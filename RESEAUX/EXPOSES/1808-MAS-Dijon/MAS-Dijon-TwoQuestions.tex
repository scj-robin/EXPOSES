%====================================================================
\subsection*{Introduction}
%====================================================================
\frame{\frametitle{Networks are useful} 

  Networks 
  \begin{itemize}
   \item have invaded most scientific fields (social sciences, economy, industry, biology, ecology, ...)
   \item provide a natural way to represent interactions between entities (humans, companies, genes, species, ...)
   \item have raised new interests in the mathematical community is the last two decades
  \end{itemize}
  
  \pause\bigskip\bigskip
  A network is actually a graph
  $$
  G = (E, V)
  $$
  \begin{itemize}
    \item $V =$ set of vertices (= nodes = entities, genes, species): 
    $$V := \llbracket 1, m \rrbracket$$
    \item $E = $ set of edges $\subset V \times V$
    % : binary or weighted, undirected or directed, univariate or multivariate.
  \end{itemize}
  }

%====================================================================
\frame{\frametitle{Two prominent statistical questions} 

  $$
  \text{\emphase{$Y =$ observed data}}
  $$
  
  \pause\bigskip
  \paragraph{Network inference.} Based on a $n \times m$ data matrix $Y = [Y_{ij}]$:
  \begin{align*}
   Y_{ij} & = \text{expression of gene $j$ in patient $i$,} \\
   & = \text{abundance of species $j$ in site $i$,}
  \end{align*}
  infer the biological / ecological network $G$ \ra \emphase{$G$ is fixed} (but unknown).
  

  \pause\bigskip\bigskip
  \paragraph{Network topology.} Based on an $m \times m$ interaction matrix $Y = [Y_{ij}]$
  \begin{align*}
   Y_{ij} & = \text{existence of a physical interaction between proteins $i$ and $j$} \\
   & = \text{number of parasites shared by species $i$ and $j$}
  \end{align*}
  understand the organization of the biological / ecological network \ra \emphase{$G = Y$ is random}.
  }
