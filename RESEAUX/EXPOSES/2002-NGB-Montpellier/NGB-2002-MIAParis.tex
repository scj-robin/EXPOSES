\documentclass[8pt]{beamer}

% Beamer style
%\usetheme[secheader]{Madrid}
% \usetheme{CambridgeUS}
\useoutertheme{infolines}
\usecolortheme[rgb={0.65,0.15,0.25}]{structure}
% \usefonttheme[onlymath]{serif}
\beamertemplatenavigationsymbolsempty
%\AtBeginSubsection

% Packages
%\usepackage[french]{babel}
\usepackage[latin1]{inputenc}
\usepackage{color}
\usepackage{xspace}
\usepackage{dsfont, stmaryrd}
\usepackage{amsmath, amsfonts, amssymb, stmaryrd}
\usepackage{epsfig}
\usepackage{tikz}
\usepackage{url}
% \usepackage{ulem}
\usepackage{/home/robin/LATEX/Biblio/astats}
%\usepackage[all]{xy}
\usepackage{graphicx}
\usepackage{xspace}
\usepackage{pifont}
\usepackage{marvosym}

% Maths
% \newtheorem{theorem}{Theorem}
% \newtheorem{definition}{Definition}
\newtheorem{proposition}{Proposition}
% \newtheorem{assumption}{Assumption}
% \newtheorem{algorithm}{Algorithm}
% \newtheorem{lemma}{Lemma}
% \newtheorem{remark}{Remark}
% \newtheorem{exercise}{Exercise}
% \newcommand{\propname}{Prop.}
% \newcommand{\proof}{\noindent{\sl Proof:}\quad}
% \newcommand{\eproof}{$\blacksquare$}

% \setcounter{secnumdepth}{3}
% \setcounter{tocdepth}{3}
\newcommand{\pref}[1]{\ref{#1} p.\pageref{#1}}
\newcommand{\qref}[1]{\eqref{#1} p.\pageref{#1}}

% Colors : http://latexcolor.com/
\definecolor{darkred}{rgb}{0.65,0.15,0.25}
\definecolor{darkgreen}{rgb}{0,0.4,0}
\definecolor{darkred}{rgb}{0.65,0.15,0.25}
\definecolor{amethyst}{rgb}{0.6, 0.4, 0.8}
\definecolor{asparagus}{rgb}{0.53, 0.66, 0.42}
\definecolor{applegreen}{rgb}{0.55, 0.71, 0.0}
\definecolor{awesome}{rgb}{1.0, 0.13, 0.32}
\definecolor{blue-green}{rgb}{0.0, 0.87, 0.87}
\definecolor{red-ggplot}{rgb}{0.52, 0.25, 0.23}
\definecolor{green-ggplot}{rgb}{0.42, 0.58, 0.00}
\definecolor{purple-ggplot}{rgb}{0.34, 0.21, 0.44}
\definecolor{blue-ggplot}{rgb}{0.00, 0.49, 0.51}

% Commands
\newcommand{\backupbegin}{
   \newcounter{finalframe}
   \setcounter{finalframe}{\value{framenumber}}
}
\newcommand{\backupend}{
   \setcounter{framenumber}{\value{finalframe}}
}
\newcommand{\emphase}[1]{\textcolor{darkred}{#1}}
\newcommand{\comment}[1]{\textcolor{gray}{#1}}
\newcommand{\paragraph}[1]{\textcolor{darkred}{#1}}
\newcommand{\refer}[1]{{\small{\textcolor{gray}{{\cite{#1}}}}}}
\newcommand{\Refer}[1]{{\small{\textcolor{gray}{{[#1]}}}}}
\newcommand{\goto}[1]{{\small{\textcolor{blue}{[\#\ref{#1}]}}}}
\renewcommand{\newblock}{}

\newcommand{\tabequation}[1]{{\medskip \centerline{#1} \medskip}}
% \renewcommand{\binom}[2]{{\left(\begin{array}{c} #1 \\ #2 \end{array}\right)}}

% Variables 
\newcommand{\Abf}{{\bf A}}
\newcommand{\Beta}{\text{B}}
\newcommand{\Bcal}{\mathcal{B}}
\newcommand{\Bias}{\xspace\mathbb B}
\newcommand{\Cor}{{\mathbb C}\text{or}}
\newcommand{\Cov}{{\mathbb C}\text{ov}}
\newcommand{\cl}{\text{\it c}\ell}
\newcommand{\Ccal}{\mathcal{C}}
\newcommand{\cst}{\text{cst}}
\newcommand{\Dcal}{\mathcal{D}}
\newcommand{\Ecal}{\mathcal{E}}
\newcommand{\Esp}{\xspace\mathbb E}
\newcommand{\Espt}{\widetilde{\Esp}}
\newcommand{\Covt}{\widetilde{\Cov}}
\newcommand{\Ibb}{\mathbb I}
\newcommand{\Fcal}{\mathcal{F}}
\newcommand{\Gcal}{\mathcal{G}}
\newcommand{\Gam}{\mathcal{G}\text{am}}
\newcommand{\Hcal}{\mathcal{H}}
\newcommand{\Jcal}{\mathcal{J}}
\newcommand{\Lcal}{\mathcal{L}}
\newcommand{\Mt}{\widetilde{M}}
\newcommand{\mt}{\widetilde{m}}
\newcommand{\Nbb}{\mathbb{N}}
\newcommand{\Mcal}{\mathcal{M}}
\newcommand{\Ncal}{\mathcal{N}}
\newcommand{\Ocal}{\mathcal{O}}
\newcommand{\pt}{\widetilde{p}}
\newcommand{\Pt}{\widetilde{P}}
\newcommand{\Pbb}{\mathbb{P}}
\newcommand{\Pcal}{\mathcal{P}}
\newcommand{\Qcal}{\mathcal{Q}}
\newcommand{\qt}{\widetilde{q}}
\newcommand{\Rbb}{\mathbb{R}}
\newcommand{\Sbb}{\mathbb{S}}
\newcommand{\Scal}{\mathcal{S}}
\newcommand{\st}{\widetilde{s}}
\newcommand{\St}{\widetilde{S}}
\newcommand{\Tcal}{\mathcal{T}}
\newcommand{\todo}{\textcolor{red}{TO DO}}
\newcommand{\Ucal}{\mathcal{U}}
\newcommand{\Un}{\math{1}}
\newcommand{\Vcal}{\mathcal{V}}
\newcommand{\Var}{\mathbb V}
\newcommand{\Vart}{\widetilde{\Var}}
\newcommand{\Zcal}{\mathcal{Z}}

% Symboles & notations
\newcommand\independent{\protect\mathpalette{\protect\independenT}{\perp}}\def\independenT#1#2{\mathrel{\rlap{$#1#2$}\mkern2mu{#1#2}}} 
\renewcommand{\d}{\text{\xspace d}}
\newcommand{\gv}{\mid}
\newcommand{\ggv}{\, \| \, }
% \newcommand{\diag}{\text{diag}}
\newcommand{\card}[1]{\text{card}\left(#1\right)}
\newcommand{\trace}[1]{\text{tr}\left(#1\right)}
\newcommand{\matr}[1]{\boldsymbol{#1}}
\newcommand{\matrbf}[1]{\mathbf{#1}}
\newcommand{\vect}[1]{\matr{#1}} %% un peu inutile
\newcommand{\vectbf}[1]{\matrbf{#1}} %% un peu inutile
\newcommand{\trans}{\intercal}
\newcommand{\transpose}[1]{\matr{#1}^\trans}
\newcommand{\crossprod}[2]{\transpose{#1} \matr{#2}}
\newcommand{\tcrossprod}[2]{\matr{#1} \transpose{#2}}
\newcommand{\matprod}[2]{\matr{#1} \matr{#2}}
\DeclareMathOperator*{\argmin}{arg\,min}
\DeclareMathOperator*{\argmax}{arg\,max}
\DeclareMathOperator{\sign}{sign}
\DeclareMathOperator{\tr}{tr}
\newcommand{\ra}{\emphase{$\rightarrow$} \xspace}

% Hadamard, Kronecker and vec operators
\DeclareMathOperator{\Diag}{Diag} % matrix diagonal
\DeclareMathOperator{\diag}{diag} % vector diagonal
\DeclareMathOperator{\mtov}{vec} % matrix to vector
\newcommand{\kro}{\otimes} % Kronecker product
\newcommand{\had}{\odot}   % Hadamard product

% TikZ
\newcommand{\nodesize}{2em}
\newcommand{\edgeunit}{2.5*\nodesize}
\newcommand{\edgewidth}{1pt}
\tikzstyle{node}=[draw, circle, fill=black, minimum width=.75\nodesize, inner sep=0]
\tikzstyle{square}=[rectangle, draw]
\tikzstyle{param}=[draw, rectangle, fill=gray!50, minimum width=\nodesize, minimum height=\nodesize, inner sep=0]
\tikzstyle{hidden}=[draw, circle, fill=gray!50, minimum width=\nodesize, inner sep=0]
\tikzstyle{hiddenred}=[draw, circle, color=red, fill=gray!50, minimum width=\nodesize, inner sep=0]
\tikzstyle{observed}=[draw, circle, minimum width=\nodesize, inner sep=0]
\tikzstyle{observedred}=[draw, circle, minimum width=\nodesize, color=red, inner sep=0]
\tikzstyle{eliminated}=[draw, circle, minimum width=\nodesize, color=gray!50, inner sep=0]
\tikzstyle{empty}=[draw, circle, minimum width=\nodesize, color=white, inner sep=0]
\tikzstyle{blank}=[color=white]
\tikzstyle{nocircle}=[minimum width=\nodesize, inner sep=0]

\tikzstyle{edge}=[-, line width=\edgewidth]
\tikzstyle{edgebendleft}=[-, >=latex, line width=\edgewidth, bend left]
\tikzstyle{edgebendright}=[-, >=latex, line width=\edgewidth, bend right]
\tikzstyle{lightedge}=[-, line width=\edgewidth, color=gray!50]
\tikzstyle{lightedgebendleft}=[-, >=latex, line width=\edgewidth, bend left, color=gray!50]
\tikzstyle{lightedgebendright}=[-, >=latex, line width=\edgewidth, bend right, color=gray!50]
\tikzstyle{edgered}=[-, line width=\edgewidth, color=red]
\tikzstyle{edgebendleftred}=[-, >=latex, line width=\edgewidth, bend left, color=red]
\tikzstyle{edgebendrightred}=[-, >=latex, line width=\edgewidth, bend right, color=red]

\tikzstyle{arrow}=[->, >=latex, line width=\edgewidth]
\tikzstyle{arrowbendleft}=[->, >=latex, line width=\edgewidth, bend left]
\tikzstyle{arrowbendright}=[->, >=latex, line width=\edgewidth, bend right]
\tikzstyle{arrowred}=[->, >=latex, line width=\edgewidth, color=red]
\tikzstyle{arrowbendleftred}=[->, >=latex, line width=\edgewidth, bend left, color=red]
\tikzstyle{arrowbendrightred}=[->, >=latex, line width=\edgewidth, bend right, color=red]
\tikzstyle{arrowblue}=[->, >=latex, line width=\edgewidth, color=blue]
\tikzstyle{dashedarrow}=[->, >=latex, dashed, line width=\edgewidth]
\tikzstyle{dashededge}=[-, >=latex, dashed, line width=\edgewidth]
\tikzstyle{dashededgebendleft}=[-, >=latex, dashed, line width=\edgewidth, bend left]
\tikzstyle{lightarrow}=[->, >=latex, line width=\edgewidth, color=gray!50]

\renewcommand{\chaptername}{Lecture}
\newcommand{\fignet}{/home/robin/RECHERCHE/RESEAUX/EXPOSES/FIGURES}
\newcommand{\figeco}{/home/robin/RECHERCHE/ECOLOGIE/EXPOSES/FIGURES}
\newcommand{\fignoisynetdata}{/home/robin/Bureau/Souhila/NoisyNetworkInference/Data21-10-19}
\newcommand{\fignoisynetsimul}{/home/robin/Bureau/Souhila/NoisyNetworkInference/FigsOld}
\newcommand{\figCMR}{/home/robin/Bureau/RECHERCHE/ECOLOGIE/CountPCA/sparsepca/Article/Network_JCGS/trunk/figs}
\newcommand{\figeconet}{/home/robin/Bureau/RECHERCHE/ECOLOGIE/EXPOSES/1904-EcoNet-Lyon/Figs}
% \newcommand{\figbarents}{/home/robin/Bureau/CountPCA/sparsepca/Pgm/PLNnetwork/barent_fish/output_barents}
% \newcommand{\figbordeaux}{/home/robin/Bureau/RECHERCHE/EXPOSES/RESEAUX/1904-Bordeaux}
% \newcommand{\nodesizeorg}{1.5em}
% \renewcommand{\nodesize}{\nodesizeorg}
% \newcommand{\edgeunitorg}{2.25*\nodesizeorg}
% \renewcommand{\edgeunit}{\edgeunitorg}

%==================================================================
%==================================================================
\begin{document}
%==================================================================
%==================================================================
\title{MIA-Paris: 'Statistical inference'}
\author[MIA-Paris]{MIA-Paris: J. Aubert, P. Barbillon, J. Chiquet, S. Donnet, R. Momal, S. Ouadah ,S. Robin}
\date[NGB, Feb'20]{NGB, Feb 2020, CEFE, Montpellier}
\maketitle

%==================================================================
\frame{\frametitle{Joint species distribution model (JSDM)}

  \paragraph{Abundance data.} $n$ sites, $p$ species: 
  
  \bigskip
  \begin{tabular}{l|l}
    \paragraph{Abundances:} $Y = n \times p$
    & 
    \paragraph{Covariates:} $X = n \times d$
    \\ \\
    \hspace{-.02\textwidth} 
    \begin{tabular}{p{.4\textwidth}}
      \begin{tabular}{rrrr}
        {\sl Hi.pl} & {\sl An.lu} & {\sl Me.ae} & ... \\
        \hline
        31  &   0  & 108 & ... \\
        4  &   0  & 110 & \\
        27  &   0  & 788 & \\
        13  &   0  & 295 & \\
        23  &   0  &  13 & \\
        20  &   0  &  97 & \\
        $\vdots$ & $\vdots$ & $\vdots$
      \end{tabular} 
    \end{tabular}
    & 
    \begin{tabular}{p{.4\textwidth}}
      \begin{tabular}{rrr}
        Lat. & Long. & Depth \\ \hline
        71.10 & 22.43 & 349 \\
        71.32 & 23.68 & 382 \\
        71.60 & 24.90 & 294 \\
        71.27 & 25.88 & 304 \\
        71.52 & 28.12 & 384 \\
        71.48 & 29.10 & 344 \\
        $\vdots$ & $\vdots$ & $\vdots$ 
      \end{tabular}
    \end{tabular}
  \end{tabular}
  \begin{itemize}
  \item $Y_{ij}=$ 'abundance' (e.g. read count) of species $j$ in site $i$
  \end{itemize}

  \bigskip
  \paragraph{Aim of JSDM's.} Model th \emphase{joint} distribution of all species accounting for
  \begin{itemize}
   \item environmental ('abiotic' covariates) effects
   \item interaction between species ('biotic')
   \item sampling effort (e.g. sampling depth)
  \end{itemize}

}

%==================================================================
\frame{\frametitle{Poisson log-normal (PLN) model}

  \paragraph{Model.} For each site $i$

  \begin{itemize}
   \item latent: \displaystyle{$Z_i \sim \Ncal_m(0, \Sigma)$}
   \item observed: \displaystyle{$Y_{ij} \mid Z_{ij} \sim \Pcal(\exp(
    \underset{\text{sampling effort}}{\underbrace{o_{ij}}} 
    + \underset{\text{covariates}}{\underbrace{x_i^\intercal \beta_j}} 
    + \underset{\text{random effect}}{\underbrace{Z_{ij}}}))$}
  \end{itemize}
  \begin{itemize}

   \item $\Sigma$: dependency structure

   \item $\beta_j$: effects of the covariates on species $j$

  \end{itemize}

  \bigskip
  \paragraph{Results.} \url{PLNmodels} R package \\
  \begin{tabular}{ccc}
    abiotic effects & species correlations & 'null' species correlations \\
    \hline
    \includegraphics[width=.3\textwidth]{\figeco/BarentsFish-coeffAll} &
    \includegraphics[width=.3\textwidth]{\figeco/BarentsFish-corrAll} &
    \includegraphics[width=.3\textwidth]{\figeco/BarentsFish-corrNull}
  \end{tabular}

}

%==================================================================
\frame{\frametitle{Network inference}

  \begin{tabular}{l|l|l}
    \emphase{Abundances:} $Y = n \times p$ & 
    \emphase{Covariates:} $X = n \times d$ & 
    \emphase{Inferred network:} $G$ \\ 
    \begin{tabular}{p{.28\textwidth}}
      \begin{tabular}{rrr}
        {\sl Hi.pl} & {\sl An.lu} & {\sl Me.ae} \\ \hline
        31  &   0  & 108 \\
         4  &   0  & 110 \\
        27  &   0  & 788 \\
        13  &   0  & 295 \\
        23  &   0  &  13 \\
        20  &   0  &  97 \\
        $\vdots$ & $\vdots$ & $\vdots$ 
      \end{tabular}
    \end{tabular}
    & 
    \begin{tabular}{p{.3\textwidth}}
      \begin{tabular}{rrr}
        Lat. & Long. & Depth \\ \hline
        71.10 & 22.43 & 349 \\
        71.32 & 23.68 & 382 \\
        71.60 & 24.90 & 294 \\
        71.27 & 25.88 & 304 \\
        71.52 & 28.12 & 384 \\
        71.48 & 29.10 & 344 \\
        $\vdots$ & $\vdots$ & $\vdots$ 
      \end{tabular} 
    \end{tabular}
    & 
    \hspace{-.02\textwidth} 
    \begin{tabular}{p{.3\textwidth}}
      \hspace{-.1\textwidth} 
      \begin{tabular}{c}
        \includegraphics[width=.35\textwidth]{\figCMR/network_BarentsFish_Gfull_full60edges}
      \end{tabular}
    \end{tabular}
  \end{tabular}

  \bigskip 
  \begin{itemize}
   \item \emphase{Approach:} $\Sigma$ should be consistent with a Gaussian graphical model (GGM), i.e. $\Sigma^{-1}$ is sparse \\ ~
   \item \emphase{Graphical lasso:} penalizing non-zero in $\Sigma^{-1}$ (Chiquet et al., ICML, 2019, \url{PLNnetwork}) \\ ~
   \item \emphase{Tree-based:} computes edge probabilities (Momal et al., Meth. Ecol. Evol., 2020, \url{EMtree})
  \end{itemize}
}

%==================================================================
\frame{\frametitle{Topological analysis (of an inferred network)}

  \paragraph{M2 internship (S. Founas).} \\ ~\\
    \begin{tabular}{c|c|c}

    {\paragraph{Abundances $Y$:} $n \times p$}

    & 

    {\paragraph{Inferred network $\widehat{G}$:} $p \times p$}

    &

    {\paragraph{SBM analysis:} }

    \\

    \hspace{-.05\textwidth} 

    \begin{tabular}{c}

      {\footnotesize \begin{tabular}{rrrr}

        Me.ae & Ra.ra & Mi.po & Ar.at \\

        \hline

        108 & 0 & 325 & 0 \\ 

        110 & 0 & 349 & 0 \\ 

        788 & 0 & 6 & 0 \\ 

        295 & 0 & 2 & 0 \\ 

        13 & 2 & 240 & 0 \\

        \vdots 

      \end{tabular}} 

    \end{tabular}

    & 

    \hspace{-.02\textwidth} 
    \begin{tabular}{c}

      \includegraphics[width=.3\textwidth]{\fignoisynetdata/BarentsFish-GhatNone.pdf}

    \end{tabular}

    & 

    \hspace{-.02\textwidth} 
    \begin{tabular}{c}

      \includegraphics[width=.3\textwidth]{\fignoisynetdata/BarentsFish-GhatNoneSBM.pdf}

    \end{tabular}

  \end{tabular}



  \bigskip 
  \paragraph{Problem:} 

  \begin{itemize}

   \item The uncertainty of network inference (step 1) 

   \item is not accounted for in the topological analysis (step 2)

  \end{itemize}


  \bigskip 
  \paragraph{Approach:} 

  \begin{itemize}

   \item Most network inference methods provide a 'score' for each edge
   \item Directly model the distribution of these scores (provided by one method of more) assuming that the network arise, say, from a stochastic blockmodel
  \end{itemize}


}

%==================================================================
\frame{\frametitle{Next}

  \paragraph{R. Momal's PhD.} 
  \begin{itemize}
   \item Network inference including a missing actor 
  \end{itemize}
  
  \bigskip \bigskip
  \paragraph{Post-doc (A. Hisi).}
  \begin{itemize}
   \item Dynamic network inference based on static data
  \end{itemize}
  
  \bigskip \bigskip
  \paragraph{M2 internship (T. Le Minh).} With F. Massol
  \begin{itemize}
   \item Analysis and comparison of mutualistic networks (ARSENIC data)
  \end{itemize}
  
}

%==================================================================
%==================================================================
\backupbegin

%====================================================================

\frame{} 


%====================================================================

\frame{\frametitle{Network inference: Barents fish} 



  \begin{center}

  \begin{tabular}{lccc}

    & no covariate & \textcolor{blue}{temp. \& depth} & \textcolor{red}{all covariates} \\

    \hline

    \rotatebox{90}{$\qquad\quad\lambda=.20$} &

    \includegraphics[width=.22\textwidth]{\figCMR/network_BarentsFish_Gnull_full60edges} &

    \includegraphics[width=.22\textwidth]{\figCMR/network_BarentsFish_Gsel_full60edges} &

    \includegraphics[width=.22\textwidth]{\figCMR/network_BarentsFish_Gfull_full60edges} 

    \vspace{-0.05\textheight} \\ \hline

    %

    \rotatebox{90}{$\qquad\quad\lambda=.28$} &

    \includegraphics[width=.22\textwidth]{\figCMR/network_BarentsFish_Gnull_sel60edges} & \includegraphics[width=.22\textwidth]{\figCMR/network_BarentsFish_Gsel_sel60edges} &

    \includegraphics[width=.22\textwidth]{\figCMR/network_BarentsFish_Gfull_sel60edges} 

    \vspace{-0.05\textheight} \\ \hline

    %

    \rotatebox{90}{$\qquad\quad\lambda=.84$} &

    \includegraphics[width=.22\textwidth]{\figCMR/network_BarentsFish_Gnull_null60edges} &

    \includegraphics[width=.22\textwidth]{\figCMR/network_BarentsFish_Gsel_null60edges} &

    \includegraphics[width=.22\textwidth]{\figCMR/network_BarentsFish_density} 

%     \includegraphics[width=.22\textwidth]{\figCMR/network_BarentsFish_Gfull_null60edges}  

  \end{tabular}

  \end{center}


}


%==================================================================

\frame{\frametitle{A pictural view}



  \begin{overprint}

  \onslide<1>\paragraph{Conceptual (generative) model:} 

  \onslide<2>\paragraph{Pipe-line:}

  \onslide<3>\paragraph{Actual pipe-line:}

  \onslide<4>\paragraph{Our aim:}

  \end{overprint}



  \bigskip

  \begin{tabular}{ccc}

    \begin{tabular}{|c|}

      \hline

      Node membership $Z$ \\

      \hline

      \includegraphics[width=.25\textwidth]{\fignet/FigNoisyNet-NodesGroup} \\

      \hline

    \end{tabular}

    & \onslide+<1>{\Huge \rotatebox{0}{\color{red}\MVRightarrow}} \onslide+<2-3>{\Huge \rotatebox{180}{\color{red}\MVRightarrow}} & 

    \begin{tabular}{|c|}

      \hline

      Graphical model $G$ \\

      \hline

      \includegraphics[width=.25\textwidth]{\fignet/FigNoisyNet-Graph} \\

      \hline

    \end{tabular} 

    \\ 

    & & \\

    \onslide+<4>{\rotatebox{90}{\Huge \color{red}\MVRightarrow}} 

    & \onslide+<3>{\rotatebox{45}{\Huge \color{red}\MVRightarrow}}

    & \onslide+<1>{\Huge \rotatebox{270}{\color{red}\MVRightarrow}} \onslide+<2>{\Huge \rotatebox{90}{\color{red}\MVRightarrow}} \\

    & & \\

    \begin{tabular}{|c|}

      \hline

      Edge scores $S$ \\

      \hline

      \scriptsize{\tt \begin{tabular}{lrrrrr}

        & sp1 & sp2 & sp3 & sp4 & sp5 \\ 

        sp1 & - & 1.5 & 0.2 & 17.7 & 0.1 \\ 

        sp3 &  & - & 26.9 & 8.9 & 1.4 \\ 

        sp3 &  &  & - & 1.3 & 5.2 \\ 

        sp4 &  &  &  & - & 10.6 \\ 

        sp5 &  &  &  &  & - \\

        \vdots 

      \end{tabular} } \\

      \hline

    \end{tabular}

    & \onslide+<1>{\rotatebox{180}{\Huge \color{red}\MVRightarrow}} \onslide+<3-4>{\rotatebox{180}{\Huge \color{red}\MVRightarrow}} & 

    \begin{tabular}{|c|}

      \hline

      Observed data $Y$ \\ 

      \hline

      \scriptsize{\tt \begin{tabular}{lrrrrr}

        & sp1 & sp2 & sp3 & sp4 & sp5 \\ 

        site1 & 0 & 2 & 8 & 2 & 0 \\ 

        site2 & 3 & 0 & 9 & 0 & 1 \\ 

        site3 & 1 & 5 & 15 & 0 & 3 \\ 

        site4 & 4 & 1 & 16 & 1 & 2 \\ 

        site5 & 1 & 3 & 104 & 0 & 4 \\ 

        site6 & 1 & 0 & 10 & 1 & 3 \\

        \vdots

      \end{tabular} } \\

      \hline

    \end{tabular} 

  \end{tabular}

  }

  

\backupend

%==================================================================
%==================================================================
\end{document}
%==================================================================
%==================================================================

\begin{tabular}{ll}
  \begin{tabular}{p{.45\textwidth}}
  \end{tabular}
  &
  \begin{tabular}{p{.45\textwidth}}
  \end{tabular} 
\end{tabular}
