\documentclass[12pt]{article}

\textwidth 16cm
\textheight 23cm 
\topmargin -1cm 
\oddsidemargin 0cm 
\evensidemargin 0cm

\begin{document}
\pagestyle{empty}

\section*{Session 6.4:  Bioinformatics}

\centerline{\it Chair: S. Schbath, INRA, France}
\vspace{0.5cm}

\subsection*{Parameter estimation in pair-hidden Markov models.}

\centerline{A. Arribas-Gil, E. Gassiat, {\bf C. Matias}}
\vspace{0.5cm}

Pair-hidden Markov models (pair-hmms) are useful tools for doing sequence alignment in a probabilistic framework. Algorithms enable to estimate the parameters of the model and to recover the alignment at the same time. It thus appears as an interesting alternative to the problem of the choice of a scoring function. We are interested in the validity of such procedures. First, we describe the pair-hmm and discuss possible definitions of likelihoods. Then we establish the validity of the pair-hmms procedures under some particular assumptions. We also provide simulation results toi cover the cases where our results do not apply.

\subsection*{A mixture model for random graphs}
  
\centerline{J.-J. Daudin, F. Picard, {\bf S. Robin}}
\vspace{0.5cm}

The Erdos-R�nyi (ER) model is probably the oldest and simplest model
for random graphs. It is simple and many explicit expressions for its
average and asymptotic properties are known. However it does not fit
well many real-word networks.

One reason is that the vertices of a network are often structured in a
priori unknown clusters (functionally related proteins, social
communities, etc.) with different connectivity properties. We define a
generalization of ER named the Erdos-R�nyi mixture for graphs
(ERMG). We derive some of its theoretical properties and propose a
pseudo-EM algorithm to estimate its parameters. 

We apply this modeling to recover the modular structure of a network
of enzymatic reactions.

\subsection*{Networks motifs: mean and variance for the count}

\centerline{C. Matias, {\bf S. Schbath}, E. Birmel\'e, J.-J. Daudin, S. Robin}
\vspace{.5cm}

Network motifs (patterns of interconnections) are at the core of
modern studies on biological networks, trying to encompass global
features such as small-world or scale-free properties.  Detection of
over- and under-represented motifs may be based on two different
approaches: either a comparison with randomized networks (requiring
the simulation of a large number of networks), or the comparison with
expected quantities in some well-chosen probabilistic model. This
second approach has been investigated here. We consider the classical
Erd�s-R�nyi model but also a random graph model that fits the vertex
degrees.  In such models, we provide exact formulas for the
expectation and the variance of the number of occurrences of a motif.
Generalizations exist for the ERMG model (see previous talk of S.
Robin) and for oriented motifs in directed graphs.


\end{document}
