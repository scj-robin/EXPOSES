\documentclass[pdf,mia,noFooter,slideColor,colorBG]{prosper}
%\documentclass[mia,noFooter]{prosper}

\usepackage{amsmath, amssymb}
\usepackage{epsfig}
\usepackage{color}

\definecolor{orange}{rgb}{.98,.3,.03}
\newcommand{\orange}[1]{{\textcolor{orange}{{#1}}}}
\newcommand{\emphase}[1]{{\textcolor{blue}{{#1}}}}
\newcommand{\publi}[1]{{\textcolor{blue}{{\bf #1}}}}
\renewcommand{\paragraph}[1]{{\large \bf  #1}}

%====================================
%====================================
\begin{document}
%====================================
%====================================
\title{Title}
\subtitle{Subtitles}
\author{Smurf}
\email{smurf@inra.fr}
\institution{INRA}

%\maketitle

%====================================
\begin{slide}{}
  \psline[linewidth=50pt,linecolor=white](-2,1.4)(10.3,1.4)
  \rput[lb](2.5cm,0.4cm){\includegraphics[width=5cm]{Page1_top.eps}}
  \rput[lb](-0.3cm,-1.4cm){\includegraphics[width=10.5cm]{logoMIA19x2.eps}}
  \rput[lb](0.8cm,-4.8cm){Evaluation of the MIA Department -- October 2006}
  \rput[lb](-1.8cm,-5.2cm){\rotatebox[origin=c]{0}{\includegraphics[width=2cm]{butterfly.eps}}}   
  \rput[lb](-2cm,-7.35cm){\includegraphics[width=13.7cm]{Page1_bottom.eps}}
 
  \rput[lb](0.5cm,-2.1cm){\Large\textbf{Statistics for Systems Biology}}
\end{slide}
%====================================

% %====================================
% \begin{slide}{SSB group}
% \begin{itemize}
% \item 3 MIA units: Jouy, Paris, Evry
% \item 30 scientists
% \item Regular meetings
% \end{itemize}
% \end{slide}
% %====================================

% %====================================
% \begin{slide}{3 Major Research fields}
% \vspace{0.5cm}
% \begin{description}
% \item[\paragraph{Sequence analysis:}] ~\\
%   \emphase{Motifs} (discrete
%   math., large deviation, Poisson approximation, scan statistics) \\
%   \emphase{Local and alignment scores} (CUSUM processes, random walks,
%   paired
%   HMM) \\
%   \emphase{Gene, motif and 2D structure prediction} (HMM, semi-Markov
%   hidden models)
% \end{description}
% \end{slide}
% %====================================

% %====================================
% \begin{slide}{}%3 Major Research Fields (2)}
% \vspace{-0.75cm}
% \begin{description}
% \item[\paragraph{Microarray data:}] ~\\
%   \emphase{Experimental design} (linear model) \\ 
%   \emphase{Differential analysis} (multiple testing, mixture model) \\
%   \emphase{Diagnostic} (supervised classification, dimension reduction, model
%   selection) \\ 
%   \emphase{Comparative Genomic Hybridisation} (breakpoints
%   detection, mixed model)\\
%   ~\\
% \item[\paragraph{Biological networks:}] ~\\
%   \emphase{Random graph models}
%   (mixture models, variational methods, discrete math.), \\
%   \emphase{Motifs} (Combinatorics, Poisson approximation), 
%   \emphase{Network inference} (Graphical models, multiple testing)
% \end{description}
% \end{slide}
% %====================================

% %====================================
% \begin{slide}{Two illustrations}
% \begin{itemize}
% \item Biological sequence analysis via HMM
  
% \item Statistical analysis of biological networks
% \end{itemize}
% \end{slide}
% %====================================

%====================================
\begin{slide}{SSB group}
\begin{itemize}
\item 24 scientists (+ 6 PhD).
\item 3 MIA units: Evry, Jouy (MIG), Paris.
\item Existence from 10 years.
\item Strong interactions.
\item Regular meetings (1 day per month).
\item \url{www.genome.jouy.inra.fr/ssb/}
\end{itemize}

% MIG : SS, FR, PN, CH, [ER, AG]

% Evry : BP, FM, AST, GN, CM, EB, MH, VM, CC, CA [FP, SL, MG]

% Paris : SR, JJD, EL, TMH, JA, MLM, AB, NC, [MM, AC, MK]
\end{slide}
%====================================

%====================================
\overlays{2}{%
\begin{slide}{3 Major Research fields}
\vspace{0.5cm}
\begin{description}
\item[\paragraph{Sequence analysis:}] ~\\
  \emphase{Motifs} (discrete
  math., large deviation, Poisson approximation, scan statistics). \\
  \emphase{Local and alignment scores} (CUSUM processes, random walks,
  paired HMM). \\
\onlySlide*{1}{
\emphase{Gene, motif and 2D structure prediction} (HMM,
  semi-Markov hidden models).
}
\onlySlide{2}{
\orange{Gene, motif and 2D structure prediction} (HMM,
  semi-Markov hidden models).
}
\end{description}
\end{slide}
}
%====================================

%====================================
\overlays{3}{%
\begin{slide}{}%3 Major Research Fields (2)}
\vspace{-0.75cm}
\begin{description}
\FromSlide{1}
\item[\paragraph{Microarray data:}] ~\\
  \emphase{Experimental design} (linear model). \\ 
  \emphase{Differential analysis} (multiple testing, mixture model). \\
  \emphase{Diagnostic} (supervised classification, dimension reduction, model
  selection). \\ 
  \emphase{Comparative Genomic Hybridisation} (breakpoint
  detection, mixed model).\\
  ~\\
\FromSlide{2}
\item[\paragraph{Biological networks:}] ~\\
\onlySlide*{2}{  
  \emphase{Network inference} (Graphical models, multiple testing).
  \emphase{Random graph models}
  (mixture models, variational methods, discrete math.). \\
  \emphase{Motifs} (Combinatorics, Poisson approximation). 
}
\onlySlide*{3}{
  \emphase{Network inference} (Graphical models, multiple testing).
 \orange{Random graph models}
  (mixture models, variational methods, discrete math.). \\
  \orange{Motifs} (Combinatorics, Poisson approximation).
}
\end{description}
\end{slide}
}

%====================================
\begin{slide}{Sequence Analysis via HMM}
  \hspace{-1cm}{The oldest theme of SSB is sequence analysis ($1^\text{st}$ paper 1995)}\\
  {\bf Markov} M$m$: $\qquad \qquad \qquad \qquad X_t \in {\cal A},
  \quad |{\cal A}| = k$
\begin{eqnarray*}
{\mathbb P}(X_t \ | \ (X_j)_{1 \leqslant j \leqslant  t-1}) &=& {\mathbb P}(X_t \ | \ (X_j)_{t-m \leqslant j \leqslant  t-1})\\
&=& \pi(X_{t-m}\dots X_{t-1} \ ; \ X_t)
\end{eqnarray*}
{\bf Hidden Markov} M$1$M$m$: $\qquad \qquad S_t \in [1, 2, \dots, r]$
\begin{eqnarray*}
{\mathbb P}(S_t \ | \ (S_j)_{1 \leqslant j \leqslant  t-1}) &=& {\mathbb P}(S_t \ | \ S_{t-1}) \\ 
&=&  \color{red}\pi_0 \color{black} (S_{t-1} \ ; S_t)\\
{\mathbb P}(X_t \ | \ (X_j)_{1 \leqslant j \leqslant  t-1} \  \ (S_j)_{1 
\leqslant j \leqslant  t}) &=& {\mathbb P}(X_t \ | \ (X_j)_{t-m 
\leqslant j \leqslant  t-1}, S_t)\\
&=&  \color{red} \pi_{S_t} \color{black}   (X_{t-m}\dots X_{t-1} \ ; X_t)
\end{eqnarray*}
\end{slide}
%====================================


%====================================
\begin{slide}{Markov : ``exceptional motifs''}
$\qquad$ Markov models are essentialy used to measure the ``exceptionality'' of a given word $w$ or of a given motif $W$\\
\color{orange} {\it Gaussian and compound Poisson approximations were known in 2001.} \color{black}\\
{\bf New} :
\begin{itemize}
\item Exact distribution \color{blue} {\bf (JRSS C 02, JCB 05)} \color{black}
\item Approach based on Large Deviations \color{blue} {\bf (JCB 04)} \color{black}
\item Waiting times for structured motifs based on generative
  functions \color{blue}{\bf (JCB 02, Disc.App.Math 06)} \color{black}
\item Efficient implementation of all methods \color{blue} {\bf (R'MES, SPatt)} \color{black}
\item Numerical comparison of methods \color{blue} {\bf (JCB 01, SAGMB 06) } \color{black}
\end{itemize}
\end{slide}
%====================================


%====================================
\begin{slide}{}
\vspace{-1.2cm}
\begin{itemize}
\item Unified approach through Deterministic Finite Automata \color{blue}{\bf (JAP 06) }  \color{black}
\end{itemize}
Consider the motif $W = $  a-x(1;2)-a ; the graph below detects the occurrences of $W$ on a sequence (passage in the red circle).
\vspace{-.4cm}
\begin{tabular}{cc}
\begin{minipage}{0.4\textwidth}
\begin{figure}
\includegraphics[width=\textwidth]{FDA.eps}
\end{figure}
\end{minipage}
\begin{minipage}{0.6\textwidth}
{\tiny
$$
\left(
\begin{array}{rrrrr}
\pi(a,a) & \pi(a,b) & 0 & 0 & 0\\
\pi(b,a) & 0 & 0 & \pi(b,b) & 0\\
0 & 0 & \pi(b,b) & 0 & \pi(b,a)\\
\pi(b,a) & 0 & \pi(b,b) & 0 & 0\\
\pi(a,a) & \pi(a,b) & 0 & 0 & 0
\end{array}
\right)
$$
}
\end{minipage}
\end{tabular}
If $X_t$ is a Markov chain, the position on this graph is also a MC (whose transition is indicated).\\
This methods overcomes the curse of the motif cardinality (ex : Prosite)


%\color{blue} \underline{Example of application} : Genomic distribution of motifs involved in DNA repair (E. coli)
%\color{black}
\end{slide}
%====================================

%====================================
\begin{slide}{Hidden Markov Models}
\begin{center}
Theoretical results
\end{center}
\begin{itemize}
\item Theoretical properties of the MLE
\item Estimation by Monte Carlo MC (MCMC)
\item Estimation of Semi Markovian Hidden Model (SHMM)  \color{blue} {\bf (Signal proc. 01) } \color{black}
\item Choice of models,
\begin{itemize}
\item using penalization of likelihood,
\item using Reversible Jump MC. \color{blue} {\bf (JCB 06)} \color{black}
\end{itemize}
\end{itemize}
\begin{center}
+ \color{blue} {\bf (3 books : Belin 03, CUP 05, Hermes 07) } \color{black}
\end{center}
\end{slide}
%====================================
%        \color{blue} {\bf ( ) } \color{black}

%====================================
\begin{slide}{Hidden Markov Models}
The modelisation by HMM turns to be a tool for the analysis of heterogeneity. In the ssb group of MIA, we used them for
\begin{itemize}
\item Alignments (pair HMM) \color{blue} {\bf (Scand. J. S 06) } \color{black}
\item Annotation 
\begin{itemize}
\item localisation of genes  \color{blue} {\bf (NAR 02) } \color{black}
\item localisation of motifs (ex : promotors) \color{blue} {\bf (JCB 06) } \color{black}
\end{itemize}
\item Secondary structure of proteins  \color{blue} {\bf (IEEE Int.Syst. 05 ) } \color{black}
\item Localisation of nucleosomes \color{blue} {\bf (submitted Science 06)} \color{black}
\end{itemize}
\end{slide}
%====================================


%====================================
\begin{slide}{Alignment : pair HMM}
A model for a pair of sequences consists in :
\begin{tabular}{cc}
\begin{minipage}{0.5\textwidth}
\noindent $\bullet$ an (hidden) MC $S_t$ taking values in $\{ \, \longrightarrow \, , \, \uparrow \, , \, \nearrow  \}$\\
\noindent $\bullet$ when $S_t = \longrightarrow$, we observe\\
\indent $\quad (X_t,-) \quad ; \quad X_t \sim P_1$\\
$\bullet$  when $S_t = \uparrow$, we observe\\
\indent $\quad (-,Y_t) \quad \, ; \quad Y_t \sim P_2$\\
$\bullet$  when $S_t = \nearrow$, we observe\\ 
\indent $\quad (X_t,Y_t) \sim P_{12}$
\end{minipage}
\begin{minipage}{0.5\textwidth}
\begin{figure}
\centering
\includegraphics[width=.8\textwidth]{pairHMM.eps}
\end{figure}
\end{minipage}
\end{tabular}

\vspace{0,5cm}

The alignment of two observed sequences can be done using the usual HMM tools.

\end{slide}
%====================================


%====================================
\begin{slide}{HMM : SHOW}
A very efficient (ssb) software implements these methods\\
 \color{blue} {\bf (NAR 06, PNAS 06 ) } \color{black}
\begin{figure}
\centering
\includegraphics[width=.8\textwidth]{papillon.eps}
\end{figure}
SHOW is part of the ``plate-forme'' AGMIAL in the INRA and is widely used (ex: M�digue's team in genopole).
\end{slide}
%====================================


%====================================
\begin{slide}{Annotation}
On $200\, 000$ bp of {\it B. subtilis}, SHOW gives the following annotation (curves), to be compared to the true one (colored arrows)
\begin{figure}
\centering
\includegraphics[width=.9\textwidth]{bsub_34.eps}
\end{figure}
\end{slide}
%====================================
%{bsub_1-200000_6etats.eps}

%====================================
\begin{slide}{SHMM}
HMM $\Rightarrow$ the length of each region follows an exponential distribution -- this hypothesis is false in the reality !\\
\color{blue} {\bf Example : Modelization of the promotors of genes} \color{black}
\vspace{-0.3cm}
\begin{figure}
\centering
\includegraphics[width=.8\textwidth]{figure1.eps}
\end{figure}
\vspace{-1.1cm}
\begin{figure}
\includegraphics[width=.8\textwidth]{figure2.eps}
\end{figure}
\end{slide}
%====================================


%====================================
\begin{slide}{The promotor model : result}
\vspace{-.5cm}
\begin{figure}
\centering
\includegraphics[width=.7\textwidth]{figure4.eps}
\end{figure}
\end{slide}
%====================================

%====================================
\begin{slide}{II$^{\text{ary}}$ structure of proteins}
The hidden chain is written in a ``structural alphabet'' describing angles in the skeleton of the protein.
\begin{tabular}{cc}
\begin{minipage}{0.4\textwidth}
\begin{figure}
\includegraphics[width=\textwidth]{diedre.eps}
\end{figure}
\begin{figure}
\includegraphics[width=\textwidth]{complex2.eps}
\end{figure}
\end{minipage}
\begin{minipage}{0.5\textwidth}
A learning set indicates how  the ``words'' of 5 a.a. can be grouped according to their spatial configuration in (here) 14 groups : the ``letters'' of the structural alphabet. The frequencies of each a.a. in each ``letter'' is estimated.\\
Given a sequence of a.a., HMM tools allows the attribution of ``letters'', and therefore a spatial reconstruction.
\end{minipage}
\end{tabular}
\end{slide}
%====================================


%====================================
\begin{slide}{Searching nucleosomes}
In eukaryotes, an important part of the chromosomes forms chromatine, a state where the double helix winds rounds beads, forming a ``collar''.
\begin{figure}
\centering
\includegraphics[width=.8\textwidth]{defnucleo.eps}
\end{figure}
Is it possible to give a \color{red} prediction of the positions \color{black} of these nucleosomes ?
\end{slide}
%====================================


%====================================
\begin{slide}{}
\begin{tabular}{cc}
\begin{minipage}{0.4\textwidth}
We will take advantage from the fact that the curvature of the DNA
differs where it  touchs the ``core'' and in the parts far from the core.
Therefore the necessary energy differs.
\end{minipage}
\begin{minipage}{0.6\textwidth}
\begin{figure}
\centering
\includegraphics[width=\textwidth]{defnucleo2.eps}
\end{figure}
\end{minipage}
\end{tabular}
\end{slide}
%====================================


%====================================
\begin{slide}{}
\begin{figure}
\centering
\includegraphics[width=\textwidth]{longsegment.eps}
\end{figure}
Using only the primary sequence ({\bf taccgtatcag...}),\\
 we have computed the energy which is locally necesary to bend the DNA (blue curve).\\
This is our ``observed sequence'' $X_t \in {\mathbb R}$
\end{slide}
%====================================


%====================================
\begin{slide}{}
$\qquad \qquad \quad$ \color{red} ``Nuc''-states ($\neq$ distributions in each position) 
\begin{figure}
\centering
\includegraphics[width=.8\textwidth]{hmmnucleo.eps}
\end{figure}
\color{black}
A HMM is ajusted to these data.\\
The hidden chain will be $S_t =$ Nuc $\quad$ or $\quad S_t =$ No-Nuc
\end{slide}
%====================================


%====================================
\begin{slide}{}
\begin{figure}
\centering
\includegraphics[width=.3\textwidth]{segmentafterHMM_2.eps}
\hspace{1cm}
\includegraphics[width=.4\textwidth]{segmentafterHMM_1.eps}
\end{figure}
The predicted positions of the nucleosomes (red dots) ``fairly'' coincide with the experimental data (red curve -- Rando et al.)
\end{slide}
%====================================

%====================================
\begin{slide}{\hspace{-6cm}Network Statistics }
  \vspace{-1cm}
  \begin{tabular}{cc}
    \hspace{-1cm}
    \begin{tabular}{p{3cm}}
      \\ \\ \\ \\ \\ \\ \\
      Yeast protein interaction network. \\
      {\small \sl Barabasi, Nat. Genet., 04} \\ \\ \\
    \end{tabular}
    &
    \begin{tabular}{c}
      \epsfig{file = ../Figures/Barabasi6.ps, clip=, bbllx=39, bblly=466,
        bburx=351, bbury=754, width=8cm}
    \end{tabular}
  \end{tabular}
\end{slide}
%====================================

%====================================
\begin{slide}{Complex Systems}
Molecular biologists more and more face data with a network
structure:
\begin{enumerate}
\item protein interaction networks
\item gene regulation networks
\item metabolic networks
\end{enumerate}

Statistical tools are needed
\begin{description}
\item[$(i)$] to understand and to lay out the global topology
of the graph \\
\item[$(ii)$] and to detect unexpected local structures.
\end{description}
\end{slide}
%====================================

%====================================
\begin{slide}{Our Contributions}
\begin{description}
\item[$(i)$] \paragraph{Models for random graphs} \\
  Random graphs are natural to describe such networks. \\
  Lack of models fitting the observed networks well. \\
  ~\\
\item[$(ii)$] \paragraph{Significant structures in a biological
    network} \\
  The search for unexpected patterns or characteristics (i.e.
  diameter) is the standard way to point out key features.
\end{description}
This field involves about \emphase{10 people} from Evry, Jouy and
Paris.
\end{slide}
%====================================

%====================================
\begin{slide}{A New Random Graph Model}
  \paragraph{Erd\"os Model.} All pairs of vertices have the \emphase{same
    probability} to be connected; All edges are independent. \\
  ~\\
  $\rightarrow$ Poor fit, probably due to some heterogeneity between
  the vertices (i.e. proteins, reactions, genes). \\
  ~\\
  \paragraph{Mixture Model.} Vertices are spread among \emphase{$Q$ groups}
  (with proportions $\alpha_q$); Vertices from groups $q$ and $\ell$
  are connected with probability $\pi_{q\ell}$. \\
  ~\\
  This includes the preferential attachment model: the groups
  correspond to the time when the vertices joined the network.
\end{slide}
%====================================

%====================================
\begin{slide}{Mixture Topologies}
%\vspace{-1cm}
\begin{tabular}{cc}
  \hspace{-1cm} 
  \begin{tabular}{c}
    Clusters (affiliation) \\
    \epsfig{file = ../figures/FigNetworks-Clusters-Col.eps,
      height=5cm, width=2.5cm, clip=, angle=270}    \\
    Stars (hubs) \\
    \epsfig{file = ../figures/FigNetworks-Star-Col.eps,
      height=5cm, width=2.5cm, clip=, angle=270}
    \end{tabular}
    &
    \begin{tabular}{c}
      Preferential \\ attachment \\
      \epsfig{file = ../figures/FigNetworks-PrefAtt-Col.eps, 
        height=5cm, width=5cm, clip=,}% angle=270}
    \end{tabular} 
  \end{tabular} 
\end{slide}
%====================================

%====================================
\begin{slide}{Parameter Estimation}
  \vspace{0.5cm}
  \paragraph{Standard strategy for mixture model: E-M.} \\
  The maximisation of the log-likelihood $\log P(X)$ of the observed
  edges $X$ involves the calculation of the conditional distribution
  of the unobserved vertex labels $Z$: $P(Z\;|\;X)$. \\
  ~\\
  \paragraph{E-M fails for random graphs.} \\
  All vertices are potentially connected so there is no local
  dependency: \\
  $\rightarrow$ The \emphase{neighbourhood of a vertex is the whole graph} \\
  $\rightarrow$ $P(Z\;|\;X)$ can  not be  computed \\
  $\rightarrow$ A new strategy is needed.
\end{slide}
%====================================

%====================================
\begin{slide}{Variational Approach}
  \paragraph{Lower bound for the likelihood.} \\
  We actually optimise
  $$
  \log P(X) - KL[P(Z\;|\;X), Q_R(Z)]
  $$
  where $Q_R$ is the \emphase{best approximation} of $P(Z\;|\;X)$
  within a class of 'nice' distributions:
  $$
  Q_R(Z) = \prod_i q(Z_i; \tau_i(X)).
  $$
  The multinomial parameters $\tau_i(X)$ can be interpreted as
  \emphase{posterior probabilities} for vertew $i$ to belong to each
  group.
\end{slide}
%====================================

%====================================
\begin{slide}{Estimation Algorithm}
  \begin{description}
  \item[\paragraph{Optimisation of $Q_R$.}] ~\\
    The optimal $\widehat{Q}_X$ is obtained with
    $\widehat{\tau}_i(X)$'s satisfying an explicit
    fix-point relation. \\
  \item[\paragraph{Parameter estimates.}] ~\\
    Given the $\widehat{\tau}_i(X)$, analytical expressions of
    $\widehat{\alpha}_q$ and $\widehat{\pi}_{q\ell}$ can be easily derived. \\
  \item[\paragraph{Model selection.}] ~\\
    A BIC-like criterion has been derived to choose the number of
    groups. \\ 
  \end{description}
  A \emphase{software} is available for both oriented and non-oriented
  graphs + \publi{under review at Stat. \& Comput. and PNAS}
\end{slide}
%====================================

%====================================
\begin{slide}{E. coli reaction network}
%\vspace{-1cm}
\hspace{-1.5cm}
\begin{tabular}{cc}
  \begin{tabular}{p{5cm}}
                                %\paragraph{Zoom (bottom left).} \\ 
                                %\\
    Submatrix of $\widehat{\pi}$:
    {\tiny      
      $$
      \begin{tabular}{c|cccc}
        $q, \ell$ & 1 & 7 & 10 & 16 \\
        \hline 
        1 & {\bf 1.0} \\ 
        7 & {\sl .11} & .65 \\ 
        10 & {\sl .43} & & .67  \\ 
        16 & {\bf 1.0} & {\sl .01} & & {\bf 1.0} \\
      \end{tabular}
      $$
      }
    Groups 1 and 16 both involve pyruvate; \\
    Only group 1 involves also
    CO2 (group 7) and acetylCoA (group 10).
                                %     \\ \\ 
  \end{tabular}
  &
  %\hspace{-1cm}
  \begin{tabular}{l}
    \epsfig{file = ../figures/Ecoli-Complet-ERMG-Ward-Q21_class.eps,
      height=5.5cm, width=5.5cm, clip=,bbllx=90, bblly=485, bburx=277,
      bbury=605.5} \\
    \publi{SSB + INRIA Helix: JOBIM 06}
  \end{tabular}  
%   \vspace{-2cm}
%   \\ \\ \\
%   \begin{tabular}{p{9cm}}
%     \paragraph{Vertices degree $K_i$.}  \\ \\ 
%     Mean degree in the last group:\\ 
%     $\overline{K}_{21} = 2.6$ \\ \\ \\
%   \end{tabular}
%   & 
%   \begin{tabular}{l}
%     \epsfig{file = ../figures/Ecoli-Complet-ERMG-Ward-Q21_class.eps,
%     width=8.5cm, height=3cm, clip=, bbllx=70, bblly=105, bburx=277,
%     bbury=245}   
%   \end{tabular}
\end{tabular}
\end{slide}
%====================================

%====================================
\begin{slide}{Network Motifs}
\begin{description}
\item[\paragraph{Regulatory motifs.}] ~\\
  $\begin{array}{ccc}
  \end{array}$
  Motifs may perform specific regulatory functions.
\item[\paragraph{Exceptional motifs.}] ~\\
  Motifs which occur more frequently than expected reflect functional
  or computational units which combine to regulate the cellular
  behaviour. \\
\item[\paragraph{Statistical significance.}] ~\\
  Need to evaluate 
  $
  \Pr\{N({\bf m}) \geq n_{obs}\}
  $
  where $N({\bf m}) = $ count of the motif ${\bf m}$ in a given
  network.
\end{description}
\end{slide}
%====================================

%====================================
\begin{slide}{State of the Art}
  \paragraph{Degree distribution fitting.} ~\\
  Consider a graph with random edges but where the degree of each
  vertex is fixed (and equal to the observed network). \\
  \begin{itemize}
  \item Shen-Orr \& al. (Nat. Genet. 01) estimate the distribution
    of $N({\bf m})$ using \emphase{simulations} \\
    $\longrightarrow$ heavy computational time.
  \item \publi{RevStat, 05} proposes \emphase{analytic formulas} for
    the moments \\
    $\longrightarrow$ difficult to implement because of the
    combinatorial complexity.
  \end{itemize}
\end{slide}
%====================================

%====================================
\begin{slide}{Stationary Models}
  \begin{tabular}{cc}
    \hspace{-1cm}
    \begin{tabular}{p{6cm}}
      Assume that the probability $\mu({\bf m})$ for motif ${\bf m}$ to occur
      \emphase{does not depend on the position} in the graph. \\
      \\
                                %\paragraph{Calculating the moments.}
      We can calculate the expected count \emphase{$\mathbb{E}
        N({\bf m})$} and variance of the count \emphase{$\mathbb{V}
        N({\bf m})$}. \\ \\ 
      $\mathbb{V} N({\bf m})$ depends on the expected count of
      {all possible overlaps} of the motif ${\bf m}$.\\ 
      (ex: ${\bf m} = {\Huge \times}$)
    \end{tabular}
    & 
    \hspace{-0.5cm}
    \begin{tabular}{c}
      \epsfig{file=
        /RECHERCHE/RESEAUX/Motifs/FIGURES/MotifStar4-Recouv3.eps,
        clip, width=4.6cm, height=6.5cm}
    \end{tabular}
  \end{tabular}
\end{slide}
%====================================

%====================================
\begin{slide}{Compound Poisson approx.}
  \paragraph{Approximate distribution of $N({\bf m})$.} ~\\
  The exact distribution of $N({\bf m})$ is still unknown. \\
  We look for an approximation involving only 2 parameters
  (corresponding to the 2 moments). \\
  ~\\
  \paragraph{Analogy with sequence motifs.} ~\\
  The compound (geometric) Poisson approximation outperforms other
  approximations for motif counts in sequences (e.g. DNA). \\ ~\\
  It assumes that the motif occurs in clumps, the number of clumps
  being Poisson and the clump size being Geometric.
  
\end{slide}
%====================================

%====================================
\begin{slide}{Examples}
  \begin{tabular}{cc}
    \hspace{-1cm}
    \begin{tabular}{p{5cm}}
      \paragraph{Geometric Poisson.}~\\
      Simulations show that it outperforms the Poisson and Gaussian
      approximations. \\ \\
      Although the geometric clump size is questionable for network
      motifs, the approximate $p$-values are accurate.
    \end{tabular}
    &
    \hspace{-0.5cm}
    \begin{tabular}{c}
      \epsfig{file=/RECHERCHE/RESEAUX/Motifs/FIGJJD-150506/fig11_200.eps, 
        bbllx=332, bblly=211, bburx=515, bbury=304, width=6cm,
        height=3.5cm, clip=} \\
      \textcolor{red}{\bf --} Gaussian, \textcolor{green}{\bf --}
        Compound Poisson \\
        \\
        \publi{submitted to Recomb 07}
    \end{tabular}
  \end{tabular}  
\end{slide}
%====================================

%====================================
\begin{slide}{Bipartite Graph}
  \begin{description}
  \item[\paragraph{Metabolic networks.}] Metabolic network involves 2
    different kinds of vertices: reactions and compounds. \\
    ~\\
  \item[\paragraph{Model.}] A stochastic model for such bipartite
    graph explains the properties of the projected graphs
    (\publi{submitted to Discr. Appl. Math.}). \\
    ~\\
  \item[\paragraph{Coloured motifs.}] Reaction motifs are connected
    sub-graphs involving reactions of a certain type (colour).
  \end{description}
\end{slide}
%====================================


%====================================
%====================================
\end{document}
%====================================
%====================================

