%====================================================================
%====================================================================
\section{Inference}
\frame{\frametitle{Outline} \tableofcontents[currentsection]}
%====================================================================
\subsection{Incomplete data models}
%====================================================================
\frame{\frametitle{Inference of incomplete data models} 

  \paragraph{Maximum likelihood inference.}
  $$
%   \widehat{\theta} = \arg
  \max_\theta \; \underset{\text{likelihood}}{\underbrace{p(Y; \theta)}}
  $$
  \ra No closed form for ${p(Y; \theta) = \int \underset{\text{complete likelihood}}{\underbrace{p(Y, Z; \theta)}} \d Z}$ in most latent variable models.

  \pause \bigskip \bigskip
  \paragraph{Incomplete data models.} EM algorithm \refer{DLR77}
  $$
  \theta^{h+1} = 
  \underset{\text{\normalsize \emphase{M step}}}{\underbrace{{\argmax}_\theta}} \;
  \underset{\text{\normalsize \emphase{E step}}}{\underbrace{\Esp_{\theta^h}}} \;
  [\log p(Y, Z; \theta) \mid Y)].
  $$
  
  \pause \bigskip \bigskip 
  \paragraph{E step = critical step:} Evaluate $p(Z \mid Y; \theta)$, i.e. \\
  \medskip
  \ra Retrieve sufficient information about the latent variables $Z$ based on the observed ones $Y$.
}

%====================================================================
\subsection*{Variational approximation}
%====================================================================
\frame{\frametitle{Case of the Poisson log-normal model} 

  \bigskip
  \paragraph{Intractable E step.} No reasonably easy way to compute $p(Z_i | Y_i; \theta)$ under PLN.

  \pause \bigskip \bigskip
  \paragraph{Twisted problem.} Approximate it with some other (nice) distribution, e.g.
  $$
  p(Z_i | Y_i; \theta) \simeq \Ncal(m_i, S_i).
  $$
  \begin{itemize}
    \setlength{\itemsep}{0.5\baselineskip}
    \item Variational approximation \refer{WaJ08,BKM17} ('E' step $\to$ 'VE' step).
    \item Computationaly efficient (can deal with $\simeq 10^3$ samples or species).
  \end{itemize}
  
  \pause \bigskip \bigskip
  \paragraph{But...} 
  \begin{itemize}
    \setlength{\itemsep}{0.5\baselineskip}
    \item $\neq$ Maximum likelihood
    \item Very few statistical guaranty (consistency, asymptotic normality, asymptotic variance, ...).
    \item No direct test or confidence interval.
    \item Alternative methods or additional steps are needed (bootstrap, jackknife, Monte-Carlo).
  \end{itemize}

}

%====================================================================
\subsection{Variational inference}
%====================================================================
\frame{\frametitle{Variational inference} \pause

  \todo{TO DO}
}

%====================================================================
\frame{\frametitle{Parameter uncertainty} \pause

  \todo{TO DO}
}
