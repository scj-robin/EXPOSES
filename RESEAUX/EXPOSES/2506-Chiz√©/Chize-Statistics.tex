%====================================================================
%====================================================================
\section{Some statistical issues}
\frame{\frametitle{Outline} \tableofcontents[currentsection]}
%====================================================================
\subsection{Latent variable models}
%====================================================================
\frame{\frametitle{Latent variable models} 

  \bigskip 
  \begin{tabular}{cc}
    \hspace{-.04\textwidth}
    \begin{tabular}{p{.5\textwidth}}
      \paragraph{Latent variable model.} Three ingredients
      \begin{itemize}
        \setlength{\itemsep}{0.5\baselineskip}
        \item $Y =$ observed variables (data)
        \item $X =$ covariates (observed)
        \item $\theta =$ model parameters (unknown)
        \item $Z =$ latent variables (unobserved)
      \end{itemize}
    \end{tabular}
    &
    \begin{tabular}{p{.45\textwidth}}
      \begin{tabular}{p{.6\textwidth}}
        \begin{tikzpicture}
  \node[observed] (X) at (-1*\edgeunit, 1*\edgeunit) {$X$};
  \node[hidden] (Z) at (0*\edgeunit, 1*\edgeunit) {$Z$};
  \node[hidden] (theta) at (1*\edgeunit, 1*\edgeunit) {$\theta$};
  \node[observed] (Y) at (0, 0) {$Y$};

  \draw[dashedarrow] (X) to (Z);
  \draw[arrow] (theta) to (Z);
  \draw[arrow] (X) to (Y);
  \draw[arrow] (Z) to (Y);
  \draw[arrow] (theta) to (Y);
\end{tikzpicture}

      \end{tabular}
    \end{tabular}
  \end{tabular}

  \bigskip \pause
  \paragraph{Examples.} 
  \begin{itemize}
    \setlength{\itemsep}{0.5\baselineskip}
    \item Poisson log-normal model (PLN)
    $$
    Z_i = \text{Gaussian vector associated with site $i$}
    $$
    (encodes species dependencies) \\ ~
    \pause
    \item Stochastic block model (SBM)
    $$
    Z_i = \text{cluster to which species $i$ belongs}
    $$
    (encodes network structure)
  \end{itemize}
  
}

%====================================================================
\frame{\frametitle{Statistical issue} 

  \paragraph{Maximum likelihood inference.} Intractable likelihood (need to integrate over $Z$)
  $$
  p_\theta(Y) = \int p_\theta(Y \mid Z) p_\theta(Z) \d Z 
  $$
  while $p_\theta(Z)$ and $p_\theta(Y \mid Z)$ are easy to evaluate

  \bigskip \bigskip \pause
  \paragraph{Usual approach: EM algorithm \refer{DLR77}.} 
  \begin{itemize}
    \item E step: with current $\theta^{(h)}$, determine the conditional distribution
    $$
    p_{\theta^{(h)}}(Z \mid Y)
    $$
    \item M step: update $\theta^{(h+1)}$ as
    $$
    \theta^{(h+)} = \arg\max_\theta \Esp_{\theta^{(h)}}\left[\log p_\theta(Z \mid Y) \mid Y\right]
    $$
  \end{itemize}

}

%====================================================================
\subsection{Variational inference}
%====================================================================
\frame{\frametitle{Variational inference} 

  \paragraph{For both PLN and SBM.} Intractable E step: 
  $$
  \text{no close form or clever way to evaluate } p_{\theta^{(h)}}(Z \mid Y)
  $$
  
  \bigskip \bigskip \pause
  \paragraph{Variational inference \refer{BKM17}.} Approximate it
  $$
  q_{\psi^{(h)}(Z)} \simeq p_{\theta^{(h)}}(Z \mid Y)
  $$
  with a well chosen distribution $q_\psi \in \Qcal$ (\refer{DPR08} for SBM, \refer{CMR18a} for PLN)
  
  \bigskip \bigskip \pause
  \paragraph{Consequence.} Loose MLE properties
  \begin{itemize}
    \item consistency: $\widehat{\theta}_n \to \theta^*$
    \item asymptotic normality $\widehat{\theta}_n \approx \Ncal(\theta^*, \Var(\widehat{\theta}_n))$ (inc. confidence intervals)
    \item model selection criteria (AIC, BIC, ...)
    \item ...
  \end{itemize}

}

