\documentclass[11pt, a4paper]{article}
\usepackage{helvet} \renewcommand{\familydefault}{\sfdefault} 
%%% PAGE DIMENSIONS
\usepackage[margin=2.5cm, head=1.27cm]{geometry} % to change the page dimensions
\usepackage{xcolor}
\usepackage{fancyhdr} % 
\pagestyle{fancy} % 
\lhead{}
\chead{\textcolor{gray}{International Biometric Society}}
\rhead{}
\renewcommand{\headrulewidth}{0pt} % customise the layout...
%
\lfoot{}
\cfoot{International Biometric Conference, Dublin, 13�-18 July 2008}
\rfoot{}
%
\parindent 0pt
%%%
\begin{document}
%
%Please leave text above unchanged
%REplace below with your information
\begin{center}
  \textbf{Uncovering Latent Structure in Valued Graphs: A Variational
  Approach}\\[1em] 
  M.~Mariadassou$^1$ and S.~Robin$^1$\\[1em]
\end{center}
$^1$ UMR 518 AgroParisTech/INRA Applied Mathematics and Computer
Sciences, France \\

As more and more network-structured datasets are available, the
statistical analysis of valued graphs has become a common place.
Looking for a latent structure is one of the many strategies used to
better understand the behavior of a network. Several methods already
exist for the binary case. \\
We present a model-based strategy to uncover groups of nodes in valued
graphs. This framework can be used for a wide span of parametric
random graphs models. Variational tools allow us to achieve
approximate maximum likelihood estimation of the parameters of these
models. \\
We provide a simulation study showing that our estimation method
performs well over a broad range of situations. We analyse several
valued biological and ecological networks to detect underlying
structures.

\paragraph{Keywords:} Latent structure, Mixture model, Random graph,
Valued graph, Variational method 
 
\end{document}



