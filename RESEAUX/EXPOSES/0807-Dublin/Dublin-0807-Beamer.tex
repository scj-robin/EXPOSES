\documentclass[5pt]{beamer}
\mode<presentation>
\usetheme{Malmoe}
% \usetheme{Copenhagen}

%%%%%%%%%%%%%%%%%%%%%%%%%%%%%%%%%%%%%%%%%%%%%%%%%%%%%%%%%%%%%%%%%%%%%%%%%%
% Couleur et graphiques
%%%%%%%%%%%%%%%%%%%%%%%%%%%%%%%%%%%%%%%%%%%%%%%%%%%%%%%%%%%%%%%%%%%%%%%%%%
\usepackage{color}
\usepackage{graphics}
\usepackage{epsfig} 
\usepackage{pstcol}

%%%%%%%%%%%%%%%%%%%%%%%%%%%%%%%%%%%%%%%%%%%%%%%%%%%%%%%%%%%%%%%%%%%%%%%%%%
% Maths
%%%%%%%%%%%%%%%%%%%%%%%%%%%%%%%%%%%%%%%%%%%%%%%%%%%%%%%%%%%%%%%%%%%%%%%%%%
\newcommand{\Bcal}{\mathcal{B}}

%%%%%%%%%%%%%%%%%%%%%%%%%%%%%%%%%%%%%%%%%%%%%%%%%%%%%%%%%%%%%%%%%%%%%%%%%%
% Texte
%%%%%%%%%%%%%%%%%%%%%%%%%%%%%%%%%%%%%%%%%%%%%%%%%%%%%%%%%%%%%%%%%%%%%%%%%%
\usepackage[latin1]{inputenc}
\newcommand{\textblue}[1]{\textcolor{blue}{#1}}
\newcommand{\textgreen}[1]{\textcolor{green}{ #1}}
\newcommand{\paragraph}[1]{\noindent{\textblue{#1}}}
\newcommand{\refer}[1]{\alert{\sl #1}}

%%%%%%%%%%%%%%%%%%%%%%%%%%%%%%%%%%%%%%%%%%%%%%%%%%%%%%%%%%%%%%%%%%%%%%%%%%
% Title page
%%%%%%%%%%%%%%%%%%%%%%%%%%%%%%%%%%%%%%%%%%%%%%%%%%%%%%%%%%%%%%%%%%%%%%%%%%
\title{Uncovering (Global) Structure and (Local) Motifs in Networks}

\author{S. Robin \\
  {\tt robin@agroparistech.fr}}

\date[IBC '08]{International Biometric Conference, Dublin}

\institute{  
  AgroParisTech / INRA, Paris, \\
%    UMR 518 Math�matique et Informatique Appliqu�es \\
  \url{www.agroparistech.fr/mia/} \\
  ~\\
  {Statistics for Biological Sequences (SSB) group:} \\
  \url{genome.jouy.inra.fr/ssb/} \\
  ~\\
%  \noindent\includegraphics[scale=0.25]{../Figures/Logo-3.png}
     \includegraphics[scale=0.1]{../Figures/LogoINRA-Couleur.png}
     \qquad
     \includegraphics[scale=0.5]{../Figures/logagroptechsolo.png}
     \qquad
     \includegraphics[scale=0.2]{../Figures/../Figures/Logo-SSB.png}
  }

%%%%%%%%%%%%%%%%%%%%%%%%%%%%%%%%%%%%%%%%%%%%%%%%%%%%%%%%%%%%%%%%%%%%%%%%%%
%%%%%%%%%%%%%%%%%%%%%%%%%%%%%%%%%%%%%%%%%%%%%%%%%%%%%%%%%%%%%%%%%%%%%%%%%%
\begin{document}
%%%%%%%%%%%%%%%%%%%%%%%%%%%%%%%%%%%%%%%%%%%%%%%%%%%%%%%%%%%%%%%%%%%%%%%%%%
%%%%%%%%%%%%%%%%%%%%%%%%%%%%%%%%%%%%%%%%%%%%%%%%%%%%%%%%%%%%%%%%%%%%%%%%%%

%%%%%%%%%%%%%%%%%%%%%%%%%%%%%%%%%%%%%%%%%%%%%%%%%%%%%%%%%%%%%%%%%%%%%%%%%%
\begin{frame}
  \titlepage
\end{frame}

%%%%%%%%%%%%%%%%%%%%%%%%%%%%%%%%%%%%%%%%%%%%%%%%%%%%%%%%%%%%%%%%%%%%%%%%%%
\begin{frame}
  \frametitle{Outline}
  \tableofcontents
\end{frame}

%%%%%%%%%%%%%%%%%%%%%%%%%%%%%%%%%%%%%%%%%%%%%%%%%%%%%%%%%%%%%%%%%%%%%%%%%%
%%%%%%%%%%%%%%%%%%%%%%%%%%%%%%%%%%%%%%%%%%%%%%%%%%%%%%%%%%%%%%%%%%%%%%%%%%
\section{Looking for structure in networks}
%%%%%%%%%%%%%%%%%%%%%%%%%%%%%%%%%%%%%%%%%%%%%%%%%%%%%%%%%%%%%%%%%%%%%%%%%%

\begin{frame}
  \frametitle{Looking for structure in networks}
  \begin{columns}
    \column{.5\textwidth}
    \begin{block}{Networks}
      \begin{itemize}
      \item Arise in many fields:
        \begin{itemize}
        \item[\bf{$\rightarrow$}] Biology, Chemistry
        \item[\bf{$\rightarrow$}] Physics, Internet.
        \end{itemize} 
        ~\\
      \item Represent an interaction pattern:
        \begin{itemize}
        \item[\bf{$\rightarrow$}] $\mathcal{O}(n^2)$ interactions
        \item[\bf{$\rightarrow$}] between $n$ elements.
        \end{itemize} 
        ~\\
      \item Have a topology which:
        \begin{itemize}
        \item[\bf{$\rightarrow$}] reflects the structure / function
          relationship
        \end{itemize}
      \end{itemize}
    \end{block}
    \column{.5\textwidth}
    \begin{block}{}
       \includegraphics[angle=270,
       width=8.5cm]{../Figures/caida2_lg(brown).pdf} \\ 
      \begin{tiny} From Barab\'asi website \end{tiny}
    \end{block}
  \end{columns}
\end{frame}

%%%%%%%%%%%%%%%%%%%%%%%%%%%%%%%%%%%%%%%%%%%%%%%%%%%%%%%%%%%%%%%%%%%%%%%%%%
%%%%%%%%%%%%%%%%%%%%%%%%%%%%%%%%%%%%%%%%%%%%%%%%%%%%%%%%%%%%%%%%%%%%%%%%%%
\section{Mixture model for random graphs}
%%%%%%%%%%%%%%%%%%%%%%%%%%%%%%%%%%%%%%%%%%%%%%%%%%%%%%%%%%%%%%%%%%%%%%%%%%

%%%%%%%%%%%%%%%%%%%%%%%%%%%%%%%%%%%%%%%%%%%%%%%%%%%%%%%%%%%%%%%%%%%%
\subsection{Heterogeneity in networks}
%%%%%%%%%%%%%%%%%%%%%%%%%%%%%%%%%%%%%%%%%%%%%%%%%%%%%%%%%%%%%%%%%%%%%

\begin{frame}
  \frametitle{Nodes may have different connectivity behaviour}  
  \begin{block}{Looking for connected sub-groups:}
    \begin{itemize}
    \item Detection of cliques or groups of highly
      connected nodes: \refer{Gethor \& Diehl, 04} 
    \item Edge betweenness: \refer{Girvan \& Newman, 02}
    \item Spectral clustering: \refer{Von Luxburg \& al.,
        07}
    \end{itemize}
  \end{block}

  \pause
  \begin{block}{Model based:}
    \begin{itemize}
    \item Underlying topology: \refer{Hoff \& al., 02}
      (Latent space)
    \item Mixture model \refer{Nowicki \& Snijders, 01}
      (Block structure), \refer{Daudin \& al., 08} (Mixture for graphs)
    \item General model for heterogeneous networks:
      \refer{Bollob\'as \ al., 07}       
%       (Topological properties: Giant component, diameter, degree
%       distribution = compound Poisson, {\it etc.}).
    \item General review on random graph models:
      \refer{Pattison \& Robbins, 07}
    \end{itemize}
  \end{block}
\end{frame}

%%%%%%%%%%%%%%%%%%%%%%%%%%%%%%%%%%%%%%%%%%%%%%%%%%%%%%%%%%%%%%%%%%%%%
\begin{frame}
  \frametitle{{Random graph model}}
  We denote $i=1..n$ the \alert{fixed nodes} and $X_{ij}$ the
  \alert{random edges}. 
  
  A random graph $G$ is completely characterised by $n$ and the
  by the \alert{joint distribution} of the $X_{ij}$'s.
  
  The distribution of the degrees of the nodes is \alert{not sufficient}.

  \pause
  \begin{block}{Classical models for binary graphs}
    \begin{itemize}
    \item \paragraph{Erd�s-R�nyi:} Edges all exist with the same
      probability: 
      $$
      \{X_{ij}\} \text{ i.i.d.}, \qquad X_{ij} \sim \Bcal(\pi).
      $$
      \pause
    \item \paragraph{Expected Degree Distribution (EDD):}
      For given (expected) degrees of the nodes $\{k_1, .. k_n\}$,
      $$
      \{X_{ij}\} \text{ indep.}, \qquad \Pr\{X_{ij} = 1\} \propto
      \lambda k_i k_j. 
      $$
    \end{itemize}
  \end{block}
\end{frame}

%%%%%%%%%%%%%%%%%%%%%%%%%%%%%%%%%%%%%%%%%%%%%%%%%%%%%%%%%%%%%%%%%%%%%%%%%%
%%%%%%%%%%%%%%%%%%%%%%%%%%%%%%%%%%%%%%%%%%%%%%%%%%%%%%%%%%%%%%%%%%%%%%%%%%
\section{Variational inference}
%%%%%%%%%%%%%%%%%%%%%%%%%%%%%%%%%%%%%%%%%%%%%%%%%%%%%%%%%%%%%%%%%%%%%%%%%%

%%%%%%%%%%%%%%%%%%%%%%%%%%%%%%%%%%%%%%%%%%%%%%%%%%%%%%%%%%%%%%%%%%%%%%%%%%
%%%%%%%%%%%%%%%%%%%%%%%%%%%%%%%%%%%%%%%%%%%%%%%%%%%%%%%%%%%%%%%%%%%%%%%%%%
\section{Motifs in interaction graphs}
%%%%%%%%%%%%%%%%%%%%%%%%%%%%%%%%%%%%%%%%%%%%%%%%%%%%%%%%%%%%%%%%%%%%%%%%%%

%%%%%%%%%%%%%%%%%%%%%%%%%%%%%%%%%%%%%%%%%%%%%%%%%%%%%%%%%%%%%%%%%%%%%%%%%%
%%%%%%%%%%%%%%%%%%%%%%%%%%%%%%%%%%%%%%%%%%%%%%%%%%%%%%%%%%%%%%%%%%%%%%%%%%
\section{Approximate distribution for the count}
%%%%%%%%%%%%%%%%%%%%%%%%%%%%%%%%%%%%%%%%%%%%%%%%%%%%%%%%%%%%%%%%%%%%%%%%%%

%%%%%%%%%%%%%%%%%%%%%%%%%%%%%%%%%%%%%%%%%%%%%%%%%%%%%%%%%%%%%%%%%%%%%%%%%%
%%%%%%%%%%%%%%%%%%%%%%%%%%%%%%%%%%%%%%%%%%%%%%%%%%%%%%%%%%%%%%%%%%%%%%%%%%
\section{Discussion \& Work in progress}
%%%%%%%%%%%%%%%%%%%%%%%%%%%%%%%%%%%%%%%%%%%%%%%%%%%%%%%%%%%%%%%%%%%%%%%%%%

%%%%%%%%%%%%%%%%%%%%%%%%%%%%%%%%%%%%%%%%%%%%%%%%%%%%%%%%%%%%%%%%%%%%%%%%%%
%%%%%%%%%%%%%%%%%%%%%%%%%%%%%%%%%%%%%%%%%%%%%%%%%%%%%%%%%%%%%%%%%%%%%%%%%%
\end{document}
%%%%%%%%%%%%%%%%%%%%%%%%%%%%%%%%%%%%%%%%%%%%%%%%%%%%%%%%%%%%%%%%%%%%%%%%%%
%%%%%%%%%%%%%%%%%%%%%%%%%%%%%%%%%%%%%%%%%%%%%%%%%%%%%%%%%%%%%%%%%%%%%%%%%%
