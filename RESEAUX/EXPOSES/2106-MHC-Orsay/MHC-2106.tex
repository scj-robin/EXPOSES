\documentclass[8pt]{beamer}

% Beamer style
%\usetheme[secheader]{Madrid}
% \usetheme{CambridgeUS}
\useoutertheme{infolines}
\usecolortheme[rgb={0.65,0.15,0.25}]{structure}
% \usefonttheme[onlymath]{serif}
\beamertemplatenavigationsymbolsempty
%\AtBeginSubsection

% Packages
%\usepackage[french]{babel}
\usepackage[latin1]{inputenc}
\usepackage{color}
% \usepackage[dvipsnames]{xcolor}
\usepackage{xspace}
\usepackage{dsfont, stmaryrd}
\usepackage{amsmath, amsfonts, amssymb, stmaryrd, mathabx}
\usepackage{epsfig}
\usepackage{tikz}
\usepackage{url}
% \usepackage{ulem}
\usepackage{/home/robin/LATEX/Biblio/astats}
%\usepackage[all]{xy}
\usepackage{graphicx}

% Maths
% \newtheorem{theorem}{Theorem}
% \newtheorem{definition}{Definition}
\newtheorem{proposition}{Proposition}
% \newtheorem{assumption}{Assumption}
% \newtheorem{algorithm}{Algorithm}
% \newtheorem{lemma}{Lemma}
% \newtheorem{remark}{Remark}
% \newtheorem{exercise}{Exercise}
% \newcommand{\propname}{Prop.}
% \newcommand{\proof}{\noindent{\sl Proof:}\quad}
% \newcommand{\eproof}{$\blacksquare$}

% \setcounter{secnumdepth}{3}
% \setcounter{tocdepth}{3}
\newcommand{\pref}[1]{\ref{#1} p.\pageref{#1}}
\newcommand{\qref}[1]{\eqref{#1} p.\pageref{#1}}

% Colors : http://latexcolor.com/
\definecolor{darkred}{rgb}{0.65,0.15,0.25}
\definecolor{darkgreen}{rgb}{0,0.4,0}
\definecolor{darkred}{rgb}{0.65,0.15,0.25}
\definecolor{amethyst}{rgb}{0.6, 0.4, 0.8}
\definecolor{asparagus}{rgb}{0.53, 0.66, 0.42}
\definecolor{applegreen}{rgb}{0.55, 0.71, 0.0}
\definecolor{awesome}{rgb}{1.0, 0.13, 0.32}
\definecolor{blue-green}{rgb}{0.0, 0.87, 0.87}
\definecolor{red-ggplot}{rgb}{0.52, 0.25, 0.23}
\definecolor{green-ggplot}{rgb}{0.42, 0.58, 0.00}
\definecolor{purple-ggplot}{rgb}{0.34, 0.21, 0.44}
\definecolor{blue-ggplot}{rgb}{0.00, 0.49, 0.51}

% Commands
\newcommand{\backupbegin}{
   \newcounter{finalframe}
   \setcounter{finalframe}{\value{framenumber}}
}
\newcommand{\backupend}{
   \setcounter{framenumber}{\value{finalframe}}
}
\newcommand{\emphase}[1]{\textcolor{darkred}{#1}}
\newcommand{\comment}[1]{\textcolor{gray}{#1}}
\newcommand{\paragraph}[1]{\textcolor{darkred}{#1}}
\newcommand{\refer}[1]{{\small{\textcolor{gray}{{\cite{#1}}}}}}
\newcommand{\Refer}[1]{{\small{\textcolor{gray}{{[#1]}}}}}
\newcommand{\goto}[1]{{\small{\textcolor{blue}{[\#\ref{#1}]}}}}
\renewcommand{\newblock}{}

\newcommand{\tabequation}[1]{{\medskip \centerline{#1} \medskip}}
% \renewcommand{\binom}[2]{{\left(\begin{array}{c} #1 \\ #2 \end{array}\right)}}

% Variables 
\newcommand{\Abf}{{\bf A}}
\newcommand{\Beta}{\text{B}}
\newcommand{\Bcal}{\mathcal{B}}
\newcommand{\Bias}{\xspace\mathbb B}
\newcommand{\Cor}{{\mathbb C}\text{or}}
\newcommand{\Cov}{{\mathbb C}\text{ov}}
\newcommand{\cl}{\text{\it c}\ell}
\newcommand{\Ccal}{\mathcal{C}}
\newcommand{\cst}{\text{cst}}
\newcommand{\Dcal}{\mathcal{D}}
\newcommand{\Ecal}{\mathcal{E}}
\newcommand{\Esp}{\xspace\mathbb E}
\newcommand{\Espt}{\widetilde{\Esp}}
\newcommand{\Covt}{\widetilde{\Cov}}
\newcommand{\Ibb}{\mathbb I}
\newcommand{\Fcal}{\mathcal{F}}
\newcommand{\Gcal}{\mathcal{G}}
\newcommand{\Gam}{\mathcal{G}\text{am}}
\newcommand{\Hcal}{\mathcal{H}}
\newcommand{\Jcal}{\mathcal{J}}
\newcommand{\Lcal}{\mathcal{L}}
\newcommand{\Mt}{\widetilde{M}}
\newcommand{\mt}{\widetilde{m}}
\newcommand{\Nbb}{\mathbb{N}}
\newcommand{\Mcal}{\mathcal{M}}
\newcommand{\Ncal}{\mathcal{N}}
\newcommand{\Ocal}{\mathcal{O}}
\newcommand{\pt}{\widetilde{p}}
\newcommand{\Pt}{\widetilde{P}}
\newcommand{\Pbb}{\mathbb{P}}
\newcommand{\Pcal}{\mathcal{P}}
\newcommand{\Qcal}{\mathcal{Q}}
\newcommand{\qt}{\widetilde{q}}
\newcommand{\Rbb}{\mathbb{R}}
\newcommand{\Sbb}{\mathbb{S}}
\newcommand{\Scal}{\mathcal{S}}
\newcommand{\st}{\widetilde{s}}
\newcommand{\St}{\widetilde{S}}
\newcommand{\Tcal}{\mathcal{T}}
\newcommand{\todo}{\textcolor{red}{TO DO}}
\newcommand{\Ucal}{\mathcal{U}}
\newcommand{\Un}{\math{1}}
\newcommand{\Vcal}{\mathcal{V}}
\newcommand{\Var}{\mathbb V}
\newcommand{\Vart}{\widetilde{\Var}}
\newcommand{\Zcal}{\mathcal{Z}}

% Symboles & notations
\newcommand\independent{\protect\mathpalette{\protect\independenT}{\perp}}\def\independenT#1#2{\mathrel{\rlap{$#1#2$}\mkern2mu{#1#2}}} 
\renewcommand{\d}{\text{\xspace d}}
\newcommand{\gv}{\mid}
\newcommand{\ggv}{\, \| \, }
% \newcommand{\diag}{\text{diag}}
\newcommand{\card}[1]{\text{card}\left(#1\right)}
\newcommand{\trace}[1]{\text{tr}\left(#1\right)}
\newcommand{\matr}[1]{\boldsymbol{#1}}
\newcommand{\matrbf}[1]{\mathbf{#1}}
\newcommand{\vect}[1]{\matr{#1}} %% un peu inutile
\newcommand{\vectbf}[1]{\matrbf{#1}} %% un peu inutile
\newcommand{\trans}{\intercal}
\newcommand{\transpose}[1]{\matr{#1}^\trans}
\newcommand{\crossprod}[2]{\transpose{#1} \matr{#2}}
\newcommand{\tcrossprod}[2]{\matr{#1} \transpose{#2}}
\newcommand{\matprod}[2]{\matr{#1} \matr{#2}}
\DeclareMathOperator*{\argmin}{arg\,min}
\DeclareMathOperator*{\argmax}{arg\,max}
\DeclareMathOperator{\sign}{sign}
\DeclareMathOperator{\tr}{tr}
\newcommand{\ra}{\emphase{$\rightarrow$} \xspace}

% Hadamard, Kronecker and vec operators
\DeclareMathOperator{\Diag}{Diag} % matrix diagonal
\DeclareMathOperator{\diag}{diag} % vector diagonal
\DeclareMathOperator{\mtov}{vec} % matrix to vector
\newcommand{\kro}{\otimes} % Kronecker product
\newcommand{\had}{\odot}   % Hadamard product

% TikZ
\newcommand{\nodesize}{2em}
\newcommand{\edgeunit}{2.5*\nodesize}
\newcommand{\edgewidth}{1pt}
\tikzstyle{node}=[draw, circle, fill=black, minimum width=.75\nodesize, inner sep=0]
\tikzstyle{square}=[rectangle, draw]
\tikzstyle{param}=[draw, rectangle, fill=gray!50, minimum width=\nodesize, minimum height=\nodesize, inner sep=0]
\tikzstyle{hidden}=[draw, circle, fill=gray!50, minimum width=\nodesize, inner sep=0]
\tikzstyle{hiddenred}=[draw, circle, color=red, fill=gray!50, minimum width=\nodesize, inner sep=0]
\tikzstyle{observed}=[draw, circle, minimum width=\nodesize, inner sep=0]
\tikzstyle{observedred}=[draw, circle, minimum width=\nodesize, color=red, inner sep=0]
\tikzstyle{eliminated}=[draw, circle, minimum width=\nodesize, color=gray!50, inner sep=0]
\tikzstyle{empty}=[draw, circle, minimum width=\nodesize, color=white, inner sep=0]
\tikzstyle{blank}=[color=white]
\tikzstyle{nocircle}=[minimum width=\nodesize, inner sep=0]

\tikzstyle{edge}=[-, line width=\edgewidth]
\tikzstyle{edgebendleft}=[-, >=latex, line width=\edgewidth, bend left]
\tikzstyle{edgebendright}=[-, >=latex, line width=\edgewidth, bend right]
\tikzstyle{lightedge}=[-, line width=\edgewidth, color=gray!50]
\tikzstyle{lightedgebendleft}=[-, >=latex, line width=\edgewidth, bend left, color=gray!50]
\tikzstyle{lightedgebendright}=[-, >=latex, line width=\edgewidth, bend right, color=gray!50]
\tikzstyle{edgered}=[-, line width=\edgewidth, color=red]
\tikzstyle{edgebendleftred}=[-, >=latex, line width=\edgewidth, bend left, color=red]
\tikzstyle{edgebendrightred}=[-, >=latex, line width=\edgewidth, bend right, color=red]

\tikzstyle{arrow}=[->, >=latex, line width=\edgewidth]
\tikzstyle{arrowbendleft}=[->, >=latex, line width=\edgewidth, bend left]
\tikzstyle{arrowbendright}=[->, >=latex, line width=\edgewidth, bend right]
\tikzstyle{arrowred}=[->, >=latex, line width=\edgewidth, color=red]
\tikzstyle{arrowbendleftred}=[->, >=latex, line width=\edgewidth, bend left, color=red]
\tikzstyle{arrowbendrightred}=[->, >=latex, line width=\edgewidth, bend right, color=red]
\tikzstyle{arrowblue}=[->, >=latex, line width=\edgewidth, color=blue]
\tikzstyle{dashedarrow}=[->, >=latex, dashed, line width=\edgewidth]
\tikzstyle{dashededge}=[-, >=latex, dashed, line width=\edgewidth]
\tikzstyle{dashededgebendleft}=[-, >=latex, dashed, line width=\edgewidth, bend left]
\tikzstyle{lightarrow}=[->, >=latex, line width=\edgewidth, color=gray!50]

\newcommand{\bEDD}{BEDD\xspace}
\newcommand{\Fbar}{\overline{F}}
\newcommand{\phibar}{\overline{\phi}}

% Directory
\newcommand{\fignet}{/home/robin/RECHERCHE/RESEAUX/EXPOSES/FIGURES}
\newcommand{\figbayes}{/home/robin/RECHERCHE/BAYES/EXPOSES/FIGURES}

%====================================================================
%====================================================================

%====================================================================
%====================================================================
\begin{document}
%====================================================================
%====================================================================

%====================================================================
\title[Network Poisson models]{Network Poisson models: \\ ~\\
From variational estimates to Bayesian inference \\}

\author[S. Robin]{S. Robin \\ \medskip
joint work with S. Donnet \refer{DoR21}\\ ~}

\institute[]{univ. Paris-Saclay / AgroParisTech / INRAE / Museum National d'Histoire Naturelle \\~ \\~}

\date[MHC, Orsay, Jun.'21]{Mixtures, Hidden Markov Models, Clustering, Orsay, Jun. 2021}

%====================================================================
%====================================================================
\maketitle
%====================================================================
\frame{\frametitle{Outline} \tableofcontents}
%====================================================================

%====================================================================
%====================================================================
\section{Network Poisson model}
%\frame{\frametitle{Outline} \tableofcontents[currentsection]}
%====================================================================
\frame{\frametitle{Interaction networks} 

  \bigskip
  \paragraph{Counting interactions.} 
  Consider $n$ entities = nodes ($1 \leq i, j \leq n$)
  \begin{itemize}
   \item $Y_{ij} =$ number of interactions observed between $i$ and $j$
   \item $x_{ij} =$ vector of covariates specific to $(i, j)$
  \end{itemize}
  
  \pause
  \begin{tabular}{cc}
    \hspace{-.04\textwidth}
    \begin{tabular}{p{.65\textwidth}}
      \paragraph{Zebra social network:} 
      \begin{itemize}
      \item $Y_{ij} =$ number of social contacts between individual $i$ and $j$
      \item $x_{ij} =$ same sex, same age, same both
      \end{itemize}
    \end{tabular}
    & 
    \hspace{-.05\textwidth}
    \begin{tabular}{p{.3\textwidth}}
      \includegraphics[width=.25\textwidth]{\figbayes/DoR21-Fig6-Zebra}
    \end{tabular}
    \\ \pause
    \hspace{-.04\textwidth}
    \begin{tabular}{p{.65\textwidth}}
      \paragraph{Tree species parasitic network:} 
      \begin{itemize}
      \item $Y_{ij} =$ number of fungal parasites shared by species $i$ and $j$
      \item $x_{ij} =$ taxonomic, distance, genetic distances \\ ~
      \end{itemize}
    \end{tabular}
    & 
    \hspace{-.05\textwidth}
    \begin{tabular}{p{.3\textwidth}}
      \includegraphics[width=.25\textwidth]{\figbayes/DoR21-Fig6-Tree}
    \end{tabular}
  \end{tabular}

}

%====================================================================
\frame{\frametitle{Network Poisson model} 

  \paragraph{Aim.} 
  \begin{itemize}
   \item Account / assess the effect of the covariates on the number of interactions
   \item Exhibit some {\sl residual} structure
  \end{itemize}

  \bigskip \bigskip \pause
  \paragraph{Poisson stochastic blockmodel, with covariates.} \refer{MRV10}
  \begin{itemize}
   \item Assume that nodes are spread among $K$ groups (= blocks = clusters)
   \item $Z_i =$ group membership of node $i$:
   $$
   \Pr\{Z_i = k\} = \pi_k, \qquad 1 \leq k \leq K
   $$
   \item \pause Emission distribution:
   $$
   Y_{ij} \sim \Pcal(\exp(\alpha_{Z_iZ_j} + x_{ij}^\intercal \beta))
   $$
   \item $\alpha_{k\ell} =$ effect of the group memberships
   \item $\beta =$ vector of regression coefficients
  \end{itemize}
  
  \bigskip \pause
  \paragraph{Unknowns:} 
  Parameter $\theta = (\pi, \alpha, \beta)$ + latent $Z = \{Z_i\}_{1 \leq i \leq n}$

}

%====================================================================
%====================================================================
\section{Variational inference}
%\frame{\frametitle{Outline} \tableofcontents[currentsection]}
%====================================================================
\frame{\frametitle{Variational approximation (1/2)} 

  \paragraph{Stochastic blockmodels (SBM) = incomplete data models.}
  \begin{itemize}
  \setlength{\itemsep}{1.25\baselineskip}
  \item Intractable likelihood: $\displaystyle{p_\theta(Y) = \sum_{z \in \llbracket K \rrbracket^n} p_\theta(Y \mid Z=z) P_\theta(Z = z)}$
  \item \pause Intractable conditional distribution: $p_\theta(Z \mid Y)$ (no nice factorization) \\
  \ra Cannot resort to a regular EM algorithm
  \end{itemize}

  \bigskip \bigskip \pause
  \paragraph{Variational approximation \refer{WaJ08,BKM17}.}
  \begin{itemize}
  \setlength{\itemsep}{1.25\baselineskip}
  \item \pause Replace the E-step with an approximation step (VE-step):
  $$
  q^{h+1} = \arg\min_{q \in \Qcal} KL\left(q(Z) \;||\; p_{\theta^h}(Z \mid Y)\right)
  $$
  \item \pause VEM aims at maximizing a lower bound of the log-likelihood ('ELBO'):
  \begin{align*}
  J(Y; \theta, q) 
  & = \log p_\theta(Y) - KL\left(q(Z) \;||\; p_\theta(Z \mid Y)\right) \\
  \\
  & = \Esp_q6 \left[\log p_\theta(Y, Z)\right] - \Esp_q \left[\log q(Z)\right]
  \end{align*}
  \end{itemize}
}

%====================================================================
\frame{\frametitle{Variational approximation (2/2)} 

  \paragraph{Approximation class.}   
  Standard choice for SBMs = product-form distribution \refer{DPR08}: 
  $$
  \Qcal = \left\{q: q(Z) = \prod_i q_i(Z_i)\right\}
  $$
  \ra mean-field approximation

  \bigskip \bigskip \pause
  \paragraph{Theoretical guaranties.}
  \begin{itemize}
  \setlength{\itemsep}{1.25\baselineskip}
  \item Consistency and asymptotic normality for binary SBMs without covariates \refer{CDP12,BCC13,MaM15}
  \item No generic property for variational estimates
  \item No measure of uncertainty for the estimates
  \item (But VEM runs fast and is $\sim $ easy to implement, see e.g. \url{blockmodels} \refer{Leg16})
  \end{itemize}

  \bigskip \pause
  \paragraph{Aim of this work.} 
  Use frequentist VEM inference to achieve (efficient) Bayesian inference for Poisson SBM with covariates
  
}

%====================================================================
%====================================================================
\section{From VEM to SMC}
%\frame{\frametitle{Outline} \tableofcontents[currentsection]}
%====================================================================
\frame{\frametitle{Bayesian framework} 

  \paragraph{Frequentist inference.} VEM provides
  $$
  (\widetilde{\theta}, \qt) = \arg\max_{\theta, q \in \Qcal} J(Y; \theta, q)
  $$

  \bigskip \bigskip \pause
  \paragraph{Bayesian inference.} 
  \begin{itemize}
  \setlength{\itemsep}{1.25\baselineskip}
  \item Prior distributions
  $$
  \pi \sim \Dcal(e_0), \qquad \qquad \gamma := (\alpha, \beta) \sim \Ncal(\gamma_0, V_0)
  $$
  \item Aim = (sample from the) posterior distributions
  $$
  p(\theta \mid Y), \qquad \qquad p(\theta, Z \mid Y)
  $$
  \item \pause Two-step inference: \\ ~ \\
    1 -- derive a proxy of the posterior $p(\theta \mid Y)$ based on $(\widetilde{\theta}, \qt)$ \\ ~ \\
    2 -- use sequential Monte-Carlo to get a sample from the true posterior
  \end{itemize}

}

% %====================================================================
% \frame{\frametitle{Guesstimate of the posterior variance} 
% 
%   \paragraph{Louis formulas \refer{Lou82}.} 
%   Compute Fisher information using EM side-products:
%   $$
%   \partial^2_{\theta^2} \log p_\theta(Y) 
%   = \Esp\left(\partial^2_{\theta^2} \log p_\theta(Y, Z) \mid Y \right) 
%   + \Var\left(\partial_\theta \log p_\theta(Y, Z) \mid Y \right)
%   $$
%   
%   \bigskip \bigskip \pause
%   \paragraph{Approximate variance for VEM estimates.} Take
%   \begin{align*}
%   \widetilde{\Var}\left(\widetilde{\theta}\right)^{-1} 
%   & \simeq \Esp_{\qt}\left(\partial^2_{\theta^2} \log p_\theta(Y, Z) \right) 
%   + \Var_{\qt}\left(\partial_\theta \log p_\theta(Y, Z) \right) \\
%   & \simeq \Esp_{\qt}\left(\partial^2_{\theta^2} \log p_\theta(Y, Z) \right) 
%   & & \text{as } \Var_{\qt}(\cdot) \simeq 0.
%   \end{align*}
%   
%   \bigskip \bigskip \pause
%   \paragraph{Approximate posterior.} Combine the prior with $\widetilde{\Var}\left(\widetilde{\theta}\right)$ to define the proxy
%   $$
%   \pt(\pi) = \Dcal(\widetilde{e}), \qquad \qquad
%   \pt(\gamma) = \Ncal(\widetilde{\gamma}, \widetilde{V}).
%   $$
%   
% }

%====================================================================
\frame{\frametitle{Approximate posterior distribution} 

  \paragraph{No simple variational \emphase{Bayes} algorithm here.}
  But variational approximation + Taylor expansion yields
  \begin{align*}
    p(\theta \mid Y) 
    & \propto \; \exp\left(\log p(\theta) + \log p_\theta(Y)\right) \\ ~ \\
    & \simeq \; \exp\left(\log p(\theta) + J(Y; \theta, \widetilde{q})\right) \\~ \\
    & \simeq \; \exp\left(\log p(\theta) + J(Y; \widetilde{\theta}, \widetilde{q}) + \frac12 (\theta - \widetilde{\theta})^\intercal \nabla^2_\theta\left(J(Y; \theta, \widetilde{q})\right) (\theta - \widetilde{\theta}) \right)
  \end{align*}
  where $\nabla^2(\cdot)$ stands for the Hessian matrix.

  \bigskip \bigskip \pause
  \paragraph{Approximate posterior.} Because $\nabla^2_\theta\left(J\right)$ is block diagonal, we end up with the proxy \refer{DoR21}
  $$
  \pt(\theta) = \Dcal(\pi; \widetilde{e}) \times \Ncal(\gamma; \widetilde{\gamma}, \widetilde{V}).
  $$
  and we set $\pt(Z) = \qt(Z)$.
}

%====================================================================
\frame{\frametitle{Sequential Monte-Carlo} 

  \begin{tabular}{cc}
    \hspace{-.04\textwidth}
    \begin{tabular}{p{.6\textwidth}}
      \paragraph{General principle.} \refer{DDJ06}
      \begin{itemize}
      \setlength{\itemsep}{1.25\baselineskip}
      \item $\textcolor{red}{\pt} = $ proposal, $\textcolor{blue}{p^*} = $ target
      \item Intermediate distributions
      $$
      \pt = p_0, p_1, ..., p_H = p^*
      $$
      \item Iteratively 
      use $p_{h-1}$ to get a weighted sample from $p_h$
      $$
      \Ecal_h = \{(U_h^m; w_h^m)\}_{1 \leq m \leq M}
      $$
      where $U = (\theta, Z)$.
      \end{itemize}
    \end{tabular}
    & 
    \hspace{-.05\textwidth}
    \begin{tabular}{p{.4\textwidth}}
      \includegraphics[width=.35\textwidth]{../FIGURES/FigVBEM-IS-Tempering.pdf}
    \end{tabular}
  \end{tabular}
  
  \bigskip \bigskip \pause
  \paragraph{Distribution path.} $0 = \rho_0 < \rho_1 < \dots < \rho_{H-1} < \rho_H = 1$,
  \begin{align*}
     p_h(U) & \propto \; \pt(U)^{\emphase{{1-\rho_h}}} \; \times \; p(U | Y)^{\emphase{{\rho_h}}} \\
%      \\
     & \propto \; \pt(U) \; \times \; r(U)^{\emphase{{\rho_h}}}, 
     & r(U) & = \frac{p(U) p(Y | U)}{\pt(U)}
  \end{align*}
  where $\rho_h$ can be tuned adaptively to guaranty a prescribed ESS at each step.

}

%====================================================================
\frame{\frametitle{What do we gain?} 

  \begin{itemize}
  \setlength{\itemsep}{2\baselineskip}
  \item \paragraph{Time:} synthetic data ($n = 40$, $K=2$, $d=4$ covariates, $M=2000$ particles)
  $$
  \begin{array}{c}
    \includegraphics[width=.4\textwidth]{\figbayes/DoR21-Fig1-simu-rho} \\
    \text{\textcolor{blue-ggplot}{starting from $\pt(U)$}, \qquad \qquad  \textcolor{awesome}{starting from $p(U)$}}
  \end{array}
  $$
  \item \pause \paragraph{All relevent quantities} posterior inference
  \end{itemize}

}
%====================================================================
\frame{\frametitle{Model selection} 

  \paragraph{Posterior distribution of the number of groups.} Given a prior $P(K)$, we look for
  $$
  P(K \mid Y) \; \propto \; P(K)p(Y \mid K)
  $$
  
  \bigskip \bigskip \pause
  \paragraph{Marginal likelihood.} \refer{DDJ06} introduce the estimate
  $$
  \widehat{p}(Y \mid K) = \prod_{h=1}^H \sum_m W_h^m r(U_h^m)^{\rho_h - \rho_{h-1}}
  $$
  
  \bigskip \bigskip \pause
  \paragraph{Model averaging.} $P(K \mid Y)$ can be use to choose $K$ or to average with respect to it, e.g.
  $$
  p(\beta \mid Y) = \sum_k P(K \mid Y) \; p(\beta \mid Y, K)
  $$

}

%====================================================================
%====================================================================
\section{Illustrations}
%\frame{\frametitle{Outline} \tableofcontents[currentsection]}
%====================================================================
\frame{\frametitle{Tree network} 

  \paragraph{Data.} 
  $n = 51$ tree species, $d = 3$ covariates (distances), $Y_{ij} =$ number of shared parasites
  
  \bigskip \bigskip \pause
  \paragraph{Sampling path and model selection.} 
  $$
  \begin{array}{ccc}
    \text{Model selection} & & \text{Distribution path ($K = 5$)} \\
    \includegraphics[width=.3\textwidth]{\figbayes/DoR21-Fig7-logpY} & \qquad &
    \includegraphics[width=.3\textwidth]{\figbayes/DoR21-Fig7-rho} \\
    \text{\textcolor{awesome}{ELBO}, \textcolor{amethyst}{ICL}, \textcolor{green-ggplot}{$\log p(Y \mid K)$}} & & 
    \text{$\rho_h = f(h)$}
  \end{array}
  $$
  
}

%====================================================================
\frame{\frametitle{Posterior for the regression coefficients} 

  $$
  \begin{array}{ccc}
    \text{taxonomy ($\beta_1$)} & \text{geography ($\beta_2$)} & \text{phylogeny ($\beta_3$)} \\
    \includegraphics[width=.25\textwidth]{\figbayes/DoR21-Fig7-beta1} & 
    \includegraphics[width=.25\textwidth]{\figbayes/DoR21-Fig7-beta2} & 
    \includegraphics[width=.25\textwidth]{\figbayes/DoR21-Fig7-beta3} \\
    \multicolumn{3}{c}{\text{\textcolor{awesome}{proxy $\pt(\beta_j)$}, 
    \qquad \textcolor{green-ggplot}{posterior $p(\beta_j \mid Y, \widehat{K})$} \qquad \textcolor{amethyst}{averaged $p(\beta_j \mid Y)$} (dotted)}}
  \end{array}
  $$
  \bigskip 
  $$
  P(\text{taxonomy} \mid Y) = 52.1\%, \qquad 
  P(\text{taxonomy + geography} \mid Y) = 46.8\%
  $$
  
  \bigskip \bigskip \pause 
  \paragraph{Why does it take 25 steps to go from $\pt(\beta)$ to $p(\beta \mid Y)$?}
  
  \begin{itemize}
  \item Partly because of correlations:
  $$
  \begin{array}{l|ccc}
    & (\beta_1, \beta_2) & (\beta_1, \beta_3) & (\beta_2, \beta_3) \\
    \hline
    \pt(\beta) & -0.037 & -0.010 & 0.235 \\
    p(\beta \mid Y) & -0.139 & -0.017 & 0.325
  \end{array}
  $$
  \end{itemize}
}

%====================================================================
\frame{\frametitle{Posterior distribution of the latent variables} 

  \begin{itemize}
  \item ... but mostly because of the sampling of the $Z_i$'s.
  \end{itemize}
  
  \bigskip \pause 
  \begin{tabular}{cc}
    \hspace{-.04\textwidth}
    \begin{tabular}{p{.6\textwidth}}
      The proxy $\pt$ assumes their conditional independence:
      $$
      \pt(Z) = \prod_i \pt(Z_i)
      $$
      \bigskip \pause \\
      At each step of the SMC, we may evaluate their mutual information
      $$
      MI_h(Z) = KL\left(p_h(Z) \;||\; \prod_i p_h(Z_i)\right)
      $$
    \end{tabular}
    & 
    \hspace{-.05\textwidth}
    \begin{tabular}{c}
      \includegraphics[width=.35\textwidth]{\figbayes/DoR21-Fig7-MI} \\
      $MI_h = f(h)$
    \end{tabular}
  \end{tabular}

}


%====================================================================
%====================================================================
\section{Conclusion}
%\frame{\frametitle{Outline} \tableofcontents[currentsection]}
%====================================================================
\frame{\frametitle{To summarize} 

  \paragraph{Combining VEM with SMC.}
  \begin{itemize}
    \setlength{\itemsep}{1.25\baselineskip}
    \item VEM is efficient and empirically accurate, but does not come along with desirable guaranties
    \item SMC enable to sample from the posterior, and is efficient provided that the initial distribution is not to far from the target
    \item Combining the two enables us to make proper statistical inference
  \end{itemize}

  \bigskip \bigskip \pause
  \paragraph{Available extensions.}
  \begin{itemize}
    \setlength{\itemsep}{1.25\baselineskip}
    \item \pause $K=1$ corresponds to a standard Poisson regression model, with no residual structure \\
    \ra $P(K = 1 \mid Y)$ is a goodness-of-fit measure for the regression
    \item \pause Stochastic block models are specific instances of $w$-graph models \\
    \ra Averaging over $K$ provides an estimation of the 'residual graphon' \refer{LaR16}
    \item \pause The methodology can be extended to many other models such as logistic regression, latent class analysis, ... \refer{DoR17}
  \end{itemize}

}

% %====================================================================
% \frame{\frametitle{Some comments} 
% 
% }

%====================================================================
\frame[allowframebreaks]{ \frametitle{References}
  {
   \tiny
   \bibliography{/home/robin/Biblio/BibGene}
   \bibliographystyle{alpha}
  }
}

%====================================================================
\backupbegin
%====================================================================

%====================================================================
%====================================================================
\section*{Backup}
%====================================================================
\frame{\frametitle{Sequential importance sampling scheme}

  Denote
  $$
  \emphase{U = (\theta, Z)}, \qquad
  \pi = \text{ prior}, \qquad
  \ell = \text{ likelihood}
  $$

  \bigskip
  \paragraph{Distribution path:} 
    set $0 = \rho_0 < \rho_1 < \dots < \rho_{H-1} < \rho_H = 1$,
  \begin{align*}
     p_h(U) & \propto \pt(U)^{\emphase{{1-\rho_h}}} \; \times \; p(U | Y)^{\emphase{{\rho_h}}} \\
%      \\
     & \propto \pt(U) \; \times \; r(U)^{\emphase{{\rho_h}}}, 
     & r(U) & = \frac{\pi(U) \ell(Y | U)}{\pt(U)}
  \end{align*}
  
  \bigskip \bigskip \pause
  \paragraph{Sequential sampling.} At each step $h$, provides
  $$
  \Ecal_h = \{(U_h^m, w_h^m)\}_m = \text{ weighted sample of } p_h
  $$

  \bigskip \pause
  \paragraph{Question.} 
  How to tune $\{\rho_h\}$ or $H$ to keep each sampling step efficient?

}
  
%====================================================================
\frame{\frametitle{Proposed algorithm}

  \begin{description}
   \item[Init.:] Sample $(U_0^m)_m$ iid $\sim \pt$, $w_0^m = 1$ \\ ~
   \pause
   \item[Step $h$:] Using the previous sample $\Ecal_{h-1} = \{(U_{h-1}^m, w_{h-1}^m)\}$ \\ ~
   \pause
   \begin{enumerate}
    \item set $\rho_h$ such that $cESS(\Ecal_{h-1}; p_{h-1}, p_h) = \emphase{\tau_1}$ \\ ~
    \pause
    \item compute $w_h^m = \emphase{w_{h-1}^m \times (r_h^m)^{\rho_h - \rho_{h-1}}}$ \\ ~
    \pause
    \item (\footnote{To avoid degeneracy. Weights set to 1 after it.}) if $ESS_h = \overline{w}_h^2 / \overline{w_h^2} < \emphase{\tau_2}$, resample the particles  \\ ~
    \pause
    \item (\footnote{$K_h$ has stationary distribution \emphase{$p_h$} (e.g. Gibbs sampler). Only propagation: no convergence needed}) propagate the particles $U_h^m \sim \emphase{K_h}(U_h^m | U_{h-1}^m)$ 
   \end{enumerate} ~ 
   \pause
   \item[Stop:] When $\rho_h$ reaches 1.
  \end{description}
  
  \bigskip \pause
  \paragraph{Justification \refer{DDJ06}.} At each step $h$, construct a distribution for the whole particle path with marginal $p_h$.

}
  
%====================================================================
\frame{\frametitle{Adaptive step size}

  \paragraph{Conditional ESS:} efficiency of sample $\Ecal$ from $q$ for distribution $p$
  $$
  cESS(\Ecal; q, p) = \frac{M \left(\sum_m W^m a^m\right)^2}{\sum_m W^m (a^m)^2}, 
  \qquad a^m = \frac{p(U^m)}{q(U^m)}
  $$
  \ra Step 1: find next $p_h$ s.t. sample $\Ecal_{h-1}$ is reasonably efficient.
  
%   \bigskip \bigskip \bigskip \pause
%   \paragraph{ESS:} intrinsic efficiency of a sample 
%   $$
%   ESS(\{(U^m, w^m)\}) = \frac{M \left(\sum_m w^m\right)^2}{\sum_m (w^m)^2}
%   $$
%   \ra Step 1c: resample if to few particles actually contribute
  
  \bigskip \bigskip \pause
  Update formula of the weights
  $$
  cESS(\Ecal_{h-1}; p_{h-1}, p_h) 
  = 
  \frac
  {M \left[\sum_m W_{h-1}^m \; (\emphase{r^m_{h-1}})^{\rho_h -\rho_{h-1}}\right]^2}
  {\sum_m W_{h-1}^m \; (\emphase{r^m_{h-1}})^{2\rho_h - 2\rho_{h-1}}}
  $$
  \ra can be computed for any $\rho_h$ \emphase{before sampling}.
  
  \bigskip
  \ra $\rho_h$ tuned to meet $\tau_1$, which controls the step size $\rho_h - \rho_{h-1}$ (and $H$)
}

%====================================================================
\frame{\frametitle{'Residual graphon'}

  $$
  \includegraphics[width=.8\textwidth]{\figbayes/DoR21-JRSSC-Fig8}
  $$
  
}

%====================================================================
\frame{\frametitle{Accounting for individual effects}

  \paragraph{Model posterior probabilities.}
  \begin{center}
%   {\small \begin{tabular}{l|rr|rr|rr|rr}
%     & \multicolumn{8}{c}{Networks} \\
%     & & & & & & & & \\
%     & \multicolumn{2}{c|}{Trees} & \multicolumn{2}{c|}{Flowers} & \multicolumn{2}{c|}{Zebras} & \multicolumn{2}{c}{Onagers} \\
%     & \multicolumn{2}{c|}{$(n=51)$} & \multicolumn{2}{c|}{$(n=105)$} & \multicolumn{2}{c|}{$(n=28)$} & \multicolumn{2}{c}{$(n=29)$} \\    
%     & & & & & & & & \\    
%     Model $M$ & ($a$) & ($b$) & ($a$) & ($b$) & ($a$) & ($b$) & ($a$)& ($b$) \\
%     \hline
%     & & & & & & & & \\
%     1st & -1382.9 & 0 & -3734.1 & 0 & -358.0 & 0 & -699.7 & 0 \\
%     1st + 2nd & -1383.0 & 0 & -- & -- & -362.2 & 0 & -- & -- \\
%     & & & & & & & & \\
%     \hline
%     & & & & & & & & \\
%     degree & -1288.1 & 0 & -3661.0 & 98.4 & -333.3 & 28.9 & -594.1 & 3.7\\ 
%     degree + 1st & -1263.7 & 97.4 & -3665.1 & 1.6 & -332.4 & 71.1 & -590.9 & 96.3\\ 
%     degree + 1st + 2nd & -1267.3 & 2.6 & -- & -- & -- & -- & -- & -- \\ 
%     & & & & & & & & \\
%     \hline
%     & & & & & & & & \\
%     indiv. & -1505.9 & 0 & -4165.2 & 0 & -428.7 & 0 & -718.2 & 0 \\ 
%     indiv. + 1st & -1443.3 & 0 & -4186.7 & 0 & -429.5 & 0 & -685.2 & 0 \\ 
%     indiv. + 1st + 2nd & -1446.1 & 0 & -- & -- & -- & -- & -- & -- \\
%   \end{tabular} }
  {\small \begin{tabular}{lrrrr}
    & \multicolumn{4}{c}{Networks} \\
    & & & \\
    & Trees &  Flowers &  Zebras &  Onagers \\
    Model $M$ & $(n=51)$ & $(n=105)$ & $(n=28)$ & $(n=29)$ \\    
    & & & \\    
    \hline
    & & & \\    
    1st best covariate & 0 & 0 & 0 & 0 \\
    1st + 2nd & 0 & -- & 0 & -- \\
    & & & \\    
    \hline
    & & & \\    
    degree & 0 & 98.4 & 28.9 & 3.7\\ 
    degree + 1st & 97.4 & 1.6 & 71.1 & 96.3\\ 
    degree + 1st + 2nd & 2.6 & -- & -- & -- \\ 
    & & & \\    
    \hline
    & & & \\    
    indiv. & 0 & 0 & 0 & 0 \\ 
    indiv. + 1st & 0 & 0 & 0 & 0 \\ 
    indiv. + 1st + 2nd & 0 & -- & -- & -- \\
  \end{tabular} }
  \end{center}
  
  \bigskip 
  {
  \begin{itemize}
%   \item $(a) = \log P(Y \mid M)$, $(b) = P(M \mid Y)$
%   \item 1st (2nd) = 1st (2nd) best covariate
  \item Prior variance for the regression coefficients: $v_0 = 10$
  \end{itemize}
  }

}

%====================================================================
\backupend
%====================================================================

%====================================================================
%====================================================================
\end{document}
%====================================================================
%====================================================================
  
  \hspace{-.025\textwidth}
  \begin{tabular}{cc}
    \begin{tabular}{p{.5\textwidth}}
    \end{tabular}
    & 
    \hspace{-.02\textwidth}
    \begin{tabular}{p{.5\textwidth}}
    \end{tabular}
  \end{tabular}

