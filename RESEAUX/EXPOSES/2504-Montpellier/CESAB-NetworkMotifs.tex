%==================================================================
%==================================================================
\section{Bipartite networks and motifs}
\frame{\frametitle{Outline} \tableofcontents[currentsection]}
%==================================================================
\frame{\frametitle{Bipartite network}

  \begin{tabular}{ll}
    \begin{tabular}{p{.45\textwidth}}
      \paragraph{Two types of actors.}
      \begin{itemize}
      \item Mutualistic: plant-pollinator
      \item Antagonistic: host-parasite
      \end{itemize}
      \bigskip \bigskip 
      \paragraph{Topological analysis:} \\
      understanding the network organisation \\
      \bigskip
      \paragraph{Local:} % \\
      node or edge properties (degree, betweenness) \\
      \bigskip
      \paragraph{Global:} % \\
      density, connected components, nestedness 
    \end{tabular}
    &
    \begin{tabular}{p{.45\textwidth}}
      \paragraph{Zackenberg network: } 
      \refer{SRO16} \\
      \includegraphics[width=.4\textwidth, trim=0 50 0 50]{\fignet/Zackenberg-1996_12-red-net} \\
%       \includegraphics[width=.4\textwidth, trim=0 50 0 50]{\fignet/Zackenberg-1996_12-red-adj}
    \end{tabular} 
  \end{tabular}
}

%==================================================================
\frame{\frametitle{Bipartite network: notations}

  \begin{tabular}{ll}
    \begin{tabular}{p{.6\textwidth}}
      \paragraph{Species.}
      \begin{itemize}
        \setlength{\itemsep}{1.15\baselineskip}
        \item $i = 1, \dots m$ pollinators = rows = bottom nodes
        \item $j = 1, \dots n$ plants = columns = top nodes
      \end{itemize}
    \end{tabular}
    &
    \hspace{-.15\textwidth}
    \begin{tabular}{p{.4\textwidth}}
      \includegraphics[width=.4\textwidth, trim=0 50 0 50]{\fignet/Zackenberg-1996_12-red-net} 
    \end{tabular} 
    \\ \pause
    \begin{tabular}{p{.6\textwidth}}
      \paragraph{Interactions.}
      \begin{itemize}
      \item $A_{ij} = 1$ if pollinator $i$ interacts with plant $j$, \\
      0 otherwise
      $$
      A_{ij} = 1 \quad \Leftrightarrow \quad i \sim j
      $$
      \item adjacency matrix : $m \times n$
      $$
      A = [A_{ij}]_{1 \leq i \leq m, 1 \leq j \leq n}
      $$
      \end{itemize}
    \end{tabular}
    &
    \hspace{-.15\textwidth}
    \begin{tabular}{p{.5\textwidth}}
      \includegraphics[width=.4\textwidth, trim=0 50 0 50]{\fignet/Zackenberg-1996_12-red-adj}
    \end{tabular} 
  \end{tabular}
}

%==================================================================
\frame{\frametitle{Bipartite motifs}

  \label{sec:motifList}
  \begin{tabular}{ll}
    \hspace{-.04\textwidth}
    \begin{tabular}{p{.45\textwidth}}
      \paragraph{'Meso-scale' analysis.} \refer{SCB19}
      \begin{itemize}
       \item Motifs ='building-blocks'
       \item between local (several nodes) and global (sub-graph)
      \end{itemize}
      \bigskip \bigskip 
      \onslide+<2->{
      \paragraph{Interest.}
      \begin{itemize}
       \item Generic description of a network
       \item Enables network comparison
       \item Even when the nodes are different
      \end{itemize} \\
      ($+$ 'species-role': out of the scope here) \\
      }
      \bigskip \bigskip 
      \onslide+<3->{
      \paragraph{Existing tool.} 
      \url{bmotif} package \refer{SSS19}: \\
      counts motif occurrences \goto{sec:motifProba} \\
      (\emphase{Not an easy task!})
      }
    \end{tabular}
    &
    \begin{tabular}{p{.45\textwidth}} 
      \hspace{-0.035\textwidth}
      \includegraphics[width=.5\textwidth]{\fignet/SCB19-Oikos-Fig3-6motifs} \\      
    \end{tabular} 
  \end{tabular}

}

%==================================================================
\frame{\frametitle{Example}

  \begin{tabular}{ll}
    \begin{tabular}{p{.45\textwidth}}
      \paragraph{Plant-pollinator network} \refer{SRO16} \\
      \hspace{-.05\textwidth}
      \includegraphics[width=.45\textwidth, trim=0 50 0 50]{\fignet/Zackenberg-1996_12-red-net} \\
    \end{tabular}
    &
    \begin{tabular}{p{.45\textwidth}}
      \paragraph{Motif counts.}  \\ ~\\ 
      4 nodes (species) \\ 
      \includegraphics[width=.09\textwidth]{\fignet/Zackenberg-1996_12-red-motif5} 
      \includegraphics[width=.09\textwidth]{\fignet/Zackenberg-1996_12-red-motif6} \\ ~\\
      5 nodes \\ 
      \includegraphics[width=.09\textwidth]{\fignet/Zackenberg-1996_12-red-motif9} 
      \includegraphics[width=.09\textwidth]{\fignet/Zackenberg-1996_12-red-motif10} 
      \includegraphics[width=.09\textwidth]{\fignet/Zackenberg-1996_12-red-motif11} 
      \includegraphics[width=.09\textwidth]{\fignet/Zackenberg-1996_12-red-motif12} \\
      \includegraphics[width=.09\textwidth]{\fignet/Zackenberg-1996_12-red-motif13} 
      \includegraphics[width=.09\textwidth]{\fignet/Zackenberg-1996_12-red-motif14} 
      \includegraphics[width=.09\textwidth]{\fignet/Zackenberg-1996_12-red-motif15} 
      \includegraphics[width=.09\textwidth]{\fignet/Zackenberg-1996_12-red-motif16} 
    \end{tabular}  
    ~ \\
    \hline
    \\ \pause
    \begin{tabular}{p{.45\textwidth}}
      top 'stars' (plants) \\
      \includegraphics[width=.09\textwidth]{\fignet/Zackenberg-1996_12-red-motif1} 
      \includegraphics[width=.09\textwidth]{\fignet/Zackenberg-1996_12-red-motif2} 
      \includegraphics[width=.09\textwidth]{\fignet/Zackenberg-1996_12-red-motif7} 
      \includegraphics[width=.09\textwidth]{\fignet/Zackenberg-1996_12-red-motif17} 
    \end{tabular}
    &    
    \begin{tabular}{p{.45\textwidth}}
      bottom 'stars' (pollinators) \\
      \includegraphics[width=.09\textwidth]{\fignet/Zackenberg-1996_12-red-motif1} 
      \includegraphics[width=.09\textwidth]{\fignet/Zackenberg-1996_12-red-motif3} 
      \includegraphics[width=.09\textwidth]{\fignet/Zackenberg-1996_12-red-motif4} 
      \includegraphics[width=.09\textwidth]{\fignet/Zackenberg-1996_12-red-motif8} 
    \end{tabular}
  \end{tabular}
  
}

