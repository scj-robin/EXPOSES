%====================================================================
%====================================================================
\section{Bayesian inference via sequential Monte-Carlo}
\subsection*{Principle}
\frame{\frametitle{Outline} \tableofcontents[currentsection]}
%====================================================================
\frame{\frametitle{Bayesian inference} 

  \paragraph{General aim.}
  \begin{align*}
   \text{prior:} & & \theta & \sim p_0(\theta) \\
   \text{likelihood:} & & Y \mid \theta & \sim \ell(Y \mid \theta) \\
   \text{posterior:} & & \theta \mid Y & \sim \emphase{p(\theta \mid Y)}
  \end{align*}
  \ra Latent variable model: 'posterior' $= p(\theta, Z \mid Y)$

  
  \bigskip \bigskip \bigskip \pause
  \paragraph{Strategy for SBM.} 
  \begin{itemize}
  \item use VEM to get a first estimate of $\theta$ (and $Z$)
  \item use VEM side-products to get an approximate posterior $\pt(\theta, Z)$
  \item design an efficient stochastic algorithm to sample from $p(\theta, Z \mid Y)$ using $\pt(\theta, Z)$ as a proposal
  \end{itemize}

}

%====================================================================
\frame{\frametitle{VEM-based proposal}

  \paragraph{VEM + Laplace approximation:}
  \begin{itemize}
  \item Define 
  $$
  (U_{VEM}, q_{VEM}) = \arg\max_{q \in \Qcal} J(U, q)
  $$
  \item Evaluate
  \begin{align*}
   \log p(U \mid Y) 
   & = \cst + \log p_0(U) + \log \ell(Y \mid U) \\ 
   & \onslide+<2->{\simeq \cst + \log p_0(U) + \emphase{J(U, q_{VEM})}} \\
   & \onslide+<3->{\simeq \cst + \log p_0(U) + J(\emphase{U_{VEM}}, q_{VEM})} \\
   & \onslide+<3->{\quad + \frac12 (U - U_{VEM})^\intercal 
   \left(\left.\partial^2_{U^2} J(U, q_{VEM})\right|_{U = U_{VEM}}\right)
   (U - U_{VEM})} \\
   & \onslide+<4->{=: \cst + \log \emphase{\pt(U)}}
  \end{align*}
  \end{itemize}
  
  \bigskip \onslide+<5->{
  \paragraph{Case of SBM.}
  \begin{itemize}
   \item Gaussian prior for $(\alpha, \beta) \Rightarrow$ Gaussian $\pt(\alpha, \beta)$
   \item Similar approximation to get Dirichlet independent $\pt(\pi)$ and $\pt(Z_i)$
  \end{itemize}
  }

}

%====================================================================
\subsection*{SMC}
%====================================================================
\frame{\frametitle{Monte Carlo sampling} 

  \paragraph{Importance sampling.} First idea: use $\pt$ to sample directly from the posterior 
  \begin{itemize}
   \item Poor effective sample size (ESS): few particles with non-zero weight
  \end{itemize}

  \bigskip \bigskip \pause
  \hspace{-.03\textwidth} 
  \begin{tabular}{ll}
    \begin{tabular}{p{.5\textwidth}}
      \paragraph{Sequential Monte-Carlo \refer{DDJ06}.} ~ \\
      $U = (\theta, Z)$
      \begin{itemize}
      \item start with an initial proposal $\pt$
      \item define a sequence of distributions 
      $$
      (q_h)_{0 \leq h \leq H}: \qquad q_H(U) = p(U \mid Y)
      $$
      \item iteratively sample: 
      $$
      S^h = (U^{h, m})_{1 \leq m \leq M}
      $$ 
      from $q_h$ using $S^{h-1}$
      \end{itemize}
    \end{tabular}
    &
    \hspace{-.05\textwidth} 
    \begin{tabular}{c}
    \includegraphics[width=.4\textwidth]{../FIGURES/FigVBEM-IS-Tempering} \\
    $\textcolor{red}{q_0}, q_1, \dots, q_H = \textcolor{blue}{p^*}$, 
    \end{tabular}
  \end{tabular}

}

%====================================================================
\frame{\frametitle{Proposed SMC sampling scheme}

  \paragraph{Distribution path:} 
    set $0 = \rho_0 < \rho_1 < \dots < \rho_{H-1} < \rho_H = 1$,
  \begin{align*}
     q_h(U) & \propto \pt(U)^{\emphase{{1-\rho_h}}} \; \times \; p(U | Y)^{\emphase{{\rho_h}}} \\
%      \\
     & \propto \pt(U) \; \times \; r(U)^{\emphase{{\rho_h}}}, 
     & r(U) & = \frac{p_0(U) \ell(Y | U)}{\pt(U)}
  \end{align*}
  
  \bigskip \bigskip \pause
  \paragraph{Sequential sampling.} At each step $h$, provides
  $$
  \Ecal_h = \{(U_h^m, w_h^m)\}_m = \text{ weighted sample of } q_h
  $$

  \bigskip \pause
  \paragraph{Theoretical justification: \refer{DDJ06}.} 
  At each step $h$, construct a distribution for the whole particle path with marginal $p_h$.

  \bigskip \bigskip \pause
  \paragraph{Additional aim.} 
  Tune (adaptively) $\{\rho_h\}$ to keep each sampling step efficient.

}
  
%====================================================================
\frame{\frametitle{Algorithm}
  
%  \paragraph{Algorithm.} 
  \begin{description}
  \item[Init.:] Sample $(U_0^m)_m$ iid $\sim \pt$, $w_0^m = 1$ \\ ~
  \pause
  \item[Step $h$:] Using the previous sample $\Ecal_{h-1} = \{(U_{h-1}^m, w_{h-1}^m)\}$ \\ ~
  \pause
  \begin{enumerate}
  \item set $\rho_h$ such that $cESS(\Ecal_{h-1}; q_{h-1}, q_h) = \emphase{\tau_1}$ \\ ~
  \pause
  \item compute $w_h^m = \emphase{w_{h-1}^m \times (r_h^m)^{\rho_h - \rho_{h-1}}}$ \\ ~
  \pause
  \item 
  (\footnote{To avoid degeneracy. Weights set to 1 after it.}) 
  if $ESS_h = \overline{w}_h^2 / \overline{w_h^2} < \emphase{\tau_2}$, resample the particles  \\ ~
  \pause
  \item 
  (\footnote{$K_h$ has stationary distribution \emphase{$q_h$} (e.g. Gibbs sampler). Only propagation: no convergence needed}) 
  propagate the particles $U_h^m \sim \emphase{K_h}(U_h^m | U_{h-1}^m)$ 
  \end{enumerate} ~
  \pause
  \item[Stop:] When $\rho_h$ reaches 1.
  \end{description}

}

%====================================================================
\subsection*{Illustrations}
%====================================================================
\frame{\frametitle{Tree network}

  \hspace{-.025\textwidth}
  \begin{tabular}{ll}
   \begin{tabular}{p{.35\textwidth}}
    \paragraph{From \refer{VPD08}.} 
    $n = 51$ tree species 

    \bigskip
    $Y_{ij} = $ number of shared fungal parasites 

    \bigskip
    3 covariates (distances): \\
    taxonomy, geography, genetics 

    \bigskip
    \paragraph{Poisson model:} \\
    $(Y_{ij} \mid Z_i=k, Z_j=\ell)$ \\ \smallskip
    $\qquad \sim \Pcal\left(e^{\alpha_{k\ell} + x_{ji}^\intercal \beta}\right)$ 
   \end{tabular}
   &
   \hspace{-.05\textwidth}
   \begin{tabular}{cc}
    \includegraphics[width=.25\textwidth]{\figtree/Tree-all-V10-M5000-net} & 
    \includegraphics[width=.25\textwidth]{../FIGURES/FigVBEM-IS-Tree-TaxonomicDistance} \\
    Interactions & Taxonomic dist. \\
    \includegraphics[width=.25\textwidth]{../FIGURES/FigVBEM-IS-Tree-GeographicDistance} & 
    \includegraphics[width=.25\textwidth]{../FIGURES/FigVBEM-IS-Tree-GeneticDistance}  \\
    Geographic dist. & Genetic dist.
   \end{tabular}
  \end{tabular}

}

%====================================================================
\frame{\frametitle{Sampling path \& choice of $K$}

  \paragraph{Full model.} All covariates

  \bigskip
  \begin{center}
    \begin{tabular}{ccc}
    $\widehat{p}(K \mid Y)$ & & Sampling path: $\rho_h$ \\
    \includegraphics[width=.3\textwidth]{\figtree/Tree-all-V10-M5000-logpY} &
    \qquad &
    \includegraphics[width=.3\textwidth]{\figtree/Tree-all-V10-M5000-rho} \\
    \textcolor{blue}{$J_K$}, \textcolor{green}{$\widetilde{BIC}$}, \textcolor{red}{$\widetilde{ICL}$} &
    \qquad &
    $\widehat{K} = \arg\max_K \widehat{p}(K \mid Y)$
    \end{tabular}
  \end{center}
  
  \bigskip 
  \paragraph{Goodness of fit.} $\widehat{P}(K = 1 \mid Y) \approx 0$ \\
  \ra The covariates do not explain the whole network structure.
}

%====================================================================
\frame{\frametitle{Regression coefficient}

  \vspace{-.05\textheight}
  \begin{center}
    \begin{tabular}{ccc}
    taxonomy & geography & genetics \\
    \includegraphics[width=.3\textwidth]{\figtree/Tree-all-V10-M5000-beta1} & 
    \includegraphics[width=.3\textwidth]{\figtree/Tree-all-V10-M5000-beta2} & 
    \includegraphics[width=.3\textwidth]{\figtree/Tree-all-V10-M5000-beta3} \\
    \multicolumn{3}{c}{\textcolor{blue}{$\pt(\beta \mid \widehat{K})$}, \quad  \textcolor{red}{$\widehat{p}(\beta \mid Y, \widehat{K})$}, \quad $\widehat{p}(\beta \mid Y) = \sum_K \widehat{p}(K \mid Y) \widehat{p}(\beta \mid Y, K)$}
    \end{tabular}
  \end{center}  

  \bigskip \pause
  \hspace{-.025\textwidth}
  \begin{tabular}{rrrr}
    \paragraph{Correlation between estimates.} 
    & $(\beta_1, \beta_2)$ & $(\beta_1, \beta_3)$ & $(\beta_2, \beta_3)$ \\
    $\pt(\beta)$ & $-0.037$ & $-0.010$ & $0.235$ \\
    $\widehat{p}(\beta \mid Y)$ & $-0.139$ & $-0.017$ & $0.325$
  \end{tabular}

  \bigskip \pause
  \paragraph{Model posterior.} 
  (taxo., geo.)= $46.5\%$, (taxo.)= $34.5\%$ , (taxo., gene.)= $18.0\%$
  
%   \hspace{-.025\textwidth}
%   \begin{tabular}{rrrr}
%     \paragraph{Model selection.} 
%     & (taxo., geo.) & (taxo.) & (taxo., gene.) \\ 
%     $\widehat{P}(\text{model} \mid Y)$ & $46.5\%$ & $34.5\%$ & $18.0\%$
%   \end{tabular}
}

%====================================================================
\frame{\frametitle{Social network of equid species}

  \paragraph{2 contact networks \refer{RSF15}.}
  \begin{itemize}
  \item $n = 28$ zebras, $n = 29$ onagers
  \item sex and age (juvenile / adult) recorded
  \end{itemize}
  
  \bigskip \bigskip
  \paragraph{Question.} Are the two networks organized according to the same variables.
  
  \bigskip \bigskip \pause
  \paragraph{Model posterior.} 
  $$
  \begin{tabular}{rrrrr}
    & () & (sex) & (age) & (sex, age) \\ \hline
    Zebras & $\simeq0\%$ & $98.7\%$ & $\simeq0\%$ & $1.3\%$ \\
    Onagers & $.4\%$ & $1.3\%$ & $98.3\%$ & $\simeq0\%$ \\
  \end{tabular}
  $$
  
  \bigskip \pause
  \paragraph{Goodness of fit.} 
  $$
  \widehat{P}(K = 1 \mid Y) \simeq 0 \quad \text{for all models}
  $$
  \ra Individual effects remain.

}
  

