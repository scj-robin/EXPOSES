\documentclass[]{article}

%% Use this file to enter your data, LaTeX-process it to a pdf-file
%% and submit it via SPA-2018 abstract submission service

\usepackage{amssymb}
\usepackage{amsfonts}
\usepackage{amsmath}
\usepackage{amsthm}
\oddsidemargin -10 mm
\topmargin -11 mm
\textheight 225 mm
\textwidth 178 mm
\newtheorem*{theorem}{Theorem}
\newtheorem*{acknowledgement}{Acknowledgement}
\newtheorem*{condition}{Condition}
\newtheorem*{conjecture}{Conjecture}
\newtheorem*{corollary}{Corollary}
\newtheorem*{criterion}{Criterion}
\newtheorem*{definition}{Definition}
\newtheorem*{example}{Example}
\newtheorem*{lemma}{Lemma}
\newtheorem*{notation}{Notation}
\newtheorem*{proposition}{Proposition}
\newtheorem*{remark}{Remark}

\begin{document}
\thispagestyle{empty}

% Top matter
%--------------------------------------------------------------------------------

\begin{flushleft}

{\Large\bfseries A CONTINOUS TIME STOCHASTIC BLOCK MODEL FOR TIME-STAMPED INTERACTIONS} \\ % CAPITALIZED
\vspace {3mm}

{\large Matthew Ludkin} % the first co-Author(s) NOT presenting at
                       % SPA-2018, if any
{\it Lancaster University, United Kingdom}, E-mail: m.ludkin1@lancaster.ac.uk \\
\vspace {1mm}

{\large Catherine Matias}  % the speaker PRESENTING the work at SPA-2018
{\it Sorbonne Universit\'es, Universit\'e Pierre  et Marie Curie, Universit\'e Paris Diderot,  Centre National de la Recherche Scientifique, France}, E-mail: catherine.matias@math.cnrs.fr \\
\vspace {1mm}

% Add more authors, if any
{\large \bfseries St\'ephane Robin}   
{\it AgroParisTech, Institut National de la Recherche Agronomique, Universit� Paris-Saclay, France}, E-mail:
email@address.org \\
\vspace{2mm}

{\large \bfseries Keywords:} \textsc{interaction process, network, stochastic block model, variational inference} \\ 

\end{flushleft}

% Abstract
%--------------------------------------------------------------------------------
\vspace{3mm}
\noindent{\bf Abstract}:  

With the increasing amount of interaction data collected along time, an increasing interest has been paid in the last few years to dynamic network models. This work introduces a novel model and algorithm for inferring group structure in continuous time networks where every pairwise interaction is recorded. The proposed model is a dynamic and continuous-time version of the stochastic block-model. Our aim is to decipher how actors of a given system may change their role along time and therefore change their interaction rules with the other actors. We perform approximate maximum likelihood inference based on simplifying assumptions combined with a variational expectation maximization algorithm. The predicted group membership of each actor at each time is obtained as a by product. We also define two model selection criteria to choose the number of groups. We study the performances of the proposed algorithm on a series of synthetic datasets and we illustrate the use of the proposed modeling on datasets arising from ecology and social networks.

% \textbf{Acknowledgement:} This work is partially funded by the COSTNET EU COST Action CA15109 and by the SNA program from the I. Newton Institute, UK.


%References
%---------------------------------------------------------------------------------------------------------------
% \vspace*{5mm}
% \noindent{\large\bf References}
% \medskip
% \renewcommand{\labelenumi}{[\arabic{enumi}]}
% \begin{enumerate}
% % The Author's name should be in the format: Initial(s).~Family name,
% % such as K.~It\^{o} or A.~A.~Markov 
% % (the tilde is unbreakable space in LaTeX)
% % Or just follow "plain" bibtex format
% \item M.~Author, N.~Author and A.~B.~Author. Paper Title. {\it Journal},
%   {\bf Volume},  Year, start page-end page.
% \item N.~Author and A.~B.~Author. {\it Book title}, Publisher. Year
% 
% \end{enumerate}

\end{document}

%% Use this file to enter your data, LaTeX-process it to a pdf-file
%% and submit it via SPA-2018 abstract submission service
