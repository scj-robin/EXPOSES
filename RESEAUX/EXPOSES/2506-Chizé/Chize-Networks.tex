%====================================================================
%====================================================================
\section{Ecological networks}
\frame{\frametitle{Outline} \tableofcontents[currentsection]}
%====================================================================
\subsection{Some definitions}
\frame{\frametitle{Ecological networks} 
%====================================================================

  \paragraph{General aim.} 
  For a given ecosystem, understand the interactions between
  \begin{itemize}
    \setlength{\itemsep}{0.5\baselineskip}
    \item the species ({\sl biotic}) and
    \item the species and their environment ({\sl abiotic}).
  \end{itemize}

  \bigskip \bigskip 
  \paragraph{Ecological network =} 
  representation of the existing interactions between the species (biotic)
  
  \bigskip \bigskip \pause
  \begin{tabular}{cc}
    \hspace{-.04\textwidth}
    \begin{tabular}{p{.5\textwidth}}
      \paragraph{Mathematical counterpart.} 
      Network = graph
      $$
      \Gcal = (\Vcal, \Ecal)
      $$
      \begin{itemize}
        \setlength{\itemsep}{0.6\baselineskip}
        \item $\Vcal =$ set of vertices = nodes (= species)
        \item $\Ecal \subset \Vcal \times \Vcal$ set of edges = pairs\footnote{interactions involve only two species} of nodes \\ (= interactions)
      \end{itemize}
    \end{tabular}
    &
    \begin{tabular}{p{.4\textwidth}}
      \includegraphics[width=.35\textwidth]{\fignet/FigGGM-4nodes-red}
    \end{tabular}
  \end{tabular}
  
}

%====================================================================
\frame{\frametitle{Various types of networks}

  \paragraph{Networks can be} \refer{Kol09}
  \begin{itemize}
    \setlength{\itemsep}{0.75\baselineskip}
    \item \emphase{undirected} ($i \sim j$) or directed ($(i \to j) \neq (j \to i)$) \\
    \item \emphase{uni-partite} (one type of node) or multi-partite (several types) \\
    \textcolor{gray}{plant-pollinator networks are bipartite}
    \item binary (presence/absence of the edges) or \emphase{weighted} (valued edges) \\
    \textcolor{gray}{plant-pollinator networks are weighted when counting the number of visits of an insect on a plant}
    \item with \emphase{one} or several types of edges \\
    \textcolor{gray}{scientific networks: co-authorship, same lab, same project, ...}
    \item \emphase{static} or dynamic,
    \item 'multi-layer', \dots
  \end{itemize}

  \bigskip \bigskip \pause
  More importantly,
  networks can be \emphase{observed or not}

}

%====================================================================
\subsection{Two problems}
%====================================================================
\frame{\frametitle{Two typical problems} 

  \paragraph{Problem 1: Network inference.}
  The network is unknown.
  \begin{itemize}
    \item Aim: infer it from available data.
    \item Typical setting: species of abundances recorded in different sites (or at different times):
    $$
    \text{abundances} \sim \Fcal(\text{covariates}, \text{parameters}, \emphase{\text{network } \Gcal})
    $$
  \end{itemize}
  
  \bigskip \bigskip \pause
  \paragraph{Problem 2: Network structure.}
  The network is observed.
  \begin{itemize}
    \item Aim: understand its organisation or structure (e.g. understand the specific roles of the different species in the network)
    \item Typical understand the specific roles of the different species in the network
    $$
    \emphase{\text{network } \Gcal} \sim \Fcal(\text{covariates}, \text{parameters})
    $$
  \end{itemize}

}
