% !TEX root = Topology-MIA518.tex
\section{Introduction } 
\frame{\frametitle{Introduction}
 \cite{Kolaczyk:2009}
\begin{itemize}
\item Object of interest : \emphase{network data}
\item General definition from  the Oxford English Dictionary :  \emphase{A network  is a collection of interconnected things}
\item Network data come from $4$ large domains
\end{itemize}

\begin{columns}[t]
  \begin{column}{5cm}
\begin{itemize}
\item Technology
\item Information


\end{itemize}
  \end{column}
    \begin{column}{5cm}
\begin{itemize}
\item Sociology
\item Biology

\end{itemize}
  \end{column}
  \end{columns}
\vspace{1em}
\begin{itemize}
\item Each domain leads to datasets with specific properties and characteristics
\end{itemize}
}


%%%%%%%%%%%%%%%%%%%%%%%%%%%%%%%%%%%%%%%%%%%%%%%%%%%%%
\subsection{Examples}


\frame{\frametitle{Examples of technological networks}

\begin{itemize}

\item Communication networks (telephone, INTERNET)
\item Transportation networks (roads, tails, airlines routes)
\item Energy networks (delivery of gas or electricity) 

\item \textbf{Common interest in this case} :

 \begin{itemize}
\item Topology of a large (giant?) network 
  \item  Flow a  certain object in the network (Internet traffic packets, units of electrical energy...)
  \item Representation? 
  \item Sampling? 
\end{itemize}
\end{itemize}
}

%%%%%%%%%%%%%%%%%%%%%%%%%%%%%%%%%%%%%%%%%%%%%%%%%%%%%

\frame{\frametitle{Example of sociological network}

\begin{itemize}
\item Dating back to at least the 1930's
\item Networks representing the interactions among a collection of social entities or ``actors''
\begin{itemize}
\item Friendships among people
\item Contacts (sexual contacts, meetings between members of terrorist cells... )
\item Exchange of resources (emails, advices,...) 
\item Corporate alliances among businesses
\item Trade agreement among nations
\end{itemize}


\item \emphase{Questions}
\begin{itemize}
\item Who interacts with whom and which factors influence the presence/ absence of relation
\item Are interactions mutual (reciprocity)?
\item Are my friends' friends my friends?  
\item Can I find social groups? 
\item Who is central / who is peripherical? 
\item Can I find actors with similar roles in the network? 
\item Study of small networks (school, club) or large networks (Facebook / Linked in / Twitter...)
\end{itemize}
\end{itemize}
}

%%%%%%%%%%%%%%%%%%%%%%%%%%%%%%%%%%%%%%%%%%%%%%%%%%%%%
\frame{\frametitle{Examples of biological networks}

\begin{itemize}
\item Networks are convenient ways to represent the internal networking of biological systems, at different scales
\item Ecology : prey-predators networks
\item Neurology : networks of neurones
\item Genetics   \footnote{possibly non directly observed}
\item Epidemics : epidemiological networks characterizing the spread of disease in a population
\end{itemize}


}

%%%%%%%%%%%%%%%%%%%%%%%%%%%%%%%%%%%%%%%%%%%%%%%%%%%%%

\frame{\frametitle{Examples of communication networks}
\begin{itemize}
\item
Networks of citations between academic journals of papers, 
\item Networks of co-autorship, 
\item Networks indicating semantic relationships
 \item Links between web pages (political blogs for instance
 \item Contributions to Wikipedia pages
\end{itemize}

}



%%%%%%%%%%%%%%%%%%%%%%%%%%%%%%%%%%%%%%%%%%%%%%%%%%%%%
\frame{\frametitle{From network data to graph}
 In general, the word ``network'' is used inter-changeably with the word ``graph'', which is formally the mathematical representation of a network. 

%For one type of relation which is the object of interest
\begin{columns}[t]
  \begin{column}{5cm}
  \begin{block}{Network data}
 \begin{itemize}
 \item A type of relation
 \item  $n$ actors and   the list of the existing relations between  any pair of actors
\end{itemize}
 \end{block} 
 \begin{block}{Visual representation}
\includegraphics[width=\linewidth]{plots/network_ex1.png}

  \end{block}  
  \end{column}
  
  \begin{column}{5cm}
  \begin{block}{Graph}
    \begin{itemize}
\item A set of $n$ vertices or nodes $V=\{1,\dots,n\}$
\item A collection of pairs $E = \{ (1,2), (1,3),\dots...\}$ 
\end{itemize}
\begin{center}or \end{center}  
\begin{itemize}
\item An adjacency matrix $(Y_{ij})_{(i,j)\in  \{1,\dots,n\}}$ 

\item Such  that $Y_{ij}=0$ for any $(i,j) \notin E$
\end{itemize}
  \end{block}   
  \end{column}
 \end{columns}  
\footnotesize{http://www.thenetworkthinkers.com}
}


%%%%%%%%%%%%%%%%%%%%%%%%%%%%%%%%%%%%%%%%%%%%%%%%%%%%%

\frame{\frametitle{Notations}

Let $\{1,\dots,n\}$ be a set of actors  and $E$ a set of pairs.  For any $(i,j) \in \{1,\dots \}$,  $Y_{ij}$ represents the relation between $i$ and $j$. 

\begin{block}{Oriented - Non oriented}
\begin{itemize}
\item If for any edge  $(i,j)=(j,i)$ then \emphase{non-oriented} graph


\begin{center}
$\left(Y_{ij}\right)_{i,j \in \{1,\dots,n\}}$ is a symetric matrix 
\end{center}
\item If not, \emphase{oriented graph}, the edges are  called \emphase{arcs}
\end{itemize}

\end{block}

\begin{block}{$Y_{ij} \in $ ???}
\begin{itemize}
\item If $(i,j) \notin E$ $$Y_{ij} = 0 $$
\item If $(i,j) \in E$ : 
\begin{itemize}
\item \emphase{In general}\;:  \hspace{4em} $Y_{ij} = 1$  \hspace{4em}  \emphase{Binary graph}
\item \emphase{Sometimes}:  \hspace{4em}  $Y_{ij} \in \mathbb{R}$  \hspace{4em}  \emphase{Weighted graph}
\end{itemize}
\end{itemize}
\end{block}
}

%%%%%%%%%%%%%%%%%%%%%%%%%%%%%%%%%%%%%%%%%%%%%%%%%%%%%
\frame{\frametitle{Graph theory in a few slides (1)}

We want to describe the graph as an object. 

\begin{block}{Adjacency and degree}
\begin{itemize}
\item Degree of a node $i$ : number of edges containing $i$  (outdegree and indegree for oriented graphes) 
\item Distribution of degrees on the whole graph
\end{itemize}
\end{block}



\begin{block}{Study of the subgraphes / motifs}
\begin{itemize}
\item Presence of clique (subset of vertices who are all connected)
\item Strongly connected component : a subgraph where all nodes in the subgraph are reachable by all other nodes in the subgraph. 
\end{itemize}
\end{block}



%\footnotesize{source  : wikipedia}

}

%\section{Algorithmic point of view}
%\frame{\frametitle{Algorithmic point of view}
%
%Connectivity, edge-betweenness, clustering coefficient, diameter, degree distribution
%
%-> descriptive statistics
%}
%
%
%%%%%%%%%%%%%%%%%%%%%%%%%%%%%%%%%%%%%%%%%%%%%%%%%%%%%%

\frame{\frametitle{Graph theory in a few slides (2)}

\begin{block}{Paths}
\begin{itemize}
\item Study of the path from one node to an other
\item \emphase{Distance}  : shortest path between $i$ and $j$
\item \emphase{Eccentricity} of a vertex = maximum of the distances with any other vertex

\item \emphase{Diameter of a graph} (maximum distance from v to any other vertex)  : maximum eccentricity over all vertices in a graph

\end{itemize} 


\end{block}



}


%%%%%%%%%%%%%%%%%%%%%%%%%%%%%%%%%%%%%%%%%%%%%%%%%%%%%

\frame{\frametitle{Graph theory in a few slides (3)}



\begin{block}{Connectivity}
\begin{itemize}
\item Existence of path between any two vertices? If yes $\rightarrow$ connected graph 
\item Existence of an important edge such that removing it lead to a disconnected graph? 
\item \emphase{Betweenness centrality}: an indicator of a node's centrality in a network. It is equal to the number of shortest paths from all vertices to all others that pass through that node
\end{itemize}
\end{block}

\begin{block}{Spectral theory}
Study of the relations between the properties of the graph and the properties of the adjacency matrix
\end{block}


}

%%%%%%%%%%%%%%%%%%%%%%%%%%%%%%%%%%%%%%%%%%%%%%%%%%%%%

\subsection{Random graphs}
\frame{\frametitle{Random graphs}

\begin{itemize}
\item Let $\{1,\dots,n\}$ be the set of nodes and $Y$ an adjacency matrix
\item $Y$ is a realization of a  random variable whose distribution as to be specified
$$ P(Y=y; \theta)$$ 
\end{itemize}
}

%%%%%%%%%%%%%%%%%%%%%%%%%%%%%%%%%%%%%%%%%%%%%%%%%%%%%
\frame{\frametitle{A first example : Exponential Random Graphical Models (ERGM)} 
$$ P(Y=y; \theta )  = \frac{e^{\theta^T S(y)}}{c(\theta)}$$
where $S(y)$ is a vector a statistics computed on the graph
\begin{itemize}
\item \emphase{Difficulty 1} : How to chose $S$? 
\begin{itemize}
\item Frequency of interaction important? \emphase{$\Rightarrow$}  $S$ should include the number of ties
\item Reciprocity  important (if directed)?  \emphase{$\Rightarrow$} $S$ should include the count of the number of mutual ties
\item ... 
\end{itemize}
\item \emphase{Difficulty 2} : $C(\theta)$  is impossible to compute explicitly
\item \emphase{Difficulty 3} : It is not a generating model  \emphase{$\Rightarrow$} difficult to interpret
\end{itemize}

\begin{block}{An other point of view} \centering Starting from independent edges and complexifying the model
\end{block}
}





