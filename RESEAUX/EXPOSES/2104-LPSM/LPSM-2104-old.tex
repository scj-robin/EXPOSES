\documentclass[8pt]{beamer}

% Beamer style
%\usetheme[secheader]{Madrid}
% \usetheme{CambridgeUS}
\useoutertheme{infolines}
\usecolortheme[rgb={0.65,0.15,0.25}]{structure}
% \usefonttheme[onlymath]{serif}
\beamertemplatenavigationsymbolsempty
%\AtBeginSubsection

% Packages
%\usepackage[french]{babel}
\usepackage[latin1]{inputenc}
\usepackage{color}
% \usepackage[dvipsnames]{xcolor}
\usepackage{xspace}
\usepackage{dsfont, stmaryrd}
\usepackage{amsmath, amsfonts, amssymb, stmaryrd, mathabx}
\usepackage{epsfig}
\usepackage{tikz}
\usepackage{url}
% \usepackage{ulem}
\usepackage{/home/robin/LATEX/Biblio/astats}
%\usepackage[all]{xy}
\usepackage{graphicx}

% Maths
% \newtheorem{theorem}{Theorem}
% \newtheorem{definition}{Definition}
\newtheorem{proposition}{Proposition}
% \newtheorem{assumption}{Assumption}
% \newtheorem{algorithm}{Algorithm}
% \newtheorem{lemma}{Lemma}
% \newtheorem{remark}{Remark}
% \newtheorem{exercise}{Exercise}
% \newcommand{\propname}{Prop.}
% \newcommand{\proof}{\noindent{\sl Proof:}\quad}
% \newcommand{\eproof}{$\blacksquare$}

% \setcounter{secnumdepth}{3}
% \setcounter{tocdepth}{3}
\newcommand{\pref}[1]{\ref{#1} p.\pageref{#1}}
\newcommand{\qref}[1]{\eqref{#1} p.\pageref{#1}}

% Colors : http://latexcolor.com/
\definecolor{darkred}{rgb}{0.65,0.15,0.25}
\definecolor{darkgreen}{rgb}{0,0.4,0}
\definecolor{darkred}{rgb}{0.65,0.15,0.25}
\definecolor{amethyst}{rgb}{0.6, 0.4, 0.8}
\definecolor{asparagus}{rgb}{0.53, 0.66, 0.42}
\definecolor{applegreen}{rgb}{0.55, 0.71, 0.0}
\definecolor{awesome}{rgb}{1.0, 0.13, 0.32}
\definecolor{blue-green}{rgb}{0.0, 0.87, 0.87}
\definecolor{red-ggplot}{rgb}{0.52, 0.25, 0.23}
\definecolor{green-ggplot}{rgb}{0.42, 0.58, 0.00}
\definecolor{purple-ggplot}{rgb}{0.34, 0.21, 0.44}
\definecolor{blue-ggplot}{rgb}{0.00, 0.49, 0.51}

% Commands
\newcommand{\backupbegin}{
   \newcounter{finalframe}
   \setcounter{finalframe}{\value{framenumber}}
}
\newcommand{\backupend}{
   \setcounter{framenumber}{\value{finalframe}}
}
\newcommand{\emphase}[1]{\textcolor{darkred}{#1}}
\newcommand{\comment}[1]{\textcolor{gray}{#1}}
\newcommand{\paragraph}[1]{\textcolor{darkred}{#1}}
\newcommand{\refer}[1]{{\small{\textcolor{gray}{{\cite{#1}}}}}}
\newcommand{\Refer}[1]{{\small{\textcolor{gray}{{[#1]}}}}}
\newcommand{\goto}[1]{{\small{\textcolor{blue}{[\#\ref{#1}]}}}}
\renewcommand{\newblock}{}

\newcommand{\tabequation}[1]{{\medskip \centerline{#1} \medskip}}
% \renewcommand{\binom}[2]{{\left(\begin{array}{c} #1 \\ #2 \end{array}\right)}}

% Variables 
\newcommand{\Abf}{{\bf A}}
\newcommand{\Beta}{\text{B}}
\newcommand{\Bcal}{\mathcal{B}}
\newcommand{\Bias}{\xspace\mathbb B}
\newcommand{\Cor}{{\mathbb C}\text{or}}
\newcommand{\Cov}{{\mathbb C}\text{ov}}
\newcommand{\cl}{\text{\it c}\ell}
\newcommand{\Ccal}{\mathcal{C}}
\newcommand{\cst}{\text{cst}}
\newcommand{\Dcal}{\mathcal{D}}
\newcommand{\Ecal}{\mathcal{E}}
\newcommand{\Esp}{\xspace\mathbb E}
\newcommand{\Espt}{\widetilde{\Esp}}
\newcommand{\Covt}{\widetilde{\Cov}}
\newcommand{\Ibb}{\mathbb I}
\newcommand{\Fcal}{\mathcal{F}}
\newcommand{\Gcal}{\mathcal{G}}
\newcommand{\Gam}{\mathcal{G}\text{am}}
\newcommand{\Hcal}{\mathcal{H}}
\newcommand{\Jcal}{\mathcal{J}}
\newcommand{\Lcal}{\mathcal{L}}
\newcommand{\Mt}{\widetilde{M}}
\newcommand{\mt}{\widetilde{m}}
\newcommand{\Nbb}{\mathbb{N}}
\newcommand{\Mcal}{\mathcal{M}}
\newcommand{\Ncal}{\mathcal{N}}
\newcommand{\Ocal}{\mathcal{O}}
\newcommand{\pt}{\widetilde{p}}
\newcommand{\Pt}{\widetilde{P}}
\newcommand{\Pbb}{\mathbb{P}}
\newcommand{\Pcal}{\mathcal{P}}
\newcommand{\Qcal}{\mathcal{Q}}
\newcommand{\qt}{\widetilde{q}}
\newcommand{\Rbb}{\mathbb{R}}
\newcommand{\Sbb}{\mathbb{S}}
\newcommand{\Scal}{\mathcal{S}}
\newcommand{\st}{\widetilde{s}}
\newcommand{\St}{\widetilde{S}}
\newcommand{\Tcal}{\mathcal{T}}
\newcommand{\todo}{\textcolor{red}{TO DO}}
\newcommand{\Ucal}{\mathcal{U}}
\newcommand{\Un}{\math{1}}
\newcommand{\Vcal}{\mathcal{V}}
\newcommand{\Var}{\mathbb V}
\newcommand{\Vart}{\widetilde{\Var}}
\newcommand{\Zcal}{\mathcal{Z}}

% Symboles & notations
\newcommand\independent{\protect\mathpalette{\protect\independenT}{\perp}}\def\independenT#1#2{\mathrel{\rlap{$#1#2$}\mkern2mu{#1#2}}} 
\renewcommand{\d}{\text{\xspace d}}
\newcommand{\gv}{\mid}
\newcommand{\ggv}{\, \| \, }
% \newcommand{\diag}{\text{diag}}
\newcommand{\card}[1]{\text{card}\left(#1\right)}
\newcommand{\trace}[1]{\text{tr}\left(#1\right)}
\newcommand{\matr}[1]{\boldsymbol{#1}}
\newcommand{\matrbf}[1]{\mathbf{#1}}
\newcommand{\vect}[1]{\matr{#1}} %% un peu inutile
\newcommand{\vectbf}[1]{\matrbf{#1}} %% un peu inutile
\newcommand{\trans}{\intercal}
\newcommand{\transpose}[1]{\matr{#1}^\trans}
\newcommand{\crossprod}[2]{\transpose{#1} \matr{#2}}
\newcommand{\tcrossprod}[2]{\matr{#1} \transpose{#2}}
\newcommand{\matprod}[2]{\matr{#1} \matr{#2}}
\DeclareMathOperator*{\argmin}{arg\,min}
\DeclareMathOperator*{\argmax}{arg\,max}
\DeclareMathOperator{\sign}{sign}
\DeclareMathOperator{\tr}{tr}
\newcommand{\ra}{\emphase{$\rightarrow$} \xspace}

% Hadamard, Kronecker and vec operators
\DeclareMathOperator{\Diag}{Diag} % matrix diagonal
\DeclareMathOperator{\diag}{diag} % vector diagonal
\DeclareMathOperator{\mtov}{vec} % matrix to vector
\newcommand{\kro}{\otimes} % Kronecker product
\newcommand{\had}{\odot}   % Hadamard product

% TikZ
\newcommand{\nodesize}{2em}
\newcommand{\edgeunit}{2.5*\nodesize}
\newcommand{\edgewidth}{1pt}
\tikzstyle{node}=[draw, circle, fill=black, minimum width=.75\nodesize, inner sep=0]
\tikzstyle{square}=[rectangle, draw]
\tikzstyle{param}=[draw, rectangle, fill=gray!50, minimum width=\nodesize, minimum height=\nodesize, inner sep=0]
\tikzstyle{hidden}=[draw, circle, fill=gray!50, minimum width=\nodesize, inner sep=0]
\tikzstyle{hiddenred}=[draw, circle, color=red, fill=gray!50, minimum width=\nodesize, inner sep=0]
\tikzstyle{observed}=[draw, circle, minimum width=\nodesize, inner sep=0]
\tikzstyle{observedred}=[draw, circle, minimum width=\nodesize, color=red, inner sep=0]
\tikzstyle{eliminated}=[draw, circle, minimum width=\nodesize, color=gray!50, inner sep=0]
\tikzstyle{empty}=[draw, circle, minimum width=\nodesize, color=white, inner sep=0]
\tikzstyle{blank}=[color=white]
\tikzstyle{nocircle}=[minimum width=\nodesize, inner sep=0]

\tikzstyle{edge}=[-, line width=\edgewidth]
\tikzstyle{edgebendleft}=[-, >=latex, line width=\edgewidth, bend left]
\tikzstyle{edgebendright}=[-, >=latex, line width=\edgewidth, bend right]
\tikzstyle{lightedge}=[-, line width=\edgewidth, color=gray!50]
\tikzstyle{lightedgebendleft}=[-, >=latex, line width=\edgewidth, bend left, color=gray!50]
\tikzstyle{lightedgebendright}=[-, >=latex, line width=\edgewidth, bend right, color=gray!50]
\tikzstyle{edgered}=[-, line width=\edgewidth, color=red]
\tikzstyle{edgebendleftred}=[-, >=latex, line width=\edgewidth, bend left, color=red]
\tikzstyle{edgebendrightred}=[-, >=latex, line width=\edgewidth, bend right, color=red]

\tikzstyle{arrow}=[->, >=latex, line width=\edgewidth]
\tikzstyle{arrowbendleft}=[->, >=latex, line width=\edgewidth, bend left]
\tikzstyle{arrowbendright}=[->, >=latex, line width=\edgewidth, bend right]
\tikzstyle{arrowred}=[->, >=latex, line width=\edgewidth, color=red]
\tikzstyle{arrowbendleftred}=[->, >=latex, line width=\edgewidth, bend left, color=red]
\tikzstyle{arrowbendrightred}=[->, >=latex, line width=\edgewidth, bend right, color=red]
\tikzstyle{arrowblue}=[->, >=latex, line width=\edgewidth, color=blue]
\tikzstyle{dashedarrow}=[->, >=latex, dashed, line width=\edgewidth]
\tikzstyle{dashededge}=[-, >=latex, dashed, line width=\edgewidth]
\tikzstyle{dashededgebendleft}=[-, >=latex, dashed, line width=\edgewidth, bend left]
\tikzstyle{lightarrow}=[->, >=latex, line width=\edgewidth, color=gray!50]

\newcommand{\bEDD}{B-EDD\xspace}

% Directory
\newcommand{\fignet}{/home/robin/RECHERCHE/RESEAUX/EXPOSES/FIGURES}

%====================================================================
%====================================================================

%====================================================================
%====================================================================
\begin{document}
%====================================================================
%====================================================================

%====================================================================
\title[Bipartite motifs]{Motif-based analysis of bipartite networks \\}

\author[S. Robin]{S. Robin \\ \medskip
joint work with S. Ouadah, P. Latouche \refer{OLR21}\\ ~}

\institute[]{INRA / AgroParisTech / univ. Paris-Saclay / Museum National d'Histoire Naturelle \\~ \\~}

\date[Paris, Apr.'21]{LPSM, Paris, Apr. 2021}

%====================================================================
%====================================================================
\maketitle
%====================================================================
\frame{\frametitle{Outline} \tableofcontents}

%====================================================================
%====================================================================
\section{Bipartite networks and motifs}
\frame{\frametitle{Outline} \tableofcontents[currentsection]}
%====================================================================
\frame{\frametitle{Bipartite networks and motifs} 

  \hspace{-.04\textwidth}
  \begin{tabular}{cc}
    \begin{tabular}{p{.53\textwidth}}
      \paragraph{Bipartite graph.} Graph connecting to distinct types of nodes: 
      $$
      G = (E^{\circ}, E^{\square}, V)
      $$ 
      \begin{itemize}
      \item $E^{\circ} = \{1, \dots m\} =$ set of 'top' nodes (vertices)
      \item $E^{\square} = \{1, \dots n\} =$ set of 'bottom' nodes
      \item $V \subset E^{\circ} \times E^{\square} =$ set of edges
      \end{itemize}
      Equivalent to a $m \times n$ adjacency matrix $G$.
      
      \bigskip \bigskip \pause
      \paragraph{Bipartite motif.} Sub-graph ($p_s \ll m$, $q_s \ll n$) of a bipartite graph
      $$
      s = (\{1, \dots p_s\}, \{1, \dots q_s\}, A_s)
      $$
      \bigskip
      $$
      \text{Example on the right:} \qquad A^s = \left( \begin{array}{cc} 0 & 1 \\ 0 & 1 \\ 1 & 1\end{array} \right)
      $$
    \end{tabular}
    & 
    \hspace{-.05\textwidth} \pause
    \begin{tabular}{c} 
      Bipartite graph $G$ \\
      \includegraphics[width=.4\textwidth, trim=0 50 0 50]{\fignet/Zackenberg-1996_12-red-net} \\
      Bipartite motif $s$ \\
      \includegraphics[width=.2\textwidth]{\fignet/FigMotifsBEDD-motif9-automorphism1}
    \end{tabular}
  \end{tabular}

}

%====================================================================
\frame{\frametitle{Motivation} 

  \paragraph{Ecological networks.}
  \begin{itemize}
  \item Plant-pollinator: $\circ =$ insects, $\square =$ plants (mutualistic)
  \item Host-parasite: $\circ =$ viruses, $\square =$ plants (antagonistic)
  \end{itemize}
      
  \bigskip 
\bigskip 
\pause
  \paragraph{Motif counts} provide a generic 'meso-scale' description of a network \refer{SCB19}
  \begin{itemize}
  \item motifs = 'building-blocks'
  \item between local (node degree) and global ('metrics')
  \end{itemize}

  \bigskip 
\bigskip 
\pause
  \paragraph{Application.}
  \begin{itemize}
  \item Characterize and compare networks (even when the nodes=species are different)
  \item \textcolor{gray}{Characterize species according their 'species-role' in various networks}
  \end{itemize} 
  + Efficient tool to count motif occurrences in a network (\url{bmotif} package \refer{SSS19})

}

%==================================================================
\frame{\frametitle{Motif count}

  \bigskip 
  \paragraph{Automorphisms =} non-redundant permutations of the motifs' adjacency matrix:
  $$
  \includegraphics[width=.1\textwidth]{\fignet/FigMotifsBEDD-motif9-automorphism1}
  \includegraphics[width=.1\textwidth]{\fignet/FigMotifsBEDD-motif9-automorphism2}
  \includegraphics[width=.1\textwidth]{\fignet/FigMotifsBEDD-motif9-automorphism3} 
  \includegraphics[width=.1\textwidth]{\fignet/FigMotifsBEDD-motif9-automorphism4}
  \includegraphics[width=.1\textwidth]{\fignet/FigMotifsBEDD-motif9-automorphism5}
  \includegraphics[width=.1\textwidth]{\fignet/FigMotifsBEDD-motif9-automorphism6} 
  $$
  
  \bigskip 
  \paragraph{Set of positions $\Pcal_s =$} choice of $p_s$ nodes (among $m$) $\times$ choice of $q_s$ nodes (among $n$) $\times$ choice of an automorphism (among $r_s$)
  $$
  c_s := |\Pcal_s| = 
  \left(\begin{array}{c}m \\ p_s\end{array}\right)
  \times \left(\begin{array}{c}n \\ q_s\end{array}\right)
  \times r_s
  $$ 

  \bigskip 
  \paragraph{Motif count.} For a position $\alpha \in \Pcal_s$, define the indicator variable 
  $$
  Y_s(\alpha) = 1 \text{ if match,} \qquad 0 \text{ otherwise}.
  $$
  Then the motif count is
  $$
  N_s = \sum_{\alpha \in \Pcal_s} Y_s(\alpha)
  $$
  and its frequency is $F_s := N_s / c_s$
  
}

%====================================================================
\section{Moments of motif count}
\frame{\frametitle{Outline} \tableofcontents[currentsection]}
%====================================================================
\frame{\frametitle{Bipartite expected degree distribution model} 

  \paragraph{\bEDD model.} A row-column exhangeable (RCE) model: \\ ~
  \begin{itemize}
   \item $\{U_i\}_{1 \leq i \leq m} \text{ iid}: U_i \sim \Ucal_{[0, 1]}$, 
   \qquad $\{V_j\}_{1 \leq j \leq n} \text{ iid}: V_j \sim \Ucal_{[0, 1]}$, \\ ~
   \item $\{G_{ij}\}_{1 \leq i, \leq m, 1 \leq j \leq n} \text{ independent } \mid \{U_i\}, \{V_j\}:$
   $$
   \Pr\{G_{ij} = 1 \mid U_i, V_j\} = \rho \; g(U_i) \; h(V_j).
   $$
  \end{itemize}
  
  \bigskip \bigskip 
  where:
  \begin{itemize}
  \item $\rho =$ network density, \\ ~
  \item $g =$ degree imbalance among top nodes ($g \geq 0$, $\int g(u) \d u = 1$), \\ ~
  \item $h =$ degree imbalance among bottom nodes ($h \geq 0$, $\int h(v) \d v = 1$).
  \end{itemize}

}

%====================================================================
\frame{\frametitle{Star motifs} 

  \paragraph{Top star motif} with degree $d$ = one top node connected to $d$ bottom nodes 

  \bigskip \bigskip 
  \paragraph{Counting (top) star motifs.} Only depends on the (top) degrees $\{D_i\}_{1 \leq i \leq m}$
  $$
  N(\text{top star with degre $d$}) = \sum_{i=1}^m \binom{D_i}{d}
  $$
  
  \bigskip \bigskip 
  \paragraph{Exchangeability.} Under any RCE model, $\Pr\{Y_s(\alpha)\}$ does not depend on $\alpha$.

  \bigskip \bigskip 
  \paragraph{Probability of a star motif.} Under the \bEDD model, the probability of a top-star motif with degree $d$ writes
  $$
  \gamma_d = \rho^d \int g(u)^d \d u.
  $$
  Obviously, $\gamma_1 = \lambda_1 = \rho$
    
}

%====================================================================
\frame{\frametitle{Motif probability} 

  \bigskip 
  \paragraph{\bEDD model.} Under the \bEDD model, the motif probability writes
  $$
  \phi_s := \Pr\{Y_s(\alpha)\} = \prod_{u = 1}^{p_s} \gamma_{d_u^s} \prod_{v = 1}^{q_s} \lambda_{e_v^s} \left/ \rho^{d_+^s} \right.
  $$
  where
  \begin{itemize}
   \item $d_u^s =$ degree of the $u$-th top node in motif $s$ (resp. $e_v^s$ for the $v$-th bottom node) \\ ~
   \item $d_+^s = \sum_u d_u^s = \sum_v e_v^s =$ number of edges in motif $s$
  \end{itemize}

  \bigskip \bigskip \pause
  That is
  $$
  \Pbb\left(
  \includegraphics[width=.06\textwidth, trim=100 200 100 0]{\fignet/FigMotifsBEDD-motif9} 
  \right)
  \; = \; 
  \frac{
  \overset{\text{top stars}}{\overbrace{
  \Pbb\left(\includegraphics[width=.05\textwidth, trim=100 200 100 0]{\fignet/FigMotifsBEDD-motif9-top1}\right) % \times 
  \Pbb\left(\includegraphics[width=.05\textwidth, trim=100 200 100 0]{\fignet/FigMotifsBEDD-motif9-top1}\right) % \times 
  \Pbb\left(\includegraphics[width=.05\textwidth, trim=100 200 100 0]{\fignet/FigMotifsBEDD-motif9-top2}\right)   }}
  \times
  \overset{\text{bottom stars}}{\overbrace{
  \Pbb\left(\includegraphics[width=.05\textwidth, trim=100 200 100 0]{\fignet/FigMotifsBEDD-motif9-bottom1}\right) % \times
  \Pbb\left(\includegraphics[width=.05\textwidth, trim=100 200 100 0]{\fignet/FigMotifsBEDD-motif9-bottom3}\right)
  }}
  }{
  \underset{\text{edges}}{\underbrace{
  \left(\Pbb\left(\includegraphics[width=.05\textwidth, trim=100 200 100 0]{\fignet/FigMotifsBEDD-motif9-top1}\right)\right)^4
  }}
  } 
  $$

}

%====================================================================
\frame{\frametitle{Variance of the count} 

}

%====================================================================
\frame{\frametitle{Plug-in estimates} 

  \paragraph{Estimated probability.} 
  $$
  {\phi}_s := \frac{\gamma_2 \lambda_3}{\rho}
  \qquad \rightarrow \qquad 
  \overline{F}_s := \frac{\Gamma_2 \Lambda_4}{F_1}
  $$
  where $F_1$ is the edge frequency and $\Gamma_d$ (resp. $\lambda_e$) the frequency of top (resp. bottom) stars
  }

}

%====================================================================
\section{Asymptotic normality}
\frame{\frametitle{Outline} \tableofcontents[currentsection]}
%====================================================================
\frame{\frametitle{Main result} 

}

%====================================================================
\frame{\frametitle{Sketch of proof} 

}

%====================================================================
\frame{\frametitle{Limitation of the plug-in step} 

}

%====================================================================
\section{Goodness-of-fit and network comparison}
\frame{\frametitle{Outline} \tableofcontents[currentsection]}
%====================================================================
\frame{\frametitle{} 

}

%====================================================================
\section{Network embedding}
\frame{\frametitle{Outline} \tableofcontents[currentsection]}
%====================================================================
\frame{\frametitle{} 

}

%====================================================================
\frame[allowframebreaks]{ \frametitle{References}
  {%\footnotesize
   \tiny
   \bibliography{/home/robin/Biblio/BibGene}
%    \bibliographystyle{/home/robin/LATEX/Biblio/astats}
   \bibliographystyle{alpha}
  }
}

%====================================================================
\backupbegin
%====================================================================
%====================================================================

%====================================================================
\backupend
%====================================================================

%====================================================================
%====================================================================
\end{document}
%====================================================================
%====================================================================
  
  \hspace{-.025\textwidth}
  \begin{tabular}{cc}
    \begin{tabular}{p{.5\textwidth}}
    \end{tabular}
    & 
    \hspace{-.02\textwidth}
    \begin{tabular}{p{.5\textwidth}}
    \end{tabular}
  \end{tabular}

