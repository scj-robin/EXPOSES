\documentclass[8pt]{beamer}

% Beamer style
%\usetheme[secheader]{Madrid}
% \usetheme{CambridgeUS}
\useoutertheme{infolines}
\usecolortheme[rgb={0.65,0.15,0.25}]{structure}
% \usefonttheme[onlymath]{serif}
\beamertemplatenavigationsymbolsempty
%\AtBeginSubsection

% Packages
%\usepackage[french]{babel}
\usepackage[latin1]{inputenc}
\usepackage{color}
% \usepackage[dvipsnames]{xcolor}
\usepackage{xspace}
\usepackage{dsfont, stmaryrd}
\usepackage{amsmath, amsfonts, amssymb, stmaryrd, mathabx}
\usepackage{epsfig}
\usepackage{tikz}
\usepackage{url}
% \usepackage{ulem}
\usepackage{/home/robin/LATEX/Biblio/astats}
%\usepackage[all]{xy}
\usepackage{graphicx}

% Maths
% \newtheorem{theorem}{Theorem}
% \newtheorem{definition}{Definition}
\newtheorem{proposition}{Proposition}
% \newtheorem{assumption}{Assumption}
% \newtheorem{algorithm}{Algorithm}
% \newtheorem{lemma}{Lemma}
% \newtheorem{remark}{Remark}
% \newtheorem{exercise}{Exercise}
% \newcommand{\propname}{Prop.}
% \newcommand{\proof}{\noindent{\sl Proof:}\quad}
% \newcommand{\eproof}{$\blacksquare$}

% \setcounter{secnumdepth}{3}
% \setcounter{tocdepth}{3}
\newcommand{\pref}[1]{\ref{#1} p.\pageref{#1}}
\newcommand{\qref}[1]{\eqref{#1} p.\pageref{#1}}

% Colors : http://latexcolor.com/
\definecolor{darkred}{rgb}{0.65,0.15,0.25}
\definecolor{darkgreen}{rgb}{0,0.4,0}
\definecolor{darkred}{rgb}{0.65,0.15,0.25}
\definecolor{amethyst}{rgb}{0.6, 0.4, 0.8}
\definecolor{asparagus}{rgb}{0.53, 0.66, 0.42}
\definecolor{applegreen}{rgb}{0.55, 0.71, 0.0}
\definecolor{awesome}{rgb}{1.0, 0.13, 0.32}
\definecolor{blue-green}{rgb}{0.0, 0.87, 0.87}
\definecolor{red-ggplot}{rgb}{0.52, 0.25, 0.23}
\definecolor{green-ggplot}{rgb}{0.42, 0.58, 0.00}
\definecolor{purple-ggplot}{rgb}{0.34, 0.21, 0.44}
\definecolor{blue-ggplot}{rgb}{0.00, 0.49, 0.51}

% Commands
\newcommand{\backupbegin}{
   \newcounter{finalframe}
   \setcounter{finalframe}{\value{framenumber}}
}
\newcommand{\backupend}{
   \setcounter{framenumber}{\value{finalframe}}
}
\newcommand{\emphase}[1]{\textcolor{darkred}{#1}}
\newcommand{\comment}[1]{\textcolor{gray}{#1}}
\newcommand{\paragraph}[1]{\textcolor{darkred}{#1}}
\newcommand{\refer}[1]{{\small{\textcolor{gray}{{[\cite{#1}]}}}}}
\newcommand{\Refer}[1]{{\small{\textcolor{gray}{{[#1]}}}}}
\newcommand{\goto}[1]{{\small{\textcolor{blue}{[\#\ref{#1}]}}}}
\renewcommand{\newblock}{}

\newcommand{\tabequation}[1]{{\medskip \centerline{#1} \medskip}}
% \renewcommand{\binom}[2]{{\left(\begin{array}{c} #1 \\ #2 \end{array}\right)}}

% Variables 
\newcommand{\Abf}{{\bf A}}
\newcommand{\Beta}{\text{B}}
\newcommand{\Bcal}{\mathcal{B}}
\newcommand{\Bias}{\xspace\mathbb B}
\newcommand{\Cor}{{\mathbb C}\text{or}}
\newcommand{\Cov}{{\mathbb C}\text{ov}}
\newcommand{\cl}{\text{\it c}\ell}
\newcommand{\Ccal}{\mathcal{C}}
\newcommand{\cst}{\text{cst}}
\newcommand{\Dcal}{\mathcal{D}}
\newcommand{\Ecal}{\mathcal{E}}
\newcommand{\Esp}{\xspace\mathbb E}
\newcommand{\Espt}{\widetilde{\Esp}}
\newcommand{\Covt}{\widetilde{\Cov}}
\newcommand{\Ibb}{\mathbb I}
\newcommand{\Fcal}{\mathcal{F}}
\newcommand{\Gcal}{\mathcal{G}}
\newcommand{\Gam}{\mathcal{G}\text{am}}
\newcommand{\Hcal}{\mathcal{H}}
\newcommand{\Jcal}{\mathcal{J}}
\newcommand{\Lcal}{\mathcal{L}}
\newcommand{\Mt}{\widetilde{M}}
\newcommand{\mt}{\widetilde{m}}
\newcommand{\Nbb}{\mathbb{N}}
\newcommand{\Mcal}{\mathcal{M}}
\newcommand{\Ncal}{\mathcal{N}}
\newcommand{\Ocal}{\mathcal{O}}
\newcommand{\pt}{\widetilde{p}}
\newcommand{\Pt}{\widetilde{P}}
\newcommand{\Pbb}{\mathbb{P}}
\newcommand{\Pcal}{\mathcal{P}}
\newcommand{\Qcal}{\mathcal{Q}}
\newcommand{\qt}{\widetilde{q}}
\newcommand{\Rbb}{\mathbb{R}}
\newcommand{\Sbb}{\mathbb{S}}
\newcommand{\Scal}{\mathcal{S}}
\newcommand{\st}{\widetilde{s}}
\newcommand{\St}{\widetilde{S}}
\newcommand{\Tcal}{\mathcal{T}}
\newcommand{\todo}{\textcolor{red}{TO DO}}
\newcommand{\Ucal}{\mathcal{U}}
\newcommand{\Un}{\math{1}}
\newcommand{\Vcal}{\mathcal{V}}
\newcommand{\Var}{\mathbb V}
\newcommand{\Vart}{\widetilde{\Var}}
\newcommand{\Zcal}{\mathcal{Z}}

% Symboles & notations
\newcommand\independent{\protect\mathpalette{\protect\independenT}{\perp}}\def\independenT#1#2{\mathrel{\rlap{$#1#2$}\mkern2mu{#1#2}}} 
\renewcommand{\d}{\text{\xspace d}}
\newcommand{\gv}{\mid}
\newcommand{\ggv}{\, \| \, }
% \newcommand{\diag}{\text{diag}}
\newcommand{\card}[1]{\text{card}\left(#1\right)}
\newcommand{\trace}[1]{\text{tr}\left(#1\right)}
\newcommand{\matr}[1]{\boldsymbol{#1}}
\newcommand{\matrbf}[1]{\mathbf{#1}}
\newcommand{\vect}[1]{\matr{#1}} %% un peu inutile
\newcommand{\vectbf}[1]{\matrbf{#1}} %% un peu inutile
\newcommand{\trans}{\intercal}
\newcommand{\transpose}[1]{\matr{#1}^\trans}
\newcommand{\crossprod}[2]{\transpose{#1} \matr{#2}}
\newcommand{\tcrossprod}[2]{\matr{#1} \transpose{#2}}
\newcommand{\matprod}[2]{\matr{#1} \matr{#2}}
\DeclareMathOperator*{\argmin}{arg\,min}
\DeclareMathOperator*{\argmax}{arg\,max}
\DeclareMathOperator{\sign}{sign}
\DeclareMathOperator{\tr}{tr}
\newcommand{\ra}{\emphase{$\rightarrow$} \xspace}

% Hadamard, Kronecker and vec operators
\DeclareMathOperator{\Diag}{Diag} % matrix diagonal
\DeclareMathOperator{\diag}{diag} % vector diagonal
\DeclareMathOperator{\mtov}{vec} % matrix to vector
\newcommand{\kro}{\otimes} % Kronecker product
\newcommand{\had}{\odot}   % Hadamard product

% TikZ
\newcommand{\nodesize}{2em}
\newcommand{\edgeunit}{2.5*\nodesize}
\newcommand{\edgewidth}{1pt}
\tikzstyle{node}=[draw, circle, fill=black, minimum width=.75\nodesize, inner sep=0]
\tikzstyle{square}=[rectangle, draw]
\tikzstyle{param}=[draw, rectangle, fill=gray!50, minimum width=\nodesize, minimum height=\nodesize, inner sep=0]
\tikzstyle{hidden}=[draw, circle, fill=gray!50, minimum width=\nodesize, inner sep=0]
\tikzstyle{hiddenred}=[draw, circle, color=red, fill=gray!50, minimum width=\nodesize, inner sep=0]
\tikzstyle{observed}=[draw, circle, minimum width=\nodesize, inner sep=0]
\tikzstyle{observedred}=[draw, circle, minimum width=\nodesize, color=red, inner sep=0]
\tikzstyle{eliminated}=[draw, circle, minimum width=\nodesize, color=gray!50, inner sep=0]
\tikzstyle{empty}=[draw, circle, minimum width=\nodesize, color=white, inner sep=0]
\tikzstyle{blank}=[color=white]
\tikzstyle{nocircle}=[minimum width=\nodesize, inner sep=0]

\tikzstyle{edge}=[-, line width=\edgewidth]
\tikzstyle{edgebendleft}=[-, >=latex, line width=\edgewidth, bend left]
\tikzstyle{edgebendright}=[-, >=latex, line width=\edgewidth, bend right]
\tikzstyle{lightedge}=[-, line width=\edgewidth, color=gray!50]
\tikzstyle{lightedgebendleft}=[-, >=latex, line width=\edgewidth, bend left, color=gray!50]
\tikzstyle{lightedgebendright}=[-, >=latex, line width=\edgewidth, bend right, color=gray!50]
\tikzstyle{edgered}=[-, line width=\edgewidth, color=red]
\tikzstyle{edgebendleftred}=[-, >=latex, line width=\edgewidth, bend left, color=red]
\tikzstyle{edgebendrightred}=[-, >=latex, line width=\edgewidth, bend right, color=red]

\tikzstyle{arrow}=[->, >=latex, line width=\edgewidth]
\tikzstyle{arrowbendleft}=[->, >=latex, line width=\edgewidth, bend left]
\tikzstyle{arrowbendright}=[->, >=latex, line width=\edgewidth, bend right]
\tikzstyle{arrowred}=[->, >=latex, line width=\edgewidth, color=red]
\tikzstyle{arrowbendleftred}=[->, >=latex, line width=\edgewidth, bend left, color=red]
\tikzstyle{arrowbendrightred}=[->, >=latex, line width=\edgewidth, bend right, color=red]
\tikzstyle{arrowblue}=[->, >=latex, line width=\edgewidth, color=blue]
\tikzstyle{dashedarrow}=[->, >=latex, dashed, line width=\edgewidth]
\tikzstyle{dashededge}=[-, >=latex, dashed, line width=\edgewidth]
\tikzstyle{dashededgebendleft}=[-, >=latex, dashed, line width=\edgewidth, bend left]
\tikzstyle{lightarrow}=[->, >=latex, line width=\edgewidth, color=gray!50]

\newcommand{\dN}{\Delta N}
\newcommand{\dtau}{\Delta \tau}

% Directory
\newcommand{\fignet}{/home/robin/RECHERCHE/RESEAUX/EXPOSES/FIGURES}
\newcommand{\figlux}{/home/robin/RECHERCHE/ECOLOGIE/EXPOSES/2012-Luxembourg/Slides/Figures}
\newcommand{\figeco}{/home/robin/RECHERCHE/ECOLOGIE/EXPOSES/FIGURES}

%====================================================================
%====================================================================

%====================================================================
%====================================================================
\begin{document}
%====================================================================
%====================================================================

%====================================================================
\title{Network inference and Graphical models}

\author[S. Robin]{S. Robin}

\institute[Sorbonne universit�]{\normalsize{Sorbonne universit�}}

\date{ANR Trident, Dec. 2022}

%====================================================================
%====================================================================
\maketitle

%====================================================================
%====================================================================
\section{Network inference}
%====================================================================
\frame{\frametitle{Species interaction network} 

  \begin{tabular}{cc}
    \hspace{-.04\textwidth}
    \begin{tabular}{p{.6\textwidth}}
      \begin{itemize}
      \item \bigskip Aim: Understand how species from a same community interact
      \item \bigskip Network representation = draw an edge between interacting pairs of species
      \item \pause \bigskip Main issue: Distinguish 
      \emphase{direct interactions} (predator-prey) from
      simple \emphase{associations} (two preys of a same predator) 
      \end{itemize}
    \end{tabular}
    &
    \hspace{-.05\textwidth}
    \begin{tabular}{ccc}
      \begin{tikzpicture}
  \node[nocircle] (pred) at ( 0.0*\edgeunit, 1.0*\edgeunit) {predator};
  \node[nocircle] (prey) at ( 0.0*\edgeunit, 0.0*\edgeunit) {prey};

  \draw[edge] (pred) to (prey);
\end{tikzpicture}
 &
      & 
      \begin{tikzpicture}
  \node[nocircle] (pred) at ( 0.6*\edgeunit, 1.0*\edgeunit) {predator};
  \node[nocircle] (prey1) at ( 0.0*\edgeunit, 0.0*\edgeunit) {prey1};
  \node[nocircle] (prey2) at ( 1.2*\edgeunit, 0.0*\edgeunit) {prey2};

  \draw[edge] (pred) to (prey1);
  \draw[edge] (pred) to (prey2);
  \draw[edgered] (prey1) to (prey2);
\end{tikzpicture}

    \end{tabular}
  \end{tabular}
  
  \pause 
  \ra Co-occurences or correlations cannot distinguish between the two \refer{PWT19}

  \pause \bigskip \bigskip 
  \paragraph{Probabilistic translation.}
  \begin{align*}
    \text{'network'} & = \text{graph} \\
    \text{'association'} & = \text{marginal dependance} \\
    \text{'direct interaction'} & = \text{conditional dependance} 
  \end{align*}
}

%====================================================================
\frame{\frametitle{Network inference}

  \bigskip 
  \paragraph{Typical dataset at hand.} $n$ sites, $p$ species, 
  \begin{itemize}
    \item $x_i =$ environmental description of site $i$, \textcolor{gray}{$t_j =$ traits of species $j$}
    \item $Y_{ij} =$ abundance (or presence) of species $j$ in site $i$
  \end{itemize}
  
  \bigskip \bigskip \pause
  \paragraph{Assumption.} 
  \begin{itemize}
    \item Species 'direct' interactions are encoded in a network (= a graph) $G$.
    \item $G$ is the same in all sites.
  \end{itemize}

  \bigskip \bigskip \pause
  \paragraph{Aim.} Retrieve the graph $G$ from the species abundances $Y$ : 
  $$  
  \begin{tabular}{c|c|c}
    {Abundances} $Y$ & 
    {Covariates} $X$ & 
    {Inferred network} $G$ \\ 
    \begin{tabular}{p{.25\textwidth}}
      {\small \begin{tabular}{rrr}
        {\sl Hi.pl} & {\sl An.lu} & {\sl Me.ae} \\ \hline
        31  &   0  & 108 \\
         4  &   0  & 110 \\
        27  &   0  & 788 \\
        13  &   0  & 295 \\
        23  &   0  &  13 \\
        20  &   0  &  97 \\
        $\vdots$ & $\vdots$ & $\vdots$ 
      \end{tabular}}
    \end{tabular}
    & 
    \begin{tabular}{p{.25\textwidth}}
      {\small \begin{tabular}{rrr}
        Lat. & Long. & Depth \\ \hline
        71.10 & 22.43 & 349 \\
        71.32 & 23.68 & 382 \\
        71.60 & 24.90 & 294 \\
        71.27 & 25.88 & 304 \\
        71.52 & 28.12 & 384 \\
        71.48 & 29.10 & 344 \\
        $\vdots$ & $\vdots$ & $\vdots$ 
      \end{tabular} }
    \end{tabular}
    & 
    \hspace{-.02\textwidth} 
    \begin{tabular}{p{.3\textwidth}}
      \hspace{-.1\textwidth} 
      \begin{tabular}{c}
        \includegraphics[width=.35\textwidth, trim=0 95 0 60, clip=]{\fignet/network_BarentsFish_Gfull_full60edges}
      \end{tabular}
    \end{tabular}
  \end{tabular}
  $$

}
  
%====================================================================
%====================================================================
\section{Graphical models}
%====================================================================
\frame{\frametitle{Probabilistic framework: Graphical models}

  \paragraph{Definition \refer{Lau96}.} The joint distribution $p$ is faithful to the graph $G$ 
  $$
  p(Z_1, \dots, Z_p) \propto \prod_{C \in \Ccal_G} \psi_C(Z_C)
  $$
  where $\Ccal_G =$ set of cliques of $G$. % \refer{HaC71}
  
  \bigskip \pause
  \paragraph{Example.} $p(Z_1, Z_2, Z_3, Z_4) \; \propto \; \psi_1(Z_1 ,Z_2, Z_3) \times \psi_2(Z_3, Z_4)$ \\
%   \vspace{-.2\textheight}
  \begin{tabular}{cc}
    \begin{tabular}{p{.4\textwidth}}
%     \hspace{-.1\textwidth}
    \includegraphics[width=.35\textwidth]{\fignet/FigGGM-4nodes-red}
    \end{tabular}
    & 
    \hspace{-.1\textwidth}
    \begin{tabular}{p{.6\textwidth}}
      \begin{itemize}
       \item Connected graph: all variables are dependent \\ ~
       \item $Z_3$ 'separates' $Z_4$ from $Z_1$ and $Z_2$: 
       $$
       Z_4 \perp (Z_1, Z_2) | Z_3
       $$
      \end{itemize}
    \end{tabular}
  \end{tabular} 

  \hspace{-.7\textheight} ~ \pause
  \paragraph{Interpretation.} ~
  \begin{itemize}
  \item The vertices of $G$ only reveal {\sl conditional dependencies}.
  \item Directed graphical models also exist (but raise identifiability/interpretability issues)
  \end{itemize}
}

%====================================================================
\frame{\frametitle{Gaussian graphical models (GGM)}

  \paragraph{Gaussian distribution.}
  $$
  Z \sim \Ncal_p(\mu, \Sigma)
  $$
  $\mu =$ vector of means, $\Sigma =$ covariance matrix, $\Omega = \Sigma^{-1} =$ precision matrix.
  
  \bigskip \bigskip \pause
  \paragraph{A nice property.} ~ \\
%   \vspace{-.2\textheight}
  \begin{tabular}{cc}
    \begin{tabular}{p{.5\textwidth}}
%     \hspace{-.15\textwidth}
    \includegraphics[width=.35\textwidth]{\fignet/FigGGM-4nodes-red}
    \end{tabular}
    & 
    \hspace{-.15\textwidth}
    \begin{overprint}
    \onslide<2>
    \begin{tabular}{p{.5\textwidth}}
	 Covariance matrix
	 $$
	 \Sigma \propto \left[ \begin{array}{rrrr}
	   1 & -.25 & -.41 &  \emphase{.25} \\
	   -.25 &  1 & -.41 &  \emphase{.25} \\
	   -.41 & -.41 &  1 & -.61 \\
	   \emphase{.25} &  \emphase{.25} & -.61 &  1
	   \end{array} \right] 
	 $$
    \end{tabular} 
    \onslide<3>
    \begin{tabular}{p{.5\textwidth}}
	 Inverse covariance matrix
	 $$
	 \Omega \; \propto \; \left[ \begin{array}{rrrr}
	   1 & .5 & .5 & \emphase{0} \\
	   .5 & 1 & .5 & \emphase{0} \\
	   .5 & .5 & 1 & .5 \\
	   \emphase{0} & \emphase{0} & .5 & 1
	   \end{array} \right] 
	 $$
    \end{tabular} 
    \onslide<4>
    \begin{tabular}{p{.5\textwidth}}
	 Adjacency matrix
	 $$
	 A = \left[ \begin{array}{rrrr}
	 0 & 1 & 1 & \emphase{0} \\
	 1 & 0 & 1 & \emphase{0} \\
	 1 & 1 & 0 & 1 \\
	 \emphase{0} & \emphase{0} & 1 & 0 
	 \end{array} \right]
	 $$
    \end{tabular} 
    \onslide<5->
    \begin{tabular}{p{.5\textwidth}}
	 Estimated inverse covariance matrix
	 $$
	 \widehat{\Omega} \; \propto \; \left[ \begin{array}{rrrr}
	   1 & .48 & .61 & \emphase{.09} \\
	   .48 & 1 & .67 & \emphase{.06} \\
	   .61 & .67 & 1 & .46 \\
	   \emphase{.09} & \emphase{.06} & .46 & 1
	   \end{array} \right] 
	 $$
	 ($n = 100$)
    \end{tabular} 
    \end{overprint}
  \end{tabular}
  
}

%====================================================================
\frame{\frametitle{A popular approach}

  \paragraph{Sparsity assumption.} Assume that
  $$
  \Omega = \Sigma^{-1} \text{ is sparse}
  $$
  = most entries of $\Omega$ are zeros.
  
  \bigskip \bigskip \pause
  \paragraph{Sparsity inducing penality.} 
  \begin{itemize}
    \item Maximum-likelihood estimation: 
    $$
    \widehat{\Omega} \text{ maximizes } \log p(Z; \Omega)
    $$
    \item Penalized maximum-likelihood estimation: 
    $$
    \widehat{\Omega} \text{ maximizes } \log p(Z; \Omega) - \emphase{\lambda \|\Omega\|_{0, 1}}
    $$
    where
    $$
    \|\Omega\|_{0, 1} = \sum_{j \neq k} |w_{jk}|
    $$
    forces a fraction of entries of $\Omega$ to be zero ('graphical lasso': \refer{FHT08}).
  \end{itemize}

}

%====================================================================
%====================================================================
\section{Joint species distribution models}
%====================================================================
\frame{\frametitle{Latent variable models}

  \paragraph{'Abundance' data.}
  \begin{itemize}
    \setlength\itemsep{1em}    
    \item The Gaussian distribution does not make sens for abundance (species counts) or presence (0/1) data
    \item Many (most ?) joint species distribution models (JSDM) resort to a latent Gaussian structure \refer{WBO15,OTN17,PWT19}
  \end{itemize}
  
  \bigskip \bigskip \pause
  \begin{tabular}{cc}
    \hspace{-.04\textwidth}
    \begin{tabular}{p{.65\textwidth}}
      \paragraph{Latent variable model.} 'Latent' = unobserved \\~
      \begin{itemize}
        \setlength\itemsep{1em}    
        \item $Z_i =$ latent Gaussian vector in site $i$
        \item $x_i =$ descriptors of site $i$
        \item $Y_i =$ abundance vector in site $i$, distributed conditionally on $Z_i$:
        $$
        Y_{ij} \sim p(\cdot \mid Z_{ij}; x_i)
        $$
      \end{itemize}
    \end{tabular}
    & 
    \hspace{-.05\textwidth}
    \begin{tabular}{p{.4\textwidth}}
      \includegraphics[width=.25\textwidth, trim=0 0 150 0, clip=]{\fignet/CMR18b-ArXiv-Fig1b}
    \end{tabular}
  \end{tabular}

}

%====================================================================
\frame{\frametitle{Poisson log-normal model}

  \paragraph{PLN = Poisson log-normal model \refer{AiH89}.} A (multivariate mixed) generalized linear model :
  $$
  Z_i \sim \Ncal(0, \Sigma), \qquad \qquad
  Y_{ij} \sim \Pcal(\exp(x_i^\top \beta_j + Z_{ij})).
  $$
  \begin{itemize}
    \setlength\itemsep{1em}    
    \item $\Sigma =$ covariance structure (of the latent layer): \emphase{biotic interactions}
    \item $\beta_j =$ regression coefficients: \emphase{abiotic effects} of the environment on species $j$
  \end{itemize}

  \bigskip \bigskip \pause
  \paragraph{Inference.}
  \begin{itemize}
    \setlength\itemsep{1em}    
    \item Not straightforward (because of the latent layer), often classical EM \refer{DLR77} not enough
    \item Two main strategies: MCMC ({\tt HSMC} package) or EM + variational approximation
    \item PLN model: {\tt PLNmodels} package \refer{CMR21} uses a variational approximation
  \end{itemize}
  

}

%====================================================================
\frame{\frametitle{PLN: Barents'fish data}

  \bigskip
  \paragraph{Data \refer{FNA06}:} $n =89$ stations, $p = 30$ fish species, $d = 4$ descriptors (lat., long., depth, temp.)

  \bigskip
  $$
  \begin{tabular}{cc|c}
    \multicolumn{2}{l|}{\emphase{Full model}} &
    \multicolumn{1}{l}{\emphase{Null model}} \\
    & & \\
    \multicolumn{2}{c|}{{$Y_{ij} \sim \Pcal(\exp(\emphase{x_i^\intercal \beta_j} + Z_{ij}))$}} &
    \multicolumn{1}{c}{{$Y_{ij} \sim \Pcal(\exp(\emphase{\mu_j} + Z_{ij}))$}} \\
    & & \\
    \multicolumn{2}{l|}{{$x_i =$ all covariates}} &
    \multicolumn{1}{l}{{no covariate}} \\ 
    & & \\
    & correlations between & \\
    inferred  correlations $\widehat{\Sigma}_{\text{full}}$ & 
    predictions: $x_i^\intercal \widehat{\beta}_j$ & 
    inferred correlations $\widehat{\Sigma}_{\text{null}}$ \\ 
    \includegraphics[width=.3\textwidth, trim=20 20 20 20]{\figeco/BarentsFish-corrAll} 
    &
    \includegraphics[width=.3\textwidth, trim=20 20 20 20]{\figeco/BarentsFish-corrPred} &
    \includegraphics[width=.3\textwidth, trim=20 20 20 20]{\figeco/BarentsFish-corrNull}
  \end{tabular}
  $$
}

%====================================================================
\frame{\frametitle{PLN: Network inference }
 
  $$
  \includegraphics[height=.8\textheight]{\fignet/CMR18b-ArXiv-Fig5}
  $$
  \refer{CMR18b}

}

%====================================================================
%====================================================================
\section{Conclusion}
%====================================================================
\frame{\frametitle{To summarize}

  \begin{itemize}
    \setlength\itemsep{2em}    
    \item Network inference aims at distinguishing between direct and undirect 'interactions' between species.
    \item Most network inference methods use the graphical model framework
    \item Most JSDM's involve a Gaussian latent layer \\
    (to take advantage of the properties of Gaussian graphical models)
    \item {\tt PLNmodels} package: JSDM inference, network inference \\
    + dimension reduction (PCA), site clustering, \dots \\
    \textcolor{gray}{+ accounting for species traits ... soon}
  \end{itemize}

}

%====================================================================
\frame{\frametitle{Some comments}

  \begin{itemize}
    \setlength\itemsep{2em}    
    \item Network inference is not an easy task: there are
    $$
    2^{p(p-1)/2}
    \qquad 
    (p = 10 \to 3 \; 10^{13} \text{ graphs})
    $$
    possible graphs connecting $p$ nodes (10 nodes $\to 10^{13}$ graphs, 30 nodes $\to 10^{131}$ graphs).
    \item \pause Understanding the functioning of a community may not require to deciphere the connexion/non-connexion of each pair of species (think of analyses based on species traits)
    \item \pause JSDMs provide a broader framework.
    \item \pause Advertising: {\sl Statistical Approaches for Hidden Variables in Ecology} \refer{PeG22b}
  \end{itemize}

}

%====================================================================
%====================================================================
\backupbegin
%====================================================================
\section*{References}
%====================================================================
\frame[allowframebreaks]{ \frametitle{References}
  {
   \footnotesize
   \bibliography{/home/robin/Biblio/BibGene}
   \bibliographystyle{alpha}
  }
}

%====================================================================
\frame{\frametitle{PLN for network inference: choosing $\lambda$}

  $$
  \includegraphics[height=.8\textheight]{\fignet/BarentsFish_Gfull_criteria}
  $$
}

%====================================================================
\backupend

%====================================================================
%====================================================================
\end{document}
%====================================================================
%====================================================================
  
  \begin{tabular}{cc}
    \hspace{-.04\textwidth}
    \begin{tabular}{p{.5\textwidth}}
    \end{tabular}
    & 
    \hspace{-.02\textwidth}
    \begin{tabular}{p{.5\textwidth}}
    \end{tabular}
  \end{tabular}

