\documentclass[11pt]{article}
 %\pdfminorversion=4

%\usepackage{amsmath}
%\usepackage{graphicx,psfrag,epsf}
\usepackage{enumerate}
\usepackage{natbib}
\usepackage{url} % not crucial - just used below for the URL 
\usepackage{amsfonts,amsmath,amssymb,epsfig,epsf,psfrag}
\usepackage{graphicx}
\usepackage{xcolor}
\usepackage{xspace}

\usepackage[latin1]{inputenc}


% \textwidth 16cm
% \textheight 24cm 
% \topmargin -1 cm 
% \oddsidemargin 0cm 
% \evensidemargin 0cm

\title{\bf Goodness of fit of logistic models for random graphs: a variational Bayes approach}
\author{Pierre Latouche$^1$, St�phane Robin$^2$, Sarah Ouadah$^2$\\
  \small{($^1$) Laboratoire    SAMM,   EA   4543,  Universit�   Paris   1, Panth�on-Sorbonne,  France}, \\
  \small{($^2$) UMR MIA-Paris, AgroParisTech, INRA, Universit� Paris-Saclay, France}
}
\date{}

\begin{document}

\maketitle

\paragraph{Keywords:}  Random graphs; logistic regression; $W$-graph
model; variational approximations

\bigskip
The logistic regression model constitutes a natural and simple tool to understand how covariates (when available) contribute to explain the topology of a binary network. Parallel to this model, the W-graph model constitutes a fairly general model to for random graph, without any side information. Still, beside the issues raised by its inference, the graphon model suffers severe identifiability issues.
 
In this work, we propose to combine the logistic model for graph with a W-graph residual term. Such an extra term aims at characterizing the residual structure of the network, that is not explained by the covariates. The goodness-of-fit of the logistic regression then amounts to check if the graphon function associated with the residual W-graph is constant.

We approximate this new generic logistic model using a class of models with blockwise constant residual structure (see \cite{LaR15}). This framework allows to derive a Bayesian procedure from a model based selection context using goodness-of-fit criteria. All these criteria depend on marginal likelihood terms for which we provide estimates relying on two series of variational approximations (see \cite{LRO15}). 

The accuracy of proposed inference procedure is assessed through a simulation study and its use will be illustrated on several networks from social and life sciences.

\nocite{}
\bibliographystyle{chicago}
\bibliography{/home/robin/Biblio/BibGene}

%\bibliographystyle{plain}
%\bibliography{biblio}



\end{document}

