\documentclass[11pt,a4paper]{article}

\usepackage{latexsym,amsthm,amsfonts,amsgen,amssymb,amscd,amsmath}
\usepackage{graphicx}         % Packages to allow inclusion of graphics


\usepackage[english]{babel}
\usepackage[LGR,T1]{fontenc}
\usepackage{palatino}
\usepackage[mathbf]{euler}
%\usepackage{euler}
\usepackage{enumerate,mathrsfs}
\usepackage{fullpage}
\usepackage{xcolor}
\usepackage{tikz}

%  For creating hyperlinks in cross references
%\usepackage[pdfpagelabels,plainpages=false]{hyperref}  

% Nice bibliography
\usepackage{natbib}


\parindent 0cm
\parskip 0.4cm


\newtheorem{thm}{Theorem}
\newtheorem{prop}[thm]{Proposition}
\newtheorem{corollaire}[thm]{Corollary}
\newtheorem{lem}[thm]{Lemma}
\newtheorem{rem}[thm]{Remark}
\newtheorem{defn}[thm]{Definition}
\theoremstyle{remark}

\theoremstyle{definition}
\newtheorem{exemple}{Exemple}
%\numberwithin{equation}{section}


\newcommand{\dd}{\mathrm d}
\newcommand{\ee}{\mathrm e}
\newcommand{\esp}{\mathbb E}
\newcommand{\prob}{\mathbb P}
\newcommand{\e}{\mathrm e}
\newcommand{\ii}{\mathrm i}
\newcommand{\eps}{\varepsilon}
\newcommand{\Vect}{\mathrm{Vect}\,}
\newcommand{\Var}{\mathrm{Var}\,}
\newcommand{\ds}{\displaystyle}
\newcommand{\trait}{\begin{center}\rule{3cm}{.3pt}\end{center}}
\renewcommand{\mathcal}{\mathscr}
\newcommand{\eqlaw}{\stackrel{\text{law}}{=}}

%\newcommand{\eps}{\varepsilon}

\newfont{\manfnt}{manfnt}
\newcommand{\danger}{{\manfnt\symbol{'177}}}
\newcommand{\mdanger}{\marginpar[\hfill\danger]{\danger\hfill}}
\newcommand{\new}{{\manfnt\symbol{30}}}
\newcommand{\mnew}{\marginpar[\hfill\new]{\hfill\new}}
\DeclareTextFontCommand{\hershey}{\fontfamily{hscs}\selectfont}

\begin{document}
%
%Please leave text above unchanged
%Replace below with your information
\begin{center}
\textbf{\Large Computing Bayesian posterior with empirical likelihood in population genetics}\\[1em]
Jean-Marie Cornuet$^1$, Rapha{\"e}l Leblois$^1$, \textbf{Pierre Pudlo}$^{1 \& 2}$ and
Christian P. Robert$^3$ \\[1em]
\end{center}
$^1$ CBGP, INRA\\
$^2$ I3M, Universit\'e Montpellier 2\\
$^3$ CEREMADE, Universit\'e Paris Dauphine and CREST, INSEE, Paris.\\

In population genetics, the computation of the likelihood is often a hard problem. Approximate Bayesian
computation (ABC) is certainly the most used algorithm to bypass this computation in a Bayesian paradigm (see
\cite{beaumont10}, for a recent survey). But ABC is quite time consuming and needs massive paralellization to be
efficient. In this talk we will present a promising alternative using the empirical likelihood \cite{owen88}.


That last method profiles the likelihood in a nonparametric way using an estimating equation on the unknown
parameters. Our proposal relies on the score functions given by the pairwise composite likelihood 
\cite{lindsay88} which can be explicitly computed in a large variety of evolutionary scenarii when considering
microsatellite loci with the stepwise mutation model \cite{ohta:kimura:73}. Numerical simulations will
exhibit that the posterior estimated with our proposal is comparable to the ABC posterior, but that the
computation is about thirty times faster.



\begin{thebibliography}{1}
\bibitem{beaumont10}
M.~Beaumont.
\newblock Approximate Bayesian computation in evolution and ecology.
\newblock {\em Annu. Rev. Ecol. Evol. Syst.}, 41:379--406, 2010.

\bibitem{lindsay88}
B. G. Lindsay.
\newblock Composite Likelihood Methods.
\newblock {\em Contemporary Mathematics}, 80:221--239, 1988.

\bibitem{ohta:kimura:73}
T.~Ohta and M.~Kimura.
\newblock A model of mutation appropriate to estimate the number of electrophoretically detectable alleles in
a finite population.
\newblock {\em Genet. Res.}, 22:201--204, 1973.

\bibitem{owen88}
A.~B. Owen.
\newblock Empirical Likelihood ratio confidence intervals for a single functional.
\newblock {\em Biometrika}, 75:237--249, 1988

\end{thebibliography}
% \bibliographystyle{plain}
% \bibliography{ibs.bib}

\end{document}
