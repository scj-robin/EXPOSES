%\documentclass[notes=onlyslideswithnotes]{beamer}
%\documentclass[handout]{beamer}

\documentclass{beamer}
\usetheme{Madrid}
%\usecolortheme[rgb={0.65,0.15,0.25}]{structure}
\usefonttheme[onlymath]{serif}
\usepackage[latin1]{inputenc}
\usepackage{color}
\usepackage{amsmath}
\usepackage{amsfonts}
\usepackage[all]{xy}
\usepackage{graphicx}
\beamertemplatenavigationsymbolsempty

\newenvironment{disarray}%
 {\everymath{\displaystyle\everymath{}}\array}%
 {\endarray}

\renewcommand{\ttdefault}{lmtt}
\newcommand\M{\mathcal{M}}
\newcommand\Lv{\mathcal{L}}
\newcommand\E{\mathbb{E}}
\newcommand\la{\ln{\bf \alpha}}
\newcommand{\Xcal}{\mathcal{X}}
\newcommand{\Zbf}{{\bf Z}}
\newcommand{\Vbf}{{\bf V}}
\newcommand{\Xbf}{{\bf X}}
\newcommand{\Zcal}{\mathcal{Z}}
\newcommand{\wbf}{{\bf w}}
\newcommand{\alphabf}{\text{\mathversion{bold}{$\alpha$}}}
\newcommand{\pibf}{\mbox{\mathversion{bold}{$\pi$}}}
\newcommand{\Pibf}{\mbox{\mathversion{bold}{$\Pi$}}}
\newcommand{\thetabf}{\mbox{\mathversion{bold}{$\theta$}}}
\newcommand{\Thetabf}{\mbox{\mathversion{bold}{$\Theta$}}}
\newcommand{\taubf}{\mbox{\mathversion{bold}{$\tau$}}}
\newcommand{\pibar}{\bar{\pi}}
\newcommand{\Qcal}{\mathcal{Q}}
\newcommand{\RX}{\mathcal{R}_{\Xbf}}
\newcommand{\Lcal}{\mathcal{L}}
\renewcommand{\ttdefault}{lmtt}
\newcommand{\Abf}{{\bf A}}
\newcommand{\Bcal}{\mathcal{B}}
\newcommand{\C}{C}
\newcommand{\Esp}{\mathbb{E}}
\newcommand{\eps}{\varepsilon}
\newcommand{\epsbar}{\overline{\eps}}
\newcommand{\etabar}{\overline{\eta}}
\newcommand{\fbf}{{\bf f}}
\newcommand{\Gcal}{\mathcal{G}}
\newcommand{\Hbf}{{\bf H}}
\newcommand{\Ibb}{\mathbb{I}}
\newcommand{\Kcal}{\mathcal{K}}
\newcommand{\lambdabar}{\overline{\lambda}}
\newcommand{\Mcal}{\mathcal{M}}
\newcommand{\Ncal}{\mathcal{N}}
\newcommand{\Pcal}{\mathcal{P}}
\newcommand{\rhobar}{\overline{\rho}}
\newcommand{\Sbf}{{\bf S}}
\newcommand{\Vcal}{\mathcal{V}}
\newcommand{\Vsf}{\mathsf{V}}


\newcommand{\biz}{\begin{itemize}}
\newcommand{\eiz}{\end{itemize}}
\newcommand{\bcent}{\begin{center}}
\newcommand{\ecent}{\end{center}}
\newcommand{\barr}{\begin{array}}
\newcommand{\earr}{\end{array}}
\newcommand{\btab}{\begin{tabular}}
\newcommand{\etab}{\end{tabular}}
\newcommand{\ben}{\begin{enumerate}}
\newcommand{\een}{\end{enumerate}}
\newcommand{\bdes}{\begin{description}}
\newcommand{\edes}{\end{description}}
\newcommand{\dps}{\displaystyle}

\newcommand{\bqas}{\begin{eqnarray*}}
\newcommand{\eqas}{\end{eqnarray*}}
\newcommand{\noi}{\noindent}

\newtheorem{Prop}{Proposition}
\newtheorem{Propri}{Propri\'et\'es}

\definecolor{darkblue}{rgb}{0,0,0.55}
\definecolor{darkgreen}{rgb}{0,0.55,0}
\definecolor{darkred}{rgb}{0.75,0,0}
\definecolor{rougeF}{rgb}{0.65,0.15,0.25}
\newcommand{\noir}[1]{\textrm{\textcolor{black}{#1}}}


\newcommand{\coul}[1]{\textcolor{rougeF}{#1}}

\title{Mod�le statistique pour les graphes al�atoires h�t�rog�nes, application aux r�seaux biologiques}
\author[J.J. Daudin]
{Jean-Jacques Daudin}

\institute{UMR518 AgroParisTech-INRA}
\date{mai 2009}

\begin{document}

\begin{frame}
\titlepage
joint work with
\begin{tabular}{p{4cm}p{4cm}}
\tiny Corinne Vacher UMR1202 INRA/Univ. Bordeaux I BioGeCo &  \tiny Laurent Pierre Universit� Paris 10 \\
%\includegraphics[scale=0.3, angle=0] {vacher2.jpg} &
%\includegraphics[scale=0.037, angle=0] {laurent.jpg}
\\

\end{tabular}
%\begin{tabular}{lr}
% \includegraphics[height=1.5cm ] {SSB.png} & \includegraphics[height=1.5cm ] {logagroptech.png} \\
%\end{tabular}
\end{frame}

\begin{frame}{Sommaire}
\tableofcontents[hideallsubsections]
\end{frame}

\section{Enjeux et Questions}
\begin{frame}
\frametitle{Enjeux}
\begin{tabular}{p{4.5cm} p{6cm}}
\vspace{-4cm}
La compr�hension des r�seaux biologiques
\begin{itemize}
  \item r�gulation de g�nes, interactions prot�iques, relations m�taboliques
  \item r�seaux �cologiques
\end{itemize}
 est un  enjeu majeur en biologie mol�culaire et en �cologie.
  &   \includegraphics[height=4cm ] {Colinet_base.png}\\
\hline
 \vspace{0.5cm}
L'�tude des r�seaux al�atoires est un sujet tr�s actif en math�matiques.
& \vspace{0.5cm} \hspace{0.5cm}
\includegraphics[scale=0.32 ] {ISIWofK2.pdf} \\
\end{tabular}
\end{frame}


\section{Mixnet}
\begin{frame}
\frametitle{Mixnet: mod�le de m�lange pour les graphes al�atoires (Daudin et al. Stat.\& Comput. 2007)}

\begin{itemize}
  \item $i=1,n$ sommets (g�nes, prot�ines, esp�ces)  \\[0.4cm]
  \item $q=1,Q$ classes de sommets \\[0.4cm]
  \item $X_{ij}=1$ s'il y a un arc du noeud  $i$ vers le noeud $j$  \\[0.4cm]
  \item $Z=Z_{iq}$ variable latente discr�te , $Z_{iq}=1$ si le noeud $i$ appartient � la classe $q$  \\[0.4cm]
  \item $(Z_{i1},Z_{i2}...Z_{iQ}) \sim \M (1,\alpha_1,\alpha_2,...\alpha_Q)$  \\[0.4cm]
  \item  $P(X_{ij}=1/Z_{iq}=1,Z_{jl}=1)=\pi_{ql}$
\end{itemize}

\end{frame}


%\begin{frame}
%\frametitle{Mixnet: un mod�le flexible}
%  \begin{table}[h]
%    \begin{center}
%      \begin{tabular}{lccc}
%        \hline
%        Description & Graphe & $Q$ & $\pibf$ \\
%        \hline
%        \begin{tabular}{p{2cm}} Erdos \end{tabular}
%        & \begin{tabular}{c}
%          \includegraphics[height=1cm,width=2.3cm]{FigNetworks-Erdos.pdf}
%        \end{tabular}
%        & 1
%        &  $p$ \\
%        \hline
%        \begin{tabular}{p{2cm}} Hubs \end{tabular}
%        & \begin{tabular}{c}
%          \includegraphics[height=1.5cm,width=2.3cm]{hubs.png}
%        \end{tabular}
%        & 4
%        & $\left( { \small \begin{array}{cccc} 0&1&0&0\\ 1&0&1&0\\0&1&0&1\\0&0&1&0\\
%          \end{array} }\right)$
%         \\
%        \hline \begin{tabular}{p{2cm}}  groupes modulaires \end{tabular}
%        & \begin{tabular}{c}
%          \includegraphics[height=1cm,width=2.3cm]{modules.png}
%        \end{tabular}
%        & 2
%        & $\left(\begin{array}{cc} 1&\varepsilon\\ \varepsilon&1\\
%          \end{array} \right)$ \\
%        \hline
%        \begin{tabular}{p{2cm}} Organisation hi�rarchique \end{tabular}
%        & \begin{tabular}{c}
%          \includegraphics[height=1.5cm,width=2.3cm]{hierarchique.png}
%        \end{tabular}
%        & 5
%        & $\left( {\tiny \begin{array}{ccccc} 0&1&1&0&0\\ 0&0&0&1&0\\0&0&0&0&1\\0&0&0&0&0\\
%        0&0&0&0&0\\
%          \end{array}} \right)$ \\
%    \end{tabular}
%    \end{center}
%  \end{table}
%\end{frame}

\begin{frame}
\frametitle{Mixnet: un mod�le flexible...mais avec 2 d�fauts}
 \begin{itemize}
 \item estimation des param�tres difficile (complexit� $Q^n$)
 \item chaque sommet est oblig� d'appartenir � un et un seul groupe
 \end{itemize}

\end{frame}

\section{ Model EVMRG}

\begin{frame}
\frametitle{Extremal Vertices Model for Random Graphs (EVMRG)}

\begin{itemize}
 \item Each vertex $i$ : weighted mean of $Q$ extreme hypothetical vertices (EHV)
 \item   weights
$Z_i = (z_{i1},\dots,z_{iQ})$,  $z_{iq} \ge 0$,  $\sum_qz_{iq}=1$.

\item $  P_{ij}=\sum_{q,l=1,Q}z_{iq}a_{ql}z_{jl}$
\item $a_{ql}$: connectivity between the EHVs $q$ and $l$.
\item $X_{ij} \sim B(P_{ij})$
\item $X_{ij}$ independent
\end{itemize}
\alert{Each vertex is a mixture of EHVs and inherits its connectivity properties from those of the EHVs} \\
\alert{$n(Q-1)+Q^2$ parameters...and $n^2$ observations }

\end{frame}



\begin{frame}
\frametitle{Model identifiability}
\begin{itemize}
  \item $P$ $(n,n)$ matrix containing the $p_{ij}$,
  \item $Z$  $(n,Q)$ matrix containing the $z_{iq}$ ,  $Z \in S_Q^n$,
  \item $A \in [0,1]^{Q^2}$, the $(Q,Q)$ matrix containing the $a_{ql}$, the connectivity matrix between the EHVs.
\end{itemize}
We choose $Z$ which maximizes $Tr(ZZ')$ among the equivalent versions of model $P=ZAZ'$.
\begin{itemize}
  \item this constraint implies unicity of $(Z,A)$ provided that $n \gg Q$ and the $n$ vertices are different.
  \item the EV should not be too far from real vertices in order to confer upon them some reality.
\end{itemize}
\end{frame}

\section{Parameter Estimation}
%%%%%%%%%%%%%%%%%%%%%%%%%%%%%%%%%%%%%%%%%%%%%%%%%%%%%%%%%%%%%%%%%%%%%
\begin{frame}
\frametitle{log-likelihood}
\begin{equation*}\label{Likelihood}
    L=\sum_{i,j}x_{ij}\log(\sum_{q,l=1,Q}z_{iq}a_{ql}z_{jl})+(1-x_{ij})\log(1-\sum_{q,l=1,Q}z_{iq}a_{ql}z_{jl})
\end{equation*}

\begin{equation*}
    \frac{\partial L}{\partial Z}=RZ A'+ R'Z A
\end{equation*}
with $R$ a $(n,n)$ matrix with $r_{ij}=\frac{x_{ij}-p_{ij}}{p_{ij}(1-p_{ij})},$

\begin{equation*}
     \frac{\partial L}{\partial A}=Z'RZ
\end{equation*}

\end{frame}


\begin{frame}
\frametitle{algorithm}
   \begin{itemize}
   \item Find initializing values $(A^{(0)},Z^{(0)})$
   \item At step $(k)$ use a linear programming algorithm to maximize the linearized log-likelihood
  under the constraints:
\begin{eqnarray*}
% \nonumber to remove numbering (before each equation)
  A & \in & [O,1]^{n^2} \\
  Z & \in & S_Q^n
\end{eqnarray*}
  \item Maximize $ L(A,Z)$  along the line $(A^{(k)},Z^{(k)}) \rightarrow (A^{LP_k},Z^{LP_k)})$ using a mixed convex interpolation and Golden section search and obtain  $(A^{(k+1)},Z^{(k+1)})$
       \item a MATLAB package is available (Aymeric Thibaud M1)
   \end{itemize}
\end{frame}

\begin{frame}
\frametitle{Choice of the Number of Groups}

for direct networks:
  $$AIC(Q)=-2L(\hat{A}_Q,\hat{Z}_Q)+2(Q^2+n(Q-1))$$
   $$ BIC(Q)=-2L(\hat{A}_Q,\hat{Z}_Q)+Q^2\log(n(n-1))+n(Q-1)\log(n)$$

\bigskip

 Simulations $\rightarrow $ AIC better than BIC for low and moderate sample sizes.
\end{frame}



\section{Example}

\begin{frame}
\frametitle{Ecological networks}
\begin{itemize}
\item
{ \it It is interesting to contemplate a tangled bank, clothed with many plants of many kinds, with birds singing on the bushes, with various insects flitting about, and with worms crawling through the damp earth, and to reflect that these elaborately constructed forms, so different from each other, and dependent on each other in so complex a manner, have all been produced by laws acting around us}. C. Darwin, 1869.
\item
 Ecological networks = networks having species as vertices and interspecific interactions as edges
 \item
   invariant topological properties = common laws governing the structure of apparently diverse species assemblages ?
   \item
    Uncovering these laws is a crucial challenge for biologists because it would allow important advances in conservation and environmental management.
    \end{itemize}
\end{frame}


\begin{frame}
\frametitle{Data}
\begin{tabular}{p{5cm}p{6cm}}

\includegraphics[scale=0.4]{armi[1].jpg} & \includegraphics[scale=0.5]{Paj[1].jpg} \\
{ \tiny Host-parasite interaction between a young pine tree and the fungi species Armillaria ostoyae (image from C. Vacher web site)} &
{ \tiny Interaction network between tree species and parasitic fungi species in the French forests (image from C. Vacher web site)} \\
\end{tabular}

\begin{itemize}
\item
543 interactions between 51 forest tree taxa  and 154 parasitic fungal species. The network is composed of 205 vertices and 543 edges.
\item  bipartite graph :  tree-fungus interactions  are the only possible ones.
\item from the database of the French governmental organization in charge of forest health monitoring (the {\it D\'epartement Sant\'e des For\^ets (DSF)}) for the 1972-2005 period.
    \item methods used for data collection described in more detail in  {\small Vacher, C., Piou, D., and Desprez-Loustau, M.-L. (2008) {\it PLoS ONE} {\bf  3,} e1740.}
        \end{itemize}



\end{frame}

\begin{frame}
\frametitle{Results}
AIC criteria $\rightarrow$ $Q=5$  \\
\begin{center}
\begin{tabular}{|c|c|c|c|c|c|}

\hline
        & FT0   &  T1  &  T2    &  F1    &  F2     \\
 \hline
  FT0   & 0     &   0  &  0     &  0     &   0     \\
  T1    & 0     &   0  &  0     &  0.996 &   0     \\
  T2    & 0     &   0  &  0     &  0     &  0.985  \\
  F1    & 0     & 0.996&  0     &  0     &   0     \\
  F2    & 0     &   0  &  0.985 &  0     &   0     \\
\hline

\end{tabular}
\end{center}

\begin{itemize}
\item  EHV0=non connected species
\item  EHV1=T1 and EHV2=T2 two Extreme Hypothetical Trees
\item  EHV3=F1 and EHV4=F2 two Extreme Hypothetical Fungus
\item the only two connected EHVs are T1 and F1 and T2 and F2
\end{itemize}
\end{frame}

\begin{frame}
\frametitle{Triangular representations of tree species  and fungal species as a function of their number of interactions}

\includegraphics[scale=0.22]{Figure2A.pdf}
\includegraphics[scale=0.22]{Figure2B.pdf}
{ \small Vertical axis = degree of the vertices }\\
{ \small Horizontal axis = differentiation between two classes of species}
\end{frame}

\begin{frame}
\frametitle{Triangular representations of tree species and fungal species as a function of their phylogenetic origin}
\includegraphics[scale=0.22]{Figure3A.pdf}
\includegraphics[scale=0.22]{Figure3B.pdf}
{ \small The differentiation along the horizontal axis is due to the phylogenetic origin of Trees. Phylogenetic origin of Fungi does not matter!}
\end{frame}

\begin{frame}
\frametitle{Triangular representations of tree species and fungal species as a function of their introduction status}
\includegraphics[scale=0.22]{Figure4A.pdf}
\includegraphics[scale=0.22]{Figure4B.pdf}
 { \small Aliens are rapidly (less than 600 years) integrated}
\end{frame}

\section{Conclusions}
\begin{frame}
\frametitle{Conclusions}
\begin{itemize}
\item  EVMRG  appears as a good approach for synthesizing the heterogeneity of ecological networks. EVMRG  confirmed, with a single analysis, several results obtained in previous studies through different analyses.
\item  The EVMRG model is more flexible than the usual mixture model for it includes the possibility for a vertex to have intermediate connectivity properties.
\item    additional work is in progress to
understand the behavior of the maximum-likelihood estimates of $n$
parameters and $n^2$ observations when $n \rightarrow \infty.$
\end{itemize}
\end{frame}

\end{document}


