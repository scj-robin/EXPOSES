\documentclass{beamer}

% Beamer style
%\usetheme[secheader]{Madrid}
\usetheme{CambridgeUS}
\usecolortheme[rgb={0.65,0.15,0.25}]{structure}
%\usefonttheme[onlymath]{serif}
\beamertemplatenavigationsymbolsempty
%\AtBeginSubsection

% Packages
%\usepackage[french]{babel}
\usepackage[latin1]{inputenc}
\usepackage{color}
\usepackage{dsfont, stmaryrd}
\usepackage{amsmath, amsfonts, amssymb}
\usepackage{stmaryrd}
\usepackage{epsfig}
\usepackage{/Latex/astats}
%\usepackage[all]{xy}
\usepackage{graphicx}

% Commands
\definecolor{darkred}{rgb}{0.65,0.15,0.25}
\newcommand{\emphase}[1]{\textcolor{darkred}{#1}}
\newcommand{\refer}[1]{\textcolor{blue}{\sl \cite{#1}}}

% Symbols
\newcommand{\Abf}{{\bf A}}
\newcommand{\Beta}{\text{B}}
\newcommand{\Bcal}{\mathcal{B}}
\newcommand{\BIC}{\text{BIC}}
\newcommand{\dd}{\text{d}}
\newcommand{\Dcal}{\mathcal{D}}
\newcommand{\Esp}{\mathbb{E}}
\newcommand{\Ebf}{{\bf E}}
\newcommand{\Ibb}{\mathbb{I}}
\newcommand{\ICL}{\text{ICL}}
\newcommand{\Cov}{\mathbb{C}\text{ov}}
\newcommand{\Var}{\mathbb{V}}
\newcommand{\pen}{\text{pen}}
\newcommand{\Hcal}{\mathcal{H}}
\newcommand{\Jcal}{\mathcal{J}}
\newcommand{\Lcal}{\mathcal{L}}
\newcommand{\Mcal}{\mathcal{M}}
\newcommand{\Ncal}{\mathcal{N}}
\newcommand{\Ocal}{\mathcal{O}}
\newcommand{\Pbf}{{\bf P}}
\newcommand{\Pcal}{\mathcal{P}}
\newcommand{\Qcal}{\mathcal{Q}}
\newcommand{\Rcal}{\mathcal{R}}
\newcommand{\Vcal}{\mathcal{V}}
\newcommand{\Tbf}{{\bf T}}
\newcommand{\Ubf}{{\bf U}}
\newcommand{\Ybf}{{\bf Y}}
\newcommand{\Zbf}{{\bf Z}}
\newcommand{\Pibf}{\mbox{\mathversion{bold}{$\Pi$}}}
\newcommand{\mubf}{\mbox{\mathversion{bold}{$\mu$}}}
\newcommand{\thetabf}{\mbox{\mathversion{bold}{$\theta$}}}
\newcommand{\BP}{\text{BP}}
\newcommand{\EM}{\text{EM}}
\newcommand{\VEM}{\text{VEM}}
\newcommand{\VBEM}{\text{VB}}


%====================================================================
\title[Variational estimates]{Accuracy of Variational Estimates for
  Random Graph Mixture Models}

\author[S. Robin]{S. Gazal, J.-J. Daudin, S. Robin}

\institute[AgroParisTech / INRA]{AgroParisTech / INRA \\
  \bigskip
  \begin{tabular}{ccccc}
    \epsfig{file=../Figures/LogoINRA-Couleur.ps, width=2.5cm} &
    \hspace{.5cm} &
    \epsfig{file=../Figures/logagroptechsolo.eps, width=3.75cm} &
    \hspace{.5cm} &
    \epsfig{file=../Figures/Logo-SSB.eps, width=2.5cm} \\
  \end{tabular} \\
  \bigskip
  }

\date{MSTGA, Toulouse, Dec. 2009}
%====================================================================

%====================================================================
%====================================================================
\begin{document}
%====================================================================
%====================================================================

%====================================================================
\frame{\titlepage}
%====================================================================

%====================================================================
%====================================================================
\section{Mixture Model for Random Graphs}
%\subsection{Model}
\frame{ \frametitle{Mixture model (MixNet)}
%==================================================================== 
  We consider an undirected graph with $n$ nodes ($i, j = 1
  \dots n$) and binary edges:
  $$
  X_{ij} = \Ibb\{i \sim j\}.
  $$
  The nodes are spread into $Q$ groups ($q, \ell = 1 \dots Q$) with
  proportions
  $$
  (\alpha_1,\dots, \alpha_Q) =: \alpha.
  $$
  The groups to which each node belongs are independent:
  $$
  \{Z_i\} \text{ i.i.d } \sim \Mcal(1; \alpha).
  $$
  The edges are independent conditionally to the node's groups
  (\refer{DPR}):
  $$
  \{X_{ij}\} \text{ independent},
  \qquad
  X_{ij} | Z_{iq} Z_{j\ell} \sim \Bcal(\pi_{q\ell}).
  $$
  }

% %====================================================================
% %====================================================================
% \subsection{Examples}
% \frame{ \frametitle{Examples}
% %====================================================================
%   \includegraphics[width=1\textwidth]{../Figures/FigNetwork-Global.jpg} \\
%   }

%====================================================================
%\subsection{Likelihoods}
\frame{ \frametitle{Likelihoods}
%==================================================================== 
  The log-likelihood of the complete data is
  $$
  \log P(X,Z)=\frac{1}{2}\sum_{i, j \neq i}
  \sum_{q,l}  Z_{iq}Z_{j\ell}\log b_{ijq\ell}+\sum_i \sum_q
  Z_{iq}\log\alpha_q    
  $$
  where $b_{ijq\ell}=\pi_{q\ell}^{X_{ij}} (1-\pi_{q\ell})^{1-X_{ij}}$.

  \bigskip
  Its conditional expectation denoted by $\Qcal$ in the EM literature
  is
  \begin{eqnarray*} 
    \Qcal(X) & = & \Esp [\log P(Z,X) | X] \nonumber \\
    & = & \frac{1}{2} \sum_{i, j \neq i} \sum_{q,l} \Delta_{ijq\ell}
    \log b_{ijq\ell} + \sum_i \sum_q \tau_{iq} \log\alpha_q  
  \end{eqnarray*}  
  where $\tau_{iq}=\Esp(Z_{iq}| X )$ and $\Delta_{ijq\ell} =
  \Esp(Z_{iq}Z_{j\ell}| X )$.   
  }

%====================================================================
%====================================================================
\section{Inference}
%\subsection{EM}
\frame{ \frametitle{Regular EM}
%==================================================================== 
  \emphase{M-step.} Parameter estimates are straightforward and
  similar for all inference methods
  $$
  { \pi}^{\EM}_{q\ell}=\frac{ \sum_{i \neq j} X_{ij} \Delta_{ijq\ell}^{\EM}  }
  { \sum_{i \neq j}  \Delta_{ijq\ell}^{\EM}}  
  \qquad \text {and} \qquad
  { \alpha}^{\EM}_{q}= \frac{1}{n} \sum_{i} \tau_{iq}^{\EM}.
  $$

  \bigskip
  \emphase{E-step.} It aims at computing the conditional
  distribution of the unobserved data $Z$:
  $$
  \tau_{iq}^{\EM} = \sum_z z_{iq} P(Z=z|X)
  \qquad \text {and} \qquad
  \Delta_{ijq\ell}^{\EM} =  \sum_z z_{iq}z_{j\ell} P(Z=z|X)\enspace .
  $$
  Except for small datasets, $\sum_z$ can not be computed.

  \bigskip
  The methods presented hereafter provide \emphase{approximations of
  $\tau_{iq}$ and $\Delta_{ijq\ell}$}.  
  }

%====================================================================
%\subsection{Variational EM}
\frame{ \frametitle{Variational EM (VEM)}
%==================================================================== 
  The variational strategy (\refer{TSJ}) aims at maximizing a lower
  bound of $\log P(X)$
  \begin{eqnarray*} 
    \Jcal(X) & =
    & \log P(X)-KL(\Rcal_X(Z),P(Z|X)) \nonumber \\
    & = & \left[\frac12 \sum_{i \neq j} \sum_{q,l} \Delta^\VEM_{ijq\ell}
      \log b_{ijq\ell} + \sum_i \sum_q \tau^\VEM_{iq} \log\alpha_q  \right] 
  \end{eqnarray*}
  $\Rcal_X$ is chosen in a set of manageable distributions:
  $$
  \Rcal_X(Z)=\prod_i \prod_q \left(\tau_{iq}\right)^{Z_{iq}}
  $$
  which implies $\Delta^\VEM_{ijq\ell} =
  \tau^\VEM_{iq}\tau^\VEM_{j\ell}$.  
  
  \bigskip 
  $\tau^\VEM_{iq}$ satisfy a fix-point equation:
  $$
  {\tau}_{iq}^{\VEM} \propto {\alpha}_q^{\VEM} \prod_{j\neq i}
  \prod_{l} ({b}_{ijq\ell})^{{\tau}_{j\ell}^{\VEM}}.
  $$
  }

%====================================================================
%\subsection{Belief Propagation}
\frame{ \frametitle{Belief Propagation (BP)}
%==================================================================== 
  Based on a message passing principle (\refer{messagepassing}), we
  get a similar ${\tau}_{iq}$, but a new version of
  $\Delta_{ijq\ell}$:
  \begin{eqnarray*}
%     {\tau}_{iq}^{\BP} & \propto & {\alpha}_q \prod_{j\neq i}
%     \prod_{l} ({b}_{ijq\ell})^{{\tau}_{j\ell}^{\BP}},  \\
    \Delta_{ijq\ell}^{\BP}& \propto & \alpha_q \alpha_\ell b_{ijq\ell} 
    \prod_{k\neq i, j} \prod_{r}
    \left(b_{ikqr}\right)^{\tau_{kl}^{\BP}} 
    \left(b_{kjqr}\right)^{\tau_{kr}^{\BP}}.
  \end{eqnarray*}

  \bigskip
  It can be seen as a slight modification of the $\VEM$ approximation:
  $$
  \Delta_{ijq\ell}^{\BP} \propto \tau_{iq}^{\BP} \tau_{j\ell}^{\BP}
  \frac{b_{ijq\ell}}{ \prod_{r} \left[
      \left(b_{ijqr}\right)^{\tau_{jr}^{\BP}}
      \left(b_{ijrl}\right)^{\tau_{ir}^{\BP}} \right] }.
  $$
  }

%====================================================================
%\subsection{Variational Bayes EM}
\frame{ \frametitle{Variational Bayes EM ({\VBEM})}
%==================================================================== 
  The variational approximation can be applied to Bayesian inference;
  the parameter $\theta = (\alpha, \pi)$ is also viewed as an
  unobserved variable (\refer{BG}).
  
  \biqkip
  We aim at finding
  $$
  \Rcal^*_X = \arg\min KL(\Rcal_X(\theta, Z),P(\theta, Z|X))
  $$

  \bigskip
  If conjugate priors are used:
  $$
  P(X, \theta, Z) \propto \exp\{\phi(\theta)' [u^0 + u(X, Z)]\}, 
  $$
  close-form approximate conditional distributions of the form
  $$
  \Rcal_X(\theta, Z) = \Rcal_{X, \theta}(\theta) \Rcal_{X, Z}(Z)
  $$
  can be derived:
  \begin{eqnarray*}
    \Rcal_{X, \theta}(\theta) & \propto & \exp\{\phi(\theta)'
    \tilde{u}(X)\}, \qquad \tilde{u}(X) = u^0 + \overline{u}(X)\\  
    \Rcal_{X, Z}(Z) & \propto & \exp\{\overline{\phi}' u(X, Z)\}
  \end{eqnarray*}
  }

%====================================================================
%\subsection{Variational Bayes EM}
\frame{ \frametitle{{\VBEM} for MixNet}
%==================================================================== 
  For Dirichlet and Beta priors, we get:
%   $$
%   \alpha \sim \Dcal(n^0), \qquad   
%   \pi_{q\ell} \sim \Beta(\eta_{q\ell}^0, \zeta_{q\ell}^0).
%   $$
%   where $n^0 = (n_1^0, \dots, n_Q^0)$.

  \emphase{$\Rcal_{X, Z}(Z)$.}
  \begin{eqnarray*}
    \tau_{iq}^{\VBEM} &\propto&
    e^{\psi(\widetilde{n_q})-\psi\left(\sum_{l=1}^Q
        \widetilde{n}_\ell\right)} \prod_{j \neq i}^n \prod_{l=1}^Q
    e^{\tau_{j\ell}^{\VBEM} \left\{\psi(\widetilde{\zeta}_{q\ell}) -
        \psi(\widetilde{\eta}_{q\ell} + \widetilde{\zeta}_{q\ell}) +
        X_{ij}
        [\psi(\widetilde{\eta}_{q\ell})-\psi(\widetilde{\zeta}_{q\ell})]
      \right\})}    
  \end{eqnarray*}
  where $\psi$ is the first derivative of the $\Gamma$ function.

  \bigskip
  \emphase{$\Rcal_{X, \theta}(\theta)$.}
  $$
  (\alpha | X) \approx  \Dcal(\widetilde{n}), \qquad 
  (\pi_{q\ell} | X) \approx \Beta(\widetilde{\eta}_{q\ell},
  \widetilde{\zeta}_{q\ell}) 
  $$
  where
  $$
  \begin{array}{rcl}
  %\begin{eqnarray*}
    \widetilde{n}_q&=&n^0_q+\sum_i \tau^\VBEM_{iq}, \\
    \widetilde{\eta}_{q\ell} & = & \eta^0_{q\ell}+ \left(1 -
      \frac12 \mathds{1}_{q=l} \right) \sum_{i \neq j}
    X_{ij}\tau^\VBEM_{iq} \tau^\VBEM_{j\ell},  \\  
    \widetilde{\zeta}_{q\ell} & = & \zeta^0_{q\ell}+\left(1 -
      \frac12 \mathds{1}_{q= l} \right) \sum_{i \neq j}
    (1-X_{ij})\tau^\VBEM_{iq}\tau^\VBEM_{j\ell}. 
  %\end{eqnarray*}  
  \end{array}  
  $$
  }

%====================================================================
\section{Simulation Study}
%\subsection{Design}
\frame{ \frametitle{Simulation Design}
%==================================================================== 
  2-group MixNet model with parameters:
  \begin{description}
  \item[Case 1:] $\alpha=\left(\begin{array}{cc}0.6 &
        0.4\end{array}\right)$,  $\pi=\left(\begin{array}{cc}0.8 & 0.2
        \\0.2 & 0.5\end{array}\right)$;
  \item[Case 2:] $\alpha=\left(\begin{array}{cc}0.6 &
        0.4\end{array}\right)$, $\pi=\left(\begin{array}{cc}0.8 & 0.2
        \\0.2 & 0.3\end{array}\right)$.
  \end{description}
  500 graphs are simulated for each case and each graph size. 
  
  \bigskip \bigskip 
  The complete comparison of the 4 methods is only made on
  small graphs ($n = 18$) because of the computation time required by
  for EM.  }

%====================================================================
%\subsection{Bias, standard deviation and likelihood}
\frame{ \frametitle{Estimates, standard deviation and likelihood}
%==================================================================== 
  {\small
    \begin{tabular}{ccccccccc} 
      $n=18$ & $\alpha_1$ & $\pi_{11}$ & $\pi_{12}$ & $\pi_{22}$ &
      $\log P(X)$ \\   
      \hline
      True value & 60\% & 80\% & 20\% & 50\% & \\ 
      \hline 
      EM & 59.1 (13.1) & 78.5 (13.5) & 20.9 (8.4) & 50.9 (15.4) & -90.68 \\  
      {\VEM} & 57.7 (16.6) & 78.8 (12.4) & 22.4 (10.7) & 50.3 (14.6) & -90.87\\  
      BP & 57.9 (16.2) & 78.9 (12.3) & 22.2 (10.5) & 50.3 (14.5) & -90.85 \\  
      {\VBEM} & 58.1 (13.3) & 78.2 (9.7) & 21.6 (7.7) & 50.8 (13.3) & -90.71 \\ 
      \hline \hline
      True value & 60\% & 80\% & 20\% & 30\% & - \\ 
      \hline
      EM & 59.5 (14.1) & 78.7 (15.6) & 21.2 (8.7) & 30.3 (14.3) & -88.18 \\  
      {\VEM} & 55.6 (19.0) & 80.1 (14.0) & 24.0 (11.8) & 30.8 (13.8) & -88.54 \\  
      BP & 56.6 (17.8) & 80.0 (13.6) & 23.2 (11.0) & 30.8 (13.8) & -88.40 \\  
      {\VBEM} & 58.4 (14.6) & 77.9 (12.0) & 22.3 (9.3) & 32.1 (12.3) & -88.26 \\  
    \end{tabular} 
    }
  \bigskip
  \begin{itemize}
  \item All methods provide similar results.
  \item {\EM} achieves the best ones.
  \item {\BP} does not significantly improve {\VEM}.
  \end{itemize}
  }

%====================================================================
%\subsection{Influence of the graph size}
\frame{ \frametitle{Influence of the graph size}
%==================================================================== 
  Comparison of \textcolor{red}{{\VEM}: $\bullet$} and
  \textcolor{blue}{{\VBEM}: $+$} in case 2 (difficult). \\
  Left to right: $\alpha_1$, $\pi_{11}$, $\pi_{12}$, $\pi_{22}$.

  \bigskip
  \emphase{Means.} \\
  \includegraphics[width=1\textwidth]{../Figures/im-etudnVB1} \\

  \emphase{Standard deviations.} \\
  \includegraphics[width=1\textwidth]{../Figures/im-etudnVB2}
  
  \begin{itemize}
  \item {\VBEM} estimates converge more rapidly than {\VEM} ones.
  \item Their precision is also better.
  \end{itemize}
  }

%====================================================================
%\subsection{{\VBEM} Credibility intervals}
\frame{ \frametitle{{\VBEM} Credibility intervals}
%==================================================================== 
  \emphase{Actual level as a function of $n$.}\\
  \includegraphics[width=1\textwidth]{../Figures/im-ICQ2-2-new} \\
  $\alpha_1$: $+$, $\pi_{11}$: \textcolor{red}{$\triangle$},
  $\pi_{12}$: \textcolor{blue}{$\circ$}, $\pi_{22}$:
  \textcolor{green}{$\bullet$}

  \begin{itemize}
  \item For all parameters, {\VBEM} posterior credibility intervals
    achieve the nominal level (90\%), as soon as $n \geq 25$.
  \item \emphase{$\rightarrow$ the {\VBEM} approximation seems to work well}.
%   \item These may be due to the concentration of $P(Z|X)$ around the
%     true value of $Z$ (work in progress).
  \end{itemize}
  }

%====================================================================
%\subsection{Convergence rate of the {\VBEM} estimates}
\frame{ \frametitle{Convergence rate of the {\VBEM} estimates}
%==================================================================== 
  \emphase{Width of the posterior credibility intervals.}
  {$\alpha_1$}, \textcolor{red}{$\pi_{11}$},
  \textcolor{blue}{$\pi_{12}$}, \textcolor{green}{$\pi_{22}$}
  \\
  \includegraphics[width=1\textwidth]{../Figures/im-ICQ2-3} \\

  \begin{itemize}
  \item The width decreases as $1/\sqrt{n}$ for $\alpha_1$.
  \item It decreases as $1/n$ for $\pi_{11}$, $\pi_{12}$ and
    $\pi_{22}$.
  \item Consistent with the penalty of the ICL criterion
    of \refer{DPR}: 
    $$
    (Q-1)\log n + Q^2 \log[n(n-1)/2].
    $$
  \end{itemize}
  }

%====================================================================
\frame{ \frametitle{Why does {\VBEM} work so well? (off the record)}
%==================================================================== 
  \emphase{Work in progress:} Daudin \& Celisse are about to prove the 
  concentration of $P(Z|X)$ around the true value $z^*$, i.e.
  $$
  P(Z|X) \underset{n \rightarrow \infty}{\longrightarrow} \delta_{z^*}(Z)
  $$
  \bigskip
  \emphase{Intuition:} If this holds, 
  \begin{enumerate}[($i$)]
  \item The limit distribution $\delta_{z^*}(Z)$ belongs to the
    distribution class over which {\VBEM} approximation achieves
    maximisation, so it is reached;
  \item The joint conditional distribution $P(\theta, Z|X) =
    P(\theta|Z, X) P(Z|X)$ tends to $P(\theta|Z, X) \delta_{z^*}(Z)$.
    Again $P(\theta|Z, X)$ belongs to the distribution class of
    {\VBEM}, so it is also reached;
  \end{enumerate}
  And the variational approximation tends to be ... exact.
  }

%====================================================================
%====================================================================
\section{{\sl E. Coli} Regulatory Network}
\subsection{Comparison of {\VEM} and {\VBEM}}
\frame{ \frametitle{Comparison of {\VEM} and {\VBEM}}
%==================================================================== 
  Network are n = 338 operons, linked if one encodes a
  transcription factor that directly regulates the other one.  \\
  {\small
    \begin{tabular}{cccccc}
      $\pi_{ql}$ & 1 & 2 & 3 & 4 & 5 \\
      \hline 
      1 & 0.03 & 0.00 & 0.03 & 0.00 & 0.00 \\
      2 & 6.40 & 1.50 & 1.34 & 0.44 & 0.00 \\
      3 & 1.21 & 0.89 & 0.58 & 0.00 & 0.00 \\
      4 & 0.00 & 0.09 & 0.00 & 0.95 & 0.00 \\
      5 & 8.64 & 17.65 & 0.05 & 72.87 & 11.01 \\
      \hline
      $\alpha$ & 65.49 & 5.18 & 7.92 & 21.10 & 0.30 \\
      \hline \hline
      1  &  [0.02;0.04] &  [0.00;0.10] &  [0.01;0.08] &   [0.00;0.03] &  [0.02;1.34] \\
      2  &  [6.14;7.60] &  [0.61;3.68] &  [1.07;3.50] &   [0.05;0.54] &  [0.33;17.62] \\
      3  &  [1.20;1.72] &  [0.35;2.02] &  [0.56;1.92] &   [0.03;0.30] &  [0.19;10.57] \\
      4  &  [0.01;0.07] &  [0.04;0.51] &  [0.01;0.20] &   [0.76;1.27] &  [0.08;4.43] \\
      5  &  [6.35;12.70] &  [4.60;33.36] &  [4.28;24.37] &  [63.56;81.28] &  [5.00;95.00] \\
      \hline
      $\alpha$ & [59.65;74.38]&  [2.88;6.74] & [5.68;10.77] & [16.02;24.04] & [0.11;1.42]
%       \hline \hline
%       1  &  0.03  [0.02;0.04] &  0.03  [0.00;0.10] &  0.03  [0.01;0.08
%       ] &  0.01  [0.00;0.03] &  0.45  [0.02;1.34] \\
%       2  &  6.85  [6.14;7.60] &  1.87  [0.61;3.68] &  2.14  [1.07;3.50
%       ] &  0.23  [0.05;0.54] &  6.08  [0.33;17.62] \\
%       3  &  1.45  [1.20;1.72] &  1.04  [0.35;2.02] &  1.15  [0.56;1.92
%       ] &  0.13  [0.03;0.30] &  3.59  [0.19;10.57] \\
%       4  &  0.04  [0.01;0.07] &  0.22  [0.04;0.51] &  0.07  [0.01;0.20
%       ] &  1  [0.76;1.27] &  1.49  [0.08;4.43] \\
%       5  &  9.32  [6.35;12.70] &  16.64  [4.60;33.36] &  12.79  [4.28 ;
%       24.37] &  72.81  [63.56;81.28] &  50.00  [5.00;95.00] \\
%       \hline
%       $\alpha$ & 66.85 [59.65;74.38]&  4.64 [2.88;6.74] & 8.06 [5.68;10.77]
%       & 19.86 [16.02;24.04] & 0.60 [0.11;1.42]
    \end{tabular} 
    }
    
  {\VEM} and {\VBEM} estimates for the $Q=5$ group model (approximate
  90\% credibility intervals).

  }

%====================================================================
%\subsection{}
\frame{ \frametitle{Approximate posterior distribution}
%==================================================================== 
  \begin{centering}
%   \begin{tabular}{ll}
%     \begin{tabular}{p{5cm}}
%       Posterior distributions of the connexion probabilities
%       $\{\pi_{q\ell}\}$ (five first rows) and proportions $\{\alpha_q\}$
%       (last row) using {\VBEM} for the operon network of {\it E . Coli}.
%     \end{tabular}
%     &
%     \begin{tabular}{l}
      \includegraphics[width=.7\textwidth]{../Figures/im-pi1BVEM}\\        
      \includegraphics[width=.7\textwidth]{../Figures/im-pi2BVEM}\\
      \includegraphics[width=.7\textwidth]{../Figures/im-pi3BVEM}\\
      \includegraphics[width=.7\textwidth]{../Figures/im-pi4BVEM}\\
      \includegraphics[width=.7\textwidth]{../Figures/im-pi5BVEM}\\
      \hline \\
      \includegraphics[width=.7\textwidth]{../Figures/im-alphaBVEM}\\
%     \end{tabular}
%   \end{tabular}
    \end{centering}
  }

%====================================================================
%\subsection{}
\frame{ \frametitle{Comparison of {\VEM} and {\VBEM} classifications}
%==================================================================== 
  \begin{tabular}{cc}
  \includegraphics[width=.45\textwidth]{../Figures/im_EcoliVEM}
  &
  \includegraphics[width=.45\textwidth]{../Figures/im_EcoliBVEM} \\
  {\VEM} & {\VBEM}
  \end{tabular}

  \begin{itemize}
  \item Only 4 nodes have different classifications.
  \end{itemize}
  }

%====================================================================
%====================================================================
\bibliographystyle{/Latex/astats}
%\bibliographystyle{elsarticle-harv}
\bibliography{biblio}
%====================================================================

%====================================================================
%====================================================================
\end{document}
%====================================================================
%====================================================================
