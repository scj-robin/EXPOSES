\documentclass{beamer}

% Beamer style
%\usetheme[secheader]{Madrid}
\usetheme{CambridgeUS}
\usecolortheme[rgb={0.65,0.15,0.25}]{structure}
%\usefonttheme[onlymath]{serif}
\beamertemplatenavigationsymbolsempty
%\AtBeginSubsection

% Packages
%\usepackage[french]{babel}
\usepackage[latin1]{inputenc}
\usepackage{color}
\usepackage{dsfont, stmaryrd}
\usepackage{amsmath, amsfonts, amssymb}
\usepackage{stmaryrd}
\usepackage{epsfig}
\usepackage{url}
\usepackage{/Latex/astats}
%\usepackage[all]{xy}
\usepackage{graphicx}

% Commands
\definecolor{darkred}{rgb}{0.65,0.15,0.25}
\newcommand{\emphase}[1]{\textcolor{darkred}{#1}}
%\newcommand{\emphase}[1]{{#1}}
\newcommand{\paragraph}[1]{\textcolor{darkred}{#1}}
\newcommand{\refer}[1]{\textcolor{blue}{\sl \cite{#1}}}
\newcommand{\Refer}[1]{\textcolor{blue}{\sl #1}}
\newcommand{\newblock}{}

% Symbols
\newcommand{\Abf}{{\bf A}}
\newcommand{\Bias}{\mathbb{B}}
\newcommand{\Bbf}{{\bf B}}
\newcommand{\Beta}{\text{B}}
\newcommand{\Bcal}{\mathcal{B}}
\newcommand{\BIC}{\text{BIC}}
\newcommand{\dd}{\text{d}}
\newcommand{\dbf}{{\bf d}}
\newcommand{\Dcal}{\mathcal{D}}
\newcommand{\Esp}{\mathbb{E}}
\newcommand{\Ebf}{{\bf E}}
\newcommand{\Ecal}{\mathcal{E}}
\newcommand{\Gcal}{\mathcal{G}}
\newcommand{\Gbf}{{\bf G}}
\newcommand{\Gam}{\mathcal{G}\mbox{am}}
\newcommand{\Ibb}{\mathbb{I}}
\newcommand{\Ibf}{{\bf I}}
\newcommand{\ICL}{\text{ICL}}
\newcommand{\Cov}{\mathbb{C}\text{ov}}
\newcommand{\Corr}{\mathbb{C}\text{orr}}
\newcommand{\Var}{\mathbb{V}}
\newcommand{\Vsf}{\mathsf{V}}
\newcommand{\pen}{\text{pen}}
\newcommand{\Fcal}{\mathcal{F}}
\newcommand{\Hbf}{{\bf H}}
\newcommand{\Hcal}{\mathcal{H}}
\newcommand{\Jcal}{\mathcal{J}}
\newcommand{\Kbf}{{\bf K}}
\newcommand{\Lcal}{\mathcal{L}}
\newcommand{\Mcal}{\mathcal{M}}
\newcommand{\mbf}{{\bf m}}
\newcommand{\mum}{\mu(\mbf)}
\newcommand{\Ncal}{\mathcal{N}}
\newcommand{\Nbf}{{\bf N}}
\newcommand{\Nm}{N(\mbf)}
\newcommand{\Ocal}{\mathcal{O}}
\newcommand{\Obf}{{\bf 0}}
\newcommand{\Omegas}{\underset{s}{\Omega}}
\newcommand{\Pbf}{{\bf P}}
\newcommand{\Pcal}{\mathcal{P}}
\newcommand{\Qcal}{\mathcal{Q}}
\newcommand{\Rbb}{\mathbb{R}}
\newcommand{\Rcal}{\mathcal{R}}
\newcommand{\sbf}{{\bf s}}
\newcommand{\Sbf}{{\bf S}}
\newcommand{\Scal}{\mathcal{S}}
\newcommand{\Ucal}{\mathcal{U}}
\newcommand{\Vcal}{\mathcal{V}}
\newcommand{\Tbf}{{\bf T}}
\newcommand{\ubf}{{\bf u}}
\newcommand{\Ubf}{{\bf U}}
\newcommand{\Vbf}{{\bf V}}
\newcommand{\Wbf}{{\bf W}}
\newcommand{\xbf}{{\bf x}}
\newcommand{\Xbf}{{\bf X}}
\newcommand{\ybf}{{\bf y}}
\newcommand{\Ybf}{{\bf Y}}
\newcommand{\zbf}{{\bf z}}
\newcommand{\Zbf}{{\bf Z}}
\newcommand{\betabf}{\mbox{\mathversion{bold}{$\beta$}}}
\newcommand{\pibf}{\mbox{\mathversion{bold}{$\pi$}}}
\newcommand{\Sigmabf}{\mbox{\mathversion{bold}{$\Sigma$}}}
\newcommand{\gammabf}{\mbox{\mathversion{bold}{$\gamma$}}}
\newcommand{\mubf}{\mbox{\mathversion{bold}{$\mu$}}}
\newcommand{\nubf}{\mbox{\mathversion{bold}{$\nu$}}}
\newcommand{\Thetabf}{\mbox{\mathversion{bold}{$\Theta$}}}
\newcommand{\thetabf}{\mbox{\mathversion{bold}{$\theta$}}}
\newcommand{\BP}{\text{BP}}
\newcommand{\EM}{\text{EM}}
\newcommand{\VEM}{\text{VEM}}
\newcommand{\VBEM}{\text{VB}}
\newcommand{\cst}{\text{cst}}
\newcommand{\obs}{\text{obs}}
\newcommand{\ra}{\emphase{\mathversion{bold}{$\rightarrow$}~}}
\newcommand{\QZ}{Q_{\Zbf}}
\newcommand{\Qt}{Q_{\thetabf}}

% Directory
\newcommand{\fignet}{/RECHERCHE/RESEAUX/Exposes/Figures}
\newcommand{\figmotif}{/RECHERCHE/RESEAUX/Motifs/Figures}


%--------------------------------------------------------------------
\title[Statistical models for quantitative variations]{Some
  statistical modeling approaches for quantitative variations}

\author{S. Robin}

\institute[INRA / AgroParisTech]{INRA / AgroParisTech \\
  \bigskip
  \begin{tabular}{ccccc}
    \epsfig{file=/RECHERCHE/RESEAUX/Exposes/Figures/LogoINRA-Couleur.ps,
      width=2.5cm} & 
    \hspace{.5cm} &
    \epsfig{file=/RECHERCHE/RESEAUX/Exposes/Figures/logagroptechsolo.eps,
      width=3.75cm} & 
    \hspace{.5cm} &
    \epsfig{file=/RECHERCHE/RESEAUX/Exposes/Figures/Logo-SSB.eps,
      width=2.5cm} \\ 
  \end{tabular} \\
  \bigskip
  }

\date[G2DC school]{G2DC school, May 2011, La Colle sur Loup}
%--------------------------------------------------------------------

%--------------------------------------------------------------------
%--------------------------------------------------------------------
\begin{document}
%--------------------------------------------------------------------
%--------------------------------------------------------------------

%--------------------------------------------------------------------
\frame{\titlepage}

% %--------------------------------------------------------------------
% \frame{ \frametitle{Outline}
%   \tableofcontents}

%--------------------------------------------------------------------
%--------------------------------------------------------------------
\section*{Modeling quantitative variations}
%--------------------------------------------------------------------
\frame{ \frametitle{Modeling quantitative variations}

  \paragraph{Statistical modeling} is the classical framework to describe the
  way data have been (or will be) collected, accounting for the
  intrinsic variability of the biological processes.

  \pause\bigskip\bigskip
  \paragraph{Complex dependency structures.} Genetic and phenotypic
  characteristics of living organisms may typically depend on space,
  time, pedigree, etc... \\
  \ra All these influences induce complex --~well described or not~--
  dependency between the observations.

  \pause\bigskip\bigskip
  \paragraph{High dimension of the observations.} High throughput
  technologies currently provide thousands to billions of data
%   \begin{itemize}
%   \item thousands (gene expression), 
%   \item millions (SNP) 
%   \item or even billions (whole genome sequence) 
%   \end{itemize}
%   of measurement 
  per individual, which almost always exceeds the
  number of individuals:
  $$
  \emphase{p \gg n.}
  $$
  }

% %--------------------------------------------------------------------
% \frame{ \frametitle{Two specificities of genetic studies (among
%     others) (1/2)}

%   \paragraph{1 - Complex dependency structures.} Genetic and phenotypic
%   characteristics of living organisms may typically depend on
%   \begin{itemize}
%   \item space (where they live),
%   \item time (when the observation is made), 
%   \item pedigree (how they are related),
%   \item \dots
%   \end{itemize}
%   All these influences induce complex --~well described or not~--
%   dependency between the observations.

%   \bigskip\bigskip\pause
%   \paragraph{Graphical models} constitute a fairly general way to
%   describe such dependencies, using hidden variables, hierarchical
%   structures, etc. 

%   \bigskip
%   \paragraph{Their inference} often raise difficult statistical issues
%   for which both frequentist and Bayesian approaches have been
%   developed.

%   \bigskip
%   \paragraph{Still, for very complex models,} only approximate inference
%     techniques are manageable.
%   }

% %--------------------------------------------------------------------
% \frame{ \frametitle{Two specificities of genetic studies (among
%     others) (2/2)}

%   \paragraph{2 - High dimension of the observations.} High throughput
%   technologies currently provide 
%   \begin{itemize}
%   \item thousands (gene expression), 
%   \item millions (SNP) 
%   \item or even billions (whole genome sequence) 
%   \end{itemize}
%   of measurement per individual, which almost always exceeds the
%   number of individuals:
%   $$
%   \emphase{p \gg n.}
%   $$
  
%   \bigskip\pause
%   \paragraph{Regularization techniques.} Even very classical methods,
%   such as linear regression need to be adapted to the $p \gg n$
%   situation. 
%   \begin{itemize}
%   \item Ridge or LASSO approaches have become a standard way to cope
%     with this dimensionality issues.
%   \end{itemize}

%   }

%--------------------------------------------------------------------
\frame{ \frametitle{Outline}
  \tableofcontents}

%--------------------------------------------------------------------
%--------------------------------------------------------------------
\section{Complex dependency structure}
\frame{ \frametitle{Complex dependency structure} }
%--------------------------------------------------------------------

%--------------------------------------------------------------------
\subsection{Graphical models}
%--------------------------------------------------------------------
\frame{ \frametitle{Graphical models}
  
  \paragraph{Definition:} A graphical model is a graph, i.e. a set of
  \emphase{nodes linked by edges}.
  \begin{itemize}
  \item Each node represent a variable involved in the model.
  \item Edges display the dependency between the variables.
  \end{itemize}

  \pause\bigskip
  \paragraph{Interest:} Graphical models constitutes a natural way to
  describe a complex model and to apprehend the underlying dependency
  structure.

  \pause\bigskip\bigskip
  \paragraph{Properties:} The topology of the graph (chains, loops,
  trees, etc.) gives insights about
  \begin{itemize}
  \item the dependencies between the variables,
  \item the conditional dependencies between the variables,
  \item the way the likelihood of the data can be calculated (or not).
  \end{itemize}
  }

%--------------------------------------------------------------------
\frame{ \frametitle{An agronomic design}
  \begin{tabular}{cc}
    \hspace{-.5cm}
    \begin{tabular}{p{.5\textwidth}}
      \paragraph{Incomplete block design:} \\
      \begin{tabular}{c|ccc}
        & Bl. 1 & Bl. 2 & Bl. 3 \\
        \hline
        Var 1 & $T_{11}$ & & $T_{13}$ \\
        Var 2 & $T_{21}$ & $T_{22}$ & \\
        Var 3 & & $T_{32}$ & $T_{33}$ \\
      \end{tabular}

      \bigskip
      \onslide+<2->{        
        \paragraph{Variance component model:} 
        $$
        T_{ij} = \mu + V_i + B_j + E_{ij}
        $$
        }

      \onslide+<3>{        
        \paragraph{Marginal model:} 
        $$
        T_{ij} = \mu + \varepsilon_{ij}, 
        \qquad
        \Var(\Tbf) = \Sigmabf
        $$
        }
    \end{tabular}
    & 
    \hspace{-.5cm}
    \begin{tabular}{p{.5\textwidth}}
      \begin{overprint}
        \onslide<2>
        \epsfig{file=../Figures/GM-FactDes-BVT.eps, clip=,
          width=0.6\textwidth}
        \onslide<3>
        \epsfig{file=../Figures/GM-FactDes-T_BV.eps, clip=,
          width=0.6\textwidth}
      \end{overprint}
    \end{tabular}
  \end{tabular}
  }

%--------------------------------------------------------------------
\frame{ \frametitle{Animal breeding}
  \begin{tabular}{cc}
    \hspace{-.5cm}
    \begin{tabular}{p{.45\textwidth}}
      \paragraph{Aim:} study the variations of a quantitative trait.
      
      \bigskip
      \paragraph{Data:} 
      \begin{itemize}
      \item a pedigree relating a set of individuals;
      \item the trait observed for some of them.
      \end{itemize}

      \bigskip
      \paragraph{Typical question:} 
      estimate the heritability of the trait.
    \end{tabular}
    & 
    %\hspace{-.5cm} 
    \pause
    \begin{tabular}{p{.5\textwidth}}
      \paragraph{A toy example:} \\ ~\\
      \epsfig{file=../Figures/GenPerf-Arbre.ps, width=0.5\textwidth,
        bbllx=105, bblly=70, bburx=440, bbury=290, clip=} \\
      Source: \Refer{E. Verrier}
%       The picture displays
%       \begin{itemize}
%       \item the pedigree;
%       \item the observed phenotypes.
%       \end{itemize}
    \end{tabular}
  \end{tabular}
  }

%--------------------------------------------------------------------
\frame{ \frametitle{Animal breeding}
  \begin{tabular}{ll}
    \hspace{-.5cm}
    \begin{tabular}{p{.5\textwidth}}
      \paragraph{From the graphical representation...} \\ 
      \epsfig{file=../Figures/GenPerf-Arbre.ps, width=0.5\textwidth,
        bbllx=105, bblly=70, bburx=440, bbury=290, clip=} \\
%       The picture displays
%       \begin{itemize}
%       \item the pedigree;
%       \item the observed phenotypes.
%       \end{itemize} 
      ~\\
      \vspace{2.2cm}
    \end{tabular}
    & 
    \hspace{-.5cm} \pause
    \begin{tabular}{p{.5\textwidth}}
      \paragraph{..to the graphical model} \\
      \epsfig{file=../Figures/GM-GenPerf-GP.eps, clip=,
      width=0.6\textwidth} \\
    Denoting
    \begin{itemize}
    \item $G_i = $ genotype of individual $i$;
    \item $T_i = $ trait for individual $i$.
    \end{itemize}
    \end{tabular}
  \end{tabular}
  }

%--------------------------------------------------------------------
\frame{ \frametitle{Animal breeding}
  \begin{tabular}{cc}
    \hspace{-.5cm}
    \begin{tabular}{p{.5\textwidth}}
      \onslide+<1->{
      \paragraph{The complete model} involves \emphase{unobserved
        variables}, e.g. the genotypes of the individual s: 
      $$
      \Tbf = \Xbf \thetabf + \Abf \Gbf + \Ebf
      $$              
      }

      \onslide+<2>{
      \paragraph{The marginal model} describes the
      correlations between the observed variables: 
      $$
      \Tbf = \Xbf \thetabf + \Ebf, \qquad \Var(\Ebf) = \Sigmabf
      $$        
      inherited from their connections with the unobserved variables.
      }
    \end{tabular}
    & 
    \hspace{-.5cm}
    \begin{tabular}{p{.5\textwidth}}
      \begin{overprint}
        \onslide<1>
        \epsfig{file=../Figures/GM-GenPerf-GP.eps, clip=,
          width=0.6\textwidth}
        \onslide<2>
        \epsfig{file=../Figures/GM-GenPerf-P.eps, clip=,
          width=0.6\textwidth}
      \end{overprint}
    \end{tabular}
  \end{tabular}
  }

%--------------------------------------------------------------------
\frame{ \frametitle{Population genetic structure}
  \paragraph{Data:} Genotypes of a set of individuals at $p$ loci:
  $G_{i1}, \dots, G_{ip}$.

  \bigskip
  \paragraph{Aim:} Find a structure in terms of populations.

  \begin{tabular}{cc}
    \hspace{-.5cm}
    \begin{tabular}{p{.5\textwidth}}
      \pause
      \paragraph{First modeling = Mixture model}
      \begin{itemize}
      \item Each individual belongs to one (unknown) population
        $P_i$.
      \item The allelic frequencies at each locus depends on the
        population.
      \end{itemize}
      ~\\~\\
    \end{tabular}
    & 
    %\hspace{-.5cm}
    \begin{tabular}{p{.5\textwidth}}
      \pause
      \epsfig{file=../Figures/GM-Structure-Mixture.eps, clip=,
        width=0.6\textwidth}
    \end{tabular}
  \end{tabular}
  }

%--------------------------------------------------------------------
\frame{ \frametitle{Population genetic structure}
  \begin{tabular}{cc}
    \hspace{-.5cm}
    \begin{tabular}{p{.5\textwidth}}
      \onslide+<1->{
        \paragraph{More subtle modeling} than the simple mixture can
        be proposed.
        }     

      \onslide+<2->{
        \bigskip\bigskip
        \paragraph{'Structure' software:} the origin may be
        different at each locus. \\
        \ra {'Admixture'}
        }

      \onslide+<3->{
        \bigskip\bigskip
        \paragraph{Linkage along the genome:} the origin of
        neighbor loci are dependent. \\ 
        \ra {'Hidden Markov model' (HMM)}
        }
    \end{tabular}
    & 
    %\hspace{-.5cm}
    \begin{tabular}{p{.5\textwidth}}
      \vspace{1cm}
      \begin{overprint}
        \onslide<1>
        \epsfig{file=../Figures/GM-Structure-Mixture.eps, clip=,
          width=0.6\textwidth}
        \onslide<2>
        \epsfig{file=../Figures/GM-Structure-Admixture.eps, clip=,
          width=0.6\textwidth}
        \onslide<3>
        \epsfig{file=../Figures/GM-Structure-HMM.eps, clip=,
          width=0.6\textwidth}
      \end{overprint}
    \end{tabular}
  \end{tabular}
  }

%--------------------------------------------------------------------
\frame{ \frametitle{Calculating the likelihood}

  \paragraph{Many different inference methods do exist} but
  likelihood-based methods are among the most popular.
  
  \bigskip
  \paragraph{The complexity} of its calculation depends on the
  topology of the graph. 
  

  \begin{overprint}
    \onslide<2>
    \begin{tabular}{cc}
      \hspace{-.5cm}
      \begin{tabular}{p{.5\textwidth}}
        ~\\
        \paragraph{Model with unobserved variables:}
        $ T_{ij} = \mu + V_i + B_j + E_{ij} $

        \bigskip
        \paragraph{Likelihood:}
        \begin{eqnarray*}
          p(\Vbf, \Bbf, \Tbf) & = & \prod_i p(V_i; \sigma^2_V) \\
          & & \times \prod_j p(B_j; \sigma^2_B) \\
          & & \times \prod_{i,j} p(T_{ij}|V_i, B_j; \mu, \sigma^2_E)
        \end{eqnarray*}          
      \end{tabular}
      & 
      %\hspace{-.5cm}
      \begin{tabular}{p{.5\textwidth}}
        \epsfig{file=../Figures/GM-FactDes-BVT.eps, clip=,
          width=0.6\textwidth}
      \end{tabular}
    \end{tabular}
    \onslide<3>
    \begin{tabular}{cc}
      \hspace{-.5cm}
      \begin{tabular}{p{.5\textwidth}}
        \paragraph{Model with observed variables only:}
         $$
         T_{ij} = \mu + \varepsilon_{ij}, 
         \qquad
         \Var(\Tbf) = \Sigmabf
         $$

        \bigskip
        \paragraph{Likelihood:}
        $$
        p(\Tbf) = p(T_{11}, \dots T_{33}; \mu,
        \underset{\Sigmabf}{\underbrace{\sigma^2_V, \sigma^2_B, \sigma^2_E}})  
        $$
        ~\\~\\~\\
      \end{tabular}
      &
    %\hspace{-.5cm}
      \begin{tabular}{p{.5\textwidth}}
        \epsfig{file=../Figures/GM-FactDes-T_BV.eps, clip=,
          width=0.6\textwidth}
      \end{tabular}
    \end{tabular}
  \end{overprint}
  }

%--------------------------------------------------------------------
\frame{ \frametitle{Calculating the likelihood}
  \begin{tabular}{cc}
    \hspace{-.5cm}
    \begin{tabular}{p{.5\textwidth}}
      \onslide+<1->{
        \paragraph{Factorization.} The likelihood can be
        factorized between isolated sets of variables.
        }
      
      \onslide+<3->{
        \bigskip\bigskip        
        \paragraph{Recursion.} It can be efficiently computed 
        for directed acyclic graphs (DAG), i.e.'tree-like'.
        }
      
      \onslide+<5->{
        \bigskip\bigskip        
        \paragraph{Non-factorable cases.} For very intricate
        dependencies (loops, large cliques, ...), the calculation may be
        out of reach.
      }
    \end{tabular}
    & 
    \hspace{-.5cm}
    \begin{tabular}{p{.5\textwidth}}
      \vspace{1cm}
      \begin{overprint}
        \onslide<1>
        \epsfig{file=../Figures/GM-GenPerf-GP.eps, clip=,
          width=0.6\textwidth}
        \onslide<2>
        \epsfig{file=../Figures/GM-GenPerf-P.eps, clip=,
          width=0.6\textwidth}
        \onslide<3>
        \epsfig{file=../Figures/GM-Structure-HMM.eps, clip=,
          width=0.6\textwidth}
        \onslide<4>
        \epsfig{file=../Figures/GM-Phylogeny.eps, clip=,
          width=0.6\textwidth}
        \onslide<5>
        \epsfig{file=../Figures/GM-HMRF-Z.eps, clip=,
          width=0.6\textwidth}
        \onslide<6>
        \epsfig{file=../Figures/GM-HMRF-ZX.eps, clip=,
          width=0.6\textwidth}
      \end{overprint}
    \end{tabular}
  \end{tabular}

  }

%--------------------------------------------------------------------
\subsection{Frequentist inference}
%--------------------------------------------------------------------
\frame{ \frametitle{Frequentist inference: Maximum likelihood}

  \paragraph{Some notations:}
  \begin{itemize}
  \item $\thetabf=$ parameters; 
    \pause
  \item $\Xbf=$ observed variables \ra \emphase{'Observed' likelihood}
      $= p (\Xbf; \thetabf)$;
    \pause
  \item $\Zbf=$ unobserved variables \ra \emphase{'Complete' likelihood}
      $= p (\Xbf, \Zbf; \thetabf)$.
  \end{itemize}  

  \pause\bigskip\bigskip
  \paragraph{General principle.} Estimate $\thetabf$ with the value that
  maximizes the 'observed' likelihood of the observed data
  $$
  \widehat{\thetabf} = \arg\max_{\thetabf} p(\Xbf; \thetabf) 
  \pause  
  = \arg\max_{\thetabf} \int p(\Xbf, \Zbf; \thetabf) \;\dd\Zbf.
  $$

  \paragraph{For complex models} the maximization is achieved numerically via
  \begin{itemize}
  \item deterministic optimization algorithms (e.g. Newton-Raphson);
  \item stochastic optimization algorithms (e.g. simulated annealing);
  \item combination of both.
  \end{itemize}
  }

%--------------------------------------------------------------------
\frame{ \frametitle{E-M strategies}

  \paragraph{Models with unobserved variables.}
  \begin{enumerate}
  \item We are often interested in the value of the unobserved
    variables $\Zbf$ (e.g. BLUP, classification, etc.).
  \item The complete likelihood $p(\Xbf, \Zbf; \thetabf)$ has often a
    simpler form than the observed (i.e. marginal) one $p(\Xbf;
    \thetabf)$.
  \end{enumerate}

  \pause\bigskip
  \paragraph{E-M algorithms} alternate
  \begin{description}
  \item[E-step:] calculation of the conditional distribution of the
    unobserved variables: $p(\Zbf|\Xbf; \widehat{\thetabf})$.
  \item[M-step:] maximization of the (expectation of) the complete
    likelihood: $\widehat{\thetabf} = \arg\max_{\thetabf} p(\Xbf, \Zbf;
    \thetabf)$.
  \end{description}
  to maximize the observed likelihood $p(\Xbf;\thetabf)$ (at least in
  theory).  }

%--------------------------------------------------------------------
\frame{ \frametitle{Example for 'Structure' (at a given locus)}
  \paragraph{Data:} Genotypes $G_i = G_{it}$.
  
  \bigskip
  \paragraph{Parameters:} 
  \begin{itemize}
  \item $\pi_p =$ proportion of individuals in population $p$,
  \item $\theta_{pg} =$ frequency of allele $g$ in population $p$.
  \end{itemize}
  
  \bigskip\pause
  \paragraph{E-step:} using Bayes formula
  $$
  \tau_{ip} := \Pr\{P_i = p | G_i=g\} = \frac{\Pr\{P_i = p,
    G_i=g\}}{\Pr\{G_i=g\}} = \frac{\widehat{\pi}_p
    \widehat{\theta}_{pg}}{\sum_q \widehat{\pi}_q
    \widehat{\theta}_{pg}}. 
  $$

  \pause
  \paragraph{M-step:} 
  $$
  \widehat{\pi}_p = \frac{\sum_i \tau_{ip}}{n}, 
  \qquad
  \widehat{\theta}_{pg} = \frac{\sum_i \tau_{ip} \Ibb\{G_i=g\}}{\sum_i
    \tau_{ip}}. 
  $$ 
  }

%--------------------------------------------------------------------
\frame{ \frametitle{Example for 'Structure' (at a given locus)}

  \paragraph{Interpretation.} The probability $\tau_{ip}(t)$ is the
  probability for locus $t$ of individual $i$ to come from population
  $p$, given its genotype:
  $$
  \tau_{ip}(t) = \Pr\{P_{it} = p | G_{it}\}.
  $$

  \pause
  \paragraph{Graphical model:} remind 
  $$
  \epsfig{file=../Figures/GM-Structure-Admixture-1-tmp.eps, clip=,
    width=.45\textwidth}
  $$
  
  \pause
  \paragraph{Structure output:} for individual $i$
  \epsfig{file=../Figures/Rosenberg4.ps, clip=,width=.95\textwidth} \\
   $\tau_{ip}(t) =$ height of the bar with color $p$ at locus $t$ 
  }

%--------------------------------------------------------------------
\frame{ \frametitle{Limitations of E-M strategies}

  \begin{tabular}{cc}
    \hspace{-.5cm}
    \begin{tabular}{p{.5\textwidth}}
      \paragraph{The conditional distribution $p(\Zbf|\Xbf)$}
      can be
      \begin{itemize}
      \item \onslide+<1->{straightforward (e.g. ad-mixture models),}
      \item \onslide+<2->{non explicit (e.g. HMM, molecular evolution),}
      \item \onslide+<3->{not possible for a large number of hidden
          variables or very complex dependency structure.}
      \end{itemize}
      
      \onslide+<5>{
        \paragraph{In practice.}
        \begin{itemize}
        \item Computation time,
        \item Local maxima,
        \item Sensitivity to initialization.
        \end{itemize}
        }
    \end{tabular}
    & 
    \hspace{-.5cm}
    \begin{tabular}{p{.5\textwidth}}
      \begin{overprint}
         \onslide<1>
         \epsfig{file=../Figures/GM-Structure-Admixture.eps, clip=,
           width=0.6\textwidth}
         \onslide<2>
         \epsfig{file=../Figures/GM-Structure-HMM.eps, clip=,
           width=0.6\textwidth}
         \onslide<3>
         \epsfig{file=../Figures/GM-FactDes-BVT.eps, clip=,
           width=0.6\textwidth}
         \onslide<4->
         \epsfig{file=../Figures/GM-FactDes-BV.eps, clip=,
           width=0.6\textwidth}
      \end{overprint}
    \end{tabular}
  \end{tabular}
  }

%--------------------------------------------------------------------
\subsection{Bayesian inference}
%--------------------------------------------------------------------
\frame{ \frametitle{Bayesian inference}
  
  \paragraph{Bayesian paradigm:} parameters ($\thetabf$) are
  themselves random.

  \pause\bigskip
  \paragraph{Prior distribution.} According to the prior (absence of)
  knowledge a distribution a priori is set for the parameters:
  \emphase{$p(\thetabf)$}. 

  \pause\bigskip
  \paragraph{Model.} The model specifies the way the data $\Xbf$ depend on
  the parameters trough the likelihood \emphase{$p(\Xbf  | \thetabf)$}.

  \pause\bigskip
  \paragraph{Posterior distribution.} The aim of Bayesian inference is
  to compute the distribution of the $\thetabf$ given the data $\Xbf$
  (using \emphase{Bayes rule}):
  $$
  p(\thetabf | \Xbf) = \frac{p(\thetabf, \Xbf)}{p(\Xbf)} =
  \frac{p(\thetabf) p(\Xbf|\thetabf)}{p(\Xbf)} = \frac{p(\thetabf)
    p(\Xbf|\thetabf)}{\int p(\thetabf) p(\Xbf|\thetabf) \;\dd \thetabf}.
  $$

  \pause\bigskip
  \paragraph{Hypothesis testing} does not (formally) make sense in a
  Bayesian setting. 
  }

%--------------------------------------------------------------------
\frame{ \frametitle{The mechanics}
  \begin{tabular}{cc}
    \hspace{-.5cm}
    \begin{tabular}{p{.5\textwidth}}
      \begin{enumerate}
        \onslide+<1->{
        \item Set a prior distribution \textcolor{blue}{$p(\theta)$}.
          }
      \onslide+<2->{
        \item Get the observed data $X$.
        }
      \onslide+<3->{
        \item Compute the posterior distribution
          \textcolor{red}{$p(\theta|X)$}. 
        }
      \onslide+<4->{
        \item Compute any posterior quantity of interest: posterior
          mean, credibility interval, etc.
        }
      \end{enumerate}
      
      \onslide+<5->{
        \bigskip\bigskip
        The posterior depends on both the data and the prior.
        }
    \end{tabular}
    & 
    \hspace{-.5cm}
    \begin{tabular}{p{.5\textwidth}}
      \vspace{-.5cm}
      \begin{overprint}
        \onslide<1>
        \epsfig{file=../Figures/BetaBinom-n30-x4-a1-b1-Prior.ps, clip=,
          height=0.5\textwidth, angle=270}
        \onslide<2>
        \epsfig{file=../Figures/BetaBinom-n30-x4-a1-b1-Data.ps, clip=,
          height=0.5\textwidth, angle=270}
        \onslide<3>
        \epsfig{file=../Figures/BetaBinom-n30-x4-a1-b1-Posterior.ps, clip=,
          height=0.5\textwidth, angle=270}
        \onslide<4->
        \epsfig{file=../Figures/BetaBinom-n30-x4-a1-b1-Credibility.ps, clip=,
          height=0.5\textwidth, angle=270}
      \end{overprint} \\  
      \vspace{-1cm}
      \begin{overprint}
        \onslide<5>
        \epsfig{file=../Figures/BetaBinom-n30-x4-a10-b5-Data.ps, clip=,
          height=0.5\textwidth, angle=270}
        \onslide<6>
        \epsfig{file=../Figures/BetaBinom-n30-x4-a10-b5-Posterior.ps, clip=,
          height=0.5\textwidth, angle=270}
      \end{overprint} 
    \end{tabular}
  \end{tabular}
  }

%--------------------------------------------------------------------
\frame{ \frametitle{Sampling integrals}

  \paragraph{Computing integrals.}
  \begin{itemize} 
  \item Due the Bayes formula, Bayesian inference often requires the
    calculation of integrals.
  \item Except for rather convenient models, these integrals can not
    be computed neither analytically, nor numerically.
  \end{itemize}
 
 \pause\bigskip
  \paragraph{Monte-Carlo.} They can still be estimated via 
  Monte-Carlo techniques based on the law of large numbers.

  \pause\bigskip
  \paragraph{Example.} The computation of $p(\thetabf|\Xbf)$ requires
  this of
  $$
  \int p(\thetabf) p(\Xbf|\thetabf) \;\dd \thetabf
  $$
  \pause that can be estimated by 
  $$
  \frac1B \sum_b p(\Xbf|\thetabf^b), 
  \qquad \text{where }
  \{\thetabf^1, \dots \thetabf^B\} \text{ i.i.d. } \sim p(\thetabf)
  $$
  (note that $p(\theta)$ is far from being the best choice).
  }

%--------------------------------------------------------------------
\frame{ \frametitle{Bayesian inference \& graphical models}

  \paragraph{Block design.}\\
  \begin{tabular}{cc}
    \hspace{-.5cm}
    \begin{tabular}{p{.5\textwidth}}
      \onslide+<1->{
        \paragraph{Marginal model} (observed variables only): 
        $$
        \thetabf = (\sigma_V, \sigma_B, \mu, \sigma_E)
        $$
        }

      \onslide+<3->{
        \paragraph{Complete model} (with unobserved variables):} 
      \begin{itemize}
      \item \onslide+<4->{$\thetabf_\Vbf = \sigma_V$,} 
      \item \onslide+<6->{$\thetabf_\Bbf = \sigma_B$,}
      \item \onslide+<8->{$\thetabf_{\Tbf|\Vbf, \Bbf} = (\mu,
          \sigma_E)$.}
      \end{itemize}
    \end{tabular}
    & 
    %\hspace{-.5cm}
    \begin{tabular}{p{.5\textwidth}}
      \begin{overprint}
        \onslide<1>
        \epsfig{file=../Figures/GM-FactDes-T-B-Theta1.eps, clip=,
          width=0.6\textwidth}
        \onslide<2>
        \epsfig{file=../Figures/GM-FactDes-T-B-Theta2.eps, clip=,
          width=0.6\textwidth}
        \onslide<3>
        \epsfig{file=../Figures/GM-FactDes-BVT-B.eps, clip=,
          width=0.6\textwidth}
        \onslide<4>
        \epsfig{file=../Figures/GM-FactDes-BVT-B-V1.eps, clip=,
          width=0.6\textwidth}
        \onslide<5>
        \epsfig{file=../Figures/GM-FactDes-BVT-B-V2.eps, clip=,
          width=0.6\textwidth}
        \onslide<6>
        \epsfig{file=../Figures/GM-FactDes-BVT-B-VB1.eps, clip=,
          width=0.6\textwidth}
        \onslide<7>
        \epsfig{file=../Figures/GM-FactDes-BVT-B-VB2.eps, clip=,
          width=0.6\textwidth}
        \onslide<8>
        \epsfig{file=../Figures/GM-FactDes-BVT-B-VBT1.eps, clip=,
          width=0.6\textwidth}
        \onslide<9>
        \epsfig{file=../Figures/GM-FactDes-BVT-B-VBT2.eps, clip=,
          width=0.6\textwidth}
      \end{overprint}
    \end{tabular}
  \end{tabular}
  }

%--------------------------------------------------------------------
\frame{ \frametitle{Posterior with unobserved variables}

  \paragraph{Posterior distributions.} Conditional distributions of
  interest, such as
  $$
  p(\thetabf|\Xbf), \qquad p(\Zbf|\Xbf), \qquad p(\Zbf,
  \thetabf|\Xbf)
  $$
  involve the double integral
  $$
  p(\Xbf) = \iint p(\thetabf) p(\Zbf|\thetabf) p(\Xbf|\Zbf,
  \thetabf) \;\dd \Zbf \;\dd \thetabf.
  $$
  
  \pause\bigskip 
  This often leads to intractable calculations and these distributions
  are either
  \begin{itemize}
  \item approximated, via e.g. variational Bayes approaches
  \item or sampled, via Monte-Carlo algorithms.
  \end{itemize}
  }

%--------------------------------------------------------------------
\frame{ \frametitle{Monte-Carlo, MCMC, Importance sampling and others}

  \paragraph{Gibbs sampling.} To simulate a joint distribution
  $p^*(\thetabf) = p^*(\theta_1, ... \theta_p)$, 
  $$
  \theta_i^{b+1} \sim p^*(\theta_1^{b+1}, \dots, \theta_{i-1}^{b+1},
  \theta_{i+1}^b, \dots, \theta_p^b).
  $$
  converge toward $p^*(\thetabf)$ for large $b$.

  \pause\bigskip\bigskip
  \paragraph{Monte-Carlo Markov chain (MCMC).} Design a Markov chain that
  converges toward the target distribution $p^*(\thetabf)$, e.g.
  Metropolis Hastings:
  \begin{eqnarray*}
  \text{Draw } \thetabf & \sim & q(\thetabf^b; \thetabf) \\
  \text{Accept } \thetabf^{b+1} & = & \thetabf \qquad \text{ with probability
  } \frac{p^*(\theta) q(\thetabf^b; \thetabf)}{p^*(\theta^b) q(\thetabf;
  \thetabf^b)}.
  \end{eqnarray*}

  \pause\bigskip
  \paragraph{Importance sampling.} Using alternative proposal:
  $\{\thetabf^1, \dots \thetabf^B\} \text{ i.i.d. } \sim q(\thetabf)$ 
  $$
  \int q(\thetabf) \frac{p(\thetabf)}{q(\thetabf)} p(\Xbf|\thetabf)
  \;\dd \thetabf
  \simeq
  \frac1B \sum_b \frac{p(\thetabf^b)}{q(\thetabf^b)} p(\Xbf|\thetabf^b).
  $$
  }

%--------------------------------------------------------------------
\frame{ \frametitle{Monte-Carlo, MCMC, Importance sampling and others}
  \begin{tabular}{cc}
    \hspace{-.5cm}
    \begin{tabular}{p{.5\textwidth}}
      \paragraph{Tuning stochastic algorithms.}
      \begin{itemize}
      \item Choosing a sufficiently large $B$ to guaranty convergence,
      \item Getting a sufficiently large acceptance rate,
      \item Defining a clever proposal, 
      \item etc.
      \end{itemize}
      \emphase{does matter...} 

      \bigskip\pause
      \paragraph{Consequences:}
      \begin{itemize}
      \item Computation time,
      \item Precision of the estimated distribution.
      \end{itemize}
    \end{tabular}
    & 
    \hspace{-.5cm}
    \begin{tabular}{p{.5\textwidth}}
      \vspace{-.5cm}
      \onslide+<2->
      \epsfig{file=../Figures/BetaBinom-n30-x4-a10-b5-SampPost-0.ps,
        clip=, height=0.5\textwidth, angle=270} \\
      \vspace{-1cm}
      \onslide+<3>
      \epsfig{file=../Figures/BetaBinom-n30-x4-a10-b5-SampPost-1.ps,
      clip=, height=0.5\textwidth, angle=270}
    \end{tabular}
  \end{tabular}
  }

%--------------------------------------------------------------------
\frame{ \frametitle{Likelihood-free methods: ABC}

  \paragraph{Approximate Bayesian Computing (ABC).} For complex
  models, it may be both impossible
  \begin{itemize}
  \item to calculate the likelihood
  \item or to get a sample of the posterior distribution.
  \end{itemize}

  \pause\bigskip
  \paragraph{Rational of ABC.} Get a sample of $\thetabf$ which
  provide simulated data that are 'similar' to the observed ones.

  \pause\bigskip
  \paragraph{3 key user-defined ingredients:.} 
  \begin{description}
  \item[Summary statistics:] $\Sbf = ($means, variances, frequencies,
    correlations, etc.$)$ that provide a relevant description of the
    data.
  \item[Distance] $d(\Sbf, \Sbf')$ to compare to vectors of summary
    statistics (e.g. Euclidian).
  \item[Threshold] $\varepsilon$ for the distance below which two
    datasets are said 'similar'.
  \end{description}
  }

%--------------------------------------------------------------------
\frame{ \frametitle{Likelihood-free methods: ABC}

  \begin{tabular}{p{.4\textwidth}p{.5\textwidth}}
    \paragraph{Summary statistics:} \\
    $
    \Sbf^* = S(\Xbf)
    $
    & 
    Compute the vector of summary statistics
    for the observed data. 
%  \end{tabular} 
  \\
%  \begin{tabular}{p{.4\textwidth}p{.5\textwidth}}
    \pause\paragraph{Sampling parameters:} \\
    $
    (\thetabf^1, \dots, \thetabf^B) \text{ i.i.d. } \sim q(\thetabf)
    $
    &
    Draw parameter values
    $\thetabf^b$ from a 'clever' proposal distribution. 
%  \end{tabular} 
  \\
%  \begin{tabular}{p{.4\textwidth}p{.5\textwidth}}
  \pause\paragraph{Simulated data:} \\
    $
    \Xbf^b \sim p(\Xbf|\thetabf^b)
    $
    & 
    From each $\thetabf^b$, simulate fake data
    according to the model.
%  \end{tabular} 
  \\  
%  \begin{tabular}{p{.4\textwidth}p{.5\textwidth}}
  \pause\paragraph{Simulated statistics:} \\
    $
    \Sbf^b = S(\Xbf^b)
    $
    & 
    For each fake dataset $\Xbf^b$, compute
    the simulated summary statistics:
%  \end{tabular} 
  \\
%  \begin{tabular}{p{.4\textwidth}p{.5\textwidth}}
  \pause\paragraph{Acceptance:} \\
    $
    d(\Sbf^b, \Sbf^*) < \varepsilon
    $
    & 
    Accept $\thetabf^b$ if $\Sbf^b$ is  close enough to $\Sbf^*$.
  \end{tabular}

  }

%--------------------------------------------------------------------
%--------------------------------------------------------------------
\section{Large dimension}
%--------------------------------------------------------------------
\frame{ \frametitle{Large dimension}
 }
%--------------------------------------------------------------------
\subsection{Genetical genomics}
%--------------------------------------------------------------------
%--------------------------------------------------------------------
\frame{ \frametitle{Genetical genomics}
  
  \paragraph{Aim.} Understand the relation between the variations of
  phenotypic traits and genomic information.
  
  \pause\bigskip
  \begin{description}
  \item[Genotype:] $10^4 - 10^6$ SNP
  \item[Trait:] Quantitative (height, growth rate, etc.) or qualitative
    (presence/absence) phenotype.
  \end{description}

  \pause\bigskip\bigskip
  \paragraph{Data.} For a set of individuals ($i = 1, \dots n$):
  \begin{itemize}
  \item Genotype at $p$ loci: $\Xbf_i = (X_{i1}, \dots, X_{ip})$,
  \item Trait $Y_i$
  \end{itemize}
  gathered into
  $$
  \underset{(n \times 1)}{\Ybf} = \left[\begin{array}{c}Y_1\\
      \vdots\\Y_n\end{array}\right],  
  \quad
    \underset{(n \times p)}{\Xbf} = \left[\begin{array}{ccc}
      X_{11} & \dots & X_{1p} \\ 
      \vdots &  \vdots \\
      X_{n1} & \dots & X_{np}
    \end{array}\right]
  \qquad \text{where}
  \quad p \gg n.
  $$
 }

%--------------------------------------------------------------------
\frame{ \frametitle{Genetical genomics}
  
  \paragraph{2 distinct questions} (and the statistical issues they raise)
  \begin{description}
  \item[Test:] Determine the loci associated with (responsible for?) the
    variations of the trait (GWAS). \\
    \ra Multiple testing, FDR, etc.
  \item[Prediction:] \pause Predict the variation of the trait based on the
    genotype (Genomic selection). \\
    \ra Precision of the prediction, regularization.
  \end{description}
  
  \pause\bigskip\bigskip
  And a third one
  \begin{description}
  \item[Variable selection:] Find the subset of locus that best predicts
    the trait. \\
    \ra Algorithmics, stability of the result.
  \end{description}
  }

%--------------------------------------------------------------------
\subsection{Regularization}
%--------------------------------------------------------------------
\frame{ \frametitle{Regression}

   \paragraph{Variability of the trait.} The value of the trait observed
   for individual $i$ can be decomposed as
   $$
   Y_i = \mu_i + E_i
   $$
   where
   \begin{itemize}
   \item $\mu_i$ stands for its 'theoretical' value; 
   \item $E_i$ is the random term accounting for its
     variability.  
   \end{itemize}

   \pause\bigskip
   \paragraph{Regression framework.} Our aim is to estimate $\mu_i$ as
   a (linear) function of the genotype:
   $$
   \widehat{\mu}_i = \widehat{\theta}_1 x_{i1} + \widehat{\theta}_1
   x_{i2} + ... + \widehat{\theta}_p x_{ip}
   \qquad \Leftrightarrow \qquad
   \widehat{\mubf} = \Xbf \widehat{\thetabf}
   $$

   \pause\bigskip
   \paragraph{Aim.} Find an estimator $\widehat{\thetabf}$ warranting
   \emphase{a small risk
   $$
   \Esp\left(\|\widehat{\mubf} - \mubf\|^2\right).
   $$}

   }

%--------------------------------------------------------------------
\frame{ \frametitle{Regression}

  \begin{tabular}{cc}
    \hspace{-.5cm}
    \begin{tabular}{p{.5\textwidth}}
      \vspace{-2cm}      
      \paragraph{Geometric view:}
      \begin{itemize}
      \item \onslide+<1->{Data: $\Ybf$.}
      \item \onslide+<2->{Truth: $\Ybf = \mubf + \Ebf$.}
      \item \onslide+<3->{Covariates: $\Xbf$.}
      \item \onslide+<4->{Fit: $\widehat{\mubf} =
          \Xbf\widehat{\thetabf}$.} 
      \end{itemize}

      \onslide+<5->{
        \bigskip
        \paragraph{Decomposition of the risk.}
        $$
        \Esp\left(\|\widehat{\mubf} - \mubf\|^2\right) = 
        [\Bias(\widehat{\mubf}]^2 + \Var(\widehat{\mubf})
        $$
        }
    \end{tabular}
    & 
    \hspace{-.5cm}
    \begin{tabular}{p{.5\textwidth}}
      \begin{overprint}
        \onslide<1>
        \epsfig{file=../Figures/Reg-Y.eps, clip=,
          width=0.7\textwidth}
        \onslide<2>
        \epsfig{file=../Figures/Reg-Ymu.eps, clip=,
          width=0.7\textwidth}
        \onslide<3>
        \epsfig{file=../Figures/Reg-YmuX.eps, clip=,
          width=0.7\textwidth}
        \onslide<4>
        \epsfig{file=../Figures/Reg-YmuX-fit.eps, clip=,
          width=0.7\textwidth}
        \onslide<5->
        \epsfig{file=../Figures/Reg-YmuX-BV.eps, clip=,
          width=0.7\textwidth}
      \end{overprint}
    \end{tabular}
  \end{tabular}
  
  \onslide+<6>{
    \vspace{-1cm}
    \paragraph{Bias-Variance compromise.} When $p$ increases,
    $$
    \Bias(\widehat{\mubf}) \text{ decreases whereas }
    \Var(\widehat{\mubf}) \text{ increases.}
    $$
    }
  }

%--------------------------------------------------------------------
\frame{ \frametitle{Ridge regression}

  \paragraph{Least square estimate.} $\widehat{\thetabf}$ defined as
  $$
  \widehat{\thetabf} = \arg\min_{\thetabf} \|\Ybf - \Xbf \thetabf\|^2
        = \arg\min_{\theta_1, ...\theta_p} \sum_i (Y_i - x_{i1}
        \theta_1 - ... - x_{ip} \theta_p)^2.
  $$

  \pause
  \paragraph{Ridge regularization.} Prevent $\widehat{\thetabf}$ from
  having large coordinates via
  $$
  \widehat{\thetabf}_{\text{Ridge}} = \arg\min_{\thetabf} \|\Ybf - \Xbf
  \thetabf\|^2 + \lambda \|\thetabf\|^2
  $$
  where $\|\thetabf\|^2 = \sum_j \theta_j^2$.

  \pause\bigskip\bigskip
  \paragraph{Several interpretations.}
  \begin{itemize}
  \item Variance reduction: $\Var(\widehat{\thetabf}_{\text{Ridge}}) <
    \Var(\widehat{\thetabf})$. 
  \item Tykhonov regularization: makes $(\Xbf'\Xbf)$ invertible when
    $p > n$.
  \item Gaussian prior for $\thetabf$ in a Bayesian context.
  \end{itemize}

  }

%--------------------------------------------------------------------
\frame{ \frametitle{Ridge regression}

  \begin{tabular}{cc}
    \hspace{-.5cm}
    \begin{tabular}{p{.5\textwidth}}
      \vspace{-2cm}      
      \paragraph{Geometric view:}
      \begin{itemize}
      \item \onslide+<1->{Least square estimate: $\widehat{\thetabf}$.}
      \item \onslide+<2->{Ball with radius $1/\lambda$.}
      \item \onslide+<3->{$\widehat{\thetabf}_{\text{Ridge}} =$
          projection of $\widehat{\thetabf}$ onto the ball.}
      \end{itemize}

    \end{tabular}
    & 
    \hspace{-.5cm}
    \begin{tabular}{p{.5\textwidth}}
      \begin{overprint}
        \onslide<1>
        \epsfig{file=../Figures/Reg-Theta.eps, clip=,
          width=0.7\textwidth}
        \onslide<2>
        \epsfig{file=../Figures/Reg-Ridge1.eps, clip=,
          width=0.7\textwidth}
        \onslide<3->
        \epsfig{file=../Figures/Reg-Ridge2.eps, clip=,
          width=0.7\textwidth}
      \end{overprint}
    \end{tabular}
  \end{tabular}

  \onslide+<4->{
    \vspace{-2cm}
    \paragraph{Shrinkage.} The projection shrinks the regression
    coefficients towards 0:   
    $$
    \|\widehat{\thetabf}_{\text{Ridge}}\| < \|\widehat{\thetabf}\|.
    $$
    }
  }

%--------------------------------------------------------------------
\frame{ \frametitle{LASSO regression}

  \begin{tabular}{cc}
    \hspace{-.5cm}
    \begin{tabular}{p{.5\textwidth}}
      \vspace{-2cm}
      \onslide+<1->{
        \paragraph{Ridge penalization} relies on the $\ell_2$ norm
        $$
        \|\thetabf\| = \sqrt{\sum_j \theta_j^2}
        $$
        for which 'balls' are round. 
        }

      \onslide+<2->{
        \bigskip
        \paragraph{LASSO penalization} relies on the $\ell_1$ norm
        $$
        |\thetabf| = \sum_j |\theta_j|
        $$
        for which 'balls' are diamond shaped. 
        }

    \end{tabular}
    & 
    \hspace{-.5cm}
    \begin{tabular}{p{.5\textwidth}}
      \begin{overprint}
        \onslide<1>
        \epsfig{file=../Figures/Reg-Ridge1.eps, clip=,
          width=0.7\textwidth}
        \onslide<2>
        \epsfig{file=../Figures/Reg-Lasso1.eps, clip=,
          width=0.7\textwidth}
        \onslide<3->
        \epsfig{file=../Figures/Reg-Lasso2.eps, clip=,
          width=0.7\textwidth}
      \end{overprint}
    \end{tabular}
  \end{tabular}

  \onslide+<3->{
    \vspace{-1cm}
    \paragraph{Projection.} The projection on such a ball sets some
    coefficients to 0. \\
    \ra Lasso achieves \emphase{variables selection}.
    }
  }

%--------------------------------------------------------------------
\frame{ \frametitle{LARS heuristic}

  \begin{tabular}{cc}
    \hspace{-.5cm}
    \begin{tabular}{p{.5\textwidth}}
      \vspace{-5cm}
      \onslide+<1->{
        \paragraph{Lasso regression.}
        $$
        \widehat{\thetabf}_{\text{Lasso}} 
        = \arg\min_{\thetabf} \| \Ybf - \Xbf \thetabf\|^2 + \lambda |\thetabf|.
        $$
        }

      \onslide+<2->{
        \paragraph{The number of non-0 coefficients
          $\widehat{\theta}_j$} increases as the penalty $\lambda$ decreases. 
        }
    \end{tabular}
    & 
    \hspace{-.5cm}
    \begin{tabular}{p{.5\textwidth}}
      \begin{overprint}
        \onslide<1>
        \epsfig{file=../Figures/Reg-Lasso2.eps, clip=,
          width=0.7\textwidth}
        \onslide<2>
        \epsfig{file=../Figures/Reg-Lasso3.eps, clip=,
          width=0.7\textwidth}
        \onslide<3>
        \epsfig{file=../Figures/Fig-LARS-coef.ps, clip=,
          height=0.5\textwidth, width=0.5\textwidth, angle=270}
      \end{overprint}
    \end{tabular}
  \end{tabular}

  \onslide+<3->{
    \vspace{-6cm}
    \paragraph{LARS algorithm.} All the estimates
    $\widehat{\thetabf}_{\text{Lasso}}(\lambda)$ (i.e. the 'solution
    path') can be computed in a linear time ($\Ocal(p)$) for all
    possible value of $\lambda$.  }  }

%--------------------------------------------------------------------
\frame{ \frametitle{How to choose $\lambda$?}

  \begin{tabular}{cc}
    \hspace{-.5cm}
    \begin{tabular}{p{.5\textwidth}}
      \onslide+<1->{
        \vspace{-1cm}
        \paragraph{$V$-fold cross-validation.} Split the data set into
        $V$ equal subsets. For a given $\lambda$ and for each subset
        $v = 1 ... V$:  
        \begin{enumerate}
        \item Compute $\widehat{\thetabf}^{-v}(\lambda)$ without $v$;
        \item For each element $i$ of $v$, predict
          $\widehat{\mu}_i^{-v}(\lambda) = \xbf_i
          \widehat{\thetabf}^{-v}(\lambda)$ ; 
        \item Compare $Y_i$ with $\widehat{\mu}_i^{-v}(\lambda)$.
        \end{enumerate}
        }
    \end{tabular}      
    & 
    \hspace{-.5cm}
    \begin{tabular}{p{.5\textwidth}}
      \onslide+<3->{
        \epsfig{file=../Figures/Fig-LARS-cv.ps, clip=,
          height=0.5\textwidth, width=0.5\textwidth, angle=270}
        }
    \end{tabular}
  \end{tabular}
  
  \onslide+<2->{
    \vspace{-0.5cm}
    Choose the $\lambda$ with best $V$-fold risk:
    $$
    R_V(\lambda) = \frac1n \sum_v \sum_{i \in v} (Y_i -
    \widehat{\mu}_i^{-v}(\lambda))^2.
    $$
    }
  }


%--------------------------------------------------------------------
\frame{ \frametitle{Playing with penalties}

  Each penalty defines a specific set of constraints on the regression
  parameters.

%   \begin{tabular}{p{.2\textwidth}p{.35\textwidth}p{.35\textwidth}}
%     \paragraph{Ridge} & $\lambda \|\thetabf\|$ & gives similar
%   regression coefficients to correlated covariates. \\ 
%     \paragraph{Lasso} & $\lambda_1 |\thetabf|$ & sets some
%     coefficients to 0. \\ 
%     \paragraph{Elastic net} & $\lambda_1 |\thetabf| + \lambda_2
%     \|\thetabf\|$ & does both. \\
%     \paragraph{Fuzzed Lasso} & $\lambda_1 |\thetabf| + \lambda_2
%     \sum_j |\theta_j - \theta_{j-1}|$ & sets some coefficients and
%     differences between neighbor coefficients to 0. \\
%     \paragraph{Group Lasso} & $\sum_g \lambda_g \|\thetabf^g\|$ & sets
%     the coefficients of some predefined groups of covariates to 0. \\
%   \end{tabular}

  \pause\medskip
  \paragraph{Ridge} gives similar regression coefficients to
  correlated covariates: \emphase{$\lambda \|\thetabf\|$}.
  
  \pause\medskip
  \paragraph{Lasso} sets some coefficients to 0: \emphase{$\lambda
  |\thetabf|$}.
  
  \pause\medskip
  \paragraph{Elastic net} does both: \emphase{$\lambda_1 |\thetabf| +
  \lambda_2 \|\thetabf\|$}.
  
  \pause\medskip
  \paragraph{Fuzzed Lasso} sets some coefficients and
  differences between neighbor coefficients to 0: \emphase{$\lambda_1
  |\thetabf| + \lambda_2 \sum_j |\theta_j - \theta_{j-1}|$}.
  
  \pause\medskip
  \paragraph{Group Lasso} sets the coefficients of some
  predefined groups of covariates to 0: \emphase{$\sum_g \lambda_g
    \|\thetabf^g\|$}.
  
  \pause\bigskip   
  An efficient optimization algorithms (or at leats a good heuristic)
  exists for each of them.

  }

%--------------------------------------------------------------------
\section{To conclude}
%--------------------------------------------------------------------
\frame{ \frametitle{To conclude (?)}

  \paragraph{Graphical models} constitute a convenient (and
  theoretically grounded) way to define models with complex dependency
  structure. 

  \pause\bigskip\bigskip
  \paragraph{The inference of complex models} involving latent
  variables can often not be achieved is a standard way but
  approximates techniques exist.

  \pause\bigskip\bigskip
  \paragraph{Regression models} involving a large number of
  covariates can be handled using penalization combined with
  efficient optimization algorithm.
  }

%====================================================================
\frame{ \frametitle{}
  \tiny{
    \nocite{WaJ08,Jaa00,MaP00,Lau96,BeG03,BZB02,MaR07}
    \bibliography{/Biblio/AST,/Biblio/ARC}
    \bibliographystyle{/Latex/astats}
    }
  }

%--------------------------------------------------------------------
%--------------------------------------------------------------------
\end{document}
%--------------------------------------------------------------------
%--------------------------------------------------------------------

  \begin{tabular}{cc}
    \hspace{-.5cm}
    \begin{tabular}{p{.5\textwidth}}
    \end{tabular}
    & 
    \hspace{-.5cm}
    \begin{tabular}{p{.5\textwidth}}
      \begin{overprint}
      \end{overprint}
    \end{tabular}
  \end{tabular}



