%------------------------------------------------------------------------
\begin{frame}
  \frametitle{Mod�le � nombre fini d'all�les}
  \begin{itemize}
  \item Les mutations sont vues comme des transitions entre $A$
    all�les possibles.
  \item Elles surviennent selon un processus Markovien homog�ne �
    temps continu $\{X_t\}_{t \geq 0}$.
  \end{itemize}
  \pause
  On note $r_{ab}$ le taux de transition de l'all�le $a$ vers l'all�le
  $b$ et $R = [r_{ab}]$ le g�n�rateur infinit�simal du
    processus. %($r_{aa} = -\sum_{b \neq a} r_{ab}$).
  \begin{itemize}
  \item La matrice de transition en une unit� de temps est $\exp(R)$.
  \item Plus g�n�ralement~: 
    $$
    \pi_{ab}(s) := P(X_{t+s} = b | X_t = a) = [exp(s R)]_{ab}.
    $$
  \end{itemize}
  
  $\pi_{66}(t_1)$ ne signifie pas qu'il n'y a pas eu de mutation entre
  temps.
\end{frame}

%------------------------------------------------------------------------
\begin{frame}
  \frametitle{Probabilit� � noeuds internes connus}

  La possibilit� de retour en arri�re ne permet pas de conna�tre les
  p�riodes de mutation.
  \begin{tabular}{ll}
    \begin{tabular}{p{.5\textwidth}}
      $P(\Dcal|\Tcal, \theta) = $
      \begin{eqnarray*}
        %
        &  & \pi_{53}(T) \\
        & \times & \pi_{56}(T-t_3) \\
        & \times & \pi_{66}(t_3-t_2) \pi_{66}(t_2-t_1) \pi_{66}(t_1)  \\  
        & \times & \pi_{66}(t_1)  \\ 
        & \times & \pi_{66}(t_2)  \\ 
        & \times & \pi_{67}(t_3)
      \end{eqnarray*}
    \end{tabular}
    &
    \begin{tabular}{p{.5\textwidth}}
      \includegraphics[width=.4\textwidth,
      height=.6\textheight,clip=]{transition2-left}  
    \end{tabular}
  \end{tabular}
\end{frame}

%------------------------------------------------------------------------
\begin{frame}
  \frametitle{Probabilit� � noeuds internes inconnus}

  \begin{tabular}{ll}
    \begin{tabular}{p{.5\textwidth}}
      $P(\Dcal|\Tcal, \theta) = $
      \begin{eqnarray*}
        &  & \sum_{a_4} \pi_{a_43}(T) \\
        & \times & \Big( \sum_{a_3} \pi_{a_4a_3}(T-t_3)
        \pi_{a_37}(t_3) \\
        & \times & \Big( \sum_{a_2} \pi_{a_3a_2}(t_3-t_2)
        \pi_{a_26}(t_2) \\
        & \times & \Big( \sum_{a_1} \pi_{a_2a_1}(t_2-t_1)  \\
        & & \qquad 
        \pi_{a_16}(t_1) \pi_{a_16}(t_1) \Big)\Big)\Big)
      \end{eqnarray*}
    \end{tabular}
    &
    \begin{tabular}{p{.5\textwidth}}
      \includegraphics[width=.4\textwidth,
      height=.6\textheight,clip=]{transition1-left}  
    \end{tabular}
  \end{tabular}
\end{frame}


%%% Local Variables: 
%%% mode: latex
%%% TeX-master: "presgen"
%%% End: 
