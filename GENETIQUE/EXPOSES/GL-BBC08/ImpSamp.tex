\section{Echantillonnage pr�f�rentiel}

%------------------------------------------------------------------------
\begin{frame}
  \frametitle{Echantillonnage pr�f�rentiel : Principe}
  On cherche � calculer (�valuer) une fonction d'int�r�t
  $$
  L(\theta) = \int g(\Tcal, \theta) P(\Tcal |\theta) \dd \Tcal
  \qquad 
  (\text{e.g. } g(\Tcal, \theta) = P(\Dcal | \Tcal, \theta))
  $$ 
  sans pouvoir calculer cette int�grale.

  \pause \bigskip
  {\bf Astuce~:} Pour toute distribution $Q$ t.q. 
  $$
  g(\Tcal, \theta) P(\Tcal |\theta) > 0 \qquad \Rightarrow \qquad
  Q(\Tcal) > 0,
  $$
  on a 
  $$
  L(\theta) = \int g(\Tcal, \theta) \frac{P(\Tcal |\theta)}{Q(\Tcal)}
  Q(\Tcal) \dd \Tcal.
  $$ 

\end{frame}

%------------------------------------------------------------------------
\begin{frame}
  \frametitle{Estimation par Monte-Carlo}
  
  Pour $\{\Tcal^{(1)}, \dots, \Tcal^{(M)}\}$ i.i.d. $\sim Q$, 
  $$
  \widehat{L}(\theta) = \frac1M \sum_{i=1}^M g(\Tcal^{(i)}, \theta)
    \frac{P(\Tcal^{(i)} |\theta)}{Q(\Tcal^{(i)})}
  $$
  \pause
  estime $L(\theta)$ sans bias et 
  $$
  \Var\left[\widehat{L}(\theta) \right] = \frac1M
  \Var_Q\left[g(\Tcal^{(i)}, \theta)
    \frac{P(\Tcal^{(i)} |\theta)}{Q(\Tcal^{(i)})} \right].
  $$
  \pause
  $Q$ est d'autant meilleure que $\Var\left[\widehat{L}(\theta)
  \right]$ est petite, i.e. que
  $$
  S(Q) = \Esp_Q\left[g^2(\Tcal^{(i)}, \theta) \frac{P^2(\Tcal^{(i)}
      |\theta)}{Q^2(\Tcal^{(i)})} \right] \qquad \text{est
    petite.}
  $$
\end{frame}

%------------------------------------------------------------------------
\begin{frame}
  \frametitle{Meilleure distribution $Q$}

  La distribution $Q^*$ est optimale si, pour toute 'contamination'
  $F$
  $$
  \left. \frac{\partial}{\partial h} \left[S(Q+h F) - \lambda \int h F(\Tcal)
      \dd \Tcal \right] \right|_{h=0} = 0
  $$
  \pause 
  Si on peut d�river sous l'int�grale (la somme), cette d�riv�e vaut
  $$
  \int \left[g^2(\Tcal, \theta) \frac{P^2(\Tcal |\theta)}{{Q^*}^2(\Tcal)}
  - \lambda \right]  F(\Tcal) \dd \Tcal
  $$
  et est nulle pour toute $F$, si
  $$
  Q^*(\Tcal) \propto g(\Tcal, \theta) P(\Tcal |\theta).
  $$
  \pause
  Pour $g(\Tcal, \theta) = P(\Dcal | \Tcal, \theta)$, on obtient $
  Q^*(\Tcal) \propto P(\Dcal | \Tcal, \theta) P(\Tcal |\theta)$
  $$
  \Rightarrow \qquad   
  Q^*(\Tcal) = \frac{P(\Dcal | \Tcal, \theta) P(\Tcal
    |\theta)}{P(\Dcal | \theta)} = P(\Tcal | \Dcal, \theta)...
  $$
\end{frame}

%------------------------------------------------------------------------
\begin{frame}
  \frametitle{Surface de vraisemblance}

  La distribution $Q^*$ pour l'estimation $\widehat{L}(\theta)$
    d�pend en $\theta: Q^* = Q^*_\theta$. \\
  \begin{itemize}
  \item \pause {\bf Point par point~:} estimer $L^*(\theta)$ avec $Q^*_\theta$
    requiert un �chantillonnage de $M$ valeurs pour chaque $\theta$ \\
    $\rightarrow$ trop long.
  \item \pause {\bf Valeur pilote $\theta_0$~:} on estime $L(\theta)$
    en utilisant $Q^*_{\theta_0}$ \\
    $\rightarrow$ marche pour $\theta$ proche de $\theta_0$.
  \item \pause {\bf Grille $(\theta_1, \dots \theta_G$)~:} on d�finit
    une distribution moyenne
    $$
    Q(\Tcal) = \sum_g Q^*_{\theta_g}(\Tcal).
    $$
  \end{itemize}
\end{frame}

%------------------------------------------------------------------------
\begin{frame}
  \frametitle{Inf�rence ancestral}
  
  On peut estimer l'esp�rance a posteriori de l'�ge du plus r�cent
  anc�tre commun $T_{MRCA}$ utilisant les arbres $\Tcal^{(i)}$ obtenus
  lors de l'�chantillonnage pr�f�rentiel.

  \pause 
  On d�finit les poids
  $$
  w^{(i)} = \frac{W^{(i)}}{\sum_j W^{(j)}}, 
  \qquad
  W^{(i)} = \frac{P(\Dcal |\Tcal^{(i)}, \theta) P(\Tcal^{(i)} |
    \theta)}{Q(\Tcal^{(i)})}.  
  $$
  \pause
  et on utilise
  \begin{eqnarray*}
    \sum_i w^{(i)} T_{MRCA}(\Tcal^{(i)}) & \approx & 
    \frac1{P(\Dcal|\theta)} \int P(\Dcal, \Tcal | \theta)
    T_{MRCA}(\Tcal) \dd \Tcal \\
    & = & \Esp(T_{MRCA} | \Dcal, \theta). 
  \end{eqnarray*}
\end{frame}

%------------------------------------------------------------------------
\begin{frame}
  \frametitle{}
\end{frame}

%%% Local Variables: 
%%% mode: latex
%%% TeX-master: "presgen"
%%% End: 
