% %====================================================================
% \frame{\frametitle{Que stocker ?}
% 
%   \paragraph{Sp�cificit� des donn�es mol�culaires.}
%   \begin{itemize}
%    \item Mesures g�n�ralement indirectes
%    \item Dont la nature varie (fr�quemment) du fait des �volutions technologiques
%    \item Effort permanent de normalisation / correction ...
%   \end{itemize}
% 
%   \bigskip \bigskip \pause
%   \paragraph{Que stocker ?}
%   \begin{itemize}
%    \item Donn�es brutes (d�finition de 'donn�e brute') + Logiciels de normalisation
%    \item Donn�es corrig�e + Version du logiciel
%    \item Refaire la manip
%   \end{itemize}
%   
% }

%====================================================================
\frame{\frametitle{Breiman, 2001: 'Statistical Modeling: The Two Cultures'}

  \paragraph{Abstract.} ~\\~\\
  There are two cultures in the use of statistical modeling to reach conclusions from data. One assumes that the data are generated by a given stochastic data model. The other uses algorithmic models and treats the data mechanism as unknown. \\ ~
  
  The statistical community has been committed to the almost exclusive use of data models. 
  This commitment has led to irrelevant theory, questionable conclusions, and has kept statisticians from working on a large range of interesting current problems. \\ ~
  
  Algorithmic modeling, both in theory and practice, has developed rapidly in fields outside statistics. It can be used both on large complex data sets and as a more accurate and informative alternative to data modeling on smaller data sets.  \\ ~
  
  If our goal as a field is to use data to solve problems, then we need to move away from exclusive dependence on data models and adopt a more diverse set of tools  
}

%====================================================================
\frame{\frametitle{Mod�le graphique, Identifiabilit�, Causalit�}

  \begin{tabular}{c|c|c}
    'V�rit�' & Mod�les non distinguables & 'Causalit�' \\  
    & \footnotesize{� partir de donn�es observationnelles, non {\sl interventionnelles}} \\  
    & & \\\hline & & \\
    \includegraphics[trim={0 15 180 0}, height=.22\textheight, clip]{Figs/Pea09-Fig2-3}
    &
    \includegraphics[trim={0 15 0 0}, height=.21\textheight, clip]{Figs/Pea09-Fig2-4}
    &
    \includegraphics[trim={180 15 0 0}, height=.22\textheight, clip]{Figs/Pea09-Fig2-3}
  \end{tabular}

  \bigskip \bigskip 
  $$
  a \leftrightarrow b \qquad = \qquad a \leftarrow u \rightarrow b
  $$
  $u$ non observ�e

}

%====================================================================
\frame{\frametitle{R�seaux g�n�ratifs adverses}

  \begin{tabular}{cc}
    \begin{tabular}{c}
      \includegraphics[width=.4\textwidth]{/home/robin/RECHERCHE/APPRENTISSAGE/EXPOSES/Figures/GANs-HumanFaces} 
    \end{tabular}
    \pause &
    \begin{tabular}{c}
      \includegraphics[width=.5\textwidth]{/home/robin/RECHERCHE/APPRENTISSAGE/EXPOSES/Figures/GANs-Principle} 
    \end{tabular}
  \end{tabular}

  \bigskip \bigskip \pause
  \paragraph{{\sl Imitation game}:}
  \begin{itemize}
   \item Un algorithme est entra�n� � distinguer au mieux les images authentiques d'images simul�es
   \item Un autre algorithme est entra�n� � simuler des images trompant le plus possible le premier
  \end{itemize}



}

