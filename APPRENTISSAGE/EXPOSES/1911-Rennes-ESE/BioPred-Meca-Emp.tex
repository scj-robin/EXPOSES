%====================================================================
%====================================================================
\subsection{Mod�les empiriques vs m�canistes}
\frame{\frametitle{Outline} \tableofcontents[currentsubsection]}
%====================================================================
\frame{\frametitle{Mod�les empiriques / m�canistes}

  \paragraph{Breiman (2001) :} '{\sl Statistical Modeling: The Two Cultures}'
  \begin{itemize}
   \item m�caniste : {\sl assumes that the data are generated by a given stochastic \textbf{data model}}
   \item empirique : {\sl uses \textbf{algorithmic models} and treats the data mechanism as unknown}
  \end{itemize}
  
  \bigskip \bigskip \pause
  \paragraph{Parti pris de l'apprentissage:}
  \begin{itemize}
   \item inutile de comprendre pour pr�dire
   \item seulement comprendre pour agir (au mieux) 
  \end{itemize}
  \ra Pr�diction et optimisation \\
  \ra Pas d'analyse des m�canismes
  
  \bigskip \bigskip \pause
  \paragraph{En pratique :} approches hydrides

}

%====================================================================
%====================================================================
\subsection{Approches hybrides}
\frame{\frametitle{Outline} \tableofcontents[currentsubsection]}
%====================================================================
\frame{\frametitle{Approche hybride : s�lection g�nomique}

  \paragraph{Pr�diction d'un ph�notype} � partir du g�notype :  \\ ~ \\
  \begin{itemize}
  \item Mod�le empirique de pr�diction fond� sur une population contemporaine \\ ~
  \item Mod�le g�n�tique d'�volution de population (soumise � s�lection) \\ ~
  \item Estimation de la qualit� des pr�dictions dans les g�n�rations futures
  \end{itemize}

}


%====================================================================
\frame{\frametitle{Approche hybride : r�seaux �cologiques}

  \begin{tabular}{ll}
    \begin{tabular}{p{.6\textwidth}}
    \paragraph{Inf�rence de r�seaux �cologiques.} ~\\ 
    \begin{itemize}
     \item Mod�le de Lotka-Volterra stochastique (d) ou mod�le graphique probabiliste (b) \\~
     \item Inf�rence par maximum de vraisemblance \\~
     \item P�nalisation lasso pour r�duire le nombre d'ar�tes
    \end{itemize}
    \end{tabular}
    &
    \hspace{-0.05\textwidth}
    \begin{tabular}{p{.4\textwidth}}
   \includegraphics[trim={360 0 160 200}, width=.25\textwidth, clip]{Figs/StaticDynamicNetwork} 
   \\ ~ \\
   \includegraphics[trim={360 205 160 0}, width=.25\textwidth, clip]{Figs/StaticDynamicNetwork}
    \end{tabular}
  \end{tabular}

}

%====================================================================
\frame{\frametitle{Approche hybride : g�n�tique des populations}

  \begin{tabular}{ll}
    \begin{tabular}{p{.45\textwidth}}
    \paragraph{Comparaison de sc�narios �volutifs.} ~\\ 
    \begin{itemize}
     \item Mod�le probabiliste d'�volution \\~
     \item Inf�rence bay�sienne (ABC) \\~
     \item For�ts al�atoires pour estimer la probabilit� {\it a posteriori} de chaque sc�nario
    \end{itemize}
    \end{tabular}
    &
    \hspace{-0.05\textwidth}
    \begin{tabular}{p{.3\textwidth}}
    \includegraphics[width=.45\textwidth]{Figs/Rob17-StatLearn-Fig1} \\
    \\
    \includegraphics[width=.45\textwidth]{Figs/Rob17-StatLearn-Fig2}
    \end{tabular}
  \end{tabular}

}


