\documentclass[8pt]{beamer}

% Beamer style
%\usetheme[secheader]{Madrid}
% \usetheme{CambridgeUS}
\useoutertheme{infolines}
\usecolortheme[rgb={0.65,0.15,0.25}]{structure}
% \usefonttheme[onlymath]{serif}
\beamertemplatenavigationsymbolsempty
%\AtBeginSubsection

% Packages
\usepackage[french]{babel}
\usepackage[latin1]{inputenc}
\usepackage{color}
\usepackage{xspace}
\usepackage{dsfont, stmaryrd}
\usepackage{amsmath, amsfonts, amssymb, stmaryrd}
\usepackage{epsfig}
\usepackage{tikz}
\usepackage{url}
% \usepackage{ulem}
\usepackage{/home/robin/LATEX/Biblio/astats}
%\usepackage[all]{xy}
\usepackage{graphicx}
\usepackage{xspace}

% Maths
% \newtheorem{theorem}{Theorem}
% \newtheorem{definition}{Definition}
\newtheorem{proposition}{Proposition}
% \newtheorem{assumption}{Assumption}
% \newtheorem{algorithm}{Algorithm}
% \newtheorem{lemma}{Lemma}
% \newtheorem{remark}{Remark}
% \newtheorem{exercise}{Exercise}
% \newcommand{\propname}{Prop.}
% \newcommand{\proof}{\noindent{\sl Proof:}\quad}
% \newcommand{\eproof}{$\blacksquare$}

% \setcounter{secnumdepth}{3}
% \setcounter{tocdepth}{3}
\newcommand{\pref}[1]{\ref{#1} p.\pageref{#1}}
\newcommand{\qref}[1]{\eqref{#1} p.\pageref{#1}}

% Colors : http://latexcolor.com/
\definecolor{darkred}{rgb}{0.65,0.15,0.25}
\definecolor{darkgreen}{rgb}{0,0.4,0}
\definecolor{darkred}{rgb}{0.65,0.15,0.25}
\definecolor{amethyst}{rgb}{0.6, 0.4, 0.8}
\definecolor{asparagus}{rgb}{0.53, 0.66, 0.42}
\definecolor{applegreen}{rgb}{0.55, 0.71, 0.0}
\definecolor{awesome}{rgb}{1.0, 0.13, 0.32}
\definecolor{blue-green}{rgb}{0.0, 0.87, 0.87}
\definecolor{red-ggplot}{rgb}{0.52, 0.25, 0.23}
\definecolor{green-ggplot}{rgb}{0.42, 0.58, 0.00}
\definecolor{purple-ggplot}{rgb}{0.34, 0.21, 0.44}
\definecolor{blue-ggplot}{rgb}{0.00, 0.49, 0.51}

% Commands
\newcommand{\backupbegin}{
   \newcounter{finalframe}
   \setcounter{finalframe}{\value{framenumber}}
}
\newcommand{\backupend}{
   \setcounter{framenumber}{\value{finalframe}}
}
\newcommand{\emphase}[1]{\textcolor{darkred}{#1}}
\newcommand{\comment}[1]{\textcolor{gray}{#1}}
\newcommand{\paragraph}[1]{\textcolor{darkred}{#1}}
\newcommand{\refer}[1]{{\small{\textcolor{gray}{{[\cite{#1}]}}}}}
\newcommand{\Refer}[1]{{\small{\textcolor{gray}{{[#1]}}}}}
\newcommand{\goto}[1]{{\small{\textcolor{blue}{[\#\ref{#1}]}}}}
\renewcommand{\newblock}{}

\newcommand{\tabequation}[1]{{\medskip \centerline{#1} \medskip}}
% \renewcommand{\binom}[2]{{\left(\begin{array}{c} #1 \\ #2 \end{array}\right)}}

% Variables 
\newcommand{\Abf}{{\bf A}}
\newcommand{\Beta}{\text{B}}
\newcommand{\Bcal}{\mathcal{B}}
\newcommand{\Bias}{\xspace\mathbb B}
\newcommand{\Cor}{{\mathbb C}\text{or}}
\newcommand{\Cov}{{\mathbb C}\text{ov}}
\newcommand{\cl}{\text{\it c}\ell}
\newcommand{\Ccal}{\mathcal{C}}
\newcommand{\cst}{\text{cst}}
\newcommand{\Dcal}{\mathcal{D}}
\newcommand{\Ecal}{\mathcal{E}}
\newcommand{\Esp}{\xspace\mathbb E}
\newcommand{\Espt}{\widetilde{\Esp}}
\newcommand{\Covt}{\widetilde{\Cov}}
\newcommand{\Ibb}{\mathbb I}
\newcommand{\Fcal}{\mathcal{F}}
\newcommand{\Gcal}{\mathcal{G}}
\newcommand{\Gam}{\mathcal{G}\text{am}}
\newcommand{\Hcal}{\mathcal{H}}
\newcommand{\Jcal}{\mathcal{J}}
\newcommand{\Lcal}{\mathcal{L}}
\newcommand{\Mt}{\widetilde{M}}
\newcommand{\mt}{\widetilde{m}}
\newcommand{\Nbb}{\mathbb{N}}
\newcommand{\Mcal}{\mathcal{M}}
\newcommand{\Ncal}{\mathcal{N}}
\newcommand{\Ocal}{\mathcal{O}}
\newcommand{\pt}{\widetilde{p}}
\newcommand{\Pt}{\widetilde{P}}
\newcommand{\Pbb}{\mathbb{P}}
\newcommand{\Pcal}{\mathcal{P}}
\newcommand{\Qcal}{\mathcal{Q}}
\newcommand{\qt}{\widetilde{q}}
\newcommand{\Rbb}{\mathbb{R}}
\newcommand{\Sbb}{\mathbb{S}}
\newcommand{\Scal}{\mathcal{S}}
\newcommand{\st}{\widetilde{s}}
\newcommand{\St}{\widetilde{S}}
\newcommand{\Tcal}{\mathcal{T}}
\newcommand{\todo}{\textcolor{red}{TO DO}}
\newcommand{\Ucal}{\mathcal{U}}
\newcommand{\Un}{\math{1}}
\newcommand{\Vcal}{\mathcal{V}}
\newcommand{\Var}{\mathbb V}
\newcommand{\Vart}{\widetilde{\Var}}
\newcommand{\Zcal}{\mathcal{Z}}

% Symboles & notations
\newcommand\independent{\protect\mathpalette{\protect\independenT}{\perp}}\def\independenT#1#2{\mathrel{\rlap{$#1#2$}\mkern2mu{#1#2}}} 
\renewcommand{\d}{\text{\xspace d}}
\newcommand{\gv}{\mid}
\newcommand{\ggv}{\, \| \, }
% \newcommand{\diag}{\text{diag}}
\newcommand{\card}[1]{\text{card}\left(#1\right)}
\newcommand{\trace}[1]{\text{tr}\left(#1\right)}
\newcommand{\matr}[1]{\boldsymbol{#1}}
\newcommand{\matrbf}[1]{\mathbf{#1}}
\newcommand{\vect}[1]{\matr{#1}} %% un peu inutile
\newcommand{\vectbf}[1]{\matrbf{#1}} %% un peu inutile
\newcommand{\trans}{\intercal}
\newcommand{\transpose}[1]{\matr{#1}^\trans}
\newcommand{\crossprod}[2]{\transpose{#1} \matr{#2}}
\newcommand{\tcrossprod}[2]{\matr{#1} \transpose{#2}}
\newcommand{\matprod}[2]{\matr{#1} \matr{#2}}
\DeclareMathOperator*{\argmin}{arg\,min}
\DeclareMathOperator*{\argmax}{arg\,max}
\DeclareMathOperator{\sign}{sign}
\DeclareMathOperator{\tr}{tr}
\newcommand{\ra}{\emphase{$\rightarrow$} \xspace}

% Hadamard, Kronecker and vec operators
\DeclareMathOperator{\Diag}{Diag} % matrix diagonal
\DeclareMathOperator{\diag}{diag} % vector diagonal
\DeclareMathOperator{\mtov}{vec} % matrix to vector
\newcommand{\kro}{\otimes} % Kronecker product
\newcommand{\had}{\odot}   % Hadamard product

% TikZ
\newcommand{\nodesize}{2em}
\newcommand{\edgeunit}{2.5*\nodesize}
\newcommand{\edgewidth}{1pt}
\tikzstyle{node}=[draw, circle, fill=black, minimum width=.75\nodesize, inner sep=0]
\tikzstyle{square}=[rectangle, draw]
\tikzstyle{param}=[draw, rectangle, fill=gray!50, minimum width=\nodesize, minimum height=\nodesize, inner sep=0]
\tikzstyle{hidden}=[draw, circle, fill=gray!50, minimum width=\nodesize, inner sep=0]
\tikzstyle{hiddenred}=[draw, circle, color=red, fill=gray!50, minimum width=\nodesize, inner sep=0]
\tikzstyle{observed}=[draw, circle, minimum width=\nodesize, inner sep=0]
\tikzstyle{observedred}=[draw, circle, minimum width=\nodesize, color=red, inner sep=0]
\tikzstyle{eliminated}=[draw, circle, minimum width=\nodesize, color=gray!50, inner sep=0]
\tikzstyle{empty}=[draw, circle, minimum width=\nodesize, color=white, inner sep=0]
\tikzstyle{blank}=[color=white]
\tikzstyle{nocircle}=[minimum width=\nodesize, inner sep=0]

\tikzstyle{edge}=[-, line width=\edgewidth]
\tikzstyle{edgebendleft}=[-, >=latex, line width=\edgewidth, bend left]
\tikzstyle{edgebendright}=[-, >=latex, line width=\edgewidth, bend right]
\tikzstyle{lightedge}=[-, line width=\edgewidth, color=gray!50]
\tikzstyle{lightedgebendleft}=[-, >=latex, line width=\edgewidth, bend left, color=gray!50]
\tikzstyle{lightedgebendright}=[-, >=latex, line width=\edgewidth, bend right, color=gray!50]
\tikzstyle{edgered}=[-, line width=\edgewidth, color=red]
\tikzstyle{edgebendleftred}=[-, >=latex, line width=\edgewidth, bend left, color=red]
\tikzstyle{edgebendrightred}=[-, >=latex, line width=\edgewidth, bend right, color=red]

\tikzstyle{arrow}=[->, >=latex, line width=\edgewidth]
\tikzstyle{arrowbendleft}=[->, >=latex, line width=\edgewidth, bend left]
\tikzstyle{arrowbendright}=[->, >=latex, line width=\edgewidth, bend right]
\tikzstyle{arrowred}=[->, >=latex, line width=\edgewidth, color=red]
\tikzstyle{arrowbendleftred}=[->, >=latex, line width=\edgewidth, bend left, color=red]
\tikzstyle{arrowbendrightred}=[->, >=latex, line width=\edgewidth, bend right, color=red]
\tikzstyle{arrowblue}=[->, >=latex, line width=\edgewidth, color=blue]
\tikzstyle{dashedarrow}=[->, >=latex, dashed, line width=\edgewidth]
\tikzstyle{dashededge}=[-, >=latex, dashed, line width=\edgewidth]
\tikzstyle{dashededgebendleft}=[-, >=latex, dashed, line width=\edgewidth, bend left]
\tikzstyle{lightarrow}=[->, >=latex, line width=\edgewidth, color=gray!50]


%====================================================================
%====================================================================
\begin{document}
%====================================================================
%====================================================================

%====================================================================
\title{Biologie et �cologie Pr�dictive : Mod�les \& Algorithmes}

\author{S. Robin}

\institute[]{MIA-Paris, AgroParisTech / INRA / univ. Paris-Saclay}

\date[Rennes, nov. 2019]{INRA, Rennes, 15 novembre 2019}

%====================================================================
%====================================================================
\maketitle
%====================================================================


%====================================================================
%====================================================================
\frame{\frametitle{}

  \paragraph{Contexte}
  \begin{enumerate}
  \item Syst�mes biologiques vus comme des syst�mes complexes
  \item Donn�es massives et essor de l'intelligence artificielle
  \end{enumerate}

  \bigskip \bigskip 
  \paragraph{Objectif}
  \begin{itemize}
  \item Compr�hension / pr�diction / gestion (contr�le) de syst�mes biologiques
  \end{itemize}
  
  \bigskip \bigskip 
  \paragraph{Biologie pr�dictive}
  \begin{itemize}    
  \item Recours � des m�thodes formelles (math�matiques, informatiques) � ces fins
  \end{itemize}
}

%====================================================================
\frame{\frametitle{Plan}
  \tableofcontents
}

\section{Syst�mes complexes \& Mod�les}
\frame{\frametitle{Outline} \tableofcontents[currentsection]}
%====================================================================
%====================================================================
\subsection{Mod�les}
\frame{\frametitle{Outline} \tableofcontents[currentsubsection]}
%====================================================================
\frame{\frametitle{Mod�le}

  $$
  y = f(x; p)
  $$
  \begin{itemize}
   \item $y =$ r�ponse du syst�me
   \item $x =$ conditions (exp�rimentales, environnementales, �tat ant�rieur du syst�me)
   \item $f =$ 'mod�le' (syst�me d'�quations, r�gles d'�volution, ...)
   \item $p =$ param�tres (coefficients, seuils, ...)
  \end{itemize}
  
  \bigskip \bigskip \pause
  \paragraph{Mod�lisation :} Choisir la forme de $f$

  \bigskip \bigskip  \pause
  \paragraph{+} mesure / recueil / d�termination / estimation de $p$ \\ ~\\

}

%====================================================================
\frame{\frametitle{Quels mod�les / quels objectifs ?}

  \paragraph{Deux arch�types :}
  \begin{description}
  \item[Empirique :] Capable de g�n�rer des donn�es $(x, y)$ similaires aux observations
  \item[M�caniste :] D�crivant effectivement les m�canismes sous-jacents
  \end{description}

  \bigskip \bigskip \pause
  \paragraph{Objectifs.} 
  \begin{enumerate}
   \item \pause Comprendre : �tablir un $f$ (et d�terminer un $p$) ``r�aliste'' \\ ~
   \item \pause Pr�dire : $\hat{y}_ 0 = f(x_0; p)$, $x_0 =$ sc�nario, g�nome, ... \\ ~\\
%    \ra Ex. : r�ponse d'un �cosyst�me � une perturbation environnementale \\ ~
   \item \pause G�rer / contr�ler : d�terminer $u^*$ tel que $f(x_0, u^*; p) =$ valeur cible $y^*$ \\ ~\\
%    \ra Ex. :  m�decine personnalis�e
  \end{enumerate}

}

%====================================================================
%====================================================================
\subsection{Syst�mes complexes}
\frame{\frametitle{Outline} \tableofcontents[currentsubsection]}
%====================================================================
\frame{\frametitle{Syst�mes biologiques $\rightarrow$ Syst�mes complexes $\rightarrow$ Mod�les complexes}

  \paragraph{Syst�me complexe :}
  \begin{itemize}
   \item 'Syst�me' = ensemble organis� d'�l�ments
   \item 'Complexe' = constitu� de plusieurs �l�ments / {\sl difficile de ce fait}
  \end{itemize}
  
  \bigskip \bigskip \pause
  \paragraph{Recours � une mod�lisation (repr�sentation) formelle}
  \begin{itemize}
   \item des entit�s biologiques
   \item de leurs interactions
   \item de leur �volution dans les temps et dans l'espace
  \end{itemize}
  
  \bigskip \bigskip \pause
  \paragraph{Cons�quences :}
  \begin{itemize}
   \item D�finition / abstraction des objets et des connaissances
   \item Formalisation de leurs relations
   \item Traduction des questions de recherche en un langage formel
  \end{itemize}
}

%====================================================================
\frame{\frametitle{Abstraction / Formalisation}

  \paragraph{D�finition des entit�s.}
  \begin{itemize}
   \item �cologie microbienne : esp�ces, OTU ({\sl operational taxonomic unit}) ou g�nes
   \item R�seaux de r�gulation : transcrits, prot�ines, m�tabolites
  \end{itemize}

  \bigskip \bigskip \pause
  \paragraph{Formalisation des relations.}
  \begin{itemize}
   \item Relations d�terministes : $y = f(x)$, probabilistes : $p(y \mid x)$, logiques : $y = (x_1 \land x_2) \lor (�x_3)$
%    \item Description quantitative : $y = a x_1 + b x_2^2$, logique : $y = (x_1 \land x_2) \lor (�x_3)$
   \item Hi�rarchies
  \end{itemize}

  \bigskip \bigskip \pause
  \paragraph{Formalisation de l'expertise.}
  \begin{itemize}
   \item 'Elicitation de prior' : repr�sentation (probabiliste) d'un savoir individuel
   \item D�finition d'ontologies
  \end{itemize}
  
}

%====================================================================
\frame{\frametitle{Exemple : Mod�les de r�seau �cologique}

  \begin{minipage}[t]{.48\linewidth}
   \paragraph{Mod�le dynamique :}
   $\dot{x}_j(t) = \sum_i a_{ij} \, x_i(t)$
   $$
   \includegraphics[trim={360 0 160 200}, width=.6\textwidth, clip]{Figs/StaticDynamicNetwork}
   $$ 
   \begin{itemize}
   \item Donn�es temporelles
   \item Description 'm�caniste'
   \item Limite d'�chelle
   \item R�silience, robustesse
   \end{itemize}
  \end{minipage} \hfill \pause
  \begin{minipage}[t]{.48\linewidth}
   \paragraph{Mod�le probabiliste :} $p(x) \propto \prod_C p(x_C)$    
   $$
   \includegraphics[trim={360 205 160 0}, width=.6\textwidth, clip]{Figs/StaticDynamicNetwork}
   $$
   \begin{itemize}
   \item Donn�es statiques
   \item Interaction directe / indirecte
   \item Mutualisme, comp�tition
   \item Centralit�, agr�gation
   \end{itemize}
  \end{minipage} 
}

%====================================================================
\frame{\frametitle{Traduction math�matique / Conclusions math�matiques}

  \paragraph{``Mod�le graphique'' :} 
  \begin{tabular}{ll}
    \begin{tabular}{p{.2\textwidth}}
    \includegraphics[trim={0 15 180 0}, width=.2\textwidth, clip]{Figs/Pea09-Fig2-3}
    \end{tabular}
    &
    \begin{tabular}{p{.7\textwidth}}
    les variations d'une entit� sont conditionnelles � celles de ses parents :
    $$
    p(a, b, c, d, e) = 
    p(a) \; p(b \mid a) \; p(c \mid a) \; p(d \mid b, c) \; p(e \mid d)
    $$ 
    \end{tabular}   
  \end{tabular}

  \bigskip \bigskip \pause
  \begin{tabular}{lll}
    \begin{tabular}{p{.6\textwidth}}
    \paragraph{Identifiabilit�} � partir de donn�es observationnelles : ~\\ 
    \includegraphics[trim={0 15 0 0}, height=.25\textheight, clip]{Figs/Pea09-Fig2-4}
    \end{tabular}
    &
    \qquad
    &  \pause
    \begin{tabular}{p{.2\textwidth}}
    \paragraph{Causalit� :} ~\\ ~\\
    \includegraphics[trim={190 15 0 10.5}, height=.22\textheight, clip]{Figs/Pea09-Fig2-3}
    \end{tabular}   
  \end{tabular}

}

%====================================================================
\frame{\frametitle{Mod�les complexes}

  \begin{tabular}{ll}
    Mod�le environmental & R�seau de r�gulation \\
    \begin{tabular}{p{.45\textwidth}}
    \includegraphics[trim={20 20 15 80}, width=.45\textwidth, clip]{Figs/aquatox-large}
    \end{tabular}
    &
    \begin{tabular}{p{.4\textwidth}}
    \includegraphics[width=.45\textwidth]{Figs/KeggActin}
    \end{tabular}
  \end{tabular}

  \bigskip \bigskip 
  \begin{itemize}
   \item Compr�hension / description (carte � l'�chelle 1)
   \item Hubris mod�lisatrice : $y = f_3\left(\underset{\widehat{x}_1}{\underbrace{f_1(u_1; p_1)}}, \underset{\widehat{x}_2}{\underbrace{f_2(u_2; p_2)}}, x_3; p_3\right)$
  \end{itemize}
}

%===================================================================
\frame{\frametitle{Enjeux}

  \paragraph{Risque :} le mod�le remplace l'objet

  \bigskip \bigskip \pause
  \paragraph{Garde-fou :} confrontation avec l'observation \\
  \begin{itemize}
  \item Pouvoir de pr�diction du mod�le
%    \item Mod�les m�t�orologiques
%    \item Exp�riences de {\sl knock-out}
  \end{itemize}
  \ra A condition que la pr�diction soit testable (ex. : interaction, causalit�)

  \bigskip \bigskip \pause
  \paragraph{Sinon ...}
  \begin{itemize}
   \item S'assurer que les concepts math�matiques refl�tent les concepts biologiques
  \end{itemize}

}



\section{Donn�es \& Algorithmes}
\frame{\frametitle{Outline} \tableofcontents[currentsection]}
%====================================================================
%====================================================================
\subsection{Donn�es massives}
\frame{\frametitle{Outline} \tableofcontents[currentsubsection]}
%====================================================================
\frame{\frametitle{Avalanche de donn�es}

  \paragraph{Historique:}
  $$
  \text{
  \begin{tabular}{llll}
  & jadis, & puis, & et puis \\ \hline
  collecte : & homme & homme   & machine \\
  analyse :  & homme & machine & machine 
  \end{tabular}
  }
  $$
  
  \bigskip \bigskip \pause
  \paragraph{Masse de donn�es}
  \begin{itemize}
   \item de toutes natures (g�nomique, ph�notypique, climatique, etc), 
   \item � diff�rentes �chelles (g�nes, cellule, individu, milieu, plan�te), 
   \item avec diff�rentes structures (nombres, images, sons, textes, ...)
  \end{itemize}

  \bigskip \bigskip \pause
  \paragraph{Gradient de strat�gie} pour leur collecte
  $$
  \text{opportuniste } \rightarrow
  \text{ participative (``protocol�e'' ou non) } \rightarrow
  \text{ syst�matique / planifi�e} 
  $$
  ~}

%====================================================================
\frame{\frametitle{Plannifi� vs opportuniste}

  \paragraph{Planification exp�rimentale.}
  \begin{itemize}
   \item Recueil optimis� du point de vue de la mod�lisation
   \item Garanties sur la valeur de g�n�ralit� des r�sultats obtenus
  \end{itemize}
%   \ra R�ponse � une question pr�cis�mment formul�e \\
  \ra Pertinence conditionnelle au mod�le


  \bigskip \bigskip \pause
  \paragraph{Science de l'observation.}
  \begin{itemize}
   \item Approche naturaliste, exploration
  \end{itemize}
  \ra Besoin d'outils automatiques pour l'``observation'' de donn�es massives 

}

%====================================================================
%====================================================================
\subsection{Apports de l'IA}
\frame{\frametitle{Outline} \tableofcontents[currentsubsection]}
%====================================================================
\frame{\frametitle{Essort de l'intelligence articifielle}

  \paragraph{Promesse de l'IA =} Apprentissage automatique \\
  {\sl ``des algorithmes plus autonomes pour extraire et synth�tiser de la connaissance''}
  
  \bigskip \bigskip \pause
  \paragraph{Progr�s r�cents~:} 
  \begin{itemize}
   \item masse croissante de donn�es disponibles
   \item accroissement des moyens de calcul
   \item progr�s th�oriques en apprentissage ({\sl algorithmes})
  \end{itemize}

  \bigskip \bigskip \pause
  \paragraph{Masses des donn�es:} 
  \begin{itemize}
   \item Parfois sans commune mesure avec les donn�es biologiques
  \end{itemize}

  \bigskip \bigskip \pause
  \paragraph{Puissance de calcul:} 
  \begin{itemize}
   \item Nerf de la guerre
   \item Des probl�mes (ex.: combinatoires) insolvable par une approche frontale
  \end{itemize}
}

%====================================================================
\frame{\frametitle{{\sl Algorithme}}

  \paragraph{D�finition~:}
  \textsl{suite finie et non ambigu� d'op�rations ou d'instructions permettant de \textbf{r�soudre une classe de probl�mes}}

  \bigskip \bigskip \pause
  \paragraph{Usage~:} ``Algorithme'' d�signe, selon les cas,
  \begin{enumerate}
   \item le probl�me � r�soudre
   \item la suite d'op�rations
   \item le r�sultat
  \end{enumerate} \pause
  Confusion encore plus fr�quente: processus d'apprentissage / formule de pr�diction

  \bigskip \bigskip \pause
  \paragraph{Biologie pr�dictive~: } typiquement des probl�mes 
  \begin{enumerate}[($a$)]
   \item de calcul num�rique, algorithmique, combinatoire
   \item supervis�s~: pr�diction, classification (optimisation)
   \item non-supervis�s~: r�duction de dimension, {\sl clustering}
  \end{enumerate}

}

%====================================================================
\frame{\frametitle{Probl�mes supervis�s}

  \paragraph{Parti pris~:} mod�le empirique
  $$
  \hat{f}(x, \hat{p}) = \hat{y} \simeq y
  $$

  \bigskip \bigskip \pause
  \paragraph{Large panoplie d'algorithmes~:} r�seaux de neurones, machines � vecteurs supports, for�ts al�atoires, .... 
  \begin{itemize}
   \item efficaces pour g�rer les donn�es de grande dimension
   \item aux performances empiriques au niveau de l'�tat de l'art
   \item assortis de certaines garanties th�oriques (erreur de g�n�ralisation)
  \end{itemize}

  \bigskip \bigskip \pause
  \paragraph{Ingr�dients du succ�s.} 
  \begin{itemize}
   \item Repr�sentation des donn�es ('{\sl with good features, learning is easy}')
   \item Choix de l'algorithme
   \item Crit�re de qualit� de la pr�diction
  \end{itemize}

}

%====================================================================
\frame{\frametitle{Choix des algorithmes (en supervis�)}

  \paragraph{Pr�diction g�nomique:} $y =$ ph�notype, $x = $ marqueurs, $\widehat{y} = \widehat{f}(x; \widehat{p})$
  
  \bigskip \pause
  \hspace{-0.05\textwidth}
  \begin{tabular}{ll}
   \begin{tabular}{p{.45\textwidth}}
   \paragraph{R�seau de neurone:} \\ ~\\
   \includegraphics[trim={0 0 0 1700}, width=.45\textwidth, clip]{Figs/neuralnetworks} \\ ~
   \end{tabular}
   & \pause
   \begin{tabular}{p{.5\textwidth}}
   \end{tabular}
   \begin{tabular}{p{.45\textwidth}}
   \paragraph{P�nalisation / r�gularisation:} \\ ~\\
   \includegraphics[trim={0 0 0 0}, width=.45\textwidth, clip]{Figs/Regularization} \\ ~\\
   $[$group, fused, Bayesian, sparse-group$]$-lasso, ridge, elastic-net, ... \\ ~\\ ~
   \end{tabular}
   \end{tabular}

  \pause
  \begin{itemize}
   \item Grande dimension~: r�gularisation / r�duction de dimension
   \item Interpr�tabilit� / sur-interpr�tation des r�sultats ($\hat{f}$, $\hat{p}$)
   \item P�nalit� � fa�on~: prise en compte du d�s�quilibre de liaison
  \end{itemize}

}

%====================================================================
\frame{\frametitle{Probl�mes non supervis�s}

  \paragraph{Parti pris~:} question non formelle, voire pas de question 

  \bigskip \bigskip \pause
  \paragraph{Large panoplie d'algorithmes~:} ...
  \begin{itemize}
   \item efficaces ... 
   \item performances ... 
   \item garanties th�oriques {\sl (conditionnelles � la formalisation du probl�me)}
  \end{itemize}

  \bigskip \bigskip \pause
  \paragraph{Ingr�dients du succ�s.} 
  \begin{itemize}
   \item Repr�sentation des donn�es
   \item Choix de l'algorithme
   \item Mesure de qualit� de la solution {\sl (conditionnelle � la formalisation du probl�me)}
  \end{itemize}

}

% %====================================================================
% \frame{\frametitle{Repr�sentation des donn�es}
% 
%   \bigskip \bigskip 
%   \paragraph{Repr�sentation.}
%   \begin{itemize}
%    \item Son (bases de fonctions), image (structure spatiale), texte ('sacs de mots'), ...
%   \end{itemize}
% 
% 
%   \bigskip 
%   \begin{itemize}
%    \item Prise en compte de la connaissance pr�alable (covariables, structure de population)
%   \end{itemize}
% 
% 
% 
% }
% 
%====================================================================
\frame{\frametitle{Crit�re d'�valuation (en non-supervis�)}

  \paragraph{{\sl Clustering}.}
  $$
  \hspace{-0.05\textwidth}
  \begin{tabular}{c|c|c|c}
   Donn�es & Algorithme 1 & Algorithme 2 & Algorithme 1' \\
   \includegraphics[width=.23\textwidth]{Figs/Classif0} & \pause
   \includegraphics[width=.23\textwidth]{Figs/Classif1} &
   \includegraphics[width=.23\textwidth]{Figs/Classif2} &
   \includegraphics[width=.23\textwidth]{Figs/Classif3} 
  \end{tabular}
  $$
  
  \bigskip % \pause
  \begin{itemize}
   \item Choix du 'meilleur' algorithme 
   \item Repr�sentation des donn�es
  \end{itemize}
  ({\sl Avoid blind use of metrics, learn formulas instead})
  
}

%====================================================================
\frame{\frametitle{Extensions naturelles}

  \paragraph{Int�gration de donn�es.}
  \begin{itemize}
   \item Repr�sentation formelle de donn�es
   \item Combinaision des repr�sentations (ex. : produit scalaire)
  \end{itemize}

  \bigskip \bigskip \pause
  \paragraph{Combinaison de pr�dicteurs.}
  \begin{itemize}
   \item Diff�rents pr�dicteurs disponibles pour une m�me t�che
   \item Combinaison de pr�dicteur = (meta-)probl�me d'apprentissage
  \end{itemize}

  \bigskip \bigskip \pause
  \paragraph{Apprentissage par renforcement.}
  \begin{itemize}
   \item Mesure locale de la qualit� de la pr�diction \\
   \ra D�tection des zones uncertaines et requ�te de nouvelles mesures
   \item Automatisation de la confrontation avec l'exp�rience \\
   \ra Biologie synth�tique
  \end{itemize}
  
}

%===================================================================
\frame{\frametitle{Enjeux}

  \paragraph{Risque~:} l'algorithme remplace la question

  \bigskip \bigskip \pause
  \paragraph{Garde-fou~:} confrontation avec l'observation
  \begin{itemize}
   \item Apprentissage supervis�
   \item Automatisation des exp�riences
  \end{itemize}
  \ra A condition que la r�ponse soit �valuable

  \bigskip \bigskip \pause
  \paragraph{Sinon ...}
  \begin{itemize}
   \item S'assurer que l'algorithme r�pond bien � la question pos�e
   \item A d�faut: conna�tre la question � laquelle il r�pond
  \end{itemize}

}



\section{Mod�lisation empirique / m�caniste}
\frame{\frametitle{Outline} \tableofcontents[currentsection]}
%====================================================================
%====================================================================
\subsection{Mod�les empiriques vs m�canistes}
\frame{\frametitle{Outline} \tableofcontents[currentsubsection]}
%====================================================================
\frame{\frametitle{Mod�les empiriques / m�canistes}

  \paragraph{Breiman (2001) :} '{\sl Statistical Modeling: The Two Cultures}'
  \begin{itemize}
   \item m�caniste : {\sl assumes that the data are generated by a given stochastic \textbf{data model}}
   \item empirique : {\sl uses \textbf{algorithmic models} and treats the data mechanism as unknown}
  \end{itemize}
  
  \bigskip \bigskip \pause
  \paragraph{Parti pris de l'apprentissage:}
  \begin{itemize}
   \item inutile de comprendre pour pr�dire
   \item seulement comprendre pour agir (au mieux) 
  \end{itemize}
  \ra Pr�diction et optimisation \\
  \ra Pas d'analyse des m�canismes
  
  \bigskip \bigskip \pause
  \paragraph{En pratique :} approches hydrides

}

%====================================================================
%====================================================================
\subsection{Approches hybrides}
\frame{\frametitle{Outline} \tableofcontents[currentsubsection]}
%====================================================================
\frame{\frametitle{Approche hybride : s�lection g�nomique}

  \paragraph{Pr�diction d'un ph�notype} � partir du g�notype :  \\ ~ \\
  \begin{itemize}
  \item Mod�le empirique de pr�diction fond� sur une population contemporaine \\ ~
  \item Mod�le g�n�tique d'�volution de population (soumise � s�lection) \\ ~
  \item Estimation de la qualit� des pr�dictions dans les g�n�rations futures
  \end{itemize}

}


%====================================================================
\frame{\frametitle{Approche hybride : r�seaux �cologiques}

  \begin{tabular}{ll}
    \begin{tabular}{p{.6\textwidth}}
    \paragraph{Inf�rence de r�seaux �cologiques.} ~\\ 
    \begin{itemize}
     \item Mod�le de Lotka-Volterra stochastique (d) ou mod�le graphique probabiliste (b) \\~
     \item Inf�rence par maximum de vraisemblance \\~
     \item P�nalisation lasso pour r�duire le nombre d'ar�tes
    \end{itemize}
    \end{tabular}
    &
    \hspace{-0.05\textwidth}
    \begin{tabular}{p{.4\textwidth}}
   \includegraphics[trim={360 0 160 200}, width=.25\textwidth, clip]{Figs/StaticDynamicNetwork} 
   \\ ~ \\
   \includegraphics[trim={360 205 160 0}, width=.25\textwidth, clip]{Figs/StaticDynamicNetwork}
    \end{tabular}
  \end{tabular}

}

%====================================================================
\frame{\frametitle{Approche hybride : g�n�tique des populations}

  \begin{tabular}{ll}
    \begin{tabular}{p{.45\textwidth}}
    \paragraph{Comparaison de sc�narios �volutifs.} ~\\ 
    \begin{itemize}
     \item Mod�le probabiliste d'�volution \\~
     \item Inf�rence bay�sienne (ABC) \\~
     \item For�ts al�atoires pour estimer la probabilit� {\it a posteriori} de chaque sc�nario
    \end{itemize}
    \end{tabular}
    &
    \hspace{-0.05\textwidth}
    \begin{tabular}{p{.3\textwidth}}
    \includegraphics[width=.45\textwidth]{Figs/Rob17-StatLearn-Fig1} \\
    \\
    \includegraphics[width=.45\textwidth]{Figs/Rob17-StatLearn-Fig2}
    \end{tabular}
  \end{tabular}

}




\backupbegin
% %====================================================================
% \frame{\frametitle{Que stocker ?}
% 
%   \paragraph{Sp�cificit� des donn�es mol�culaires.}
%   \begin{itemize}
%    \item Mesures g�n�ralement indirectes
%    \item Dont la nature varie (fr�quemment) du fait des �volutions technologiques
%    \item Effort permanent de normalisation / correction ...
%   \end{itemize}
% 
%   \bigskip \bigskip \pause
%   \paragraph{Que stocker ?}
%   \begin{itemize}
%    \item Donn�es brutes (d�finition de 'donn�e brute') + Logiciels de normalisation
%    \item Donn�es corrig�e + Version du logiciel
%    \item Refaire la manip
%   \end{itemize}
%   
% }

%====================================================================
\frame{\frametitle{}

}

%====================================================================
\frame{\frametitle{Breiman, 2001: 'Statistical Modeling: The Two Cultures'}

  \paragraph{Abstract.} ~\\~\\
  There are two cultures in the use of statistical modeling to reach conclusions from data. One assumes that the data are generated by a given stochastic data model. The other uses algorithmic models and treats the data mechanism as unknown. \\ ~
  
  The statistical community has been committed to the almost exclusive use of data models. 
  This commitment has led to irrelevant theory, questionable conclusions, and has kept statisticians from working on a large range of interesting current problems. \\ ~
  
  Algorithmic modeling, both in theory and practice, has developed rapidly in fields outside statistics. It can be used both on large complex data sets and as a more accurate and informative alternative to data modeling on smaller data sets.  \\ ~
  
  If our goal as a field is to use data to solve problems, then we need to move away from exclusive dependence on data models and adopt a more diverse set of tools  
}

%====================================================================
\frame{\frametitle{Mod�le graphique, Identifiabilit�, Causalit�}

  \begin{tabular}{c|c|c}
    'V�rit�' & Mod�les non distinguables & 'Causalit�' \\  
    & \footnotesize{� partir de donn�es observationnelles, non {\sl interventionnelles}} \\  
    & & \\\hline & & \\
    \includegraphics[trim={0 15 180 0}, height=.22\textheight, clip]{Figs/Pea09-Fig2-3}
    &
    \includegraphics[trim={0 15 0 0}, height=.21\textheight, clip]{Figs/Pea09-Fig2-4}
    &
    \includegraphics[trim={180 15 0 0}, height=.22\textheight, clip]{Figs/Pea09-Fig2-3}
  \end{tabular}

  \bigskip \bigskip 
  $$
  a \leftrightarrow b \qquad = \qquad a \leftarrow u \rightarrow b
  $$
  $u$ non observ�e

}


\backupend

%====================================================================
%====================================================================
\end{document}
%====================================================================
%====================================================================
  
