\documentclass[10pt]{beamer}

% Beamer style
%\usetheme[secheader]{Madrid}
% \usetheme{CambridgeUS}
\useoutertheme{infolines}
\usecolortheme[rgb={0.65,0.15,0.25}]{structure}
% \usefonttheme[onlymath]{serif}
\beamertemplatenavigationsymbolsempty
%\AtBeginSubsection

% Packages
%\usepackage[french]{babel}
\usepackage[latin1]{inputenc}
\usepackage{color}
\usepackage{xspace}
\usepackage{dsfont, stmaryrd}
\usepackage{amsmath, amsfonts, amssymb}
\usepackage{epsfig}
\usepackage{tikz}
\usepackage{url}
\usepackage{/home/robin/LATEX/Biblio/astats}
%\usepackage[all]{xy}
\usepackage{graphicx}

% Commands
% % TikZ
\newcommand{\nodesize}{1.8em}
\newcommand{\edgeunit}{2.2*\nodesize}
\tikzstyle{hidden}=[draw, circle, fill=gray!50, minimum width=\nodesize, inner sep=0]
\tikzstyle{observed}=[draw, circle, minimum width=\nodesize, inner sep=0]
\tikzstyle{eliminated}=[draw, circle, minimum width=\nodesize, color=gray!50, inner sep=0]
\tikzstyle{empty}=[]
\tikzstyle{arrow}=[->, >=latex, line width=1pt]
\tikzstyle{edge}=[-, line width=1pt]
\tikzstyle{dashedarrow}=[->, >=latex, dashed, line width=1pt]
\tikzstyle{lightarrow}=[->, >=latex, line width=1pt, fill=gray!50, color=gray!50]


\definecolor{darkred}{rgb}{0.65,0.15,0.25}
\newcommand{\backupbegin}{
   \newcounter{finalframe}
   \setcounter{finalframe}{\value{framenumber}}
}
\newcommand{\backupend}{
   \setcounter{framenumber}{\value{finalframe}}
}
\newcommand{\emphase}[1]{\textcolor{darkred}{#1}}
% \newcommand{\emphase}[1]{{#1}}
\newcommand{\paragraph}[1]{\textcolor{darkred}{#1}}
\newcommand{\refer}[1]{{\small{\textcolor{blue}{{[\cite{#1}]}}}}}
% \newcommand{\Refer}[1]{{\small{\textcolor{gray}{{[#1]}}}}}
\renewcommand{\newblock}{}

% Symbols
\newcommand{\Abf}{{\bf A}}
\newcommand{\Beta}{\text{B}}
\newcommand{\Bcal}{\mathcal{B}}
\newcommand{\BIC}{\text{BIC}}
\newcommand{\Ccal}{\mathcal{C}}
\newcommand{\dd}{\text{~d}}
\newcommand{\dbf}{{\bf d}}
\newcommand{\Dcal}{\mathcal{D}}
\newcommand{\Esp}{\mathbb{E}}
\newcommand{\Ebf}{{\bf E}}
\newcommand{\Ecal}{\mathcal{E}}
\newcommand{\Gcal}{\mathcal{G}}
\newcommand{\Gam}{\mathcal{G}\text{am}}
\newcommand{\Hcal}{\mathcal{H}}
\newcommand{\Ibb}{\mathbb{I}}
\newcommand{\Ibf}{{\bf I}}
\newcommand{\ICL}{\text{ICL}}
\newcommand{\Cov}{\mathbb{C}\text{ov}}
\newcommand{\Corr}{\mathbb{C}\text{orr}}
\newcommand{\Var}{\mathbb{V}}
\newcommand{\Jcal}{\mathcal{J}}
\newcommand{\Lcal}{\mathcal{L}}
\newcommand{\mt}{\widetilde{m}}
\newcommand{\Nbb}{\mathbb{N}}
\newcommand{\Ncal}{\mathcal{N}}
\newcommand{\Ocal}{\mathcal{O}}
\newcommand{\Omegas}{\underset{s}{\Omega}}
\newcommand{\pt}{\widetilde{p}}
\newcommand{\Pcal}{\mathcal{P}}
\newcommand{\Qcal}{\mathcal{Q}}
\newcommand{\St}{\widetilde{S}}
\newcommand{\cst}{\text{cst}}
\newcommand{\ra}{\emphase{\mathversion{bold}{$\rightarrow$}~}}
%\newcommand{\transp}{\text{{\tiny $\top$}}}

% Directory
\newcommand{\fignet}{/home/robin/Bureau/RECHERCHE/RESEAUX/EXPOSES/FIGURES}
\newcommand{\figchp}{/home/robin/Bureau/RECHERCHE/RUPTURES/EXPOSES/FIGURES}


%====================================================================
%====================================================================

%====================================================================
%====================================================================
\begin{document}
%====================================================================
%====================================================================

%====================================================================
\title[Probabilistic PCA for counts]{Probabilistic multivariate analysis of count data: Application to community ecology}

\author[S. Robin]{S. Robin \\ ~\\
  \begin{tabular}{ll}
    Joint work with J. Chiquet \& M. Mariadassou
  \end{tabular}
  }

\institute[INRA / AgroParisTech]{~ \\%INRA / AgroParisTech \\
  \vspace{-.1\textwidth}
  \begin{tabular}{ccc}
    \includegraphics[height=.25\textheight]{\fignet/LogoINRA-Couleur} & 
    \hspace{.02\textheight} &
    \includegraphics[height=.06\textheight]{\fignet/logagroptechsolo} % & 
%     \hspace{.02\textheight} &
%     \includegraphics[height=.09\textheight]{\fignet/logo-ssb}
    \\ 
  \end{tabular} \\
  \bigskip
  }

\date[EMS'17, Helsinki]{Eutopean Meeting of Statisticians, July 2017, Helsinki}

%====================================================================
%====================================================================
\maketitle
%====================================================================

%====================================================================
%====================================================================
\section{Multivariate analysis of abundance data}
\frame{\tableofcontents[currentsection]}
%====================================================================

%====================================================================
\subsection*{Abundance data}
%====================================================================
\frame{\frametitle{Community ecology}

  \paragraph{Abundance data.} $Y = [Y_{ij}]: n \times p$: 
  \begin{eqnarray*}
   Y_{ij} & = & \text{abundance of species $j$ in sample $i$ (old)} \\
    & = & \text{number of reads associated with species $j$ in sample $i$ (new)}
  \end{eqnarray*}
%   $$
%   Y = [Y_{ij}]: n \times p, \qquad \text{either $n$ or $p$ 'large'}
%   $$
%   \begin{itemize}
%    \item $Y_{ij} =$ count of $k$-mer $j$ in sample $i$
%    \item $Y_{ij} =$ abundance of species $j$ in sample $i$
%    \item ...
%   \end{itemize}
  
  \bigskip \bigskip 
  \paragraph{Need for multivariate analysis:} 
  \begin{itemize}
   \item to summarize the information from $Y$
   \item to exhibit patterns of diversity
   \item to understand between-species interactions
   \item ...
  \end{itemize}
  
  \bigskip
  \pause More generally, to \emphase{model dependences between count variables}
  $$
  \text{\ra Need for a generic (probabilistic) framework}
  $$

}

%====================================================================
\frame{\frametitle{Models for multivariate count data.}

  \paragraph{Abundance vector:} $Y_i = (Y_{i1}, \dots Y_{ip})$, $Y_{ij} = \emphase{\text{ counts } \in \Nbb}$

  \bigskip \bigskip \pause
  \paragraph{No generic model for multivariate counts.}
  \begin{itemize}
   \item Data transformation ($\widetilde{Y}_{ij} = \log (1+Y_{ij}), \sqrt{Y_{ij}}$) \\ 
   \ra Pb when many counts are zero. \\ ~
   \item Poisson multivariate distributions \\
   \ra Constraints of the form of the dependency \refer{IYA16} \\ ~
   \item Latent variable models \\
   \ra Poisson-Gamma (negative binomial \ra positive dependency) \\
   \ra \emphase{Poisson-log normal} \refer{AiH89}
  \end{itemize}
}

%====================================================================
\frame{\frametitle{{P}oisson-log normal (PLN) distribution}

  \paragraph{Latent Gaussian model:} 
  \begin{itemize}
   \item $Z_i:$ latent vector $\sim \Ncal_p(\mu, \Sigma)$ \\ 
   \item $Y_i = (Y_{ij})_j:$ counts independent conditional on $Z_i$ 
   $$
   Y_{ij} \,|\, Z_{ij} \sim \Pcal\left(e^{Z_{ij}}\right)
   $$
  \end{itemize}

  \bigskip \bigskip \pause
  \paragraph{Some properties:}
%   \begin{align*}
%    \Esp(Y_{ij}) & = e^{\mu_j + \Sigma_{jj}/2} =: \lambda_j > 0 \\
%    \Var(Y_{ij}) & = \lambda_j + \lambda_j^2 \left( e^{\Sigma_{jj}} - 1\right) 
%    \qquad (\text{over-dispersion}) \\
%    \Cov(Y_{ij}, Y_{ik}) & = \lambda_j \lambda_k \left( e^{\Sigma_{jk}} - 1\right)
%    \quad \qquad (\text{arbitrary sign}) 
%   \end{align*}
  \begin{itemize}
   \item $\Var(Y_{ij}) \geq \Esp(Y_{ij})$ : \text{over-dispersion}
   \item $\Cov(Y_{ij}, Y_{ik})$ : \text{arbitrary sign}
   \item Borrows nice properties of the multivariate Gaussian distribution
  \end{itemize}
}

%====================================================================
%====================================================================
\section{Probabilistic PCA for counts}
\frame{\tableofcontents[currentsection]}
%====================================================================

%====================================================================
\subsection*{pPCA}
%====================================================================
\frame{\frametitle{Reminder: Regular probabilistic PCA (pPCA)}

  \paragraph{Gaussian model with constrained covariance.} $Y_i = (Y_{i1}, \dots Y_{ip})$
  $$
  \{Y_i\} \text{ iid } \sim \Ncal_p(\mu, \Sigma)
  $$
  where $\Sigma$ as almost rank $q <p$: $\Sigma = B' B + \sigma^2 I_p$.
  
  \bigskip \bigskip \pause
  \paragraph{Latent variable version.} 
  \begin{itemize}
   \item scores: $\{W_i\}_i \text{ iid } \sim \Ncal_q(0, I_q)$
   \item noise: $\{E_i\}_i \text{ iid } \sim \Ncal_p(0, \sigma^2 I_p)$
   \item loadings: $B = q \times p$ matrix
  \end{itemize}

  $$
  Y_i = \mu + B' W_i + E_i
  $$
}

%====================================================================
\frame{\frametitle{Accounting for covariates}

  \paragraph{Additional information.}
  \begin{itemize}
   \item $x_i: d$-vector of descriptors for observation $i$
   \item $\beta: d \times p$ matrix of coefficients, $\beta_{kj} =$ effect of covariate $k$ on species $j$
  \end{itemize}
  
  $$
  Y_i = \beta' x_i + B' W_i + E_i
  $$

  \bigskip \bigskip \pause
  \paragraph{General matrix form.}
  $$
  Y = X \beta + W B + E
  $$
}

%====================================================================
\frame{\frametitle{Inference}

  \paragraph{Incomplete data model:} 
  \begin{itemize}
   \item $Y :$ observed variables, 
   \item $W :$ unobserved variables, 
   \item $\theta = (\beta, B, \sigma^2):$ unknown parameters
  \end{itemize}

  \bigskip \bigskip 
  \paragraph{EM algorithm.} \refer{DLR77}
  \begin{itemize}
   \item E step: given $\theta$ compute (some moments of)
   $$
   \emphase{p_\theta(W_i \,|\, Y_i)}
   $$
   explicit for Gaussian pPCA \refer{TiB99}. 
   \item M step: update $\theta$ using the 'completed' likelihood 
   $$
   \Esp [\log p_\theta(Y, W) \,|\, Y]
   $$
  \end{itemize}
}

%====================================================================
\subsection*{PLN-PCA}
%====================================================================
\frame{\frametitle{pPCA for count data}

  \paragraph{Idea.} Merge pPCA with Poisson log-normal \refer{CMR17}.
  
  \bigskip \bigskip \pause
  \begin{tabular}{p{.5\textwidth}p{.5\textwidth}}
  \paragraph{Latent variable:}
  \begin{itemize}
  \item $W_i \sim \Ncal_q(0, I_q)$ 
  \item $Z_i = \beta' x_i + B' W_i $
  \end{itemize}
  &
  \paragraph{Covariance structure:}
  \begin{itemize}
  \item $\Sigma = B' B$ \quad (no noise)
  \item $Z_i \sim \Ncal_p(\beta' x_i, \Sigma)$ 
  \end{itemize}
  \end{tabular}
  
  \begin{itemize}
   \item $(Y_{ij})$ independent $\,|\, \; (Z_{ij}):$ 
   $$
   Y_{ij} \sim \Pcal\left(e^{Z_{ij}}\right)
   $$
  \end{itemize}
  
}

%====================================================================
\frame{\frametitle{Variational inference  \refer{CMR17}}

  \paragraph{EM does not apply} because $p(W_i \,|\, Y_i)$ is intractable
  
  \bigskip \bigskip \pause
  \paragraph{Variational approximation \refer{WaJ08,Jaa00}.}
  $$
  p(W_i \,|\, Y_i) \approx \pt_{Y_i}(W_i) := \Ncal_q(W_i; \mt_i, \St_i)
  $$
  $\mt_i$, $\St_i$: variational parameters, chosen to minimize $KL$ divergence.

  \bigskip \bigskip \pause
  \paragraph{Variational EM.} Aim at maximizing the lower bound of $\log p_\theta(Y)$
  $$
  \Jcal(Y; \theta, \pt) := \log p_\theta(Y) - \sum_i KL[\pt_{Y_i}(W_i) || p(W_i \,|\, Y_i)]
  $$
}

%====================================================================
\frame{\frametitle{Natural exponential family}

  \paragraph{Generic pPCA for the natural exponential family.}
%   \begin{eqnarray*}
%    \{W_i\} \text{ iid} & \sim & \Ncal_q(0_q, I_q) \\
%    Z_i & = & \beta' x_i + B W_i' \\
%    \{Y_{ij}\} \text{ indep. } | \{Z_{ij}\}, \quad Y_{ij} | Z_{ij} & \sim & \exp\left(Y_{ij} Z_{ij} - b(Z_{ij}) - a(Y_{ij})\right)
%   \end{eqnarray*}
%   $\{W_i\} \text{ iid} \sim \Ncal_q(0_q, I_q)$, $Z_i = \beta' x_i + B W_i'$, $\{Y_{ij}\} \text{ indep. } | \{Z_{ij}\}$.
  \begin{itemize}
   \item scores: $\{W_i\} \text{ iid} \sim \Ncal_q(0_q, I_q)$
   \item 'parameters': $Z_i = \beta' x_i + B' W_i$
   \item observations: $\{Y_{ij}\} \text{ indep. } | \{Z_{ij}\}$, $Y_{ij} | Z_{ij} \sim P_{Z_{ij}}$
  \end{itemize}
  
  $$
  P_\lambda: \quad p_\lambda(y) = \exp \left[ y \lambda - b(\lambda) - a(y) \right]
  $$

  \bigskip \bigskip \pause
  \paragraph{Property.} 
  $$
  \text{$\Jcal(Y; \theta, \pt)$ is convex wrt both $(B, \beta)$ and $(\mt, \St)$}
  $$
  
  \bigskip \bigskip 
  \begin{itemize}
   \item No need for E and M steps \ra Classical gradient descent algorithm.
   \item Also hold for simple PLN, with explicit estimate of $\Sigma$
  \end{itemize}

}

%====================================================================
\frame{\frametitle{Model selection}

  \paragraph{Number of component $q$:} needs to be chosen.
  
  \bigskip \bigskip %\pause
  \paragraph{Penalized 'likelihood'.}
  \begin{itemize}
   \item $\log p_{\widehat{\theta}}(Y)$ intractable: replaced with $\Jcal(Y; {\widehat{\theta}}, \pt)$ \\ ~
   \item $BIC$ \refer{Sch78} \ra $vBIC_q = \Jcal(Y; {\widehat{\theta}}, \pt) - pq \log(n)/2$ \\ ~ 
   \item $ICL$ \refer{BCG00} \ra $vICL_q = vBIC_q - \Hcal(\pt)$ \\ ~ 
  \end{itemize}
  
  \bigskip
  \paragraph{Chosen rank:}
  $$
  \widehat{q} = \arg\max_q vBIC_q
  \qquad \text{or} \qquad
  \widehat{q} = \arg\max_q vICL_q
  $$
}

%====================================================================
\frame{\frametitle{Visualization}

  \paragraph{PCA:} Optimal subspaces nested when $q$ increases.
  
  \bigskip \bigskip 
  \paragraph{PLN-pPCA:} Non-nested subspaces.

  \bigskip
  \ra For a the selected dimension $\widehat{q}$: \\~
  \begin{itemize}
   \item Compute the estimated latent positions $\widetilde{P} = \widetilde{M} \widehat{B}$ \\~
   \item Perform PCA on the $\widetilde{P}$ \\~
   \item Display results in any dimension $q \leq \widehat{q}$
  \end{itemize}
}

%====================================================================
\section{Illustrations}
\frame{\tableofcontents[currentsection]}
%====================================================================

%====================================================================
\frame{\frametitle{Metagenomic experiment}

  \paragraph{Data.} Metabarcoding
  $$
  Y_{ij} = \text{number of reads associated with species $j$ in sample $i$}
  $$

  \bigskip \bigskip \pause
  \paragraph{Generic model.} 
  $$
  Y_{ij} \sim \Pcal(e^{Z_{ij}}), \qquad Z_{ij} = o_i + x_i' \beta^j + W_i' B^j
  $$
  \begin{itemize}
   \item $o_i =$ \emphase{offset} = sequencing depth in sample $i$
   \item $x_i =$ covariates
   \item $W_i =$ scores = principal coordinates = PC
   \item $\beta^j = j$-th column of $\beta$ (idem $B^j$)
  \end{itemize} 
}

%====================================================================
\subsection*{Pathobiome}
%====================================================================
\frame{\frametitle{Pathobiome: Oak powdery mildew}

  \paragraph{Data from \refer{JFS16}.} 
  \begin{itemize}
   \item $n = 116$ oak leaves = samples
   \item $p_1 = 66$ bacterial species (OTU)
   \item $p_2 = 48$ fungal species ($p = 114$)
   \item covariates: tree (resistant, intermediate, susceptible), branch height, distance to trunk, ...
   \item offsets: $o_{i1}, o_{i2} =$ offset for bacteria, fungi
  \end{itemize}

%   \bigskip \bigskip \pause
%   \paragraph{Aim.} Understand the interaction between the species, including the oak mildew pathogene {\sl E. alphitoides}.
}

%====================================================================
\frame{\frametitle{Pathobiome: PLN model ($q = p$)}

  \begin{overprint}
   \onslide<1>
   \paragraph{Without covariates}
   $$
   \includegraphics[width=.8\textwidth]{../FIGURES/PLN_oaks_plot2-1}
   $$
   \onslide<2>
   \paragraph{With covariates}
   $$
   \includegraphics[width=.8\textwidth]{../FIGURES/PLN_oaks_plot3-1}
   $$
  \end{overprint}
}

%====================================================================
\frame{\frametitle{Pathobiome: PCA rank selection}

  $$
  \includegraphics[width=.9\textwidth]{../FIGURES/CMR17-Fig1a_ModSel.pdf}
  $$
  \qquad \qquad offset only: $\widehat{q} = 24$; \qquad \qquad offset + covariates: $\widehat{q} = 21$, 
}

%====================================================================
\frame{\frametitle{Pathobiome: First 2 PCs}

  $$
  \includegraphics[width=.9\textwidth]{../FIGURES/CMR17-Fig1c_IndMap.pdf}
  $$
  \qquad \qquad \qquad offset only; \qquad \qquad \qquad offset + covariates, 
}


%====================================================================
\frame{\frametitle{Pathobiome: Precision of $\widehat{Z}_{ij}$}
  
  \begin{center}
     \begin{tabular}{cc}
	 $\sqrt{\widetilde{\Var}}(Z_{ij})$ & \begin{tabular}{c}
	 \includegraphics[width=.5\textwidth]{../FIGURES/CMR17-Fig2_VarZ.pdf} 
	 \end{tabular} \\
	 & $Y_{ij}$
	\end{tabular}
  \end{center}

  \bigskip
  Due to the link function (log) $\widetilde{\Var}(Z_{ij})$ is higher when $Y_{ij}$ is close to 0.}

%====================================================================
\subsection*{Microbiome}
%====================================================================
\frame{\frametitle{Microbiome: Weaning of piglets}

  \begin{tabular}{cc}
    \begin{tabular}{p{.4\textwidth}}
    \paragraph{Data from \refer{MBE15}.} 
    \begin{itemize}
    \item $n = 115$ samples % (= $51$ piglets at $5$ times)
    \item $p = 4031$ bacterial species % (OTU)
    %    \item covariates: piglet, time
    \item offsets: $o_i =$ offset % for sample $i$
    \end{itemize}

    \bigskip \bigskip
    Considering only a fraction of (most abundant) species.
    \end{tabular}
    & 
    \hspace{-.02\textwidth}
    \begin{tabular}{p{.5\textwidth}}
    \includegraphics[width=.5\textwidth]{../FIGURES/CMR17-Fig3_timings.pdf} \\
    ~ \\
    $x$: number species; \\
    $y$: computation time (s)
    \end{tabular}
  \end{tabular}
}

%====================================================================
\frame{\frametitle{Microbiome: Model selection}

  \begin{tabular}{ccc}
    $\log_{10}$ abundances & $R^2$ criterion & chosen rank $\widehat{q}$ \\
    \includegraphics[width=.3\textwidth]{../FIGURES/CMR17-Fig4a-CritAbundance.pdf}
    &
    \includegraphics[width=.3\textwidth]{../FIGURES/CMR17-Fig4b-CritR2.pdf}
    &
    \includegraphics[width=.3\textwidth]{../FIGURES/CMR17-Fig4c-CritRank.pdf}
  \end{tabular}
  ~ \\
  $x-$axis: number of species
}

%====================================================================
\frame{\frametitle{Microbiome: Standard PCA outputs}
  $$
  \includegraphics[height=.85\textheight]{../FIGURES/CMR17-Fig5.pdf}
  $$
}
%====================================================================
%====================================================================
\section{Discussion}
\frame{\tableofcontents[currentsection]}
%====================================================================

%====================================================================
\frame{\frametitle{Summary}

  \begin{itemize}
   \item Generic PLN model combining pPCA and GLM \\~
   \item Efficient variational algorithm \\~
   \item Model selection criteria 
  \end{itemize}
  
  \bigskip \bigskip \pause
  \paragraph{And } (see \refer{CMR17})\\~
  \begin{itemize}
   \item Goodness-of-fit \\~
%    \item Visualization (non-nested latent spaces when $q$ increases) \\~
   \item Package available at \emphase{\tt https://github.com/jchiquet/PLNmodels} %\\
%    Syntax:
%     $$
%     \text{\url{
%     PCA = PLNPCA(Y ~ 1 + X + offset(O), Q=1:10, control = ...)
%     }}
%     $$
  \end{itemize}
}

%====================================================================
\frame{\frametitle{Extensions}

  \begin{itemize}
   \item Theoretical property of variational estimates (starting with regular PLN) \\~
   \item Extension to other distributions (Bernoulli, ...) \\ ~
   \item Force sparsity in $B$ ('sparse pPCA') \\ ~
   \item Other classical mutivariate analyses (e.g. LDA) \\ ~
   \item ...
  \end{itemize}

}

%====================================================================
\frame{ \frametitle{References}
{\tiny
  \bibliography{/home/robin/Biblio/BibGene}
%   \bibliographystyle{/home/robin/LATEX/Biblio/astats}
  \bibliographystyle{alpha}
  }
}

%====================================================================
%====================================================================
%====================================================================
\appendix 
\backupbegin
\section{Appendix}
% \frame{\tableofcontents[currentsection]}
%====================================================================
% \frame{\frametitle{Appendix}}
%====================================================================

% %====================================================================
% \frame{\frametitle{Covariance structure}
% 
%   \begin{tabular}{cc}
%     Empirical covariance & \hspace{-.15\textwidth} Rotated covariance \\
%     \begin{tabular}{p{.5\textwidth}}
%     \includegraphics[width=.45\textwidth]{../FIGURES/Fig-pPCA-Sigma.pdf}
%     \end{tabular}
%     & 
%     \hspace{-.1\textwidth}
%     \begin{tabular}{p{.5\textwidth}}
%     \includegraphics[width=.45\textwidth]{../FIGURES/Fig-pPCA-rotSigma.pdf}
%     \end{tabular} \\
%     $p = 20$ & \hspace{-.15\textwidth} $q = 5$ 
%   \end{tabular}
% }
% 
%====================================================================
\frame{\frametitle{Goodness of fit}

  \paragraph{pPCA:} Cumulated sum of the eigenvalues = \% of variance preserved on the first $q$ components.
  
  \bigskip \bigskip 
  \paragraph{PLN-pPCA:} Deviance based criterion.
  \begin{itemize}
   \item Compute $\widetilde{Z}^q = O + X \widehat{\beta} + \widetilde{M}^q \widehat{B}^q$
   \item Take $\lambda_{ij}^q = \exp(\widetilde{Z}_{ij}^q)$
   \item Define $\lambda_{ij}^{\min} = \exp( \widetilde{Z}_{ij}^0)$ and $\lambda_{ij}^{\max} = Y_{ij}$ 
   \item Compute the Poisson log-likelihood $\ell_q = \log P(Y; \lambda^q)$
  \end{itemize}
  
  \bigskip \bigskip 
  \paragraph{Pseudo-$R^2$:} 
  $$
  R_q^2 = \frac{\ell_q - \ell_{\min}}{\ell_{\max} - \ell_{\min}}
  $$
}

%====================================================================
\frame{\frametitle{Pathobiome: Goodness of fit}
  
  $$
  \includegraphics[width=.9\textwidth]{../FIGURES/CMR17-Fig1b_GoodnessEntropy.pdf}
  $$
  $x-$axis: $q$; $y-$axis: left = $R_q^2$, right = $-\Hcal(\pt) = vBIC_q - vICL_q$
}

\backupend

%====================================================================
%====================================================================
\end{document}
%====================================================================
%====================================================================

  \begin{tabular}{cc}
    \begin{tabular}{p{.5\textwidth}}
    \end{tabular}
    & 
    \hspace{-.02\textwidth}
    \begin{tabular}{p{.5\textwidth}}
    \end{tabular}
  \end{tabular}

