\documentclass[dvips, lscape]{foils}
%\documentclass[dvips, french]{slides}
\textwidth 18cm
\textheight 25cm 
\topmargin -1cm 
\oddsidemargin  -1cm 
\evensidemargin  -1cm

% Maths
\usepackage{amsfonts, amsmath, amssymb}
\newcommand{\Bcal}{\mathcal{B}}
\newcommand{\C}{C}
\newcommand{\Esp}{\mathbb{E}}
\newcommand{\Gcal}{\mathcal{G}}
\newcommand{\Hcal}{\mathcal{H}}
\newcommand{\Hbf}{{\bf H}}
\newcommand{\ellt}{\tilde{\ell}}
\newcommand{\lambdat}{\tilde{\lambda}}
\newcommand{\Lcal}{\mathcal{L}}
\newcommand{\mut}{\tilde{\mu}}
\newcommand{\Ncal}{\mathcal{N}}
\newcommand{\Pcal}{\mathcal{P}}
\newcommand{\Pibf}{\mbox{\mathversion{bold}{$\Pi$}}}
\newcommand{\Sbf}{{\bf S}}
\newcommand{\wbf}{{\bf w}}
\newcommand{\w}{{\bf w}}

\newcommand{\binomcoef}[2]{\left(\begin{array}{c}#1\\#2\end{array}\right)}

% Couleur et graphiques
\usepackage{color}
\usepackage{graphics}
\usepackage{epsfig} 
\usepackage{pstcol}

% Texte
\usepackage{lscape}
\usepackage{../../../../Latex/fancyheadings, rotating, enumerate}
\usepackage[french]{babel}
\usepackage[latin1]{inputenc}
\definecolor{darkgreen}{cmyk}{0.5, 0, 0.5, 0.5}
\definecolor{orange}{cmyk}{0, 0.6, 0.8, 0}
\definecolor{jaune}{cmyk}{0, 0.5, 0.5, 0}
\newcommand{\textblue}[1]{\textcolor{blue}{#1}}
\newcommand{\textred}[1]{\textcolor{red}{#1}}
\newcommand{\textgreen}[1]{\textcolor{green}{ #1}}
\newcommand{\textlightgreen}[1]{\textcolor{green}{#1}}
%\newcommand{\textgreen}[1]{\textcolor{darkgreen}{#1}}
\newcommand{\textorange}[1]{\textcolor{orange}{#1}}
\newcommand{\textyellow}[1]{\textcolor{yellow}{#1}}

% Sections
\newcommand{\chapter}[1]{\centerline{\LARGE \textblue{#1}}}
\newcommand{\section}[1]{\centerline{\Large \textblue{#1}}}
\newcommand{\subsection}[1]{\noindent{\large \textblue{#1}}}
\newcommand{\emphase}[1]{{\textblue{\sl #1}}}
\newcommand{\paragraph}[1]{\noindent {\textblue{#1}}}

%%%%%%%%%%%%%%%%%%%%%%%%%%%%%%%%%%%%%%%%%%%%%%%%%%%%%%%%%%%%%%%%%%%%%%
%%%%%%%%%%%%%%%%%%%%%%%%%%%%%%%%%%%%%%%%%%%%%%%%%%%%%%%%%%%%%%%%%%%%%%
%%%%%%%%%%%%%%%%%%%%%%%%%%%%%%%%%%%%%%%%%%%%%%%%%%%%%%%%%%%%%%%%%%%%%%
%%%%%%%%%%%%%%%%%%%%%%%%%%%%%%%%%%%%%%%%%%%%%%%%%%%%%%%%%%%%%%%%%%%%%%
\begin{document}
%%%%%%%%%%%%%%%%%%%%%%%%%%%%%%%%%%%%%%%%%%%%%%%%%%%%%%%%%%%%%%%%%%%%%%
%%%%%%%%%%%%%%%%%%%%%%%%%%%%%%%%%%%%%%%%%%%%%%%%%%%%%%%%%%%%%%%%%%%%%%
%%%%%%%%%%%%%%%%%%%%%%%%%%%%%%%%%%%%%%%%%%%%%%%%%%%%%%%%%%%%%%%%%%%%%%
%%%%%%%%%%%%%%%%%%%%%%%%%%%%%%%%%%%%%%%%%%%%%%%%%%%%%%%%%%%%%%%%%%%%%%
\landscape
\headrulewidth 0pt 
\pagestyle{empty} 
\cfoot{}
\rfoot{}
\rhead{\begin{rotate}{90}{
      \hspace{-.5cm} \tiny \thepage
      }\end{rotate}}
\setcounter{page}{0}

\begin{center}
  \chapter{Statistical tests to compare}
  \medskip
  \chapter{motif count exceptionalities} 

  \bigskip

  {\sl \large S. Schbath S. Robin and V. Vandewalle}

  \bigskip

  {CBME meeting: 10 March 2009, Jouy}
\end{center}

\vspace{3cm}
\paragraph{Reference:}\\
Schbath, S. Robin and Vandewalle (2207). \\ 
{\sl Statistical tests to compare motif count exceptionalities}. \\
BMC Bioinformatics, 8(84):1--12, 2007. doi:10.1186/1471-2105-8-84.

%%%%%%%%%%%%%%%%%%%%%%%%%%%%%%%%%%%%%%%%%%%%%%%%%%%%%%%%%%%%%%%%%%%%%
%%%%%%%%%%%%%%%%%%%%%%%%%%%%%%%%%%%%%%%%%%%%%%%%%%%%%%%%%%%%%%%%%%%%%
\newpage
\section{Introduction}
%%%%%%%%%%%%%%%%%%%%%%%%%%%%%%%%%%%%%%%%%%%%%%%%%%%%%%%%%%%%%%%%%%%%%

%%%%%%%%%%%%%%%%%%%%%%%%%%%%%%%%%%%%%%%%%%%%%%%%%%%%%%%%%%%%%%%%%%%%%
\subsection{Motif counts in DNA sequences}
%%%%%%%%%%%%%%%%%%%%%%%%%%%%%%%%%%%%%%%%%%%%%%%%%%%%%%%%%%%%%%%%%%%%%

%%%%%%%%%%%%%%%%%%%%%%%%%%%%%%%%%%%%%%%%%%%%%%%%%%%%%%%%%%%%%%%%%%%%%
\paragraph{8-letter palindrome.}
{\tt cagcgctg}, respectively on the loops (sequence 1) and the
backbone (sequence 2) of the \textit{E.  coli} K12 leading strands.

The have respective length $\ell_1 = 758\,434$ and $\ell_2 =
3\,882\,513$ and respective counts $N_1 = 30$ and $N_2 = 113$.

%%%%%%%%%%%%%%%%%%%%%%%%%%%%%%%%%%%%%%%%%%%%%%%%%%%%%%%%%%%%%%%%%%%%%
\bigskip\bigskip
\subsection{Extension to CBME}
%%%%%%%%%%%%%%%%%%%%%%%%%%%%%%%%%%%%%%%%%%%%%%%%%%%%%%%%%%%%%%%%%%%%%

%%%%%%%%%%%%%%%%%%%%%%%%%%%%%%%%%%%%%%%%%%%%%%%%%%%%%%%%%%%%%%%%%%%%%
\paragraph{Differential analysis} of NGS data also
leads to compare counts $N_i$ in runs (e.g. conditions or sample)
$i=1$ and $i=2$.

%%%%%%%%%%%%%%%%%%%%%%%%%%%%%%%%%%%%%%%%%%%%%%%%%%%%%%%%%%%%%%%%%%%%%
%%%%%%%%%%%%%%%%%%%%%%%%%%%%%%%%%%%%%%%%%%%%%%%%%%%%%%%%%%%%%%%%%%%%%
\newpage
\section{Poisson model and hypothesis testing}
%%%%%%%%%%%%%%%%%%%%%%%%%%%%%%%%%%%%%%%%%%%%%%%%%%%%%%%%%%%%%%%%%%%%%

%%%%%%%%%%%%%%%%%%%%%%%%%%%%%%%%%%%%%%%%%%%%%%%%%%%%%%%%%%%%%%%%%%%%%
\subsection{Poisson model} \label{Sec:PoissonModel}
%%%%%%%%%%%%%%%%%%%%%%%%%%%%%%%%%%%%%%%%%%%%%%%%%%%%%%%%%%%%%%%%%%%%%

In sequence (\emphase{or sample}) $i$, the count $N_i$ is supposed to
have a Poisson distribution with mean (and variance) $\lambda_i$:
$$
N_i \sim \Pcal(\lambda_i)
$$
\paragraph{For motif count in sequences}, the mean $\lambda_i$ in
sequence $i$ must account for three parameters: 
\begin{enumerate}[($i$)]
\item the length $\ell_i$ of the sequence, 
\item the composition of the sequence (Markov model), 
\item the possible exceptionality of the motif in the sequence.
\end{enumerate}

\bigskip
\paragraph{For NGS counts} it should account for (???) 
\begin{enumerate}[($i$)]
\item the total number of reads in run $i$ (simular to $\ell_i$)
\item some expected transcriptional level ($\mu_i$ ?)
\item some expected abundance of a species ($\mu'_i$ ?)
\end{enumerate}

%%%%%%%%%%%%%%%%%%%%%%%%%%%%%%%%%%%%%%%%%%%%%%%%%%%%%%%%%%%%%%%%%%%%%
\newpage
\paragraph{Expected intensity.}
$$
\begin{array}{ll}
\text{model  M0:} & 
\displaystyle{\mu_i=\frac{\prod_{j=1}^{h}N_i(w_j)}{\ell_i^{h}}},\\
\text{model  M1:} & 
\displaystyle{\mu_i=\frac{\prod_{j=1}^{h-1}N_i(w_j w_{j+1})}
                        {\ell_i \prod_{j=2}^{h-1}N_i(w_j)}},\\
\text{model } M(h-2)\text{:} & 
\displaystyle{\mu_i=\frac{N_i(w_1 w_2 \cdots w_{h-1})\,N_i(w_2 \cdots w_h)}
                        {(\ell_i -m+1) N_i(w_2 \cdots w_{h-1})}},
\end{array}
$$
where $N_i(\cdot)$ denotes the count in the sequence $i$.

\bigskip
\begin{table}[h]
  \begin{center}
    \begin{tabular}{lcccc}
      Model          & M00 & M0 & M1 & M6  \\
      \hline
      $\ell_1 \mu_1$ & 11.6 & 9.4 & 13.9 & 24.8 \\
      $\ell_2 \mu_2$ & 59.2 & 66.0 & 106.2 & 126.1
    \end{tabular}
    \caption{Expected count for {\tt cagcgctg} in the loops (1) and in the
      bakbone (2) of the genome of {\it E. coli} under different
      models.}
    \label{Tab:EN-cagcgctg}
%     \begin{tabular}{lcccc}
%       Model          & M00 & M0 & M1 & M6  \\
%       \hline
%       $\ell_1 \mu_1$ & 59.2 & 75.4 & 73.6 & 556.3 \\
%       $\ell_2 \mu_2$ & 11.2 & 10.7 & 11.3 & 48.7
%     \end{tabular}
%     \caption{Expected count for the Chi motif in the backbone (1) and in the
%       loops (2) of the genome of {\it E. coli} under different
%       models.}
%     \label{Tab:EN-Chi}
  \end{center}
\end{table}

%%%%%%%%%%%%%%%%%%%%%%%%%%%%%%%%%%%%%%%%%%%%%%%%%%%%%%%%%%%%%%%%%%%%%
\newpage
\paragraph{Exceptionality coefficient.}
We introduce an exceptionality coefficient $k_i$ which
allows $\lambda_i$ to be greater (or smaller) than the expected value:
$$
\lambda_i := k_i \ell_i \mu_i.
$$
In the following, parameters $\ell_i$ and $\mu_i$ will be supposed
to be known a priori: they can be considered as two correction terms.
The inference will only be made on $k_i$.
 
%%%%%%%%%%%%%%%%%%%%%%%%%%%%%%%%%%%%%%%%%%%%%%%%%%%%%%%%%%%%%%%%%%%%%
\bigskip
\subsection{Hypothesis testing \label{Subsec:TestP}}
%%%%%%%%%%%%%%%%%%%%%%%%%%%%%%%%%%%%%%%%%%%%%%%%%%%%%%%%%%%%%%%%%%%%%

%%%%%%%%%%%%%%%%%%%%%%%%%%%%%%%%%%%%%%%%%%%%%%%%%%%%%%%%%%%%%%%%%%%%%
\paragraph{Model.} 
$$
N_1 \perp N_2, \qquad N_i \sim \Pcal(k_1 \ell_1\mu_1), \quad  i=1, 2. 
$$


%%%%%%%%%%%%%%%%%%%%%%%%%%%%%%%%%%%%%%%%%%%%%%%%%%%%%%%%%%%%%%%%%%%%%
\bigskip
\paragraph{Null hypothesis.} 
$$
\Hbf_0 = \{k_1 = k_2\}.
$$


%%%%%%%%%%%%%%%%%%%%%%%%%%%%%%%%%%%%%%%%%%%%%%%%%%%%%%%%%%%%%%%%%%%%%
%%%%%%%%%%%%%%%%%%%%%%%%%%%%%%%%%%%%%%%%%%%%%%%%%%%%%%%%%%%%%%%%%%%%%
\newpage
\section{Tests} 
%%%%%%%%%%%%%%%%%%%%%%%%%%%%%%%%%%%%%%%%%%%%%%%%%%%%%%%%%%%%%%%%%%%%%

%%%%%%%%%%%%%%%%%%%%%%%%%%%%%%%%%%%%%%%%%%%%%%%%%%%%%%%%%%%%%%%%%%%%%
\subsection{Exact binomial test} \label{Subsec:BinomTest}
%%%%%%%%%%%%%%%%%%%%%%%%%%%%%%%%%%%%%%%%%%%%%%%%%%%%%%%%%%%%%%%%%%%%%

If $N_1$ and $N_2$ are two Poisson counts with respective means
$\lambda_1$ and $\lambda_2$, the distribution of $N_1$ given their sum
$N_+ := N_1+N_2$ is binomial : 
$$
N_1 \sim \Bcal(N_+, \pi) \qquad \text{with} \pi = \lambda_1 /
(\lambda_1 +\lambda_2). 
$$  
Under $\Hbf_0$, we have
\begin{equation} \label{Eq:PiBinom}
\pi_0 
%= \frac{k_1 \ell_1 \mu_1}{k_1 \ell_1 \mu_1 + k_2 \ell_2 \mu_2}
= \frac{\ell_1 \mu_1}{\ell_1 \mu_1 + \ell_2 \mu_2}
\end{equation}
because $k_1 = k_2$.  In absence of correction (M00 model) , we have
$\pi_0 = \ell_1 / (\ell_1 +\ell_2)$. 

\begin{table}[h]
  \begin{center}
    \begin{tabular}{lcccc}
      Model          & M00 & M0 & M1 & M6  \\
      \hline
      $\pi_0$ (\%) & 16.3 & 12.4 & 11.6 & 16.4 \\
      $p_B$ & $8.6\;10^{-2}$ & $2.7\;10^{-3}$ & $9.1\;10^{-4}$ & $9.1\;10^{-2}$\\
    \end{tabular}
    \caption{Probability $\pi_0$ and $p$-value for {\tt cagcgctg}
      under different models.}
    \label{Tab:EN-cagcgctg}
%     \begin{tabular}{lcccc}
%       Model          & M00 & M0 & M1 & M6  \\
%       \hline
%       $\pi_0$ (\%) & 83.7 & 87.6 & 86.7 & 92.0 \\
%       $p_B$ & $6.8\;10^{-10}$ & $7.1\;10^{-4}$ & $5.1\;10^{-5}$ &
%       $7.7\;10^{-1}$
%     \end{tabular}
%     \caption{Probability $\pi_0$ and $p$-value for the Chi motif
%       under different models.}
%     \label{Tab:EN-Chi}
  \end{center}
\end{table}

%%%%%%%%%%%%%%%%%%%%%%%%%%%%%%%%%%%%%%%%%%%%%%%%%%%%%%%%%%%%%%%%%%%%%
We first propose an exact test based on a general property of the
Poisson distribution. 
%%%%%%%%%%%%%%%%%%%%%%%%%%%%%%%%%%%%%%%%%%%%%%%%%%%%%%%%%%%%%%%%%%%%%
\newpage
\subsection{Likelihood ratio test (LRT)}
%%%%%%%%%%%%%%%%%%%%%%%%%%%%%%%%%%%%%%%%%%%%%%%%%%%%%%%%%%%%%%%%%%%%%

$\Hbf_0$ versus $\Hbf'_1 = \{k_1 \neq k_2\}$ can be derived. 
\begin{eqnarray*}
LRT 
% & = & 2\left\{ N_1 \ln\left[ \frac{N_1 / \mu_1 \ell_1}{N_+ / (\mu_2
%       \ell_2 + \mu_1 \ell_1)} \right] + N_2 \ln\left[ \frac{N_2 /
%       \mu_2 \ell_2}{N_+ / (\mu_1 \ell_1 + \mu_2 \ell_2)} \right] \right\} \\
& = & 2\left[ N_1 \ln\left( \frac{N_1 / N_+}{\pi_0} \right) + N_2
  \ln\left( \frac{N_2 / N_+ }{1 - \pi_0} \right) \right] \\
& \underset{\Hbf_0}{\approx} & \chi^2_1
\end{eqnarray*}

\bigskip
\begin{table}[h]
  \begin{center}
    \begin{tabular}{ccccc}
      Model          & M00 & M0 & M1 & M6  \\
      \hline
      $LRT$ &  2.1 & 8.2 & 10.2 & 2.0 \\
      $p_L$ & $1.5\;10^{-1}$ & $4.2\;10^{-3}$ & $1.4\;10^{-3}$ &
      $1.6\;10^{-1}$ \\
    \end{tabular}
    \caption{$LRT$ statistic and associated $p$-value for {\tt cagcgctg}.}
    \label{Tab:EN-cagcgctg}
%     \begin{tabular}{ccccc}
%       Model          & M00 & M0 & M1 & M6  \\
%       \hline
%       $LRT$ & 37.6 & 10.6 & 15.7 & 0.4 \\
%       $p_L$ & $8.7\;10^{-10}$ & $1.1\;10^{-3}$ & $7.5\;10^{-5}$ &
%       $5.1\;10^{-1}$
%     \end{tabular}
%     \caption{$LRT$ statistic and associated $p$-value for the Chi motif
%       under different models.}
%     \label{Tab:EN-Chi}
  \end{center}
\end{table}

%%%%%%%%%%%%%%%%%%%%%%%%%%%%%%%%%%%%%%%%%%%%%%%%%%%%%%%%%%%%%%%%%%%%%
%%%%%%%%%%%%%%%%%%%%%%%%%%%%%%%%%%%%%%%%%%%%%%%%%%%%%%%%%%%%%%%%%%%%%
\newpage
\section{Usage conditions}
%%%%%%%%%%%%%%%%%%%%%%%%%%%%%%%%%%%%%%%%%%%%%%%%%%%%%%%%%%%%%%%%%%%%%

%%%%%%%%%%%%%%%%%%%%%%%%%%%%%%%%%%%%%%%%%%%%%%%%%%%%%%%%%%%%%%%%%%%%%
\subsection{LRT Distribution}
%%%%%%%%%%%%%%%%%%%%%%%%%%%%%%%%%%%%%%%%%%%%%%%%%%%%%%%%%%%%%%%%%%%%%

Figure \ref{Fig:FitLRTChi2} compares both levels (actual and nominal).
We see that the nominal level is only reached with $N_+ \simeq 1000$
for $\pi_0 = 0.5$ and even later for $\pi_0 = 0.95$.  Since the counts are
discrete, the actual level can never be exactly $\alpha$ leading to
oscillations in the plot.

\begin{figure}[h]
  \begin{center}
    \epsfig{file = ../Figures/FitLRTChi2.eps, clip=, width=16cm,
      height=6cm}
    \caption{Actual level (log scale) of the LRT test as a function of $N_+$
      (log scale) for a nominal level $\alpha = $ 0.1, 1, 5 or 10\%
      and probability $\pi_0 =$ 0.5 (left) and 0.95 (right). (Since the
      LRT test is two-sided, the right plot also holds for $\pi_0$ = 0.05)}
    \label{Fig:FitLRTChi2} 
  \end{center}
\end{figure}

%%%%%%%%%%%%%%%%%%%%%%%%%%%%%%%%%%%%%%%%%%%%%%%%%%%%%%%%%%%%%%%%%%%%%
\newpage
\subsection{Decidability limits for the binomial test}
%%%%%%%%%%%%%%%%%%%%%%%%%%%%%%%%%%%%%%%%%%%%%%%%%%%%%%%%%%%%%%%%%%%%%

This figure gives this critical value of $N_+$ ($y$-axis) for various
values of $\pi_0$ ($x$-axis) and $\alpha$ ($0.1\%, 1\%, 5\%, 10\%$).
$$
  \epsfig{file = ../Figures/Decidability-pi.eps, clip=, width=10cm,
    height=10cm}
$$
We see that for $\pi_0 = 0.7$ and $N_+ = 10$, one may get significant
results at a level greater than 5\% but not at a level smaller than
1\%.

%%%%%%%%%%%%%%%%%%%%%%%%%%%%%%%%%%%%%%%%%%%%%%%%%%%%%%%%%%%%%%%%%%%%%
\newpage
%%%%%%%%%%%%%%%%%%%%%%%%%%%%%%%%%%%%%%%%%%%%%%%%%%%%%%%%%%%%%%%%%%%%%


\begin{figure}[h]
  % VERSION MATLAB
  \begin{center}
    \begin{tabular}{lc}
      \begin{tabular}{p{8cm}}
        \subsection{Power} \\ \\ 
        Power of the exact binomial test with level $\alpha = 5\%$ as
        a function of $k_1/k_2$ ($x$-axis) for different value of
        $\pi_0$.  \\ 
        \\
        $N_+ =$ 5 (dashed black), 10 (dashed red), 20 (dashed blue),
        50 (dashed green), 100 (solid black), 500 (solid red) and 1000
        (solid blue). 
      \end{tabular}
      &
      \begin{tabular}{cc}
        $\pi_0$ & Power curves  \\
        \hline
        \vspace{-0.5cm}
        \\
        \begin{tabular}{c} 0.1 \end{tabular} 
        & \begin{tabular}{c} \epsfig{file = ../Figures/PowerBin.eps, clip=, width=10cm,
            height=2.5cm, bbllx=70, bblly=634, bburx=550, bbury=770} \end{tabular} \\
        \begin{tabular}{c} 0.2 \end{tabular} 
        & \begin{tabular}{c} \epsfig{file = ../Figures/PowerBin.eps, clip=, width=10cm,
            height=2.5cm, bbllx=70, bblly=498, bburx=550, bbury=634} \end{tabular} \\
        \begin{tabular}{c} 0.5 \end{tabular} 
        & \begin{tabular}{c} \epsfig{file = ../Figures/PowerBin.eps, clip=, width=10cm,
            height=2.5cm, bbllx=70, bblly=362, bburx=550, bbury=498} \end{tabular} \\
        \begin{tabular}{c} 0.8 \end{tabular} 
        & \begin{tabular}{c} \epsfig{file = ../Figures/PowerBin.eps, clip=, width=10cm,
            height=2.5cm, bbllx=70, bblly=226, bburx=550, bbury=362} \end{tabular} \\
        \begin{tabular}{c} 0.9 \end{tabular} 
        & \begin{tabular}{c} \epsfig{file = ../Figures/PowerBin.eps, clip=, width=10cm,
            height=2.5cm, bbllx=70, bblly=90, bburx=550, bbury=226} \end{tabular} \\
      \end{tabular}
    \end{tabular}
  \end{center}
\end{figure}

% %%%%%%%%%%%%%%%%%%%%%%%%%%%%%%%%%%%%%%%%%%%%%%%%%%%%%%%%%%%%%%%%%%%%%
% %%%%%%%%%%%%%%%%%%%%%%%%%%%%%%%%%%%%%%%%%%%%%%%%%%%%%%%%%%%%%%%%%%%%%
% \newpage
% \section{Case of overlapping-word}
% %%%%%%%%%%%%%%%%%%%%%%%%%%%%%%%%%%%%%%%%%%%%%%%%%%%%%%%%%%%%%%%%%%%%%

% %%%%%%%%%%%%%%%%%%%%%%%%%%%%%%%%%%%%%%%%%%%%%%%%%%%%%%%%%%%%%%%%%%%%%
% \subsection{Compound Poisson model}
% %%%%%%%%%%%%%%%%%%%%%%%%%%%%%%%%%%%%%%%%%%%%%%%%%%%%%%%%%%%%%%%%%%%%%
% The distribution of overlapping word occurrences can be modeled with a
% compound Poisson process (see \cite{Rob02}) in the following way:
% \begin{itemize}
% \item The word occurs in clumps distributed according to a Poisson
%   process.  
% \item The size $V_{ic}$ of the $c$th clump (in sequence $i$) is random
%   with geometric distribution.  Parameter $a_i$ is the {\it
%     overlapping probability} of the motif.
% \end{itemize}

% %%%%%%%%%%%%%%%%%%%%%%%%%%%%%%%%%%%%%%%%%%%%%%%%%%%%%%%%%%%%%%%%%%%%%
% \bigskip
% \subsection{Tests} \label{Subsec:Tests}
% %%%%%%%%%%%%%%%%%%%%%%%%%%%%%%%%%%%%%%%%%%%%%%%%%%%%%%%%%%%%%%%%%%%%%

% %%%%%%%%%%%%%%%%%%%%%%%%%%%%%%%%%%%%%%%%%%%%%%%%%%%%%%%%%%%%%%%%%%%%%
% \paragraph{Comparison of the number of clumps.}
% The comparison of the counts $C_1$ and $C_2$
% is then exactly equivalent to the comparison of the counts $N_1$ and
% $N_2$ studied in Section \ref{Subsec:TestP}, replacing $\lambda_i$
% with $\lambdat_i$ and $\mu_i$ with $\mut_i = (1-a_i)\mu_i$.

% %%%%%%%%%%%%%%%%%%%%%%%%%%%%%%%%%%%%%%%%%%%%%%%%%%%%%%%%%%%%%%%%%%%%%
% \bigskip
% \paragraph{Exact test for the overlapping probability under M00.}
% $\Hbf_0 = \{a_1 = a_2\}$. 
% \begin{equation} \label{Eq:CondDistCi}
% C_i - 1 \sim \Bcal(N_i-1, 1 - a_i)
% \end{equation}
% which means that the expected number of clumps decreases when the
% overlapping probability increases.

% $$
% \Pr\{C_1 = c_1 \;|\; N_1, N_2, C_+\} =
% \frac{\binomcoef{N_1-1}{c_1-1}
%   \binomcoef{N_2-1}{C_+-c_1-1}}{\binomcoef{N_+ - 2}{C_+-2} }.
% $$
% The overlapping probability $a_1$ is then significantly greater than
% $a_2$ if the probability $\Pr\{C_1 \leq c_1 \;|\; N_1, N_2, C_+\}$ is
% smaller than a given level $\alpha$.

% %%%%%%%%%%%%%%%%%%%%%%%%%%%%%%%%%%%%%%%%%%%%%%%%%%%%%%%%%%%%%%%%%%%%%
% \paragraph{Exact test in the general case.}
% In this case, the true overlapping probability in sequence $i$ is $b_i
% = h_i a_i$, where $h_i$ is an exceptionality coefficient (analogous to
% $k_i$ defined in Section \ref{Sec:PoissonModel}). The problem is then
% to test $ \Hbf_0 = \{h_1 = h_2\}$. 
% % Such a test is proposed in Appendix
% % \ref{Subsec:DefHyp}: it involves the generalized negative
% % hyper-geometric distribution.

%%%%%%%%%%%%%%%%%%%%%%%%%%%%%%%%%%%%%%%%%%%%%%%%%%%%%%%%%%%%%%%%%%%%%
%%%%%%%%%%%%%%%%%%%%%%%%%%%%%%%%%%%%%%%%%%%%%%%%%%%%%%%%%%%%%%%%%%%%%
\newpage
\section{Application: Comparison of backbone and loops  of {\it E. coli}}
%%%%%%%%%%%%%%%%%%%%%%%%%%%%%%%%%%%%%%%%%%%%%%%%%%%%%%%%%%%%%%%%%%%%%

%%%%%%%%%%%%%%%%%%%%%%%%%%%%%%%%%%%%%%%%%%%%%%%%%%%%%%%%%%%%%%%%%%%%%
\subsection{Comparison of exceptionalities} \label{Subsec:CompExcep}
%%%%%%%%%%%%%%%%%%%%%%%%%%%%%%%%%%%%%%%%%%%%%%%%%%%%%%%%%%%%%%%%%%%%%
\begin{center}
  \begin{tabular}{cc}
    \epsfig{file = ../Figures//C1C2_pB_8_B0.eps, clip=, angle=90,
    width=8cm, height=8cm} 
    &
    \epsfig{file = ../Figures/C1C2_pB_8_M1.eps, clip=, angle=90,
      width=8cm, height=8cm} 
  \end{tabular}
\end{center}
\begin{center}
  \begin{tabular}{rclcccc}
    \multicolumn{3}{r}{Model:}                &  M00  &   M0  &   M1
    &   M6   \\
    \hline
    & $p_B$ & $< 10^{-4}$ & 277   & 126   & 83
    & 37 \\
    $10^{-4} \leq$ & $p_B$ & $< 10^{-3}$ & 519   & 303   & 247
    & 4 \\
    $10^{-3} \leq$ & $p_B$ & $< 10^{-2}$ & 1758  & 1330  & 1143
    & 104 \\
    $10^{-2} \leq$ & $p_B$ &             & 62982 & 63777 & 64063
    & 65391 \\
  \end{tabular}
\end{center}

\newpage
\begin{table}[h]
  \begin{center}
    {\footnotesize
      \begin{tabular}{llllllll}
       \multicolumn{2}{c}{M00} & \multicolumn{2}{c}{M0} &
       \multicolumn{2}{c}{M1} & \multicolumn{2}{c}{M6} \\
%      \begin{tabular}{cccc}
%      \multicolumn{2}{c}{M00} & \multicolumn{2}{c}{M1}\\
      \hline
 {\tt ggcgctgg*} & $< 10^{-20}   $  &  {\tt ctggaaga~} & $6.8\;10^{-10}$  &  {\tt ctggaaga~} & $1.2\;10^{-10}$  &  {\tt tcggttac~} & $4.9\;10^{-4}$ \\
 {\tt gcgctgga~} & $2.5\;10^{-14}$  &  {\tt cgatgaag~} & $2.9\;10^{-9}$  &  {\tt atctggtg~} & $3.3\;10^{-8}$  &  {\tt ggttgatg*} & $5.4\;10^{-4}$ \\
 {\tt cggcgctg~} & $3.0\;10^{-13}$  &  {\tt gaagtgct~} & $7.2\;10^{-9}$  &  {\tt gaagtgct~} & $4.6\;10^{-8}$  &  {\tt gcgcatcc~} & $6.8\;10^{-4}$ \\
 {\tt tggcgctg*} & $5.8\;10^{-12}$  &  {\tt tgaaactg*} & $4.0\;10^{-8}$  &  {\tt ggcgctgg*} & $5.2\;10^{-8}$  &  {\tt taggccgc~} & $8.5\;10^{-4}$ \\
 {\tt gcgctggt~} & $7.2\;10^{-12}$  &  {\tt atctggtg~} & $4.9\;10^{-8}$  &  {\tt cgatgaag~} & $6.6\;10^{-8}$  &  {\tt aagcttcg~} & $1.1\;10^{-3}$ \\
 {\tt cgctggtg~} & $8.9\;10^{-12}$  &  {\tt gcgctgga~} & $8.0\;10^{-8}$  &  {\tt tatctggt*} & $1.1\;10^{-7}$  &  {\tt cgatgaag~} & $1.1\;10^{-3}$ \\
 {\tt cgcgctgg~} & $1.0\;10^{-10}$  &  {\tt cggtaaag~} & $1.1\;10^{-7}$  &  {\tt cggtaaag~} & $1.4\;10^{-7}$  &  {\tt cggataaa~} & $1.2\;10^{-3}$ \\
 {\tt gctggcga~} & $1.3\;10^{-10}$  &  {\tt ggttgatg*} & $1.4\;10^{-7}$  &  {\tt ggttgatg*} & $2.0\;10^{-7}$  &  {\tt ggggggac~} & $1.4\;10^{-3}$ \\
 {\tt tggcgcag~} & $1.7\;10^{-10}$  &  {\tt gtgctgga~} & $1.6\;10^{-7}$  &  {\tt gtgctgga~} & $2.5\;10^{-7}$  &  {\tt caggcgtt~} & $1.6\;10^{-3}$ \\
 {\tt ctggaaga~} & $3.1\;10^{-10}$  &  {\tt aattgtcg~} & $2.1\;10^{-7}$  &  {\tt tgggcttc~} & $5.6\;10^{-7}$  &  {\tt acgccttc~} & $1.8\;10^{-3}$ \\
\hline
\hline
 {\tt cggataag~} & $1.2\;10^{-19}$  &  {\tt cggataag~} & $3.9\;10^{-20}$  &  {\tt cggataag~} & $2.7\;10^{-18}$  &  {\tt gggataaa~} & $2.4\;10^{-4}$ \\
 {\tt ggataagg*} & $8.6\;10^{-16}$  &  {\tt ccgcatcc*} & $2.0\;10^{-16}$  &  {\tt taaggcgt*} & $9.1\;10^{-15}$  &  {\tt tcgaccaa~} & $3.0\;10^{-4}$ \\
 {\tt taaggcgt*} & $4.6\;10^{-15}$  &  {\tt ggataagg*} & $3.0\;10^{-16}$  &  {\tt ccgcatcc*} & $4.0\;10^{-14}$  &  {\tt agttttta*} & $4.5\;10^{-4}$ \\
 {\tt gataaggc~} & $1.2\;10^{-14}$  &  {\tt tgtaggcc~} & $1.1\;10^{-15}$  &  {\tt acgccgca*} & $4.0\;10^{-14}$  &  {\tt aagtgata*} & $5.3\;10^{-4}$ \\
 {\tt taataaaa~} & $1.9\;10^{-14}$  &  {\tt tcaggcct*} & $2.9\;10^{-15}$  &  {\tt ataaggcg~} & $3.2\;10^{-13}$  &  {\tt gatagcgc~} & $8.1\;10^{-4}$ \\
 {\tt ataaggcg~} & $5.6\;10^{-14}$  &  {\tt taaggcgt*} & $2.9\;10^{-15}$  &  {\tt gccgcatc~} & $1.0\;10^{-12}$  &  {\tt gggtcagg*} & $1.5\;10^{-3}$ \\
 {\tt ctgataag~} & $1.2\;10^{-13}$  &  {\tt gataaggc~} & $4.9\;10^{-15}$  &  {\tt gataaggc~} & $2.2\;10^{-12}$  &  {\tt agccgaga*} & $1.7\;10^{-3}$ \\
 {\tt tgtaggcc~} & $4.0\;10^{-13}$  &  {\tt ggcctaca~} & $1.1\;10^{-14}$  &  {\tt gttccccg*} & $4.0\;10^{-12}$  &  {\tt gaggttac~} & $1.7\;10^{-3}$ \\
 {\tt cttatccg~} & $5.5\;10^{-13}$  &  {\tt ccggccta~} & $1.2\;10^{-14}$  &  {\tt cgcatccg*} & $4.4\;10^{-12}$  &  {\tt cagagtcc*} & $1.8\;10^{-3}$ \\
 {\tt ccttatcc*} & $6.0\;10^{-13}$  &  {\tt aggcctac~} & $1.4\;10^{-14}$  &  {\tt tgtaggcc~} & $4.7\;10^{-12}$  &  {\tt ccctggcc*} & $2.0\;10^{-3}$ \\
      \end{tabular}
      }
    \caption{10 most significant motifs (and corresponding value of
      $p_B$) for model M00, M0, M1 and M6. $*$ indicates overlapping
      words. Top: over-represented in the backbone, bottom:
      over-represented in the loops.} 
    \label{Tab:Top20-pB}
    \end{center}
\end{table}

\paragraph{LRT versus binomial.} 
To make the graph more readable, we transform the $p$-value $p_B$ into
a Gaussian score $S_B$:
$
S_B = \Phi^{-1}(1-p_B).
$

\begin{center}
  \epsfig{file =
    /RECHERCHE/OCCURRENCES/CompExcep/ColiBackLoop8recouv/CompLRT_Bin_8_M1.eps,
    clip=, angle=90, width=10cm}
\end{center}

\begin{table}[h]
  \begin{center}
    \begin{tabular}{lcccc}
      Model          & M00 & M0 & M1 & M6  \\
      \hline
      Spearman (\%) & 99.7 & 99.7 & 99.7 & 99.3 \\
      Kendall (\%) & 96.0 & 95.6 & 95.5 & 93.3
    \end{tabular}
    \caption{Spermann and Kendall correlation coefficients between $LRT^s$
      and $p_B$ for different models.}
    \label{Tab:CorrCoef}
  \end{center}
\end{table}


% %%%%%%%%%%%%%%%%%%%%%%%%%%%%%%%%%%%%%%%%%%%%%%%%%%%%%%%%%%%%%%%%%%%%%
% \newpage
% \subsection{Test for overlaps}

% Very few motifs have significant differences in their clumps sizes.
% Tables \ref{Tab:SignifOverlaps} presents the results for the 4 words
% having a $p$-value smaller than 10\%. For all of them, no overlap is
% observed in the backbone ($C_1 = N_1 \Rightarrow$ all clumps are of size 1) while
% few are observed in the loops ($C_2 < N_2$). The probability $a$ is
% the overlapping probability under model M00.

% \begin{table}[h]
%   \begin{center}
%     \begin{tabular}{lcccccc}
%       Word & \multicolumn{2}{c}{backbone} & \multicolumn{2}{c}{loops} & $p$ & $a$ \\
%        & $C$ & $N$ & $C$ & $N$ & (\%) & (\%) \\
%       \hline
%       {\tt accactac} &    44 &    44 &    7 &    9 &   2.20 & 0.02 \\ %    1     0.021961
%       {\tt tattatta} &    69 &    69 &   38 &   41 &   4.83 & 1.56 \\ %    1     0.048394
%       {\tt tcggggtc} &    24 &    24 &    2 &    3 &   8.00 & 0.02 \\ %    1     0.080000
%       {\tt cgcgccgc} &   246 &   246 &   27 &   28 &   9.93 & 0.10 \\ %    1     0.099265
%     \end{tabular}
%     \caption{Words with significant differences in terms of overlaps
%       in the backbone / loops comparison.}
%     \label{Tab:SignifOverlaps}
%   \end{center}
% \end{table}

% \paragraph{RAG motifs.}

%%%%%%%%%%%%%%%%%%%%%%%%%%%%%%%%%%%%%%%%%%%%%%%%%%%%%%%%%%%%%%%%%%%%%
%%%%%%%%%%%%%%%%%%%%%%%%%%%%%%%%%%%%%%%%%%%%%%%%%%%%%%%%%%%%%%%%%%%%%
\newpage
\section{Discussion}
%%%%%%%%%%%%%%%%%%%%%%%%%%%%%%%%%%%%%%%%%%%%%%%%%%%%%%%%%%%%%%%%%%%%%

% \paragraph{Aggregation test.} The intermediate results
% \eqref{Eq:CondDistCi} and \eqref{Eq:CondDistNi} allow to compare the
% number of clumps with the number of occurrences in one single
% sequence. We can hence derive a test to check whether the occurrences
% of the motif are more aggregated than expected. Such a behavior would
% be characterized by an observed number of clumps smaller than
% expected.

% \paragraph{Motif family.}

%\appendix

%%%%%%%%%%%%%%%%%%%%%%%%%%%%%%%%%%%%%%%%%%%%%%%%%%%%%%%%%%%%%%%%%%%%%
\bigskip
\subsection{What we Past}
%%%%%%%%%%%%%%%%%%%%%%%%%%%%%%%%%%%%%%%%%%%%%%%%%%%%%%%%%%%%%%%%%%%%%

The proposed test allows to compare \emphase{motifs counts in DNA
  sequences} as well as \emphase{NGS counts}.

Its power of is known for a given total count $N_+$.

%%%%%%%%%%%%%%%%%%%%%%%%%%%%%%%%%%%%%%%%%%%%%%%%%%%%%%%%%%%%%%%%%%%%%
\bigskip\bigskip
\subsection{Future}
%%%%%%%%%%%%%%%%%%%%%%%%%%%%%%%%%%%%%%%%%%%%%%%%%%%%%%%%%%%%%%%%%%%%%

Is it useful for NGS data (yes?).

Which effects should be accounted for?

In the experimental setting for the power computation similar?

%%%%%%%%%%%%%%%%%%%%%%%%%%%%%%%%%%%%%%%%%%%%%%%%%%%%%%%%%%%%%%%%%%%%%%
%%%%%%%%%%%%%%%%%%%%%%%%%%%%%%%%%%%%%%%%%%%%%%%%%%%%%%%%%%%%%%%%%%%%%%
%%%%%%%%%%%%%%%%%%%%%%%%%%%%%%%%%%%%%%%%%%%%%%%%%%%%%%%%%%%%%%%%%%%%%%
%%%%%%%%%%%%%%%%%%%%%%%%%%%%%%%%%%%%%%%%%%%%%%%%%%%%%%%%%%%%%%%%%%%%%%
\end{document}
%%%%%%%%%%%%%%%%%%%%%%%%%%%%%%%%%%%%%%%%%%%%%%%%%%%%%%%%%%%%%%%%%%%%%%
%%%%%%%%%%%%%%%%%%%%%%%%%%%%%%%%%%%%%%%%%%%%%%%%%%%%%%%%%%%%%%%%%%%%%%
%%%%%%%%%%%%%%%%%%%%%%%%%%%%%%%%%%%%%%%%%%%%%%%%%%%%%%%%%%%%%%%%%%%%%%
%%%%%%%%%%%%%%%%%%%%%%%%%%%%%%%%%%%%%%%%%%%%%%%%%%%%%%%%%%%%%%%%%%%%%%

