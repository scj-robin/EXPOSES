%====================================================================
%====================================================================
\subsection*{Latent variable models}
%====================================================================
\frame{\frametitle{Latent variable models (1/2)} 

  \paragraph{Latent  ('hidden', 'unobserved', ...) variables} are widely used in statistical ecology  \refer{PeG22b} to 
  \begin{itemize}
    \setlength{\itemsep}{.5\baselineskip}
    \item account for heterogeneity (species clustering, over-dispersion or 'excess' of zeros in population sizes), 
    \item encode dependency (due to space, time, parental relatedness, ...), 
    \item represent a 'true' signal, observed with noise (animal movement), 
    \item ...
  \end{itemize}

  \bigskip \bigskip \pause
  \paragraph{Statistical perspective.} 
  \begin{itemize}
    \setlength{\itemsep}{.5\baselineskip}
    \item Number of latent variables $\simeq$ number of observed variables.
    \item Inference of the model parameters much easier if the latent variables were observed.
  \end{itemize}
}

%====================================================================
\frame{\frametitle{Latent variable models (2/2} 

\paragraph{Notations.}
  \begin{itemize}
    \item $Y =$ observed variable of interest (response),
    \item $Z =$ unobserved (latent) variable,
    \item $\theta =$ unknown parameter (to be inferred),
    \item $X =$ set of covariates,
  \end{itemize}

  \bigskip \bigskip \pause 
  \paragraph{General model.}
  \begin{align*}
      \text{hidden layer:} \qquad \qquad Z & \sim p_\theta(Z; \textcolor{gray}{X}), \\ ~ \\
      \text{observed layer:} \qquad Y \mid Z & \sim p_\theta(Y \mid Z; \textcolor{gray}{X}).
  \end{align*}

}

%====================================================================
\frame{\frametitle{Graphical model} 

  \begin{tabular}{cc}
    \hspace{-.04\textwidth}
    \begin{tabular}{p{.5\textwidth}}
      \begin{itemize}
        \item $Y =$ response,
        \item $Z =$ latent,
        \item $\theta =$ parameter,
        \item $X =$ covariates
      \end{itemize}
    \end{tabular}
    & 
    \hspace{-.25\textwidth}
    \begin{tabular}{p{.5\textwidth}}
      \input{\figeco/GMlatentVarModel-XYZtheta}
    \end{tabular}
  \end{tabular}
  
  \bigskip \bigskip \pause
  \paragraph{Nature of the variables.}
  $$
  \begin{tabular}{l|cc}
    & Fixed & Random \\
    \hline
    & \\
    Observed & covariates $\emphase{X}$ & response $\emphase{Y}$ \\
    & \\
    Unobserved & parameter $\emphase{\theta}$ & latent $\emphase{Z}$
  \end{tabular}
  $$
}

%====================================================================
\frame{\frametitle{Inference specificity} 

  \paragraph{'Observed likelihood'} = likelihood of the observed data $Y$:
  $$
  p_\theta(Y) 
  = 
  \int_\Zcal p_\theta(Z) p_\theta(Y \mid Z) \d Z
  $$
  \pause
  but (EM decomposition \refer{DLR77})
  $$
  \log p_\theta(Y) = \Esp[\log p_\theta(Y, Z) \mid Y] + \Hcal(p_\theta(Z \mid Y)
  $$
  where $\Hcal$ stands for the entropy\footnote{$\Hcal(q) = - \Esp_q[\log q(X)]$}.
  
  \bigskip \bigskip \pause
  \paragraph{Three typical situations:}
  \begin{enumerate}
    \setlength{\itemsep}{0.75\baselineskip}
    \item \pause Integration over $Z$ can be done for free,
    \item \pause Integration over $Z$ is intractable, but $p_\theta(Z \mid Y)$ is accessible,
    \item \pause Integration over $Z$ is intractable, but $p_\theta(Z \mid Y)$ is inaccessible.
  \end{enumerate}

}
