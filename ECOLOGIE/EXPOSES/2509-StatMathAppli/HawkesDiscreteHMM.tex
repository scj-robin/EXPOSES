%====================================================================
%====================================================================
\subsection{A hidden Markov model?} 
\frame{\frametitle{Discrete time Markov-switching Hawkes process} 
%====================================================================

  \bigskip 
  \paragraph{Data.} $Y_k =$ number of bat calls during the $k$-th time bin.

  \bigskip \pause
  \paragraph{Markov switching Hawkes process model.} In discrete time:
  \begin{itemize}
    \setlength{\itemsep}{.75\baselineskip}
    \item Hidden path $(Z_k)_{k \geq 1} =$ homogeneous Markov chain with $Q$ states
    $$
    (Z_k)_{k \geq 1} \sim MC_Q(\nu, \pi)
    $$
    $\nu =$ intial distribution, $\pi =$ transition matrix;
    \item Observed counts: for $k\geq 1$ and
      $$
      \left(Y_k \mid (Y_\ell)_{\ell \leq k-1}, \emphase{Z_k=q}\right) 
      \sim \mathcal{P}\left(\emphase{\mu_q} + \alpha \sum_{\ell = 1}^{k-1} \beta^{\ell-1}Y_{k-\ell} \right);
      $$  
    \item Model parameters: $\theta = (\nu, \pi, \mu, \alpha, \beta)$
  \end{itemize}
  
  \bigskip \pause
  \paragraph{Proposition: Identifiability \refer{BoR25}\footnote{The proof does not rely on \refer{AMR09}}.}
  \begin{itemize}
    \setlength{\itemsep}{.75\baselineskip}
    \item The model parameter $\theta$ is identifiable from the joint distribution $p_\theta(Y_1, Y_2, Y_3)$. \\
    \textcolor{gray}{($\theta \neq \theta' \Rightarrow p_\theta(\cdot, \cdot, \cdot) \neq p_{\theta'}(\cdot, \cdot, \cdot)$)} 
  \end{itemize}

}

%====================================================================
\frame{\frametitle{Markovian representation (homogeneous case)} 

  \paragraph{Homogeneous discrete-time Hawkes process $Y = \{Y_k\}_{k \geq 1}$.} 
  $$
  Y_k \mid (Y_\ell)_{\ell \leq k-1} \sim \mathcal{P}\left(\mu + \alpha \sum_{\ell = 1}^{k-1}\beta^{\ell-1} Y_{k-\ell} \right)
  $$
  \bigskip
  $(Y_k)_{k\geq 1}$ is not a Markov chain (because of infinite memory).

  \bigskip \pause
  \paragraph{Markovian representation.}
  \begin{itemize}
    \item Define
    $$
    U_1 = 0, \qquad \qquad 
    U_k = \alpha \sum_{\ell = 1}^k \beta^{\ell-1} Y_{k-\ell}, 
    $$
    \item \pause then, for $k \geq 1$ \textcolor{gray}{(with $U_0 = Y_0 = 0$)}
    $$
    U_k = {\alpha Y_{k-1} + \beta U_{k-1}}, 
    \qquad \qquad 
    Y_k \mid U_k \sim \mathcal{P}(\mu + U_k).
    $$
  \end{itemize}
  \pause 
  \medskip
  so $\left((Y_k,U_k)\right)_{k \geq 1}$ forms a Markov chain.
  
}

%====================================================================
\frame{\frametitle{Markovian representation (Markov switching case)} 

  \paragraph{Markov switching Hawkes process model.} Can be rephrased as
  $$
  \left(Y_k \mid (Y_\ell)_{\ell \leq k-1}, \emphase{Z_k=q}\right) 
  \sim \mathcal{P}\left(\emphase{\mu_q} + U_k \right)
  $$  
  with
  $$
  U_1 = 0, \qquad \qquad 
  U_k = \alpha \sum_{\ell = 1}^k \beta^{\ell-1} Y_{k-\ell}, 
  $$      
  
  \bigskip \pause
  \paragraph{Consequence.}
  \begin{itemize}
    \setlength{\itemsep}{.75\baselineskip}
    \item The model is a regular Hidden Markov Model (HMM)
    \item With graphical model
    \begin{figure}
      \begin{centering}
        \begin{tikzpicture}
\node[] (Zt_2) at (-\edgeunit, \edgeunit) {}; 
\node[] (Zt_1) at (0, \edgeunit) {$Z_{k-1}$}; 
\node[] (Zt) at (\edgeunit, \edgeunit) {$Z_{k}$}; 
\node[] (Zt1) at (2*\edgeunit, \edgeunit) {$Z_{k+1}$}; 
\node[] (Zt2) at (3*\edgeunit, \edgeunit) {}; 
\node[] (Ut_1) at (-0.5*\edgeunit, 0.5*\edgeunit) {$U_{k-1}$}; 
\node[] (Ut) at (0.5*\edgeunit, 0.5*\edgeunit) {$U_{k}$}; 
\node[] (Ut1) at (1.5*\edgeunit, 0.5*\edgeunit) {$U_{k+1}$}; 
\node[] (Ut2) at (2.5*\edgeunit, 0.5*\edgeunit) {$U_{k+2}$}; 
\node[] (Ut3) at (3.5*\edgeunit, 0.5*\edgeunit) {}; 
\node[] (Yt_2) at (-\edgeunit, 0) {}; 
\node[] (Yt_1) at (0, 0) {$Y_{k-1}$}; 
\node[] (Yt) at (\edgeunit, 0) {$Y_{k}$}; 
\node[] (Yt1) at (2*\edgeunit, 0) {$Y_{k+1}$}; 
\node[] (Yt2) at (3*\edgeunit, 0) {}; 

\draw[->,dashed] (Zt_2) -- (Zt_1); \draw[->] (Zt_1) -- (Zt); \draw[->] (Zt) -- (Zt1); \draw[->,dashed] (Zt1) -- (Zt2);
\draw[->] (Zt_1) -- (Yt_1); \draw[->] (Zt) -- (Yt); \draw[->] (Zt1) -- (Yt1);
\draw[->] (Ut_1) -- (Yt_1); \draw[->] (Ut) -- (Yt); \draw[->] (Ut1) -- (Yt1);
\draw[->] (Ut_1) -- (Ut); \draw[->] (Ut) -- (Ut1); \draw[->] (Ut1) -- (Ut2); \draw[->,dashed] (Ut2) -- (Ut3);
\draw[->,dashed] (Yt_2) -- (Ut_1); \draw[->] (Yt_1) -- (Ut); \draw[->] (Yt) -- (Ut1); \draw[->] (Yt1) -- (Ut2);
\end{tikzpicture}
 \\
        $(Z_k)_{k \geq 1} =$ \emphase{hidden path}, 
        $\quad (U_k)_{k \geq 1} =$ memory, 
        $\quad (Y_k)_{k \geq 1} =$ observed process.
      \end{centering}
    \end{figure}
  \end{itemize}
}

%====================================================================
\frame{\frametitle{Inference}

  \bigskip
  \paragraph{Maximum likelihood inference:} 
  $
  \widehat{\theta} = \argmax_\theta \log p_\theta(Y)
  $

  \bigskip \bigskip \pause
  \paragraph{EM algorithm for HMM:} \refer{DLR77,CMR05}
  $$
  \theta^{(h+1)} 
  = \underset{\text{\normalsize \emphase{M step}}}{\underbrace{\argmax_\theta}} \; \underset{\text{\normalsize \emphase{E step}}}{\underbrace{\Esp_{\theta^{(h)}}}}[\log p_\theta(Y, Z) \mid Y]
  $$
  \begin{itemize}
    \setlength{\itemsep}{.75\baselineskip}
    \item E step: Evaluate $Q(\theta \mid \theta^{(h)}) = \Esp_{\theta^{(h)}}[\log p_\theta(Y, Z) \mid Y]$ (forward-backward recursion)
    \item M step: Gradient descent, computing $\nabla_\theta Q(\theta \mid \theta^{(h)})$ by recursion
  \end{itemize}

  \bigskip \bigskip \pause
  \paragraph{Model selection.} Penalized likelihood
  \begin{align*}
    AIC_Q & = \log p_{\widehat{\theta}_Q}(Y) - D_Q, &
    BIC_Q & = \log p_{\widehat{\theta}_Q}(Y) - D_Q \frac{\log(\emphase{N})}2     
  \end{align*}
  with $D_Q =$ number of parameters $= 2 + Q^2$ and \emphase{$N =$ number of time bins}.
}

%====================================================================
\frame{ \frametitle{Simulation study (not shown)}
  
  \paragraph{Design.}
  \begin{itemize}
    \setlength{\itemsep}{.75\baselineskip}
    \item Simulate a \emphase{continuous time} Markov-switching Hawkes process
    \item With more or less events (control parameter $\lambda$)
    \item Then \emphase{discretise} with more or less bins (control parameter $N \; \propto$ nb events)
  \end{itemize}

  \bigskip \pause
  \paragraph{Conclusions.}
  \begin{itemize}
    \setlength{\itemsep}{.75\baselineskip}
    \item Inference more accurate when more signal (large $\lambda$)!!!
    \goto{back:HawkesFit}
    \item Inference more accurate with thinner discretization step (large $N$) \\
    But at the price of a higher computational cost  \goto{back:HawkesClassif}
    \item BIC does not capture the right number of states \\
    \textcolor{gray}{Sequences not simulated according to the model}
    \item AIC does, with reasonable signal ($\lambda$) and discretization ($N$) \\
    \textcolor{gray}{Blind to the simulation shift from the model?} \goto{back:HawkesAIC}
%     \item Clear advantage wrt Poisson HMM.
  \end{itemize} \label{sec:HawkesSimuls}
  
  \bigskip \pause
  \paragraph{Practical recommendations:}
  Take $N = 2 n$ \quad and \quad use AIC to choose $Q$.

}
    
%====================================================================
%====================================================================
\subsection{Bats calls sequences}
\frame{\frametitle{Bats calls sequences}} 
%====================================================================

%====================================================================
\frame{\frametitle{Vigie-chiro project} 

  \bigskip
  \paragraph{Data set.}
  \begin{itemize}
    \item Vigie-chiro project French participatory project to monitor bats echolocation calls (\url{https://www.vigienature.fr/fr/chauves-souris}).
    \item 2354 overnight recordings collected between October 2010 and January 2020 in 755 locations. 
    \item Restricted to sequences with at least 50 cries $\to$ 1555 time sequences. 
  \end{itemize}
  
  \bigskip \bigskip \pause
  \paragraph{Poisson vs Hawkes / Homogeneous vs HMM.} Best model according to AIC
  $$
  \begin{tabular}{r|cc|c}
    & Poisson & Hawkes & Total \\
    \hline
    Homogeneous & 34 & 353 & 387\\
    Hidden Markov  & 24 & \emphase{1144} & \emphase{1168} \\
    \hline
    Total & 58 & \emphase{1497} & 1555
  \end{tabular}
  $$
  \begin{itemize}
    \item Memory ($95\%$) and heterogeneity ($75\%$) are present in most sequences 
    \item Hawkes-HMM best fits almost 3 sequences out of 4.
  \end{itemize}

}

%====================================================================
\frame{ \frametitle{An example}

  $$
  \begin{tabular}{ccc}
    Hawkes HMM ($\widehat{Q} = 3$) & \qquad &
    Poisson HMM ($\widehat{Q} = 4$) \\ 
    ~ \\
    \includegraphics[width=.3\textwidth, trim=10 40 20 50, clip=]{\figchiro/Chiro-seq1776-N1048-Qmax5-classif-seg} &
    \qquad & 
    \includegraphics[width=.3\textwidth, trim=10 40 20 50, clip=]{\figchiro/Chiro-seq1776-N1048-Qmax5-classifP-seg} 
  \end{tabular}
  $$
  \bigskip
  \begin{itemize}
    \setlength{\itemsep}{.75\baselineskip}
    \item Interpretation of the states: \textcolor{red}{absence} of calls, transit and \textcolor{green}{foraging} (high call frequency)
    \item Hawkes-HMM state changes do not correspond to slope changes
    \item Poisson-HMM needs many state changes to account for self-excitation
  \end{itemize}
}
    
%====================================================================
\frame{ \frametitle{States and species}

  The number of bat species was also recorded each night
  
  $$
  \includegraphics[width=.45\textwidth]{\figchiro/boxplot_species_Coef2_allQ.pdf}
  $$
  \begin{itemize}
    \setlength{\itemsep}{.75\baselineskip}
    \item The number of states does not match the number of species
    \item More discussion to come with members of the Vigie-chiro project
  \end{itemize}

}
