%====================================================================
\frame{\frametitle{Summary} 

  \paragraph{Latent variable models.}
  \begin{itemize}
    \setlength{\itemsep}{.75\baselineskip}
    \item Latent variable models are ubiquite in statistiscal ecology.
    \item For various modelling purposes (inferring $Z$ is critical in Model 2, not in Models 1 and 3).
    \item Latent variables mays play different roles, from mechanistic to purely instrumental.
  \end{itemize}
  
  \bigskip \bigskip \pause
  \paragraph{Inference: No big picture.}
  \begin{itemize}
    \setlength{\itemsep}{.75\baselineskip}
    \item Dealing with latent variable yields specific difficulties.
    \item Ranging from trivial to intractable.
    \item Often model-dependant, requiring specific developments.
    \item Still some generic questions (replace EM with gradient ascent?).
  \end{itemize}
  
}

%====================================================================
\frame{\frametitle{Discussion (some home works ?)} \label{sec:discussion}

  \bigskip
  \paragraph{1 - Network motifs (plant-pollinator)}
  \begin{enumerate}[a -]
%     \setlength{\itemsep}{.75\baselineskip}
    \item No clear understanding of the information brought by each motif;
    \item BEDD is not consistent with the actual sampling process of the network;
    \item (Asymptotic) normality does not hold for the networks at hand \goto{back:asymNormality} \\
    (Could explain the poor power of the tests?).
  \end{enumerate}
  
  \bigskip \pause
  \paragraph{2 - Hawkes hidden Markov model (bats calls)}
  \begin{enumerate}[a -]
%     \setlength{\itemsep}{.75\baselineskip}
    \item The Markovian representation also holds for non-exponential kernels \goto{back:HawkesKernel};
    \item No theoretical problem to define a continuous time version of the proposed model \\
    (but many practical ones);
    \item A proper model selection criterion accounting for the discretization step is still needed.
  \end{enumerate}
  
  \bigskip \pause
  \paragraph{3 - Poisson log-normal (species abundances)}
  \begin{enumerate}[a -]
%     \setlength{\itemsep}{.75\baselineskip}
    \item Account for the 'excess' of null abundances \refer{BCG24}
    \item Could we say more about the properties of VEM estimates?
    \item The expression of $p_\theta(Z \mid Y)$ is hugly, but the function is actually very regular \\
    $\to$ Could we 'learn' a deterministic transformation allowing, say, to sample from it?
  \end{enumerate}
  
}

