\documentclass{beamer}

% Beamer style
%\usetheme[secheader]{Madrid}
\usetheme{CambridgeUS}
\usecolortheme[rgb={0.65,0.15,0.25}]{structure}
%\usefonttheme[onlymath]{serif}
\beamertemplatenavigationsymbolsempty
%\AtBeginSubsection

% Packages
%\usepackage[french]{babel}
\usepackage[latin1]{inputenc}
\usepackage{color}
\usepackage{dsfont, stmaryrd}
\usepackage{amsmath, amsfonts, amssymb}
\usepackage{stmaryrd}
\usepackage{epsfig}
\usepackage{/Latex/astats}
%\usepackage[all]{xy}
\usepackage{graphicx}

% Commands
\definecolor{darkred}{rgb}{0.65,0.15,0.25}
\newcommand{\emphase}[1]{\textcolor{darkred}{#1}}
\newcommand{\refer}[1]{\textcolor{blue}{\sl \cite{#1}}}

% Symbols
\newcommand{\Abf}{{\bf A}}
\newcommand{\BIC}{\text{BIC}}
\newcommand{\dd}{\text{d}}
\newcommand{\Esp}{\mathbb{E}}
\newcommand{\Ebf}{{\bf E}}
\newcommand{\ICL}{\text{ICL}}
\newcommand{\Cov}{\mathbb{C}\text{ov}}
\newcommand{\Var}{\mathbb{V}}
\newcommand{\pen}{\text{pen}}
\newcommand{\Hcal}{\text{H}}
\newcommand{\Lcal}{\mathcal{L}}
\newcommand{\Mcal}{\mathcal{M}}
\newcommand{\Ncal}{\mathcal{N}}
\newcommand{\Ocal}{\mathcal{O}}
\newcommand{\Pbf}{{\bf P}}
\newcommand{\Pcal}{\mathcal{P}}
\newcommand{\Vcal}{\mathcal{V}}
\newcommand{\Tbf}{{\bf T}}
\newcommand{\Ubf}{{\bf U}}
\newcommand{\Ybf}{{\bf Y}}
\newcommand{\Zbf}{{\bf Z}}
\newcommand{\Pibf}{\mbox{\mathversion{bold}{$\Pi$}}}
\newcommand{\mubf}{\mbox{\mathversion{bold}{$\mu$}}}
\newcommand{\thetabf}{\mbox{\mathversion{bold}{$\theta$}}}


%====================================================================
\title{Sequence classification \& Mixture model}

\author[S. Robin]{Ph. Randrianomenjanahary, E. Lebarbier, S. Robin, F.
  Picard}

\institute[AgroParisTech / INRA]{AgroParisTech / INRA \\
  \bigskip
  \begin{tabular}{ccccc}
    \epsfig{file=../Figures/LogoINRA-Couleur.ps, width=2.5cm} &
    \hspace{.5cm} &
    \epsfig{file=../Figures/logagroptechsolo.eps, width=3.75cm} &
    \hspace{.5cm} &
    \epsfig{file=../Figures/Logo-SSB.eps, width=2.5cm} \\
  \end{tabular} \\
  \bigskip
  }

\date{MetaSoil, Evry, October 2009}
%====================================================================

%====================================================================
%====================================================================
\begin{document}
%====================================================================
%====================================================================

%====================================================================
\frame{\titlepage}
%====================================================================

% %====================================================================
% \frame{ \frametitle{Outline}
% %====================================================================
  
%   \setcounter{tocdepth}{1}
%   \tableofcontents
% %  \tableofcontents[pausesections]
%   }

%====================================================================
\section{Sequence classification}
\subsection{Problem}
\frame{ \frametitle{Sequence classification}
%==================================================================== 
  \emphase{Original problem: Comparatic genomics.}
  \begin{itemize}
  \item Alignment of 3 {\sl E. coli} strain genome \\
    $\rightarrow$ definition of 'backbone' (common) and 'loop'
    (specific) sequences.
  \item Classification of loop sequences to trace back their origins 
  \end{itemize}

  \bigskip
  \emphase{Metagenomics \& next generation sequencing.}
  \begin{itemize}
  \item Determine a typology of unaligned reads
  \end{itemize}

  \bigskip
  \emphase{2 families of classification methods.}
  \begin{itemize}
  \item Distance based: define a distance between sequence and gather
    'closest' distance into groups ($K$-means, hierarchical
    clustering)
  \item Model based: suppose that each sequence is generated according
    to a model with parameter specific to its group 
  \end{itemize}
  }

%====================================================================
\subsection{Mixture model}
\frame{ \frametitle{Mixture model}
%==================================================================== 
  \emphase{Data.} Sample of $n$ sequences $\{S_1, \dots, S_n\}$:
  $$
  S_i = (S_{i1}, \dots, S_{it}, \dots, S_{i\ell_i}), 
  \qquad \ell_i = \text{length of } S_i.
  $$

  \emphase{Mixture of Markov chains.}
  \begin{itemize}
  \item The $n$ sequences are spread into $K$ groups:
    $$
    Z_{ik} = 1 \text{ if } i \in k, \quad 0 \text{ otherwise}.
    $$
  \item Each sequence $i$ belongs to group $k$ ($k = 1, \dots K$) with
    probability $\pi_k$:
    $$
    \Pr\{i \in k\} = \Pr\{Z_{ik} = 1\} = \pi_k
    $$
    interpreted as the \emphase{prior probability} to belong to
    group $k$.
  \item Provided sequence $i$ belongs to group $k$, it is generated
    according to a Markov chain with parameter $\phi^k$:
    $$
    (S_i | Z_{ik}=1) \sim \text{MC}(\phi^k)
    $$
  \end{itemize}
  }

%====================================================================
\subsection{Mixture model}
\frame{ \frametitle{Mixture model}
%==================================================================== 
  \emphase{Markov chains.} Sequence $S$ is generated according to
  MC($\phi$) iff
  $$
  \Pr\{S_t = s_t | S_{t-m}=s_{t-m}, \dots, S_{t-1}=s_{t-1}\} =
    \phi(s_{y-m}, \dots s_{t-1}; s_t).
  $$
  \begin{itemize}
  \item $m$ is the \emphase{order} of the Markov chain.
  \item $\phi(\cdot; \cdot)$ are the \emphase{transition
      probabilities}. They fit the frequencies of the ($m+1$)-mers:
    $$
    \widehat{\phi}({\tt gc}; {\tt a}) = \frac{N({\tt gca})}{N({\tt gc})}.
    $$
  \end{itemize}

  \bigskip
  \emphase{Interpretation.} Markov chains of order $m$ account for the
  \emphase{sequence contents in ($m+1$)-mers}, e.g
  \begin{itemize}
  \item M0 is fitted to the nucleotide frequencies;
  \item M2 is fitted to the codon frequencies;
  \item M5 is fitted to the di-codon frequencies.
  \end{itemize}
  }

%====================================================================
\section{Statistical inference}
\subsection{E-M algorithm}
\frame{ \frametitle{Statistical inference}
%==================================================================== 
  {Mixture models} are \emphase{incomplete data models} since we miss
  the group to which each sequence belongs.

  \bigskip
  \emphase{E-M algorithm:} provides maximum likelihood estimates.
  \begin{itemize}
  \item \emphase{E-step:} calculates the probability for eahc sequence
    to belong to each class:
    $$
    \tau_{ik} = \Pr\{Z_{ik} = 1 | S_i\}
    $$
    interpreted as the \emphase{posterior probability} to belong to
    group $k$.
  \item \emphase{M-step:} using the $\tau_{ik}$'s as estimates of the
    $Z_{ik}$'s, estimates $\widehat{\pi}_k$ and $\widehat{\phi}^k$ can
    easily be derived.
  \end{itemize}

  \bigskip
  \emphase{Classification.} The $\tau_{ik}$ can be used to perform
  'fuzzy' classification, or 'maximum a posteriori' (MAP)
  classification:
  $$
  MAP_i = \arg\max_k \tau_{ik}.
  $$
  }

%====================================================================
\subsection{Some issues}
\frame{ \frametitle{Some issues}
%==================================================================== 
  \emphase{Model selection.} The order $m$ of the Markov chain and the
  number of group $K$ have to be chosen in some way.
  \begin{itemize}
  \item The order $m$ can be fixed according to biological
    considerations (see M0, M2, M5).
  \item The likelihood of the model $\widehat{P}(S|K)$ necessarily increases
    with $K$ (so the optimum would be $K = n$...). \\
    $\rightarrow$ Penalized criterion can be used:
    $$
    BIC(K) = \log \widehat{P}(S|K) - \frac12 \log(\text{\# data}) \times
    (\text{\# parameters}).
    $$
  \item \emphase{Number of data =} number of sequences or number of
    nucleotides?
  \item \emphase{Number of parameters =} total number of parms, or
    some number of efficient parms?
  \end{itemize}
  }

%====================================================================
\section{Some results}
\subsection{Sequence classification}
\frame{ \frametitle{Some results}
%==================================================================== 
  \emphase{Sequence classification for comparative genomics.} Cross
  classification according the length (L) and the di-codon content
  (MC).
  
  {\small
    \begin{center}
      \begin{tabular}{lc|ccccc|c}
        & length (bp) & MC1 & MC2 & MC3   & MC4  & MC5 & \\
        \hline
        L1 & $[2417;40120]$ & 2 (IS3) & 5 (rhs) & 0   & 34  & 36  & 77\\
        L2 & $[335:2270]$   & 5 (IS) & 1 (near rhs) & 13  & 47  &
        90  & 156\\ 
        L3 & $[43;323]$     & 0 & 4 & 116 & 106 & 139 & 365\\
        L4 & $[20;42]$      & 4 & 3 & 58  & 96  & 128 & 289\\
        \hline & & 11 & 13 & 187 & 283 & 393
      \end{tabular}
    \end{center}
    }
  
  \emphase{Ex.:} BIME's (repeated elements) are clustered in MC3. The
  transition matrix $\phi^3$ reveals characteristics 6-mers that
  corresponds to palindromic units.
  }

%====================================================================
\subsection{Comparison to reference sequence}
\frame{ \frametitle{Comparison to reference sequence (1/2)}
%==================================================================== 
  \emphase{Problem.} Identify sequences similar to a given one
  ($S^*$).
  
  \bigskip 
  \emphase{Similarity measure.}
  \begin{itemize}
  \item \emphase{Distance-based:} $d(S_i, S^*)$ should be small when
    $S_i$ is similar to $S^*$.
  \item \emphase{Mixture:} Assuming that $S^*$ belongs to group 0. The
    similarity can be measured with $d(S_i, S^*) = 1 - \tau_{i0}$. \\
    (Could be averaged to avoid the choice of the number of groups.)
  \end{itemize}

  \bigskip 
  \emphase{Comparative study.}
  \begin{itemize}
  \item For a given threshold $d^*$, one can predict if each sequence
    is similar ($d(S_i, S^*) < d^*$) to the reference or not.
  \item For each  threshold $d^*$, one can then compute False Positive
    (FP) and False Negative (FN) rates.
  \end{itemize}
  }

%====================================================================
\frame{ \frametitle{Comparison to reference sequence (2/2)}
%==================================================================== 
  \emphase{Data from Wu \& al., 01.}
  \begin{itemize}
  \item 20 sequences similar to the reference (HSLIPAS)
  \item 19 heterogenous sequences, different from HSLIPAS
  \end{itemize}
  $$
  \begin{tabular}{cc}
    \begin{tabular}{c}
      \epsfig{file=../FIGURES/Rapport-Fig41.ps, width=4.5cm, height=4.5cm,
        clip=} 
    \end{tabular}
    &
    \begin{tabular}{c}
      \begin{tabular}{ll}
        Distance & Surface \\
        \hline
        MC(2) & 1.0000 \\
        rre32 & 0.7316 \\
        wre32 & 0.8395 \\
        S132 & 0.7289 \\
        euclid3 & 0.7394 \\
        cos3 & 0.7211 \\
        S232 & 0.7342 \\
        KL3 & 0.7473 \\
      \end{tabular}
    \end{tabular}
  \end{tabular}
  $$
  \emphase{Important remark:} No cross-validation for distance based
  methods.  
  }

%====================================================================
\section{Connexion with other projects/actions}
\subsection{ANR CBME: Estimation of abundance}
\frame{ \frametitle{ANR CBME: Estimation of abundance}
%==================================================================== 
  \emphase{New topic:}
  \begin{itemize}
  \item Post-doc of S. Li-Thiao-T� recently started (ANR CBME);
  \item Invitation of J. Bunge in Paris last week.
  \end{itemize}

  \bigskip
  \emphase{General problem.}
  \begin{itemize}
  \item Data: In a given place, we observe the frequencies (i.e.
    number of individuals) of $c$ species: $X_1, X_2, \dots, X_c$.
  \item Question: How many species actually live in this place?
  \end{itemize}

  \bigskip
  \emphase{Statistical problem: .}
  \begin{itemize}
  \item How to infer the number of unobserved species $N_0$ from the
    distribution of $X_1, X_2, \dots, X_c$?
  \item What is the precision of the estimated total number of
    species:
    $$
    C = c + \widehat{N}_0.
    $$
  \end{itemize}
  }

%====================================================================
\subsection{COST Action: StatSeq}
\frame{ \frametitle{COST Action: StatSeq} 
%==================================================================== 
  \emphase{Objectif}
  \begin{itemize}
  \item Statistical challenges on the 1000E genome sequences in
    \emphase{plants}
  \item E.U. action to gather and promote interaction between E.U. labs
  \end{itemize}

  \bigskip
  \emphase{Addressed issues}
  \begin{itemize}
  \item Data normalization, sequencing biases
  \item Exp�rimental designs
  \item Alignment
  \item Genome re-sequencing
  \item Transcriptome, SNP, ChIPseq, RNAseq
  \end{itemize}
  }

%====================================================================
%====================================================================
\end{document}
%====================================================================
%====================================================================

