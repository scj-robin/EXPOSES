\documentclass[8pt]{beamer}

% Beamer style
%\usetheme[secheader]{Madrid}
% \usetheme{CambridgeUS}
\useoutertheme{infolines}
\usecolortheme[rgb={0.65,0.15,0.25}]{structure}
% \usefonttheme[onlymath]{serif}
\beamertemplatenavigationsymbolsempty
%\AtBeginSubsection

% Packages
%\usepackage[french]{babel}
\usepackage[latin1]{inputenc}
\usepackage{color}
\usepackage{xspace}
\usepackage{dsfont, stmaryrd}
\usepackage{amsmath, amsfonts, amssymb, stmaryrd}
\usepackage{epsfig}
\usepackage{tikz}
\usepackage{url}
% \usepackage{ulem}
\usepackage{/home/robin/LATEX/Biblio/astats}
%\usepackage[all]{xy}
\usepackage{graphicx}
\usepackage{xspace}

% Maths
% \newtheorem{theorem}{Theorem}
% \newtheorem{definition}{Definition}
\newtheorem{proposition}{Proposition}
% \newtheorem{assumption}{Assumption}
% \newtheorem{algorithm}{Algorithm}
% \newtheorem{lemma}{Lemma}
% \newtheorem{remark}{Remark}
% \newtheorem{exercise}{Exercise}
% \newcommand{\propname}{Prop.}
% \newcommand{\proof}{\noindent{\sl Proof:}\quad}
% \newcommand{\eproof}{$\blacksquare$}

% \setcounter{secnumdepth}{3}
% \setcounter{tocdepth}{3}
\newcommand{\pref}[1]{\ref{#1} p.\pageref{#1}}
\newcommand{\qref}[1]{\eqref{#1} p.\pageref{#1}}

% Colors : http://latexcolor.com/
\definecolor{darkred}{rgb}{0.65,0.15,0.25}
\definecolor{darkgreen}{rgb}{0,0.4,0}
\definecolor{darkred}{rgb}{0.65,0.15,0.25}
\definecolor{amethyst}{rgb}{0.6, 0.4, 0.8}
\definecolor{asparagus}{rgb}{0.53, 0.66, 0.42}
\definecolor{applegreen}{rgb}{0.55, 0.71, 0.0}
\definecolor{awesome}{rgb}{1.0, 0.13, 0.32}
\definecolor{blue-green}{rgb}{0.0, 0.87, 0.87}
\definecolor{red-ggplot}{rgb}{0.52, 0.25, 0.23}
\definecolor{green-ggplot}{rgb}{0.42, 0.58, 0.00}
\definecolor{purple-ggplot}{rgb}{0.34, 0.21, 0.44}
\definecolor{blue-ggplot}{rgb}{0.00, 0.49, 0.51}

% Commands
\newcommand{\backupbegin}{
   \newcounter{finalframe}
   \setcounter{finalframe}{\value{framenumber}}
}
\newcommand{\backupend}{
   \setcounter{framenumber}{\value{finalframe}}
}
\newcommand{\emphase}[1]{\textcolor{darkred}{#1}}
\newcommand{\comment}[1]{\textcolor{gray}{#1}}
\newcommand{\paragraph}[1]{\textcolor{darkred}{#1}}
\newcommand{\refer}[1]{{\small{\textcolor{gray}{{[\cite{#1}]}}}}}
\newcommand{\Refer}[1]{{\small{\textcolor{gray}{{[#1]}}}}}
\newcommand{\goto}[1]{{\small{\textcolor{blue}{[\#\ref{#1}]}}}}
\renewcommand{\newblock}{}

\newcommand{\tabequation}[1]{{\medskip \centerline{#1} \medskip}}
% \renewcommand{\binom}[2]{{\left(\begin{array}{c} #1 \\ #2 \end{array}\right)}}

% Variables 
\newcommand{\Abf}{{\bf A}}
\newcommand{\Beta}{\text{B}}
\newcommand{\Bcal}{\mathcal{B}}
\newcommand{\Bias}{\xspace\mathbb B}
\newcommand{\Cor}{{\mathbb C}\text{or}}
\newcommand{\Cov}{{\mathbb C}\text{ov}}
\newcommand{\cl}{\text{\it c}\ell}
\newcommand{\Ccal}{\mathcal{C}}
\newcommand{\cst}{\text{cst}}
\newcommand{\Dcal}{\mathcal{D}}
\newcommand{\Ecal}{\mathcal{E}}
\newcommand{\Esp}{\xspace\mathbb E}
\newcommand{\Espt}{\widetilde{\Esp}}
\newcommand{\Covt}{\widetilde{\Cov}}
\newcommand{\Ibb}{\mathbb I}
\newcommand{\Fcal}{\mathcal{F}}
\newcommand{\Gcal}{\mathcal{G}}
\newcommand{\Gam}{\mathcal{G}\text{am}}
\newcommand{\Hcal}{\mathcal{H}}
\newcommand{\Jcal}{\mathcal{J}}
\newcommand{\Lcal}{\mathcal{L}}
\newcommand{\Mt}{\widetilde{M}}
\newcommand{\mt}{\widetilde{m}}
\newcommand{\Nbb}{\mathbb{N}}
\newcommand{\Mcal}{\mathcal{M}}
\newcommand{\Ncal}{\mathcal{N}}
\newcommand{\Ocal}{\mathcal{O}}
\newcommand{\pt}{\widetilde{p}}
\newcommand{\Pt}{\widetilde{P}}
\newcommand{\Pbb}{\mathbb{P}}
\newcommand{\Pcal}{\mathcal{P}}
\newcommand{\Qcal}{\mathcal{Q}}
\newcommand{\qt}{\widetilde{q}}
\newcommand{\Rbb}{\mathbb{R}}
\newcommand{\Sbb}{\mathbb{S}}
\newcommand{\Scal}{\mathcal{S}}
\newcommand{\st}{\widetilde{s}}
\newcommand{\St}{\widetilde{S}}
\newcommand{\Tcal}{\mathcal{T}}
\newcommand{\todo}{\textcolor{red}{TO DO}}
\newcommand{\Ucal}{\mathcal{U}}
\newcommand{\Un}{\math{1}}
\newcommand{\Vcal}{\mathcal{V}}
\newcommand{\Var}{\mathbb V}
\newcommand{\Vart}{\widetilde{\Var}}
\newcommand{\Zcal}{\mathcal{Z}}

% Symboles & notations
\newcommand\independent{\protect\mathpalette{\protect\independenT}{\perp}}\def\independenT#1#2{\mathrel{\rlap{$#1#2$}\mkern2mu{#1#2}}} 
\renewcommand{\d}{\text{\xspace d}}
\newcommand{\gv}{\mid}
\newcommand{\ggv}{\, \| \, }
% \newcommand{\diag}{\text{diag}}
\newcommand{\card}[1]{\text{card}\left(#1\right)}
\newcommand{\trace}[1]{\text{tr}\left(#1\right)}
\newcommand{\matr}[1]{\boldsymbol{#1}}
\newcommand{\matrbf}[1]{\mathbf{#1}}
\newcommand{\vect}[1]{\matr{#1}} %% un peu inutile
\newcommand{\vectbf}[1]{\matrbf{#1}} %% un peu inutile
\newcommand{\trans}{\intercal}
\newcommand{\transpose}[1]{\matr{#1}^\trans}
\newcommand{\crossprod}[2]{\transpose{#1} \matr{#2}}
\newcommand{\tcrossprod}[2]{\matr{#1} \transpose{#2}}
\newcommand{\matprod}[2]{\matr{#1} \matr{#2}}
\DeclareMathOperator*{\argmin}{arg\,min}
\DeclareMathOperator*{\argmax}{arg\,max}
\DeclareMathOperator{\sign}{sign}
\DeclareMathOperator{\tr}{tr}
\newcommand{\ra}{\emphase{$\rightarrow$} \xspace}

% Hadamard, Kronecker and vec operators
\DeclareMathOperator{\Diag}{Diag} % matrix diagonal
\DeclareMathOperator{\diag}{diag} % vector diagonal
\DeclareMathOperator{\mtov}{vec} % matrix to vector
\newcommand{\kro}{\otimes} % Kronecker product
\newcommand{\had}{\odot}   % Hadamard product

% TikZ
\newcommand{\nodesize}{2em}
\newcommand{\edgeunit}{2.5*\nodesize}
\newcommand{\edgewidth}{1pt}
\tikzstyle{node}=[draw, circle, fill=black, minimum width=.75\nodesize, inner sep=0]
\tikzstyle{square}=[rectangle, draw]
\tikzstyle{param}=[draw, rectangle, fill=gray!50, minimum width=\nodesize, minimum height=\nodesize, inner sep=0]
\tikzstyle{hidden}=[draw, circle, fill=gray!50, minimum width=\nodesize, inner sep=0]
\tikzstyle{hiddenred}=[draw, circle, color=red, fill=gray!50, minimum width=\nodesize, inner sep=0]
\tikzstyle{observed}=[draw, circle, minimum width=\nodesize, inner sep=0]
\tikzstyle{observedred}=[draw, circle, minimum width=\nodesize, color=red, inner sep=0]
\tikzstyle{eliminated}=[draw, circle, minimum width=\nodesize, color=gray!50, inner sep=0]
\tikzstyle{empty}=[draw, circle, minimum width=\nodesize, color=white, inner sep=0]
\tikzstyle{blank}=[color=white]
\tikzstyle{nocircle}=[minimum width=\nodesize, inner sep=0]

\tikzstyle{edge}=[-, line width=\edgewidth]
\tikzstyle{edgebendleft}=[-, >=latex, line width=\edgewidth, bend left]
\tikzstyle{edgebendright}=[-, >=latex, line width=\edgewidth, bend right]
\tikzstyle{lightedge}=[-, line width=\edgewidth, color=gray!50]
\tikzstyle{lightedgebendleft}=[-, >=latex, line width=\edgewidth, bend left, color=gray!50]
\tikzstyle{lightedgebendright}=[-, >=latex, line width=\edgewidth, bend right, color=gray!50]
\tikzstyle{edgered}=[-, line width=\edgewidth, color=red]
\tikzstyle{edgebendleftred}=[-, >=latex, line width=\edgewidth, bend left, color=red]
\tikzstyle{edgebendrightred}=[-, >=latex, line width=\edgewidth, bend right, color=red]

\tikzstyle{arrow}=[->, >=latex, line width=\edgewidth]
\tikzstyle{arrowbendleft}=[->, >=latex, line width=\edgewidth, bend left]
\tikzstyle{arrowbendright}=[->, >=latex, line width=\edgewidth, bend right]
\tikzstyle{arrowred}=[->, >=latex, line width=\edgewidth, color=red]
\tikzstyle{arrowbendleftred}=[->, >=latex, line width=\edgewidth, bend left, color=red]
\tikzstyle{arrowbendrightred}=[->, >=latex, line width=\edgewidth, bend right, color=red]
\tikzstyle{arrowblue}=[->, >=latex, line width=\edgewidth, color=blue]
\tikzstyle{dashedarrow}=[->, >=latex, dashed, line width=\edgewidth]
\tikzstyle{dashededge}=[-, >=latex, dashed, line width=\edgewidth]
\tikzstyle{dashededgebendleft}=[-, >=latex, dashed, line width=\edgewidth, bend left]
\tikzstyle{lightarrow}=[->, >=latex, line width=\edgewidth, color=gray!50]

% \newcommand{\GMSBM}{/home/robin/RECHERCHE/RESEAUX/EXPOSES/1903-SemStat/}
% \newcommand{\figeconet}{/home/robin/RECHERCHE/ECOLOGIE/EXPOSES/1904-EcoNet-Lyon/Figs}
\newcommand{\fignet}{/home/robin/RECHERCHE/RESEAUX/EXPOSES/FIGURES}
\newcommand{\figeco}{/home/robin/RECHERCHE/ECOLOGIE/EXPOSES/FIGURES}
% \newcommand{\figbayes}{/home/robin/RECHERCHE/BAYES/EXPOSES/FIGURES}
% \newcommand{\figCMR}{/home/robin/Bureau/RECHERCHE/ECOLOGIE/CountPCA/sparsepca/Article/Network_JCGS/trunk/figs}
% \newcommand{\figtree}{/home/robin/RECHERCHE/BAYES/VBEM-IS/VBEM-IS.git/Data/Tree/Fig}
% \newcommand{\figDoR}{/home/robin/RECHERCHE/BAYES/VBEM-IS/VBEM-IS.git/Paper/JRSSC-V3/Figs}

\renewcommand{\nodesize}{1.75em}
\renewcommand{\edgeunit}{2.25*\nodesize}

%====================================================================
%====================================================================
\begin{document}
%====================================================================
%====================================================================

\title[R\'eseaux et variables latentes pour l'\'ecologie des communaut\'es]{Mod\`eles de r\'eseaux et mod\`eles \`a variables latentes \\
(pour l'\'ecologie des communaut\'es)}

\author{S. Robin}

\date[Oct'21, LPSM]{October 2021, LPSM}

\maketitle

%====================================================================
\frame{\frametitle{Community ecology}

{\sl A community is a group [\dots] of populations of [\dots] different species occupying the same geographical area at the same time }

\medskip
{\sl Community ecology [\dots] is the study of the interactions between species in communities [\dots].} \Refer{Wikipedia}

\bigskip \bigskip
\paragraph{Need for statistical models to}
\begin{itemize}
 \item \bigskip decipher / describe / evaluate environmental effects on species ({\sl abiotic} interactions) and between species interactions ({\sl biotic} interactions) \\ ~\\
 \ra \emphase{joint species distribution models}
 \item \bigskip describe / understand the organisation of species interaction networks \\ ~\\
  \ra \emphase{network models}
\end{itemize}

}

%====================================================================
%====================================================================
\section{Models with latent variables}
\frame{\frametitle{Outline} \tableofcontents[currentsection]}
%====================================================================

%====================================================================
\subsection{Two models}
%====================================================================
\frame{\frametitle{Joint species distribution models}

  \paragraph{Abundance data:} 
  \begin{itemize}
  \item $n$ sites, $p$ species, $d$ covariates (site descriptors)
  \item Abundance table: $Y = [Y_{ij}]$
  $$
  Y_{ij} = \text{number of individual from species $j$ observed in site $i$}
  $$
  \item Covariate table: $X = [x_{ik}]$
  $$
  x_{ik} = \text{value of descriptor $k$ for site $i$}
  $$
  \end{itemize}

  \bigskip \bigskip 
  \paragraph{Poisson log-normal model (PLN):} \refer{AiH89}
  \begin{itemize}
  \item $Z_i =$ latent vector associated with site $i$
  $$
  \{Z_i\}_{1 \leq i \leq n} \text{ iid}, \qquad 
  Z_i \sim \Ncal_p(0, \Sigma)
  $$
  \item $Y_{ij} = $ observed abundance for species $j$ in site $i$
  $$
  \{Y_{ij}\}_{1 \leq i \leq n, 1 \leq j \leq p} \text{ indep} \mid Z, \qquad 
  Y_{ij} \mid Z_{ij} \sim \Pcal(\exp(x_i^\intercal \beta_j + Z_{ij}))
  $$
  \item $\theta = (\beta = \text{ abiotic}, \Sigma = \text{ biotic})$
  \end{itemize}

}

%====================================================================
\frame{\frametitle{Illustration}

  \paragraph{Barents fish:}
  $n = 89$ sites, $p = 30$ species, $d = 4$ covariates (lat., long., depth, temp.)

  \begin{tabular}{rlrl}
  \begin{tabular}{c} $\widehat{\beta} = $ \end{tabular} & 
  \hspace{-.05\textwidth} \begin{tabular}{c} \includegraphics[width=.35\textwidth]{\figeco/BarentsFish-coeffAll-woIntercept}  \end{tabular} & 
%   \begin{tabular}{c} \qquad \end{tabular} &
  \begin{tabular}{c} $\widehat{\Sigma} = $ \end{tabular} &  
  \hspace{-.05\textwidth} \begin{tabular}{c} \includegraphics[width=.35\textwidth]{\figeco/BarentsFish-corrAll} \end{tabular}
  \end{tabular}

  \bigskip \bigskip 
  \begin{itemize}
   \item Several extensions (dimension reduction, network inference, clustering, discriminant analysis)
   \item \url{PLNmodels} R package \refer{CMR18a,CMR19,CMR21}
  \end{itemize}

}

%====================================================================
\frame{\frametitle{Models for interaction networks}
  \paragraph{Interaction data:} 
  \begin{itemize}
  \item $p$ species, $d$ covariates 
  \item Interaction matrix: $Y = [Y_{ij}]$
  $$
  Y_{ij} = \text{intensity of the interaction between species $i$ and $j$}
  $$
  \item Covariates: $x_{ij} = [x_{ij1}, \dots, x_{ijd}]$
  $$
  x_{ijk} = \text{value of descriptor $k$ for the species pair $(i, j)$}
  $$
  \end{itemize}
  
  \bigskip \bigskip 
  \paragraph{Stochastic block model (SBM):} \refer{HoL79} $K$ groups,
  \begin{itemize}
  \item $Z_i =$ latent variable indicating to which group node $i$ belongs to
  $$
  \{Z_i\}_{1 \leq i \leq n} \text{ iid}, \qquad 
  \Pr\{Z_i = k\} = {\pi_k}
  $$
  \item $Y_{ij} =$ observed number of interactions between species $i$ and $j$
  $$
  \{Y_{ij}\}_{1 \leq i \leq n, 1 \leq j \leq p} \text{ indep} \mid Z, \qquad 
  Y_{ij} \mid Z_i, Z_j \sim \Pcal(\exp(x_{ij}^\intercal {\beta} + \alpha_{Z_i Z_j}))
  $$
  \item ${\theta} = (\pi, \alpha, \beta)$ \qquad (and $K$)
  \end{itemize}
  
}

%====================================================================
\frame{\frametitle{Illustration}

  \begin{tabular}{cc}
    \hspace{-.04\textwidth}
    \begin{tabular}{p{.525\textwidth}}
      \paragraph{Data.}
      \begin{itemize}
        \setlength{\itemsep}{1.25\baselineskip}
        \item $n = 51$ tree species, 
        \item $Y_{ij} = $ number of fungal parasites shared by species $i$ and $j$,
        \item $x_{ij} =$ taxonomic distance btw species $i$ and $j$.
      \end{itemize}

    
      \bigskip \bigskip \pause
      \paragraph{Results.} \refer{MRV10}
      \begin{itemize}
        \setlength{\itemsep}{1.25\baselineskip}
        \item {$\widehat{\beta} = -.417$}
        \item $\widehat{K} = 4$ species groups \\
        (not due to taxonomic similarity)
      \end{itemize}

    \end{tabular}
    &
    \begin{tabular}{p{.45\textwidth}}
%       \includegraphics[height=.3\textwidth,width=.3\textwidth]{\fignet/Tree-ICL-SBMall}  \\
      \includegraphics[height=.35\textwidth,width=.35\textwidth]{\fignet/Tree-adjMat-SBMtaxo}
    \end{tabular}
  \end{tabular}
  
\bigskip \bigskip 
\paragraph{R packages:} \url{blockmodels} \refer{Leg16}, \url{sbm}.
  
}

%====================================================================
\subsection{Variational inference}
%====================================================================
\frame{\frametitle{A reminder on the EM algorithm}

  \paragraph{Maximum likelihood inference.}
  $$
  \widehat{\theta} = \arg\max_\theta \; \log p_\theta(Y)
  $$
  \ra no closed form for $\displaystyle{{p_\theta(Y) = \int \underset{\text{complete likelihood}}{\underbrace{p_\theta(Y, Z)}} \d Z}}$ in latent variable models

  \bigskip \bigskip
  \paragraph{Incomplete data models.} EM algorithm \refer{DLR77}
  $$
  \theta^{h+1} 
  = \underset{\text{\normalsize \emphase{M step}}}{\underbrace{\argmax_\theta}} \; \underset{\text{\normalsize \emphase{E step}}}{\underbrace{\Esp_{\theta^h}}}[\log p_\theta(Y, Z) \mid Y]
  $$

  \bigskip \bigskip
  \paragraph{Critical step = E step.} Evaluate $p_\theta(Z \mid Y)$
  \begin{itemize}
  \item easy: mixture models, Gaussian linear mixed-models, ...
  \item use a trick: hidden Markov models, phylogenetic or evolutionary models, ...
  \item intractable: many models, including PLN \& SBM
  \end{itemize}
  
}

%====================================================================
\frame{\frametitle{Variational approximation} 

  \paragraph{Twisted problem:} \refer{Jaa01,WaJ08,BKM17} maximize a lower bound of the log-likelihood:
  \begin{align*}
  J(Y; \theta, q)
  & = \log p_\theta(Y)
  - \underset{\text{\normalsize {VE step}}}{\underbrace{\emphase{KL(q(Z) \mid\mid p_\theta(Z \mid Y))}}} \\
  & = 
  \underset{\text{\normalsize {M step}}}{\underbrace{\emphase{\Esp_q (\log p_\theta(Y, Z))}}} 
  - \Esp_q(\log q(Z)) 
  \end{align*}
  where $q(\cdot) \simeq p_\theta(\cdot \mid Y)$ is chosen within the approximation class $\Qcal$.
  

  \bigskip \bigskip \pause
  \paragraph{Many avatars.} 
  \begin{itemize}
    \item Alternative divergences \refer{Min01,Min05}
    \item Extension to Bayesian inference w/o latent variables \refer{BeG03} 
  \end{itemize}   

  \bigskip
  \ra Reasonably easy to implement and computationally efficient

}

%====================================================================
\frame{\frametitle{Approximate distribution} 

  \paragraph{Critical choice:} $\Qcal$ needs to be 
  \begin{itemize}
  \item 'large' enough to include good approximations of $p_\theta(Z \mid Y)$ 
  \item 'small' enough to make the calculations tractable
  \end{itemize}

  \bigskip \bigskip
  \paragraph{Examples:} 
  \begin{itemize}
   \item PLN model for species abundance ($Z \sim \Ncal, Y\mid Z \sim \Pcal$):
   $$
   \Qcal := \{\text{multivariate Gaussian distributions}\}
   $$ \\ ~
   \item SBM model for networks ($Z \sim \Mcal, Y\mid Z \sim \Pcal$): \refer{DPR08,MRV10}
   $$
   \Qcal := \{\text{factorable discrete distributions}\}
   $$ 
   ('mean-field' approximation)
  \end{itemize}
  
}

%====================================================================
\frame{\frametitle{Statistical guarantees for variational estimates} \label{back:statGuarantees} 
  \paragraph{Statistical guaranties for variational inference.} 
  \begin{itemize}
  \item Very scarce \\
  \ra binary SBM without covariate \refer{CDP12,BCC13,MaM15}, specific PLN \refer{HOW11,WeM19}
  \item Many (positive) empirical results %\Refer{\#\ref{goto:vbemSBM}}
  \item No generic theoretical result about consistency, asymptotic normality, \dots
  \end{itemize}
 
  \bigskip \bigskip \pause
  \paragraph{VEM with statistical guarantees?}
  \begin{itemize}
  \item Make a theoretical analysis of
  $$
  \widehat{\theta}_{VEM} = \argmax_\theta \left(\argmax_{q \in \Qcal} J(Y; \theta, q) \right)
  $$
  \ra Very problem specific \refer{WeM19} \\ ~
  \item Use VEM to accelerate computationally demanding procedures for frequentist or Bayesian inference \refer{DoR21}
  \end{itemize}

}

%====================================================================
%====================================================================
\section{A model for exchangeable graphs}
\frame{\frametitle{Outline} \tableofcontents[currentsection]}
%====================================================================

%====================================================================
\subsection{Row-column exchangeable matrices}
%====================================================================
\frame{\frametitle{Row-column exchangeable matrix}

  \paragraph{Row-column exchangeabibility (RCE).} $Y = [Y_{ij}] = (m \times n)$ random matrix.
  $$
  Y \text{ is RCE} \qquad \Leftrightarrow \qquad
  \forall \sigma_1, \sigma_2: \quad \Lcal([Y_{ij}]) = \Lcal([Y_{\sigma_1(i)\sigma_2(j)}])
  $$
  
  \bigskip \bigskip \pause
  \paragraph{Example: Bipartite expected 'degree' distribution (BEDD).} 
  \begin{itemize}
    \setlength{\itemsep}{1.25\baselineskip}
    \item $f, g: [0, 1] \mapsto \Rbb^+$, with $\int f = \int g = 1$;
    \item $\{U_i\}_{1 \leq i \leq m}, \{V_j\}_{1 \leq j \leq n} \text{ iid } \sim \Ucal[0, 1]$;
    \item $\{Y_{ij}\}_{1 \leq i \leq m, 1 \leq j \leq n} \text{ independent } \mid \{U_i\}, \{V_j\}$, with
    $$
    Y_{ij} \mid U_i, V_j \sim 
      \left\{\begin{array}{ll}
        \Bcal(\rho \; f(U_i) \; g(V_j)) & \text{binary matrix} \\ ~
        \\
        \Pcal(\lambda \; f(U_i) \; g(V_j)) & \text{count matrix} \\ ~
        \\
        \Fcal(\theta, f(U_i), g(V_j)) & \text{general case}
      \end{array}\right.
    $$
  \end{itemize}

}

%====================================================================
\subsection{Analyzing bipartite network}
%====================================================================
%====================================================================
\frame{\frametitle{Bipartite species interaction networks}

  \begin{tabular}{cc}
    \hspace{-.04\textwidth}
    \begin{tabular}{p{.55\textwidth}}
      \paragraph{Species interactions:} 
      \begin{itemize}
      \setlength{\itemsep}{1.25\baselineskip}
      \item plant-pollinator (mutualistic)
      \item plant-herbivors (antagonistic)
      \end{itemize}

      \bigskip \bigskip 
      \onslide+<2->{
        \paragraph{Specialists vs generalists:} some insects interact with almost any plant, whereas some others interact only with few specific plants. \\
        (same for plants $\leftrightarrow$ insects).
      }
      
      \bigskip \bigskip 
      \onslide+<3->{
        \paragraph{$f$ encodes} the difference between specialist and generalist plants:
        $$
        \{f \equiv 1\} \quad \Leftrightarrow \quad \text{no generalist vs specialist}.
        $$
        (same for $g$ and insects)
        }
    \end{tabular}
    &
    \begin{tabular}{c}
      \includegraphics[width=.35\textwidth, trim=0 50 0 50]{\fignet/Zackenberg-1996_12-red-net} \\
      \includegraphics[width=.35\textwidth, angle=90, trim=0 50 0 50, clip]{\fignet/Zackenberg-1996_12-red-6node-orderedAdj}  
    \end{tabular}
  \end{tabular}
 
}

%====================================================================
\frame{\frametitle{Motif in bipartite networks}

  \begin{tabular}{cc}
    \hspace{-.04\textwidth}
    \begin{tabular}{p{.5\textwidth}}
      \paragraph{Motif count $N_s =$} number of occurrences of motif $s$ in the (binary) network $Y$.
      
      \onslide+<2->{
        \bigskip 
        \paragraph{Moments under BEDD.} $\Esp N_s$ and $\Var N_s$ are functions of $\rho$, $F_k = \int f^k$ and $G_k = \int g^k$.
      }
      
      \onslide+<3->{
        \bigskip 
        \paragraph{Estimates} of $\rho$, $F_k$ and $G_k$ can be derived from the counts of 'star' motifs.
      }
      
      \onslide+<4->{
        \bigskip 
        \paragraph{Asymptotic normality.} Under sparsity conditions $\rho \propto m^{-a} n^{-b}$ 
        $$
        \left(N_s - \widehat{\Esp}N_s\right) \left/ \sqrt{\widehat{\Var}N_s}\right.
        \overset{\Lcal}{\longrightarrow}
        \Ncal(0, 1).
        $$
      }
    \end{tabular}
    &
    \begin{tabular}{c}
      \includegraphics[width=.4\textwidth]{\fignet/SCB19-Oikos-Fig3-6motifs} \\
      \refer{SCB19,SSS19}
    \end{tabular}
  \end{tabular}
  
  \onslide+<5->{
  \bigskip 
  \paragraph{Tests} \refer{OLR21} can be derived for
  \begin{align*}
    \text{Goodness of fit:} & & H_0 & = \{Y \sim BEDD\} \\
    \text{Absence of imbalance:} & & H_0 & = \{f \equiv 1\}, & 
    \text{Network comparison:} & & H_0 & = \{f^A = f^B\}
  \end{align*}
  }

}

%====================================================================
\frame{ \frametitle{$U$-statistics for weighted RCE networks}
  
  \paragraph{Order-2 $U$-statistic.} Given a kernel $h : \Rbb^4 \mapsto \Rbb$, define
  $$
  U_{m, n} = \binom{m}{2}^{-1} \binom{n}{2}^{-1} 
  \sum_{1 \leq i_1 < i_2 \leq m} \sum_{1 \leq j_1 < j_2 \leq n} 
  h(Y_{i_1j_1}, Y_{i_1j_2}, Y_{i_2j_1}, Y_{i_2j_2}).
  $$
  Tam Le Minh's PhD \refer{LeM21}: asymptotic normality of $U_{m, n}$ when $Y$ is {\sl separable} RCE\footnote{plus some technical conditions}. \\
  \medskip
  \ra Order-2 $U$-statistics can be used for estimation or tests.
  
  \bigskip \bigskip \pause
  \paragraph{Example.} For $Y_{ij} \mid U_i, V_j \sim \Pcal(\lambda f(U_i) g(V_j))$:
  \begin{align*}
    h_1 & = \frac14 (Y_{i_1j_1} + Y_{i_1j_2} + Y_{i_2j_1} + Y_{i_2j_2}) & 
    \Rightarrow \quad \Esp h_1 & = \lambda, \\
    h_2 & = \frac12 (Y_{i_1j_1}Y_{i_1j_2} + Y_{i_2j_1}Y_{i_2j_2}) & 
    \Rightarrow \quad \Esp h_2 & = \lambda^2 F_2, \\
    h_3 & = \frac12 (Y_{i_1j_1}Y_{i_2j_2} + Y_{i_2j_1}Y_{i_1j_2}) & 
    \Rightarrow \quad \Esp h_3 & = \lambda^2.
  \end{align*}
  
}
  
%====================================================================
\frame{ \frametitle{Variance degeneracy for some kernels}

  \paragraph{Technical conditions} impose that $\Var U_{m, n}$ is controlled by the 'leading' covariances
  \begin{align*}
    \Cov(h(Y_{i_1j_1}, Y_{i_1j_2}, Y_{i_2j_1}, Y_{i_2j_2}), h(Y_{i_1j_3}, Y_{i_1j_4}, Y_{i_3j_3}, Y_{i_3j_3})) & & & (\text{one common row}) \\
    \text{and} \quad \Cov(h(Y_{i_1j_1}, Y_{i_1j_2}, Y_{i_2j_1}, Y_{i_2j_2}), h(Y_{i_3j_1}, Y_{i_3j_3}, Y_{i_4j_1}, Y_{i_4j_3})) & & & (\text{one common column})  
  \end{align*}
  (if not: wrong scaling for the TCL).
  
  \bigskip \bigskip \pause
  \paragraph{Not only a technical issue.} Suppose we want to test $H_0 = \{F_2 = 1\}$ (no imbalance among plants), a natural kernel is
  \begin{align*}
    h & = h_2 - h_3 
    = \frac12 (Y_{i_1j_1}Y_{i_1j_2} + Y_{i_2j_1}Y_{i_2j_2} - Y_{i_1j_1}Y_{i_2j_2} - Y_{i_2j_1}Y_{i_1j_2}) \\
    \Rightarrow \quad \Esp h & = \lambda^2(F_2 - 1) \overset{H_0}{=} 0
  \end{align*}
  but then, both leading covariances are 0...

}

%====================================================================
%====================================================================
\backupbegin 
\section*{Backup}
%====================================================================
\frame[allowframebreaks]{\frametitle{References} 

  {\tiny 
  \bibliography{/home/robin/Biblio/BibGene}
  \bibliographystyle{alpha}
  }

}

%====================================================================
\backupend 

%====================================================================
%====================================================================
\end{document}
%====================================================================
%====================================================================

%====================================================================
\frame{\frametitle{}
}

  \begin{tabular}{cc}
    \hspace{-.04\textwidth}
    \begin{tabular}{p{.5\textwidth}}
    \end{tabular}
    &
    \begin{tabular}{p{.45\textwidth}}
    \end{tabular}
  \end{tabular}

