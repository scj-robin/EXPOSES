\documentclass[9pt]{beamer}

% Beamer style
%\usetheme[secheader]{Madrid}
% \usetheme{CambridgeUS}
\useoutertheme{infolines}
\usecolortheme[rgb={0.65,0.15,0.25}]{structure}
% \usefonttheme[onlymath]{serif}
\beamertemplatenavigationsymbolsempty
%\AtBeginSubsection

% Packages
%\usepackage[french]{babel}
\usepackage[latin1]{inputenc}
\usepackage{color}
\usepackage{xspace}
\usepackage{dsfont, stmaryrd}
\usepackage{amsmath, amsfonts, amssymb}
\usepackage{epsfig}
\usepackage{array}
\usepackage{tikz}
\usepackage{url}
\usepackage{/home/robin/LATEX/Biblio/astats}
%\usepackage[all]{xy}
\usepackage{graphicx}

% Maths
% \newtheorem{theorem}{Theorem}
% \newtheorem{definition}{Definition}
\newtheorem{proposition}{Proposition}
% \newtheorem{assumption}{Assumption}
% \newtheorem{algorithm}{Algorithm}
% \newtheorem{lemma}{Lemma}
% \newtheorem{remark}{Remark}
% \newtheorem{exercise}{Exercise}
% \newcommand{\propname}{Prop.}
% \newcommand{\proof}{\noindent{\sl Proof:}\quad}
% \newcommand{\eproof}{$\blacksquare$}

% \setcounter{secnumdepth}{3}
% \setcounter{tocdepth}{3}
\newcommand{\pref}[1]{\ref{#1} p.\pageref{#1}}
\newcommand{\qref}[1]{\eqref{#1} p.\pageref{#1}}

% Colors : http://latexcolor.com/
\definecolor{darkred}{rgb}{0.65,0.15,0.25}
\definecolor{darkgreen}{rgb}{0,0.4,0}
\definecolor{darkred}{rgb}{0.65,0.15,0.25}
\definecolor{amethyst}{rgb}{0.6, 0.4, 0.8}
\definecolor{asparagus}{rgb}{0.53, 0.66, 0.42}
\definecolor{applegreen}{rgb}{0.55, 0.71, 0.0}
\definecolor{awesome}{rgb}{1.0, 0.13, 0.32}
\definecolor{blue-green}{rgb}{0.0, 0.87, 0.87}
\definecolor{red-ggplot}{rgb}{0.52, 0.25, 0.23}
\definecolor{green-ggplot}{rgb}{0.42, 0.58, 0.00}
\definecolor{purple-ggplot}{rgb}{0.34, 0.21, 0.44}
\definecolor{blue-ggplot}{rgb}{0.00, 0.49, 0.51}

% Commands
\newcommand{\backupbegin}{
   \newcounter{finalframe}
   \setcounter{finalframe}{\value{framenumber}}
}
\newcommand{\backupend}{
   \setcounter{framenumber}{\value{finalframe}}
}
\newcommand{\emphase}[1]{\textcolor{darkred}{#1}}
\newcommand{\comment}[1]{\textcolor{gray}{#1}}
\newcommand{\paragraph}[1]{\textcolor{darkred}{#1}}
\newcommand{\refer}[1]{{\small{\textcolor{gray}{{[\cite{#1}]}}}}}
\newcommand{\Refer}[1]{{\small{\textcolor{gray}{{[#1]}}}}}
\newcommand{\goto}[1]{{\small{\textcolor{blue}{[\#\ref{#1}]}}}}
\renewcommand{\newblock}{}

\newcommand{\tabequation}[1]{{\medskip \centerline{#1} \medskip}}
% \renewcommand{\binom}[2]{{\left(\begin{array}{c} #1 \\ #2 \end{array}\right)}}

% Variables 
\newcommand{\Abf}{{\bf A}}
\newcommand{\Beta}{\text{B}}
\newcommand{\Bcal}{\mathcal{B}}
\newcommand{\Bias}{\xspace\mathbb B}
\newcommand{\Cor}{{\mathbb C}\text{or}}
\newcommand{\Cov}{{\mathbb C}\text{ov}}
\newcommand{\cl}{\text{\it c}\ell}
\newcommand{\Ccal}{\mathcal{C}}
\newcommand{\cst}{\text{cst}}
\newcommand{\Dcal}{\mathcal{D}}
\newcommand{\Ecal}{\mathcal{E}}
\newcommand{\Esp}{\xspace\mathbb E}
\newcommand{\Espt}{\widetilde{\Esp}}
\newcommand{\Covt}{\widetilde{\Cov}}
\newcommand{\Ibb}{\mathbb I}
\newcommand{\Fcal}{\mathcal{F}}
\newcommand{\Gcal}{\mathcal{G}}
\newcommand{\Gam}{\mathcal{G}\text{am}}
\newcommand{\Hcal}{\mathcal{H}}
\newcommand{\Jcal}{\mathcal{J}}
\newcommand{\Lcal}{\mathcal{L}}
\newcommand{\Mt}{\widetilde{M}}
\newcommand{\mt}{\widetilde{m}}
\newcommand{\Nbb}{\mathbb{N}}
\newcommand{\Mcal}{\mathcal{M}}
\newcommand{\Ncal}{\mathcal{N}}
\newcommand{\Ocal}{\mathcal{O}}
\newcommand{\pt}{\widetilde{p}}
\newcommand{\Pt}{\widetilde{P}}
\newcommand{\Pbb}{\mathbb{P}}
\newcommand{\Pcal}{\mathcal{P}}
\newcommand{\Qcal}{\mathcal{Q}}
\newcommand{\qt}{\widetilde{q}}
\newcommand{\Rbb}{\mathbb{R}}
\newcommand{\Sbb}{\mathbb{S}}
\newcommand{\Scal}{\mathcal{S}}
\newcommand{\st}{\widetilde{s}}
\newcommand{\St}{\widetilde{S}}
\newcommand{\Tcal}{\mathcal{T}}
\newcommand{\todo}{\textcolor{red}{TO DO}}
\newcommand{\Ucal}{\mathcal{U}}
\newcommand{\Un}{\math{1}}
\newcommand{\Vcal}{\mathcal{V}}
\newcommand{\Var}{\mathbb V}
\newcommand{\Vart}{\widetilde{\Var}}
\newcommand{\Zcal}{\mathcal{Z}}

% Symboles & notations
\newcommand\independent{\protect\mathpalette{\protect\independenT}{\perp}}\def\independenT#1#2{\mathrel{\rlap{$#1#2$}\mkern2mu{#1#2}}} 
\renewcommand{\d}{\text{\xspace d}}
\newcommand{\gv}{\mid}
\newcommand{\ggv}{\, \| \, }
% \newcommand{\diag}{\text{diag}}
\newcommand{\card}[1]{\text{card}\left(#1\right)}
\newcommand{\trace}[1]{\text{tr}\left(#1\right)}
\newcommand{\matr}[1]{\boldsymbol{#1}}
\newcommand{\matrbf}[1]{\mathbf{#1}}
\newcommand{\vect}[1]{\matr{#1}} %% un peu inutile
\newcommand{\vectbf}[1]{\matrbf{#1}} %% un peu inutile
\newcommand{\trans}{\intercal}
\newcommand{\transpose}[1]{\matr{#1}^\trans}
\newcommand{\crossprod}[2]{\transpose{#1} \matr{#2}}
\newcommand{\tcrossprod}[2]{\matr{#1} \transpose{#2}}
\newcommand{\matprod}[2]{\matr{#1} \matr{#2}}
\DeclareMathOperator*{\argmin}{arg\,min}
\DeclareMathOperator*{\argmax}{arg\,max}
\DeclareMathOperator{\sign}{sign}
\DeclareMathOperator{\tr}{tr}
\newcommand{\ra}{\emphase{$\rightarrow$} \xspace}

% Hadamard, Kronecker and vec operators
\DeclareMathOperator{\Diag}{Diag} % matrix diagonal
\DeclareMathOperator{\diag}{diag} % vector diagonal
\DeclareMathOperator{\mtov}{vec} % matrix to vector
\newcommand{\kro}{\otimes} % Kronecker product
\newcommand{\had}{\odot}   % Hadamard product

% TikZ
\newcommand{\nodesize}{2em}
\newcommand{\edgeunit}{2.5*\nodesize}
\newcommand{\edgewidth}{1pt}
\tikzstyle{node}=[draw, circle, fill=black, minimum width=.75\nodesize, inner sep=0]
\tikzstyle{square}=[rectangle, draw]
\tikzstyle{param}=[draw, rectangle, fill=gray!50, minimum width=\nodesize, minimum height=\nodesize, inner sep=0]
\tikzstyle{hidden}=[draw, circle, fill=gray!50, minimum width=\nodesize, inner sep=0]
\tikzstyle{hiddenred}=[draw, circle, color=red, fill=gray!50, minimum width=\nodesize, inner sep=0]
\tikzstyle{observed}=[draw, circle, minimum width=\nodesize, inner sep=0]
\tikzstyle{observedred}=[draw, circle, minimum width=\nodesize, color=red, inner sep=0]
\tikzstyle{eliminated}=[draw, circle, minimum width=\nodesize, color=gray!50, inner sep=0]
\tikzstyle{empty}=[draw, circle, minimum width=\nodesize, color=white, inner sep=0]
\tikzstyle{blank}=[color=white]
\tikzstyle{nocircle}=[minimum width=\nodesize, inner sep=0]

\tikzstyle{edge}=[-, line width=\edgewidth]
\tikzstyle{edgebendleft}=[-, >=latex, line width=\edgewidth, bend left]
\tikzstyle{edgebendright}=[-, >=latex, line width=\edgewidth, bend right]
\tikzstyle{lightedge}=[-, line width=\edgewidth, color=gray!50]
\tikzstyle{lightedgebendleft}=[-, >=latex, line width=\edgewidth, bend left, color=gray!50]
\tikzstyle{lightedgebendright}=[-, >=latex, line width=\edgewidth, bend right, color=gray!50]
\tikzstyle{edgered}=[-, line width=\edgewidth, color=red]
\tikzstyle{edgebendleftred}=[-, >=latex, line width=\edgewidth, bend left, color=red]
\tikzstyle{edgebendrightred}=[-, >=latex, line width=\edgewidth, bend right, color=red]

\tikzstyle{arrow}=[->, >=latex, line width=\edgewidth]
\tikzstyle{arrowbendleft}=[->, >=latex, line width=\edgewidth, bend left]
\tikzstyle{arrowbendright}=[->, >=latex, line width=\edgewidth, bend right]
\tikzstyle{arrowred}=[->, >=latex, line width=\edgewidth, color=red]
\tikzstyle{arrowbendleftred}=[->, >=latex, line width=\edgewidth, bend left, color=red]
\tikzstyle{arrowbendrightred}=[->, >=latex, line width=\edgewidth, bend right, color=red]
\tikzstyle{arrowblue}=[->, >=latex, line width=\edgewidth, color=blue]
\tikzstyle{dashedarrow}=[->, >=latex, dashed, line width=\edgewidth]
\tikzstyle{dashededge}=[-, >=latex, dashed, line width=\edgewidth]
\tikzstyle{dashededgebendleft}=[-, >=latex, dashed, line width=\edgewidth, bend left]
\tikzstyle{lightarrow}=[->, >=latex, line width=\edgewidth, color=gray!50]


% Directory
\newcommand{\fignet}{/home/robin/Bureau/RECHERCHE/RESEAUX/EXPOSES/FIGURES}
\newcommand{\figchp}{/home/robin/Bureau/RECHERCHE/RUPTURES/EXPOSES/FIGURES}


%====================================================================
%====================================================================

%====================================================================
%====================================================================
\begin{document}
%====================================================================
%====================================================================

%====================================================================
\title[(Variational)Inference of the PLN model]{(Variational) Inference of the Poisson log-normal model}

\author[S. Robin]{J. Chiquet, M. Mariadassou, S. Robin}

\institute[INRA / AgroParisTech]{~ \\%INRA / AgroParisTech \\
%   \vspace{-.1\textwidth}
  \begin{tabular}{ccc}
    \includegraphics[height=.06\textheight]{\fignet/LogoINRA-Couleur} & 
    \hspace{.02\textheight} &
    \includegraphics[height=.06\textheight]{\fignet/logagroptechsolo} % & 
%     \hspace{.02\textheight} &
%     \includegraphics[height=.09\textheight]{\fignet/logo-ssb}
    \\ 
  \end{tabular} \\
  \bigskip
  }

\date[Rochebrune]{Mar. 2018, Rochebrune}

%====================================================================
%====================================================================
\maketitle
%====================================================================

%====================================================================
%====================================================================
\section{Multivariate analysis of abundance data}
\frame{\tableofcontents[currentsection]}
%====================================================================


%====================================================================
\subsection*{Abundance data}
%====================================================================
\frame{\frametitle{Community ecology}

  \paragraph{Abundance data.} $Y = [Y_{ij}]: n \times p$: 
  \begin{eqnarray*}
   Y_{ij} & = & \text{abundance of species $j$ in sample $i$ (old)} \\
    & = & \text{number of reads associated with species $j$ in sample $i$ (new)}
  \end{eqnarray*}
  
  \bigskip \bigskip 
  \paragraph{Need for multivariate analysis:} 
  \begin{itemize}
   \item to summarize the information from $Y$
   \item to exhibit patterns of diversity
   \item to understand between-species interactions
   \item to account for the effect of covariates
  \end{itemize}
  
  \bigskip
  \pause More generally, to \emphase{model dependences between count variables}
  $$
  \text{\ra Need for a generic (probabilistic) framework}
  $$
}

%====================================================================
\frame{\frametitle{Models for multivariate count data.}

  \paragraph{Abundance vector:} $Y_i = (Y_{i1}, \dots Y_{ip})$, $Y_{ij} = \emphase{\text{ counts } \in \Nbb}$

  \bigskip \bigskip \pause
  No generic model for multivariate counts.
  \bigskip 
  \begin{description}
    \item[Data transformation] $\widetilde{Y}_{ij} = \log (1+Y_{ij}), \sqrt{Y_{ij}}, \text{clr}(Y) \refer{Ait86}$
    \begin{itemize}
    \item Pb when many counts are zero. 
    \item No obvious way to account for covariates \\ ~
    \end{itemize}
  \item[Poisson multivariate distributions] ~
    \begin{itemize}
    \item Constraints of the form of the dependency \refer{IYA17} \\ ~
    \end{itemize}
  \item[Latent variable models] ~
    \begin{itemize}
    \item Poisson-Gamma (= negative binomial): positive dependency
    \item \emphase{Poisson-log normal} \refer{AiH89} 
    \end{itemize}
  \end{description}
}

%====================================================================
%====================================================================
\section{Poisson-log normal model}
\frame{\tableofcontents[currentsection]}
%====================================================================

%====================================================================
\subsection*{PLN distribution}
%====================================================================
\frame{\frametitle{{P}oisson-log normal (PLN) distribution}

  \paragraph{Latent Gaussian model.} 
  $$
  \left.
    \begin{tabular}{l}
	 latent vectors $(Z_i)_i$ iid $\sim \Ncal_p(0, \emphase{\Sigma})$ \\ 
	 independent counts $Y_i = (Y_{ij})_j$ conditional on $Z_i$ 
    \end{tabular}
  \right\}
  \quad 
  Y_{ij} \,|\, Z_{ij} \sim \Pcal\left(e^{\emphase{\mu_j} + Z_{ij}}\right)
  $$

  \bigskip \pause
  \paragraph{Properties:} over-dispersion: $\Var(Y_{ij}) > \Esp Y_{ij}$, $\Cov(Y_{ij}, Y_{ik})$: same sign as $\sigma_{jk}$.

  \bigskip \bigskip \pause
  \paragraph{Extensions.} 
  $$
  \left.
    \begin{tabular}{l}
	 $x_i =$ vector of covariates for observation $i$ \\
	 $o_{ij} = $ known 'offset'
    \end{tabular}
  \right\}
  \quad 
  Y_{ij} \,|\, Z_{ij} \sim \Pcal(e^{\emphase{o_{ij} + x_i^\intercal \beta_j} + Z_{ij}})
  $$

  \bigskip \pause
  \paragraph{Interpretation.} 
  \begin{itemize}
   \item Dependency in the \emphase{Gaussian} latent space (encoded by $\Sigma$)
   \item Additional effects are fixed
   \item Conditional Poisson distribution = noise model
  \end{itemize}
}

%====================================================================
\subsection*{Some avatars of the PLN model}
%====================================================================
\frame{\frametitle{First avatar: Dimension reduction}

  \begin{tabular}{cc}
    \begin{tabular}{p{.5\textwidth}}
	 \paragraph{Probabilistic PCA:} PLN model +
	 $$
	 \text{rank}(\Sigma) = q
	 $$
	 
	 \pause \bigskip
	 \paragraph{Pathobiome dataset \refer{JFS16}.} 
	 \begin{itemize}
	 \item $n = 116$ oak leaves = samples
	 \item $p_1 = 66$ bacterial species (OTU)
	 \item $p_2 = 48$ fungal species ($p = 114$)
	 \item covariates: tree type, branch height, distance to trunk,
	 ...
	 \item offsets: $o_{i1}, o_{i2} =$ offset for bacteria, fungi
	 \end{itemize} \\ ~
    \end{tabular}
    & 
    \hspace{-.05\textwidth}
    \begin{tabular}{p{.5\textwidth}}
	 \begin{overprint}
	  \onslide<3>
	  Offset only: $\widehat{q} = 24$ \\ ~\\
	  \includegraphics[width=.4\textwidth]{../FIGURES/CMR17-Fig1a1_ModSel.pdf}
	  \onslide<4>
	  Offset only: PC 1-2 \\ ~\\
	  \includegraphics[width=.4\textwidth]{../FIGURES/CMR17-Fig1c1_IndMap.pdf}
	  \onslide<5>
	  Offset + covariates: $\widehat{q} = 21$ \\ ~\\
	  \includegraphics[width=.4\textwidth]{../FIGURES/CMR17-Fig1a2_ModSel.pdf}
	  \onslide<6>
	  Offset + covariates: PC 1-2 \\ ~\\
	  \includegraphics[width=.4\textwidth]{../FIGURES/CMR17-Fig1c2_IndMap.pdf}
	 \end{overprint} 
    \end{tabular}
  \end{tabular}
}

%====================================================================
\frame{\frametitle{Second avatar: Network inference}

  \begin{tabular}{cc}
    \begin{tabular}{p{.5\textwidth}}
	 \paragraph{Aim:} 'infer the ecological network' \\
	 \ra statistical interpretation: \\
	 infer the {\sl latent} GGM.
	 
	 \bigskip \bigskip \pause
	 \paragraph{PLN network:} PLN model +
	 $$
	 \Sigma^{-1} \text{ sparse}
	 $$ 
	 
	 \bigskip
	 \ra sparsity-inducing penalty
	 $$
	 \lambda \; \|\Omega\|_{1, \text{off}}
	 $$ \\ ~ \\ ~ 
    \end{tabular}
    & 
    \hspace{-.1\textwidth}
    \begin{tabular}{p{.5\textwidth}}
	 \begin{overprint}
	  \onslide<3>
	  Offset only: \\ ~\\
	  \includegraphics[width=.4\textwidth]{../FIGURES/network_oak_offset_network}
	  \onslide<4>
	  Offset + covariates: \\ ~\\
	  \includegraphics[width=.4\textwidth]{../FIGURES/network_oak_tree_or_network}
	 \end{overprint}
	 \onslide+<3->
	 {
	   Ea = {\sl Erysiphe alphitoides} \\ = pathogene responsible for oak mildew
	   } 
    \end{tabular}
  \end{tabular}
}

%====================================================================
%====================================================================
\section{Inference}
\frame{\tableofcontents[currentsection]}
%====================================================================

%====================================================================
\subsection*{Variational inference}
%====================================================================
\frame{\frametitle{Intractable EM}
  
  \paragraph{Aim of the inference:} 
  \begin{itemize}
   \item estimate $\theta = (\beta, \Sigma)$ 
   \item predict the $Z_i$'s
  \end{itemize}

  \bigskip \bigskip \pause
  \paragraph{Maximum likelihood.} 
  \begin{itemize}
  \item PLN is an incomplete data model
  \ra Try EM, but no close form for $p(Z_i \,|\,  Y_i)$. \\~
  \pause
  \item \refer{Kar05} resorts to numerical or Monte-Carlo integration. \\ ~
  \pause
  \item Variational approach \refer{WaJ08}: use a proxy of $p(Z\,|\,Y)$.
  \end{itemize}
}

%====================================================================
\frame{\frametitle{Variational EM}

  \paragraph{Variational approximation:} choose a class of distribution $\Qcal$
  $$
  \qquad \Qcal = \Big\{\pt: \quad \pt(Z) = \prod_i \pt_i(Z_i), \quad \pt_i(Z_i) = \emphase{\Ncal}(Z_i; \mt_i, \St_i) \Big\}  
  $$
  and maximize the lower bound ($\Espt$ = expectation under $\pt$)
  \begin{eqnarray*}
    J(\theta, \pt) 
    = \log p_\theta(Y) - KL[\pt(Z) \,||\, p_\theta(Z\,|\,Y)] 
    = \Espt [\log p_\theta(Y, Z)] + \Hcal[\pt(Z)]
  \end{eqnarray*}

  \bigskip \pause
  \paragraph{Variational EM.} 
  \begin{itemize}
   \item VE step: find the optimal $\pt$: 
   $$
   \pt^h = \argmax J(\theta^h, \pt) = \argmin_{\pt \in \Qcal} KL[\pt(Z) \,||\, p_{\theta^h}(Z\,|\,Y)] 
   $$
   \item M step: update $\widehat{\theta}$
   $$
   \widehat{\theta}^h = \argmax J(\theta, \pt^h) = \argmax_\theta \Espt [\log p_\theta(Y, Z)]
   $$
  \end{itemize}
}

%====================================================================
\frame{\frametitle{Variational EM}
  
  \paragraph{Property:} The lower $J(\theta, \pt)$ is bi-concave, i.e.
  \begin{itemize}
  \item wrt $\pt = (\Mt, \St)$ for given $\theta$ 
  \item wrt $\theta = (\Sigma, \beta)$ for given $\pt$ % (close form for $\widehat{\Sigma} = n^{-1} (\Mt^\trans \Mt + \St_+)$)
  \end{itemize}
  but not jointly concave in general. 
  
  \bigskip
  \ra Still holds for PCA ($\text{rank}(\Sigma) = q$) and network inference ($J(\theta, \pt) - \lambda \; \|\Omega\|_{1, \text{off}}$)


  \bigskip \bigskip 
  \paragraph{Implementation:} Gradient ascent for the complete parameter $(\Mt, \St, \theta)$
%   \begin{itemize}
%    \item No formal VEM algorithm.
%   \end{itemize}
  
  \bigskip \bigskip \pause
  \paragraph{{\tt PLNmodels} package:} \refer{CMR17}
  $$
  \text{\url{https://github.com/jchiquet/PLNmodels}}
  $$
}

%====================================================================
\frame{\frametitle{Limitations of VEM}

  \paragraph{(Absence of) Properties of variational estimates}
  \begin{itemize}
   \item No generic properties (consistency, asymptotic normality) of variational estimates. \\~
   \pause
   \item Many negative results, including
   $$
   \argmax J(\theta, \pt) \neq \argmax \log p(Y; \theta)
   $$ \\ ~
   \pause
   \item Some positive results, in cases (SBM, LBM) where
   $$
   p(Z | Y) \underset{n \rightarrow \infty}{\longrightarrow} \Qcal
   $$ 
  \end{itemize}
  
  \bigskip \bigskip \pause
  \ra Not able to make inference on $\theta$


}
%====================================================================
\subsection*{Composite likelihood}
%====================================================================
\frame{\frametitle{Likelihoods}

  \paragraph{Likelihood} 
  \begin{align*}
  \log p(Y) 
%   & = \sum_i \log p(Y_i; \theta) \\
  & = \sum_i \log \int p(Z_i) p(Y_i|Z_i) \d Z_i
  \end{align*}
  \ra $n$ terms but intractable integral over $\Rbb^p$
  
  \bigskip \bigskip \pause
  \paragraph{Composite likelihood} \refer{VRF11}
  \begin{align*}
  \cl(Y) 
  & = \sum_{j < k} \log p(Y^j, Y^k) \\
%   = \sum_i \sum_{j < k} \log p(Y_{ij}, Y_{ik}; \theta) \\
  & = \sum_i \sum_{j < k} \log \int p(Y_{ij}, Y_{jk} |Z_{ij}, Z_{ik}) p(Z_{ij}, Z_{ik}) \d Z_{ij} \d Z_{ik} 
  \end{align*}
  \ra $\displaystyle{\frac{n p^2}2}$ terms but tractable integral over $\Rbb^2$
}

%====================================================================
\frame{\frametitle{A reminder on $M$-estimators (1/2)}

  \paragraph{Maximum composite likelihood} estimators are $M$-estimators \refer{vdV98}:
  $$
  M_n(\theta) = \frac1n \cl(Y; \theta) = \frac1n \sum_i m(Y_i; \theta)
  $$
  
  \paragraph{Properties.} If $M_n(\theta) \overset{p}{\rightarrow} M(\theta)$, $\sup_\theta |M_n(\theta) - M(\theta)| \overset{p}{\rightarrow} 0$ and $M(\theta)$ has a unique maximum in $\theta^*$
  \begin{enumerate}
  \item Consistency:
  $$\widehat{\theta}_n \overset{p}{\rightarrow} \theta^*$$
  \item Asymptotic normality:
  $$
  \sqrt{n}(\widehat{\theta}_n - \theta^*) \overset{d}{\rightarrow} \Ncal\left(0, G^{-1}(\theta^*)\right)
  $$
  where \refer{VRF11}
  $$
  G(\theta^*) = \text{Godambe matrix} = H(\theta^*) J(\theta^*)^{-1} H(\theta^*)
  $$
  
  $$
  J(\theta^*) = \Esp[\dot{m}(Y_i; \theta^*) \dot{m}(Y_i; \theta^*)^\intercal], \qquad
  H(\theta^*) = - \Esp[\ddot{m}(Y_i; \theta^*)].
  $$
  \end{enumerate}
}

%====================================================================
\frame{\frametitle{A reminder on $M$-estimators (2/2)}

  \paragraph{Sketch of proof of property 2 \refer{vdV98}.} $\theta$ with dimension 1 ($\dot{f}= \partial_\theta f$, $\ddot{f}= \partial^2_{\theta^2} f$, ...):
  $$
  0 = \dot{M}_n(\widehat{\theta}_n) 
  = \dot{M}_n(\theta^*) 
  + (\widehat{\theta}_n - \theta^*) \ddot{M}_n(\theta^*) 
  + \frac12 (\widehat{\theta}_n - \theta^*)^2 \dddot{M}_n(\overline{\theta}_n)
  $$
  that is
  $$
  \sqrt{n} \; (\widehat{\theta}_n - \theta^*) 
  = - \sqrt{n} \; \dot{M}_n(\theta^*) \left/ \left(
    \ddot{M}_n(\theta^*) 
    + \frac12 (\widehat{\theta}_n - \theta^*) \dddot{M}_n(\overline{\theta}_n) 
    \right) \right.
  $$
  where
  \begin{align*}
  (\widehat{\theta}_n - \theta^*) & \overset{p}{\rightarrow} 0, \\
  \ddot{M}_n(\theta^*) & \overset{p}{\rightarrow} \Esp [\ddot{m}(Y_i; \theta)], \\
  \sqrt{n} \; \dot{M}_n(\theta^*) & \overset{d}{\rightarrow} \Ncal\left(0, \Esp[\dot{m}^2(Y_i; \theta^*)]\right)
  \end{align*}
  because, $\dot{M}_n(\theta^*)$ is a mean. Then use Slutsky's lemma.
}

%====================================================================
\frame{\frametitle{Back to composite likelihood for PLN}

  \paragraph{$M$-estimator properties.} $m(Y_i; \theta) = \sum_{j < k} \log p(Y_{ij}, Y_{ik}; \theta)$ and use
  $$
  \widehat{J}(\widehat{\theta}) = \frac1n \sum_i \dot{m}(Y_i; \widehat{\theta}) \dot{m}(Y_i; \widehat{\theta})^\intercal , \qquad
  \widehat{H}(\widehat{\theta}) = - \frac1n \sum_i \ddot{m}(Y_i; \widehat{\theta})
  $$

  \bigskip \pause
  \paragraph{Good news.} 
  \begin{itemize}
  \item \pause {\tt poilog} R package: pdf of the bivariate PLN distribution \refer{GrE08} \\ ~
  \item \pause PLN inherits from Poisson:
  $$
  p_\mu(y) = e^{-\mu} {\mu^y}/{y!} \quad \Rightarrow \quad
  \partial_\mu p_\mu(y) = y p_\mu(y) - (y+1) p_\mu(y+1)
  $$
  \ra $\dot{m}$ and $\ddot{m}$ are functions of $p(Y_{ij}, Y_{ik})$. \\ ~
  \end{itemize}
  
}

%====================================================================
\frame{\frametitle{Proposed algorithm}

  To infer $\theta$ in the PLN model \\ ~
  
  \begin{enumerate}
   \item Use VEM to get the variational estimate $\widehat{\theta}_{VEM}$ \\
    \ra {\tt PLNmodels} is fast \\~
   \item Use gradient ascent to get the composite likelihood $\widehat{\theta}_{CL}$ \emphase{starting from $\widehat{\theta}_{VEM}$} \\
    \ra hopefully, few iterations are needed ({\tt poilog} is not that fast) \\~
   \item Estimate the Godambe matrix to perform inference (CI, tests, etc.)
  \end{enumerate}
  
}

%====================================================================
\subsection*{Some simulations}
%====================================================================
\frame{\frametitle{Simulations}

  \paragraph{Proof-of-concept:} Inference on $\beta$ with known $\Sigma$

%   \bigskip \pause
%   \paragraph{Iterations \& computation time.}   Mean (sd) over 100 simulations:
%   $$
%   \begin{array}{rrr|rr|rrr|rr}
%     \multicolumn{3}{c}{\text{parms}} & \multicolumn{2}{c}{\text{\# terms}} & \multicolumn{3}{c}{\text{time (s)}} & \multicolumn{2}{c}{\text{iterations}} \\
%     d & p & n & \beta & \cl 
% 	 & \widehat{\beta}_{VEM} & \widehat{\beta}_{CL} & \Var_\infty & VEM & CL \\
%     \hline
% 2  & 3  & 50  & 6  & 3  & 0.07\; (0.07) & 4.15\; (0.86) & 0.21\; (0) & 1681\; (1840) & 15\; (3) \\ 
% 2  & 10  & 50  & 20  & 45  & 0.86\; (0.34) & 64.11\; (8.91) & 3.02\; (0.06) & 4019\; (1655) & 16\; (2) \\ 
% 2  & 10  & 100  & 20  & 45  & 0.75\; (0.18) & 1.93\; (0.23) & 15.32\; (0.11) & 2441\; (616) & 14\; (2) \\   
%   \end{array}
%   $$

  \bigskip \pause
  \paragraph{Computation time.}   Mean (sd) over 100 simulations:
  $$
  \begin{array}{rrr|rr|rrrrrr}
    \multicolumn{3}{c}{\text{parms}} & \multicolumn{2}{c}{\text{\# terms}} & \multicolumn{6}{c}{\text{time (s)}} \\
    d & p & n & \beta & \cl & \multicolumn{2}{c}{\widehat{\beta}_{VEM}} & \multicolumn{2}{c}{\widehat{\beta}_{CL}} & \multicolumn{2}{c}{\Var_\infty} \\
    \hline
2  & 3  & 50  & 6  & 3  & 0.07 & (0.07) & 4.15 & (0.86) & 0.21 & (0) \\ 
2  & 10  & 50  & 20  & 45  & 0.86 & (0.34) & 64.11 & (8.91) & 3.02 & (0.06) \\ 
2  & 10  & 100  & 20  & 45  & 2.26 & (1.31) & 120.9 & (23.41) & 6.07 & (0.2) \\ 
  \end{array}
  $$
  
  \bigskip \pause
  \paragraph{Nb iterations.}   Mean (sd) over 100 simulations:  
  $$
  \begin{array}{rrr|rr|rrrr}
    \multicolumn{3}{c}{\text{parms}} & \multicolumn{2}{c}{\text{\# terms}} & \multicolumn{4}{c}{\text{iterations}} \\
    d & p & n & \beta & \cl & \multicolumn{2}{c}{VEM} & \multicolumn{2}{c}{CL} \\
    \hline
2  & 3  & 50  & 6  & 3  & 1681 & (1840) & 15 & (3) \\ 
2  & 10  & 50  & 20  & 45  & 4019 & (1655) & 16 & (2) \\ 
2  & 10  & 100  & 20  & 45  & 5787 & (3336) & 15 & (3) \\ 
  \end{array}
  $$
}

%====================================================================
\frame{\frametitle{Simulations}

  \paragraph{Distribution of the test statistic} $T = (\widehat{\beta} - \beta^*) \left/ \sqrt{\Var_\infty(\widehat{\beta})} \right.$
  
  \bigskip
  \begin{overprint}
    \onslide<2>
    $$d = 2 \text{ covariates}, \qquad p = 3 \text{ species}, \qquad n = 50 \text{ samples}$$
    $$
    \begin{tabular}{cc}
    \includegraphics[width=.35\textwidth]{../FIGURES/PLN-CL-Simul-p3-n50-d2-HistStatCL.pdf} &
    \includegraphics[width=.35\textwidth]{../FIGURES/PLN-CL-Simul-p3-n50-d2-QQplotCL.pdf} 
    \end{tabular}
    $$
    \onslide<3>
    $$d = 2 \text{ covariates}, \qquad p = 10 \text{ species}, \qquad n = 50 \text{ samples}$$
    $$
    \begin{tabular}{cc}
    \includegraphics[width=.35\textwidth]{../FIGURES/PLN-CL-Simul-p10-n50-d2-HistStatCL.pdf} &
    \includegraphics[width=.35\textwidth]{../FIGURES/PLN-CL-Simul-p10-n50-d2-QQplotCL.pdf} 
    \end{tabular}
    $$
    \onslide<4>
    $$d = 2 \text{ covariates}, \qquad p = 10 \text{ species}, \qquad n = 100 \text{ samples}$$
    $$
    \begin{tabular}{cc}
    \includegraphics[width=.35\textwidth]{../FIGURES/PLN-CL-Simul-p10-n100-d2-HistStatCL.pdf} &
    \includegraphics[width=.35\textwidth]{../FIGURES/PLN-CL-Simul-p10-n100-d2-QQplotCL.pdf} 
    \end{tabular}
    $$
  \end{overprint}
}

%====================================================================
\section*{Discussion}
% \frame{\tableofcontents[currentsection]}
%====================================================================
\frame{ \frametitle{Discussion}

  \paragraph{Summary}
  \begin{itemize}
   \item PLN = generic model for multivariate count data analysis
   \item Allows for covariates
   \item Flexible modeling of the covariance structure
   \item Efficient VEM algorithm
   \item {\tt PLNmodels} package: \url{https://github.com/jchiquet/PLNmodels}
   \item But no statistical guaranties
  \end{itemize}

  \bigskip \bigskip \pause
  \paragraph{Extensions}
  \begin{itemize}
   \item Composite likelihood inference: identifiability, consistency, asymptotic normality
   \item Model selection criterion for network inference
   \item Tree-based network inference
   \item Other covariance structures (spatial, time series, ...)
  \end{itemize}
}

%====================================================================
\frame{ \frametitle{References}
{\tiny
  \bibliography{/home/robin/Biblio/BibGene}
%   \bibliographystyle{/home/robin/LATEX/Biblio/astats}
  \bibliographystyle{alpha}
  }
}

%====================================================================
\appendix 
\backupbegin
\section{Appendix}

%====================================================================
\frame{\frametitle{Model selection for PLN-PCA}

  \paragraph{Number of components $q$:} needs to be chosen.
  
  \bigskip \bigskip %\pause
  \paragraph{Penalized 'likelihood'.}
  \begin{itemize}
   \item $\log p_{\widehat{\theta}}(Y)$ intractable: replaced with $J(\widehat{\theta}, \pt)$ \\ ~
   \item $BIC$ \ra $\widetilde{BIC}_q = J(\widehat{\theta}, \pt) - pq \log(n)/2$ \\ ~ 
   \item $ICL$ \ra $\widetilde{ICL}_q = \widetilde{BIC}_q - \Hcal(\pt)$ \\ ~ 
  \end{itemize}
  
  \bigskip
  \paragraph{Chosen rank:}
  $$
  \widehat{q} = \arg\max_q \widetilde{BIC}_q
  \qquad \text{or} \qquad
  \widehat{q} = \arg\max_q \widetilde{ICL}_q
  $$
}

%====================================================================
\frame{\frametitle{Visualization in PLN-PCA}

  \paragraph{PCA:} Optimal subspaces nested when $q$ increases.
  
  \bigskip \bigskip 
  \paragraph{PLN-pPCA:} Non-nested subspaces.

  \bigskip
  \ra For a the selected dimension $\widehat{q}$: \\~
  \begin{itemize}
   \item Compute the estimated latent positions $\widetilde{M}$ \\~
   \item Perform PCA on the $\widetilde{M}$ \\~
   \item Display results in any dimension $q \leq \widehat{q}$
  \end{itemize}
}

%====================================================================
\frame{\frametitle{Network inference in PLN}

  \paragraph{Graphical model \refer{Lau96}} of the $Y_i = (Y_{i1}, \dots Y_{ip})$, i.e. the graph $G$ such that
  $$
  p(Y_i) \propto \prod_{C \in \Ccal(G)} \psi_C(Y_i^C)
  $$
  where $\Ccal(G) =$ set of cliques of $G$ \\ 
  \pause \bigskip
  \paragraph{Cheating with PLN:} Use the PLN model and infer the graphical model of $Z$
  \renewcommand{\nodesize}{1.75em}
  \renewcommand{\edgeunit}{2.5*\nodesize}
  \bigskip  
  \begin{overprint}
   \onslide<2>
    $$
    \begin{array}{ccc}
    {\footnotesize \input{Fig2-jointZY} }
    & \qquad &
    {\footnotesize \input{Fig2-margY} }
    \end{array}
    $$
  \onslide<3>
   $$
   \begin{array}{ccc}
   {\footnotesize \begin{tikzpicture}
\node[hidden] (Z1) at ( 0.95*\edgeunit,  0.31*\edgeunit) {$Z_1$};
\node[hidden] (Z2) at (-0.00*\edgeunit,  1.00*\edgeunit) {$Z_2$};
\node[hidden] (Z3) at (-0.95*\edgeunit,  0.31*\edgeunit) {$Z_3$};
\node[hidden] (Z4) at (-0.59*\edgeunit, -0.81*\edgeunit) {$Z_4$};
\node[hidden] (Z5) at ( 0.59*\edgeunit, -0.81*\edgeunit) {$Z_5$};

\draw[edge] (Z1) to (Z2);  \draw[edge] (Z1) to (Z4);  
\draw[edge] (Z1) to (Z5);  \draw[edge] (Z2) to (Z3);  
\draw[edge] (Z2) to (Z4);  \draw[edge] (Z3) to (Z4); 

\node[observed] (Y1) at ( 1.05*\edgeunit, -0.39*\edgeunit) {$Y_1$};
\node[observed] (Y2) at (-0.00*\edgeunit,  0.30*\edgeunit) {$Y_2$};
\node[observed] (Y3) at (-1.05*\edgeunit, -0.39*\edgeunit) {$Y_3$};
\node[observed] (Y4) at (-0.59*\edgeunit, -1.51*\edgeunit) {$Y_4$};
\node[observed] (Y5) at ( 0.59*\edgeunit, -1.51*\edgeunit) {$Y_5$};

\draw[arrow] (Z1) to (Y1); 
\draw[arrow] (Z2) to (Y2);
\draw[arrow] (Z3) to (Y3);
\draw[arrow] (Z4) to (Y4);
\draw[arrow] (Z5) to (Y5);
\end{tikzpicture}
 }
   & \qquad &
   {\footnotesize \input{Fig1-margY} }
   \end{array}
  $$
  \end{overprint}
  \onslide+<3->{
  $$
  \text{Graphical model of $Z$ \emphase{$\neq$} Graphical model of $Y$}
  $$
  }
}

%====================================================================
\frame{ \frametitle{Model selection}

  \paragraph{Network density:} controlled by $\lambda$ 
  
  \bigskip \bigskip \pause
  \paragraph{Penalized 'likelihood'.} 
  \begin{itemize}
   \item $\widetilde{BIC}(\lambda) = J(\widehat{\theta}, \pt)- \frac12 {\log n} \left(p q + \,|\,\text{Support}(\widehat{\Omega}_\lambda)\,|\, \right)$
   \item $EBIC(\lambda):$ Extended BIC \refer{FoD10}
  \end{itemize}

  \bigskip \bigskip \pause
  \paragraph{Stability selection.} 
  \begin{itemize}
   \item Get $B$ subsamples
   \item Get $\widehat{\Omega}^b_\lambda$ for an intermediate $\lambda$ and $b = 1...B$
   \item Count the selection frequency of each edge
  \end{itemize}

}

%====================================================================
\frame{\frametitle{Oak powdery mildew: stability selection}

  \begin{tabular}{cc}
% p   no covariates & covariate = tree + orientation \\
   \begin{tabular}{c}
    \includegraphics[height=.6\textheight]{../FIGURES/network_oak_offset_network_stabsel}
   \end{tabular}
   &
   \begin{tabular}{c}
    \includegraphics[height=.6\textheight]{../FIGURES/network_oak_tree_or_network_stabsel}
   \end{tabular}
  \end{tabular}
  }

%====================================================================
\frame{\frametitle{Comparison VEM / CL estimates}

  \paragraph{With know $\Sigma$:} $d=2$, $p=10$, $n=30$
  
  $$
  \begin{array}{cc}
   VEM & CL \\
    \includegraphics[width=.35\textwidth]{../FIGURES/PLN-CL-Simul-p10-n50-d2-BoxplotVEM}
   &
    \includegraphics[width=.35\textwidth]{../FIGURES/PLN-CL-Simul-p10-n50-d2-BoxplotCL}
   \end{array}
  $$
  }

\backupend

%====================================================================
%====================================================================
\end{document}
%====================================================================
%====================================================================

  \begin{tabular}{cc}
    \begin{tabular}{p{.5\textwidth}}
    \end{tabular}
    & 
    \hspace{-.02\textwidth}
    \begin{tabular}{p{.5\textwidth}}
    \end{tabular}
  \end{tabular}
