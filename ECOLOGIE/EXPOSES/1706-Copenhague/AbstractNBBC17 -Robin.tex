%%%%%%%%%%%%%%%%%%%%%%%%%%%%%%%%%%%%%%%%%%%%%%%%%%%%
% This is a sample input file for your abstract.
% Please use it as a template for your own input, and please
% follow the instructions in this document.
%%%%%%%%%%%%%%%%%%%%%%%%%%%%%%%%%%%%%%%%%%%%%%%%%%%%
\documentclass[a4paper,english]{article}
\usepackage{mathpazo}
\usepackage{babel}
\usepackage[latin1]{inputenc}
\usepackage[T1]{fontenc}

\textheight 18.9cm\textwidth 11.7cm\topmargin 0cm \hoffset=0cm \setlength\oddsidemargin   {2cm}
\setlength\evensidemargin  {2cm}
\def\title#1{{\Large\bf  \begin{center} \textsc{#1} \vspace{0pt} \end{center}  } }
\def\authors#1{{\large \begin{center} #1 \vspace{0pt} \end{center} } }
\def\university#1{{ \begin{center} #1 \vspace{0pt} \end{center} } }
\def\inst#1{\unskip$^{#1}$}
\newcommand{\keywords}[1]{\bigskip\noindent {\large\bf Keywords:} \ #1\vspace{1cm}}
\newenvironment{references}{\noindent {\large\bf References:} \medskip \small \begin{description} \itemindent -7.5ex }{\end{description}}
\begin{document}

\title{
%%% YOUR TITLE HERE:
Detection of adaptive shifts on phylogenies using shifted stochastic processes on a tree
}

\authors{
%%% NAME(S) OF AUTHOR(S) HERE, THE PRESENTING AUTHOR UNDERLINED:
Paul Bastide\inst{1}\inst{2}\inst{3} and Mahendra Mariadassou\inst{2}\inst{3} and \underline{St\'ephane Robin}\inst{1}\inst{2}\inst{3} 
% \author[Bastide, Mariadassou, Robin]{Paul Bastide}
% \address{AgroParisTech, UMR518 MIA-Paris, F-75231 Paris Cedex 05, France.\\
% INRA, UMR518 MIA-Paris, F-75231 Paris Cedex 05, France.\\
% INRA, UR1404 Unit\'e Math\'ematiques et Informatique Appliqu\'ees du G\'enome \`a l'Environnement, F78352 Jouy-en-Josas, France.}
% \email{paul.bastide@agroparistech.fr}
% 
% \author{Mahendra Mariadassou}
% \address{INRA, UR1404 Unit\'e Math\'ematiques et Informatique Appliqu\'ees du G\'enome \`a l'Environnement, F78352 Jouy-en-Josas, France.}
% 
% \author{St\'ephane Robin}
% \address{AgroParisTech, UMR518 MIA-Paris, F-75231 Paris Cedex 05, France.\\
% INRA, UMR518 MIA-Paris, F-75231 Paris Cedex 05, France.}
}

 \university{
%%% AUTHOR(S) INSTITUTION(S) HERE:
 \inst{1} AgroParisTech, France\\
 \inst{2} INRA, France\\
 \inst{3} University Paris-Saclay, France\\
 }

%% ENTER THE TEXT OF YOUR ABSTRACT HERE:
Comparative and evolutive ecologists are interested in the distribution of quantitative traits among related species. The classical framework for these distributions consists of a random process running along the branches of a phylogenetic tree relating the species. We consider shifts in the process parameters, which reveal fast adaptation to changes of ecological niches. We show that models with shifts are not identifiable in general. Constraining the models to be parsimonious in the number of shifts partially alleviates the problem but several evolutionary scenarios can still provide the same joint distribution for the extant species. We provide a recursive algorithm to enumerate all the equivalent scenarios and to count the number of effectively different scenarios. We introduce an incomplete-data framework and develop a maximum likelihood estimation procedure based on the EM algorithm. We also propose a model selection procedure, based on the cardinal of effective scenarios, to estimate the number of shifts and for which we prove an oracle inequality. Eventually, we will discuss the generalization of this approach to multivariate traits as weel as the case were the phylogenetic tree is not actually a tree.

%% ENTER SOME KEYWORDS HERE:
\keywords{Random process on tree, Ornstein-Uhlenbeck process, Change-point detection, Adaptive shifts, Phylogeny, Model selection.}

%% ENTER THE REFERENCES HERE (OR DELETE THE FOLLOWING, IF NO REFERENCES ADDED)
\begin{references}
\item
Bastide, P., Mariadassou, M., & Robin, S. (2016). J. Royal Stat. Soc. B. DOI:10.1111/rssb.12206
\end{references}

\end{document} 