\documentclass[12pt]{beamer}

% Beamer style
%\usetheme[secheader]{Madrid}
% \usetheme{CambridgeUS}
\useoutertheme{infolines}
\usecolortheme[rgb={0.65,0.15,0.25}]{structure}
% \usefonttheme[onlymath]{serif}
\beamertemplatenavigationsymbolsempty
%\AtBeginSubsection

% Packages
%\usepackage[french]{babel}
\usepackage[latin1]{inputenc}
\usepackage{color}
\usepackage{xspace}
\usepackage{dsfont, stmaryrd}
\usepackage{amsmath, amsfonts, amssymb}
\usepackage{epsfig}
\usepackage{tikz}
\usepackage{url}
\usepackage{/home/robin/LATEX/Biblio/astats}
%\usepackage[all]{xy}
\usepackage{graphicx}

% Commands
% % TikZ
\newcommand{\nodesize}{1.8em}
\newcommand{\edgeunit}{2.2*\nodesize}
\tikzstyle{hidden}=[draw, circle, fill=gray!50, minimum width=\nodesize, inner sep=0]
\tikzstyle{observed}=[draw, circle, minimum width=\nodesize, inner sep=0]
\tikzstyle{eliminated}=[draw, circle, minimum width=\nodesize, color=gray!50, inner sep=0]
\tikzstyle{empty}=[]
\tikzstyle{arrow}=[->, >=latex, line width=1pt]
\tikzstyle{edge}=[-, line width=1pt]
\tikzstyle{dashedarrow}=[->, >=latex, dashed, line width=1pt]
\tikzstyle{lightarrow}=[->, >=latex, line width=1pt, fill=gray!50, color=gray!50]


\definecolor{darkred}{rgb}{0.65,0.15,0.25}
\newcommand{\emphase}[1]{\textcolor{darkred}{#1}}
% \newcommand{\emphase}[1]{{#1}}
\newcommand{\paragraph}[1]{\textcolor{darkred}{#1}}
\newcommand{\refer}[1]{{\small{\textcolor{blue}{{[\cite{#1}]}}}}}
% \newcommand{\Refer}[1]{{\small{\textcolor{gray}{{[#1]}}}}}
\renewcommand{\newblock}{}

% Symbols
\newcommand{\Abf}{{\bf A}}
\newcommand{\Beta}{\text{B}}
\newcommand{\Bcal}{\mathcal{B}}
\newcommand{\BIC}{\text{BIC}}
\newcommand{\Ccal}{\mathcal{C}}
\newcommand{\dd}{\text{~d}}
\newcommand{\dbf}{{\bf d}}
\newcommand{\Dcal}{\mathcal{D}}
\newcommand{\Esp}{\mathbb{E}}
\newcommand{\Ebf}{{\bf E}}
\newcommand{\Ecal}{\mathcal{E}}
\newcommand{\Gcal}{\mathcal{G}}
\newcommand{\Gam}{\mathcal{G}\text{am}}
\newcommand{\Hcal}{\mathcal{H}}
\newcommand{\Ibb}{\mathbb{I}}
\newcommand{\Ibf}{{\bf I}}
\newcommand{\ICL}{\text{ICL}}
\newcommand{\Cov}{\mathbb{C}\text{ov}}
\newcommand{\Corr}{\mathbb{C}\text{orr}}
\newcommand{\Var}{\mathbb{V}}
\newcommand{\Vsf}{\mathsf{V}}
\newcommand{\pen}{\text{pen}}
\newcommand{\Fcal}{\mathcal{F}}
\newcommand{\Hbf}{{\bf H}}
\newcommand{\Jcal}{\mathcal{J}}
\newcommand{\Kbf}{{\bf K}}
\newcommand{\Lcal}{\mathcal{L}}
\newcommand{\Mcal}{\mathcal{M}}
\newcommand{\mbf}{{\bf m}}
\newcommand{\mum}{\mu(\mbf)}
\newcommand{\Nbb}{\mathbb{N}}
\newcommand{\Ncal}{\mathcal{N}}
\newcommand{\Nbf}{{\bf N}}
\newcommand{\Nm}{N(\mbf)}
\newcommand{\Ocal}{\mathcal{O}}
\newcommand{\Obf}{{\bf 0}}
\newcommand{\Omegas}{\underset{s}{\Omega}}
\newcommand{\Pbf}{{\bf P}}
\newcommand{\pt}{\widetilde{p}}
\newcommand{\Pt}{\widetilde{P}}
\newcommand{\Pcal}{\mathcal{P}}
\newcommand{\Qcal}{\mathcal{Q}}
\newcommand{\Rbb}{\mathbb{R}}
\newcommand{\Rcal}{\mathcal{R}}
\newcommand{\Scal}{\mathcal{S}}
\newcommand{\Tcal}{\mathcal{T}}
\newcommand{\Ucal}{\mathcal{U}}
\newcommand{\Vcal}{\mathcal{V}}
\newcommand{\BP}{\text{BP}}
\newcommand{\EM}{\text{EM}}
\newcommand{\VEM}{\text{VEM}}
\newcommand{\VBEM}{\text{VBEM}}
\newcommand{\cst}{\text{cst}}
\newcommand{\obs}{\text{obs}}
\newcommand{\ra}{\emphase{\mathversion{bold}{$\rightarrow$}~}}
%\newcommand{\transp}{\text{{\tiny $\top$}}}
\newcommand{\transp}{\text{{\tiny \mathversion{bold}{$\top$}}}}
\newcommand{\logit}{\text{logit}\xspace}

% Directory
\newcommand{\fignet}{/home/robin/Bureau/RECHERCHE/RESEAUX/EXPOSES/FIGURES}
\newcommand{\figchp}{/home/robin/Bureau/RECHERCHE/RUPTURES/EXPOSES/FIGURES}


%====================================================================
%====================================================================

%====================================================================
%====================================================================
\begin{document}
%====================================================================
%====================================================================

%====================================================================
\title[Probabilistic PCA for counts]{Probabilistic PCA for counts: A variational approach}

\author[S. Robin]{S. Robin \\ ~\\
  \begin{tabular}{ll}
    Joint work with J. Chiquet \& M. Mariadassou
  \end{tabular}
  }

\institute[INRA / AgroParisTech]{~ \\%INRA / AgroParisTech \\
  \vspace{-.1\textwidth}
  \begin{tabular}{ccc}
    \includegraphics[height=.25\textheight]{\fignet/LogoINRA-Couleur} & 
    \hspace{.02\textheight} &
    \includegraphics[height=.06\textheight]{\fignet/logagroptechsolo} % & 
%     \hspace{.02\textheight} &
%     \includegraphics[height=.09\textheight]{\fignet/logo-ssb}
    \\ 
  \end{tabular} \\
  \bigskip
  }

\date[May 2017, Rennes]{Hydrogene, May 2017, Rennes}

%====================================================================
%====================================================================
\maketitle
%====================================================================

%====================================================================
%====================================================================
\section{Multivariate analysis of 'abundance' data}
%====================================================================

%====================================================================
\subsection{Multivariate analysis}
%====================================================================
\frame{\frametitle{Abundance data}

  \paragraph{Data table.} 
  $$
  Y = [Y_{ij}]: n \times p, \qquad \text{either $n$ or $p$ 'large'}
  $$
  \begin{itemize}
   \item $Y_{ij} =$ count of $k$-mer $j$ in sample $i$
   \item $Y_{ij} =$ abundance of species $j$ in sample $i$
   \item ...
  \end{itemize}
  
  \bigskip \bigskip \pause
  \paragraph{Multivariate analysis:} summarize the information.
  \begin{enumerate}
   \item Find clusters of samples, of species, of both;
   \item \emphase{Understand the co-variations of the species abundances};
   \item Determine the species characterizing sub-groups of samples;
   \item ...
  \end{enumerate}
}

%====================================================================
\frame{\frametitle{Principal component analysis (PCA)}

  2 points of view.

  \bigskip \bigskip 
  \paragraph{Descriptive:} summarize $Y$ with a reduced $n \times q$ table, with $q \ll p$
  \begin{itemize}
  \item 'faithful' visualization of $Y$ (summary)
  \item no model
  \end{itemize}

  \bigskip \bigskip \pause 
  \paragraph{Probabilistic:} analyze the dependency structure between the columns of $Y$
  \begin{itemize} 
  \item understand co-variations
  \item probabilistic model
  \end{itemize}
}

%====================================================================
\frame{\frametitle{Example ({\tt prcomp} vignette)}

  \begin{tabular}{cc}
    \begin{tabular}{p{.4\textwidth}}
    Crime data:
    \begin{itemize}
     \item $n =51$ observations \\  
     \item $p = 4$ variables \\ 
     \item $q = 2$ principal components (PC)
    \end{itemize}
    \qquad \includegraphics[width=.3\textwidth]{../FIGURES/Fig-pPCA-exPCA-inertia.pdf}
    \end{tabular}
    & 
    \hspace{-.1\textwidth}
    \begin{tabular}{p{.5\textwidth}}
	 \includegraphics[width=.6\textwidth]{../FIGURES/Fig-pPCA-exPCA-biplot.pdf}
    \end{tabular}
  \end{tabular}
}

%====================================================================
\frame{\frametitle{Rational of PCA}

  \paragraph{Idea.} The $n \times p$ matrix $Y$ is a noisy version of a $n \times q$ matrix $W$:
  $$
  \underset{n \times p}{Y} = \underset{n \times q}{W} \quad \underset{p \times q}{B} + \underset{n \times p}{E}
  $$
  
  \begin{itemize}
   \item $Y:$ observed data
   \item $W:$ latent 'factors'
   \item $B:$ coefficients linking latent and observed variables
   \item $E:$ non-informative noise
  \end{itemize}

}

%====================================================================
\subsection{Probabilistic PCA}
%====================================================================
\frame{\frametitle{Probabilistic PCA}

  \paragraph{Multivariate Gaussian model.} \refer{TiB99}
  \begin{itemize}
   \item $W_i:$ vector of latent variables (dimension $q$)
   $$
   W_i \sim \Ncal_q(0, I_q)
   $$
   \item $Y_i:$ vector of observed variables (dimension $p$)
   $$
   Y_i \;|\; W_i \sim \Ncal_p(B W_i, \sigma^2 I_p), \qquad B: p \times q
   $$
   $B$ 'distributes' the variations of $q$ dimensions over $p$ dimensions.
  \end{itemize}
  
  \bigskip \bigskip \pause
  \paragraph{Property.}
  $$
  \Var(Y_i) = B' B + \sigma^2 I_p
  $$
  $B' B = $ matrix of rank $q$, \quad $\sigma^2 I_p =$ noise

}

%====================================================================
\frame{\frametitle{Covariance structure}

  \begin{tabular}{cc}
    Empirical covariance & \hspace{-.15\textwidth} Rotated covariance \\
    \begin{tabular}{p{.5\textwidth}}
    \includegraphics[width=.45\textwidth]{../FIGURES/Fig-pPCA-Sigma.pdf}
    \end{tabular}
    & 
    \hspace{-.1\textwidth}
    \begin{tabular}{p{.5\textwidth}}
    \includegraphics[width=.45\textwidth]{../FIGURES/Fig-pPCA-rotSigma.pdf}
    \end{tabular} \\
    $p = 20$ & \hspace{-.15\textwidth} $q = 5$ 
  \end{tabular}
}

%====================================================================
\frame{\frametitle{Accounting for covariates}

  \paragraph{Additional information.}
  \begin{itemize}
   \item $x_i:$ vector of descriptors for observation $i$
   \item $X :$ $n \times d$ matrix of descriptors
  \end{itemize}
  
  \bigskip \bigskip \pause
  \paragraph{Interest of modeling.} Model accounting for covariates
  $$
  Y_i \;|\; W_i \sim \Ncal_p(x_i \beta + W_i B, \sigma^2 I_p)
  $$
  with $\beta:$ $d \times q$ matrix of regression coefficient (effect of each covariate on each species).
  
}

%====================================================================
\frame{\frametitle{Inference 1/2}

  \paragraph{Incomplete data model.} 
  \begin{itemize}
   \item $Y :$ observed variables
   \item $W :$ unobserved variables
   \item $\theta = (B, \sigma^2):$ unknown parameters
  \end{itemize}

  \bigskip \bigskip 
  \paragraph{Maximum likelihood estimate:} $\widehat{\theta} = \arg\max_\theta \log p_\theta(Y)$ 
  $$
  \qquad p_\theta(Y) = \int p_\theta(Y, W) \dd W
  $$
  \ra Often hard to manage $p_\theta(Y)$ directly.

}

%====================================================================
\frame{\frametitle{Inference 2/2}

  \paragraph{EM algorithm.} \refer{DLR77}
  \begin{itemize}
   \item E step: given $\theta$ 'retrieve' the unobserved variables % (+ some moments)
   $$
   \widehat{W}_i = \Esp_{\theta^h}(W_i\;|\;Y_i), \qquad \left( + \; \Var_{\theta^h}(W_i\;|\;Y_i) \right)
   $$
   \item M step: update $\theta$ using the 'completed' likelihood
   $$
   \theta^{h+1} = \arg\max_\theta \Esp_{\theta^h} [\log p_\theta(Y, W) \;|\; Y]
   $$
  \end{itemize}

  \bigskip \bigskip \pause
  \paragraph{Gaussian pPCA.} EM manageable \refer{TiB99} because 
  $$
  p_\theta(W \;|\; Y) \text{ explicit.}
  $$
  ... but Gaussian distribution not suitable for counts.

}

%====================================================================
%====================================================================
\section{Poisson - log normal model}
%====================================================================

%====================================================================
\subsection{Multivariate count data}
%====================================================================
\frame{\frametitle{Models for multivariate count data.}

  \paragraph{Abundance vector:} $Y_i = (Y_{i1}, \dots Y_{ip})$, $Y_{ij} \in \Nbb$

  \bigskip \bigskip \pause
  \paragraph{No generic model for multivariate counts.}
  \begin{itemize}
   \item Data transformation ($\widetilde{Y}_{ij} = \log(Y_{ij}), \sqrt(Y_{ij})$) \\ 
   \ra Pb when many counts are zero.
   \item Poisson multivariate distributions \\
   \ra Constraints of the form of the dependency \refer{IYA16}
   \item Latent variable models \\
   \ra Poisson-Gamma (negative binomial) \\
   \ra \emphase{Poisson-log normal (PLN)} \refer{AiH89}
  \end{itemize}

}

%====================================================================
\frame{\frametitle{Multivariate {P}oisson-log normal distribution}

  \paragraph{Model.}
  \begin{itemize}
   \item $Z_i:$ latent vector $\sim \Ncal_p(0, \Sigma)$ \\ ~
   \item $(Y_{ij})_j:$ counts independent conditional on $Z_i$ \\ ~
   \item $Y_{ij} | Z_{ij} \sim \Pcal(\exp Z_{ij})$
  \end{itemize}

}

%====================================================================
\frame{\frametitle{}
}

%====================================================================
\subsection{}
%====================================================================
\frame{\frametitle{}
}

%====================================================================
\frame{\frametitle{}
}

%====================================================================
\frame{\frametitle{}
}

%====================================================================
%====================================================================
\section{Variational inference}
%====================================================================

%====================================================================
\subsection{}
%====================================================================
\frame{\frametitle{}
}

%====================================================================
\frame{\frametitle{}
}

%====================================================================
\frame{\frametitle{}
}

%====================================================================
\subsection{}
%====================================================================
\frame{\frametitle{}
}

%====================================================================
\frame{\frametitle{}
}

%====================================================================
\frame{\frametitle{}
}

%====================================================================
%====================================================================
\section{Illustration}
%====================================================================

%====================================================================
\subsection{}
%====================================================================
\frame{\frametitle{}
}

%====================================================================
\frame{\frametitle{}
}

%====================================================================
\frame{\frametitle{}
}

%====================================================================
\subsection{}
%====================================================================
\frame{\frametitle{}
}

%====================================================================
\frame{\frametitle{}
}

%====================================================================
\frame{\frametitle{}
}

%====================================================================
\frame{ \frametitle{References}
{\tiny
  \bibliography{/home/robin/Biblio/BibGene}
%   \bibliographystyle{/home/robin/LATEX/Biblio/astats}
  \bibliographystyle{alpha}
  }
}



%====================================================================
%====================================================================
\end{document}
%====================================================================
%====================================================================

  \begin{tabular}{cc}
    \begin{tabular}{p{.5\textwidth}}
    \end{tabular}
    & 
    \hspace{-.02\textwidth}
    \begin{tabular}{p{.5\textwidth}}
    \end{tabular}
  \end{tabular}

