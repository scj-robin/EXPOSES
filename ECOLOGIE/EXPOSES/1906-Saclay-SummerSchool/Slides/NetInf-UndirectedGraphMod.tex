%==================================================================
\subsection*{Undirected graphical models}
%==================================================================
\frame{\frametitle{Undirected graphs = Markov random fields}

  \paragraph{Definition.} Let $G$ be an {\sl undirected graph}, the distribution $p$ is said to factorize in $G$ iff
  $$
  p(x_1, \dots x_n) \propto \prod_{C \in \Ccal(G)} \psi_C(x_C).
  $$
  where $\Ccal(G)$ is the set of maximal cliques of $G$

  \bigskip \bigskip \pause
  \begin{tabular}{cc}
    \begin{tabular}{c}
    $  \begin{tikzpicture}
  \input{\figeconet/GM7nodes-positions}
  
  \draw[edge] (a) to (b);  \draw[edge] (a) to (c);  \draw[edge] (a) to (d);
  \draw[edge] (b) to (c);  \draw[edge] (b) to (d);  \draw[edge] (c) to (d);
  \draw[edge] (d) to (e);  \draw[edge] (d) to (f);  \draw[edge] (e) to (f);
  \draw[edge] (f) to (g);  
  \end{tikzpicture}
$
    \end{tabular}
    &
    \begin{tabular}{p{.7\textwidth}}
      \begin{eqnarray*}
%         p(a, \dots g) & \propto 
%         & \psi_1(a, b, c) \; \psi_2(a, b, d) \; \psi_3(a, c, d) \; \psi_4(b, c, d) \\
%         & & \psi_5(d, e, f) \; \psi_6(f, g) \\\pause
%       \text{but also} \qquad \\
        p(a, \dots g) & \propto 
        & \psi_1(a, b, c, d) \\
        & & \psi_2(d, e, f) \; \psi_3(f, g) 
      \end{eqnarray*}
%       \ra Only consider \emphase{maximal} cliques
    \end{tabular}
  \end{tabular}
}

%====================================================================
\frame{\frametitle{Conditional independence}

  \paragraph{Property.} If $p(x) > 0$, 
  $$
  \text{separation} \qquad \Leftrightarrow \qquad \text{conditional independence}
  $$

  \bigskip \pause
  \begin{tabular}{ccc}
    \hspace{.2\textwidth} 
    &
    \begin{tabular}{c}
    $  \begin{tikzpicture}
  \input{\figeconet/GM7nodes-positions}
  
  \draw[edge] (a) to (b);  \draw[edge] (a) to (c);  \draw[edge] (a) to (d);
  \draw[edge] (b) to (c);  \draw[edge] (b) to (d);  \draw[edge] (c) to (d);
  \draw[edge] (d) to (e);  \draw[edge] (d) to (f);  \draw[edge] (e) to (f);
  \draw[edge] (f) to (g);  
  \end{tikzpicture}
$
    \end{tabular}
    &
    \begin{tabular}{p{.7\textwidth}}
    \begin{itemize}
     \item $A \not\independent B$ \\ ~
     \item $A \not\independent D \mid B$ \\ ~
     \item $A \independent D \mid \{B, C\}$ \\ ~
     \item $\{A, B, C\} \independent \{E, F, G\} \mid D$ \\ ~
     \item $\{B, C\} \independent \{E, F\} \mid D$ \\ ~
    \end{itemize}
    \end{tabular}
  \end{tabular}

  \bigskip
  \ra Inferring $G$ = Inferring conditional independences
}
  
%====================================================================
\subsection*{Missing actors}
%====================================================================
\frame{\frametitle{Missing 'actors'}

  \paragraph{Incomplete observations.} Most of the time, not all actors (species, environmental covariate, ...) are observed
  
  \bigskip \bigskip 
  \paragraph{Missing variable = marginalisation.} 
  \begin{center}
  \begin{tabular}{ccccc}
    $B \independent C \mid A$ & & $A$ missing & & $B \not\independent C$ \\
    \begin{tabular}{c}
    $  \begin{tikzpicture}
    \node[observed] (a) at (.5*\edgeunit, 0.75*\edgeunit) {$A$};
  \node[observed] (b) at (0*\edgeunit, 0*\edgeunit) {$B$};
  \node[observed] (c) at (1*\edgeunit, 0*\edgeunit) {$C$};

  
  \draw[edge] (a) to (b);  \draw[edge] (a) to (c);
  \end{tikzpicture}
$
    \end{tabular}
    & \qquad \qquad &
    \begin{tabular}{c}
    $  \begin{tikzpicture}
  \input{\figeco/MissingNode-Amissing-positions}
  
  \draw[lightedge] (a) to (b);  \draw[lightedge] (a) to (c);
  \end{tikzpicture}
$
    \end{tabular}
    & \qquad \qquad &
    \begin{tabular}{c}
    $  \begin{tikzpicture}
  \input{\figeco/MissingNode-Amissing-positions}
  
  \draw[edge] (b) to (c);
  \end{tikzpicture}
$
    \end{tabular}
  \end{tabular}
  \end{center}
  
  \bigskip
  \paragraph{Indeed:}
  $$
  p(a, b, c) = p(a) p(b \mid a) p(c \mid a)
  $$
  but $A$ is not observed, so only $p(b, c)$ can be considered:
  $$
  p(b, c) = \sum_a p(a, b, c) = \sum_a p(\emphase{a}) p(b \mid \emphase{a}) p(c \mid \emphase{a}) \neq p(b) p(c)
  $$
}

%====================================================================
\frame{\frametitle{'Spurious' edges}

  Possibly dramatic effect on the observable dependency structure:
  \begin{center}
  \begin{tabular}{ccccc}
    'Truth' && $C$ missing && $D$ missing \\
    \begin{tabular}{c}
    $  \begin{tikzpicture}
  \input{\figeconet/GM7nodes-positions}
  
  \draw[edge] (a) to (b);  \draw[edge] (a) to (c);  \draw[edge] (a) to (d);
  \draw[edge] (b) to (c);  \draw[edge] (b) to (d);  \draw[edge] (c) to (d);
  \draw[edge] (d) to (e);  \draw[edge] (d) to (f);  \draw[edge] (e) to (f);
  \draw[edge] (f) to (g);  
  \end{tikzpicture}
$
    \end{tabular}
    & \qquad \qquad &
    \begin{tabular}{c}
    $  \begin{tikzpicture}
  \input{\figeco/GM7nodes-Cmissing-positions}
  
  \draw[edge] (a) to (b);  \draw[lightedge] (a) to (c);  \draw[edge] (a) to (d);
  \draw[lightedge] (b) to (c);  \draw[edge] (b) to (d);  \draw[lightedge] (c) to (d);
  \draw[edge] (d) to (e);  \draw[edge] (d) to (f);  \draw[edge] (e) to (f);
  \draw[edge] (f) to (g);  
  \end{tikzpicture}
$
    \end{tabular}
    & \qquad \qquad &
    \begin{tabular}{c}
    $  \begin{tikzpicture}
  \input{\figeco/GM7nodes-Dmarginal-positions}
  
  \draw[edge] (a) to (b);  \draw[edge] (a) to (c);  \draw[lightedge] (a) to (d);
  \draw[edge] (b) to (c);  \draw[lightedge] (b) to (d);  \draw[lightedge] (c) to (d);
  \draw[lightedge] (d) to (e);  \draw[lightedge] (d) to (f);  \draw[edge] (e) to (f);
  \draw[edge] (f) to (g);  

  \draw[edgered] (a) to (e);  \draw[edgered] (b) to (e);  \draw[edgebendrightred] (c) to (e);    
  \draw[edgered] (a) to (f);  \draw[edgebendleftred] (b) to (f);  \draw[edgered] (c) to (f);  
  \end{tikzpicture}
$
    \end{tabular}
  \end{tabular}
  \end{center}
  
  \bigskip
  \ra Need to account for 'all' available information to avoid 'spurious' edges

}


