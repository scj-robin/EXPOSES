%==================================================================
\subsection*{Gaussian graphical models (GGM)}
\frame{\frametitle{Gaussian graphical models (GGM)}

  \paragraph{Gaussian setting.} Suppose we observe an 'abundance' vector $Y_i = (Y_{i1} \dots Y_{ip})$ in each site $i$ and assume that
  $$
  Y_i \sim \Ncal_p(\mu, \Sigma)
  $$
  \begin{itemize}
   \item $\mu = (p \times 1)$ vector of mean 'abundances'
   \item $\Sigma = (p \times p)$ covariance matrix:
   $$
   \sigma_{jj} = \sigma_j^2 = \Var(Y_{ij}), \qquad
   \sigma_{jk} = \Cov(Y_{ij}, Y_{ik}), \qquad
   \rho_{jk} = \frac{\sigma_{jk}}{\sigma_j \sigma_k}
   $$
  \end{itemize}

  \bigskip \pause
  \paragraph{Property.}
  $$
  \text{null correlation} \quad \Leftrightarrow \qquad
  \text{null covariance} \quad \Leftrightarrow \qquad
  \text{independence} 
  $$
  i.e.
  $$
  \sigma_{jk} = 0 \quad \Leftrightarrow \qquad Y_{ij} \independent Y_{ik} 
  \pause \emphase{\text{ marginally}}
  $$
}

%==================================================================
\frame{\frametitle{Regression point of view}

  \paragraph{Similarly to temporal networks,} finding the neighbors of a node is equivalent to find its 'parents' in a regression:
  $$
  Y_{ik} = \sum_{j \neq k} \beta_{jk} Y_{ij} + E_{ik}.
  $$
  \ra Requires a testing procedure \refer{VeV09} or a penalization \refer{MeB06} to determine non-zero coefficients

  \bigskip \bigskip \pause
  \paragraph{Remarks.}
  \begin{itemize}
   \item Regression needs reconciliation when $\widehat{\beta}_{jk} = 0$ and $\widehat{\beta}_{jk} \neq 0$ 
   \item \refer{Ver12} provides bounds for the recovery of the list of neighbors: let $k =$ degree of a given node 
   \begin{align*}
    k \log(p/k) & \leq n & & \text{possible recovery} \\
    k \log(p/k) & > n \log n & & \text{impossible recovery}
   \end{align*}
  \end{itemize}
}

%==================================================================
\frame{\frametitle{A nice property of GGM's}

  \begin{tabular}{cc}
    \begin{tabular}{p{.5\textwidth}}
    \hspace{-.15\textwidth}
    \includegraphics[width=.6\textwidth]{\fignet/FigGGM-4nodes}
    \end{tabular}
    & 
    \hspace{-.15\textwidth}
    \begin{overprint}
    \onslide<2>
    \begin{tabular}{p{.5\textwidth}}
	 \paragraph{Graphical model.}
	 $$
	 G = \left[ \begin{array}{cccc}
	 0 & 1 & 1 & \emphase{0} \\
	 1 & 0 & 1 & \emphase{0} \\
	 1 & 1 & 0 & 1 \\
	 \emphase{0} & \emphase{0} & 1 & 0 
	 \end{array} \right]
	 $$
	 \bigskip
	 \begin{itemize}
	  \item Connected
	  \item 3 separates 4 from (1, 2)
	 \end{itemize}
    \end{tabular} 
    \onslide<3>
    \begin{tabular}{p{.5\textwidth}}
	 \paragraph{Covariance matrix.}
	 $$
	 \Sigma \propto \left[ \begin{array}{cccc}
	   1 & -.25 & -.41 &  \emphase{.25} \\
	   -.25 &  1 & -.41 &  \emphase{.25} \\
	   -.41 & -.41 &  1 & -.61 \\
	   \emphase{.25} &  \emphase{.25} & -.61 &  1
	   \end{array} \right] 
	 $$
	 \bigskip
	 \begin{itemize}
	  \item No zero because $G$ is Connected
	 \end{itemize}
    \end{tabular} 
    \onslide<4>
    \begin{tabular}{p{.5\textwidth}}
	 \paragraph{Inverse covariance matrix.}
	 $$
	 \Sigma^{-1} \propto \left[ \begin{array}{cccc}
	   1 & .5 & .5 & \emphase{0} \\
	   .5 & 1 & .5 & \emphase{0} \\
	   .5 & .5 & 1 & .5 \\
	   \emphase{0} & \emphase{0} & .5 & 1
	   \end{array} \right] 
	 $$
	 \bigskip
	 \begin{itemize}
	  \item 0's at (1, 4) and (2, 4)
	  \item Conditional independence
	  \item $\Omega := \Sigma^{-1} =$ \emphase{precision} matrix
	 \end{itemize}
    \end{tabular} 
    \end{overprint}
  \end{tabular}

}

%==================================================================
\frame{\frametitle{More formally}

  \paragraph{If $Y \sim \Ncal(0, \Sigma)$,} then
  \begin{align*}
  p(Y) & \propto \exp\left(-\frac12 \|Y\|^2_{\Sigma^{-1}} \right)\\
  & = \exp\left(-\frac12 \sum_{j, k} \omega_{jk} Y_j Y_k\right)  
  & & \text{where } \Omega = [\omega_{jk}] = \Sigma^{-1} \\
  & = \prod_{j, k} \underset{\phi_{jk}(Y_j, Y_k)}{\underbrace{\exp\left(-\frac12 \omega_{jk} Y_j Y_k\right)}}
  \end{align*}

  \bigskip \pause
  \begin{itemize}
   \item The non-zeros of $\Omega$ correspond to the edges of $G$
   \item Furthermore:
   $$
   - \omega_{jk} \; \propto \; 
   \rho\left(Y_j, Y_k \mid Y_{\{j, k\}}\right)
   = \text{ \emphase{'partial'} correlation}
   $$
  \end{itemize}
  
}

%==================================================================
\frame{\frametitle{Inference}

\begin{tabular}{cc}
    \begin{tabular}{p{.5\textwidth}}
    \hspace{-.15\textwidth}
    \includegraphics[width=.6\textwidth]{\fignet/FigGGM-4nodes}
    \end{tabular}
    & 
    \hspace{-.15\textwidth}
    \begin{overprint}
    \onslide<1>
    \begin{tabular}{p{.5\textwidth}}
	 \paragraph{Graphical model.}
	 $$
	 G = \left[ \begin{array}{cccc}
	 0 & 1 & 1 & \emphase{0} \\
	 1 & 0 & 1 & \emphase{0} \\
	 1 & 1 & 0 & 1 \\
	 \emphase{0} & \emphase{0} & 1 & 0 
	 \end{array} \right]
	 $$
    \end{tabular} 
    \onslide<2>
    \begin{tabular}{p{.5\textwidth}}
	 \paragraph{Inverse covariance matrix.}
	 $$
	 \Omega \propto \left[ \begin{array}{cccc}
	   1 & .5 & .5 & \emphase{0} \\
	   .5 & 1 & .5 & \emphase{0} \\
	   .5 & .5 & 1 & .5 \\
	   \emphase{0} & \emphase{0} & .5 & 1
	   \end{array} \right] 
	 $$
	 \bigskip
	 \begin{itemize}
	  \item Same zeros as $G$
	 \end{itemize}
    \end{tabular} 
    \onslide<3>
    \begin{tabular}{p{.5\textwidth}}
	 \paragraph{Estimated inverse covariance matrix.}
	 $$
	 \widehat{\Omega} \propto \left[ \begin{array}{cccc}
	   1 & .48 & .61 & \emphase{.09} \\
	   .48 & 1 & .67 & \emphase{.06} \\
	   .61 & .67 & 1 & .46 \\
	   \emphase{.09} & \emphase{.06} & .46 & 1
	   \end{array} \right] 
	 $$
	 ($n = 100$)
	 \bigskip
	 \begin{itemize}
	  \item No 'true zero' in the estimate
	  \item Need to 'force' zeros to appear
	 \end{itemize}
    \end{tabular} 
    \end{overprint}
  \end{tabular}
}

%==================================================================
\frame{\frametitle{Making $\Omega$ sparse}

  \paragraph{Graphical lasso \refer{FHT08}.} Penalized likelihood:
  $$
  \max_{\mu, \Omega} \log p(Y; \mu, \Omega) - \lambda \sum_{j < k} |\omega_{jk}|
  $$
  the '$\ell_1$' penalty ($|\omega_{jk}|$) forces some $\omega_{jk}$ to be zero. \\
  \ra Again a convex problem \\
  \ra Fast solution ({\tt huge} R package)
  
  \bigskip \bigskip \pause
  \paragraph{Playing with penalties.}
  \begin{itemize}
   \item Other forms of penalties ($\ell_1$, $\ell_2$, combinations) induces other topologies
   \item Ex.: \refer{ACM09} induce clustered networks.
  \end{itemize}
  
  \bigskip \bigskip \pause
  \paragraph{Alternatives.}
  \begin{itemize}
   \item Mixture of tree-shaped distributions: \refer{MeJ06,Kir07,JFS16,ScR17,MRA19}
   \item ...
  \end{itemize}
}

%==================================================================
\subsection*{Latent GGM}
\frame{\frametitle{Back to ecology}

  \paragraph{Non-Gaussian data.} 
  \begin{itemize}
   \item The normal distribution does not fit count or presence/absence data
   \item Not many flexible {\sl multivariate} distributions for count or binary data do exist (\refer{IYA17})
  \end{itemize}

  \bigskip \bigskip \pause
  \paragraph{A common trick:} latent variable models
  \begin{itemize}
  \item See \refer{WBO15} for an introduction for species abundance distributions \\ ~
  \item Most popular structure = latent GGM: \\
  Spiec-Easi \refer{KMM15}, gCODA \refer{FHZ17}, MiNT \refer{BML16}, PLNnetwork \refer{CMR18b}, tree-based PLN \refer{MRA19} \\ ~
  \item Next: focus on PLNnetwork (abundance data)
  \end{itemize}
}

%==================================================================
\frame{\frametitle{Poisson log-normal model}

  \paragraph{Data:} $n$ independent sites (no spatial structure), $p$ species, 
  $$
  Y_{ij} = \text{ abundance of species $j$ in site $i$}
  $$
  
  \bigskip \pause
  \paragraph{Poisson log-normal (PLN) model \refer{AiH89}:} 
  \begin{itemize}
   \item For each site, draw independently
   $$
   Z_i \sim \Ncal_p(0, \Omega^{-1})
   $$
   \item \pause For each species in each site, draw independently (given $Z$)
   $$
   Y_{ij} \sim \Pcal(\exp(\mu_j + Z_{ij}))
   $$
  \end{itemize} \pause
  summarized as
  $$
  \{Y_i\} \text{ iid} \sim PLN(\mu, \Omega^{-1})
  $$

  \bigskip \pause
  \paragraph{Interpretation.}
  \begin{itemize}
%   \item PLN is a mixed model
  \item $\mu_j =$ mean (log-)abundance of species $j$
  \item $\Omega = $ dependency structure (encoded in the latent layer)
  \end{itemize}
}

%==================================================================
\frame{\frametitle{PLNnetworks}

  \paragraph{PLN network model.} Same model as PLN + sparsity assumption:
  $$
  \{Y_i\} \text{ iid} \sim PLN(\mu, \Omega)
  \qquad \qquad + \emphase{\Omega \text{ sparse}}
  $$
  
  \bigskip \bigskip \pause
  \paragraph{Inference algorithm.} Variational EM + sparsity inducing norm \refer{CMR18,CMR19}:
  $$
  \max_{\mu, \Omega} \widetilde{\log p}(Y; \mu, \Omega) - \lambda \sum_{j < k} |\omega_{jk}|
  $$
  \ra Alternate convex problems \\
  \ra Fast solution ({\tt PLNmodels} R package)
  
  \bigskip
  \begin{itemize}
   \item Resampling (StARS \refer{LRW10}) is highly recommended for robustness
  \end{itemize}
}

%==================================================================
\frame{\frametitle{Modeling limitation}

  \paragraph{All latent (GGM) models} infer the dependency struture of the latent $Z$, not of the observed abundances $Y$
  
  \bigskip \bigskip 
  \renewcommand{\nodesize}{2em}
  \begin{overprint}
    \onslide<2>
    $$
    \begin{array}{ccc}
    p(Z, Y) & & p(Y) \\ ~\\
    \input{\fignet/CMR18b-Fig1c}
    & \qquad \qquad &
      \begin{tikzpicture}[scale=.8]
  \node[observed] (Y1) at ( 0.95*\edgeunit,  0.31*\edgeunit) {$Y_1$};
  \node[observed] (Y2) at (-0.00*\edgeunit,  1.00*\edgeunit) {$Y_2$};
  \node[observed] (Y3) at (-0.95*\edgeunit,  0.31*\edgeunit) {$Y_3$};
  \node[observed] (Y4) at (-0.59*\edgeunit, -0.81*\edgeunit) {$Y_4$};
  \node[observed] (Y5) at ( 0.59*\edgeunit, -0.81*\edgeunit) {$Y_5$};
  \node[empty] (YY) at ( 0.59*\edgeunit, -1.51*\edgeunit) {};

  \draw[edge] (Y1) to (Y5);  \draw[edge] (Y2) to (Y3);  
  \draw[edge] (Y2) to (Y4);  \draw[edge] (Y3) to (Y4);  
  
  \end{tikzpicture}

    \end{array}
    $$
    \onslide<3>
    $$
    \begin{array}{ccc}
    p(Z, Y) & & p(Y) \\ ~\\
    \input{\fignet/CMR18b-Fig1a}
    & \qquad \qquad &
    \input{\fignet/CMR18b-Fig1b}
    \end{array}
    $$
  \end{overprint}
  \renewcommand{\nodesize}{\commonnodesize}
}

%==================================================================
\frame{\frametitle{Illustration}

  \paragraph{Barents fish dataset:} $n = 89$ sites, $p = 30$ species
  
  \bigskip
  \begin{tabular}{p{.45\textwidth}p{.45\textwidth}}
    \begin{tabular}{c}
      \vspace{.04\textheight}
      Regularization path ($\lambda$) \\
      \includegraphics[width=.45\textwidth, height=.45\textheight]{\figbarents/BarentsFish_Gnull_criteria}
    \end{tabular}
    &
%     \hspace{-.1\textwidth}
    \begin{overprint}
      \onslide<2>
        \begin{tabular}{c}
          Covariance ($\widehat{\Sigma}$) \\ 
          \includegraphics[width=.4\textwidth]{\figbarents/BarentsFish_Gnull_sigma}
        \end{tabular}
      \onslide<3>
        \begin{tabular}{c}
          Precision ($\widehat{\Omega}$) \\ 
          \includegraphics[width=.4\textwidth]{\figbarents/BarentsFish_Gnull_omega}
        \end{tabular}
      \onslide<4>
        \begin{tabular}{c}
          Network ($\widehat{G}$) \\ 
          \includegraphics[width=.4\textwidth]{\figbarents/BarentsFish_Gnull_network}
        \end{tabular}
    \end{overprint}
  \end{tabular}

  {\tt PLNnetwork} R package  \refer{CMR19}: \qquad
  {\tt PLNnetwork(Y $\sim$  1)}
}

%==================================================================
\subsection*{Accounting for covariates}
\frame{\frametitle{Accounting for covariates}

  \paragraph{Aim of network inference:} determine which species are 'in direct interaction' \\
  \ra Avoid 'spurious edges'
  
  \hspace{-.05\textwidth}
  \begin{tabular}{p{.45\textwidth}p{.45\textwidth}}
    \begin{tabular}{p{.4\textwidth}}
      \onslide+<2->{
        \paragraph{Environmental variations} may jointly affect species, which do not actually interact \\
        }
      \onslide+<4->{
        \bigskip \bigskip 
        \paragraph{Introducing covariates.} ~ \\
        $x_i =$ vector of covariates for site $i$ \\
        e.g. $x_i = (\text{temperature}, \text{altitude}, ...)$
        }
    \end{tabular}
    &
%     \hspace{-.1\textwidth}
    \begin{tabular}{p{.45\textwidth}}
      \onslide+<3->{
        \includegraphics[width=.4\textwidth]{\figbarents/BarentsFish_Sigmanull_Sigmafull}
        }
    \end{tabular}
  \end{tabular}
  
  \onslide+<4->{
    Add a regression term in the PLN model:
    $$
    Y_{ij} \mid Z_{ij} \sim \Pcal(\exp(\emphase{x_i^\intercal \beta_j} + Z_{ij}))
    $$
    $\beta_j =$ vector of environmental effects on species $j$
    }

}

%====================================================================
\frame{\frametitle{Barents fish} 

  \begin{center}
  \begin{tabular}{lccc}
    & no covariate & \textcolor{blue}{temp. \& depth} & \textcolor{red}{all covariates} \\
    \hline
    \rotatebox{90}{$\qquad\quad\lambda=.20$} &
    \includegraphics[width=.22\textwidth]{\figCMR/network_BarentsFish_Gnull_full60edges} &
    \includegraphics[width=.22\textwidth]{\figCMR/network_BarentsFish_Gsel_full60edges} &
    \includegraphics[width=.22\textwidth]{\figCMR/network_BarentsFish_Gfull_full60edges} 
    \vspace{-0.05\textheight} \\ \hline
    %
    \rotatebox{90}{$\qquad\quad\lambda=.28$} &
    \includegraphics[width=.22\textwidth]{\figCMR/network_BarentsFish_Gnull_sel60edges} & \includegraphics[width=.22\textwidth]{\figCMR/network_BarentsFish_Gsel_sel60edges} &
    \includegraphics[width=.22\textwidth]{\figCMR/network_BarentsFish_Gfull_sel60edges} 
    \vspace{-0.05\textheight} \\ \hline
    %
    \rotatebox{90}{$\qquad\quad\lambda=.84$} &
    \includegraphics[width=.22\textwidth]{\figCMR/network_BarentsFish_Gnull_null60edges} &
    \includegraphics[width=.22\textwidth]{\figCMR/network_BarentsFish_Gsel_null60edges} &
    \includegraphics[width=.22\textwidth]{\figCMR/network_BarentsFish_density} 
%     \includegraphics[width=.22\textwidth]{\figCMR/network_BarentsFish_Gfull_null60edges}  
  \end{tabular}
  \end{center}

}

