%==================================================================
%==================================================================
\subsection*{Erd\"os model}
%==================================================================
\frame{ \frametitle{The simplest one}

  \paragraph{Erd\"os model $\Gcal(n, \pi)$\footnote{Alternate version $\Gcal(n, m)$, $m =$ fixed number of edges}:} 
  \begin{itemize}
   \item $n$ (fixed) nodes (= species) 
   \item undirected binary edges (= interactions) 
   \item all pairs of nodes are connected (= interact) independently 
   \item with same probability $\pi$
  \end{itemize}
  $$
  \{Y_{ij}\} \text{ iid } \sim \Bcal(\pi)
  $$
  
  \bigskip \bigskip \pause
  \paragraph{Properties.} ~
  \begin{itemize}
   \item Graph density $\pi$ 
   \item Degree of a node $\sim \Bcal(n-1, \pi) \approx \Pcal(n \pi)$ %\\
   \ra Homegeneous degree distribution
   \item Clustering coefficient $c = P\{j \sim k \mid i \sim j, i \sim k\} = \pi$ %\\
   \ra No 'assortativity'
  \end{itemize}

%   \bigskip \bigskip \pause
%   \paragraph{Remark:} 
%   $$
%   Y \sim \Ucal\{\text{set of graphs with $n$ nodes and $m$ edges}\}
%   $$
%   \begin{itemize}
%    \item Density $m/N$, where $N = n(n-1)/2 =$ number of pairs
%    \item Edges are not independent
%   \end{itemize}
}

%==================================================================
%==================================================================
\subsection*{Degree-based models}
%==================================================================
\frame{ \frametitle{Degree-based models}

  \bigskip 
  \paragraph{Aim:} account for heterogeneous number of neighbors $D_i = \sum_{j \neq i} Y_{ij}$
  
  \bigskip \bigskip \pause
  \hspace{-.04\textwidth}
  \begin{tabular}{p{.45\textwidth}p{.45\textwidth}}  
    \begin{tabular}{p{.45\textwidth}}
      \paragraph{Expected degree distribution (EDD) \refer{ChL02}:} \\
      $$
      p_{ij} := \Pr\{i \sim j\} = d_i d_j / \lambda 
      $$
      $d_i =$ observed degree of node $i$, \\ ~ \\
      $\lambda = \sum_i d_i 
      \quad \Rightarrow \quad 
      \emphase{\Esp D_i = d_i}$ \\ ~ \\
      ($p_{ij}$ may exceed 1...) \\ ~ \\ 
      ~ \\ 
      \onslide+<3>{
      \paragraph{Configuration model} (looks similar but actually different): requires
      $$
      D_i = d_i
      $$
      }
    \end{tabular}
    &
    \hspace{-.05\textwidth}
    \begin{tabular}{p{.45\textwidth}}
      \includegraphics[width=.5\textwidth]{\figeco/florida-EDD-simul.pdf}
    \end{tabular} 
  \end{tabular}

}

%==================================================================
\frame{ \frametitle{EDD for oriented graphs}

  \bigskip
  \paragraph{Degrees:} out-degree $D^+_i = \sum_j Y_{ij}$, in-degree  $D^-_i = \sum_j Y_{ji}$

  \bigskip \bigskip \pause
  \hspace{-.04\textwidth}
  \begin{tabular}{p{.45\textwidth}p{.45\textwidth}}  
    \begin{tabular}{p{.4\textwidth}}
      Connexion probability
      $$
      p_{ij} := \Pr\{i \rightarrow j\} = d^+_i d^-_j / \lambda 
      $$
      \bigskip
      Expected degrees:
      $$
      \Esp D^+_i = d^+_i, 
      \quad 
      \Esp D^-_i = d^-_i
      $$ 
      
      \bigskip
      Only accounts for 'generalists' vs 'specialists' (see top left node) \\
      \bigskip
      \onslide+<3>{
      \paragraph{Again:} similar but different from 'edge rewiring', which imposes
      $$
      D^+_i = d^+_i, \quad D^-_i = d^-_i
      $$}
    \end{tabular}
    &
    \hspace{-.05\textwidth}
    \begin{tabular}{p{.45\textwidth}}
      \includegraphics[width=.5\textwidth]{\figeco/foodweb-baydry-EDD-simul.pdf}
    \end{tabular} 
  \end{tabular}

}

%==================================================================
%==================================================================
\subsection*{Latent-space models}
%==================================================================
\frame{ \frametitle{Latent-space models}

  \paragraph{Fact:} observed networks are far from 'random' or 'uniform' (i.e. Erd\"os )
  
  \bigskip \bigskip 
  \paragraph{Rational:} the observed heterogeneity is due to (unobserved) node specificities

  \bigskip \bigskip \pause
  \paragraph{General framework:} latent space models \refer{BJR07,MaR14}
  \begin{itemize}
   \item A latent (= hidden = unobserved) variable $Z_i$ is associated with each node
   \item The connections are independent conditionally on $Z = \{Z_i\}$:
   $$
   \{Y_{ij}\} \text{ indep. } \mid Z:
   \qquad 
   P(Y_{ij} = 1) = \gamma(Z_i, Z_j)
   $$
  \end{itemize}
  
  \bigskip \bigskip \pause
  \paragraph{Exchangeable graphs:} provided that the $Z_i$'s are iid, for any permutation $\sigma$,
  $$
  p(\{Y_{ij}\}) = p(\{Y_{\emphase{\sigma}(i)\emphase{\sigma}(j)}\})
  $$

}

%==================================================================
\frame{ \frametitle{Latent positions}

  \paragraph{Latent position model \refer{HRH02}.} $Z_i \in \Rbb^d$,
  $$
  \log \frac{p_{ij}}{1 - p_{ij}} = \alpha - \|Z_i-Z_j\|
  $$

  \bigskip \pause
  \begin{tabular}{p{.45\textwidth}p{.45\textwidth}}
%     \hspace{-.1\textwidth}
    \begin{tabular}{p{.45\textwidth}}
      Latent positions: \\
      \includegraphics[width=.4\textwidth]{\fignet/LatentPositionModel-Network}
    \end{tabular} 
    &
%     \hspace{-.1\textwidth}
    \begin{tabular}{p{.45\textwidth}}
      Observed data: \\
      \includegraphics[width=.4\textwidth]{\fignet/LatentPositionModel-Adjacency}
    \end{tabular} 
  \end{tabular}
  
  \pause
  \paragraph{Clustering version:} \refer{HRT07}

}

%==================================================================
\frame{ \frametitle{Stochastic blockmodel (SBM)}

  \bigskip
  \paragraph{Mixture model \refer{HoL79,NoS01}.} $Z_i \in \{1, \dots K\}$ (in proportion $\pi_1, \dots \pi_K$),
  $$
  p_{ij} = \gamma_{Z_i Z_j}
  $$

  \pause
  \begin{tabular}{p{.45\textwidth}p{.45\textwidth}}
%     \hspace{-.1\textwidth}
    \begin{tabular}{p{.45\textwidth}}
      Latent groups: \\
      \includegraphics[width=.4\textwidth]{\fignet/SBM-Network} \\
%         $[\gamma_{k\ell}] =$
%         \includegraphics[width=.1\textwidth]{\fignet/SBM-ConnexionProba} \\
      \begin{tabular}{ll}
        $[\gamma_{k\ell}] =$
        &
        \includegraphics[width=.1\textwidth]{\fignet/SBM-ConnexionProba}
      \end{tabular}
    \end{tabular} 
    &
%     \hspace{-.1\textwidth}
    \begin{tabular}{p{.45\textwidth}}
      Observed data: \\
      \includegraphics[width=.4\textwidth]{\fignet/SBM-Adjacency}
    \end{tabular} 
  \end{tabular}
  
}

%==================================================================
\frame{ \frametitle{A generic framework: $W$-graph}

  \paragraph{Limit for dense graphs \refer{LoS06}.} $Z_i \sim \Ucal[0, 1]$, \emphase{graphon} function $\gamma: [0, 1]^2 \mapsto [0, 1]$ 
  $$
  p_{ij} = \gamma(Z_i, Z_j)
  $$

  \bigskip \pause
  \begin{tabular}{p{.45\textwidth}p{.45\textwidth}}
%     \hspace{-.1\textwidth}
    \begin{tabular}{p{.45\textwidth}}
      Graphon function: \\
      \includegraphics[trim=30 30 30 30, width=.4\textwidth, clip]{\fignet/Graphon-Function}
    \end{tabular} 
    &
%     \hspace{-.1\textwidth}
    \begin{tabular}{p{.45\textwidth}}
      Observed network: \\
      \includegraphics[width=.4\textwidth]{\fignet/Graphon-Network}
    \end{tabular} 
  \end{tabular}
  
}

%==================================================================
%==================================================================
\subsection*{Accounting for covariates}
%==================================================================
\frame{ \frametitle{Accounting for covariates}

  \paragraph{Same story:} accounting for available information on nodes ($x_i$) or pairs ($x_{ij}$) may avoid to re-discover the wheel
  
  \bigskip \pause
  \paragraph{Including covariates} is often easy (at least formally):
  \begin{align*}
   \log\frac{p_{ij}}{1 - p_{ij}} = \;
   & \emphase{x_{ij}^\intercal \beta} + \|Z_i - Z_j\| & & \text{latent position model} \\
   & \emphase{x_{ij}^\intercal \beta} + \gamma_{Z_i, Z_j} & & \text{stochastic block model} \\
   & \emphase{x_{ij}^\intercal \beta} + \gamma(Z_i, Z_j) & & \text{$W$ graph} 
  \end{align*}
  
  \bigskip \pause
  \begin{itemize}
   \item Aims at analysing the structure of the network, that is {\sl not explained} by the covariates ('residual' structure)
   \item The edges are not exchangeable anymore
   \item Easier to account for edge covariates ($x_{ij}$), than node covariates ($x_i$)
  \end{itemize}

}


%==================================================================
%==================================================================
\subsection*{Statistical inference}
%==================================================================
\frame{ \frametitle{Statistical inference}

  \paragraph{Exercise:} draw the graphical model of latent variable model for graph
  
  \bigskip
  \hspace{-0.05\textwidth}
  \begin{tabular}{p{.6\textwidth}p{.4\textwidth}}
    \begin{tabular}{p{.5\textwidth}}
      \onslide+<2->{\paragraph{Solution} for $p(Y, Z)$ ~\\}
      \bigskip
      \onslide+<3->{
        \paragraph{Incomplete data model:} $Z$ is not observed \\ ~\\
        \ra Standard statistical inference (e.g. EM algorithm) requires to compute \emphase{$p(Z \mid Y)$} ~\\
        }
      \bigskip
      \onslide+<4->{
        \paragraph{Graph moralization:} the dependency structure of $p(Z \mid Y)$ is (very) intricate \\ ~\\
        \ra Need to resort to Monte-Carlo sampling \refer{NoS01,MSP05} or variational approximations \refer{GoN05,DPR08}
        }
    \end{tabular}
    &
    \begin{tabular}{p{.4\textwidth}}
      \begin{overprint}
        \onslide<2-3>
%           \begin{centering}
            \renewcommand{\nodesize}{1.5em}
              \begin{tikzpicture}
  \node[hidden] (Z1) at (0, \edgeunit) {$Z_1$};
  \node[hidden] (Z2) at (\edgeunit, \edgeunit) {$Z_2$};
  \node[hidden] (Z3) at (0, 0) {$Z_3$};
  \node[hidden] (Z4) at (\edgeunit, 0) {$Z_4$};

  \node[observed] (Y12) at (.5*\edgeunit, 1.75*\edgeunit) {$Y_{12}$};
  \node[observed] (Y13) at (-.75*\edgeunit, .5*\edgeunit) {$Y_{13}$};
  \node[observed] (Y14) at (1.75*\edgeunit, 1.75*\edgeunit) {$Y_{14}$};
  \node[observed] (Y23) at (-.75*\edgeunit, 1.75*\edgeunit) {$Y_{23}$};
  \node[observed] (Y24) at (1.75*\edgeunit, .5*\edgeunit) {$Y_{24}$};
  \node[observed] (Y34) at (.5*\edgeunit, -.75*\edgeunit) {$Y_{34}$};
  
  \draw[arrow] (Z1) to (Y12);  \draw[arrow] (Z2) to (Y12);
  \draw[arrow] (Z1) to (Y13);  \draw[arrow] (Z3) to (Y13);
  \draw[arrow] (Z1) to (Y14);  \draw[arrow] (Z4) to (Y14);
  \draw[arrow] (Z2) to (Y23);  \draw[arrow] (Z3) to (Y23);
  \draw[arrow] (Z2) to (Y24);  \draw[arrow] (Z4) to (Y24);
  \draw[arrow] (Z3) to (Y34);  \draw[arrow] (Z4) to (Y34);
  \end{tikzpicture}

 
%           \end{centering}       
        \onslide<4>
%           \begin{centering}
            \renewcommand{\nodesize}{1.5em}
              \begin{tikzpicture}
  \node[hidden] (Z1) at (0, \edgeunit) {$Z_1$};
  \node[hidden] (Z2) at (\edgeunit, \edgeunit) {$Z_2$};
  \node[hidden] (Z3) at (0, 0) {$Z_3$};
  \node[hidden] (Z4) at (\edgeunit, 0) {$Z_4$};

  \node[observed] (Y12) at (.5*\edgeunit, 1.75*\edgeunit) {$Y_{12}$};
  \node[observed] (Y13) at (-.75*\edgeunit, .5*\edgeunit) {$Y_{13}$};
  \node[observed] (Y14) at (1.75*\edgeunit, 1.75*\edgeunit) {$Y_{14}$};
  \node[observed] (Y23) at (-.75*\edgeunit, 1.75*\edgeunit) {$Y_{23}$};
  \node[observed] (Y24) at (1.75*\edgeunit, .5*\edgeunit) {$Y_{24}$};
  \node[observed] (Y34) at (.5*\edgeunit, -.75*\edgeunit) {$Y_{34}$};
  
  \draw[lightarrow] (Z1) to (Y12);  \draw[lightarrow] (Z2) to (Y12);
  \draw[lightarrow] (Z1) to (Y13);  \draw[lightarrow] (Z3) to (Y13);
  \draw[lightarrow] (Z1) to (Y14);  \draw[lightarrow] (Z4) to (Y14);
  \draw[lightarrow] (Z2) to (Y23);  \draw[lightarrow] (Z3) to (Y23);
  \draw[lightarrow] (Z2) to (Y24);  \draw[lightarrow] (Z4) to (Y24);
  \draw[lightarrow] (Z3) to (Y34);  \draw[lightarrow] (Z4) to (Y34);
  
  \draw[dashededge] (Z1) to (Z2);  \draw[dashededge] (Z1) to (Z3);
  \draw[dashededge] (Z1) to (Z4);  \draw[dashededge] (Z2) to (Z3);
  \draw[dashededge] (Z2) to (Z4);  \draw[dashededge] (Z3) to (Z4);
  \end{tikzpicture}

 
%           \end{centering}       
      \end{overprint}
    \end{tabular} 
  \end{tabular}

}
