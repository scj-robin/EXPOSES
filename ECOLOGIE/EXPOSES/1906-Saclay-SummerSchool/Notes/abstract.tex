In the first lecture, we will consider the problem of network inference (or reconstruction), that is the attempt to establish which species directly interact with each other. The typical situation is the following: considering a series of similar sites and measuring the abundance of a same set of species in each site, one aim at recovering the 'ecological network', that is the graph where two species are linked if they have a direct interaction with each other. We will introduce the statistical formulation of this problem, which relies on probabilistic graphical models where 'interaction' is equivalent to conditional dependence. We will then describe a series of methods that have been proposed for the reconstruction of ecological network based on abundance data, which rely on (latent) Gaussian graphical models.

The second lecture will be dedicated to the analysis of the organization of an ecological network, also called topological analysis. This problem arises when considering a network encoding the interactions between a (possibly large) set of species. A typical task is then to cluster species, that is to find groups of them, having a similar connectivity profile. We will introduce the stochastic block-model and the latent block-model, which are both dedicated to this task for undirected and directed interaction networks, respectively. We will briefly introduce the statistical issues raised by these models and illustrate there use to analyze various ecological networks.

