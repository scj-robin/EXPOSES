Network = nice way to represent ecological 'interactions' \cite{Bas09}

%******************************************************************************
\subsection{Different types of networks}

\begin{itemize}
 \item Trophic, plant-pollinator, 'interaction'
 \item Directed / undirected, binary / valued ('weighted') / multivariate ('multiplex')
 \item Notations: $i$, $j$, $Y_{ij}$
\end{itemize}

%******************************************************************************
\subsection{A series of indicators}

\begin{itemize}
 \item Density, connectivity, centrality, nestedness: \cite{BJM03}
 \item ... to be interpreted, most often in the framework of an (implicit) (population) dynamic model (e.g.Lotka-Volterra): \cite{ThF10}, \cite{GiB12}
 \item \cite{Bas09}, 
 \item Global: clusters
 \item Local: motifs
\end{itemize}

