\documentclass[11pt]{beamer}

% Beamer style
%\usetheme[secheader]{Madrid}
% \usetheme{CambridgeUS}
\useoutertheme{infolines}
\usecolortheme[rgb={0.65,0.15,0.25}]{structure}
% \usefonttheme[onlymath]{serif}
\beamertemplatenavigationsymbolsempty
%\AtBeginSubsection

% Packages
%\usepackage[french]{babel}
\usepackage[latin1]{inputenc}
\usepackage{color}
\usepackage{xspace}
\usepackage{dsfont, stmaryrd}
\usepackage{amsmath, amsfonts, amssymb}
\usepackage{epsfig}
\usepackage{tikz}
\usepackage{url}
\usepackage{/home/robin/LATEX/Biblio/astats}
%\usepackage[all]{xy}
\usepackage{graphicx}

% Commands
% TikZ
\newcommand{\nodesize}{1.8em}
\newcommand{\edgeunit}{2.2*\nodesize}
\tikzstyle{hidden}=[draw, circle, fill=gray!50, minimum width=\nodesize, inner sep=0]
\tikzstyle{observed}=[draw, circle, minimum width=\nodesize, inner sep=0]
\tikzstyle{eliminated}=[draw, circle, minimum width=\nodesize, color=gray!50, inner sep=0]
\tikzstyle{empty}=[]
\tikzstyle{arrow}=[->, >=latex, line width=1pt]
\tikzstyle{edge}=[-, line width=1pt]
\tikzstyle{dashedarrow}=[->, >=latex, dashed, line width=1pt]
\tikzstyle{lightarrow}=[->, >=latex, line width=1pt, fill=gray!50, color=gray!50]


\definecolor{darkred}{rgb}{0.65,0.15,0.25}
\newcommand{\emphase}[1]{\textcolor{darkred}{#1}}
% \newcommand{\emphase}[1]{{#1}}
\newcommand{\paragraph}[1]{\textcolor{darkred}{#1}}
\newcommand{\refer}[1]{{\small{\textcolor{gray}{{[\cite{#1}]}}}}}
% \newcommand{\Refer}[1]{{\small{\textcolor{gray}{{[#1]}}}}}
\renewcommand{\newblock}{}

% Symbols
\newcommand{\Abf}{{\bf A}}
\newcommand{\Beta}{\text{B}}
\newcommand{\Bcal}{\mathcal{B}}
\newcommand{\BIC}{\text{BIC}}
\newcommand{\Ccal}{\mathcal{C}}
\newcommand{\dd}{\text{~d}}
\newcommand{\dbf}{{\bf d}}
\newcommand{\Dcal}{\mathcal{D}}
\newcommand{\Esp}{\mathbb{E}}
\newcommand{\Ebf}{{\bf E}}
\newcommand{\Ecal}{\mathcal{E}}
\newcommand{\Gcal}{\mathcal{G}}
\newcommand{\Gam}{\mathcal{G}\text{am}}
\newcommand{\Hcal}{\mathcal{H}}
\newcommand{\Ibb}{\mathbb{I}}
\newcommand{\Ibf}{{\bf I}}
\newcommand{\ICL}{\text{ICL}}
\newcommand{\Cov}{\mathbb{C}\text{ov}}
\newcommand{\Corr}{\mathbb{C}\text{orr}}
\newcommand{\Var}{\mathbb{V}}
\newcommand{\Vsf}{\mathsf{V}}
\newcommand{\pen}{\text{pen}}
\newcommand{\Fcal}{\mathcal{F}}
\newcommand{\Hbf}{{\bf H}}
\newcommand{\Jcal}{\mathcal{J}}
\newcommand{\Kbf}{{\bf K}}
\newcommand{\Lcal}{\mathcal{L}}
\newcommand{\Mcal}{\mathcal{M}}
\newcommand{\mbf}{{\bf m}}
\newcommand{\mum}{\mu(\mbf)}
\newcommand{\Ncal}{\mathcal{N}}
\newcommand{\Nbf}{{\bf N}}
\newcommand{\Nm}{N(\mbf)}
\newcommand{\Ocal}{\mathcal{O}}
\newcommand{\Obf}{{\bf 0}}
\newcommand{\Omegas}{\underset{s}{\Omega}}
\newcommand{\Pbf}{{\bf P}}
\newcommand{\pt}{\widetilde{p}}
\newcommand{\Pt}{\widetilde{P}}
\newcommand{\Pcal}{\mathcal{P}}
\newcommand{\Qcal}{\mathcal{Q}}
\newcommand{\Rbb}{\mathbb{R}}
\newcommand{\Rcal}{\mathcal{R}}
\newcommand{\Scal}{\mathcal{S}}
\newcommand{\Tcal}{\mathcal{T}}
\newcommand{\Ucal}{\mathcal{U}}
\newcommand{\Vcal}{\mathcal{V}}
\newcommand{\BP}{\text{BP}}
\newcommand{\EM}{\text{EM}}
\newcommand{\VEM}{\text{VEM}}
\newcommand{\VBEM}{\text{VBEM}}
\newcommand{\cst}{\text{cst}}
\newcommand{\obs}{\text{obs}}
\newcommand{\ra}{\emphase{\mathversion{bold}{$\rightarrow$}~}}
%\newcommand{\transp}{\text{{\tiny $\top$}}}
\newcommand{\transp}{\text{{\tiny \mathversion{bold}{$\top$}}}}
\newcommand{\logit}{\text{logit}\xspace}

% Directory
\newcommand{\fignet}{/home/robin/Bureau/RECHERCHE/RESEAUX/EXPOSES/FIGURES}
\newcommand{\figchp}{/home/robin/Bureau/RECHERCHE/RUPTURES/EXPOSES/FIGURES}


%====================================================================
%====================================================================

%====================================================================
%====================================================================
\begin{document}
%====================================================================
%====================================================================

%====================================================================
\title[LBM for metagenomics]{Latent Block Model for Overdispersed Count Data
Application in Microbial Ecology}

\author[S. Robin]{S. Robin \\ ~\\
  \begin{tabular}{ll}
    Joint work with \underline{J. Aubert}, S. Schbath
  \end{tabular}
  }

\institute[INRA / AgroParisTech]{~ \\%INRA / AgroParisTech \\
  \vspace{-.1\textwidth}
  \begin{tabular}{ccc}
    \includegraphics[height=.25\textheight]{\fignet/LogoINRA-Couleur} & 
    \hspace{.02\textheight} &
    \includegraphics[height=.06\textheight]{\fignet/logagroptechsolo} % & 
%     \hspace{.02\textheight} &
%     \includegraphics[height=.09\textheight]{\fignet/logo-ssb}
    \\ 
  \end{tabular} \\
  \bigskip
  }

\date[Mar. 2017, AgroParisTech]{ABS4NGS, Mar. 2017, AgroParisTech}

%====================================================================
%====================================================================
\maketitle
%====================================================================

%====================================================================
%====================================================================
\section{Bi-clustering for metagenomics}
%====================================================================
% \frame{\frametitle{}}

%====================================================================
\subsection{Metagnomics experiment}
%====================================================================
\frame{\frametitle{A typical metagenomic experiment}

  \paragraph{Amplicon-based sampling.} Consider 
  \begin{itemize}
   \item $n$ different (bacterial, fungal, ...) species / OTU and
   \item $m$ different samples / patients / media / conditions.
  \end{itemize}

  \bigskip \bigskip 
  NGS provides
  \begin{eqnarray*}
     Y_{ij} & = & \text{number of reads from species $i$ in sample $j$} \\
	 & \propto & \text{abundance of species $i$ in sample $j$}
  \end{eqnarray*}

  \bigskip \bigskip 
  \paragraph{Question.} 
  Can we exhibit some patterns in the distribution of the species abundances across samples?
}

%====================================================================
\subsection{Need for dedicated bi-clustering methods}
%====================================================================
\frame{\frametitle{Bi-clustering problem}

  \paragraph{Rephrased problem:} Find 
  \begin{itemize}
   \item groups of species having similar abundance profile across the samples and 
   \item groups of samples histing the different species in similar proportions.
  \end{itemize}
  
  \bigskip \bigskip \pause
  \paragraph{Bi-clustering problem:} Simultaneously determine
  \begin{itemize}
   \item row clusters and
   \item column clusters
  \end{itemize}
  in a $n \times m$ matrix of counts.
}

%====================================================================
\frame{\frametitle{Approach}

  \paragraph{Model-based clustering:}
  $$
  \rightarrow \text{LBM = Latent Block-Model \refer{GoN05,BrM15}}
  $$
  
  \bigskip \bigskip \pause
  \paragraph{Specificities of NGS data:}
  \begin{itemize}
   \item count data,
   \item over dispersed (wrt Poisson),
   \item with heterogeneous sampling effort (= sequencing depth), 
   \item with high variation among the species abundances,
   \item \textcolor{gray}{possibly with replicates}.
  \end{itemize}
}

%====================================================================
%====================================================================
\section{Model and inference}
%====================================================================

%====================================================================
\subsection{Model}
%====================================================================
\frame{\frametitle{Latent Block Model}

  \paragraph{Bi-clustering.} $K$ species groups, $G$ sample groups
  \begin{itemize}
   \item $Z_i =$ group to which species $i$ belongs to ($\in \{1, ... K\}$);
   \item $W_j =$ group to which sample $j$ belongs to ($\in \{1, ... G\}$)
  \end{itemize}
  both latent = hidden = unobserved.
  
  \bigskip
  \ra Incomplete data model

  \bigskip \bigskip \pause
  \paragraph{Ex: Poisson LBM.}
  \begin{eqnarray*}
   (Z_i) \text{ iid} & \sim & \pi \qquad \qquad (\text{species prop.}) \\
   (W_j) \text{ iid} & \sim & \rho \qquad \qquad (\text{sample prop.}) \\
   (Y_{ij}) \text{ indep } | (Z_i); (W_j) & \sim & \Pcal(\lambda_{Z_iW_j}) 
  \end{eqnarray*}
  Does not accommodate for NGS data specificities.
}

%====================================================================
\frame{\frametitle{Over-dispersion}

  \paragraph{Negative-binomial.} Most popular distribution of NGS counts:
  $$
  Y \sim \Ncal\Bcal(\lambda, \phi) 
  \qquad \quad 
  \Esp Y = \lambda, 
  \qquad
  \Var Y = \lambda (1 + \phi \lambda) \geq \lambda.   
  $$
  
  \bigskip \bigskip \pause
  \paragraph{Gamma-Poisson representation.} Take $a = 1/\phi$ and draw
  $$
  U \sim \Gam(a, a), 
  \quad
  Y \; | \; U \sim \Pcal(\lambda U)
  \qquad \Rightarrow \qquad 
  Y \sim \Ncal\Bcal(\lambda, \phi).
  $$
  Negative binomial = Poisson with latent Gamma
  
  \bigskip \bigskip 
  \ra Incomplete data model ($Y$ is observed, $U$ is not) \refer{BPR15}.
}

%====================================================================
\frame{\frametitle{LBM for metagenomic data}

  \paragraph{Hidden layer:}
  \begin{eqnarray*}
   (Z_i) \text{ iid} & \sim & \pi \qquad \qquad (\text{species prop.}) \\
   (W_j) \text{ iid} & \sim & \rho \qquad \qquad (\text{sample prop.}) \\
   (U_{ij}) \text{ iid} & \sim & \Gam(a, a) 
  \end{eqnarray*}
  
  \bigskip \pause
  \paragraph{Observed counts:} (interest of model-based approaches)
  $$
  Y_{ij} \; | \; Z, W, U \sim \Pcal\left(\mu_i \; \nu_j \; \alpha_{Z_i W_j} \; U_{ij} \right)
  $$
  where
  \begin{itemize}
   \item $\mu_i:$ mean abundance of species $i$
   \item $\nu_j:$ sequencing depth in sample $j$ (fixed)
   \item \emphase{$\alpha_{k\ell}:$ interaction term between group species $k$ and sample group $\ell$.}
  \end{itemize}
}

%====================================================================
\subsection{Variational inference}
%====================================================================
\frame{\frametitle{Inference}

  \paragraph{Aim:} Retrieve 
  \begin{itemize}
   \item $Z_i =$ species group, or at least $P(i \in k | Y)$;
   \item $W_j =$ sample group, or at least $P(j \in \ell | Y)$;
  \end{itemize}
  and estimate the interaction parameter $\alpha = (\alpha_{k\ell})$.

  \bigskip \bigskip \pause
  \paragraph{Which means} (maximum-likelihood approach)
  \begin{itemize}
   \item Compute $p(Z, W, U |Y)$;
   \item Maximize $\log p_\theta(Y)$, where $\theta = (\alpha, \mu)$.
  \end{itemize}
  
  \bigskip \bigskip 
  Most popular algorithm: EM \refer{DLR77}.

}

%====================================================================
\frame{\frametitle{Variational approximation}

  Species group $Z_i$ and sample group $W_j$ are not independent given $Y_{ij}$
  $$
  \text{\ra $p(Z, W, U \; | \; Y)$ intractable}
  $$
  
  \bigskip 
  \paragraph{Variational approximation \refer{WaJ08,Jaa00}.} Find
  \begin{eqnarray*}
  \pt(Z, W, U) & \simeq & p(Z, W, U |Y) \\ 
  \text{such that} \qquad
  \pt(Z, W, U) & = & \pt_1(Z) \; \pt_2(W) \; \pt_3(U)
  \end{eqnarray*}
  (mean-field approximation).

%   $$
%   \pt(Z, W, U) \simeq p(Z, W, U |Y)
%   $$
%   such that
%   $$
%   \pt(Z, W, U) = \pt_1(Z) \; \pt_2(W) \; \pt_3(U).
%   $$
%   (mean-field approximation)
  
  \bigskip \bigskip 
  \ra Variational EM (VEM) algorithm provide a lower bound 
  $$
  J(Y, \pt, \widehat{\theta}) \leq \log p_{\widehat{\theta}}(Y).
  $$

}

%====================================================================
\subsection{Model selection}
%====================================================================
\frame{\frametitle{Penalized 'likelihood' criteria}

  \paragraph{Penalized criterion.} $\log p_{\widehat{\theta}}(Y)$ intractable
  $$
  \log p_{\widehat{\theta}}(Y) - \text{pen}(p_{\widehat{\theta}})
  \qquad \rightarrow \qquad
  J(Y, \pt, \widehat{\theta}) - \text{pen}(p_{\widehat{\theta}})
  $$
  
  \bigskip \bigskip 
  \paragraph{BIC \& ICL.} $\Hcal =$ entropy
  \begin{eqnarray*}
   \text{pen}_{BIC} & = & \left[ (K-1) \log n - (G-1) \log m - KG \log (nm) \right] / 2\\
   ~ \\
   \text{pen}_{ICL_1} & = & \text{pen}_{BIC} + \Hcal(\pt_Z)  + \Hcal(\pt_W) \qquad (\text{classif. entropy}) \\
   ~ \\
   \text{pen}_{ICL_2} & = & \text{pen}_{ICL_1} + \Hcal(\pt_U) \qquad \qquad \quad (\text{dispersion entropy})
  \end{eqnarray*}
}

%====================================================================
%====================================================================
\section{Illustrations}
%====================================================================
\frame{\frametitle{Global patterns}

  \paragraph{Dataset:} \refer{McH13}
  \begin{itemize}
   \item $n = 215$ OTU
   \item $m = 26$ samples (ocean, soil, human body, ..., several of each)
  \end{itemize}

  \bigskip \bigskip 
  \paragraph{Results:}
  \begin{itemize}
   \item $\widehat{K} = 22$ groups of bacteria
   \item $\widehat{G} = 16$ groups of plants
  \end{itemize}

  \bigskip \bigskip 
  \paragraph{Comments:} 
  \begin{itemize}
   \item Sample group retrieve the sample types
   \item Clear interaction pattern
  \end{itemize}
}

%====================================================================
\frame{\frametitle{Global patterns}
  \vspace{-.05\textheight}
  $$
  \includegraphics[height=.9\textheight]{../FIGURES/ARS17-GlobalPatterns}
  $$
}

%====================================================================
\frame{\frametitle{Meta-rhizo}

  \paragraph{Dataset:}
  \begin{itemize}
   \item $n = 288$ bacteria (genus)
   \item $m = 483$ samples = rhizosphere of different plants (genotypes)
   \item $Y_{+j} \simeq 32000$
  \end{itemize}

  \bigskip \bigskip 
  \paragraph{Results:}
  \begin{itemize}
   \item $\widehat{K} = 13$ groups of bacteria
   \item $\widehat{G} = 12$ groups of samples
  \end{itemize}

  \bigskip \bigskip 
  \paragraph{Extension:} 
  \begin{itemize}
   \item Despite $\nu_j$, bacteria groups correspond to abundance groups
   \item Plant groups corresponds to diversity levels (Shannon index)
  \end{itemize}
}

%====================================================================
\frame{\frametitle{Meta-rhizo}
  \vspace{-.05\textheight}
  $$
  \includegraphics[height=.9\textheight]{../FIGURES/ARS17-MetaRhizo}
  $$
}

%====================================================================
\frame{\frametitle{Oak mildew pathobiome}

  \paragraph{Dataset:}
  \begin{itemize}
   \item $n = 114 = 48$ fungi $+ 66$ bacteria
   \item $m = 116$ leaves collected from 3 trees (resistant, intermediate, susceptible)
  \end{itemize}

  \bigskip \bigskip 
  \paragraph{Results:} 
  \begin{itemize}
   \item $\widehat{K} = 13$ groups of OTU, 
   \item $\widehat{G} = 2$ groups of leafs
  \end{itemize}
 
  \bigskip \bigskip 
  \paragraph{Results:} 
  \begin{itemize}
   \item Heterogeneous over-dispersion parameters ($a_{kg}$),
   \item Groups reveal the abundance of {\sl E. alphitoides} (pathogene)
  \end{itemize}
}

%====================================================================
\frame{\frametitle{Meta-rhizo}

%   $$
%   \begin{tabular}{cc}
%     & {\sl E. alphitoides} \\
%     $\log (\alpha_{k1} / \alpha_{k2})$ & abundance in sample groups \\
%     \includegraphics[height=.6\textheight]{../FIGURES/ARS17-TreeB}   
%     &
% %     \vspace{.15\textheight}
%     \includegraphics[height=.5\textheight]{../FIGURES/ARS17-TreeA}
%   \end{tabular}
%   $$

  $$
  \begin{tabular}{c}
    $\log (\alpha_{k1} / \alpha_{k2})$ \\
    \includegraphics[height=.6\textheight]{../FIGURES/ARS17-TreeB}   
  \end{tabular}
  \begin{tabular}{c}
    {\sl E. alphitoides} \\
    abundance in sample groups \\ ~\\ 
    \includegraphics[height=.5\textheight]{../FIGURES/ARS17-TreeA}
  \end{tabular}
  $$
}

% %====================================================================
% %====================================================================
% \section{Conclusion}
% %====================================================================
% \frame{\frametitle{Conclusion}
% 
%   \paragraph{Comments:} 
%   \begin{itemize}
%    \item Over-dispersion parameter depending on the group
%   \end{itemize}
% }


%====================================================================
\frame{ \frametitle{References}
{\tiny
  \bibliography{/home/robin/Biblio/BibGene}
%   \bibliographystyle{/home/robin/LATEX/Biblio/astats}
  \bibliographystyle{alpha}
  }
}



%====================================================================
%====================================================================
\end{document}
%====================================================================
%====================================================================

  \begin{tabular}{cc}
    \begin{tabular}{p{.5\textwidth}}
    \end{tabular}
    & 
    \hspace{-.02\textwidth}
    \begin{tabular}{p{.5\textwidth}}
    \end{tabular}
  \end{tabular}

