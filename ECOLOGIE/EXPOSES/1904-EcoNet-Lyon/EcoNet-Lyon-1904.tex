\documentclass[9pt]{beamer}

% Beamer style
%\usetheme[secheader]{Madrid}
% \usetheme{CambridgeUS}
\useoutertheme{infolines}
\usecolortheme[rgb={0.65,0.15,0.25}]{structure}
% \usefonttheme[onlymath]{serif}
\beamertemplatenavigationsymbolsempty
%\AtBeginSubsection

% Packages
%\usepackage[french]{babel}
\usepackage[latin1]{inputenc}
\usepackage{color}
\usepackage{xspace}
\usepackage{dsfont, stmaryrd}
\usepackage{amsmath, amsfonts, amssymb, stmaryrd}
\usepackage{epsfig}
\usepackage{tikz}
\usepackage{url}
% \usepackage{ulem}
\usepackage{/home/robin/LATEX/Biblio/astats}
%\usepackage[all]{xy}
\usepackage{graphicx}
\usepackage{xspace}

% Maths
% \newtheorem{theorem}{Theorem}
% \newtheorem{definition}{Definition}
\newtheorem{proposition}{Proposition}
% \newtheorem{assumption}{Assumption}
% \newtheorem{algorithm}{Algorithm}
% \newtheorem{lemma}{Lemma}
% \newtheorem{remark}{Remark}
% \newtheorem{exercise}{Exercise}
% \newcommand{\propname}{Prop.}
% \newcommand{\proof}{\noindent{\sl Proof:}\quad}
% \newcommand{\eproof}{$\blacksquare$}

% \setcounter{secnumdepth}{3}
% \setcounter{tocdepth}{3}
\newcommand{\pref}[1]{\ref{#1} p.\pageref{#1}}
\newcommand{\qref}[1]{\eqref{#1} p.\pageref{#1}}

% Colors : http://latexcolor.com/
\definecolor{darkred}{rgb}{0.65,0.15,0.25}
\definecolor{darkgreen}{rgb}{0,0.4,0}
\definecolor{darkred}{rgb}{0.65,0.15,0.25}
\definecolor{amethyst}{rgb}{0.6, 0.4, 0.8}
\definecolor{asparagus}{rgb}{0.53, 0.66, 0.42}
\definecolor{applegreen}{rgb}{0.55, 0.71, 0.0}
\definecolor{awesome}{rgb}{1.0, 0.13, 0.32}
\definecolor{blue-green}{rgb}{0.0, 0.87, 0.87}
\definecolor{red-ggplot}{rgb}{0.52, 0.25, 0.23}
\definecolor{green-ggplot}{rgb}{0.42, 0.58, 0.00}
\definecolor{purple-ggplot}{rgb}{0.34, 0.21, 0.44}
\definecolor{blue-ggplot}{rgb}{0.00, 0.49, 0.51}

% Commands
\newcommand{\backupbegin}{
   \newcounter{finalframe}
   \setcounter{finalframe}{\value{framenumber}}
}
\newcommand{\backupend}{
   \setcounter{framenumber}{\value{finalframe}}
}
\newcommand{\emphase}[1]{\textcolor{darkred}{#1}}
\newcommand{\comment}[1]{\textcolor{gray}{#1}}
\newcommand{\paragraph}[1]{\textcolor{darkred}{#1}}
\newcommand{\refer}[1]{{\small{\textcolor{gray}{{[\cite{#1}]}}}}}
\newcommand{\Refer}[1]{{\small{\textcolor{gray}{{[#1]}}}}}
\newcommand{\goto}[1]{{\small{\textcolor{blue}{[\#\ref{#1}]}}}}
\renewcommand{\newblock}{}

\newcommand{\tabequation}[1]{{\medskip \centerline{#1} \medskip}}
% \renewcommand{\binom}[2]{{\left(\begin{array}{c} #1 \\ #2 \end{array}\right)}}

% Variables 
\newcommand{\Abf}{{\bf A}}
\newcommand{\Beta}{\text{B}}
\newcommand{\Bcal}{\mathcal{B}}
\newcommand{\Bias}{\xspace\mathbb B}
\newcommand{\Cor}{{\mathbb C}\text{or}}
\newcommand{\Cov}{{\mathbb C}\text{ov}}
\newcommand{\cl}{\text{\it c}\ell}
\newcommand{\Ccal}{\mathcal{C}}
\newcommand{\cst}{\text{cst}}
\newcommand{\Dcal}{\mathcal{D}}
\newcommand{\Ecal}{\mathcal{E}}
\newcommand{\Esp}{\xspace\mathbb E}
\newcommand{\Espt}{\widetilde{\Esp}}
\newcommand{\Covt}{\widetilde{\Cov}}
\newcommand{\Ibb}{\mathbb I}
\newcommand{\Fcal}{\mathcal{F}}
\newcommand{\Gcal}{\mathcal{G}}
\newcommand{\Gam}{\mathcal{G}\text{am}}
\newcommand{\Hcal}{\mathcal{H}}
\newcommand{\Jcal}{\mathcal{J}}
\newcommand{\Lcal}{\mathcal{L}}
\newcommand{\Mt}{\widetilde{M}}
\newcommand{\mt}{\widetilde{m}}
\newcommand{\Nbb}{\mathbb{N}}
\newcommand{\Mcal}{\mathcal{M}}
\newcommand{\Ncal}{\mathcal{N}}
\newcommand{\Ocal}{\mathcal{O}}
\newcommand{\pt}{\widetilde{p}}
\newcommand{\Pt}{\widetilde{P}}
\newcommand{\Pbb}{\mathbb{P}}
\newcommand{\Pcal}{\mathcal{P}}
\newcommand{\Qcal}{\mathcal{Q}}
\newcommand{\qt}{\widetilde{q}}
\newcommand{\Rbb}{\mathbb{R}}
\newcommand{\Sbb}{\mathbb{S}}
\newcommand{\Scal}{\mathcal{S}}
\newcommand{\st}{\widetilde{s}}
\newcommand{\St}{\widetilde{S}}
\newcommand{\Tcal}{\mathcal{T}}
\newcommand{\todo}{\textcolor{red}{TO DO}}
\newcommand{\Ucal}{\mathcal{U}}
\newcommand{\Un}{\math{1}}
\newcommand{\Vcal}{\mathcal{V}}
\newcommand{\Var}{\mathbb V}
\newcommand{\Vart}{\widetilde{\Var}}
\newcommand{\Zcal}{\mathcal{Z}}

% Symboles & notations
\newcommand\independent{\protect\mathpalette{\protect\independenT}{\perp}}\def\independenT#1#2{\mathrel{\rlap{$#1#2$}\mkern2mu{#1#2}}} 
\renewcommand{\d}{\text{\xspace d}}
\newcommand{\gv}{\mid}
\newcommand{\ggv}{\, \| \, }
% \newcommand{\diag}{\text{diag}}
\newcommand{\card}[1]{\text{card}\left(#1\right)}
\newcommand{\trace}[1]{\text{tr}\left(#1\right)}
\newcommand{\matr}[1]{\boldsymbol{#1}}
\newcommand{\matrbf}[1]{\mathbf{#1}}
\newcommand{\vect}[1]{\matr{#1}} %% un peu inutile
\newcommand{\vectbf}[1]{\matrbf{#1}} %% un peu inutile
\newcommand{\trans}{\intercal}
\newcommand{\transpose}[1]{\matr{#1}^\trans}
\newcommand{\crossprod}[2]{\transpose{#1} \matr{#2}}
\newcommand{\tcrossprod}[2]{\matr{#1} \transpose{#2}}
\newcommand{\matprod}[2]{\matr{#1} \matr{#2}}
\DeclareMathOperator*{\argmin}{arg\,min}
\DeclareMathOperator*{\argmax}{arg\,max}
\DeclareMathOperator{\sign}{sign}
\DeclareMathOperator{\tr}{tr}
\newcommand{\ra}{\emphase{$\rightarrow$} \xspace}

% Hadamard, Kronecker and vec operators
\DeclareMathOperator{\Diag}{Diag} % matrix diagonal
\DeclareMathOperator{\diag}{diag} % vector diagonal
\DeclareMathOperator{\mtov}{vec} % matrix to vector
\newcommand{\kro}{\otimes} % Kronecker product
\newcommand{\had}{\odot}   % Hadamard product

% TikZ
\newcommand{\nodesize}{2em}
\newcommand{\edgeunit}{2.5*\nodesize}
\newcommand{\edgewidth}{1pt}
\tikzstyle{node}=[draw, circle, fill=black, minimum width=.75\nodesize, inner sep=0]
\tikzstyle{square}=[rectangle, draw]
\tikzstyle{param}=[draw, rectangle, fill=gray!50, minimum width=\nodesize, minimum height=\nodesize, inner sep=0]
\tikzstyle{hidden}=[draw, circle, fill=gray!50, minimum width=\nodesize, inner sep=0]
\tikzstyle{hiddenred}=[draw, circle, color=red, fill=gray!50, minimum width=\nodesize, inner sep=0]
\tikzstyle{observed}=[draw, circle, minimum width=\nodesize, inner sep=0]
\tikzstyle{observedred}=[draw, circle, minimum width=\nodesize, color=red, inner sep=0]
\tikzstyle{eliminated}=[draw, circle, minimum width=\nodesize, color=gray!50, inner sep=0]
\tikzstyle{empty}=[draw, circle, minimum width=\nodesize, color=white, inner sep=0]
\tikzstyle{blank}=[color=white]
\tikzstyle{nocircle}=[minimum width=\nodesize, inner sep=0]

\tikzstyle{edge}=[-, line width=\edgewidth]
\tikzstyle{edgebendleft}=[-, >=latex, line width=\edgewidth, bend left]
\tikzstyle{edgebendright}=[-, >=latex, line width=\edgewidth, bend right]
\tikzstyle{lightedge}=[-, line width=\edgewidth, color=gray!50]
\tikzstyle{lightedgebendleft}=[-, >=latex, line width=\edgewidth, bend left, color=gray!50]
\tikzstyle{lightedgebendright}=[-, >=latex, line width=\edgewidth, bend right, color=gray!50]
\tikzstyle{edgered}=[-, line width=\edgewidth, color=red]
\tikzstyle{edgebendleftred}=[-, >=latex, line width=\edgewidth, bend left, color=red]
\tikzstyle{edgebendrightred}=[-, >=latex, line width=\edgewidth, bend right, color=red]

\tikzstyle{arrow}=[->, >=latex, line width=\edgewidth]
\tikzstyle{arrowbendleft}=[->, >=latex, line width=\edgewidth, bend left]
\tikzstyle{arrowbendright}=[->, >=latex, line width=\edgewidth, bend right]
\tikzstyle{arrowred}=[->, >=latex, line width=\edgewidth, color=red]
\tikzstyle{arrowbendleftred}=[->, >=latex, line width=\edgewidth, bend left, color=red]
\tikzstyle{arrowbendrightred}=[->, >=latex, line width=\edgewidth, bend right, color=red]
\tikzstyle{arrowblue}=[->, >=latex, line width=\edgewidth, color=blue]
\tikzstyle{dashedarrow}=[->, >=latex, dashed, line width=\edgewidth]
\tikzstyle{dashededge}=[-, >=latex, dashed, line width=\edgewidth]
\tikzstyle{dashededgebendleft}=[-, >=latex, dashed, line width=\edgewidth, bend left]
\tikzstyle{lightarrow}=[->, >=latex, line width=\edgewidth, color=gray!50]

\newcommand{\GMSBM}{/home/robin/RECHERCHE/RESEAUX/EXPOSES/1903-SemStat/}
\newcommand{\figeconet}{/home/robin/Bureau/RECHERCHE/ECOLOGIE/EXPOSES/1904-EcoNet-Lyon/Figs}
\renewcommand{\nodesize}{1.75em}
\renewcommand{\edgeunit}{2.25*\nodesize}

%====================================================================
%====================================================================
\begin{document}
%====================================================================
%====================================================================

\title[Graphical models]{Introduction to probabilistic graphical models}

\author[S. Robin]{S. Robin: based on Forbes (2016) \& Schwaller (2016)} \nocite{For16,Sch16}

\date[EcoNet, Apr '19, Lyon]{EcoNet, April 2019, Lyon}

\maketitle

%====================================================================
\frame{ \frametitle{References}
  {\footnotesize
   %\tiny
   \bibliography{/home/robin/Biblio/BibGene}
%    \bibliographystyle{/home/robin/LATEX/Biblio/astats}
   \bibliographystyle{alpha}
  }
}

%====================================================================
\frame{\frametitle{Graphical models}

  Useful and mathematically grounded tool to describe the dependency structure among a set of random variables.
  
  \bigskip \bigskip \pause
  \paragraph{Example.} Let $A, B, C, ... =$ species abundances, environmental covariates, ...
  
  What do we mean when drawing 
  $$
  \input{\figeconet/DAG6nodes}
  \qquad \qquad
    \begin{tikzpicture}
  \input{\figeconet/GM7nodes-positions}
  
  \draw[edge] (a) to (b);  \draw[edge] (a) to (c);  \draw[edge] (a) to (d);
  \draw[edge] (b) to (c);  \draw[edge] (b) to (d);  \draw[edge] (c) to (d);
  \draw[edge] (d) to (e);  \draw[edge] (d) to (f);  \draw[edge] (e) to (f);
  \draw[edge] (f) to (g);  
  \end{tikzpicture}

  $$
}
  

%====================================================================
%====================================================================
\section{Elements of probability}
\frame{\frametitle{Outline} \tableofcontents[currentsection]}
%====================================================================
\frame{\frametitle{Elements of probability}

  \paragraph{Notation.}
  $$
  p(x) = p(x_1, x_2, \dots, x_n) = \Pr\{X_1=x_1, X_2=x_2, \dots, X_n=x_n\}
  $$
  
  \bigskip 
  \onslide+<2->{\paragraph{Definitions\footnote{$\sum_{x_2, \dots, x_n} ()$ to be replaced with $\int () \d x_2 \dots \d x_n$ when continuous}.}
  \begin{align*}
   \text{Marginal distribution of $X_1$:} & &  p(x_1) & =\sum_{x_2, \dots, x_n} p(x_1, x_2, \dots, x_n) \\ 
   ~ \\
   \text{Conditional distribution of $Y \mid X$:} & & p(y \mid x) & = {p(x, y)} \left/ {p(x)} \right. 
  \end{align*} }
  
  \bigskip 
  \onslide+<3->{\paragraph{Properties.}
  \begin{align*}
   \text{Product rule:} & & p(x, y) & =p(x) p(y \mid x) \\ 
   ~ \\
   \text{Chain rule:} & & \quad p(x_1, x_2, \dots, x_n) & =p(x_1) p(x_2 \mid x_1) \\
   & & & \quad \dots \times p(x_n \mid x_1, x_2, \dots x_{n-1})
  \end{align*}}
}
  
%====================================================================
%====================================================================
\section{Directed graphs and Bayesian networks}
\frame{\frametitle{Outline} \tableofcontents[currentsection]}
%====================================================================
\frame{\frametitle{Directed graphs and Bayesian networks}

  \paragraph{Definition.} Let $D$ be a {\sl directed acyclic graph} (\emphase{DAG}), the distribution $p$ is said to factorize in $D$ iff
  $$
  p(x_1, \dots x_n) = \prod_{i=1}^n p(x_i \mid x_{pa_D(i)})
  $$
  where $pa_D(i)$ stands for the set of parents of $i$ in $D$.

  \bigskip \bigskip \pause
  \begin{tabular}{cc}
    \begin{tabular}{c}
    $\input{\figeconet/DAG6nodes}$
    \end{tabular}
    &
    \begin{tabular}{p{.7\textwidth}}
      \begin{eqnarray*}
        pa_D(A) = \emptyset, & & pa_D(D) = \{B, C\}, \qquad \dots  \\
        \\
        p(a, \dots f) & = 
        & p(a) \; p(b \mid a) \; p(c \mid a) \\
        & & p(d \mid b, c) \; p(e \mid d) \\
        & & p(f \mid b, d)
        \end{eqnarray*}
    \end{tabular}
  \end{tabular}

}

%====================================================================
\frame{\frametitle{Dynamic Bayesian Networks (DBN)}

  \renewcommand{\nodesize}{2em}
  \begin{tabular}{p{.48\textwidth}p{.48\textwidth}}
   \paragraph{Genuine graphical model.} A DAG: &
   \paragraph{Popular representation.} Not a DAG: \\ 
   $$
     \begin{tikzpicture}
  \node[observed] (a1) at (0*\edgeunit, 2.25*\edgeunit) {$A_{t-1}$};
  \node[observed] (b1) at (0*\edgeunit, 1.5*\edgeunit) {$B_{t-1}$};
  \node[observed] (c1) at (0*\edgeunit, .75*\edgeunit) {$C_{t-1}$};
  \node[observed] (d1) at (0*\edgeunit, 0*\edgeunit) {$D_{t-1}$};
  \node[observed] (a2) at (1*\edgeunit, 2.25*\edgeunit) {$A_t$};
  \node[observed] (b2) at (1*\edgeunit, 1.5*\edgeunit) {$B_t$};
  \node[observed] (c2) at (1*\edgeunit, 0.75*\edgeunit) {$C_t$};
  \node[observed] (d2) at (1*\edgeunit, 0*\edgeunit) {$D_t$};
  \node[observed] (a3) at (2*\edgeunit, 2.25*\edgeunit) {$A_{t+1}$};
  \node[observed] (b3) at (2*\edgeunit, 1.5*\edgeunit) {$B_{t+1}$};
  \node[observed] (c3) at (2*\edgeunit, 0.75*\edgeunit) {$C_{t+1}$};
  \node[observed] (d3) at (2*\edgeunit, 0*\edgeunit) {$D_{t+1}$};
  
  \draw[lightarrow] (a1) to (a2);  \draw[lightarrow] (b1) to (b2);
  \draw[lightarrow] (c1) to (c2);  \draw[lightarrow] (d1) to (d2);
  \draw[lightarrow] (a2) to (a3);  \draw[lightarrow] (b2) to (b3);
  \draw[lightarrow] (c2) to (c3);  \draw[lightarrow] (d2) to (d3);

  \draw[arrow] (a1) to (b2);  \draw[arrow] (a1) to (c2);
  \draw[arrow] (b1) to (d2);  
  \draw[arrow] (c1) to (b2);  \draw[arrow] (c1) to (d2);
  \draw[arrow] (d1) to (c2);  \draw[arrow] (d1) to (a2);  

  \draw[arrow] (a2) to (b3);  \draw[arrow] (a2) to (c3);
  \draw[arrow] (b2) to (d3);  
  \draw[arrow] (c2) to (b3);  \draw[arrow] (c2) to (d3);
  \draw[arrow] (d2) to (c3);  \draw[arrow] (d2) to (a3);  
  \end{tikzpicture}

   $$
   &
   $$
    \begin{tikzpicture}
  \node[observed] (a) at (0*\edgeunit, 1*\edgeunit) {$A$};
  \node[observed] (b) at (1*\edgeunit, 1*\edgeunit) {$B$};
  \node[observed] (c) at (0*\edgeunit, 0*\edgeunit) {$C$};
  \node[observed] (d) at (1*\edgeunit, 0*\edgeunit) {$D$};
  
  \draw[arrow] (a) to (b);  \draw[arrow] (a) to (c);
  \draw[arrow] (b) to (d);  
  \draw[arrow] (c) to (b);  \draw[arrowbendleft] (c) to (d);
  \draw[arrowbendleft] (d) to (c);  \draw[arrow] (d) to (a);  
  
  \end{tikzpicture}

   $$
  \end{tabular}
  \renewcommand{\nodesize}{1.75em}
}
  
%====================================================================
\frame{\frametitle{A simple (interesting) example}

  Consider $D =$
  $$
  \input{\figeconet/LeftRightChain}
  $$
  $p(x, y, z)$ is faithful to $D$ iff
  $$
  p(x, y, z) = p(x) \; p(y \mid x) \; p(z \mid y) 
  $$ \pause
  But
  \begin{align*}
   p(x) \; p(y \mid x) \; p(z \mid y) 
%    & = p(x) \; \frac{p(x, y)}{p(x)} \; \frac{p(y, z)}{p(y)} \\
%    & = \frac{p(x, y)}{p(y)} \; \frac{p(y, z)}{p(z)} \; p(z) \\
   & = p(x \mid y) \; p(y \mid z) \; p(z) 
  \end{align*}
  so $p$ is also faithful to $D' =$
  $$
    \begin{tikzpicture}
  \node[observed] (x) at (0*\edgeunit, 0*\edgeunit) {$X$};
  \node[observed] (y) at (1*\edgeunit, 0*\edgeunit) {$Y$};
  \node[observed] (z) at (2*\edgeunit, 0*\edgeunit) {$Z$};
  
  \draw[arrow] (z) to (y);  \draw[arrow] (y) to (x);
  \end{tikzpicture}
 
  $$
  \pause and to $D'' =$
  $$
  \input{\figeconet/AwayFromCenter}
  $$
  
  \bigskip \pause
  \paragraph{Conclusions.} 
  \begin{itemize}
   \item $p(x)$ is not enough to retrieve edge orientations'
   \item No causal interpretation
  \end{itemize}
}
  
%====================================================================
%====================================================================
\section{Conditional independance and Markov properties}
\frame{\frametitle{Outline} \tableofcontents[currentsection]}
%====================================================================
\frame{\frametitle{Conditional independance}

  \paragraph{Definition.} $X$ is independent of $Y$ conditional on $Z$ ($X \independent Y \mid Z$) iff
  $$
  p(x, y \mid z) = p(x \mid z ) p(y \mid z)
  \qquad \Leftrightarrow \qquad 
  p(x \mid z, y) = p(x \mid z ) 
  $$

  \bigskip \bigskip \pause
  \paragraph{Example.} $A \independent C \mid B$ in the three DAGs:
  $$
  \input{\figeconet/LeftRightChain} \qquad
    \begin{tikzpicture}
  \node[observed] (x) at (0*\edgeunit, 0*\edgeunit) {$X$};
  \node[observed] (y) at (1*\edgeunit, 0*\edgeunit) {$Y$};
  \node[observed] (z) at (2*\edgeunit, 0*\edgeunit) {$Z$};
  
  \draw[arrow] (z) to (y);  \draw[arrow] (y) to (x);
  \end{tikzpicture}
 \qquad
  \input{\figeconet/AwayFromCenter}
  $$
  Indeed, for first one,
  $$
  p(x, z \mid y)
  = \frac{p(x, y, z)}{p(y)}
  = \frac{p(x) p(y \mid x) p(z \mid y)}{p(y)}
  = p(x \mid y) p(z \mid y)
  $$
  because $p(x) p(y \mid x) = p(y) p(x \mid y)$.
  
}

%====================================================================
\frame{\frametitle{V-structure}

%   \paragraph{Counter-example.} 
  In the V-structured (or 'head to head') DAG:
  $$
    \begin{tikzpicture}
  \node[observed] (x) at (0*\edgeunit, 0*\edgeunit) {$X$};
  \node[observed] (y) at (1*\edgeunit, 0*\edgeunit) {$Y$};
  \node[observed] (z) at (2*\edgeunit, 0*\edgeunit) {$Z$};
  
  \draw[arrow] (x) to (y);  \draw[arrow] (z) to (y);
  \end{tikzpicture}

  $$
  $X$ and $Z$ are \emphase{conditionally dependent} ($X \not\independent Y \mid Z$):
  \begin{align*}
  p(x, y, z) & = p(x) p(z) p(y \mid x, z) \\ ~\\
  \Rightarrow \quad
  p(x, z \mid y) & = \frac{p(x, y, z)}{p(y)}   = \frac{p(x) p(z) p(y \mid x, z)}{p(y)}  
  \end{align*}
  
  
  \bigskip \bigskip \pause
  \paragraph{Remark.} 
  $X$ and $Z$ are \emphase{marginally independent}:
  $$p(x, z)
  = \sum_y p(x) p(z) p(y \mid x, z)
  = p(x) p(z) \underset{= 1}{\underbrace{\sum_y p(y \mid x, z)}}
  $$

}

%====================================================================
\frame{\frametitle{Markov properties}

  \paragraph{Theorem.} Two DAGs are Markov equivalent (i.e. induce the same conditional dependences and independences) if they share
  \begin{itemize}
   \item the same skeleton (i.e. the same undirected edges)
   \item the same V-structures.
  \end{itemize}

}
  
%====================================================================
\frame{\frametitle{Equivalent DAGs}

  \begin{tabular}{c|cc|cc}
  $D$ &
  \multicolumn{2}{c|}{Equivalent to $D$} &
  \multicolumn{2}{c}{Not equivalent to $D$} \\
  (\textcolor{red}{V-stuctures}) & & & & \\ 
  & & & & \\ \hline & & & & \\ 
  $  \begin{tikzpicture}
    \node[observed] (a) at (.5*\edgeunit, 2.75*\edgeunit) {$A$};
  \node[observed] (b) at (0*\edgeunit, 2*\edgeunit) {$B$};
  \node[observed] (c) at (1*\edgeunit, 2*\edgeunit) {$C$};
  \node[observed] (d) at (0.5*\edgeunit, 1*\edgeunit) {$D$};
  \node[observed] (e) at (0*\edgeunit, 0*\edgeunit) {$E$};
  \node[observed] (f) at (1*\edgeunit, 0*\edgeunit) {$F$};


  \draw[arrow] (a) to (b);  \draw[arrow] (a) to (c);
  \draw[arrowred] (b) to (d);  \draw[arrowbendleft] (b) to (f);
  \draw[arrowred] (c) to (d);  \draw[arrow] (d) to (e);
  \draw[arrow] (d) to (f);  
  \end{tikzpicture}
$ \pause &
  $  \begin{tikzpicture}
    \node[observed] (a) at (.5*\edgeunit, 2.75*\edgeunit) {$A$};
  \node[observed] (b) at (0*\edgeunit, 2*\edgeunit) {$B$};
  \node[observed] (c) at (1*\edgeunit, 2*\edgeunit) {$C$};
  \node[observed] (d) at (0.5*\edgeunit, 1*\edgeunit) {$D$};
  \node[observed] (e) at (0*\edgeunit, 0*\edgeunit) {$E$};
  \node[observed] (f) at (1*\edgeunit, 0*\edgeunit) {$F$};


  \draw[arrowblue] (b) to (a);  \draw[arrow] (a) to (c);
  \draw[arrow] (b) to (d);  \draw[arrowbendleft] (b) to (f);
  \draw[arrow] (c) to (d);  \draw[arrow] (d) to (e);
  \draw[arrow] (d) to (f);  
  \end{tikzpicture}
$ \pause &
  $\input{\figeconet/DAG6nodes-vtruct-equiv2}$ \pause &
  $  \begin{tikzpicture}
    \node[observed] (a) at (.5*\edgeunit, 2.75*\edgeunit) {$A$};
  \node[observed] (b) at (0*\edgeunit, 2*\edgeunit) {$B$};
  \node[observed] (c) at (1*\edgeunit, 2*\edgeunit) {$C$};
  \node[observed] (d) at (0.5*\edgeunit, 1*\edgeunit) {$D$};
  \node[observed] (e) at (0*\edgeunit, 0*\edgeunit) {$E$};
  \node[observed] (f) at (1*\edgeunit, 0*\edgeunit) {$F$};

  
  \draw[arrowred] (b) to (a);  \draw[arrowred] (c) to (a);
  \draw[arrow] (b) to (d);  \draw[arrowbendleft] (b) to (f);
  \draw[arrow] (c) to (d);  \draw[arrow] (d) to (e);
  \draw[arrow] (d) to (f);  
  \end{tikzpicture}
$ \pause &
  $  \begin{tikzpicture}
    \node[observed] (a) at (.5*\edgeunit, 2.75*\edgeunit) {$A$};
  \node[observed] (b) at (0*\edgeunit, 2*\edgeunit) {$B$};
  \node[observed] (c) at (1*\edgeunit, 2*\edgeunit) {$C$};
  \node[observed] (d) at (0.5*\edgeunit, 1*\edgeunit) {$D$};
  \node[observed] (e) at (0*\edgeunit, 0*\edgeunit) {$E$};
  \node[observed] (f) at (1*\edgeunit, 0*\edgeunit) {$F$};

  
  \draw[arrow] (a) to (b);  \draw[arrow] (a) to (c);
  \draw[arrowred] (b) to (d);  \draw[arrowbendleft] (b) to (f);
  \draw[arrow] (c) to (d);  \draw[arrowred] (e) to (d);
  \draw[arrow] (d) to (f);  
  \end{tikzpicture}
$ 
  \end{tabular}

}
  
%====================================================================
%====================================================================
\section{Undirected graphs and Markov random fields}
\frame{\frametitle{Outline} \tableofcontents[currentsection]}
%====================================================================
\frame{\frametitle{Undirected graphical model}

  \paragraph{Definition.} Let $G$ be an {\sl undirected graph}, the distribution $p$ is said to factorize in $G$ iff
  $$
  p(x_1, \dots x_n) \propto \prod_{C \in \Ccal(G)} \psi_C(x_C).
  $$
  where $\Ccal(G)$ is the set of cliques of $G$

  \bigskip \bigskip \pause
  \begin{tabular}{cc}
    \begin{tabular}{c}
    $  \begin{tikzpicture}
  \input{\figeconet/GM7nodes-positions}
  
  \draw[edge] (a) to (b);  \draw[edge] (a) to (c);  \draw[edge] (a) to (d);
  \draw[edge] (b) to (c);  \draw[edge] (b) to (d);  \draw[edge] (c) to (d);
  \draw[edge] (d) to (e);  \draw[edge] (d) to (f);  \draw[edge] (e) to (f);
  \draw[edge] (f) to (g);  
  \end{tikzpicture}
$
    \end{tabular}
    &
    \begin{tabular}{p{.7\textwidth}}
      \begin{eqnarray*}
        p(a, \dots g) & \propto 
        & \psi_1(a, b, c) \; \psi_2(a, b, d) \; \psi_3(a, c, d) \; \psi_4(b, c, d) \\
        & & \psi_5(d, e, f) \; \psi_6(f, g) \\\pause
      \text{but also} \qquad \\
        p(a, \dots g) & \propto 
        & \psi_1(a, b, c, d) \\
        & & \psi_2(d, e, f) \; \psi_3(f, g) 
      \end{eqnarray*}
      \ra Only consider \emphase{maximal} cliques
    \end{tabular}
  \end{tabular}

}

%====================================================================
\frame{\frametitle{Conditional independence}

  \paragraph{Property.} If $p(x) > 0$, 
  $$
  \text{separation} \qquad \Leftrightarrow \qquad \text{conditional independence}
  $$

  \bigskip \pause
  \begin{tabular}{cc}
    \begin{tabular}{c}
    $  \begin{tikzpicture}
  \input{\figeconet/GM7nodes-positions}
  
  \draw[edge] (a) to (b);  \draw[edge] (a) to (c);  \draw[edge] (a) to (d);
  \draw[edge] (b) to (c);  \draw[edge] (b) to (d);  \draw[edge] (c) to (d);
  \draw[edge] (d) to (e);  \draw[edge] (d) to (f);  \draw[edge] (e) to (f);
  \draw[edge] (f) to (g);  
  \end{tikzpicture}
$
    \end{tabular}
    &
    \begin{tabular}{p{.7\textwidth}}
    \begin{itemize}
     \item $A \not\independent B$ \\ ~
     \item $A \not\independent D \mid B$ \\ ~
     \item $A \independent D \mid \{B, C\}$ \\ ~
     \item $\{A, B, C\} \independent \{E, F, G\} \mid D$ \\ ~
     \item $\{B, C\} \independent \{E, F\} \mid D$ \\ ~
    \end{itemize}
    \end{tabular}
  \end{tabular}

}
  
%====================================================================
\frame{\frametitle{From directed to undirected graphical models}

  \paragraph{Resolve {\sl immoralities}.} Graph {\sl moralization} (parents must be married):
  $$
  \begin{array}{ccccc}
     \begin{tikzpicture} 
    \node[observed] (a) at (0*\edgeunit, 1*\edgeunit) {$A$};
  \node[observed] (b) at (1*\edgeunit, 1*\edgeunit) {$B$};
  \node[observed] (c) at (.5*\edgeunit, 0*\edgeunit) {$C$};
  

  
  \draw[arrow] (a) to (c);  \draw[arrow] (b) to (c);
  \end{tikzpicture}
  

   & \qquad & 
     \begin{tikzpicture} 
    \node[observed] (a) at (0*\edgeunit, 1*\edgeunit) {$A$};
  \node[observed] (b) at (1*\edgeunit, 1*\edgeunit) {$B$};
  \node[observed] (c) at (.5*\edgeunit, 0*\edgeunit) {$C$};
  

  
  \draw[arrow] (a) to (c);  \draw[arrow] (b) to (c);
  \draw[dashededge] (a) to (b);
  \end{tikzpicture}
  

   & \qquad & 
     \begin{tikzpicture} 
    \node[observed] (a) at (0*\edgeunit, 1*\edgeunit) {$A$};
  \node[observed] (b) at (1*\edgeunit, 1*\edgeunit) {$B$};
  \node[observed] (c) at (.5*\edgeunit, 0*\edgeunit) {$C$};
  

  
  \draw[edge] (a) to (c);  \draw[edge] (b) to (c);
  \draw[edge] (a) to (b);
  \end{tikzpicture}
  

  \end{array}
  $$
  
  \pause
  \paragraph{Example.}
  $$
  \begin{array}{ccccc}
   \input{\figeconet/DAG6nodes}
   & \qquad & 
     \begin{tikzpicture}
    \node[observed] (a) at (.5*\edgeunit, 2.75*\edgeunit) {$A$};
  \node[observed] (b) at (0*\edgeunit, 2*\edgeunit) {$B$};
  \node[observed] (c) at (1*\edgeunit, 2*\edgeunit) {$C$};
  \node[observed] (d) at (0.5*\edgeunit, 1*\edgeunit) {$D$};
  \node[observed] (e) at (0*\edgeunit, 0*\edgeunit) {$E$};
  \node[observed] (f) at (1*\edgeunit, 0*\edgeunit) {$F$};

  
  \draw[arrow] (a) to (b);  \draw[arrow] (a) to (c);
  \draw[dashededge] (b) to (c);  
  \draw[arrow] (b) to (d);  \draw[arrowbendleft] (b) to (f);
  \draw[arrow] (c) to (d);  \draw[arrow] (d) to (e);
  \draw[arrow] (d) to (f);  
  \end{tikzpicture}

   & \qquad & 
     \begin{tikzpicture}
    \node[observed] (a) at (.5*\edgeunit, 2.75*\edgeunit) {$A$};
  \node[observed] (b) at (0*\edgeunit, 2*\edgeunit) {$B$};
  \node[observed] (c) at (1*\edgeunit, 2*\edgeunit) {$C$};
  \node[observed] (d) at (0.5*\edgeunit, 1*\edgeunit) {$D$};
  \node[observed] (e) at (0*\edgeunit, 0*\edgeunit) {$E$};
  \node[observed] (f) at (1*\edgeunit, 0*\edgeunit) {$F$};

  
  \draw[edge] (a) to (b);  \draw[edge] (a) to (c);
  \draw[edge] (b) to (c);  
  \draw[edge] (b) to (d);  \draw[edgebendleft] (b) to (f);
  \draw[edge] (c) to (d);  \draw[edge] (d) to (e);
  \draw[edge] (d) to (f);  
  \end{tikzpicture}

  \end{array}
  $$
}

%====================================================================
%====================================================================
\section{Two useful models}
\frame{\frametitle{Outline} \tableofcontents[currentsection]}
%====================================================================
%====================================================================
\subsection{Gaussian graphical models (GGM)}
%====================================================================
\frame{\frametitle{Gaussian graphical models (GGM)}

  \paragraph{Multivariate Gaussian distribution.} $\Omega =$ {\sl precision} matrix:
  \begin{align*}
  X = (X_1, \dots X_n) & \sim \Ncal(0, \Sigma), 
  \qquad \qquad \text{define } [\omega_{ij}] = \Omega := \Sigma^{-1} \\ ~\\ 
  \Rightarrow \quad p(x_1, \dots, x_n) 
  & \propto \exp\left( - x^\intercal \Omega x \left/ 2 \right. \right) 
  = \exp\left( - \sum_{i, j} x_i \omega_{ij} x_j \left/ 2 \right. \right) \\
  & = \prod_{i, j} \exp(- x_i \omega_{ij} x_j / 2)
  \end{align*}
  
  \bigskip \bigskip \pause
  \paragraph{Consequence.} $p(x)$ is faithful to $G$ such that: 
  $$
  i \sim j \quad \Leftrightarrow \quad \omega_{ij} \neq 0
  $$
  
}
  
%====================================================================
\frame{\frametitle{GGM: an example}

  \begin{tabular}{ll}
   \paragraph{Covariance.} &    \paragraph{Correlations.} \\
   \begin{tabular}{p{.4\textwidth}}
    \footnotesize{$$
    \Sigma = \left(\begin{array}{rrrr} 
    1.6  & -1.2  & -0.8  & 0.7 \\ 
     & 2.3  & 1.5  & -1.4 \\ 
     & & 2.6  & -1.8 \\ 
     & &  & 2.4 
    \end{array} \right) 
    $$}
   \end{tabular}
   &
   \begin{tabular}{p{.4\textwidth}}
    \footnotesize{$$
    R = \left(\begin{array}{rrrr} 
    1.0 & -0.6  & -0.4  & 0.4 \\ 
    & 1.0 & 0.6  & -0.6 \\ 
    & & 1.0 & -0.7 \\ 
    & & & 1.0
    \end{array} \right)
    $$}
   \end{tabular}
   \\ ~ \\
   \paragraph{Inverse covariance.} &     \paragraph{Graphical model.} \\
   \begin{tabular}{p{.4\textwidth}}
    \footnotesize{$$
    \Omega = \left(\begin{array}{rrrr} 
    1.0 & 0.5  & 0  & 0 \\ 
    & 1.0 & -0.3  & 0.2 \\ 
    & & 1.0 & 0.6 \\ 
    & & & 1.0
    \end{array} \right) 
    $$}
   \end{tabular}
   &
   \begin{tabular}{p{.4\textwidth}}
   $$
     \begin{tikzpicture} 
    \node[observed] (a) at (.5*\edgeunit, 2*\edgeunit) {$A$};
  \node[observed] (b) at (.5*\edgeunit, 1*\edgeunit) {$B$};
  \node[observed] (c) at (0*\edgeunit, 0*\edgeunit) {$C$};
  \node[observed] (d) at (1*\edgeunit, 0*\edgeunit) {$D$};
  

  
  \draw[edge] (a) to (b);  \draw[edge] (b) to (c);
  \draw[edge] (b) to (d);  \draw[edge] (c) to (d);
  \end{tikzpicture}
  

   $$
   \end{tabular}
  \end{tabular}

}

%====================================================================
\frame{\frametitle{Infering GGM}

  \paragraph{A simulation.} 50 replicates \\~ \\~

  \begin{tabular}{ll}
   \paragraph{Empirical covariance.} &    \paragraph{Empirical precision.} \\
   \begin{tabular}{p{.4\textwidth}}
    \footnotesize{$$
    \widehat{\Sigma} = \left(\begin{array}{rrrr} 
    2.15  & -1.36  & -0.59  & 0.57 \\ 
    & 1.83  & 1.24  & -1.05 \\ 
    & & 2.58  & -1.62 \\ 
    & & & 1.81 
    \end{array} \right) 
    $$}
   \end{tabular}
   &
   \begin{tabular}{p{.4\textwidth}}
    \footnotesize{$$
    \widehat{\Omega} = \left(\begin{array}{rrrr} 
    0.93  & 0.82  & -0.15  & 0.04 \\ 
    & 1.59  & -0.37  & 0.33 \\ 
    & & 0.97  & 0.7 \\ 
    & & & 1.36     
    \end{array} \right)
    $$}
   \end{tabular}
  \end{tabular}
  
  \ra Need to force $\widehat{\Omega}$ to be sparse (e.g. graphical lasso)

}

%====================================================================
%====================================================================
\subsection{Stochastic block-model (SBM)}
%====================================================================
\frame{\frametitle{Stochastic block-model (SBM)}

  \paragraph{Reminder.} Consider a network with $n$ nodes
  \begin{itemize}
   \item Each node $i$ belongs to a unobserved class: $Z_i \in \{1, \dots, K\}$, $Z_i$ iid
   \item Connections between node are to node memberships: $P(i \sim j \mid Z_i=k, Z_j=\ell) = \gamma_{kl}$
  \end{itemize}
  \ra Statistical inference requires to evaluate $p(Z \mid Y)$.

  \bigskip \bigskip 
  \begin{overprint}
  \onslide<2>
  \begin{centering}
  \input{\GMSBM/SBM-GraphModel-pZ} 
  \end{centering}
  \onslide<3>
  \begin{centering}
    \begin{tikzpicture}
  \node[hidden] (Z1) at (0, \edgeunit) {$Z_1$};
  \node[hidden] (Z2) at (\edgeunit, \edgeunit) {$Z_2$};
  \node[hidden] (Z3) at (0, 0) {$Z_3$};
  \node[hidden] (Z4) at (\edgeunit, 0) {$Z_4$};

  \node[observed] (Y12) at (.5*\edgeunit, 1.75*\edgeunit) {$Y_{12}$};
  \node[empty] (Y13) at (-.75*\edgeunit, .5*\edgeunit) {$Y_{13}$};
  \node[empty] (Y14) at (1.75*\edgeunit, 1.75*\edgeunit) {$Y_{14}$};
  \node[empty] (Y23) at (-.75*\edgeunit, 1.75*\edgeunit) {$Y_{23}$};
  \node[empty] (Y24) at (1.75*\edgeunit, .5*\edgeunit) {$Y_{24}$};
  \node[empty] (Y34) at (.5*\edgeunit, -.75*\edgeunit) {$Y_{34}$};
  
  \draw[arrow] (Z1) to (Y12); \draw[arrow] (Z2) to (Y12);
  \end{tikzpicture}

 
  \end{centering}
  \onslide<4>
  \begin{centering}
  \input{\GMSBM/SBM-GraphModel-pZY13} 
  \end{centering}
  \onslide<5>
  \begin{centering}
    \begin{tikzpicture}
  \node[hidden] (Z1) at (0, \edgeunit) {$Z_1$};
  \node[hidden] (Z2) at (\edgeunit, \edgeunit) {$Z_2$};
  \node[hidden] (Z3) at (0, 0) {$Z_3$};
  \node[hidden] (Z4) at (\edgeunit, 0) {$Z_4$};

  \node[observed] (Y12) at (.5*\edgeunit, 1.75*\edgeunit) {$Y_{12}$};
  \node[observed] (Y13) at (-.75*\edgeunit, .5*\edgeunit) {$Y_{13}$};
  \node[observed] (Y14) at (1.75*\edgeunit, 1.75*\edgeunit) {$Y_{14}$};
  \node[observed] (Y23) at (-.75*\edgeunit, 1.75*\edgeunit) {$Y_{23}$};
  \node[observed] (Y24) at (1.75*\edgeunit, .5*\edgeunit) {$Y_{24}$};
  \node[observed] (Y34) at (.5*\edgeunit, -.75*\edgeunit) {$Y_{34}$};
  
  \draw[arrow] (Z1) to (Y12);  \draw[arrow] (Z2) to (Y12);
  \draw[arrow] (Z1) to (Y13);  \draw[arrow] (Z3) to (Y13);
  \draw[arrow] (Z1) to (Y14);  \draw[arrow] (Z4) to (Y14);
  \draw[arrow] (Z2) to (Y23);  \draw[arrow] (Z3) to (Y23);
  \draw[arrow] (Z2) to (Y24);  \draw[arrow] (Z4) to (Y24);
  \draw[arrow] (Z3) to (Y34);  \draw[arrow] (Z4) to (Y34);
  \end{tikzpicture}

 
  \end{centering}
  \onslide<6>
  \begin{centering}
  \input{\GMSBM/SBM-GraphModel-pZY-moral} 
  \end{centering}
  \onslide<7>
  \begin{centering}
  \input{\GMSBM/SBM-GraphModel-pZY-GM} 
  \end{centering}
  \onslide<8>
  \begin{centering}
    \begin{tikzpicture}
  \node[hidden] (Z1) at (0, \edgeunit) {$Z_1$};
  \node[hidden] (Z2) at (\edgeunit, \edgeunit) {$Z_2$};
  \node[hidden] (Z3) at (0, 0) {$Z_3$};
  \node[hidden] (Z4) at (\edgeunit, 0) {$Z_4$};

  \node[eliminated] (Y12) at (.5*\edgeunit, 1.75*\edgeunit) {$Y_{12}$};
  \node[eliminated] (Y13) at (-.75*\edgeunit, .5*\edgeunit) {$Y_{13}$};
  \node[eliminated] (Y14) at (1.75*\edgeunit, 1.75*\edgeunit) {$Y_{14}$};
  \node[eliminated] (Y23) at (-.75*\edgeunit, 1.75*\edgeunit) {$Y_{23}$};
  \node[eliminated] (Y24) at (1.75*\edgeunit, .5*\edgeunit) {$Y_{24}$};
  \node[eliminated] (Y34) at (.5*\edgeunit, -.75*\edgeunit) {$Y_{34}$};
  
  \draw[edge] (Z1) to (Z2);  \draw[edge] (Z1) to (Z3);
  \draw[edge] (Z1) to (Z4);  \draw[edge] (Z2) to (Z3);
  \draw[edge] (Z2) to (Z4);  \draw[edge] (Z3) to (Z4);
  \end{tikzpicture}

 
  \end{centering}
  \end{overprint}

}

%====================================================================
%====================================================================
\backupbegin 
\section*{Backup}

%====================================================================
\frame{\frametitle{DAG for a hidden Markov model (HMM)}

  \paragraph{Model.} 
  \begin{itemize}
   \item (Hidden) $(Z_t) \sim$ Markov chain$(\pi)$
   \item (Observed) $Y_t \mid Z_t = k \sim F(\mu_k)$
  \end{itemize}

  \bigskip \bigskip \pause
  \begin{tabular}{p{.4\textwidth}p{.4\textwidth}}
  \paragraph{Frequentist model.} & \paragraph{Bayesian model.} \\
   $$
   \input{\figeconet/HMM-freq}
   $$
   &
   $$
     \begin{tikzpicture} 
  \node[hidden] (pi) at (1.5*\edgeunit, 3*\edgeunit) {$\pi$};
  
  \node[hidden] (z1) at (0*\edgeunit, 2*\edgeunit) {$Z_1$};
  \node[hidden] (z2) at (1*\edgeunit, 2*\edgeunit) {$Z_2$};
  \node[hidden] (z3) at (2*\edgeunit, 2*\edgeunit) {$Z_3$};
  \node[empty] (z4) at (3*\edgeunit, 2*\edgeunit) {$Z_4$};

  \node[observed] (y1) at (0*\edgeunit, 1*\edgeunit) {$Y_1$};
  \node[observed] (y2) at (1*\edgeunit, 1*\edgeunit) {$Y_2$};
  \node[observed] (y3) at (2*\edgeunit, 1*\edgeunit) {$Y_3$};
  \node[empty] (y4) at (3*\edgeunit, 1*\edgeunit) {$Y_4$};

  \node[hidden] (mu) at (1.5*\edgeunit, 0*\edgeunit) {$\mu$};

  \draw[arrow] (pi) to (z1);  \draw[arrow] (pi) to (z2);  \draw[arrow] (pi) to (z3);  
  \draw[dashedarrow] (pi) to (z4);
  \draw[arrow] (z1) to (z2);  \draw[arrow] (z2) to (z3);  \draw[dashedarrow] (z3) to (z4);
  \draw[arrow] (z1) to (y1);  \draw[arrow] (z2) to (y2);  \draw[arrow] (z3) to (y3);
  \draw[arrow] (mu) to (y1);  \draw[arrow] (mu) to (y2);  \draw[arrow] (mu) to (y3);
  \draw[dashedarrow] (mu) to (y4);
  \end{tikzpicture}
  

   $$
  \end{tabular}

}

%====================================================================
\frame{\frametitle{Reading conditional independence: $d$-separation}

  \paragraph{Definition.} $X$ and $Z$ are $d$-separated by a set of nodes $S$ if in any undirected path from $X$ to $Z$, there exist a node $Y$ such as
  \begin{itemize}
   \item $Y \in S$ and the path does not form a v-structure in $Y$
   \item $Y \notin S$ (nor its descendants) and the path forms a v-structure in $Y$
  \end{itemize}

  \bigskip \bigskip \pause
  \paragraph{Property.} $d$ separation is equivalent to conditional independence, i.e.
%   $$
%   \{\text{$S$ $d$-separates $X$ and $Z$}\} \qquad \Leftrightarrow \qquad \{X \independent Z \mid S\}
%   $$

  \bigskip \bigskip \pause
  \paragraph{Remark \refer{For16}.} 
  {\sl The definition $d$-separation implies that a variable is are conditionally independent from its non-descendants given its parents.}
}

%====================================================================
\frame{\frametitle{Understanding $d$-separation}

%   \begin{tabular}{p{.48\textwidth}p{.48\textwidth}}
%     $  \begin{tikzpicture} 
  \node[observed] (a) at (0*\edgeunit, 1*\edgeunit) {$A$};
  \node[observed] (b) at (1*\edgeunit, 1*\edgeunit) {$B$};
  \node[observed] (c) at (1*\edgeunit, 0*\edgeunit) {$C$};
  \node[observed] (d) at (2*\edgeunit, 1*\edgeunit) {$D$};
  \node[observed] (e) at (3*\edgeunit, 1*\edgeunit) {$E$};

  \draw[arrow] (a) to (b);  \draw[arrow] (b) to (c);
  \draw[arrow] (d) to (b);  \draw[arrow] (d) to (e);
  \end{tikzpicture}
  
$
%     & 
%     $\input{\figeconet/Dseparation2}$ 
%     \\
%     \begin{itemize}
%      \item $D$ does separate $A$ and $E$
%      \item $C$ does not separate $A$ and $E$
%      \item $\{C, D\}$ does separate $A$ and $E$
%     \end{itemize}
%     & 
%     \begin{itemize}
%      \item $D$ does separate $A$ and $E$
%      \item $C$ does not separate $A$ and $E$
%      \item $\{C, D\}$ does not separate $A$ and $E$
%     \end{itemize}
%   \end{tabular}
  

  \begin{tabular}{p{.48\textwidth}p{.48\textwidth}}
   $  \begin{tikzpicture} 
  \node[observed] (a) at (0*\edgeunit, 1*\edgeunit) {$A$};
  \node[observed] (b) at (1*\edgeunit, 1*\edgeunit) {$B$};
  \node[observed] (c) at (1*\edgeunit, 0*\edgeunit) {$C$};
  \node[observed] (d) at (2*\edgeunit, 1*\edgeunit) {$D$};
  \node[observed] (e) at (3*\edgeunit, 1*\edgeunit) {$E$};

  \draw[arrow] (a) to (b);  \draw[arrow] (b) to (c);
  \draw[arrow] (d) to (b);  \draw[arrow] (d) to (e);
  \end{tikzpicture}
  
$
   & 
   \begin{itemize}
   \item $D$ $d$-separates $A$ and $E$
   \item $C$ does not $d$-separate $A$ and $E$
   \item $\{C, D\}$ $d$-separates $A$ and $E$
   \end{itemize}
  \end{tabular}


  
}

%====================================================================
\frame{\frametitle{Conditional independence: an example}

  \paragraph{Proof} of $\{B, C\} \independent \{E, F\} \mid D$

  \begin{tabular}{cc}
    \begin{tabular}{l}
    $  \begin{tikzpicture}
  \input{\figeconet/GM7nodes-positions}
  
  \draw[edge] (a) to (b);  \draw[edge] (a) to (c);  \draw[edge] (a) to (d);
  \draw[edge] (b) to (c);  \draw[edge] (b) to (d);  \draw[edge] (c) to (d);
  \draw[edge] (d) to (e);  \draw[edge] (d) to (f);  \draw[edge] (e) to (f);
  \draw[edge] (f) to (g);  
  \end{tikzpicture}
$
    \end{tabular}
    &
    \begin{tabular}{p{.7\textwidth}}
      \begin{eqnarray*}
       p(b, c, d, e, f) 
       & = & \sum_a \sum_g p(a, b, c, d, e, f, g) \\
       & \propto & \sum_a \sum_g \psi_1(a, b, c, d) \psi_2(d, e, f) \psi_3(f, g) \\
       & \propto & \underset{\Psi_1(b, c, d)}{\underbrace{\sum_a \psi_1(a, b, c, d)}} \; \underset{\Psi_2(d, e, f)}{\underbrace{\sum_g \psi_2(d, e, f) \psi_3(f, g)}} \\
      \end{eqnarray*}
      so, for any fixed $d$,
      \begin{eqnarray*}
       p(b, c, e, f \mid d) 
       & = & p(b, c, d, e, f) \left/p(d) \right. \\
       & \propto & \Psi_1(b, c, d) \times \Psi_2(d, e, f)
      \end{eqnarray*}
    \end{tabular}
  \end{tabular}
  
}

%====================================================================
\frame{\frametitle{Causality \refer{Pea09}}

  \begin{tabular}{c|c|c}
    'Truth' & Equivalent & 'Causality' \\  
    & {based on observational data} \\  
    & & \\\hline & & \\
    \includegraphics[trim={0 15 180 0}, height=.22\textheight, clip]{../FIGURES/Pea09-Fig2-3}
    &
    \includegraphics[trim={0 15 0 0}, height=.21\textheight, clip]{../FIGURES/Pea09-Fig2-4}
    &
    \includegraphics[trim={180 15 0 0}, height=.22\textheight, clip]{../FIGURES/Pea09-Fig2-3}
  \end{tabular}

  \bigskip \bigskip 
  $$
  \{a \leftrightarrow b\} \qquad = \qquad \{a \leftarrow u \rightarrow b, 
  \quad \text{$u$ unobserved}\}
  $$

}

%====================================================================
\backupend 

%====================================================================
%====================================================================
\end{document}
%====================================================================
%====================================================================
