\documentclass[8pt]{beamer}

% Beamer style
%\usetheme[secheader]{Madrid}
% \usetheme{CambridgeUS}
\useoutertheme{infolines}
\usecolortheme[rgb={0.65,0.15,0.25}]{structure}
% \usefonttheme[onlymath]{serif}
\beamertemplatenavigationsymbolsempty
%\AtBeginSubsection

% Packages
%\usepackage[french]{babel}
\usepackage[latin1]{inputenc}
\usepackage{color}
\usepackage{xspace}
\usepackage{dsfont, stmaryrd}
\usepackage{amsmath, amsfonts, amssymb, stmaryrd}
\usepackage{epsfig}
\usepackage{tikz}
\usepackage{url}
% \usepackage{ulem}
\usepackage{/home/robin/LATEX/Biblio/astats}
%\usepackage[all]{xy}
\usepackage{graphicx}
\usepackage{xspace}

% Maths
% \newtheorem{theorem}{Theorem}
% \newtheorem{definition}{Definition}
\newtheorem{proposition}{Proposition}
% \newtheorem{assumption}{Assumption}
% \newtheorem{algorithm}{Algorithm}
% \newtheorem{lemma}{Lemma}
% \newtheorem{remark}{Remark}
% \newtheorem{exercise}{Exercise}
% \newcommand{\propname}{Prop.}
% \newcommand{\proof}{\noindent{\sl Proof:}\quad}
% \newcommand{\eproof}{$\blacksquare$}

% \setcounter{secnumdepth}{3}
% \setcounter{tocdepth}{3}
\newcommand{\pref}[1]{\ref{#1} p.\pageref{#1}}
\newcommand{\qref}[1]{\eqref{#1} p.\pageref{#1}}

% Colors : http://latexcolor.com/
\definecolor{darkred}{rgb}{0.65,0.15,0.25}
\definecolor{darkgreen}{rgb}{0,0.4,0}
\definecolor{darkred}{rgb}{0.65,0.15,0.25}
\definecolor{amethyst}{rgb}{0.6, 0.4, 0.8}
\definecolor{asparagus}{rgb}{0.53, 0.66, 0.42}
\definecolor{applegreen}{rgb}{0.55, 0.71, 0.0}
\definecolor{awesome}{rgb}{1.0, 0.13, 0.32}
\definecolor{blue-green}{rgb}{0.0, 0.87, 0.87}
\definecolor{red-ggplot}{rgb}{0.52, 0.25, 0.23}
\definecolor{green-ggplot}{rgb}{0.42, 0.58, 0.00}
\definecolor{purple-ggplot}{rgb}{0.34, 0.21, 0.44}
\definecolor{blue-ggplot}{rgb}{0.00, 0.49, 0.51}

% Commands
\newcommand{\backupbegin}{
   \newcounter{finalframe}
   \setcounter{finalframe}{\value{framenumber}}
}
\newcommand{\backupend}{
   \setcounter{framenumber}{\value{finalframe}}
}
\newcommand{\emphase}[1]{\textcolor{darkred}{#1}}
\newcommand{\comment}[1]{\textcolor{gray}{#1}}
\newcommand{\paragraph}[1]{\textcolor{darkred}{#1}}
\newcommand{\refer}[1]{{\small{\textcolor{gray}{{[\cite{#1}]}}}}}
\newcommand{\Refer}[1]{{\small{\textcolor{gray}{{[#1]}}}}}
\newcommand{\goto}[1]{{\small{\textcolor{blue}{[\#\ref{#1}]}}}}
\renewcommand{\newblock}{}

\newcommand{\tabequation}[1]{{\medskip \centerline{#1} \medskip}}
% \renewcommand{\binom}[2]{{\left(\begin{array}{c} #1 \\ #2 \end{array}\right)}}

% Variables 
\newcommand{\Abf}{{\bf A}}
\newcommand{\Beta}{\text{B}}
\newcommand{\Bcal}{\mathcal{B}}
\newcommand{\Bias}{\xspace\mathbb B}
\newcommand{\Cor}{{\mathbb C}\text{or}}
\newcommand{\Cov}{{\mathbb C}\text{ov}}
\newcommand{\cl}{\text{\it c}\ell}
\newcommand{\Ccal}{\mathcal{C}}
\newcommand{\cst}{\text{cst}}
\newcommand{\Dcal}{\mathcal{D}}
\newcommand{\Ecal}{\mathcal{E}}
\newcommand{\Esp}{\xspace\mathbb E}
\newcommand{\Espt}{\widetilde{\Esp}}
\newcommand{\Covt}{\widetilde{\Cov}}
\newcommand{\Ibb}{\mathbb I}
\newcommand{\Fcal}{\mathcal{F}}
\newcommand{\Gcal}{\mathcal{G}}
\newcommand{\Gam}{\mathcal{G}\text{am}}
\newcommand{\Hcal}{\mathcal{H}}
\newcommand{\Jcal}{\mathcal{J}}
\newcommand{\Lcal}{\mathcal{L}}
\newcommand{\Mt}{\widetilde{M}}
\newcommand{\mt}{\widetilde{m}}
\newcommand{\Nbb}{\mathbb{N}}
\newcommand{\Mcal}{\mathcal{M}}
\newcommand{\Ncal}{\mathcal{N}}
\newcommand{\Ocal}{\mathcal{O}}
\newcommand{\pt}{\widetilde{p}}
\newcommand{\Pt}{\widetilde{P}}
\newcommand{\Pbb}{\mathbb{P}}
\newcommand{\Pcal}{\mathcal{P}}
\newcommand{\Qcal}{\mathcal{Q}}
\newcommand{\qt}{\widetilde{q}}
\newcommand{\Rbb}{\mathbb{R}}
\newcommand{\Sbb}{\mathbb{S}}
\newcommand{\Scal}{\mathcal{S}}
\newcommand{\st}{\widetilde{s}}
\newcommand{\St}{\widetilde{S}}
\newcommand{\Tcal}{\mathcal{T}}
\newcommand{\todo}{\textcolor{red}{TO DO}}
\newcommand{\Ucal}{\mathcal{U}}
\newcommand{\Un}{\math{1}}
\newcommand{\Vcal}{\mathcal{V}}
\newcommand{\Var}{\mathbb V}
\newcommand{\Vart}{\widetilde{\Var}}
\newcommand{\Zcal}{\mathcal{Z}}

% Symboles & notations
\newcommand\independent{\protect\mathpalette{\protect\independenT}{\perp}}\def\independenT#1#2{\mathrel{\rlap{$#1#2$}\mkern2mu{#1#2}}} 
\renewcommand{\d}{\text{\xspace d}}
\newcommand{\gv}{\mid}
\newcommand{\ggv}{\, \| \, }
% \newcommand{\diag}{\text{diag}}
\newcommand{\card}[1]{\text{card}\left(#1\right)}
\newcommand{\trace}[1]{\text{tr}\left(#1\right)}
\newcommand{\matr}[1]{\boldsymbol{#1}}
\newcommand{\matrbf}[1]{\mathbf{#1}}
\newcommand{\vect}[1]{\matr{#1}} %% un peu inutile
\newcommand{\vectbf}[1]{\matrbf{#1}} %% un peu inutile
\newcommand{\trans}{\intercal}
\newcommand{\transpose}[1]{\matr{#1}^\trans}
\newcommand{\crossprod}[2]{\transpose{#1} \matr{#2}}
\newcommand{\tcrossprod}[2]{\matr{#1} \transpose{#2}}
\newcommand{\matprod}[2]{\matr{#1} \matr{#2}}
\DeclareMathOperator*{\argmin}{arg\,min}
\DeclareMathOperator*{\argmax}{arg\,max}
\DeclareMathOperator{\sign}{sign}
\DeclareMathOperator{\tr}{tr}
\newcommand{\ra}{\emphase{$\rightarrow$} \xspace}

% Hadamard, Kronecker and vec operators
\DeclareMathOperator{\Diag}{Diag} % matrix diagonal
\DeclareMathOperator{\diag}{diag} % vector diagonal
\DeclareMathOperator{\mtov}{vec} % matrix to vector
\newcommand{\kro}{\otimes} % Kronecker product
\newcommand{\had}{\odot}   % Hadamard product

% TikZ
\newcommand{\nodesize}{2em}
\newcommand{\edgeunit}{2.5*\nodesize}
\newcommand{\edgewidth}{1pt}
\tikzstyle{node}=[draw, circle, fill=black, minimum width=.75\nodesize, inner sep=0]
\tikzstyle{square}=[rectangle, draw]
\tikzstyle{param}=[draw, rectangle, fill=gray!50, minimum width=\nodesize, minimum height=\nodesize, inner sep=0]
\tikzstyle{hidden}=[draw, circle, fill=gray!50, minimum width=\nodesize, inner sep=0]
\tikzstyle{hiddenred}=[draw, circle, color=red, fill=gray!50, minimum width=\nodesize, inner sep=0]
\tikzstyle{observed}=[draw, circle, minimum width=\nodesize, inner sep=0]
\tikzstyle{observedred}=[draw, circle, minimum width=\nodesize, color=red, inner sep=0]
\tikzstyle{eliminated}=[draw, circle, minimum width=\nodesize, color=gray!50, inner sep=0]
\tikzstyle{empty}=[draw, circle, minimum width=\nodesize, color=white, inner sep=0]
\tikzstyle{blank}=[color=white]
\tikzstyle{nocircle}=[minimum width=\nodesize, inner sep=0]

\tikzstyle{edge}=[-, line width=\edgewidth]
\tikzstyle{edgebendleft}=[-, >=latex, line width=\edgewidth, bend left]
\tikzstyle{edgebendright}=[-, >=latex, line width=\edgewidth, bend right]
\tikzstyle{lightedge}=[-, line width=\edgewidth, color=gray!50]
\tikzstyle{lightedgebendleft}=[-, >=latex, line width=\edgewidth, bend left, color=gray!50]
\tikzstyle{lightedgebendright}=[-, >=latex, line width=\edgewidth, bend right, color=gray!50]
\tikzstyle{edgered}=[-, line width=\edgewidth, color=red]
\tikzstyle{edgebendleftred}=[-, >=latex, line width=\edgewidth, bend left, color=red]
\tikzstyle{edgebendrightred}=[-, >=latex, line width=\edgewidth, bend right, color=red]

\tikzstyle{arrow}=[->, >=latex, line width=\edgewidth]
\tikzstyle{arrowbendleft}=[->, >=latex, line width=\edgewidth, bend left]
\tikzstyle{arrowbendright}=[->, >=latex, line width=\edgewidth, bend right]
\tikzstyle{arrowred}=[->, >=latex, line width=\edgewidth, color=red]
\tikzstyle{arrowbendleftred}=[->, >=latex, line width=\edgewidth, bend left, color=red]
\tikzstyle{arrowbendrightred}=[->, >=latex, line width=\edgewidth, bend right, color=red]
\tikzstyle{arrowblue}=[->, >=latex, line width=\edgewidth, color=blue]
\tikzstyle{dashedarrow}=[->, >=latex, dashed, line width=\edgewidth]
\tikzstyle{dashededge}=[-, >=latex, dashed, line width=\edgewidth]
\tikzstyle{dashededgebendleft}=[-, >=latex, dashed, line width=\edgewidth, bend left]
\tikzstyle{lightarrow}=[->, >=latex, line width=\edgewidth, color=gray!50]

\newcommand{\GMSBM}{/home/robin/RECHERCHE/RESEAUX/EXPOSES/1903-SemStat/}
\newcommand{\figeconet}{/home/robin/RECHERCHE/ECOLOGIE/EXPOSES/1904-EcoNet-Lyon/Figs}
\newcommand{\fignet}{/home/robin/RECHERCHE/RESEAUX/EXPOSES/FIGURES}
\newcommand{\figeco}{/home/robin/RECHERCHE/ECOLOGIE/EXPOSES/FIGURES}
\newcommand{\figbayes}{/home/robin/RECHERCHE/BAYES/EXPOSES/FIGURES}
\newcommand{\figCMR}{/home/robin/Bureau/RECHERCHE/ECOLOGIE/CountPCA/sparsepca/Article/Network_JCGS/trunk/figs}
\newcommand{\figtree}{/home/robin/RECHERCHE/BAYES/VBEM-IS/VBEM-IS.git/Data/Tree/Fig}

\renewcommand{\nodesize}{1.75em}
\renewcommand{\edgeunit}{2.25*\nodesize}

%====================================================================
%====================================================================
\begin{document}
%====================================================================
%====================================================================
\title[PLN as a JSDM]{The Poisson log-normal model as a joint species distribution model}

\author[S. Robin]{S. Robin \\ ~\\
  {\small Sorbonne universit\'e, LPSM} \\ ~\\
  joint work with {\bf J. Chiquet}, {\bf M. Mariadassou} \\ ~\\ ~\\
  \nocite{CMR18a,CMR19,CMR20,CMR21}}

\date[IA \& microbiote, SCAI, nov'23]{\small Apprentissage, IA et analyse du microbiote intestinal \\ 
  ~ \\ 
  SCAI, Sorbonne Universit\'e, nov'23}

\maketitle

%====================================================================
\frame{\frametitle{Outline} \tableofcontents}

%====================================================================
%====================================================================
\section{The Poisson log-normal model}
\frame{\frametitle{Outline} \tableofcontents[currentsection]}
%====================================================================
\subsection*{Joint species distribution models}
%====================================================================
\frame{\frametitle{Joint species distribution models} \pause

  \paragraph{Foreword.} 
  \begin{itemize}
    \setlength{\itemsep}{0.5\baselineskip}
    \item Not a specialist of microbiome
    \item Used to work on applications in bioinformatics, then microbial ecology, then (macroscopic?) ecology
    \item Examples borrowed from different fields
  \end{itemize}
  
  \pause \bigskip \bigskip
  \paragraph{Joint species distribution model (JSDM \refer{WBO15}):} aim at modelling the joint distribution of the abundance of a set of 'species' accounting for
  \begin{itemize}
    \setlength{\itemsep}{0.5\baselineskip}
    \item environmental (or experimental) conditions
    \item 'interactions' between species
    \item \textcolor{gray}{heterogeneity of the sampling protocole}
  \end{itemize}
}

%====================================================================
\frame{\frametitle{Typical example} \pause

  \begin{tabular}{cc}
    \hspace{-.04\textwidth}
    \begin{tabular}{p{.5\textwidth}}
      \paragraph{Fish species in Barents sea \refer{FNA06}:} 
      \begin{itemize}
       \item $89$ sites (stations), 
       \item $30$ fish species, 
       \item $4$ covariates
      \end{itemize}

      \bigskip \bigskip \bigskip 
      \paragraph{Questions:} 
      \begin{itemize}
        \item Do environmental conditions affect species abundances? (abiotic) \\~ 
        \item Do species abundances vary independently? (biotic)
      \end{itemize} 
    \end{tabular}
    &
    \begin{tabular}{p{.45\textwidth}}
      \paragraph{Abundance table $=Y$:} ~ \\
        {\footnotesize \begin{tabular}{rrrr}
        {\sl Hi.pl}\footnote{{\sl Hi.pl}: Long rough dab, {\sl An.lu}: Atlantic wolffish, {\sl Me.ae}: Haddock} & {\sl An.lu} & {\sl Me.ae} & \dots \\
%         \\ 
  %       Dab & Wolffish & Haddock \\ 
        \hline
        31  &   0  & 108 & \\
         4  &   0  & 110 & \\
        27  &   0  & 788 & \\
        13  &   0  & 295 & \\
        23  &   0  &  13 & \\
        20  &   0  &  97 & \\
        . & . & . & 
      \end{tabular}} 
      \\
      \bigskip 
      \bigskip 
      \paragraph{Environmental covariates $=X$:} ~ \\
        {\footnotesize \begin{tabular}{rrrr}
        Lat. & Long. & Depth & Temp. \\
        \hline
        71.10 & 22.43 & 349 & 3.95 \\
        71.32 & 23.68 & 382 & 3.75 \\
        71.60 & 24.90 & 294 & 3.45 \\
        71.27 & 25.88 & 304 & 3.65 \\
        71.52 & 28.12 & 384 & 3.35 \\
        71.48 & 29.10 & 344 & 3.65 \\
        . & . & . & .
      \end{tabular}}
      \bigskip
    \end{tabular}
  \end{tabular}
  
}

%====================================================================
\frame{\frametitle{Latent variable models}

  \bigskip
  \paragraph{Modelling the dependency.}
  \begin{itemize}
    \setlength{\itemsep}{0.5\baselineskip}
    \item Not always easy to propose a joint distribution for a set of dependent variables (abundances), especially when {\sl dealing with counts}.
    \item May resort to a set of unobserved (latent) variables to encode the dependency.
  \end{itemize}
  
  \pause \bigskip \bigskip 
  \paragraph{Some popular examples of latent variable models.}
  \begin{itemize}
    \setlength{\itemsep}{0.5\baselineskip}
    \item Mixed models (e.g. to account for parental structure in genetics)
    \item Principal component analysis (for dimension reduction)
    \item Hidden Markov models (for time series, genomic structure, ...)
  \end{itemize}
  
  
  \pause \bigskip \bigskip 
  \paragraph{Many (most?) joint species distribution models:} SpiecEasi \refer{KMM15}, HMSC \refer{OTN17}, gCoda \refer{FHZ17}, MRFcov \refer{CWL18}, ...
  
}

%====================================================================
\frame{\frametitle{The Poisson log-normal (PLN) model}

  \bigskip
  \paragraph{Data at hand} in each site (or sample)
  \begin{itemize}
    \setlength{\itemsep}{0.5\baselineskip}
    \item $Y_i =$ abundance vector for the $p$ species under study:
    $$
    Y_i = [Y_{i1} \; \dots \; Y_{ip}],
    $$
    \item $x_i =$ vector of environmental covariates (with dimension $d$):
    $$
    x_i = [x_{i1} \; \dots \; x_{id}].
    $$
  \end{itemize}

  \pause \bigskip \bigskip 
  \paragraph{PLN = latent variable model:} \refer{AiH89}
  \begin{itemize}
    \setlength{\itemsep}{0.5\baselineskip}
    \item A latent Gaussian vector $Z_i$ with covariance matrix $\Sigma$ is associated to each site $i$
    $$
    Z_i = [Z_{i1} \; \dots \; Z_{ip}] \sim \Ncal_p(0, \emphase{\Sigma}).
    $$
    \item The abundance $Y_{ij}$ of species $j$ in site $i$ depends on both the covariates and the corresponding latent $Z_{ij}$:
    $$
    Y_{ij} \textcolor{gray}{\; \mid Z_{ij}} \sim \Pcal\left(\exp(\mu_{ij})\right), 
    \qquad
    \mu_{ij} = \textcolor{gray}{o_{ij}\;+\;} x_i^\top \emphase{\beta_j} + Z_{ij}.
    $$
  \end{itemize}

}

%====================================================================
\frame{\frametitle{Interpretation of the parameters}
 
  \bigskip
  \begin{tabular}{cc}
    \hspace{-.04\textwidth}
    \begin{tabular}{p{.5\textwidth}}
      \paragraph{'Environmental' effects:} 
      $$
      \beta_{hj} = \text{effect of covariate $h$ on species $j$}
      $$
      \begin{itemize}
        \item $\beta = d \times p$ regression coefficient matrix, 
        \item 'abiotic' effects.
      \end{itemize}
      \bigskip ~
    \end{tabular}
    & \pause
    \begin{tabular}{p{.45\textwidth}}
      \includegraphics[width=.325\textwidth]{\figeco/BarentsFish-coeffAll}
    \end{tabular} 
    \\
    \begin{tabular}{p{.5\textwidth}} \pause
      \paragraph{Species 'interactions':} 
      $$
      \sigma_{jk} = \text{(latent) covariance between species $j$ and $k$}
      $$
      \begin{itemize}
        \item $\Sigma= p \times p$ (latent) covariance matrix,
        \item 'biotic interactions'.
      \end{itemize}
      \bigskip ~
    \end{tabular}
    & \pause
    \begin{tabular}{p{.45\textwidth}}
      \includegraphics[width=.325\textwidth]{\figeco/BarentsFish-corrAll} 
    \end{tabular}    
  \end{tabular}
 
}

%====================================================================
\frame{\frametitle{Distinguishing between environmental effects and species interactions}

  \pause
  \begin{tabular}{cc|c}
    \multicolumn{2}{l|}{\emphase{Barents fishes: Full model}} &
    \multicolumn{1}{l}{\onslide+<3>{\emphase{Null model}}} \\
    & & \\
    \multicolumn{2}{c|}{{$Y_{ij} \sim \Pcal(\exp(\emphase{x_i^\intercal \beta_j} + Z_{ij}))$}} &
    \multicolumn{1}{c}{{\onslide+<3>{$Y_{ij} \sim \Pcal(\exp(\emphase{\mu_j} + Z_{ij}))$}}} \\
    & & \\
    \multicolumn{2}{l|}{{$x_i =$ all covariates}} &
    \multicolumn{1}{l}{{\onslide+<3>{no covariate}}} \\ 
    & & \\
    & correlations between & \\
    inferred  correlations $\widehat{\Sigma}_{\text{full}}$ & 
    predictions: $x_i^\intercal \widehat{\beta}_j$ & 
    \onslide+<3>{inferred correlations $\widehat{\Sigma}_{\text{null}}$} \\ 
    \includegraphics[width=.3\textwidth, trim=20 20 20 20]{\figeco/BarentsFish-corrAll} 
    &
    \includegraphics[width=.3\textwidth, trim=20 20 20 20]{\figeco/BarentsFish-corrPred} &
    \onslide+<3>{\includegraphics[width=.3\textwidth, trim=20 20 20 20]{\figeco/BarentsFish-corrNull}}
  \end{tabular}

  }
%====================================================================
\frame{\frametitle{Some properties of the Poisson log-normal distribution}

  \pause \bigskip
  \paragraph{Overdispersion.} Due to the random effect $Z$
  \begin{align*}
    \Var(PLN) > \Var(\text{Poisson})
  \end{align*}
  
  \pause \bigskip \bigskip
  \paragraph{Latent correlations sign ($Z$) = observed correlation sign ($Y$).}
  $$
  \sign(\sigma_{jk}) = \sign(\Cor(Y_{ij}, Y_{ik})). 
  $$

  \pause \bigskip \bigskip
  \paragraph{Sampling effort.} Offset $o_{ij}$
  $$
  Y_{ij} 
  \sim \Pcal(\exp(o_{ij} + x_i^\intercal \beta_j + Z_{ij})) 
  = \Pcal(\emphase{e^{o_{ij}}} \exp(x_i^\intercal \beta_j + Z_{ij}))
  $$
  \begin{itemize}
    \item 'macroscopic' ecology: $o_{ij} =$ log-time of observation,
    \item metabarcoding: $o_{ij} =$ log-sequencing depth.
  \end{itemize}

  \pause \bigskip \bigskip
  \paragraph{Prediction.} Expected abundance:
  $\Esp(Y_{ij}) = \exp(x_i^\intercal \beta_j + \sigma_{jj}/2)W$.

}

%====================================================================
%====================================================================
\section{Some applications and avatars}
\frame{\frametitle{Outline} \tableofcontents[currentsection]}
%====================================================================
\subsection{Regular PLN}
%====================================================================
\frame{\frametitle{Regular PLN} \pause

  \paragraph{Fruit flies in La R\'eunion.} \refer{FHC21}
  $p = 8$ fly species: 
  \begin{itemize}
    \setlength{\itemsep}{0.5\baselineskip}
    \item 4 generalists ({\it B. zonata}, {\it C. capitata}, {\it C. catoirii} and {\it C. quilicii}),  
    \item 3 specialists of Cucurbitaceae ({\it D. ciliatus}, {\it D. demmerezi} and {\it Z. cucurbitae}), 
    \item 1 specialist of Solanaceae ({\it N. cyanescens}).
  \end{itemize}
  
  \bigskip
  $n \simeq 5000$ plants
  
  \pause \bigskip \bigskip
  \paragraph{Questions.}
  \begin{itemize}
    \setlength{\itemsep}{0.5\baselineskip}
    \item Do sympatric species do actually share the same niche (e.g. plants)?
    \item Do climatic factors affect their respective abundances?
    \item Are species actually in competition?
  \end{itemize}
}

%====================================================================
\frame{\frametitle{Fruit flies in La R\'eunion} 

  \paragraph{Regular PLN model.} Estimated covariance matrices $\Sigma$:
  $$
  \includegraphics[width=.5\textwidth]{\figeco/FHC20-EcolLetters-Fig2}
  $$
  $$
  \begin{tabular}{clcl}
    \qquad & (a) No covariate & \qquad \qquad & (b) Climat \\
    \qquad & (c) Plant species & & (d) Plant species and climat
  \end{tabular}
  $$
  
  \bigskip
  \ra Weak competition among species specialists of the same plant.
}

%====================================================================
\subsection{Dimension reduction}
%====================================================================
\frame{\frametitle{Dimension reduction (PCA)} \pause

  \paragraph{Dimension reduction.} 
  \begin{itemize}
    \setlength{\itemsep}{0.5\baselineskip}
  \item Metagenomic, metabarcoding, environmental genomics: $p = 10^2, 10^3$ species.
  \item (Probabilistic) PCA \refer{Tib99}: the latent dimension of the data is actually $q \ll p$.
  \end{itemize}
  
  \pause \bigskip \bigskip 
  \paragraph{PLN-PCA model = reduced latent dimension.}  \refer{CMR18a}
  \begin{itemize}
    \setlength{\itemsep}{0.5\baselineskip}
    \item Do not need a latent vector $Z_i$ with $p$ free coordinates in each site. 
    \item \pause A latent vector $W_i$ with $q \ll p$ free coordinates suffices:
    $$
    W_i = [W_{i1} \; \dots \; W_{iq}] \sim \Ncal(0, I_q).
    $$ 
    \item \pause $Z_i$ can then be reconstructed from $W_i$ as $Z_i = B W_i$, where $B: p \times q$.
  \end{itemize}

  \pause \bigskip
  \ra The covariance matrix $\Sigma$ has rank $q \ll p$: 'low-rank' approximation.
  
  \pause \bigskip \bigskip 
  \paragraph{Model selection.} The latent dimension $q$ can be selected using standard criteria (BIC, ICL, ...).

}

%====================================================================
\frame{\frametitle{Oak powdery mildew dataset (1/2)}

  \bigskip 
  \paragraph{Metabarcoding data.} \refer{JFS16} 
  \begin{itemize}
    \setlength{\itemsep}{1\baselineskip}
    \item $p = 114$ OTUs (bacteria and fungi) \\ 
      \ra different extraction protocol for each group.
    \item $n = 116$ leaves = samples \\
      \ra collected from 3 different trees, at different positions in the tree.
    \item $Y_{ij} =$ read count for species $j$ in sample $i$.
  \end{itemize}
  
  \pause \bigskip \bigskip 
  \paragraph{Questions.}
  \begin{itemize}
    \setlength{\itemsep}{0.5\baselineskip}
    \item Effect of the covariates (tree, position) on the abundance of the species.
    \item Visualization.
    \item Account for the heterogeneity of the sampling protocol.
  \end{itemize}

  \pause \bigskip \bigskip 
  \paragraph{Use of the offset.}
  $$
  o_{ij} = \log(\text{total sequencing depth of sample $i$ for the group of species $j$}).
  $$
}

%====================================================================
\frame{\frametitle{Oak powdery mildew dataset (2/2)}

  $$
  \begin{tabular}{c}
    \includegraphics[height=.38\textheight]{\fignet/CMR18-AnnApplStat-Fig4a} \\
    \pause ~ % \\
    \includegraphics[height=.38\textheight]{\fignet/CMR18-AnnApplStat-Fig5a} 
  \end{tabular}
  $$
  \pause
  \ra Other main effects = distance to the ground, then orientation. 
}

%====================================================================
\frame{\frametitle{Microbiote of young pigs}

  \paragraph{Data.} \refer{MBE15}
  \begin{itemize}
    \item $n = 155$ piglets, 
    \item $p = 500$ OTUs ($= 90\%$ of the total abundance of the original 4031 OTU)
  \end{itemize}
  
  \pause \bigskip \bigskip 
  \paragraph{Question.} 
  Effect of weaning

  \pause \bigskip \bigskip 
  \paragraph{PLN-PCA.} No covariate:
  $$
  \includegraphics[height=.45\textheight]{\figeco/CMR18-AnnApplStat-Fig3}
  $$  

}

%====================================================================
\subsection{Network inference}
%====================================================================
\frame{\frametitle{Network inference} \pause

  \bigskip
  \paragraph{Distinguishing between direct and indirect interactions.}
  \begin{itemize}
    \setlength{\itemsep}{0.5\baselineskip}
    \item The variations of species abundance can be all correlated,
    \item Some correlations result from 'direct' interactions ($A$ eats $B$). 
    \item Some others result from 'indirect' interactions  (both $A$ and $B$ are preys of $C$).
  \end{itemize}
   \pause \medskip
  \ra Probabilistic translation: \emphase{\sl conditional} dependence vs \emphase{\sl marginal} dependence.

  \pause \bigskip \bigskip
  \paragraph{Graphical model} encodes the dependency structure into a graph $G$ in which only conditionally {\sl dependent variables} are connected.
  
  \pause \bigskip \bigskip \bigskip
  \begin{tabular}{cc}
    \hspace{-.04\textwidth}
    \begin{tabular}{p{.65\textwidth}}
      \paragraph{Example:}
      \begin{itemize}
        \setlength{\itemsep}{0.5\baselineskip}
        \item All variables (species) are dependent
        \item But 4 is independent from 1 and 2 conditionally on 3 \\ ~
        \item \textcolor{gray}{Formally: $p(y_1, y_2, y_3, y_4) \propto \psi_1(y_1, y_2, y_3) \psi_2(y_3, y_4).$}
      \end{itemize}
      \bigskip \bigskip ~
    \end{tabular}
    &
    \begin{tabular}{p{.25\textwidth}}
      \hspace{-.1\textwidth}
      \includegraphics[width=.3\textwidth, trim=90 90 50 90, clip=]{\fignet/FigGGM-4nodes}
    \end{tabular}
  \end{tabular}

}

%====================================================================
\frame{\frametitle{Network inference in the PLN model} \pause

  \bigskip
  \paragraph{PLN model.} The dependency is encoded in the Gaussian latent layer, with covariance matrix $\Sigma$.
  
  \pause \bigskip \bigskip
  \paragraph{Gaussian case.} The graph $G$ of direct interactions is given by the precision matrix
  $$
  \Omega = \Sigma^{-1}
  $$
  ({\sl partial} correlations vs {\sl marginal} correlations).
  
  \pause \bigskip \bigskip \bigskip 
  \begin{tabular}{cc}
    \hspace{-.04\textwidth}
    \begin{tabular}{p{.6\textwidth}}
      \paragraph{PLN version.} \refer{CMR19}
      \medskip
      \begin{itemize}
        \setlength{\itemsep}{1\baselineskip}
        \item Fit a PLN model, forcing $\Omega = \Sigma^{-1}$ to be sparse.
        \item Algorithm similar to the graphical lasso \refer{FHT08}. \\ ~
        \item \pause The sparsity of $\Omega$ is tuned via a penalty parameter $\lambda$.
        \item $\lambda$ can be chosen via standard criteria (BIC, ...).
      \end{itemize}
      \bigskip \bigskip ~
    \end{tabular}
    &
    \hspace{-.05\textwidth}
    \begin{tabular}{p{.4\textwidth}}
      \includegraphics[height=.5\textheight, trim=25 5 10 0, clip=]{\fignet/BarentsFish_Gfull_criteria}
    \end{tabular}
  \end{tabular}

}

%====================================================================
\frame{\frametitle{Barents' fish} 
  
  \vspace{-.05\textheight}
  \begin{tabular}{cc}
    \hspace{-.04\textwidth}
    \begin{tabular}{p{.22\textwidth}}
      \paragraph{Data:} \\ ~
      \begin{itemize}
      \item $n=89$ sites \\~
      \item $p=30$ species \\~
      \item $d=4$ covariates
        \begin{itemize}
        \item latitude
        \item longitude
        \item temperature
        \item depth
        \end{itemize}
      \end{itemize}
    \end{tabular}
    &
    \begin{tabular}{c}
      \includegraphics[height=.85\textheight]{\fignet/CMR18b-ArXiv-Fig5}
    \end{tabular}
  \end{tabular}
  }

%====================================================================
\frame{\frametitle{Oak powdery mildew}

  \begin{tabular}{cc}
    \hspace{-.04\textwidth}
    \begin{tabular}{p{.35\textwidth}}
      \paragraph{Network inference.} \refer{JFS16} ~
      
      \bigskip
      {\sl Erysiphe alphitoides} (Ea) = \\
      ~ \\
      fungus reponsible for the disease
    \end{tabular}
    &
%     \hspace{-.075\textwidth}
    \begin{tabular}{p{.65\textwidth}}
      \includegraphics[height=.7\textheight, trim=50 50 50 50, clip=]{\fignet/Oaks-p20-network}
    \end{tabular}
  \end{tabular}

  
  \pause \bigskip
  \ra See also tree-based network inference, including {\sl missing actors} \refer{SRS19,MRA20,MRA21}
}

%====================================================================
%====================================================================
\section{Inference}
\frame{\frametitle{Outline} \tableofcontents[currentsection]}
%====================================================================
\subsection*{EM algorithm}
%====================================================================
\frame{\frametitle{Inference of incomplete data models} 

  \paragraph{Maximum likelihood inference.}
  $$
%   \widehat{\theta} = \arg
  \max_\theta \; \underset{\text{likelihood}}{\underbrace{p(Y; \theta)}}
  $$
  \ra No closed form for ${p(Y; \theta) = \int \underset{\text{complete likelihood}}{\underbrace{p(Y, Z; \theta)}} \d Z}$ in most latent variable models.

  \pause \bigskip \bigskip
  \paragraph{Incomplete data models.} EM algorithm \refer{DLR77}
  $$
  \theta^{h+1} = 
  \underset{\text{\normalsize \emphase{M step}}}{\underbrace{{\argmax}_\theta}} \;
  \underset{\text{\normalsize \emphase{E step}}}{\underbrace{\Esp_{\theta^h}}} \;
  [\log p(Y, Z; \theta) \mid Y)].
  $$
  
  \pause \bigskip \bigskip 
  \paragraph{E step = critical step:} Evaluate $p(Z \mid Y; \theta)$, i.e. \\
  \medskip
  \ra Retrieve sufficient information about the latent variables $Z$ based on the observed ones $Y$.
}

%====================================================================
\subsection*{Variational approximation}
%====================================================================
\frame{\frametitle{Case of the Poisson log-normal model} 

  \bigskip
  \paragraph{Intractable E step.} No reasonably easy way to compute $p(Z_i | Y_i; \theta)$ under PLN.

  \pause \bigskip \bigskip
  \paragraph{Twisted problem.} Approximate it with some other (nice) distribution, e.g.
  $$
  p(Z_i | Y_i; \theta) \simeq \Ncal(m_i, S_i).
  $$
  \begin{itemize}
    \setlength{\itemsep}{0.5\baselineskip}
    \item Variational approximation \refer{WaJ08,BKM17} ('E' step $\to$ 'VE' step).
    \item Computationaly efficient (can deal with $\simeq 10^3$ samples or species).
  \end{itemize}
  
  \pause \bigskip \bigskip
  \paragraph{But...} 
  \begin{itemize}
    \setlength{\itemsep}{0.5\baselineskip}
    \item $\neq$ Maximum likelihood
    \item No statistical guaranty (consistency, asymptotic normality, asymptotic variance, ...).
    \item No direct test or confidence interval.
    \item Alternative methods or additional steps are needed (bootstrap, jackknife, Monte-Carlo).
  \end{itemize}

}

%====================================================================
%====================================================================
\section{Conclusion}
%====================================================================
\frame{\frametitle{Conclusion} 

  \paragraph{Poisson log-normal model.} \refer{CMR20}
  \begin{itemize}
    \setlength{\itemsep}{1\baselineskip}
    \item Flexible modeling framework for multivariate abundance data.
    \item \pause Other avatars: sample classification, sample clustering, ...
    \item \pause Implemented in the \emphase{R package \url{PLNmodels}} .
    \item \pause Jackknife estimates of the variances (with computational cost).
  \end{itemize}

  \pause \bigskip \bigskip 
  \paragraph{On-going works:}
  \begin{itemize}
    \setlength{\itemsep}{1\baselineskip}
    \item Include species traits.
    \item Account for dependence between sites (or samples).
    \item 'Zero-inflated' version.
    \item Monte-Carlo E step to achieve maximum (composite) likelihood.
  \end{itemize}

}
  
%====================================================================
%====================================================================
\backupbegin 
\section*{Backup}
%====================================================================
\frame[allowframebreaks]{ \frametitle{References}
  {%\footnotesize
   \tiny
   \bibliography{/home/robin/Biblio/BibGene}
%    \bibliographystyle{/home/robin/LATEX/Biblio/astats}
   \bibliographystyle{alpha}
  }
}

%====================================================================
\frame{\frametitle{Microbiote of young pigs}

  \paragraph{Effect of species selection.} 
  $$
  \includegraphics[height=.6\textheight]{\figeco/CMR18-AnnApplStat-Fig2}
  $$  

}

%====================================================================
\frame{\frametitle{Oak powdery mildew}

  \paragraph{Tree-specific network inference.}
  $$
  \begin{tabular}{ccc}
    susceptible & intermediate & resistant \\
    \includegraphics[height=.42\textheight, trim=70 70 70 70, clip=]{\fignet/Oaks-p20-network-susceptible} &
    \includegraphics[height=.42\textheight, trim=70 70 70 70, clip=]{\fignet/Oaks-p20-network-intermediate} &
    \includegraphics[height=.42\textheight, trim=70 70 70 70, clip=]{\fignet/Oaks-p20-network-resistant}
  \end{tabular}
  $$
}

%====================================================================
\backupend 

%====================================================================
%====================================================================
\end{document}
%====================================================================
%====================================================================

  \begin{tabular}{cc}
    \hspace{-.04\textwidth}
    \begin{tabular}{p{.5\textwidth}}
    \end{tabular}
    &
    \begin{tabular}{p{.45\textwidth}}
    \end{tabular}
  \end{tabular}

