\documentclass[8pt]{beamer}

% Beamer style
%\usetheme[secheader]{Madrid}
% \usetheme{CambridgeUS}
\useoutertheme{infolines}
\usecolortheme[rgb={0.65,0.15,0.25}]{structure}
% \usefonttheme[onlymath]{serif}
\beamertemplatenavigationsymbolsempty
%\AtBeginSubsection

% Packages
%\usepackage[french]{babel}
\usepackage[latin1]{inputenc}
\usepackage{color}
\usepackage{xspace}
\usepackage{dsfont, stmaryrd}
\usepackage{amsmath, amsfonts, amssymb, stmaryrd}
\usepackage{epsfig}
\usepackage{tikz}
\usepackage{url}
% \usepackage{ulem}
\usepackage{/home/robin/LATEX/Biblio/astats}
%\usepackage[all]{xy}
\usepackage{graphicx}
\usepackage{xspace}

% Maths
% \newtheorem{theorem}{Theorem}
% \newtheorem{definition}{Definition}
\newtheorem{proposition}{Proposition}
% \newtheorem{assumption}{Assumption}
% \newtheorem{algorithm}{Algorithm}
% \newtheorem{lemma}{Lemma}
% \newtheorem{remark}{Remark}
% \newtheorem{exercise}{Exercise}
% \newcommand{\propname}{Prop.}
% \newcommand{\proof}{\noindent{\sl Proof:}\quad}
% \newcommand{\eproof}{$\blacksquare$}

% \setcounter{secnumdepth}{3}
% \setcounter{tocdepth}{3}
\newcommand{\pref}[1]{\ref{#1} p.\pageref{#1}}
\newcommand{\qref}[1]{\eqref{#1} p.\pageref{#1}}

% Colors : http://latexcolor.com/
\definecolor{darkred}{rgb}{0.65,0.15,0.25}
\definecolor{darkgreen}{rgb}{0,0.4,0}
\definecolor{darkred}{rgb}{0.65,0.15,0.25}
\definecolor{amethyst}{rgb}{0.6, 0.4, 0.8}
\definecolor{asparagus}{rgb}{0.53, 0.66, 0.42}
\definecolor{applegreen}{rgb}{0.55, 0.71, 0.0}
\definecolor{awesome}{rgb}{1.0, 0.13, 0.32}
\definecolor{blue-green}{rgb}{0.0, 0.87, 0.87}
\definecolor{red-ggplot}{rgb}{0.52, 0.25, 0.23}
\definecolor{green-ggplot}{rgb}{0.42, 0.58, 0.00}
\definecolor{purple-ggplot}{rgb}{0.34, 0.21, 0.44}
\definecolor{blue-ggplot}{rgb}{0.00, 0.49, 0.51}

% Commands
\newcommand{\backupbegin}{
   \newcounter{finalframe}
   \setcounter{finalframe}{\value{framenumber}}
}
\newcommand{\backupend}{
   \setcounter{framenumber}{\value{finalframe}}
}
\newcommand{\emphase}[1]{\textcolor{darkred}{#1}}
\newcommand{\comment}[1]{\textcolor{gray}{#1}}
\newcommand{\paragraph}[1]{\textcolor{darkred}{#1}}
\newcommand{\refer}[1]{{\small{\textcolor{gray}{{\cite{#1}}}}}}
\newcommand{\Refer}[1]{{\small{\textcolor{gray}{{[#1]}}}}}
\newcommand{\goto}[1]{{\small{\textcolor{blue}{[\#\ref{#1}]}}}}
\renewcommand{\newblock}{}

\newcommand{\tabequation}[1]{{\medskip \centerline{#1} \medskip}}
% \renewcommand{\binom}[2]{{\left(\begin{array}{c} #1 \\ #2 \end{array}\right)}}

% Variables 
\newcommand{\Abf}{{\bf A}}
\newcommand{\Beta}{\text{B}}
\newcommand{\Bcal}{\mathcal{B}}
\newcommand{\Bias}{\xspace\mathbb B}
\newcommand{\Cor}{{\mathbb C}\text{or}}
\newcommand{\Cov}{{\mathbb C}\text{ov}}
\newcommand{\cl}{\text{\it c}\ell}
\newcommand{\Ccal}{\mathcal{C}}
\newcommand{\cst}{\text{cst}}
\newcommand{\Dcal}{\mathcal{D}}
\newcommand{\Ecal}{\mathcal{E}}
\newcommand{\Esp}{\xspace\mathbb E}
\newcommand{\Espt}{\widetilde{\Esp}}
\newcommand{\Covt}{\widetilde{\Cov}}
\newcommand{\Ibb}{\mathbb I}
\newcommand{\Fcal}{\mathcal{F}}
\newcommand{\Gcal}{\mathcal{G}}
\newcommand{\Gam}{\mathcal{G}\text{am}}
\newcommand{\Hcal}{\mathcal{H}}
\newcommand{\Jcal}{\mathcal{J}}
\newcommand{\Lcal}{\mathcal{L}}
\newcommand{\Mt}{\widetilde{M}}
\newcommand{\mt}{\widetilde{m}}
\newcommand{\Nbb}{\mathbb{N}}
\newcommand{\Mcal}{\mathcal{M}}
\newcommand{\Ncal}{\mathcal{N}}
\newcommand{\Ocal}{\mathcal{O}}
\newcommand{\pt}{\widetilde{p}}
\newcommand{\Pt}{\widetilde{P}}
\newcommand{\Pbb}{\mathbb{P}}
\newcommand{\Pcal}{\mathcal{P}}
\newcommand{\Qcal}{\mathcal{Q}}
\newcommand{\qt}{\widetilde{q}}
\newcommand{\Rbb}{\mathbb{R}}
\newcommand{\Sbb}{\mathbb{S}}
\newcommand{\Scal}{\mathcal{S}}
\newcommand{\st}{\widetilde{s}}
\newcommand{\St}{\widetilde{S}}
\newcommand{\Tcal}{\mathcal{T}}
\newcommand{\todo}{\textcolor{red}{TO DO}}
\newcommand{\Ucal}{\mathcal{U}}
\newcommand{\Un}{\math{1}}
\newcommand{\Vcal}{\mathcal{V}}
\newcommand{\Var}{\mathbb V}
\newcommand{\Vart}{\widetilde{\Var}}
\newcommand{\Zcal}{\mathcal{Z}}

% Symboles & notations
\newcommand\independent{\protect\mathpalette{\protect\independenT}{\perp}}\def\independenT#1#2{\mathrel{\rlap{$#1#2$}\mkern2mu{#1#2}}} 
\renewcommand{\d}{\text{\xspace d}}
\newcommand{\gv}{\mid}
\newcommand{\ggv}{\, \| \, }
% \newcommand{\diag}{\text{diag}}
\newcommand{\card}[1]{\text{card}\left(#1\right)}
\newcommand{\trace}[1]{\text{tr}\left(#1\right)}
\newcommand{\matr}[1]{\boldsymbol{#1}}
\newcommand{\matrbf}[1]{\mathbf{#1}}
\newcommand{\vect}[1]{\matr{#1}} %% un peu inutile
\newcommand{\vectbf}[1]{\matrbf{#1}} %% un peu inutile
\newcommand{\trans}{\intercal}
\newcommand{\transpose}[1]{\matr{#1}^\trans}
\newcommand{\crossprod}[2]{\transpose{#1} \matr{#2}}
\newcommand{\tcrossprod}[2]{\matr{#1} \transpose{#2}}
\newcommand{\matprod}[2]{\matr{#1} \matr{#2}}
\DeclareMathOperator*{\argmin}{arg\,min}
\DeclareMathOperator*{\argmax}{arg\,max}
\DeclareMathOperator{\sign}{sign}
\DeclareMathOperator{\tr}{tr}
\newcommand{\ra}{\emphase{$\rightarrow$} \xspace}

% Hadamard, Kronecker and vec operators
\DeclareMathOperator{\Diag}{Diag} % matrix diagonal
\DeclareMathOperator{\diag}{diag} % vector diagonal
\DeclareMathOperator{\mtov}{vec} % matrix to vector
\newcommand{\kro}{\otimes} % Kronecker product
\newcommand{\had}{\odot}   % Hadamard product

% TikZ
\newcommand{\nodesize}{2em}
\newcommand{\edgeunit}{2.5*\nodesize}
\newcommand{\edgewidth}{1pt}
\tikzstyle{node}=[draw, circle, fill=black, minimum width=.75\nodesize, inner sep=0]
\tikzstyle{square}=[rectangle, draw]
\tikzstyle{param}=[draw, rectangle, fill=gray!50, minimum width=\nodesize, minimum height=\nodesize, inner sep=0]
\tikzstyle{hidden}=[draw, circle, fill=gray!50, minimum width=\nodesize, inner sep=0]
\tikzstyle{hiddenred}=[draw, circle, color=red, fill=gray!50, minimum width=\nodesize, inner sep=0]
\tikzstyle{observed}=[draw, circle, minimum width=\nodesize, inner sep=0]
\tikzstyle{observedred}=[draw, circle, minimum width=\nodesize, color=red, inner sep=0]
\tikzstyle{eliminated}=[draw, circle, minimum width=\nodesize, color=gray!50, inner sep=0]
\tikzstyle{empty}=[draw, circle, minimum width=\nodesize, color=white, inner sep=0]
\tikzstyle{blank}=[color=white]
\tikzstyle{nocircle}=[minimum width=\nodesize, inner sep=0]

\tikzstyle{edge}=[-, line width=\edgewidth]
\tikzstyle{edgebendleft}=[-, >=latex, line width=\edgewidth, bend left]
\tikzstyle{edgebendright}=[-, >=latex, line width=\edgewidth, bend right]
\tikzstyle{lightedge}=[-, line width=\edgewidth, color=gray!50]
\tikzstyle{lightedgebendleft}=[-, >=latex, line width=\edgewidth, bend left, color=gray!50]
\tikzstyle{lightedgebendright}=[-, >=latex, line width=\edgewidth, bend right, color=gray!50]
\tikzstyle{edgered}=[-, line width=\edgewidth, color=red]
\tikzstyle{edgebendleftred}=[-, >=latex, line width=\edgewidth, bend left, color=red]
\tikzstyle{edgebendrightred}=[-, >=latex, line width=\edgewidth, bend right, color=red]

\tikzstyle{arrow}=[->, >=latex, line width=\edgewidth]
\tikzstyle{arrowbendleft}=[->, >=latex, line width=\edgewidth, bend left]
\tikzstyle{arrowbendright}=[->, >=latex, line width=\edgewidth, bend right]
\tikzstyle{arrowred}=[->, >=latex, line width=\edgewidth, color=red]
\tikzstyle{arrowbendleftred}=[->, >=latex, line width=\edgewidth, bend left, color=red]
\tikzstyle{arrowbendrightred}=[->, >=latex, line width=\edgewidth, bend right, color=red]
\tikzstyle{arrowblue}=[->, >=latex, line width=\edgewidth, color=blue]
\tikzstyle{dashedarrow}=[->, >=latex, dashed, line width=\edgewidth]
\tikzstyle{dashededge}=[-, >=latex, dashed, line width=\edgewidth]
\tikzstyle{dashededgebendleft}=[-, >=latex, dashed, line width=\edgewidth, bend left]
\tikzstyle{lightarrow}=[->, >=latex, line width=\edgewidth, color=gray!50]

\newcommand{\GMSBM}{/home/robin/RECHERCHE/RESEAUX/EXPOSES/1903-SemStat/}
\newcommand{\figeconet}{/home/robin/RECHERCHE/ECOLOGIE/EXPOSES/1904-EcoNet-Lyon/Figs}
\newcommand{\fignet}{/home/robin/RECHERCHE/RESEAUX/EXPOSES/FIGURES}
\newcommand{\figeco}{/home/robin/RECHERCHE/ECOLOGIE/EXPOSES/FIGURES}
\newcommand{\figbayes}{/home/robin/RECHERCHE/BAYES/EXPOSES/FIGURES}
\newcommand{\figCMR}{/home/robin/Bureau/RECHERCHE/ECOLOGIE/CountPCA/sparsepca/Article/Network_JCGS/trunk/figs}
\newcommand{\figtree}{/home/robin/RECHERCHE/BAYES/VBEM-IS/VBEM-IS.git/Data/Tree/Fig}

\renewcommand{\nodesize}{1.75em}
\renewcommand{\edgeunit}{2.25*\nodesize}

%====================================================================
%====================================================================
\begin{document}
%====================================================================
%====================================================================
\title[Poisson log-normal model]{Variational inference of the Poisson log-normal model}

\author[S. Robin]{S. Robin \\ ~\\
  {\small Sorbonne universit\'e (LPSM)} \\ ~\\ ~\\ ~\\
  joint work with J. Chiquet, M. Mariadassou % \\ ~\\ ~\\
  % \refer{CMR21}: \Refer{\tt www.frontiersin.org/article/10.3389/fevo.2021.588292}
  }

\date[Lille, Apr'22]{S\'eminaire Lille, April 2022}

\maketitle

%====================================================================
\frame{\frametitle{Outline} \tableofcontents}

%====================================================================
%====================================================================
\section{The Poisson log-normal model}
% \frame{\frametitle{Outline} \tableofcontents[currentsection]}
%====================================================================
\subsection*{Multivariate count data}
%====================================================================
\frame{\frametitle{Multivariate count data} 

  \paragraph{General setting.} Consider a dataset made of $n$ observations $\{(x_i, Y_i)\}_{1 \leq i \leq n}$:
  \begin{align*}
   x_i & = (x_{i1}, \dots x_{id}) \in \Rbb^d: \text{vector of covariates (regressors, descriptors)},\\
   \\
   Y_i & = (Y_{i1}, \dots Y_{ip}) \in \Nbb^p: \text{vector of counts (responses)}
  \end{align*}
  
  \bigskip \bigskip \pause
  \paragraph{Aim.} Model $Y_i$ as a function of $x_i$:
  $$
  Y_i \sim \Fcal(x_i, \theta)
  $$
  accouting for dependency between the counts $\{Y_{ij}\}_{1 \leq j \leq p}$.

}

%====================================================================
\frame{\frametitle{Some examples} 

  \paragraph{Community ecology:} $n$ sites, $p$ species,
  \begin{align*}
  Y_{ij} & = \text{abundance of species $j$ in site $i$}, \\
  x_i & = \text{environmental descriptors of site $i$}. 
  \end{align*}
    
  \bigskip \bigskip \pause 
  \paragraph{Microbiome:} $n$ patients, $p$ OTUs, 
  \begin{align*}
  Y_{ij} & = \text{number of reads arising from OTU $j$ in the microbiome of patient $i$}, \\
  x_i & = \text{descriptors/treatment of patient $i$}.
  \end{align*}
    
  \bigskip \bigskip \pause 
  \paragraph{Single cell:} $n$ cells, $p$ genes, 
  \begin{align*}
  Y_{ij} & = \text{number of reads mapped on gene $j$ in cell $i$}, \\
  x_i & = \text{characteristics of cell $i$}.
  \end{align*}

}
  
%====================================================================
\frame{\frametitle{Models for multivariate counts} 

  No that many generic model for multivariate counts.
  
  \bigskip \bigskip
  \paragraph{Data transformation:} ($\widetilde{Y}_{ij} = \log (1+Y_{ij}), (\sqrt{Y_{ij}} + \sqrt{1+Y_{ij}})/2$) 
  \begin{itemize}
  \item Unknown resulting distribution, pb when many counts are zero. 
  \end{itemize}
   
  \bigskip \bigskip \pause
  \paragraph{Poisson multivariate distributions.}
  \begin{itemize}
  \item Constraints of the form of the dependency \refer{IYA17} 
  \end{itemize}
   
  \bigskip \bigskip \pause
  \paragraph{Latent variable models.} \\
  \begin{itemize}
  \item Poisson-Gamma (= negative binomial): positive dependency 
  \item \emphase{Poisson-log normal} \refer{AiH89}
  \end{itemize}

}

%====================================================================
\frame{\frametitle{Running example: Community ecology} \pause

  \begin{tabular}{cc}
    \hspace{-.04\textwidth}
    \begin{tabular}{p{.5\textwidth}}
      \paragraph{Fish species in Barents sea \refer{FNA06}:} 
      \begin{itemize}
       \item $89$ sites (stations), 
       \item $30$ fish species, 
       \item $4$ covariates
      \end{itemize}

      \bigskip \bigskip 
      \paragraph{Questions:} 
      \begin{itemize}
       \item Do environmental conditions affect species' abundances? (abiotic) \\~ 
       \item Do species' abundances vary independently? (biotic)
      \end{itemize} 
      
      \bigskip
      See \refer{WBO15}
    \end{tabular}
    &
    \begin{tabular}{p{.45\textwidth}}
      \paragraph{Abundance table $=Y$:} ~ \\
        {\footnotesize \begin{tabular}{rrrr}
        {\sl Hi.pl} & {\sl An.lu} & {\sl Me.ae} & \dots \\
%         \footnote{{\sl Hi.pl}: Long rough dab, {\sl An.lu}: Atlantic wolffish, {\sl Me.ae}: Haddock} \\ 
  %       Dab & Wolffish & Haddock \\ 
        \hline
        31  &   0  & 108 & \\
         4  &   0  & 110 & \\
        27  &   0  & 788 & \\
        13  &   0  & 295 & \\
        23  &   0  &  13 & \\
        20  &   0  &  97 & \\
        . & . & . & 
      \end{tabular}} 
      \\
      \bigskip \bigskip 
      \paragraph{Environmental covariates $=X$:} ~ \\
        {\footnotesize \begin{tabular}{rrrr}
        Lat. & Long. & Depth & Temp. \\
        \hline
        71.10 & 22.43 & 349 & 3.95 \\
        71.32 & 23.68 & 382 & 3.75 \\
        71.60 & 24.90 & 294 & 3.45 \\
        71.27 & 25.88 & 304 & 3.65 \\
        71.52 & 28.12 & 384 & 3.35 \\
        71.48 & 29.10 & 344 & 3.65 \\
        . & . & . & .
      \end{tabular}} 
    \end{tabular}
  \end{tabular}  
}

%====================================================================
\frame{\frametitle{Poisson log-normal (PLN) model \refer{AiH89}}
 
  \bigskip 
  \begin{itemize}
    \item {$Z_i =$ latent vector} associated with site $i$
      $$
      Z_i \sim \Ncal_p(0, {\Sigma})
      $$
      \item $Y_{ij} = $ observed abundance for species $j$ in site $i$
      $$
      Y_{ij} \sim \Pcal(\exp(\textcolor{gray}{o_{ij} \,+\,} x_i^\intercal {\beta_j} + Z_{ij}))
      $$
  \end{itemize}
  \ra Parameter \emphase{${\theta} = (\beta, \Sigma)$}
% (\beta = [\beta_{hj}]_{1 \leq h \leq d, 1 \leq j \leq p}, \Sigma = [\sigma_{jk}]_{1 \leq j,k \leq p})

  \pause \bigskip \bigskip
  \paragraph{Some properties:} 
  \begin{align*}
    \text{Prediction:} & & 
    \Esp(Y_{ij}) & = \exp(x_i^\intercal \beta_j + \emphase{\sigma_{jj}/2}) \\ ~ \\
    \text{Overdispersion:} & & 
    \Var(Y_{ij}) & = \Esp(Y_{ij}) + \Esp^2(Y_{ij}) (e^{\sigma_{jj}} - 1)
    & \geq \; \Esp(Y_{ij}) \\ ~ \\
    \text{Correlation sign:} & &
    \sign(\sigma_{jk}) & = \sign(\Cor(Y_{ij}, Y_{ik})) 
  \end{align*}
}

%====================================================================
\frame{\frametitle{PLN for Barents' fishes}
 
  \begin{tabular}{cc}
    \hspace{-.04\textwidth}
    \begin{tabular}{p{.55\textwidth}}
      \paragraph{Data:} 
      \begin{itemize}
       \item $n=89$ sites, $p=30$ species, $d=4$ covariates \\ ~
       \item Abundance table ($n \times p$): ${Y} = [Y_{ij}]$ \\ ~
       \item Covariate table ($n \times d$): ${X} = [x_{ih}]$ 
      \end{itemize}
 
      \pause \bigskip \bigskip \bigskip 
      \paragraph{Interpretation:}
      \begin{itemize}
        \item $\beta = [\beta_{hj}] =$ abiotic effects (specific to each species) \\ ~
        \item $\Sigma = [\sigma_{jk}] =$ biotic covariance (same for all sites) 
      \end{itemize}
    \end{tabular}
    & \pause
    \begin{tabular}{p{.45\textwidth}}
      \paragraph{Regression cofficients $\widehat{\beta}$:} abiotic \\ 
      \includegraphics[width=.3\textwidth]{\figeco/BarentsFish-coeffAll} \\
      ~\\
      \paragraph{Covariance matrix $\widehat{\Sigma}$:} biotic \\ 
      \includegraphics[width=.3\textwidth]{\figeco/BarentsFish-corrAll} 
    \end{tabular}    
  \end{tabular}
 
}

%====================================================================
%====================================================================
\section{(Variational) inference of incomplete data models}
\frame{\frametitle{Outline} \tableofcontents[currentsection]}
%====================================================================
\subsection*{A reminder on the EM algorithm}
%====================================================================
\frame{\frametitle{A reminder on the EM algorithm} 

  \paragraph{Maximum likelihood inference.}
  $$
%   \widehat{\theta} = \arg
  \max_\theta \; \log p(Y; \theta)
  $$
  \ra Most latent variable models: no closed form for ${p(Y; \theta) = \int \underset{\text{complete likelihood}}{\underbrace{p(Y, Z; \theta)}} \d Z}$ 

  \pause \bigskip \bigskip
  \paragraph{Incomplete data models.} EM algorithm \refer{DLR77}
  $$
  \log p(Y; \theta) 
  = \underset{\text{\normalsize \emphase{M step}}}{\underbrace{\Esp(\log p(Y, Z; \theta) \mid Y)}} 
  - \Esp(\log \underset{\text{\normalsize \emphase{E step}}}{\underbrace{p(Z \mid Y; \theta)}} \mid Y)
  $$
  
%   \pause \bigskip
%   \paragraph{Expectation-maximisation algorithm.}
%   \begin{description}
%    \item[E step:] Evaluate (some moments of) $p(Z \mid Y)$
%    \item[M step:] Maximize $\Esp(\log p(Y, Z; \theta) \mid Y )$ with respect to $\theta$
%   \end{description}

  \pause \bigskip \bigskip
  \paragraph{E step = critical step.} Evaluate $p(Z \mid Y; \theta)$
  \begin{itemize}
  \item easy: mixture models, Gaussian mixed-models
  \item use a trick: hidden Markov models, phylogenetic models
  \item intractable: many models, including PLN
  \end{itemize}
}

%====================================================================
\subsection*{Variational EM algorithms}
%====================================================================
\frame{\frametitle{Variational approximation} 

  \paragraph{Twisted problem:} \refer{WaJ08,BKM17} maximize the lower bound
  \begin{align*}
  J(\theta, q)
  & = \log p(Y; \theta) 
  - \underset{\text{\normalsize {VE step}}}{\underbrace{\emphase{KL(q(Z) \mid\mid p(Z \mid Y))}}} \\
  & = 
  \underset{\text{\normalsize {M step}}}{\underbrace{\emphase{\Esp_q (\log p(Y, Z; \theta))}}} 
  - \Esp_q(\log q(Z)) 
  \end{align*}
  where $q(Z) \simeq p(Z \mid Y)$ is chosen within the class of \emphase{approximate distributions $\Qcal$}
  
%   \pause \bigskip \bigskip
%   \paragraph{Many avatars.} 
%   \begin{itemize}
%     \item Alternative divergences: $\alpha$-divergences \refer{Min05}, $KL(p(Z \mid Y) \mid\mid q(Z))$\footnote{$KL(p \mid\mid q) = \Esp_p \log(p/q)$} (expectation-propagation = EP \refer{Min01}) 
%     \item Bayesian inference (variational Bayes = VB): look for $q(\theta) \simeq p(\theta \mid Y)$ 
%     \item Bayesian inference for incomplete data model (VBEM): $q(\theta, Z) \simeq p(\theta, Z \mid Y)$ \refer{BeG03} 
%   \end{itemize} 
  
%   \bigskip
%   \ra Reasonably easy to implement and computationally efficient
  
  \pause \bigskip \bigskip \bigskip
  \paragraph{Critical choice:} $\Qcal$ needs to be 
  \begin{itemize}
  \item 'large' enough to include good approximations of $p(Z \mid Y)$ 
  \item 'small' enough to make the calculations tractable
  \end{itemize}

}

%====================================================================
\frame{\frametitle{VEM for the Poisson log-normal model} 

%   \paragraph{Critical choice:} $\Qcal$ needs to be 
%   \begin{itemize}
%   \item 'large' enough to include good approximations of $p(Z \mid Y)$ 
%   \item 'small' enough to make the calculations tractable
%   \end{itemize}
  
%   \pause \bigskip \bigskip
  \paragraph{Approximate distribution:} 
  $\Qcal := \{\text{multivariate Gaussian distributions}\}$
  $$
  p(Z_i \mid Y_i) \simeq q_i(Z_i) := \Ncal_p(Z_i; {m_i, S_i}), 
  $$
  $\{(m_i, S_i)\}_{1 \leq i \leq n}$ = {\sl variational parameters}.
  
  \pause \bigskip \bigskip \bigskip
  \paragraph{VE step:} minimize wrt $\{(m_i, S_i)\}_{1 \leq i \leq n}$
  $$
  KL(q(Z) \mid\mid p(Z \mid Y; \theta)) = KL(q(Z) \mid\mid p(Z, Y; \theta)) + \text{cst}
  $$
  \ra convex problem
  
  \pause \bigskip \bigskip 
  \paragraph{M step:} maximize wrt $\theta = (\beta, \Sigma)$
  $$
  \Esp_q (\log p(Y, Z; \theta))
  $$
  \ra convex problem (weighted GLM)
}

%====================================================================
%====================================================================
\section{Avatars of the Poisson log-normal model: Modeling $\Sigma$}
\frame{\frametitle{Outline} \tableofcontents[currentsection]}
%====================================================================
\subsection{Basic model}
%====================================================================
\frame{\frametitle{Abiotic vs biotic interactions} \pause

%   \paragraph{Variational EM \refer{CMR18a,CMR19}:} $p(Z_i \mid Y_i) \simeq \Ncal_p(Z_i; m_i, S_i)$
  
%   \pause \bigskip 
  \begin{tabular}{cc|c}
    \multicolumn{2}{l|}{\emphase{Barents fishes: Full model}} &
    \multicolumn{1}{l}{\onslide+<3>{\emphase{Null model}}} \\
    & & \\
    \multicolumn{2}{c|}{{$Y_{ij} \sim \Pcal(\exp(\emphase{x_i^\intercal \beta_j} + Z_{ij}))$}} &
    \multicolumn{1}{c}{{\onslide+<3>{$Y_{ij} \sim \Pcal(\exp(\emphase{\mu_j} + Z_{ij}))$}}} \\
    & & \\
    \multicolumn{2}{l|}{{$x_i =$ all covariates}} &
    \multicolumn{1}{l}{{\onslide+<3>{no covariate}}} \\ 
    & & \\
    & correlations between & \\
    inferred  correlations $\widehat{\Sigma}_{\text{full}}$ & 
    predictions: $x_i^\intercal \widehat{\beta}_j$ & 
    \onslide+<3>{inferred correlations $\widehat{\Sigma}_{\text{null}}$} \\ 
    \includegraphics[width=.3\textwidth, trim=20 20 20 20]{\figeco/BarentsFish-corrAll} 
    &
    \includegraphics[width=.3\textwidth, trim=20 20 20 20]{\figeco/BarentsFish-corrPred} &
    \onslide+<3>{\includegraphics[width=.3\textwidth, trim=20 20 20 20]{\figeco/BarentsFish-corrNull}}
  \end{tabular}

}

% %====================================================================
% \frame{\frametitle{Ease of modeling with latent variables} 
% 
%   \paragraph{Latent layer = multivariate Gaussian:} flexible dependency structure (\url{PLNmodels} package)
%   
%   \pause \bigskip \bigskip
%   \paragraph{Dimension reduction.} 
%   \begin{itemize}
%   \item Metagenomic, metabarcoding, environmental genomics: $p = 10^2, 10^3$ species
%   \item Probabilistic PCA \refer{Tib99}: suppose that \emphase{$\Sigma$ has rank $r \ll p$}
%   \item PLN-PCA: PLN model with rank constraint on $\Sigma$ \refer{CMR18a}
%   \end{itemize}
%   
%   \pause \bigskip \bigskip
%   \paragraph{Network inference.} 
%   \begin{itemize}
%   \item $\Sigma$ include both direct and indirect interactions
%   \item Gaussian graphical models: \emphase{$\Sigma^{-1}$ should be sparse}
%   \item PLN-network: PLN model with graphical lasso penalty \refer{CMR19}
%   \end{itemize}
%   $$
%   \arg\max_{\beta, \Sigma, q \in \Qcal} \; 
%   \underset{{\text{\normalsize log-likelihood}}}{\underbrace{\log p(Y; \beta, \Sigma)}}
%   - \underset{{\text{\normalsize variational approx.}}}{\underbrace{KL(q(Z) \mid\mid p(Z \mid Y))}}
%   - \underset{{\text{\normalsize $\ell_1$ penalty}}}{\underbrace{\emphase{\lambda \|\Sigma^{-1}\|_1}}}
%   $$
%   
% }
% 
%====================================================================
\subsection{Dimension reduction}
%====================================================================
\frame{\frametitle{PLN for dimension reduction (PCA)} \pause

  \paragraph{Dimension reduction.} 
  \begin{itemize}
  \item Metagenomic, environmental genomics, gene expression: $p = 10^2, 10^3$ species
  \item Probabilistic PCA \refer{Tib99}: suppose that \emphase{$\Sigma$ has rank $r \ll p$}
  \end{itemize}
  
  \pause \bigskip \bigskip 
  \paragraph{PLN-PCA = PLN model with rank constraint on $\Sigma$ \refer{CMR18a}.}
  For each independent site $i$:
  $$
  \text{latent } W_i \sim \Ncal_{\emphase{r}}(0, {I}),
  \qquad
  Z_i = \underset{p \times r}{\underbrace{B}} W_i
  \qquad \Rightarrow \qquad 
  \Sigma = B B^\intercal 
  $$
  
  \pause \bigskip 
  \paragraph{Model selection =} choosing $r$
  \begin{align*}
    vBIC & = J(\widehat{\theta}, \widehat{q}) - \text{pen}_{BIC}, & 
    \text{pen}_{BIC} & = \frac{\log n}2 \left(pd + \frac{(p-r+1)(p+r)}2\right), \\ 
    \\
    vICL & = J(\widehat{\theta}, \widehat{q}) - \text{pen}_{BIC} - \underset{\text{entropy}}{\underbrace{\Hcal(q)}}. 
  \end{align*}

}

%====================================================================
\frame{\frametitle{Oak powdery mildew dataset}
  \begin{tabular}{cl}
    \hspace{-.04\textwidth}
    \begin{tabular}{p{.3\textwidth}}
      \paragraph{Metabarcoding data} \refer{JFS16} \\~ \\~
      \begin{itemize}
      \item $p = 114$ OTUs \\
      (bacteria and fungi \\
      \ra use of offsets) \\~
      \item $n = 116$ leaves \\~ \\~
      \item collected on 3 trees 
        \begin{itemize}
        \item resistant 
        \item intermediate
        \item susceptible       
      \end{itemize}
      to oak powdery mildew
      \end{itemize}

    \end{tabular}
    &
    \begin{tabular}{c}
    \includegraphics[height=.4\textheight]{\fignet/CMR18-AnnApplStat-Fig5a} \\
    ~ \\
    \includegraphics[height=.4\textheight]{\fignet/CMR18-AnnApplStat-Fig4a} 
    \end{tabular}
  \end{tabular}
}

%====================================================================
\subsection{Network inference}
%====================================================================
\frame{\frametitle{Network inference} \pause

  \paragraph{A reminder on Gaussian graphical models (GGM).}  
  \begin{itemize}
  \item $\Sigma$ include both 'direct' and 'indirect' interactions:
  $$
  \sigma_{jk} = 0 
  \qquad \Leftrightarrow \qquad
  (Z_{ij}, Z_{ik}) \text{ independent}
  $$ \\ ~ \pause
  \item $\Omega = \Sigma^{-1}$ only includes 'direct' interactions:
  $$
  \omega_{jk} = 0 
  \qquad \Leftrightarrow \qquad
  (Z_{ij}, Z_{ik}) \text{ independent conditionally on } (Z_{ih})_{h \neq j, k}
  $$ \\ ~ \pause
  \item Common assumption: few species are in direct interaction 
  $$
  \Rightarrow \quad \Omega \text{ should be \emphase{sparse} \qquad (many 0's)}
  $$ \\ ~ \pause
  \item Sparsity-inducing penalty (graphical lasso)
  $$
  \max_{\Omega} \; \log p(Z; \Omega) - \lambda \underset{\ell_1 \text{ penalty}}{\underbrace{\sum_{j \neq k} |\omega_{jk}|}}
  $$
  \end{itemize}

}

%====================================================================
\frame{\frametitle{Network inference: (PLNnetwork)} 

  \paragraph{PLN-network.} PLN model with graphical lasso penalty \refer{CMR19}
  $$
  \arg\max_{\beta, \Omega, q \in \Qcal} \; 
  J(\beta, \Omega, q)
  - \underset{{\text{$\ell_1$ penalty}}}{\underbrace{\lambda \sum_{j \neq k} |\omega_{jk}|}}
  $$

  \pause \bigskip \bigskip  
  \paragraph{Inferring the {\sl latent} dependency structure}, not the abundance one
  $$
  \begin{array}{c|c}
  \text{good case} & \text{bad case} \\
  \hline
  \includegraphics[width=.4\textwidth]{\fignet/CMR18b-ArXiv-Fig1b}
  &
  \includegraphics[width=.4\textwidth]{\fignet/CMR18b-ArXiv-Fig1a}
  \end{array}
  $$

}

%====================================================================
\frame{\frametitle{Barents' fish species} 
  
  \vspace{-.05\textheight}
  \begin{tabular}{cc}
    \hspace{-.04\textwidth}
    \begin{tabular}{p{.22\textwidth}}
      \paragraph{Data:} \\ ~
      \begin{itemize}
      \item $n=89$ sites \\~
      \item $p=30$ species \\~
      \item $d=4$ covariates
        \begin{itemize}
        \item latitude
        \item longitude
        \item temperature
        \item depth
        \end{itemize}
      \end{itemize}
    \end{tabular}
    &
    \begin{tabular}{c}
      \includegraphics[height=.85\textheight]{\fignet/CMR18b-ArXiv-Fig5}
    \end{tabular}
  \end{tabular}
  }

%====================================================================
\frame{\frametitle{Barents' fish species: choosing $\lambda$}

  $$
  \includegraphics[height=.7\textheight]{\fignet/BarentsFish_Gfull_criteria}
  $$
}

%====================================================================
%====================================================================
\section{Conclusion}
%====================================================================
\frame{\frametitle{Conclusion} 

  \paragraph{Poisson log-normal model.} \refer{CMR21}
  \begin{itemize}
  \item Generic modeling framework 
  \item Other extension of PLN: sample comparison, species clustering, ...
  \item Account for the sampling effort via offset ({\sl relative abundances})
  \item R package \emphase{\url{PLNmodels}} 
  \item See also tree-based network inference \refer{MRA20}, including {\sl missing actors} \refer{MRA20b}
  \end{itemize}

  \pause \bigskip \bigskip 
  \paragraph{On-going modeling extensions:}
  \begin{itemize}
  \item Include species traits
  \item Include spatial dependency between sites
  \end{itemize}

  \pause \bigskip \bigskip 
  \paragraph{On-going inference developments:}
  \begin{itemize}
  \item Approximate confidence/credibility intervals
  \item VEM = starting point for frequentist inference (importance sampling / composite likelihood)
  \item Accelerated gradient descent for high-dimensional data
  \end{itemize}

}
  
%====================================================================
%====================================================================
\backupbegin 
\section*{Backup}
%====================================================================
\frame[allowframebreaks]{ \frametitle{References}
  {%\footnotesize
   \footnotesize
   \bibliography{/home/robin/Biblio/BibGene}
%    \bibliographystyle{/home/robin/LATEX/Biblio/astats}
   \bibliographystyle{alpha}
  }
}

%====================================================================
\backupend 

%====================================================================
%====================================================================
\end{document}
%====================================================================
%====================================================================

  \begin{tabular}{cc}
    \hspace{-.04\textwidth}
    \begin{tabular}{p{.5\textwidth}}
    \end{tabular}
    &
    \begin{tabular}{p{.45\textwidth}}
    \end{tabular}
  \end{tabular}

