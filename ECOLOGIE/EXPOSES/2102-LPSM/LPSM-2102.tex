\documentclass[8pt]{beamer}

% Beamer style
%\usetheme[secheader]{Madrid}
% \usetheme{CambridgeUS}
\useoutertheme{infolines}
\usecolortheme[rgb={0.65,0.15,0.25}]{structure}
% \usefonttheme[onlymath]{serif}
\beamertemplatenavigationsymbolsempty
%\AtBeginSubsection

% Packages
%\usepackage[french]{babel}
\usepackage[latin1]{inputenc}
\usepackage{color}
\usepackage{xspace}
\usepackage{dsfont, stmaryrd}
\usepackage{amsmath, amsfonts, amssymb, stmaryrd}
\usepackage{epsfig}
\usepackage{tikz}
\usepackage{url}
% \usepackage{ulem}
\usepackage{/home/robin/LATEX/Biblio/astats}
%\usepackage[all]{xy}
\usepackage{graphicx}
\usepackage{xspace}

% Maths
% \newtheorem{theorem}{Theorem}
% \newtheorem{definition}{Definition}
\newtheorem{proposition}{Proposition}
% \newtheorem{assumption}{Assumption}
% \newtheorem{algorithm}{Algorithm}
% \newtheorem{lemma}{Lemma}
% \newtheorem{remark}{Remark}
% \newtheorem{exercise}{Exercise}
% \newcommand{\propname}{Prop.}
% \newcommand{\proof}{\noindent{\sl Proof:}\quad}
% \newcommand{\eproof}{$\blacksquare$}

% \setcounter{secnumdepth}{3}
% \setcounter{tocdepth}{3}
\newcommand{\pref}[1]{\ref{#1} p.\pageref{#1}}
\newcommand{\qref}[1]{\eqref{#1} p.\pageref{#1}}

% Colors : http://latexcolor.com/
\definecolor{darkred}{rgb}{0.65,0.15,0.25}
\definecolor{darkgreen}{rgb}{0,0.4,0}
\definecolor{darkred}{rgb}{0.65,0.15,0.25}
\definecolor{amethyst}{rgb}{0.6, 0.4, 0.8}
\definecolor{asparagus}{rgb}{0.53, 0.66, 0.42}
\definecolor{applegreen}{rgb}{0.55, 0.71, 0.0}
\definecolor{awesome}{rgb}{1.0, 0.13, 0.32}
\definecolor{blue-green}{rgb}{0.0, 0.87, 0.87}
\definecolor{red-ggplot}{rgb}{0.52, 0.25, 0.23}
\definecolor{green-ggplot}{rgb}{0.42, 0.58, 0.00}
\definecolor{purple-ggplot}{rgb}{0.34, 0.21, 0.44}
\definecolor{blue-ggplot}{rgb}{0.00, 0.49, 0.51}

% Commands
\newcommand{\backupbegin}{
   \newcounter{finalframe}
   \setcounter{finalframe}{\value{framenumber}}
}
\newcommand{\backupend}{
   \setcounter{framenumber}{\value{finalframe}}
}
\newcommand{\emphase}[1]{\textcolor{darkred}{#1}}
\newcommand{\comment}[1]{\textcolor{gray}{#1}}
\newcommand{\paragraph}[1]{\textcolor{darkred}{#1}}
\newcommand{\refer}[1]{{\small{\textcolor{gray}{{[\cite{#1}]}}}}}
\newcommand{\Refer}[1]{{\small{\textcolor{gray}{{[#1]}}}}}
\newcommand{\goto}[1]{{\small{\textcolor{blue}{[\#\ref{#1}]}}}}
\renewcommand{\newblock}{}

\newcommand{\tabequation}[1]{{\medskip \centerline{#1} \medskip}}
% \renewcommand{\binom}[2]{{\left(\begin{array}{c} #1 \\ #2 \end{array}\right)}}

% Variables 
\newcommand{\Abf}{{\bf A}}
\newcommand{\Beta}{\text{B}}
\newcommand{\Bcal}{\mathcal{B}}
\newcommand{\Bias}{\xspace\mathbb B}
\newcommand{\Cor}{{\mathbb C}\text{or}}
\newcommand{\Cov}{{\mathbb C}\text{ov}}
\newcommand{\cl}{\text{\it c}\ell}
\newcommand{\Ccal}{\mathcal{C}}
\newcommand{\cst}{\text{cst}}
\newcommand{\Dcal}{\mathcal{D}}
\newcommand{\Ecal}{\mathcal{E}}
\newcommand{\Esp}{\xspace\mathbb E}
\newcommand{\Espt}{\widetilde{\Esp}}
\newcommand{\Covt}{\widetilde{\Cov}}
\newcommand{\Ibb}{\mathbb I}
\newcommand{\Fcal}{\mathcal{F}}
\newcommand{\Gcal}{\mathcal{G}}
\newcommand{\Gam}{\mathcal{G}\text{am}}
\newcommand{\Hcal}{\mathcal{H}}
\newcommand{\Jcal}{\mathcal{J}}
\newcommand{\Lcal}{\mathcal{L}}
\newcommand{\Mt}{\widetilde{M}}
\newcommand{\mt}{\widetilde{m}}
\newcommand{\Nbb}{\mathbb{N}}
\newcommand{\Mcal}{\mathcal{M}}
\newcommand{\Ncal}{\mathcal{N}}
\newcommand{\Ocal}{\mathcal{O}}
\newcommand{\pt}{\widetilde{p}}
\newcommand{\Pt}{\widetilde{P}}
\newcommand{\Pbb}{\mathbb{P}}
\newcommand{\Pcal}{\mathcal{P}}
\newcommand{\Qcal}{\mathcal{Q}}
\newcommand{\qt}{\widetilde{q}}
\newcommand{\Rbb}{\mathbb{R}}
\newcommand{\Sbb}{\mathbb{S}}
\newcommand{\Scal}{\mathcal{S}}
\newcommand{\st}{\widetilde{s}}
\newcommand{\St}{\widetilde{S}}
\newcommand{\Tcal}{\mathcal{T}}
\newcommand{\todo}{\textcolor{red}{TO DO}}
\newcommand{\Ucal}{\mathcal{U}}
\newcommand{\Un}{\math{1}}
\newcommand{\Vcal}{\mathcal{V}}
\newcommand{\Var}{\mathbb V}
\newcommand{\Vart}{\widetilde{\Var}}
\newcommand{\Zcal}{\mathcal{Z}}

% Symboles & notations
\newcommand\independent{\protect\mathpalette{\protect\independenT}{\perp}}\def\independenT#1#2{\mathrel{\rlap{$#1#2$}\mkern2mu{#1#2}}} 
\renewcommand{\d}{\text{\xspace d}}
\newcommand{\gv}{\mid}
\newcommand{\ggv}{\, \| \, }
% \newcommand{\diag}{\text{diag}}
\newcommand{\card}[1]{\text{card}\left(#1\right)}
\newcommand{\trace}[1]{\text{tr}\left(#1\right)}
\newcommand{\matr}[1]{\boldsymbol{#1}}
\newcommand{\matrbf}[1]{\mathbf{#1}}
\newcommand{\vect}[1]{\matr{#1}} %% un peu inutile
\newcommand{\vectbf}[1]{\matrbf{#1}} %% un peu inutile
\newcommand{\trans}{\intercal}
\newcommand{\transpose}[1]{\matr{#1}^\trans}
\newcommand{\crossprod}[2]{\transpose{#1} \matr{#2}}
\newcommand{\tcrossprod}[2]{\matr{#1} \transpose{#2}}
\newcommand{\matprod}[2]{\matr{#1} \matr{#2}}
\DeclareMathOperator*{\argmin}{arg\,min}
\DeclareMathOperator*{\argmax}{arg\,max}
\DeclareMathOperator{\sign}{sign}
\DeclareMathOperator{\tr}{tr}
\newcommand{\ra}{\emphase{$\rightarrow$} \xspace}

% Hadamard, Kronecker and vec operators
\DeclareMathOperator{\Diag}{Diag} % matrix diagonal
\DeclareMathOperator{\diag}{diag} % vector diagonal
\DeclareMathOperator{\mtov}{vec} % matrix to vector
\newcommand{\kro}{\otimes} % Kronecker product
\newcommand{\had}{\odot}   % Hadamard product

% TikZ
\newcommand{\nodesize}{2em}
\newcommand{\edgeunit}{2.5*\nodesize}
\newcommand{\edgewidth}{1pt}
\tikzstyle{node}=[draw, circle, fill=black, minimum width=.75\nodesize, inner sep=0]
\tikzstyle{square}=[rectangle, draw]
\tikzstyle{param}=[draw, rectangle, fill=gray!50, minimum width=\nodesize, minimum height=\nodesize, inner sep=0]
\tikzstyle{hidden}=[draw, circle, fill=gray!50, minimum width=\nodesize, inner sep=0]
\tikzstyle{hiddenred}=[draw, circle, color=red, fill=gray!50, minimum width=\nodesize, inner sep=0]
\tikzstyle{observed}=[draw, circle, minimum width=\nodesize, inner sep=0]
\tikzstyle{observedred}=[draw, circle, minimum width=\nodesize, color=red, inner sep=0]
\tikzstyle{eliminated}=[draw, circle, minimum width=\nodesize, color=gray!50, inner sep=0]
\tikzstyle{empty}=[draw, circle, minimum width=\nodesize, color=white, inner sep=0]
\tikzstyle{blank}=[color=white]
\tikzstyle{nocircle}=[minimum width=\nodesize, inner sep=0]

\tikzstyle{edge}=[-, line width=\edgewidth]
\tikzstyle{edgebendleft}=[-, >=latex, line width=\edgewidth, bend left]
\tikzstyle{edgebendright}=[-, >=latex, line width=\edgewidth, bend right]
\tikzstyle{lightedge}=[-, line width=\edgewidth, color=gray!50]
\tikzstyle{lightedgebendleft}=[-, >=latex, line width=\edgewidth, bend left, color=gray!50]
\tikzstyle{lightedgebendright}=[-, >=latex, line width=\edgewidth, bend right, color=gray!50]
\tikzstyle{edgered}=[-, line width=\edgewidth, color=red]
\tikzstyle{edgebendleftred}=[-, >=latex, line width=\edgewidth, bend left, color=red]
\tikzstyle{edgebendrightred}=[-, >=latex, line width=\edgewidth, bend right, color=red]

\tikzstyle{arrow}=[->, >=latex, line width=\edgewidth]
\tikzstyle{arrowbendleft}=[->, >=latex, line width=\edgewidth, bend left]
\tikzstyle{arrowbendright}=[->, >=latex, line width=\edgewidth, bend right]
\tikzstyle{arrowred}=[->, >=latex, line width=\edgewidth, color=red]
\tikzstyle{arrowbendleftred}=[->, >=latex, line width=\edgewidth, bend left, color=red]
\tikzstyle{arrowbendrightred}=[->, >=latex, line width=\edgewidth, bend right, color=red]
\tikzstyle{arrowblue}=[->, >=latex, line width=\edgewidth, color=blue]
\tikzstyle{dashedarrow}=[->, >=latex, dashed, line width=\edgewidth]
\tikzstyle{dashededge}=[-, >=latex, dashed, line width=\edgewidth]
\tikzstyle{dashededgebendleft}=[-, >=latex, dashed, line width=\edgewidth, bend left]
\tikzstyle{lightarrow}=[->, >=latex, line width=\edgewidth, color=gray!50]

\newcommand{\GMSBM}{/home/robin/RECHERCHE/RESEAUX/EXPOSES/1903-SemStat/}
\newcommand{\figeconet}{/home/robin/RECHERCHE/ECOLOGIE/EXPOSES/1904-EcoNet-Lyon/Figs}
\newcommand{\fignet}{/home/robin/RECHERCHE/RESEAUX/EXPOSES/FIGURES}
\newcommand{\figeco}{/home/robin/RECHERCHE/ECOLOGIE/EXPOSES/FIGURES}
\newcommand{\figbayes}{/home/robin/RECHERCHE/BAYES/EXPOSES/FIGURES}
\newcommand{\figCMR}{/home/robin/Bureau/RECHERCHE/ECOLOGIE/CountPCA/sparsepca/Article/Network_JCGS/trunk/figs}
\newcommand{\figtree}{/home/robin/RECHERCHE/BAYES/VBEM-IS/VBEM-IS.git/Data/Tree/Fig}
\newcommand{\figDoR}{/home/robin/RECHERCHE/BAYES/VBEM-IS/VBEM-IS.git/Paper/JRSSC-V3/Figs}

\renewcommand{\nodesize}{1.75em}
\renewcommand{\edgeunit}{2.25*\nodesize}

%====================================================================
%====================================================================
\begin{document}
%====================================================================
%====================================================================

\title{Models with latent variables for community ecology}

\author[S. Robin]{S. Robin \\ ~\\
  {\small INRAE / AgroParisTech / univ. Paris-Saclay / MNHN} \\ ~\\
  joint work with J. Chiquet, S. Donnet, M. Mariadassou, \dots}

\date[Feb'21, LPSM, Paris]{February 2021\\
Laboratoire de Probabilit\'es, Statistique et Mod\'elisation, Paris}

\maketitle

%====================================================================
\frame{\frametitle{Community ecology}

{\sl A community is a group [\dots] of populations of [\dots] different species occupying the same geographical area at the same time }

\medskip
{\sl Community ecology [\dots] is the study of the interactions between species in communities [\dots].} \Refer{Wikipedia}

\bigskip \bigskip
\paragraph{Need for statistical models to}
\begin{itemize}
 \item \bigskip decipher / describe / evaluate environmental effects on species ({\sl abiotic} interactions) and between species interactions ({\sl biotic} interactions) \\ ~\\
 \ra \emphase{joint species distribution models}
 \item \bigskip describe / understand the organisation of species interaction networks \\ ~\\
  \ra \emphase{network models}
\end{itemize}

}

%====================================================================
\frame{\frametitle{Outline} \tableofcontents}

%====================================================================
%====================================================================
\section{Two models with latent variables}
% \frame{\frametitle{Outline} \tableofcontents[currentsection]}
%====================================================================
\subsection[Joint species distribution models]{Joint species distribution models: Poisson log-normal}
\frame{\frametitle{Outline} \tableofcontents[currentsubsection]}
%====================================================================
\frame{\frametitle{Joint species distribution models (JDSM)} 

  \begin{tabular}{cc}
    \hspace{-.04\textwidth}
    \begin{tabular}{p{.5\textwidth}}
      \paragraph{Fish species in Barents sea \refer{FNA06}:} 
      \begin{itemize}
       \item $89$ sites (stations), 
       \item $30$ fish species, 
       \item $4$ covariates 
      \end{itemize}

      \bigskip \bigskip 
      \paragraph{Questions:} 

      \begin{itemize}
       \item Do environmental conditions affect species abundances? (abiotic) \\~ 
       \item Do species abundances vary independently? (biotic)
      \end{itemize} 
      
      \bigskip
      See \refer{WBO15}
    \end{tabular}
    &
    \begin{tabular}{p{.45\textwidth}}
      \paragraph{Abundance table:} ~ \\
        {\footnotesize \begin{tabular}{rrrr}
        {\sl Hi.pl} & {\sl An.lu} & {\sl Me.ae} & \dots \\
%         \footnote{{\sl Hi.pl}: Long rough dab, {\sl An.lu}: Atlantic wolffish, {\sl Me.ae}: Haddock} \\ 
  %       Dab & Wolffish & Haddock \\ 
        \hline
        31  &   0  & 108 & \\
         4  &   0  & 110 & \\
        27  &   0  & 788 & \\
        13  &   0  & 295 & \\
        23  &   0  &  13 & \\
        20  &   0  &  97 & \\
        . & . & . & 
      \end{tabular}} 
      \\
      \bigskip \bigskip 
      \paragraph{Environmental covariates:} ~ \\
        {\footnotesize \begin{tabular}{rrrr}
        Lat. & Long. & Depth & Temp. \\
        \hline
        71.10 & 22.43 & 349 & 3.95 \\
        71.32 & 23.68 & 382 & 3.75 \\
        71.60 & 24.90 & 294 & 3.45 \\
        71.27 & 25.88 & 304 & 3.65 \\
        71.52 & 28.12 & 384 & 3.35 \\
        71.48 & 29.10 & 344 & 3.65 \\
        . & . & . & .
      \end{tabular}} 
    \end{tabular}
  \end{tabular}
  
}

%====================================================================
\frame{\frametitle{Poisson log-normal model (PLN)}
 
  \begin{tabular}{cc}
    \hspace{-.04\textwidth}
    \begin{tabular}{p{.5\textwidth}}
      \paragraph{Data:} 
      \begin{itemize}
       \item $n$ sites, $p$ species, $d$ covariates 
       \item Abundance table: ${Y} = [Y_{ij}]$
       \item Covariate table: ${X} = [x_{ik}]$
      \end{itemize}
 
      \bigskip \bigskip \bigskip 
      \paragraph{PLN model:} \refer{AiH89}
      \begin{itemize}
       \item \emphase{$Z_i =$ latent vector} associated with site $i$
       $$
       Z_i \sim \Ncal_p(0, {\Sigma})
       $$
       \item $Y_{ij} = $ observed abundance for species $j$ in site $i$
       $$
       Y_{ij} \mid Z_{ij} \sim \Pcal(\exp(x_i^\intercal {\beta_j} + Z_{ij}))
       $$
      \item ${\theta} = (\beta, \Sigma)$
      \end{itemize}
    \end{tabular}
    & 
    \begin{tabular}{p{.45\textwidth}}
      \paragraph{Covariance matrix $\widehat{\Sigma}$:} biotic \\ 
      \includegraphics[width=.3\textwidth]{\figeco/BarentsFish-corrAll} \\
      ~\\
      \paragraph{Regression cofficients $\widehat{\beta}$:} abiotic \\ 
      \includegraphics[width=.3\textwidth]{\figeco/BarentsFish-coeffAll}
    \end{tabular}    
  \end{tabular}
 
}

%====================================================================
\subsection[Models for species interaction networks]{Models for species interaction networks: Stochastic block model}
\frame{\frametitle{Outline} \tableofcontents[currentsubsection]}
%====================================================================
\frame{\frametitle{Models for species interaction networks} 

  \begin{tabular}{cc}
    \hspace{-.04\textwidth}
    \begin{tabular}{p{.5\textwidth}}
      \paragraph{Tree species interactions \refer{VPD08}:} 
      \begin{itemize}
       \item $n = 51$ tree species
       \item $Y_{ij} =$ number of fungal parasites shared by species $i$ and $j$
       \item $x_{ij} =$ vector of similarities (taxonomic, geographic, genetic) between species $i$ and $j$
      \end{itemize}
      
      \bigskip \bigskip
      \paragraph{Questions:} 
      \begin{itemize}
       \item Is the network 'organized' in some way? \\~
       \item Do the covariate contribute to explain why species interact?
      \end{itemize}

      
%       \bigskip \bigskip 
%       \paragraph{Other types of network.} 
%       \begin{itemize}
%       \item plant-pollinator: mutualistic network (bipartite) 
%       \item predator-prey: trophic network (multipartite)
%       \end{itemize}
    \end{tabular}
    &
    \begin{tabular}{p{.45\textwidth}}
      \paragraph{Network (weighted):} \\
%       \includegraphics[height=.25\textwidth,width=.25\textwidth,trim=30 30 30 30]     
      \includegraphics[height=.3\textwidth,width=.3\textwidth]{\fignet/Tree-netCircle}
 \\
      \paragraph{Adjacency matrix (counts):} \\
      \includegraphics[height=.3\textwidth,width=.3\textwidth]{\fignet/Tree-adjMat}
    \end{tabular}
  \end{tabular}
      
}

%====================================================================
\frame{\frametitle{Stochastic block models (SBM)}

  \begin{tabular}{cc}
    \hspace{-.04\textwidth}
    \begin{tabular}{p{.5\textwidth}}
      \paragraph{Data:} 
      \begin{itemize}
       \item $n$ species, $d$ covariates
       \item Adjacency matrix: ${Y} = [Y_{ij}]$
       \item Covariate matrices: ${X^1, \dots X^d}$
      \end{itemize}
      \bigskip \bigskip 
      \paragraph{Stochastic block model:} \refer{HoL79} $K$ groups, 
      \begin{itemize}
      \item \emphase{$Z_i =$ latent variable} indicating to which group node $i$ belongs to
      $$
      \Pr\{Z_i = k\} = {\pi_k}
      $$
      \item $Y_{ij} =$ observed number of interactions between species $i$ and $j$
      $$
      Y_{ij} \mid Z_i, Z_j \sim \Pcal(\exp(x_{ij}^\intercal {\beta} + \alpha_{Z_i Z_j}))
      $$
      \item ${\theta} = (\pi, \alpha, \beta)$ \qquad (and $K$)
      \end{itemize}
    \end{tabular}
    &
    \begin{tabular}{p{.45\textwidth}}
      \paragraph{Observed adjacency matrix:} \\
      \includegraphics[height=.3\textwidth,width=.3\textwidth]{\fignet/Tree-adjMat} \\
      
      \paragraph{Clustered matrix:} (no covariate) \\
      \includegraphics[height=.3\textwidth,width=.3\textwidth]{\fignet/Tree-adjMat-SBMnull}
    \end{tabular}
  \end{tabular}

}

%====================================================================
%====================================================================
\section{Variational inference of incomplete data models}
\frame{\frametitle{Outline} \tableofcontents[currentsection]}
%====================================================================
\subsection*{A reminder on the EM algorithm}
%====================================================================
\frame{\frametitle{A reminder on the EM algorithm} 

  \paragraph{Maximum likelihood inference.}
  $$
  \widehat{\theta} = \arg\max_\theta \; \log p(Y; \theta)
  $$
  \ra no closed form for ${p(Y; \theta) = \int \underset{\text{complete likelihood}}{\underbrace{p(Y, Z; \theta)}} \d Z}$ in latent variable models

  \bigskip \bigskip
  \paragraph{Incomplete data models.} EM algorithm \refer{DLR77}
  $$
  \theta^{h+1} 
  = \underset{\text{\normalsize \emphase{M step}}}{\underbrace{\argmax_\theta}} \;  \underset{\text{\normalsize \emphase{E step}}}{\underbrace{\Esp_{\theta^h}}}[\log p_\theta(Y, Z) \mid Y]
  $$
%   $$
%   \log p(Y; \theta) 
%   = \underset{\text{\normalsize \emphase{M step}}}{\underbrace{\Esp(\log p(Y, Z; \theta) \mid Y)}} 
%   - \Esp(\log \underset{\text{\normalsize \emphase{E step}}}{\underbrace{p(Z \mid Y; \theta)}} \mid Y)
%   $$
  
%   \bigskip
%   \paragraph{Expectation-maximisation algorithm.}
%   \begin{description}
%    \item[E step:] Evaluate (some moments of) $p(Z \mid Y)$
%    \item[M step:] Maximize $\Esp(\log p(Y, Z; \theta) \mid Y )$ with respect to $\theta$
%   \end{description}

  \bigskip \bigskip
  \paragraph{Critical step = E step.} Evaluate $p(Z \mid Y; \theta)$
  \begin{itemize}
  \item easy: mixture models, Gaussian linear mixed-models, ...
  \item use a trick: hidden Markov models, phylogenetic or evolutionary models, ...
  \item intractable: many models, including PLN \& SBM
  \end{itemize}
}

%====================================================================
\subsection*{Variational EM algorithms}
%====================================================================
\frame{\frametitle{Variational approximation} 

  \paragraph{Twisted problem:} \refer{Jaa01,WaJ08,BKM17} maximize the 'evidence lower bound' (ELBO)
  \begin{align*}
  J(Y; \theta, q)
  & = \log p(Y; \theta) 
  - \underset{\text{\normalsize {VE step}}}{\underbrace{\emphase{KL(q(Z) \mid\mid p(Z \mid Y))}}} \\
  & = 
  \underset{\text{\normalsize {M step}}}{\underbrace{\emphase{\Esp_q (\log p(Y, Z; \theta))}}} 
  - \Esp_q(\log q(Z)) 
  \end{align*}
  where $q(Z) \simeq p(Z \mid Y)$ is chosen within a \emphase{class of distributions $\Qcal$}
  
  \bigskip \bigskip \pause
  \paragraph{Many avatars.} 
  \begin{itemize}
    \item Alternative divergences: $\alpha$-divergences \refer{Min05}, $KL(p(Z \mid Y) \mid\mid q(Z))$\footnote{$KL(p \mid\mid q) = \Esp_p \log(p/q)$} (expectation-propagation = EP \refer{Min01}) 
    \item Bayesian inference (variational Bayes = VB): look for $q(\theta) \simeq p(\theta \mid Y)$ 
    \item Bayesian inference for incomplete data model (VBEM): $q(\theta, Z) \simeq p(\theta, Z \mid Y)$ \refer{BeG03} 
  \end{itemize} 
  
  \bigskip
  \ra Reasonably easy to implement and computationally efficient
  
}

% %====================================================================
% \frame{\frametitle{Approximate distribution} 
% 
%   \paragraph{Critical choice:} $\Qcal$ needs to be \\ ~
%   \begin{itemize}
%   \item 'large' enough to include good approximations of $p(Z \mid Y)$ \\ ~
%   \item 'small' enough to make the calculations tractable
%   \end{itemize}
%   
%   \bigskip \bigskip
%   \paragraph{Examples:} 
%   \begin{itemize}
%    \item PLN model for species abundance: $$\Qcal := \{\text{multivariate Gaussian distributions}\}$$ \\~
%    \item SBM model for networks: $$\Qcal := \{\text{factorable discrete distributions}\}$$ \\
%    ('mean-field' approximation)
%   \end{itemize}
% 
% }

%====================================================================
%====================================================================
\section{Latent variable models for community ecology}
% \frame{\frametitle{Outline} \tableofcontents[currentsection]}
%====================================================================
\subsection{Poisson log-normal model}
\frame{\frametitle{Outline} \tableofcontents[currentsubsection]}
%====================================================================
\frame{\frametitle{Poisson log-normal model} 

  \paragraph{Join species distribution model.}
  \begin{itemize}
  \item $n$ independent sites, $p$ species
  \item $Z_i=$ latent vector for site $i$
  $$
  Z_i \sim \Ncal_p(0, \Sigma)
  $$
  \item $x_i =$ vector of covariates for site $i$
  \item $Y_{ij} =$ abundance of species $j$ in site $i$:
  $$
  Y_{ij} \sim \Pcal(\exp(x_i^\intercal \beta_j + Z_{ij}))
  $$
  \end{itemize}

  \bigskip
  \paragraph{Interpretation:}
  \begin{align*}
  \Sigma & = \text{biotic interactions} &
  \beta & = \text{abiotic effects}
  \end{align*}
 
  \bigskip \bigskip
  \paragraph{Desirable properties:}
  \begin{align*}
    \Var(Y_{ij}) & > \Var(\Pcal(\exp(x_i^\intercal \beta_j)), &
    \sign(\sigma_{jk}) & = \sign(\Cor(Y_{ij}, Y_{ik})) 
  \end{align*}
  
}

%====================================================================
\frame{\frametitle{VEM inference} 

  \paragraph{Variational approximation.} $\Qcal = \{\text{multivariate Gaussian}\}$
  $$
  p(Z_i \mid Y_i) \simeq q_i(Z_i) := \Ncal(Z_i; {m_i, S_i}), 
  $$
  $(m_i, S_i)_{1 \leq i \leq n}$ = {\sl variational parameters}.
  
  \bigskip \bigskip 
  \paragraph{VE step:} minimize wrt $(m_i, S_i)_{1 \leq i \leq n}$
  $$
  KL(q(Z) \mid\mid p(Z \mid Y; \theta)) = KL(q(Z) \mid\mid p(Z, Y; \theta)) + \text{cst}
  $$
  \ra convex problem \refer{CMR18a}
  
  \bigskip \bigskip
  \paragraph{M step:} maximize wrt $\theta = (\beta, \Sigma)$
  $$
  \Esp_q (\log p(Y, Z; \theta))
  $$
  \ra convex problem (weighted GLM)
}

%====================================================================
\frame{\frametitle{Abiotic vs biotic interactions} 

%   \paragraph{Variational EM \refer{CMR18a,CMR19}:} $p(Z_i \mid Y_i) \simeq \Ncal(Z_i; m_i, S_i)$
  
%   \bigskip 
  \begin{tabular}{cc|c}
    \multicolumn{2}{l|}{\emphase{Full model}} &
    \multicolumn{1}{l}{\emphase{Null model}} \\
    & & \\
    \multicolumn{2}{c|}{{$Y_{ij} \sim \Pcal(\exp(\emphase{x_i^\intercal \beta_j} + Z_{ij}))$}} &
    \multicolumn{1}{c}{{$Y_{ij} \sim \Pcal(\exp(\emphase{\mu_j} + Z_{ij}))$}} \\
    & & \\
    \multicolumn{2}{l|}{{$x_i =$ all covariates}} &
    \multicolumn{1}{l}{{no covariate}} \\ 
    & & \\
    & correlations between & \\
    inferred  correlations $\widehat{\Sigma}_{\text{full}}$ & 
    predictions: $x_i^\intercal \widehat{\beta}_j$ & 
    inferred correlations $\widehat{\Sigma}_{\text{null}}$ \\ 
    \includegraphics[width=.3\textwidth, trim=20 20 20 20]{\figeco/BarentsFish-corrAll} 
    &
    \includegraphics[width=.3\textwidth, trim=20 20 20 20]{\figeco/BarentsFish-corrPred} &
    \includegraphics[width=.3\textwidth, trim=20 20 20 20]{\figeco/BarentsFish-corrNull}
  \end{tabular}

}

%====================================================================
\frame{\frametitle{Ease of modeling with latent variables} \label{back:plnExtensions}

  \paragraph{Latent layer = multivariate Gaussian:} flexible dependency structure (\url{PLNmodels} package) \refer{CMR20}
  
  \bigskip \bigskip
  \paragraph{Dimension reduction.} \Refer{\#\ref{goto:plnPCA}}
  \begin{itemize}
  \item Metagenomic, metabarcoding, environmental genomics: $p = 10^2, 10^3$ species
  \item Probabilistic PCA \refer{Tib99}: suppose that \emphase{$\Sigma$ has rank $r \ll p$}
  \item PLN-PCA: PLN model with rank constraint on $\Sigma$ \refer{CMR18a}
  \end{itemize}
  
  \bigskip \bigskip
  \paragraph{Network inference.} \Refer{\#\ref{goto:plnNetwork}}
  \begin{itemize}
  \item $\Sigma$ include both direct and indirect interactions
  \item Gaussian graphical models: \emphase{$\Sigma^{-1}$ should be sparse}
  \item PLN-network: PLN model with graphical lasso penalty \refer{CMR19}
  \end{itemize}
  $$
  \arg\max_{\beta, \Sigma, q \in \Qcal} \; 
  \underset{{\text{\normalsize log-likelihood}}}{\underbrace{\log p(Y; \beta, \Sigma)}}
  - \underset{{\text{\normalsize variational approx.}}}{\underbrace{KL(q(Z) \mid\mid p(Z \mid Y))}}
  - \underset{{\text{\normalsize $\ell_1$ penalty}}}{\underbrace{\emphase{\lambda \|\Sigma^{-1}\|_1}}}
  $$
  
}

%====================================================================
\subsection{Stochastic block model}
\frame{\frametitle{Outline} \tableofcontents[currentsubsection]}
%====================================================================
\frame{\frametitle{Stochastic block models}

  \paragraph{Network model.}
  \begin{itemize}  
  \item $n$ species, $d$ covariates, $K$ groups
  \item $Z_i=$ latent group of species $i$
  $$
  \Pr\{Z_i = k\} = \pi_k
  $$
  \item $x_{ij} =$ vector of covariates for species $(i, j)$
  \item $Y_{ij} =$ observed nb of interactions $(i, j)$ % $i$ and $j$
  $$
  Y_{ij} \sim \Pcal(\exp(x_{ij}^\intercal {\beta} + \alpha_{Z_i Z_j}))
  $$
  \end{itemize}

  \bigskip \bigskip 
  \paragraph{Interpretation:} 
  \begin{itemize}
  \item $\beta =$ effects of the covariates on the network structure
  \item $\pi =$ group proportions
  \item $\alpha =$ residual structure encoded by the groups
  \end{itemize} 
  
}

%====================================================================
\frame{\frametitle{VEM inference} 

  \paragraph{Variational approximation.} $\Qcal = \{\text{factorable distribution}\}$
  $$
  p(Z \mid Y) \simeq q(Z) := \prod_i \Mcal(Z_i; 1, \tau_i)
  $$
  $(\tau_i)_{1 \leq i \leq n} =$ variational parameters
  
  \bigskip \bigskip
  \paragraph{VE step:} minimize wrt $(\tau_i)_{1 \leq i \leq n}$
  $$
  KL(q(Z) \mid\mid p(Z \mid Y; \theta)) = KL(q(Z) \mid\mid p(Z, Y; \theta)) + \text{cst}
  $$
  \ra fix-point problem \refer{DPR08} (mean-field approximation \refer{WaJ08})
   
  \bigskip \bigskip
  \paragraph{M step:} maximize wrt $\theta = (\pi, \alpha, \beta)$
  $$
  \Esp_q (\log p(Y, Z; \theta))
  $$
  \ra closed form + weighted GLM
}

%====================================================================
\frame{\frametitle{Interest of the covariates} \label{back:treeCovariates}

  \begin{tabular}{cc}
    \hspace{-.04\textwidth}
    \begin{tabular}{p{.5\textwidth}}
      \paragraph{Accounting for taxonomy.} \refer{MRV10}
      \begin{itemize}
        \item $x_{ij} =$ taxonomic distance btw species $i$ and $j$ 
        \item {$\widehat{\beta} = -.417$}
        \item $\widehat{K} = 4$
      \end{itemize}

      \bigskip
      \paragraph{Conclusion:} 
      \begin{itemize}
        \item Makes sense! \Refer{\#\ref{goto:treeInterpretation}}
      \end{itemize}

      \bigskip
      \paragraph{Questions:} 
      \begin{itemize}
        \item Significance of $\widehat{\beta}$?
        \item Model selection to choose $K$?
      \end{itemize}
    \end{tabular}
    &
    \begin{tabular}{p{.45\textwidth}}
%       \includegraphics[height=.3\textwidth,width=.3\textwidth]{\fignet/Tree-ICL-SBMall}  \\
      \includegraphics[height=.35\textwidth,width=.35\textwidth]{\fignet/Tree-adjMat-SBMtaxo}
    \end{tabular}
  \end{tabular}
  
}

%====================================================================
%====================================================================
\section{Beyond variational inference}
\frame{\frametitle{Outline} \tableofcontents[currentsection]}

%====================================================================
\frame{\frametitle{Statistical guarantees for variational estimates} \label{back:statGuarantees} 

  \paragraph{Statistical guaranties for variational inference?} 
  \begin{itemize}
  \item Very scarce \\
  \ra binary SBM without covariate \refer{CDP12,BCC13,MaM15}, specific PLN \refer{HOW11,WeM19}
  \item Many (positive) empirical results \Refer{\#\ref{goto:vbemSBM}}
  \item No generic theoretical result about consistency, asymptotic normality, \dots
  \end{itemize}
  
  \bigskip \bigskip 
  \paragraph{Can we use VEM to make statistically grounded inference?}
  \begin{itemize}
  \item Make a theoretical analysis of
  $$
  \widehat{\theta}_{VEM} = \argmax_\theta \left(\argmax_{q \in \Qcal} J(Y; \theta, q) \right)
  $$
  \ra Very problem specific \refer{WeM19}
  \item Use VEM to accelerate computationally demanding procedures for frequentist or Bayesian inference
  \end{itemize}

}

%====================================================================
\frame{\frametitle{Frequentist inference.} 
  
  \paragraph{Composite log-likelihood.} \refer{VRF11} \\ ~
  \begin{itemize}
  \item $p_\theta(Y) = \int p_\theta(Y, Z) \d Z$ is intractable because of the large dimensional integral \\ ~  \medskip
  \item $\cl_\theta(Y) =$ sum of log-likelihoods of blocks of data 
  \begin{align*}
  \cl_\theta(Y) & = \sum_{C \in \Ccal} w_C \log p_\theta(Y_C) & 
  \widehat{\theta}_{CL} & = \argmax_\theta \cl_\theta(Y) 
  \end{align*}
  where $\log p_\theta(Y_C) = \int \log p_\theta(Y_C, Z_C) \d Z_C$ may become tractable \\ ~   \medskip
  \item $\widehat{\theta}_{CL}$ consistent and asymptotic normal with \refer{vdV98}
  $$
  \Var_\infty(\widehat{\theta}_{CL}) = 
  \Esp(\partial^2_{\theta^2} \cl_\theta)^{-1} \Var(\partial_{\theta} \cl_\theta) \Esp(\partial^2_{\theta^2} \cl_\theta)^{-1}
  $$ \\ ~   \medskip
  \item The EM algorithm still applies to composite likelihood \\ 
    But may be computationally demanding, depending on the number of blocks
  \end{itemize}

}

%====================================================================
\frame{\frametitle{VEM-aided frequentist inference.} 
  
  \paragraph{Stochastic block-model.}
  \begin{align*}
    \text{binary SBM:} & & Y_C & = (Y_{ij}, Y_{ik}, Y_{jk}) & |\Ccal| & \approx n^3 \qquad \text{\refer{AmM12}} \\
    ~ \\
    \text{non-binary SBM:} & & Y_C & = Y_{ij} & |\Ccal| & \approx n^2 \\ 
    ~ \\
    \text{SBM:} & & Y_C & = (Y_{ij})_{1 \leq j \leq n} & |\Ccal| & = n \\
  \end{align*}

  \bigskip \bigskip 
  \paragraph{Poisson log-normal.}
  \begin{align*}
    Y_C & = (Y_{ij}, Y_{ik})_{1 \leq i \leq n} & |\Ccal| & \approx p^2 \\
    ~ \\
    Y_C & = (Y_{ij_1}, \dots Y_{ij_q})_{1 \leq i \leq n}  & |\Ccal| & = \text{number of blocks in an incomplete block design}
  \end{align*}

}

%====================================================================
\frame{\frametitle{Bayesian inference.} 

  \paragraph{Bayesian inference.} 
  \begin{itemize}
  \item {Prior}: $p(\theta)$
  \item Latent variable: $p(Z \mid \theta)$
  \item Observed data: $p(Y \mid Z, \theta)$
  \item Aim: evaluate / sample from the \emphase{joint posterior} $p(\theta, Z \mid Y) \propto \pi(\theta) p(Z \mid \theta) p(Y \mid Z, \theta)$
  \end{itemize}
  
  \bigskip \pause
  \begin{tabular}{cc}
    \hspace{-.04\textwidth}
    \begin{tabular}{p{.5\textwidth}} 
      \paragraph{Sequential Monte-Carlo sampling.} \refer{DDJ06} 
      \begin{itemize}
        \item Construct a sequence of distributions
        $$
        \textcolor{red}{p_0}, \; p_1, \; \dots, \; p_{H-1}, \; \textcolor{blue}{p_H = p(\cdot \mid Y)}
        $$
        e.g. $p_h \propto p_0^{1 - \rho_h} p_H^{\rho_h}$ with
        $$
        0 = \rho_0 < \rho_1 \ < \dots < \rho_H = 1
        $$
        \item $H = 1$: importance sampling
        \item Usually $p_0 =$ prior $\pi$
      \end{itemize}
    \end{tabular}
    &
    \begin{tabular}{p{.45\textwidth}}
      \includegraphics[width=.35\textwidth]{\figbayes/FigVBEM-IS-Tempering}
    \end{tabular}
  \end{tabular}

}

%====================================================================
\frame{\frametitle{VEM to initiate Sequential Monte-Carlo} 

  \begin{tabular}{cc}
    \hspace{-.04\textwidth}
    \begin{tabular}{p{.6\textwidth}}
      \paragraph{Proposed procedure.} \refer{DoR19} \\ ~
      \begin{enumerate}
      \item Run VEM to get $\widehat{\theta}_{VEM}$ and $\widehat{q}(Z)$ \\ ~
      \item Derive a proxy for $\Var(\widehat{\theta}_{VEM})$ using Louis formulas \refer{Lou82} and 
      $$
      \widehat{q}(Z) \simeq p_{\widehat{\theta}_{VEM}}(Z \mid Y)
      $$ \\ ~
      %     \begin{align*}
      %       \partial^2_{\theta^2} \log p(Y; \theta)
      %       & = \Esp\left(\partial^2_{\theta^2} \log p(Y, Z; \theta) \mid Y\right)
      %       + \Var\left(\partial_\theta \log p(Y, Z; \theta) \mid Y\right) 
      %       & & \text{\refer{Lou82}} \\
      %       & \simeq \Esp_{\emphase{\widehat{q}}}(\partial^2_{\theta^2} \log p(Y, Z; \theta))
      %       + {\Var_{\emphase{\widehat{q}}}(\partial_\theta \log p(Y, Z; \theta))} 
      %       \end{align*} \\ ~
      \item Combine $(\widehat{\theta}_{VEM}, \widetilde{\Var}(\widehat{\theta}_{VEM}), \widehat{q}(Z))$ with the prior $\pi(\theta)$ to build the proposal distribution $p_0(\theta, Z)$
      \end{enumerate}
    \end{tabular}
    &
    \hspace{-.075\textwidth}
    \begin{tabular}{p{.45\textwidth}}
      \includegraphics[width=.4\textwidth]{\figDoR/simu_traj_rho}
    \end{tabular}
  \end{tabular}


}

%====================================================================
\frame{\frametitle{Back to the tree interaction network} \label{back:treeNetwork}

  \bigskip 
  \paragraph{Do the distance between the tree species contribute to structure network?} 
  $$
  \begin{tabular}{ccc}
    taxonomy & geography & genetics \\
    \includegraphics[width=.25\textwidth]{\figtree/Tree-all-V10-M5000-beta1} & 
    \includegraphics[width=.25\textwidth]{\figtree/Tree-all-V10-M5000-beta2} & 
    \includegraphics[width=.25\textwidth]{\figtree/Tree-all-V10-M5000-beta3}
  \end{tabular}
  $$
  
  \bigskip 
  \hspace{-.025\textwidth}
  \begin{tabular}{rrrrl}
    \paragraph{Correlation between estimates.} 
    & $(\beta_1, \beta_2)$ & $(\beta_1, \beta_3)$ & $(\beta_2, \beta_3)$ & \\
    $p_{VEM}(\beta)$    & $-0.012$ & $ 0.021$ & $ 0.318$ & \\
    $p(\beta \mid Y)$ & $-0.274$ & $-0.079$ & $-0.088$ & \quad \Refer{SMC = 25 steps: \#\ref{goto:smcPath}}
  \end{tabular}

  \bigskip \bigskip 
  \paragraph{Model selection.} 
  $$
  P\{x = \text{(taxo., geo.)} \mid Y \} \simeq 52\%, \qquad
  P\{x = \text{(taxo.)} \mid Y \} \simeq 47\%
  $$

}

%====================================================================
%====================================================================
\section{Conclusion}
%====================================================================
\frame{\frametitle{Conclusion} 

  \paragraph{Models with latent variables.}
  \begin{itemize}
  \item A confortable framework for community ecology modeling
  \item Many avatars of PLN%: dimension reduction, network inference, sample comparison,   species clustering, offset, ...
  \item Many avatars of SBM%: bipartite networks (LBM), multi-partite networks, multi-layer networks, dynamic SBM...
  \end{itemize}

  \bigskip \bigskip 
  \paragraph{Variational approximations} allow to cope with complex dependency structures
  \begin{itemize}
  \item Poisson log-normal model for species abundances (\emphase{\url{PLNmodels}}) 
  \item Block models for species interaction networks (\emphase{\url{block models}}, \emphase{\url{sbm}})
  \end{itemize}

  \bigskip \bigskip 
  \paragraph{Several open questions about inference.}
  \begin{itemize}
  \item General results about variational estimates are still missing
  \item Alternative methods (composite likelihood, SEM, SAEM, Bayesian) come along with statistical guaranties, but are computationally demanding
  \item VEM may provide a very reliable starting point for most of them
  \end{itemize}

}
  
%====================================================================
%====================================================================
\backupbegin 
\section*{Backup}
%====================================================================
\frame[allowframebreaks]{ \frametitle{References}
  {%\footnotesize
   \tiny
   \bibliography{/home/robin/Biblio/BibGene}
%    \bibliographystyle{/home/robin/LATEX/Biblio/astats}
   \bibliographystyle{alpha}
  }
}

%====================================================================
\frame{\frametitle{PLN for dimension reduction (PCA)} \label{goto:plnPCA}

  \begin{tabular}{cl}
    \hspace{-.04\textwidth}
    \begin{tabular}{p{.3\textwidth}}
      \paragraph{Metabarcoding data} \refer{JFS16} \\ ~
      \begin{itemize}
      \item $p = 114$ OTUs (bacteria and fungi) \\ ~
      \item $n = 116$ leaves \\ ~
      \item collected on 3 trees 
        \begin{itemize}
        \item resistant 
        \item intermediate
        \item susceptible       
      \end{itemize}
      to oak powdery mildew
      \end{itemize}

      \bigskip \bigskip 
      \paragraph{Model selection:} 
      \begin{itemize}
        \item Penalized lower-bound $J(Y; \widehat{\theta}, \widehat{q})$ \\
        (pseudo BIC, pseudo ICL, \dots)
      \end{itemize}

    \end{tabular}
    &
    \begin{tabular}{p{.55\textwidth}}
    \includegraphics[height=.4\textheight]{\fignet/CMR18-AnnApplStat-Fig4a} \\
    ~ \\
    \includegraphics[height=.4\textheight]{\fignet/CMR18-AnnApplStat-Fig5a} 
    \end{tabular}
  \end{tabular}
  
  \bigskip
  \Refer{Back to \#\ref{back:plnExtensions}}

}

%====================================================================
\frame{\frametitle{Network inference (Barents' fish species)} \label{goto:plnNetwork}
  
  \vspace{-.05\textheight}
  $$
  \includegraphics[height=.85\textheight]{\fignet/CMR18b-ArXiv-Fig5}
  $$
  }

%====================================================================
\frame{\frametitle{PLN for network inference: choosing $\lambda$}

  $$
  \includegraphics[height=.8\textheight]{\fignet/BarentsFish_Gfull_criteria}
  \text{\Refer{Back to \#\ref{back:plnExtensions}}}
  $$

}

%====================================================================
\frame{\frametitle{Poisson SBM with covariates: choosing $K$} \label{goto:treeInterpretation}

  \begin{tabular}{c|c|c}
    & & \paragraph{Including the} \\
    \Refer{Back to \#\ref{back:treeCovariates}} & \paragraph{No covariate} & \paragraph{taxonomic distance} \\
    & & \\
    Observed network & Clustering (no covariate) & Group composition \\
    \includegraphics[height=.25\textwidth,width=.25\textwidth]{\fignet/Tree-adjMat}
    &
    \includegraphics[height=.25\textwidth,width=.25\textwidth]{\fignet/Tree-adjMat-SBMnull}
    &
    \includegraphics[height=.25\textwidth,width=.25\textwidth]{\fignet/Tree-adjMat-SBMtaxo}
    \\
    Model selection & Clustering & Group composition \\
    \includegraphics[height=.25\textwidth,width=.25\textwidth]{\fignet/Tree-ICL-SBMall}
    &
    \includegraphics[height=.3\textwidth,width=.25\textwidth,angle=270,origin=c,trim=0 20 0 0]{\fignet/MRV10_AoAS_Q7_group.png}
    &
    \includegraphics[height=.3\textwidth,width=.25\textwidth,angle=270,origin=c,trim=0 20 0 0]{\fignet/MRV10_AoAS_Q4_group.png} 
  \end{tabular}

}

%====================================================================
\frame{\frametitle{VBEM for binary SBM}  \label{goto:vbemSBM}

  \paragraph{Posterior credibility intervals (CI) \refer{GDR12}:}  
  Actual level for $\pi_1$ ($+$), $\gamma_{11}$
  (\textcolor{red}{$\triangle$}), $\gamma_{12}$
  (\textcolor{blue}{$\circ$}), $\gamma_{22}$
  (\textcolor{green}{$\bullet$}) \\
  \includegraphics[width=.9\textwidth]{\fignet/im-ICQ2-2-new} \\
%   \ra For all parameters, {\VBEM} posterior credibility
%   intervals achieve the nominal level (90\%), as soon as $n \geq 30$.

  \emphase{Width of the posterior CI.}
  {$\pi_1$}, \textcolor{red}{$\gamma_{11}$},
  \textcolor{blue}{$\gamma_{12}$}, \textcolor{green}{$\gamma_{22}$}
  \\
  \includegraphics[width=1\textwidth]{\fignet/im-ICQ2-3} \\

  \bigskip
  \ra 
  Width $\approx 1/\sqrt{n}$ for $\pi_1$ and $\approx 1/n = 1/\sqrt{n^2}$ for $\gamma_{11}$, $\gamma_{12}$ and $\gamma_{22}$.

  \bigskip
  \Refer{Back to \#\ref{back:statGuarantees}}

}

%====================================================================
\frame{\frametitle{SMC path} \label{goto:smcPath}

  \paragraph{Tree network, $S = \{taxo., geo.\}$.}
  $$
  \begin{array}{cc}
    \rho_h & \displaystyle{KL\left(p_h(Z) \; \| \; \prod_i p_h(Z_i)\right)} \\
    \includegraphics[width=.35\textwidth]{\figDoR/Tree-all-V10-M5000-rho} & 
    \includegraphics[width=.35\textwidth]{\figDoR/Tree-all-V10-M5000-MI} 
  \end{array}
  $$
  
  \bigskip
  \Refer{Back to \#\ref{back:treeNetwork}}
  
}

%====================================================================
\backupend 

%====================================================================
%====================================================================
\end{document}
%====================================================================
%====================================================================

  \begin{tabular}{cc}
    \hspace{-.04\textwidth}
    \begin{tabular}{p{.5\textwidth}}
    \end{tabular}
    &
    \begin{tabular}{p{.45\textwidth}}
    \end{tabular}
  \end{tabular}

