\documentclass[12pt]{article}

\usepackage{amsmath, amsfonts, amssymb, xspace}

% Mise en page
\textwidth 18cm
\textheight 27cm 
\topmargin -2.5 cm 
\oddsidemargin -1cm 
\evensidemargin -1cm
\renewcommand{\baselinestretch}{1.1}  

\newcommand{\Ncal}{\mathcal{N}}
\newcommand{\Esp}{\mathbb{E}}
\newcommand{\Ibb}{\mathbb{I}}
\newcommand{\pa}{\text{pa}}
\renewcommand{\d}{\text{d}}
\newcommand{\ra}{$\rightarrow$\xspace}

%---------------------------------------------------------------
%---------------------------------------------------------------
\begin{document}
%---------------------------------------------------------------
%---------------------------------------------------------------

{\large
%---------------------------------------------------------------
%---------------------------------------------------------------
\section{Model}
%---------------------------------------------------------------

%---------------------------------------------------------------
\paragraph{Data:}
\begin{itemize}
\item an {\sl ultrametric} phylogenetic tree $T$ with $n$ leaves and $m$ internal nodes \\
	($m = n-1$ if $T$ is binary)
\item $Y = (Y_j)_{j = 1..n}$ the value a trait $Y$ at each leaves
\end{itemize}

%---------------------------------------------------------------
\paragraph{Model without shift:}
\begin{itemize}
\item a branching Brownian motion starting in $\mu$ with variance $\sigma^2$ going along the branches of $T$:
$$
X_i | X_{\pa(i)} \sim \Ncal(X_{\pa(i)}, \ell_i\sigma^2)
$$
where $\pa(i)$ stands for the parent node of node $i$ in $T$.
\item joint distribution
$$
Y \Ncal(\mu_Y := \mu {\bf 1}, \Sigma = \sigma^2 W), \qquad [W]_{ii'} = t_{ii'}
$$
$t_{ii'} =$ length of the common history of nodes $i$ and $i'$.
\end{itemize}

%---------------------------------------------------------------
\paragraph{Model with shifts:}
\begin{itemize}
\item Same model as above except that
$$
X_i | X_{\pa(i)} \sim \Ncal(X_{\pa(i)} + \Delta_i, \ell_i\sigma^2)
$$
\ra not possible to locate to position of the shift with the branch.
\item $K =$ total number of shifts:
$$
K = \sum_{i=1}^N \Ibb\{\Delta_i \neq 0\}, \qquad N = n+m.
$$
\item Parameters: $\theta = (\sigma^2, \Delta)$
$$
\text{variance }\sigma^2, \text{location and magnitude of the shifts }\Delta = (\Delta_i)_i
$$
\ra both discrete and continuous parameters.
\end{itemize}

%---------------------------------------------------------------
%---------------------------------------------------------------
\newpage
\section{Maximum likelihood inference}
%---------------------------------------------------------------

%---------------------------------------------------------------
\paragraph{Incomplete data model point of view:}
\begin{itemize}
\item $Z = (Z_i)_{i = 1..m}$ the (unknown value) of the traits at each internal node
\item $X = (Z, Y)$ 'complete' dataset: 
$$
(X_i)_{i = 1..m} = (Z_i)_{i = 1..m}, \qquad (Y_i)_{i = (m+1)..(m+n)} = (Y_j)_{j = 1..n}
$$
\item joint distribution
$$
\begin{bmatrix}Z \\ Y \end{bmatrix}
\sim \Ncal\left(
\begin{bmatrix}\mu_Z \\ \mu_Y \end{bmatrix}, 
\begin{bmatrix} \Sigma_{XX} & \Sigma_{XY} \\ \Sigma_{YX} & \Sigma_{YY} = \Sigma \end{bmatrix}
\right)
$$
\item complete likelihood
$$
\log p(X; \theta) = \log p(Z, Y; \theta) = \sum_{i=1}^N \log p(X_i | X_{\pa(i)}; \theta)
$$
\end{itemize}

%---------------------------------------------------------------
\paragraph{Expectation-Maximization algorithm:}
\begin{itemize}
\item $\log p(X; \theta)$ easier to handle but not computable be cause $Z$ is unknown \\
\ra maximize 
$$
\Esp [\log p(Z, Y; \theta) | Y]
$$
\item E-step: compute some moments of $X | Y$: easy because the joint distribution is Gaussian
\item M-step: maximize wrt $\theta$, which includes to determine the branches hosting the shifts  \\
\ra Actually easy: just need to compute
$$
\log p(X_i | X_{\pa(i)}; \Delta_i=0, \sigma^2)
\qquad \text{and} \qquad
\log p(X_i | X_{\pa(i)}; \Delta_i, \sigma^2)
$$
and allocate the shifts to the $K$ branches with higher difference
\end{itemize}

%---------------------------------------------------------------
\paragraph{Linear regression point of view:}
\begin{itemize}
\item the model writes
$$
Y = T \Delta + E, \qquad E \sim \Ncal({\bf 0}_n, \Sigma)
$$
\ra regression with non iid residuals, equivalent to
\begin{eqnarray*}
W^{-1/2} Y & = & W^{-1/2} T \Delta + F, \qquad F \sim \Ncal({\bf 0}_n, \sigma^2 I_n) \\
\widetilde{Y} & = & \widetilde{T} \Delta + F
\end{eqnarray*}
\item Possible to initiate EM with lasso
\item + useful representation for model selection
\end{itemize}

%---------------------------------------------------------------
%---------------------------------------------------------------
\bigskip
\section{Identifiability}
%---------------------------------------------------------------

\begin{itemize}
\item Number of possible allocation of $K$ shifts on $N-1$ branches 
$$
\begin{pmatrix}N-1 \\ K \end{pmatrix}
$$
\item Number of coloration $n$ leaves into $K+1$ colors
$$
S_K^{PI}  \leq \begin{pmatrix}N-1 \\ K \end{pmatrix}, \qquad S_K^{PI} = \begin{pmatrix} 2n-2-K \\ K \end{pmatrix} \text{ if $T$ is binary}.
$$
\item Also possible to enumerate the set of allocations equivalent to a given one.
\end{itemize}

%---------------------------------------------------------------
%---------------------------------------------------------------
\bigskip
\section{Model selection}
%---------------------------------------------------------------

\begin{itemize}
\item Same settings as variable selection linear regression
\item Penalty criterion based on Barraud \& al.
\item Need to compute the complexity of the model collection: $S_K^{PI}$.
\end{itemize}

%---------------------------------------------------------------
%---------------------------------------------------------------
\bigskip
\section{Example}
%---------------------------------------------------------------

%---------------------------------------------------------------
%---------------------------------------------------------------
\bigskip
\section{Extensions}
%---------------------------------------------------------------

\begin{itemize}
\item From Brownian motion to Ornstein-Uhlenbeck:
$$
\d X(t) = - \alpha (X(t) - \mu) \d t + \sigma \d W(t)
$$
\ra stationary process
\item Multivariate trait
\item From tree $T$ to network $N$ to account for horizontal transfers
\end{itemize}
}
%---------------------------------------------------------------
%---------------------------------------------------------------
\end{document}
%---------------------------------------------------------------
%---------------------------------------------------------------


\begin{itemize}
\item 
\item 
\item 
\end{itemize}
