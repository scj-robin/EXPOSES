\documentclass[10pt]{beamer}

% Beamer style
%\usetheme[secheader]{Madrid}
% \usetheme{CambridgeUS}
\useoutertheme{infolines}
\usecolortheme[rgb={0.65,0.15,0.25}]{structure}
% \usefonttheme[onlymath]{serif}
\beamertemplatenavigationsymbolsempty
%\AtBeginSubsection

% Packages
%\usepackage[french]{babel}
\usepackage[latin1]{inputenc}
\usepackage{color}
\usepackage{xspace}
\usepackage{dsfont, stmaryrd}
\usepackage{amsmath, amsfonts, amssymb}
\usepackage{epsfig}
\usepackage{tikz}
\usepackage{url}
\usepackage{/home/robin/LATEX/Biblio/astats}
%\usepackage[all]{xy}
\usepackage{graphicx}



\input{/home/robin/RECHERCHE/EXPOSES/LATEX/Commands}

% Directory
\newcommand{\fignet}{/home/robin/Bureau/RECHERCHE/RESEAUX/EXPOSES/FIGURES}
\newcommand{\figchp}{/home/robin/Bureau/RECHERCHE/RUPTURES/EXPOSES/FIGURES}


%====================================================================
%====================================================================

%====================================================================
%====================================================================
\begin{document}
%====================================================================
%====================================================================

%====================================================================
\title[Variational inference of the PLN model]{Variational inference of the Poisson log-normal model \\
  Some applications in ecology}

\author[S. Robin]{S. Robin \\ ~\\
  \begin{tabular}{ll}
    Joint work with J. Chiquet \& M. Mariadassou
  \end{tabular}
  }

\institute[INRA / AgroParisTech]{~ \\%INRA / AgroParisTech \\
%   \vspace{-.1\textwidth}
  \begin{tabular}{ccc}
    \includegraphics[height=.06\textheight]{\fignet/LogoINRA-Couleur} & 
    \hspace{.02\textheight} &
    \includegraphics[height=.06\textheight]{\fignet/logagroptechsolo} % & 
%     \hspace{.02\textheight} &
%     \includegraphics[height=.09\textheight]{\fignet/logo-ssb}
    \\ 
  \end{tabular} \\
  \bigskip
  }

\date[AIGM, Toulouse]{AIGM, Dec. 2017, Toulouse}

%====================================================================
%====================================================================
\maketitle
%====================================================================

%====================================================================
%====================================================================
\section{Multivariate analysis of abundance data}
\frame{\tableofcontents[currentsection]}
%====================================================================

%====================================================================
\subsection*{Abundance data}
%====================================================================
\frame{\frametitle{Community ecology}

  \paragraph{Abundance data.} $Y = [Y_{ij}]: n \times p$: 
  \begin{eqnarray*}
   Y_{ij} & = & \text{abundance of species $j$ in sample $i$ (old)} \\
    & = & \text{number of reads associated with species $j$ in sample $i$ (new)}
  \end{eqnarray*}
%   $$
%   Y = [Y_{ij}]: n \times p, \qquad \text{either $n$ or $p$ 'large'}
%   $$
%   \begin{itemize}
%    \item $Y_{ij} =$ count of $k$-mer $j$ in sample $i$
%    \item $Y_{ij} =$ abundance of species $j$ in sample $i$
%    \item ...
%   \end{itemize}
  
  \bigskip \bigskip 
  \paragraph{Need for multivariate analysis:} 
  \begin{itemize}
   \item to summarize the information from $Y$
   \item to exhibit patterns of diversity
   \item to understand between-species interactions
%    \item ...
  \end{itemize}
  
  \bigskip
  \pause More generally, to \emphase{model dependences between count variables}
  $$
  \text{\ra Need for a generic (probabilistic) framework}
  $$

}

%====================================================================
\frame{\frametitle{Models for multivariate count data.}

  \paragraph{Abundance vector:} $Y_i = (Y_{i1}, \dots Y_{ip})$, $Y_{ij} = \emphase{\text{ counts } \in \Nbb}$

  \bigskip \bigskip \pause
  \paragraph{No generic model for multivariate counts.}
  \begin{itemize}
   \item Data transformation ($\widetilde{Y}_{ij} = \log (1+Y_{ij}), \sqrt{Y_{ij}}$) \\ 
   \ra Pb when many counts are zero. \\ ~
   \item Poisson multivariate distributions \\
   \ra Constraints of the form of the dependency \refer{IYA16} \\ ~
   \item Latent variable models \\
   \ra Poisson-Gamma (= negative binomial): positive dependency \\
   \ra \emphase{Poisson-log normal} \refer{AiH89}
  \end{itemize}
}

%====================================================================
\frame{\frametitle{{P}oisson-log normal (PLN) distribution}

  \paragraph{Latent Gaussian model:} 
  \begin{itemize}
   \item $(Z_i)_i:$ iid latent vectors $\sim \Ncal_p(0, \Sigma)$ \\ 
   \item $Y_i = (Y_{ij})_j:$ counts independent conditional on $Z_i$ 
   $$
   Y_{ij} \,|\, Z_{ij} \sim \Pcal\left(e^{\mu_j + Z_{ij}}\right)
   $$
  \end{itemize}

  \bigskip \pause
  \paragraph{Properties:}
  \begin{align*}
   \Esp(Y_{ij}) & = e^{\mu_j + \sigma^2_j/2} =: \lambda_j > 0 \\
   \Var(Y_{ij}) & = \lambda_j + \lambda_j^2 \left( e^{\sigma^2_j} - 1\right) 
   \qquad (\text{over-dispersion}) \\
   \Cov(Y_{ij}, Y_{ik}) & = \lambda_j \lambda_k \left( e^{\sigma_{jk}} - 1\right)
   \qquad \quad (\text{same sign as $\sigma_{jk}$}) 
   \end{align*}
}

%====================================================================
\frame{\frametitle{{P}oisson-log normal (PLN) distribution}

  \paragraph{Extensions.} 
  \begin{itemize}
   \item $x_i =$ vector of covariates for observation $i$;
   \item $o_{ij} = $ known 'offset'.
  \end{itemize}
  $$
  Y_{ij} \; | \; Z_{ij} \sim \Pcal(e^{\emphase{o_{ij} + x_i^\intercal \beta_j} + Z_{ij}})
  $$

  \bigskip \bigskip \pause
  \paragraph{Interpretation.} 
  \begin{itemize}
   \item Dependency structure encoded in the latent space (i.e. in $\Sigma$)
   \item Additional effects are fixed
   \item Conditional Poisson = noise model
  \end{itemize}

}

%====================================================================
%====================================================================
\section{Variational inference of PLN}
\frame{\tableofcontents[currentsection]}
%====================================================================

%====================================================================
\subsection*{Variational inference}
%====================================================================
\frame{\frametitle{Intractable EM}
  
  \paragraph{Aim of the inference:} 
  \begin{itemize}
   \item estimate $\theta = (\beta, \Sigma)$ 
   \item predict the $Z_i$'s
  \end{itemize}

  \bigskip \bigskip \pause
  \paragraph{Maximum likelihood.} EM requires to evaluate (some moments of)
  $$
  p(Z \; | \;  Y) = \prod_i p(Z_i \; | \;  Y_i)
  $$
  but no close form for $p(Z_i \; | \;  Y_i)$.
 
  \bigskip
  \begin{itemize}
   \item \refer{Kar05} resorts to numerical or Monte-Carlo integration.
  \end{itemize}
}

%====================================================================
\frame{\frametitle{Variational EM}
  
  \paragraph{Variational approximation:} replace $p(Z \; | \;  Y)$ with
  $$
  \pt(Z) = \prod_i \Ncal(Z_i; \mt_i, \St_i)
  $$
  and maximize the lower bound ($\Espt = $ expectation under $\pt$)
  \begin{eqnarray*}
    J(\theta, \pt) 
    & = & \log p_\theta(Y) - KL[\pt(Z) || p(Z|Y)] \\
    \\
    & = & \Espt [\log p_\theta(Y, Z)] + \Hcal[\pt(Z)]
  \end{eqnarray*}

  \bigskip \pause
  \paragraph{Variational EM.} 
  \begin{itemize}
   \item VE step: find the optimal $\pt$ (i.e. $\mt_i$'s and diagonal $\St_i$'s)
   \item M step: update $\widehat{\theta}$.
  \end{itemize}
}

%====================================================================
\frame{\frametitle{Variational EM}
  
  \paragraph{Property:} The lower $J(\theta, \pt)$ is bi-concave, i.e.
  \begin{itemize}
  \item wrt $\pt = (\Mt, \St)$ for fixed $\theta$ 
  \item wrt $\theta = (\Sigma, \beta)$ for fixed $\pt$ (close form for $\widehat{\Sigma} = n^{-1} (\Mt^\trans \Mt + \St_+)$)
  \end{itemize}
  but not jointly concave in general.

  \bigskip \bigskip 
  \paragraph{Implementation:} Gradient ascent for the complete parameter $(\Mt, \St, \theta)$
  \begin{itemize}
   \item No formal VEM algorithm.
  \end{itemize}
  
  \bigskip \bigskip \pause
  \paragraph{{\tt PLNmodels} package:} 
  $$
  \text{\url{https://github.com/jchiquet/PLNmodels}}
  $$
}

%====================================================================
%====================================================================
\section{Probabilistic PCA for counts}
\frame{\tableofcontents[currentsection]}
%====================================================================
%====================================================================
\subsection*{pPCA}
%====================================================================
\frame{\frametitle{Probabilistic PCA}

  \paragraph{Dimension reduction.} Typical task in multivariate analysis

  \bigskip \bigskip \pause
  \paragraph{Model:} Probabilistic PCA (pPCA):
  \begin{eqnarray*}
   (Z_i)_i \text{ iid} & \sim & \Ncal_p(0, \Sigma), \qquad \emphase{\text{rank}(\Sigma) = q \ll p} \\
   Y_{ij} | Z_{ij} & \sim & \Pcal(e^{o_{ij} + x_i^\trans \beta_j + Z_{ij}})
  \end{eqnarray*}
  Recall that: $\text{rank}(\Sigma) = q \quad \Leftrightarrow \quad \exists B (p \times q): 
  \Sigma = B B^\intercal$.
  
  \bigskip \bigskip \pause
  \paragraph{pPCA in the PLN model.} Variational inference:
  $$
  \text{maximize } J(\theta, \pt) %\qquad \text{s.t. rank}(\Sigma) \leq  q 
  $$
  \ra Still bi-concave in $\theta = (B, \beta)$ and $(\Mt, \St)$

}

%====================================================================
\frame{\frametitle{Model selection}

  \paragraph{Number of components $q$:} needs to be chosen.
  
  \bigskip \bigskip %\pause
  \paragraph{Penalized 'likelihood'.}
  \begin{itemize}
   \item $\log p_{\widehat{\theta}}(Y)$ intractable: replaced with $J(\widehat{\theta}, \pt)$ \\ ~
   \item $BIC$ \refer{Sch78} \ra $vBIC_q = J(\widehat{\theta}, \pt) - pq \log(n)/2$ \\ ~ 
   \item $ICL$ \refer{BCG00} \ra $vICL_q = vBIC_q - \Hcal(\pt)$ \\ ~ 
  \end{itemize}
  
  \bigskip
  \paragraph{Chosen rank:}
  $$
  \widehat{q} = \arg\max_q vBIC_q
  \qquad \text{or} \qquad
  \widehat{q} = \arg\max_q vICL_q
  $$
}

%====================================================================
\subsection*{Illustration}
%====================================================================

%====================================================================
\frame{\frametitle{Pathobiome: Oak powdery mildew}

  \paragraph{Data from \refer{JFS16}.} 
  \begin{itemize}
   \item $n = 116$ oak leaves = samples
   \item $p_1 = 66$ bacterial species (OTU)
   \item $p_2 = 48$ fungal species ($p = 114$)
   \item covariates: tree (resistant, intermediate, susceptible), branch height, distance to trunk, ...
   \item offsets: $o_{i1}, o_{i2} =$ offset for bacteria, fungi
  \end{itemize}

%   \bigskip \bigskip \pause
%   \paragraph{Aim.} Understand the interaction between the species, including the oak mildew pathogene {\sl E. alphitoides}.
}

%====================================================================
\frame{\frametitle{Pathobiome: PLN model ($q = p$)}

  \begin{overprint}
   \onslide<1>
   \paragraph{Without covariates:} offset only
   $$
   \includegraphics[width=.8\textwidth]{../FIGURES/PLN_oaks_plot2-1}
   $$
   \onslide<2>
   \paragraph{With covariates:} offset, tree (suscept., interm, resist.), orientation
   $$
   \includegraphics[width=.8\textwidth]{../FIGURES/PLN_oaks_plot3-1}
   $$
  \end{overprint}
}

%====================================================================
\frame{\frametitle{Pathobiome: PCA rank selection}

  $$
  \includegraphics[width=.9\textwidth]{../FIGURES/CMR17-Fig1a_ModSel.pdf}
  $$
  \qquad \qquad \qquad offset only: $\widehat{q} = 24$ \qquad \qquad offset + covariates: $\widehat{q} = 21$
}

%====================================================================
\frame{\frametitle{Visualization}

  \paragraph{PCA:} Optimal subspaces nested when $q$ increases.
  
  \bigskip \bigskip 
  \paragraph{PLN-pPCA:} Non-nested subspaces.

  \bigskip
  \ra For a the selected dimension $\widehat{q}$: \\~
  \begin{itemize}
   \item Compute the estimated latent positions $\widetilde{M}$ \\~
   \item Perform PCA on the $\widetilde{M}$ \\~
   \item Display results in any dimension $q \leq \widehat{q}$
  \end{itemize}
}


%====================================================================
\frame{\frametitle{Pathobiome: First 2 PCs}

  $$
  \includegraphics[width=.9\textwidth]{../FIGURES/CMR17-Fig1c_IndMap.pdf}
  $$
  \qquad \qquad \qquad offset only \qquad \qquad \qquad \qquad offset + covariates 
}


%====================================================================
\frame{\frametitle{Pathobiome: Precision of $\widehat{Z}_{ij}$}
  
  \begin{center}
     \begin{tabular}{cc}
	 $\sqrt{\widetilde{\Var}(Z_{ij})}$ & \begin{tabular}{c}
	 \includegraphics[width=.5\textwidth]{../FIGURES/CMR17-Fig2_VarZ.pdf} 
	 \end{tabular} \\
	 & $Y_{ij}$
	\end{tabular}
  \end{center}

  \bigskip
  Due to the link function (log), $\widetilde{\Var}(Z_{ij})$ is higher when $Y_{ij}$ is close to 0.}



%====================================================================
%====================================================================
\section{Network inference}
\frame{\tableofcontents[currentsection]}
%====================================================================

%====================================================================
\subsection*{Problem}
%====================================================================
\frame{\frametitle{Problem}

  \paragraph{Aim:} 'infer the ecological network'
  
  \bigskip \bigskip 
  \paragraph{Statistical interpretation:} infer the graphical model of the $Y_i = (Y_{i1}, \dots Y_{ip})$, i.e. the graph $G$ such that
  $$
  p(Y_i) \propto \prod_{C \in \Ccal(G)} \psi_C(Y_i^C)
  $$
  where $\Ccal(G) =$ set of cliques of $G$
  
  \bigskip \bigskip 
  \paragraph{Count data:} No generic framework (see Intro)
}

%====================================================================
\frame{\frametitle{PLN network inference}
  
  \renewcommand{\nodesize}{1.75em}
  \renewcommand{\edgeunit}{2.5*\nodesize}
  \paragraph{Cheat:} Use the PLN model and infer the graphical model of $Z$
  
  \bigskip  
  \begin{overprint}
   \onslide<2>
    $$
    \begin{array}{ccc}
    {\footnotesize \input{Fig2-jointZY} }
    & \qquad &
    {\footnotesize \input{Fig2-margY} }
    \end{array}
    $$
  \onslide<3>
   $$
   \begin{array}{ccc}
   {\footnotesize \begin{tikzpicture}
\node[hidden] (Z1) at ( 0.95*\edgeunit,  0.31*\edgeunit) {$Z_1$};
\node[hidden] (Z2) at (-0.00*\edgeunit,  1.00*\edgeunit) {$Z_2$};
\node[hidden] (Z3) at (-0.95*\edgeunit,  0.31*\edgeunit) {$Z_3$};
\node[hidden] (Z4) at (-0.59*\edgeunit, -0.81*\edgeunit) {$Z_4$};
\node[hidden] (Z5) at ( 0.59*\edgeunit, -0.81*\edgeunit) {$Z_5$};

\draw[edge] (Z1) to (Z2);  \draw[edge] (Z1) to (Z4);  
\draw[edge] (Z1) to (Z5);  \draw[edge] (Z2) to (Z3);  
\draw[edge] (Z2) to (Z4);  \draw[edge] (Z3) to (Z4); 

\node[observed] (Y1) at ( 1.05*\edgeunit, -0.39*\edgeunit) {$Y_1$};
\node[observed] (Y2) at (-0.00*\edgeunit,  0.30*\edgeunit) {$Y_2$};
\node[observed] (Y3) at (-1.05*\edgeunit, -0.39*\edgeunit) {$Y_3$};
\node[observed] (Y4) at (-0.59*\edgeunit, -1.51*\edgeunit) {$Y_4$};
\node[observed] (Y5) at ( 0.59*\edgeunit, -1.51*\edgeunit) {$Y_5$};

\draw[arrow] (Z1) to (Y1); 
\draw[arrow] (Z2) to (Y2);
\draw[arrow] (Z3) to (Y3);
\draw[arrow] (Z4) to (Y4);
\draw[arrow] (Z5) to (Y5);
\end{tikzpicture}
 }
   & \qquad &
   {\footnotesize \input{Fig1-margY} }
   \end{array}
  $$
  \end{overprint}

  \bigskip
  \onslide+<3->{
  $$
  \text{Graphical model of $Z$ \emphase{$\neq$} Graphical model of $Y$}
  $$
  }
}

%====================================================================
\frame{ \frametitle{PLN network model}

  \paragraph{Model:}
  \begin{eqnarray*}
   (Z_i)_i \text{ iid} & \sim & \Ncal_p(0, \emphase{\Omega^{-1}}), \qquad \emphase{\Omega \text{ sparse}} \\
   Y_{ij} | Z_{ij} & \sim & \Pcal(e^{o_{ij} + x_i^\trans \beta_j + Z_{ij}})
  \end{eqnarray*}
  
  \bigskip \pause
  \paragraph{Interest:} Similar to Gaussian graphical model (GGM) inference

  \bigskip \bigskip \pause
  \paragraph{Sparsity-inducing regularization:} graphical lasso (gLasso, \refer{FHT08})
  $$
  \log p_\theta(Y) - \lambda \; \|\Omega\|_{1, \text{off}}
  $$
}

%====================================================================
\subsection*{Variational inference}
%====================================================================
\frame{ \frametitle{Variational inference}

  \paragraph{Same problem:} $\log p_\theta(Y)$ is intractable
  
  \bigskip \bigskip \pause
  \paragraph{Variational approximation:} maximize
  $$
  J(\theta, \pt) - \lambda \; \|\Omega\|_{1, \text{off}}
  =
  \Espt[\log p_\theta(Y, Z)] + \Hcal[\pt(Z)] \emphase{- \lambda \; \|\Omega\|_{1, \text{off}}}
  $$
  with
  $$
  \pt(Z) = \prod \Ncal(Z_i; \mt_i, \St_i)
  $$
  
  \bigskip \bigskip \pause
  \ra Still bi-concave in $\theta = (\Omega, \beta)$ and $\pt = (\Mt, \St)$. Ex:
  $$
  \widehat{\Omega} = \arg\max_\Omega \, \frac{n}2 \left(\log |\Omega| - \tr(\widehat{\Sigma} \Omega)\right) - \lambda \|\Omega\|_{1, \text{off}}:
  \quad \text{gLasso problem}
  $$
}

%====================================================================
\frame{ \frametitle{Model selection}

  \paragraph{Network density:} controlled by $\lambda$ 
  
  \bigskip \bigskip \pause
  \paragraph{Penalized 'likelihood'.} 
  \begin{itemize}
   \item $vBIC(\lambda) = J(\widehat{\theta}, \pt)- \frac{\log n}2 \left(p q + |\text{Support}(\widehat{\Omega}_\lambda)| \right)$
   \item $EBIC(\lambda):$ Extended BIC \refer{FoD10}
  \end{itemize}

  \bigskip \bigskip \pause
  \paragraph{Stability selection.} 
  \begin{itemize}
   \item Get $B$ subsamples
   \item Get $\widehat{\Omega}^b_\lambda$ for an intermediate $\lambda$ and $b = 1...B$
   \item Count the selection frequency of each edge
  \end{itemize}

}

%====================================================================
\subsection*{Illustration}
%====================================================================
\frame{\frametitle{Oak powdery mildew: {\tt PLNmodels} package}

  \paragraph{Syntax:} ~\\ ~
  
  {\tt formula.offset <- Count $\sim$ 1 + offset(log(Offset))} \\~
  
  {\tt models.offset  <- PLNnetwork(formula.offset)} \\~ 

  {\tt best.offset <- models.offset\$getBestModel("BIC")} 

  }

%====================================================================
\frame{\frametitle{Oak powdery mildew: no covariates}

 \bigskip
 {\tt models.offset\$plot()}
 $$
 \includegraphics[height=.7\textheight]{../FIGURES/network_oak_offset_criteria}
 $$
}
 
%====================================================================
\frame{\frametitle{Oak powdery mildew: no covariates}

 \bigskip
 {\tt best.offset\$plot()}
 $$
 \includegraphics[height=.7\textheight]{../FIGURES/network_oak_offset_plot}
 $$
}

%====================================================================
\frame{\frametitle{Oak powdery mildew: no covariates}

 \bigskip
 {\tt best.offset\$plot\_network()}
 $$
 \includegraphics[height=.7\textheight]{../FIGURES/network_oak_offset_plot_network}
 $$
}

%====================================================================
\frame{\frametitle{Oak powdery mildew: effect of the covariates}

  \begin{tabular}{cc}
   no covariates & covariate = tree + orientation \\
   \begin{tabular}{c}
    \includegraphics[height=.6\textheight]{../FIGURES/network_oak_offset_network}
   \end{tabular}
   &
   \begin{tabular}{c}
    \includegraphics[height=.6\textheight]{../FIGURES/network_oak_tree_or_network}
   \end{tabular}
  \end{tabular}
  
  Ea = {\sl Erysiphe alphitoides} = pathogene responsible for oak mildew
  }

%====================================================================
\frame{\frametitle{Oak powdery mildew: stability selection}

  \begin{tabular}{cc}
   no covariates & covariate = tree + orientation \\
   \begin{tabular}{c}
    \includegraphics[height=.6\textheight]{../FIGURES/network_oak_offset_network_stabsel}
   \end{tabular}
   &
   \begin{tabular}{c}
    \includegraphics[height=.6\textheight]{../FIGURES/network_oak_tree_or_network_stabsel}
   \end{tabular}
  \end{tabular}
  }

%====================================================================
\section{Discussion}
\frame{\tableofcontents[currentsection]}
%====================================================================
\frame{ \frametitle{Discussion}

  \paragraph{Summary}
  \begin{itemize}
   \item PLN = generic model for multivariate count data analysis
   \item Allows for covariates
   \item Flexible modeling of the covariance structure
   \item Efficient VEM algorithm
   \item {\tt PLNmodels} package: \url{https://github.com/jchiquet/PLNmodels}
  \end{itemize}

  \bigskip \bigskip \pause
  \paragraph{To do list}
  \begin{itemize}
   \item Model selection criterion for network inference
   \item Tree-based network inference (R. Momal's PhD)
   \item Other covariance structures (spatial, time series, ...)
   \item Statistical properties of the variational estimates (for regular PLN)
  \end{itemize}
}

%====================================================================
\frame{ \frametitle{References}
{\tiny
  \bibliography{/home/robin/Biblio/BibGene}
%   \bibliographystyle{/home/robin/LATEX/Biblio/astats}
  \bibliographystyle{alpha}
  }
}

%====================================================================
%====================================================================
%====================================================================
\appendix 
\backupbegin
\section{Appendix}


\backupend

%====================================================================
%====================================================================
\end{document}
%====================================================================
%====================================================================

  \begin{tabular}{cc}
    \begin{tabular}{p{.5\textwidth}}
    \end{tabular}
    & 
    \hspace{-.02\textwidth}
    \begin{tabular}{p{.5\textwidth}}
    \end{tabular}
  \end{tabular}

