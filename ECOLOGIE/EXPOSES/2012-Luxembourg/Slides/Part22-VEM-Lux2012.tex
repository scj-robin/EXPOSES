%====================================================================
%====================================================================
\section{Variational EM}
\frame{\frametitle{Outline} \tableofcontents[currentsection]}
%====================================================================
\frame{\frametitle{General aim} 

  \paragraph{Problem.} $p_\theta(Z \mid Y)$ being intractable, we look for a 'good' approximation of it:
  $$
  q(Z) \approx p_\theta(Z \mid Y)
  $$

  \bigskip \bigskip 
  More specifically, given
  \begin{itemize}
  \item \pause \bigskip a \emphase{set of approximating distributions $\Qcal$} and
  \item \pause \bigskip a \emphase{divergence measure $D[q \| p]$},
  \end{itemize}
  \pause \bigskip we look for
  $$
  q^* = \argmin_{q \in \emphase{\Qcal}} \; \emphase{D}\left[q(Z) \| p_\theta(Z \mid Y)\right]
  $$

}

%====================================================================
\frame{\frametitle{Variational approximations} 

  \paragraph{References.} Huge literature; see \refer{WaJ08} for a general introduction or \refer{BKM17} for a more recent and concise review
  
  \pause \bigskip \bigskip 
  \paragraph{Not all methods} enter the framework described above
  \begin{itemize}
  \item loopy belief propagation \refer{MWJ99}
  \item minimization of Bethe's free energy \refer{YFW01b}
  \end{itemize}

  \pause \bigskip \bigskip 
  \paragraph{Choice of the divergence measure.} 
  \begin{itemize}
  \item Most popular choice = K\"ullback--Leibler: 
  $$
  D[q \| p] = KL[q \| p] = \Esp_q \log \left({q}/{p}\right)
  $$
  \ra the error $\log(q/p)$ is averaged wrt the approximation $q$ itself \\ ~
  \item \pause Expectation propagation (EP, \refer{Min01}): $D[q \| p] = KL[p \| q]$ \\
  \ra more sensible, but requires integration wrt $p$ \\ ~
  \item \pause Many others (see e.g. \refer{Min05})
  \end{itemize}

}

%====================================================================
\frame{\frametitle{Variational EM algorithm} 

  \paragraph{In a nutshell:} replace the E step with an approximation ('VE') step
  
  \pause \bigskip \bigskip
  \paragraph{'Evidence\footnotetext[1]{Actually log-evidence, as the evidence is $p(Y)$} lower bound' (ELBO) =} lower bound of the log-likelihood:
  \begin{align*}
  J_{\theta, q}(Y) = \log p_\theta(Y) - KL\left[q(Z) \| p_\theta(Z \mid Y)\right]
  \end{align*}
  
  \pause \bigskip \bigskip 
  \paragraph{VEM algorithm.}
  \begin{description}
  \item[VE step:] \bigskip maximize $J_{\theta, q}(Y)$ wrt $q$
  \item[M step:] \bigskip maximize $J_{\theta, q}(Y)$ wrt $\theta$
  \end{description}

  \pause \bigskip \bigskip 
  \paragraph{Property:}
  $J_{\theta, q}(Y)$ increases at each step.

}

%====================================================================
\frame{\frametitle{Variational EM algorithm} 

  The ELBO can written in two ways:
  \begin{align*}
  J_{\theta, q}(Y) 
  & = \log p_\theta(Y) - \emphase{KL\left[q(Z) \| p_\theta(Z \mid Y)\right]} \\ 
  ~ \\
  & = \emphase{\Esp_q \log p_\theta(Y, Z)} - \Esp_q \log q(Z)
  \end{align*}
  \bigskip
  \ra See \#\ref{goto:ELBO} \label{back:ELBO}

  \pause \bigskip \bigskip 
  \paragraph{VEM algorithm.} 
  \begin{itemize}
  \item \emphase{VE step} (approximation):
  $$
  q^{h+1} = \argmin_{q \in \Qcal} \; KL\left[q(Z) \| p_{\theta^h}(Z \mid Y)\right]
  $$
  \item \emphase{M step} (parameter update):
  $$
  \theta^{h+1} = \argmax_\theta \; \Esp_{q^{h+1}} \log p_\theta(Y, Z)
  $$
  \end{itemize}

}

%====================================================================
\frame{\frametitle{EM as a VEM algorithm} 

  We have that
  \begin{align*}
  \log p_\theta(Y) 
  & = \Esp[\log p_\theta(Y, Z) \mid Y] - \Esp[\log p_\theta(Z \mid Y) \mid Y]
  & & (\text{EM}) \\ ~ \\
  J_{\theta, q}(Y) 
  & = \Esp_q [\log p_\theta(Y, Z)] - \Esp_q [\log q(Z)]
  & & (\text{VEM})
  \end{align*}
  
  \bigskip \bigskip 
  \begin{itemize}
  \item \pause Both are the same iff $q(Z) = p_\theta(Z \mid Y)$ \qquad \qquad (as $KL\left[q^{h+1}(Z) \| p_{\theta^h}(Z \mid Y)\right] = 0$)
  \item \pause \bigskip This happens when $\Qcal$ is \emphase{unrestricted}, that is
  $$
  q^{h+1}(Z) 
  = \argmin_{q} \; KL\left[q(Z) \| p_{\theta^h}(Z \mid Y)\right]
  = p_{\theta^h}(Z \mid Y)
  $$
  \item \pause \bigskip This provides us with a second proof of EM's main property
  \end{itemize}


}

%====================================================================
\frame{\frametitle{'Mean-field' approximations} 

  \pause 
  \paragraph{Choice of the approximation class.} A popular choice is
  \begin{align*}
    \Qcal_{\text{fact}} 
    = \{\text{\emphase{factorable} distributions}\} 
    = \{q: q(Z) = \prod_i q_i(Z_i)\}
  \end{align*}
  
  \pause \bigskip \bigskip 
  \paragraph{Property.} For a given distribution $p(Z)$, 
  $$
  q^* = \argmin_{q \in \Qcal_{\text{fact}}} \; KL[q \| p] 
  $$
  satisfies
  $$
  q^*_i(Z_i) \propto \exp\left(\Esp_{\bigotimes_{j \neq i} q^*_j} \log p(Z) \right)
  $$
  \ra Proof in \refer{Bea03} (sketch in \#\ref{goto:meanfield}) \label{back:meanfield}

  \pause \bigskip \bigskip 
  \begin{itemize}
  \item $\log q^*_i(Z_i)$ is obtained by setting the $\{Z_j\}_{j \neq i}$ 'to their respective mean' (each wrt to $q^*_j$).
  \end{itemize}

}
