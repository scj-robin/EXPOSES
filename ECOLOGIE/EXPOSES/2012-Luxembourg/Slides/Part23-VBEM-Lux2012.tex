%====================================================================
%====================================================================
\section{Variational Bayes EM}
\frame{\frametitle{Outline} \tableofcontents[currentsection]}
%====================================================================
\frame{\frametitle{Bayesian inference} 

  \paragraph{Bayesian setting:} The parameters in $\theta$ are random 
  \qquad \qquad (no latent variable yet)
  
  \bigskip 
  \begin{itemize}
  \item \pause '\emphase{Prior}' = marginal distribution of the parameter
  $$
  p(\theta)
  $$
  \item \pause \bigskip '\emphase{Likelihood}' = conditional distribution of the observations 
  $$
  p(Y \mid \theta)
  $$
  \item \pause \bigskip '\emphase{Posterior}' = conditional distribution of the parameters given the data
  $$
  p(\theta \mid Y) = \frac{p(\theta) p(Y \mid \theta)}{\int p(\theta) p(Y \mid \theta) \d \theta}
  $$
  \end{itemize}

}
  
%====================================================================
\frame{\frametitle{Variational Bayes} 

  \paragraph{Ideal case:} Explicit posterior 
  \ra Conjugate priors 
  
  \pause \bigskip \bigskip 
  \paragraph{Most of the time:} No explicit form for $p(\theta \mid Y)$
  \begin{itemize}
  \item \pause \bigskip Sample from it, i.e. try to get
  $$
  \{\theta^b\}_{1 \leq b \leq B} \overset{\text{iid}}{\approx} p(\theta \mid Y)
  $$
  \ra Monte-Carlo (MC), MCMC, SMC, HMC, ...
  \item \pause \bigskip Approximate it, i.e. look for
  $$
  q(\theta) \simeq p(\theta \mid Y)
  $$
  \ra Variational Bayes (VB) \refer{Att00} \\
  \end{itemize}
  
  \pause \bigskip \bigskip 
  \paragraph{Example.} Consider $\Ncal = \{\text{Gaussian distributions}\}$
  $$
  q^*(\theta) 
  = \argmin_{q \in \Ncal} \; KL[q(\theta) \mid p(\theta \mid Y)]
  $$
  (or $KL[p(\theta \mid Y) \mid q(\theta)]$)

}


%====================================================================
\frame{\frametitle{Including latent variables} 

  \paragraph{Bayesian model with latent variables.} 
  \begin{align*}
    \theta & \sim p(\theta) & & \text{prior distribution} \\ ~
    Z & \sim p(Z \mid \theta) & & \text{latent variables} \\ ~
    Y & \sim p(Y \mid \theta, Z) & & \text{observed variables} \\ ~
  \end{align*}

  \pause \bigskip \bigskip 
  \paragraph{Aim of Bayesian inference.} Determine the joint conditional distribution
  $$
  p(\theta, Z \mid Y) = \frac{p(\theta) \; p(Z \mid \theta) \; p(Y \mid \theta, Z)}{p(Y)}
  $$
  where
  $$
  p(Y) = \int \int p(\theta) \; p(Z \mid \theta) \; p(Y \mid \theta, Z) \d \theta \d Z
  $$
  is most often intractable
  
  }

%====================================================================
\frame{\frametitle{Variational Bayes EM} 

  \paragraph{Variational approximation} of the joint conditional $p(\theta, Z \mid Y)$
  $$
  q(\theta, Z) = \argmin_{q \in \Qcal} \; KL[q(\theta, Z) \| p(\theta, Z \mid Y)]
  $$
  taking $\Qcal = \Qcal_{\text{fact}} = \{q: q(\theta, Z) = q_\theta(\theta) q_Z(Z)\}$ \refer{Bea03,BeG03}
  
  \pause \bigskip \bigskip 
  \paragraph{Variational Bayes EM (VBEM) algorithm.} Makes use of the mean-field approximation 
  \begin{itemize}
  \item \pause \bigskip \emphase{VBE step =} update of the latent variable distribution
  $$
  q_Z^{h+1}(Z) \propto \exp\left(\Esp_{q_\theta^{h}} \log p(Y, Z, \theta)\right)
  $$
  \item \pause \bigskip \emphase{VBM step =} update of the parameter distribution
  $$
  q_\theta^{h+1}(\theta) \propto \exp\left(\Esp_{q_Z^{h+1}} \log p(Y, Z, \theta)\right)
  $$
  \end{itemize}

}

%====================================================================
\frame{\frametitle{VBEM in practice} 

  \paragraph{Exponential family / conjugate prior.} If
  \begin{align*}
  p(Y, Z \mid \theta) & \text{ belongs to the exponential family} \\ ~ \\
  \text{and } p(\theta) & \text{ is the corresponding conjugate prior}
  \end{align*}
  then both the VBE and VBM steps are completely explicit \refer{BeG03}
  
  \pause \bigskip \bigskip 
  \paragraph{Many VBEM's.} 
  \begin{itemize}
  \item \bigskip Force further factorization among the $Z$ (see e.g. \refer{LBA12,GDR12,KBC15} for block-models)
  \item \bigskip Use further approximations when conjugacy does not hold \refer{JaJ00}
  \end{itemize}

}
