%====================================================================
%====================================================================
\section{Models with latent variables in (community) ecology}
\frame{\frametitle{Outline} \tableofcontents[currentsection]}
%====================================================================
% \frame{\frametitle{} 
% 
%   \begin{description}
%    \item[Mixture models:] species richness \url{https://esajournals.onlinelibrary.wiley.com/doi/full/10.1890/04-1078, https://esajournals.onlinelibrary.wiley.com/doi/full/10.1002/ecy.2093}
%    \item[Mixed (G)LMM:] species distribution models=SDM \url{https://onlinelibrary.wiley.com/doi/epdf/10.1111/j.1472-4642.2008.00482.x, https://onlinelibrary.wiley.com/doi/abs/10.1111/j.1600-0587.2010.06433.x}
%    \item[HMM:] animal movement \url{https://arxiv.org/abs/1806.10639}
%   \end{description}
% 
%   \bigskip
%   \paragraph{'Hidden' variables vs 'instrumental' variables}
% 
%   \bigskip
%   \paragraph{Notations.} $X$, $Y$, $Z$, $\theta$, $p_\theta$
% 
% }

%====================================================================
\frame{\frametitle{Community ecology}

  {\sl A community is a group [\dots] of populations of [\dots] different species occupying the same geographical area at the same time.} \Refer{Wikipedia}

  \bigskip 
  {\sl Community ecology [\dots] is the study of the interactions between species in communities [\dots].} \Refer{Wikipedia}

  \pause \bigskip \bigskip
  \paragraph{Need for statistical models to}
  \begin{itemize}
  \item \bigskip decipher / describe / evaluate \\
  \qquad {\sl abiotic} interactions: environmental effects on species \\
  \qquad {\sl biotic} interactions: between-species interactions \\ ~\\
  \ra \emphase{joint species distribution models}
  \item \pause \bigskip describe / understand the organisation of species
  interaction networks \\ ~\\
  \ra \emphase{network models}
  \end{itemize}
  
}

