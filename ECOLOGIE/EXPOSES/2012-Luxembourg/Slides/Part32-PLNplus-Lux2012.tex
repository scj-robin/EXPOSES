%====================================================================
%====================================================================
\section{Extensions of the Poisson log-normal model}
\frame{\frametitle{Outline} \tableofcontents[currentsection]}
% %====================================================================
% \frame{\frametitle{Flexibility of the latent layer} 
% 
% }
  
%====================================================================
\subsection{Dimension reduction}
%====================================================================
\frame{\frametitle{Dimension reduction} 

  \paragraph{Typical context.} 
  \begin{itemize}
  \item Microbial ecology: $p = 10^2$, $10^3$, $10^4$ species
  \item \bigskip 'Abundance' = 'read' count = number of genomic sequences associated with each species sampled via high-troughput sequencing ('metagenomic')
  \end{itemize}

  \pause \bigskip \bigskip 
  \paragraph{Aim.} 
  \begin{itemize}
  \item Dimension reduction (visualization)
  \item \bigskip Accounting for major known effects
  \end{itemize}

  \pause \bigskip \bigskip 
  \paragraph{Probabilistic principal component analysis.} Gaussian setting \refer{TiB99}:
  $$
  \Sigma = \underset{\text{low rank}}{\underbrace{B B^\intercal}} + \sigma^2 I_p, \qquad \qquad \text{where } B (p \times r)
  $$

}
  
%====================================================================
\frame{\frametitle{(PLN-)probabilistic PCA} 

  \paragraph{PLN-PCA model.} \refer{CMR18a}
  \begin{itemize}
  \item \pause Low dimension latent vector
  $$
  W_i \sim \Ncal_r(0, I), \qquad \qquad \text{where } r \ll p
  $$
  \item \pause \bigskip $p$-dimensional latent vector
  $$
  Z_i = \emphase{B} W_i \qquad \qquad  \text{where } \emphase{B} (p \times r) =
  \text{loading matrix}
  $$
  \item \pause \bigskip Observed counts
  $$
  Y_{ij} \sim \Pcal(\exp(\emphase{o_{ij}} + x_i^\intercal \beta + Z_{ij}))
  $$ 
  ~\\
  $\emphase{o_{ij}} =$ known 'offset' coefficient, accounting for the sampling
  effort
  \item \pause \bigskip Parameters
  $$
  \theta = (\text{loading matrix }B, \text{regression coefficient } \beta) \qquad
  \qquad (+ \text{rank } r)
  $$
  \end{itemize}

}

%====================================================================
\frame{\frametitle{Variational inference for PLN-PCA} 

  \paragraph{VEM algorithm.}
  \begin{itemize}
  \item \pause \emphase{VE step:} update the variational parameters 
  $m^{h+1}_i = \Esp_{q_i^{h+1}}(\emphase{W_i})$ and $S^{h+1}_i = \Esp_{q_i^{h+1}}(\emphase{W_i})$ \\
  \ra Similar to the VE step of regular PLN
  \item \pause \bigskip \emphase{M step:} update the model parameters $B^{h+1}$ and $\beta^{h+1}$ \\
  \ra no close form, but still convex problem (gradient descent)
  \end{itemize}

  \bigskip 
  \paragraph{Model selection.}
  \begin{itemize}
  \item \pause BIC penalty \refer{Sch78} (Laplace approximation):
  $
  \text{pen}_{BIC}(\theta) = (\underset{\beta}{\underbrace{p\; d}} + \underset{B}{\underbrace{p \; r}}) {\log n} / 2
  $
  \item \pause Heuristic adaptation (replace $\log p_\theta(Y)$ with $J_{\theta, q}(Y)$)
  $$
  vBIC = J_{\theta, q}(Y) - \text{pen}_{BIC}(\theta) 
  $$
  \item \pause Inspired from \refer{BCG00} (additional penalty for the conditional entropy the $W_i$'s)
  \begin{align*}
  vICL 
  = J_{\theta, q}(Y) - \text{pen}_{BIC}(\theta)  - \Hcal(q)
  \textcolor{gray}{\; =  \Esp_q \log p_\theta(Y, Z)  - \text{pen}_{BIC}(\theta)} 
  \end{align*}
  \end{itemize}

}

%====================================================================
\frame{\frametitle{Oak powdery mildew}
  \begin{tabular}{cl}
    \hspace{-.04\textwidth}
    \begin{tabular}{p{.3\textwidth}}
      \paragraph{Metabarcoding data} \refer{JFS16} 
      \begin{itemize}
      \item \bigskip $p = 114$ OTUs \\
      (66 bacteria and 48 fungi) 
      \item \bigskip $n = 116$ leaves 
      \item \bigskip collected on 3 trees 
        \begin{itemize}
        \item resistant 
        \item intermediate
        \item susceptible       
      \end{itemize}
      to oak powdery mildew;
      \item \bigskip different protocole for bacteria and fungi \\
      $o_{ij} =$ sequencing depth
      \end{itemize}

    \end{tabular}
    &
    \begin{tabular}{c}
    \pause \includegraphics[height=.4\textheight]{\fignet/CMR18-AnnApplStat-Fig5a} \\
    ~ \\
    \pause \includegraphics[height=.4\textheight]{\fignet/CMR18-AnnApplStat-Fig4a} 
    \end{tabular}
  \end{tabular}
  
}

%====================================================================
\subsection{Network inference}
%====================================================================
\frame{\frametitle{Network inference} 

  \begin{tabular}{cc}
    \hspace{-.04\textwidth}
    \begin{tabular}{p{.6\textwidth}}
      \paragraph{Species interaction networks.}
      \begin{itemize}
      \item \bigskip Aim: Understand how species from a same community interact
      \item \bigskip Network representation = draw an edge between interacting pairs of species
      \item \pause \bigskip Main issue: Distinguish 
      \emphase{direct interactions} (predator-prey) from
      simple \emphase{associations} (two preys of a same predator) 
      \end{itemize}
    \end{tabular}
    &
    \hspace{-.05\textwidth}
    \begin{tabular}{ccc}
      \begin{tikzpicture}
  \node[nocircle] (pred) at ( 0.0*\edgeunit, 1.0*\edgeunit) {predator};
  \node[nocircle] (prey) at ( 0.0*\edgeunit, 0.0*\edgeunit) {prey};

  \draw[edge] (pred) to (prey);
\end{tikzpicture}
 &
      & 
      \begin{tikzpicture}
  \node[nocircle] (pred) at ( 0.6*\edgeunit, 1.0*\edgeunit) {predator};
  \node[nocircle] (prey1) at ( 0.0*\edgeunit, 0.0*\edgeunit) {prey1};
  \node[nocircle] (prey2) at ( 1.2*\edgeunit, 0.0*\edgeunit) {prey2};

  \draw[edge] (pred) to (prey1);
  \draw[edge] (pred) to (prey2);
  \draw[edgered] (prey1) to (prey2);
\end{tikzpicture}

    \end{tabular}
  \end{tabular}
  
  \pause 
  \ra Obviously, analyses based on co-occurences or correlations are not sufficient \refer{PWT19}

  \pause \bigskip \bigskip 
  \paragraph{Probabilistic translation.}
  \begin{align*}
    \text{association} & = \text{marginal dependance} \\
    \text{direct interaction} & = \text{conditional dependance} 
  \end{align*}

}
  
%====================================================================
\frame{\frametitle{Undirected graphical models} 

  \paragraph{Definition.} 
  $p(U_1, \dots U_k)$ is {\sl faithful} to the (chordal) graph $G = ([k], E)$ iff
  $$
  p(U_1, \dots U_k) \propto \prod_{C \in \Ccal} \psi_C(U_C)
  $$
  where $\Ccal = \{\text{cliques of } G\}$ and $U_C = (Y_j)_{j \in C}$.
  
  \pause \bigskip \bigskip 
  \paragraph{Property.} 
  $$
  \text{separation}
  \qquad \Leftrightarrow \qquad 
  \text{conditional independance}
  $$

  \pause \bigskip 
  \paragraph{Example.} ~ \\
  \begin{tabular}{ccc}
    \hspace{-.04\textwidth}
    \begin{tabular}{c}
      \begin{tikzpicture}
    \node[observed] (U1) at (0.0*\edgeunit, 0.0*\edgeunit) {$U_1$};
  \node[observed] (U2) at (0.0*\edgeunit, 1.0*\edgeunit) {$U_2$};
  \node[observed] (U3) at (1.0*\edgeunit, 0.0*\edgeunit) {$U_3$};
  \node[observed] (U4) at (1.0*\edgeunit, 1.0*\edgeunit) {$U_4$};
  

  
  \draw[edge] (U1) to (U2);  \draw[edge] (U2) to (U3);  
  \draw[edge] (U1) to (U3);  \draw[edge] (U3) to (U4);  
\end{tikzpicture}
 \\
      ~ \\
      $C_1 = \{1, 2, 3\}, C_2 = \{3, 4\}$ \\
    \end{tabular}
    & &
    \pause
    \begin{tabular}{p{.5\textwidth}}
      $$
      p(U_1, U_2, U_3, U_4) \propto \psi_1(U_1, U_2, U_3) \; \psi_2(U_3, U_4)
      $$
      \begin{itemize}
      \item $(U_1, U_2, U_3, U_4)$ all dependent 
%        \item $U_1 \not\independent U_2$,  $U_1 \not\independent U_3$, $U_1 \not\independent U_4$, $U_2 \not\independent U_3$, ...
      \item $U_1 \not\independent U_2 \mid (U_3, U_4)$ 
      \item $U_4 \not\independent U_1 \mid U_2$ 
      \item $U_4 \independent (U_1, U_2) \mid U_3$ 
      \end{itemize}
    \end{tabular}
  \end{tabular}

}
  

%====================================================================
\frame{\frametitle{Gaussian graphical models} \pause

  Suppose $Z \sim \Ncal(0, \Sigma)$ and denote by $\Omega = [\omega_{jk}] = \Sigma^{-1}$ the {\sl precision} matrix:
  \begin{align*}
  \sigma_{jk} = 0 
  & \Leftrightarrow 
  (Z_j, Z_k) \text{ independent}
  & & (\text{'correlation'}) \\
  ~ \\ 
  \omega_{jk} = 0 
  & \Leftrightarrow 
  (Z_j, Z_k) \text{ independent } \mid (Z_h)_{h \neq j, k}
  & & (\text{'partial correlation'})
  \end{align*}
  ~ \\
  \pause \ra $\Omega$ only refers to 'direct' dependencies $\Rightarrow$ $G$ given by the \emphase{support of $\Omega$} 
  
  \pause \bigskip \bigskip 
  \paragraph{Graphical lasso.} \refer{FHT08}
  \begin{itemize}
  \item Common assumption: few species are in direct interaction 
  $$
  \Rightarrow \quad \Omega \text{ should be \emphase{sparse} \qquad (many 0's)}
  $$ 
  \item \pause \bigskip Sparsity-inducing penalty (graphical lasso)
  $$
  \max_{\Omega} \; \log p(Z; \Omega) - \lambda \underset{\ell_1 \text{ penalty}}{\underbrace{\sum_{j \neq k} |\omega_{jk}|}}
  $$
  \end{itemize}

}

%====================================================================
\frame{\frametitle{Poisson log-normal model for network inference} 

  \paragraph{PLN-network.} PLN model with graphical lasso penalty \refer{CMR19}
  $$
  \arg\max_{\beta, \Omega, q \in \Qcal} \; 
  J(\beta, \Omega, q)
  - \underset{{\text{$\ell_1$ penalty}}}{\underbrace{\lambda \sum_{j \neq k} |\omega_{jk}|}}
  $$
  \ra Convex problem for both the VE and the M step

  \onslide<2->{\bigskip \bigskip  
  \paragraph{Inferring the {\sl latent} dependency structure}, not the abundance one}
  $$
  \begin{array}{c|c}
  \onslide<3->{\text{good case}} & \onslide<4->{\text{bad case}} \\
  \hline
  \onslide<3->{\includegraphics[width=.4\textwidth]{\fignet/CMR18b-ArXiv-Fig1b}}
  &
  \onslide<4->{\includegraphics[width=.4\textwidth]{\fignet/CMR18b-ArXiv-Fig1a}}
  \end{array}
  $$
  \onslide<5->{\ra Similar setting for most approaches in statistical ecology \refer{WBO15,KMM15,FHZ17,PHW18}}

}

%====================================================================
\frame{\frametitle{Barents' fish species} 
  
  \vspace{-.05\textheight}
  \begin{tabular}{cc}
    \hspace{-.04\textwidth}
    \begin{tabular}{p{.22\textwidth}}
      \paragraph{Data:} \\ ~
      \begin{itemize}
      \item $n=89$ sites \\~
      \item $p=30$ species \\~
      \item $d=4$ covariates
        \begin{itemize}
        \item latitude
        \item longitude
        \item temperature
        \item depth
        \end{itemize}
      \end{itemize}
    \end{tabular}
    &
    \begin{tabular}{c}
      \includegraphics[height=.85\textheight]{\fignet/CMR18b-ArXiv-Fig5}
    \end{tabular}
  \end{tabular}
  }

%====================================================================
\frame{\frametitle{Barents' fish species: choosing $\lambda$}

  \begin{tabular}{ll}
    %\hspace{-.04\textwidth}
    \begin{tabular}{p{.45\textwidth}}
      \includegraphics[height=.7\textheight]{\fignet/BarentsFish_Gfull_criteria}
    \end{tabular}
    &
    \begin{tabular}{p{.4\textwidth}}
      \paragraph{Alternatively.} ~ \\
      ~ \\
      Use resampling and select edges based on selection frequency \\
      ~ \\
      \refer{LRW10}
    \end{tabular}
  \end{tabular}

}


% %====================================================================
% \frame{\frametitle{Other interesting extensions} 
% 
%   \begin{itemize}
%     \item Species traits
%     \item Spatial dependency
%   \end{itemize}
% 
% 
% }


