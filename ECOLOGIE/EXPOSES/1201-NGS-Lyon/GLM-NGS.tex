\documentclass{beamer}

% Beamer style
%\usetheme[secheader]{Madrid}
\usetheme{CambridgeUS}
\useinnertheme{rectangles}
%\usefonttheme[onlymath]{serif}
\usecolortheme[rgb={0.65,0.15,0.25}]{structure}
\beamertemplatenavigationsymbolsempty
%\AtBeginSubsection

% Packages
%\usepackage[french]{babel}
\usepackage[latin1]{inputenc}
\usepackage{color}
\usepackage{dsfont, stmaryrd}
\usepackage{amsmath, amsfonts, amssymb}
\usepackage{stmaryrd}
\usepackage{verbatim}
\usepackage{epsfig}
\usepackage{url}
\usepackage{/media/donnees/LATEX/astats}
%\usepackage[all]{xy}
\usepackage{graphicx}
\usepackage{arydshln}

% Commands
\definecolor{darkred}{rgb}{0.65,0.15,0.25}
\newcommand{\emphase}[1]{\textcolor{darkred}{#1}}
\newcommand{\paragraph}[1]{\emphase{#1}}
\newcommand{\refer}[1]{\textcolor{blue}{\sl \cite{#1}}}
\newcommand{\Refer}[1]{\textcolor{blue}{\sl #1}}
\newcommand{\newblock}{}
\newcommand{\Pause}{\pause}
%\newcommand{\Pause}{}

% Symbols
\newcommand{\Abf}{{\bf A}}
\newcommand{\Beta}{\text{B}}
\newcommand{\Bcal}{\mathcal{B}}
\newcommand{\BIC}{\text{BIC}}
\newcommand{\cbf}{{\bf c}}
\newcommand{\dd}{\text{d}}
\newcommand{\dbf}{{\bf d}}
\newcommand{\Dcal}{\mathcal{D}}
\newcommand{\Esp}{\mathbb{E}}
\newcommand{\Ebf}{{\bf E}}
\newcommand{\Ecal}{\mathcal{E}}
\newcommand{\Gcal}{\mathcal{G}}
\newcommand{\Gam}{\mathcal{G}\mbox{am}}
\newcommand{\Ibb}{\mathbb{I}}
\newcommand{\Ibf}{{\bf I}}
\newcommand{\ICL}{\text{ICL}}
\newcommand{\Cov}{\mathbb{C}\text{ov}}
\newcommand{\Corr}{\mathbb{C}\text{orr}}
\newcommand{\Var}{\mathbb{V}}
\newcommand{\Vsf}{\mathsf{V}}
\newcommand{\pen}{\text{pen}}
\newcommand{\Fcal}{\mathcal{F}}
\newcommand{\Hbf}{{\bf H}}
\newcommand{\Hcal}{\mathcal{H}}
\newcommand{\Jcal}{\mathcal{J}}
\newcommand{\Kbf}{{\bf K}}
\newcommand{\Lcal}{\mathcal{L}}
\newcommand{\Mcal}{\mathcal{M}}
\newcommand{\mbf}{{\bf m}}
\newcommand{\mum}{\mu(\mbf)}
\newcommand{\Ncal}{\mathcal{N}}
\newcommand{\Nbf}{{\bf N}}
\newcommand{\Nm}{N(\mbf)}
\newcommand{\Ocal}{\mathcal{O}}
\newcommand{\Obf}{{\bf 0}}
\newcommand{\Omegas}{\underset{s}{\Omega}}
\newcommand{\Pbf}{{\bf P}}
\newcommand{\Pcal}{\mathcal{P}}
\newcommand{\Qcal}{\mathcal{Q}}
\newcommand{\Rbb}{\mathbb{R}}
\newcommand{\Rcal}{\mathcal{R}}
\newcommand{\sbf}{{\bf s}}
\newcommand{\Sbf}{{\bf S}}
\newcommand{\Scal}{\mathcal{S}}
\newcommand{\Ucal}{\mathcal{U}}
\newcommand{\Vcal}{\mathcal{V}}
\newcommand{\Tbf}{{\bf T}}
\newcommand{\ubf}{{\bf u}}
\newcommand{\Ubf}{{\bf U}}
\newcommand{\Wbf}{{\bf W}}
\newcommand{\xbf}{{\bf x}}
\newcommand{\Xbf}{{\bf X}}
\newcommand{\Ybf}{{\bf Y}}
\newcommand{\Zbf}{{\bf Z}}
\newcommand{\pibf}{\mbox{\mathversion{bold}{$\pi$}}}
\newcommand{\Sigmabf}{\mbox{\mathversion{bold}{$\Sigma$}}}
\newcommand{\gammabf}{\mbox{\mathversion{bold}{$\gamma$}}}
\newcommand{\mubf}{\mbox{\mathversion{bold}{$\mu$}}}
\newcommand{\nubf}{\mbox{\mathversion{bold}{$\nu$}}}
\newcommand{\Thetabf}{\mbox{\mathversion{bold}{$\Theta$}}}
\newcommand{\thetabf}{\mbox{\mathversion{bold}{$\theta$}}}
\newcommand{\BP}{\text{BP}}
\newcommand{\EM}{\text{EM}}
\newcommand{\VEM}{\text{VEM}}
\newcommand{\VBEM}{\text{VB}}
\newcommand{\cst}{\text{cst}}
\newcommand{\obs}{\text{obs}}
\newcommand{\ra}{\emphase{${\bf \rightarrow}$~}}
\newcommand{\QZ}{Q_{\Zbf}}
\newcommand{\Qt}{Q_{\thetabf}}

\newcommand{\vecOne}{\begin{array}{c} 1\\ \vdots\\ 1\end{array}}
\newcommand{\vecZero}{\begin{array}{c} 0\\ \vdots\\ 0\end{array}}

%====================================================================
\title[Statistical models for count data]{Some statistical models for
  count data analysis}

\author{S. Robin}

\institute[AgroParisTech / INRA]{AgroParisTech / INRA \\
  \bigskip
  \begin{tabular}{ccccc}
    \epsfig{file=/media/donnees/RECHERCHE/RESEAUX/EXPOSES/FIGURES/LogoINRA-Couleur.ps, 
    width=2.5cm} & 
    \hspace{.5cm} &
    \epsfig{file=/media/donnees/RECHERCHE/RESEAUX/EXPOSES/FIGURES/logagroptechsolo.eps, 
    width=3.75cm} & 
    \hspace{.5cm} &
    \epsfig{file=/media/donnees/RECHERCHE/RESEAUX/EXPOSES/FIGURES/logo-ssb.eps,
    width=2.5cm} \\ 
  \end{tabular} \\
  \bigskip
  }

\date[NGS, Jan 2012, Lyon]{De la production � l'analyse des donn�es NGS,
  January 2012, Lyon}

%====================================================================
%====================================================================
\begin{document}
%====================================================================
%====================================================================

%====================================================================
\frame{\titlepage}

%====================================================================
\frame{\frametitle{Outline}
  \tableofcontents
  }

%====================================================================
%====================================================================
\section{A statistical model: what for?}
\frame{\frametitle{A statistical model: what for?} \Pause
  \paragraph{Aim of an experiment:} answer to a biological question

  \Pause \bigskip 
  \paragraph{Results of an experiment:} (numerous, numerical)
  measurements
  
  \Pause \bigskip 
  \paragraph{Model:} mathematical formula that relates the experimental
  conditions and the observed measurements (response) 
  
  \Pause \bigskip 
  \paragraph{(Statistical) modelling:} translating a biological
  question into a mathematical model \Pause (\emphase{\ra $\neq$ PIPELINE!})

  \Pause \bigskip 
  \paragraph{Statistical model:} mathematical formula involving
  \begin{itemize}
  \item the \emphase{experimental conditions},  
  \item the \emphase{biological response},  
  \item the \emphase{parameters} that describe the influence of the
    conditions on the (mean, theoretical) response,  
  \item and a description of the (technical, biological)
    \emphase{variability}.
  \end{itemize}
  }

%====================================================================
\subsection*{A general framework}
\frame{\frametitle{A general framework}
  
  \paragraph{Linear models:} an old, flexible, well established framework.  
  $$
  \begin{tabular}{cc}
    \hspace{-.5cm} \Pause
    \begin{tabular}{p{.45\textwidth}}
      \paragraph{Most common setting:}
      \begin{itemize}
      \item \emphase{Continuous} (Gaussian) measurements (e.g.
        \emphase{microarray} signal),
      \item with arbitrary experimental conditions (covariates, factor),
      \item and (reasonable) \emphase{dependency} between
        observations.
      \end{itemize}
    \end{tabular}
    & 
    \hspace{-.5cm} \Pause
    \begin{tabular}{p{.45\textwidth}}
      \paragraph{Generalisation:}
      \begin{itemize}
      \item \emphase{Almost arbitrary} (binary, count) measurements
        (e.g. \emphase{NGS}),
      \item with arbitrary experimental conditions (covariates, factor),
      \item and (\emphase{rather simple}) dependency between
        observations.
      \end{itemize}
    \end{tabular}
  \end{tabular}
  $$
  
  \Pause
  \paragraph{(Commercial and open-source) softwares} provide packages
  and procedures for the inference for such models 
  ... at least for \emphase{most standard designs}.
  }
  
%====================================================================
\frame{\frametitle{An example} \Pause
  
  \paragraph{Response $Y$:} 
  \begin{itemize}
  % \item abundance of a given microbial species measured by RNA 16S arrays,
  % \item abundance of a given microbial species measured by NGS,
  % \item (or expression level of a given gene).
  \item expression level of a given gene measured with microarray,
  \item expression level of a given gene measured with RNAseq,
  \item (or abundance of a given species).
  \end{itemize}
  
  \Pause \bigskip 
  \paragraph{Conditions $\xbf$:} 
  \begin{itemize}
  \item medium ($A, B$), 
  \item dose (or concentration: $0, 1, ..., 4$),
  \item sample preparation ($1, 2$).
  \end{itemize}

  \Pause \bigskip 
  \paragraph{Questions:} 
  \begin{enumerate}
  % \item Does the species have the same abundance in the two
  %   mediums at maximal dose?
  % \item Does the species have same response dynamics to the dose in
  %   the two mediums?
  \item Does the gene have the same expression level in the two
    mediums at maximal dose?
  \item Does the gene have same response dynamics to the dose in
    the two mediums?
  \item (Does the sample preparation affect the measurements?)
  \end{enumerate}
  }
  
%====================================================================
\frame{\frametitle{An example (cont'd)}

  \paragraph{Data:} Series of $n$ couples 
  $
  \qquad (\xbf_i, Y_i), 
  \quad i = 1..n.
  $  
  \bigskip 
  \begin{tabular}{cc}
     \Pause \hspace{-.5cm}
    \begin{tabular}{p{.45\textwidth}}
      \includegraphics[clip=,
      width=.45\textwidth]{../FIGURES/MICA-Ancova.eps} \\
      Medium: \textcolor{red}{$A$} / \textcolor{green}{$B$},
      Prep.: $\vartriangle / \triangledown$.
    \end{tabular}
    &
    \hspace{-.75cm}
    \begin{tabular}{p{.5\textwidth}}      
      \Pause\paragraph{Statistical model:}
      \begin{itemize}
      \item Mean (or 'theoretical') response for condition $\xbf$:
        $$
        \Esp(Y) = \mu(\xbf, \thetabf),
        $$
      \item \Pause Observed response at experiment
        ${\emphase{i}}$:
        $$
        Y_{\emphase{i}} \sim \Fcal[\mu(\xbf_{\emphase{i}}, \thetabf)].
        $$
      \end{itemize}
      
      \Pause\paragraph{Aim:} Estimate $\thetabf$ or
      some function of it.
    \end{tabular}
  \end{tabular}
  }

%====================================================================
\frame[containsverbatim]{\frametitle{An example (cont'd)}

  \paragraph{Data format.}  

  \bigskip 
  \begin{tabular}{cc}
    \hspace{-.5cm}
    \begin{tabular}{p{.5\textwidth}}
      \paragraph{One row = one experimental unit:} \\
      e.g. one culture, \\
      in a given medium, \\
      at a given dose, \\
      with a given preparation.

      \bigskip
      \paragraph{One column = one variable:} \\
      conditions (medium, dose, prep.) \\
      % and response (e.g. abundance).
      and response (e.g. expression).
    \end{tabular}
    &
    \hspace{-1cm}
    \begin{tabular}{p{.5\textwidth}}    
\begin{verbatim}
"i" "Med" "Prep" "Dose" "Resp" 
"1"    1     1     0    0.358
"2"    1     1     1    3.712
"3"    1     1     2   10.113
"6"    1     1     0    1.330
"7"    1     1     1    3.822
"8"    1     1     2   11.268
"29"   1     2     3   18.756
"45"   2     1     4   35.297
\end{verbatim}
    \end{tabular}
  \end{tabular} 
  
  \bigskip\bigskip
  Not optimal in terms of space storage ... but required by most softwares.

  }

% %====================================================================
% \frame{\frametitle{An example: 3 models}
%   \paragraph{Additive effects.} The respective effects of the medium ($m$),
%   dose ($d$) and preparation ($p$) \emphase{simply add} to each other:
%   $$
%   \Esp(Y_{i = m, d, p}) = \alpha_m + \beta_d + \delta_p
%   \qquad (\text{or } \mu + \alpha_m + \beta_d + \delta_p).
%   $$

%   \bigskip\bigskip\Pause
%   \paragraph{Model with interaction.} The dose effect \emphase{depends
%     on the medium} (and conversely):
%   $$
%   \Esp(Y_{i = m, d, p}) = \alpha_m + \beta_d + \gamma_{md} + \delta_p.
%   $$

%   \bigskip\bigskip\Pause
%   \paragraph{Dose as a continuous covariate.} The effect of the dose
%   has a \emphase{specific form} (e.g. quadratic) with medium-specific
%   coefficients:
%   $$
%   \Esp(Y_{i = m, d, p}) = \alpha_m + b_m d  + c_m d^2 + \delta_p.
%   $$
%   }

%====================================================================
\frame{\frametitle{An example: 3 models}


  \begin{tabular}{cc}
    \hspace{-.5cm}
    \begin{tabular}{p{.5\textwidth}}
      \onslide+<1->{
        \paragraph{Response $Y = f(\text{dose } d)$.} \\ 
        \includegraphics[clip=,
        width=.45\textwidth]{../FIGURES/MICA-Ancova.eps} \\
        Medium ($m$): \textcolor{red}{$A$} / \textcolor{green}{$B$}, \\
        Prep. ($p$): $\vartriangle / \triangledown$. \\
        ~\\ ~\\
      }
    \end{tabular}
    &
    \hspace{-1.25cm}
    \begin{tabular}{p{.5\textwidth}}
      \onslide+<2->{
        \paragraph{Additive model.} Effects {simply add}:
        $$
        \Esp(Y_{i = m, d, p}) = \alpha_m + \beta_d + \delta_p.
        $$
        % $
        % (\text{or } \mu + \alpha_m + \beta_d + \delta_p).
        % $ \\ ~\\
      }

      \onslide+<3->{
        \smallskip\Pause
        \paragraph{Model with interaction} between the dose and the medium:
        $$
        \Esp(Y_{i = m, d, p}) = \alpha_m + \beta_d +
        \emphase{\gamma_{md}} + \delta_p. 
        $$
      }
      
      \onslide+<4->{
        \smallskip\Pause
        \paragraph{Dose as a covariate.} The dose effect has
        {specific form}: 
        $$
        \Esp(Y_{i = m, d, p}) = \alpha_m + \emphase{b_m d  + c_m d^2} + \delta_p.
        $$ \\
        }
    \end{tabular} 
  \end{tabular}
  }

%====================================================================
%====================================================================
\section{Linear models}
\subsection*{General definition}
\frame{\frametitle{Linear models: Regression, Analysis of variance,
    etc.} \Pause
  
  \paragraph{'Linear':} the \emphase{mean response} $\mu$ can be
  written as a scalar product
  $$
  \emphase{\mu(\xbf, \thetabf) = \xbf . \thetabf}.
  $$
  \Pause 
  \vspace{-.4cm}\hspace{-.4cm}$
  \begin{array}{llc}
    \text{Model} & \hspace{-.4cm} \xbf \text{ (medium $A$, dose $d=3$,
    prep. $2$)} & \hspace{-.25cm} \thetabf \\ 

    \vspace{-.4cm}
    \begin{array}{l}
      \text{\emphase{Additive effects}} \\
      \alpha_m + \beta_d + \delta_p 
    \end{array}
    & \hspace{-.4cm} [\underset{medium}{\underbrace{1\;0}} \;
    \underset{dose}{\underbrace{0\;0\;0\;1\;0}} \;
    \underset{prep.}{\underbrace{0\;1}}] 
    & \hspace{-.25cm} \left[ \begin{array}{c} (\alpha_m) \\ (\beta_d) \\
        (\delta_p) \end{array} \right] \\
    \\

    \vspace{-.4cm}
    \Pause \begin{array}{l}
      \text{\emphase{Interaction}} \\
      \alpha_m + \beta_d + \gamma_{md} + \delta_p 
    \end{array}
    & \hspace{-.4cm} [\underset{medium}{\underbrace{1\;0}} \;
    \underset{dose}{\underbrace{0\;0\;0\;1\;0}} \;
    \underset{medium*dose}{\underbrace{0\;0\;0\;1\;0\;0\;0\;0\;0\;0}} \;
    \underset{prep.}{\underbrace{0\;1}}] 
    & \hspace{-.25cm} \left[ \begin{array}{c} (\alpha_m) \\ (\beta_d) \\
        (\gamma_{md}) \\ (\delta_p) \end{array} \right] \\
    \\
    
    \Pause \begin{array}{l}
      \text{\emphase{Polynomial}} \\
      \alpha_m + b_m d + c_m d^2 + \delta_p 
    \end{array}
    & \hspace{-.4cm} [\underset{medium}{\underbrace{1\;0}} \;
    \underset{dose}{\underbrace{3\;0}} \;
    \underset{dose^2}{\underbrace{9\;0}} \;
    \underset{prep.}{\underbrace{0\;1}}] 
    & \hspace{-.25cm} \left[ \begin{array}{c} (\alpha_m) \\ (b_m) \\ (c_m) \\
        (\delta_p) \end{array} \right] \\
  \end{array}
  $
  }

%====================================================================
\frame{\frametitle{Contrasts of interest}

  \paragraph{Model:} $m=$ medium, $d=$ dose, $p=$ preparation
  $$
  \mu(\xbf, \thetabf) = \alpha_m + \beta_d + \gamma_{md} + \delta_p 
  $$
  
  \Pause \bigskip 
  \paragraph{Aim = answering questions} based on the value of $\thetabf$
  \begin{enumerate}
  % \item \Pause Does the species have the same abundance in the two
  %   mediums at dose $d_{\max}$?
  \item \Pause Does the gene have the same expression level in the two
    mediums at dose $d_{\max}$ (with same preparation $p$)?
    $$
    (\alpha_A + \gamma_{A, d_{\max}}) - (\alpha_B + \gamma_{B, d_{\max}})
    \overset{?}{=} 0 
    $$
  % \item \Pause Does the species have same response dynamics to the dose in
  %   the two mediums (at dose $d$)?
  \item \Pause Does the gene have same response dynamics to the dose in
    the two mediums (at dose $d$,  with same preparation $p$)?
    $$
    (\gamma_{A,d+1} - 2 \gamma_{A,d} + \gamma_{A,d-1}) -
    (\gamma_{B,d+1} - 2 \gamma_{B,d} + \gamma_{B,d-1}) \overset{?}{=}
    0
    $$
  \item \Pause Does the sample preparation affect the measurements?
    $$
    \delta_1 - \delta_2 \overset{?}{=} 0
    $$
  \end{enumerate}

  }

%====================================================================
\frame{\frametitle{Contrasts of interest in the polynomial model}

  \paragraph{Model:} $m=$ medium, $d=$ dose, $p=$ preparation
  $$
  \mu(\xbf, \thetabf) =  \alpha_m + b_m d + c_m d^2 + \delta_p 
  $$
  
  \Pause \bigskip 
  \paragraph{Aim = answering questions} based on the value of $\thetabf$
  \begin{enumerate}
  % \item \Pause Does the species have the same abundance in the two
  %   mediums at dose $d_{\max}$?
  \item \Pause Does the gene have the same expression level in the two
    mediums at dose $d_{\max} = 4$ (with same preparation $p$)?
    $$
    (\alpha_A + 4 b_A + 16 c_A) - (\alpha_B + 4 b_B + 16 c_B)
    \overset{?}{=} 0 
    $$
  % \item \Pause Does the species have same response dynamics to the dose in
  %   the two mediums (at dose $d$)?
  \item \Pause Does the gene have same response dynamics to the dose in
    the two mediums (with same preparation $p$)?
    $$
    c_A - c_B \overset{?}{=} 0 
    \qquad \qquad \text{(at any dose $d$)}.
    $$
  \item \Pause Does the sample preparation affect the measurements?
    $$
    \delta_1 - \delta_2 \overset{?}{=} 0
    $$
  \end{enumerate}

  }

%====================================================================
\frame{\frametitle{Most classical (and simplest) setting}
  
  The responses $(Y_i)$ are \emphase{independent} with \emphase{same
    variance} $\sigma^2$.
  
  \Pause \bigskip 
  The experimental conditions and the responses can be gathered into
  $$
  \Xbf = \left[ \begin{array}{c} \xbf_1 \\ \vdots \\ \xbf_n
    \end{array} \right] =  \text{\emphase{design matrix}}, 
  \qquad
  \Ybf = \left[ \begin{array}{c} Y_1 \\ \vdots \\ Y_n
    \end{array} \right] = \text{\emphase{response vector.}}
  $$
  \Pause The model can then rewritten as (with $\Esp(\Ebf) = \Obf$,
  $\Var(\Ebf) = \sigma^2 \Ibf$) 
  $$
  \underset{\emphase{response}}{\Ybf} =
  \underset{\emphase{design}}{\Xbf} .
  \underset{\emphase{parameter}}{\thetabf } +
  \underset{\emphase{variability}}{\Ebf}. 
  $$

  \Pause \bigskip 
  \paragraph{Least square estimate.} \qquad
  $
  \widehat{\thetabf} = (\Xbf' \Xbf)^{-1} \Xbf' \Ybf.
  $

  }

%====================================================================
\frame{\frametitle{Experimental design}

  \paragraph{Precision of the estimates:}
  $$
  \Var(\widehat{\thetabf}) = \emphase{(\Xbf' \Xbf)^{-1}} \sigma^2
  $$
  depends neither on $\Ybf$, nor on $\thetabf$.

  \Pause \bigskip 
  \paragraph{Variance of a contrast:} $\phi = (\alpha_A + \gamma_{Ad_{\max}})
  - (\alpha_B + \gamma_{Bd_{\max}}) = \cbf' \thetabf$, 
  \begin{eqnarray*}
    \text{where } \cbf' & = &
%     \phi & = & (\alpha_A + \gamma_{Ad_{\max}}) - (\alpha_B + \gamma_{Bd_{\max}}) \\
    [\underset{medium}{\underbrace{1\;-1}} \;
    \underset{dose}{\underbrace{0\;0\;0\;0\;0}} \;
    \underset{medium*dose}{\underbrace{0\;0\;0\;0\;1\;0\;0\;0\;0\;-1}} \;
    \underset{prep.}{\underbrace{0\;0}}] . \thetabf \\
%     \Rightarrow \qquad \Var(\widehat{\phi)} & = & \emphase{\cbf' (\Xbf'
%     \Xbf)^{-1} \cbf} \; \sigma^2 
%     \text{or } \cbf' & = &
% %     \phi & = & (\alpha_A + \gamma_{Ad_{\max}}) - (\alpha_B + \gamma_{Bd_{\max}}) \\
%     [\underset{medium}{\underbrace{0\;0}} \;
%     \underset{dose}{\underbrace{0\;0\;0\;0\;0}} \;
%     \underset{medium*dose}{\underbrace{1\;-2\;1\;0\;0\;-1\;2\;1\;0\;0} \;
%     \underset{prep.}{\underbrace{0\;0}}] . \thetabf \\
    \Rightarrow \qquad \Var(\widehat{\phi)} & = & \emphase{\cbf' (\Xbf'
    \Xbf)^{-1} \cbf} \; \sigma^2 
  \end{eqnarray*}
  \Pause
  \begin{itemize}
  \item The \emphase{power of a test depends on the precision} of the
    estimate $\widehat{\phi}$.
  \item The experimental design $\Xbf$ \emphase{can be optimised for
    the contrast of interest}.
  \item A same design $\Xbf$ is, in general, \emphase{not
    optimal for different contrasts}. 
  \end{itemize}
  }

%====================================================================
%====================================================================
\subsection*{Continuous data}
\frame{\frametitle{Continuous data: Gaussian linear models} \Pause

  \paragraph{Continuous response:} e.g. (log-)intensity measured in a
  microarray experiment.

  \Pause \bigskip 
  \paragraph{Independent homoscedastic Gaussian setting.} For experiment $i$:
  $$
  (Y_i) \text{ independent,} \qquad Y_i \sim \Ncal(\xbf_i.\thetabf,
  \sigma^2),
  $$
  comprises linear regression, analysis of variance, etc.

  \Pause \bigskip 
  \paragraph{Most comfortable setting:}
  \begin{itemize}
  \item \emphase{Explicite} estimates of the parameters (maximum
    likelihood = least squares, no numerical optimisation);
  \item \emphase{Exact} confidence intervals and tests (not asymptotic);
  \item \emphase{Robust} to (reasonable) non-normality and
    non-homogeneity of the variance.
  \end{itemize}
  }

%====================================================================
\frame[containsverbatim]{\frametitle{Practical implementation with R}
  
  All (reasonable) statistical softwares include Gaussian linear model
  inference as a basic option.

  \bigskip %\Pause
  \paragraph{Model:}
  $$
  \mu(\xbf, \thetabf) = \alpha_m + \beta_d + \gamma_{md} + \delta_p.
  $$
  \paragraph{Call:} \\
{\tt 
\emphase{lm}(\emphase{Response $\sim$ Medium + Dose +
  Medium:Dose + Prep}) \\ 
}

\bigskip %\Pause
\paragraph{Result:} 
\begin{verbatim}
Coefficients:
  (Intercept)        Medium2          Dose1          Dose2  
       1.5791        -0.4794         2.2577         8.6825  
        Dose3          Dose4          Prep2  Medium2:Dose1  
      17.9051        29.7461        -0.6982         1.1839  
Medium2:Dose2  Medium2:Dose3  Medium2:Dose4  
       2.0200         3.0369         4.0448  
\end{verbatim}
  }

%====================================================================
\frame[containsverbatim]{\frametitle{Practical implementation with SAS}
  
{\tt
proc \emphase{GLM}; \\
\qquad class Medium Dose Prep; \\
\qquad model \emphase{Response = Medium Dose Medium*Dose Prep}; \\
\qquad estimate 'DoseMax' Medium 1 -1 \\
\qquad \qquad \qquad Medium*Dose 0 0 0 0 1 0 0 0 0 -1; \\
\qquad estimate 'TakeOff' Medium*Dose 1 -2 1 0 0 -1 2 -1 0 0; \\
} 
%\bigskip %\Pause 
\paragraph{Result:} 
\begin{verbatim}
Source      DF     Type I SS   Carr� moyen        F   Pr > F
Medium       1     37.339135     37.339135    90.02   <.0001
Dose         4   8114.016004   2028.504001  4890.59   <.0001
Medium*Dose  4     29.719231      7.429808    17.91   <.0001
Prep         1      7.313087      7.313087    17.63   0.0001
------------------------------------------------------------
Param�tre   Estimation      Ecart type    Test t    Pr > |t|
DoseMax    -3.56543357      0.37183181     -9.59      <.0001
TakeOff     0.34774687      0.91079820      0.38      0.7043
\end{verbatim}
  }

%====================================================================
\frame{\frametitle{Data transformation}
  
  As this setting is so comfortable, we may try to force the data to
  enter the homoscedastic Gaussian framework.

  \Pause \bigskip 
  \paragraph{Variance stabilisation.} Transformations exist for many situations
  $$
  \begin{array}{rlll}
%     \text{Variance heterogeneity} & \qquad & \text{Data transformation} \\
%     \hline
    \Var(Y) \propto & \Esp(Y) & \quad\rightarrow\quad & \widetilde{Y} =
    \sqrt{Y} \\ 
     \\
    \sqrt{\Var(Y)} \propto & \Esp(Y) & \quad\rightarrow\quad &
    \widetilde{Y} = \log{Y} \\ 
     \\
    Y = & \text{ proportion} & \quad\rightarrow\quad & \widetilde{Y} = \arcsin
    \sqrt{Y} \\  
    \\
    \text{Box-Cox} & & \quad\rightarrow\quad & \widetilde{Y} =
    (Y^\lambda - 1) / \lambda  \Pause
  \end{array}
  $$
  ... but the interpretability of the results may not be easy.
  }

%====================================================================
%====================================================================
\subsection*{Count data}
\frame{\frametitle{Count data: Generalised linear models} \Pause

  Linear models can be generalised to
  other distributions than Gaussian
  $$
  \rightarrow \text{General\emphase{ised} Linear Model } \qquad
  (\neq \text{General Linear Model}).
  $$
  
  \Pause \bigskip 
  \paragraph{General framework.} 
  The standard methods refer to distributions from the
  \emphase{exponential family}, which includes the most commonly used:
  $$
  \begin{tabular}{lll}
    Gaussian, & \qquad & Binomial, \\
    Poisson, & & Exponential, \\
    Gamma, & & Negative binomial, \\
    etc.
  \end{tabular}
  $$

  }

%====================================================================
\frame{\frametitle{Link function}

  \paragraph{Constraint on the mean.} For many distributions, the mean
  parameter cannot take any real value, e.g.:
  $$
  \text{Bernoulli: } \mu \in [0, 1], 
  \qquad
  \text{Poisson: } \mu \geq 0.
  $$
  The direct modelling $\mu = \xbf \thetabf$ may lead to
  \emphase{meaningless estimates}.

  \bigskip \Pause \bigskip 
  \paragraph{Modelling the mean.} We assume that a given (link)
  function of the mean is linear:
  $$
  \emphase{g}[\mu(\xbf, \thetabf)] = \xbf. \thetabf, 
  \qquad \Leftrightarrow \qquad
  \mu(\xbf, \thetabf) = \emphase{g^{-1}}(\xbf. \thetabf).
  $$
  \Pause \begin{itemize}
  \item The \emphase{link function does matter for the interpretation}
    of the results.  
  \item It is the price to pay to preserve the
    \emphase{flexibility of linear models} and the
    \emphase{statistical guaranty} for the inference. 
  \end{itemize}
  }

%====================================================================
\frame{\frametitle{Link function (cont'd)}

  \begin{tabular}{cc}
    \hspace{-.5cm}
    \begin{tabular}{p{.5\textwidth}}
      \paragraph{Binomial:} $\mu = $ 'success' probability. \Pause 
      $$
      \begin{array}{c}
        \text{logit}(\mu) = \log[\mu/(1-\mu)] = \xbf.\thetabf \\ 
        \includegraphics[clip=,width=.35\textwidth]{../FIGURES/MICA-Logit.eps}
        \\ \Pause 
        \mu = \exp(\xbf.\thetabf) / [1-\exp(\xbf.\thetabf)] \\ 
        \\ 
        \rightarrow \text{\emphase{'Odd-ratio'}}
        =\displaystyle{\frac{\pi_{Trt} / (1-\pi_{Trt})}{\pi_{Ref} /
        (1-\pi_{Ref})}}.   
      \end{array}
      $$
    \end{tabular}
    & 
    \hspace{-.5cm}\Pause 
    \begin{tabular}{p{.5\textwidth}}
      \paragraph{Poisson:} $\mu = $ 'intensity'. \Pause 
      $$
      \begin{array}{c}
        \log(\mu) = \xbf.\thetabf \\
        \includegraphics[clip=,width=.35\textwidth]{../FIGURES/MICA-Log.eps}
        \\ \Pause 
        \mu = \exp(\xbf.\thetabf) \\ 
        \\
        \rightarrow \text{\emphase{Multiplicative model}:} \\
        e^{\alpha_m + \beta_d + \delta_p}. 
        ~
      \end{array}
      $$
    \end{tabular}
  \end{tabular}
  }

%====================================================================
\frame{\frametitle{Statistical inference}

  \paragraph{Parameter estimates.} Most common strategy = Maximum
  likelihood (but also generalized least-squares).
  
  \Pause \bigskip \bigskip 
  \paragraph{General case.}
  \begin{itemize}
  \item \Pause \emphase{No explicite} estimate of the parameters \ra
    \emphase{Numerical optimisation} \\
    Newton-Raphson algorithm \emphase{will converge} (for the
    'canonic' link function); 
  \item \Pause \emphase{No exact} confidence intervals and tests \ra
    \emphase{Asymptotic results} \\
    Wald or likelihood ratio (LR) tests.
  \end{itemize}

  \Pause \bigskip \bigskip 
  \paragraph{Some simple cases.} e.g. two sample comparison for Poisson: \\
  \ra Both explicit estimates and exact tests can be derived.
  
  }

%====================================================================
\frame[containsverbatim]{\frametitle{Practical implementation}

  \begin{tabular}{ll}
  \paragraph{R:} & {\tt \emphase{glm}(Response $\sim$
    Medium + Dose + Medium:Dose + Prep,} \\ 
    & {\tt \qquad \emphase{family = poisson(link = "log")})} \\
  \paragraph{SAS:} & {\tt proc \emphase{GenMod};} \\
    & {\tt \qquad class Medium Dose Prep;} \\
    & {\tt \qquad model Response = Medium Dose Medium*Dose Prep} \\
    & {\tt \qquad \qquad \qquad / \emphase{dist = Poisson link = log};} \\
  \end{tabular}
  
%   \paragraph{Call with R:} \\
%   {\tt
%     \emphase{glm}(Response $\sim$ Medium+Dose+Medium:Dose+Prep, \\
%     \qquad     \emphase{family = poisson(link = "log")}) \\
%     }

%   \bigskip %\Pause 
%   \paragraph{Call with SAS:} \\
%   {\tt
%     proc \emphase{GenMod}; \\
%     \qquad class Medium Dose Prep; \\
%     \qquad model Response = Medium Dose Medium*Dose Prep \\
%     \qquad \qquad \qquad / \emphase{dist = Poisson link = log}; \\
%     }
  
  \bigskip %\Pause 
  \paragraph{Results with SAS:} 
\begin{verbatim}
          Algorithm converged.
     Scaled param.          1.2615
\end{verbatim}
\begin{verbatim}
Source              �cart        DF      Khi 2    Pr > Khi 2
Medium           689.3750         1       4.62        0.0317
../..
\end{verbatim}
\begin{verbatim}
Param.    Estimat   E-type   IC 95% (Wald)    Khi2  Pr>Khi2
DoseMax   -0.2364   0.1092  -0.4505 -0.0223   4.68   0.0304
TakeOff    0.6066   1.1066  -1.5624  2.7755   0.30   0.5836
\end{verbatim}
  }


%====================================================================
\frame{\frametitle{Practical use of generalised linear models}

  \paragraph{Modelling.} 
  \begin{itemize}
  \item Completely equivalent to the Gaussian case.
  \item Same way of thinking about the relevant contrasts.
  \end{itemize}

  \Pause \bigskip 
  \paragraph{Parameter estimation and inference.}
  \begin{itemize}
  \item Numerical optimisation (but softwares do the job).
  \item Asymptotic CI and tests \ra reasonably large samples.
  \end{itemize}

  \Pause \bigskip 
  \paragraph{Interpretation.}
  \begin{itemize}
  \item Contribution of each effect: equivalent to the Gaussian case.
  \item Idem for the significance of a contrast.
  \item The interpretation of parameters depends on the link function.
  \end{itemize}
  }

%====================================================================
%====================================================================
\section{Accounting for correlations}
\subsection*{Different sources of variability}
\frame{\frametitle{Different sources of variability} \Pause

  \begin{tabular}{cc}
    \hspace{-.5cm}
    \begin{tabular}{p{.5\textwidth}}
      \paragraph{An example.} 
      \begin{itemize}
      \item consider 3 {treatments}, 
      \item with 2 {individuals} per treatments ('biological' rep.),
      \item and 3 ('technical') replicates per individual.
      \end{itemize}

      \Pause \paragraph{Aim.}
      \begin{itemize}
      \item \emphase{Differences between the treatments?}
      \item Accounting for the \emphase{correlation between technical
          replicates}.
      \end{itemize}
    \end{tabular}
    & 
    \hspace{-.5cm} \Pause
    \begin{tabular}{c}
      \textcolor{red}{'biological'} / \textcolor{blue}{'technical'}
      variability \\
      \includegraphics[clip=,
      width=.5\textwidth]{../FIGURES/MICA-VarComp.eps} \\
    \end{tabular}
  \end{tabular}

  \Pause \bigskip 
  \paragraph{'Mixed' model:} 
  $
  \qquad Y_{Trt, Ind, Rep} = \alpha_{Trt} + \textcolor{red}{A_{Trt, Ind}} +
  \textcolor{blue}{E_{Trt, Ind, Rep}}.
  $
  }

%====================================================================
%====================================================================
\subsection*{Linear mixed models}
\frame{\frametitle{Gaussian mixed models}
 
  \paragraph{Modelling the variance.} The 
  $$
  \text{'biological' variance} = \gamma^2 \quad  \text{and 'technical'
    variance} = \sigma^2
  $$
  are called the \emphase{variance components}:
  $$
  \text{response variance } = \Var(Y_{Trt, Ind, Rep}) =
  \gamma^2 + \sigma^2.
  $$
  \Pause The hierarchical structure induces a
  \emphase{within-individual correlation}:
  $$
  \Corr(Y_{Trt, Ind, 1}, Y_{Trt, Ind, 2}) =  {\gamma^2}/{(\gamma^2 +
    \sigma^2)}. 
  $$

  \Pause \bigskip 
  \paragraph{'Mixed' models} involve both
  \begin{itemize}
  \item \emphase{fix} effects: medium, dose, treatment, ... preparation?,
  \item and \emphase{random} (Gaussian) effects: individual.
  \end{itemize}
  }

%====================================================================
\frame{\frametitle{Inference}

  \paragraph{Parameters.} In such a model, we need to both estimate
  \begin{itemize}
  \item \emphase{fix effects} parameters: $\thetabf$
  \item and \emphase{variance} parameter: $\gamma^2$, $\sigma^2$.
  \end{itemize}

  \Pause \bigskip 
  \paragraph{Estimation.} Several methods exist
  \begin{itemize}
  \item Maximum likelihood or Restricted maximum likelihood
    (\emphase{ReML}): general but \emphase{iterative, sometimes long}
    (but computers still do the job),
  \item Moments (e.g. Henderson: direct but \emphase{limited to some 
      models}).
  \end{itemize}

  \Pause \bigskip 
  \paragraph{Tests.}
  \begin{itemize}
  \item Exact for balanced designs,
  \item (Asymptotic) approximations are used in general,
  \item Depend on the experimental unit and on the reference
    variability.
  \end{itemize}
  }

%====================================================================
\frame[containsverbatim]{\frametitle{Practical implementation}
  \paragraph{Call with SAS:} \\
{\tt
proc \emphase{Mixed}; \\
\qquad class Trt Indiv; \\
\qquad model Response = Trt; \\
\qquad \emphase{random Indiv(Trt)}; \\
\qquad estimate 'Trt1-Trt2' Trt 1 -1 0; \\
}

\bigskip %\Pause 
\paragraph{Results with SAS:} 
\begin{verbatim}
           Estimation Method:        REML
              Convergence criteria met.
\end{verbatim}
\begin{verbatim}
             Parm cov       Estimation
             Indiv(Trt)         0.3456
             Residual          0.09990
\end{verbatim}
\begin{verbatim}
Param�tre   Estimat   Ecart type   Ddl   Test t    Pr > |t|
Trt1-Trt2   -5.5741       0.6156     3    -9.06      0.0028
\end{verbatim}
  }

%====================================================================
\frame{\frametitle{Tests in 'split-plot' designs}
  
  \paragraph{Design.} 2 genotypes, 5 individuals / genotype, 3 treatments,
  $n=30$
  $$
  \begin{tabular}{ll}
    Genotype 1 &   
    \begin{tabular}{|c|c|c|c|c|c|c|c|c|}
      \hline
      \multicolumn{3}{|c|}{Indiv. 1} & \multicolumn{3}{|c|}{\quad
      \dots \quad~} &
      \multicolumn{3}{|c|}{Indiv. 5} \\
      Trt 1 & Trt 2 & Trt 3 & \multicolumn{3}{|c|}{\quad
      \dots \quad~} & Trt 1 & Trt 2 & Trt 3 \\
      \hline
    \end{tabular} \\
%    & \\
    Genotype 2 &   
    \begin{tabular}{|c|c|c|c|c|c|c|c|c|}
      \hline
      \multicolumn{3}{|c|}{Indiv. 1} & \multicolumn{3}{|c|}{\quad
      \dots \quad~} &
      \multicolumn{3}{|c|}{Indiv. 5} \\
      Trt 1 & Trt 2 & Trt 3 & \multicolumn{3}{|c|}{\quad
      \dots \quad~} & Trt 1 & Trt 2 & Trt 3 \\
      \hline
    \end{tabular} 
  \end{tabular}
  $$

  \Pause \paragraph{Model.}   
  $$
  Y_{Gen,Ind,Trt} = \textcolor{red}{\alpha_{Gen} + A_{Gen, Ind}}
  \textcolor{blue}{ + \beta_{Trt} + \delta_{Gen,Trt} + E_{Trt,
      Ind, Rep}}.
  $$

  \Pause \paragraph{Tests.}
  $$
  \begin{tabular}{llcc}
    Effect & Exp. unit & Nb "obs" & Ref variance \\
    \hline
    Genotype & Individual & \emphase{10} & \emphase{$\gamma^2 +
    \sigma^2/3$} \\ 
    Treatment & Indiv*Treat & \emphase{30} & \emphase{$\sigma^2$}
  \end{tabular}
  $$
  \ra The two tests have different powers.
  }

%====================================================================
%====================================================================
\subsection*{Generalised linear mixed models}
\frame{\frametitle{Generalised mixed linear models} \Pause

  \paragraph{Random effects} can be introduced similarly, via the link
  function.

  \Pause \bigskip
  \paragraph{Count data: Poisson model.} 
  \begin{itemize}
  \item Design: Treatment / Individual / Replicate
  \item \emphase{Fix Treatment} effect $\alpha_{Trt}$,
  \item \emphase{Random Individual} effect $A_{Trt, Ind}$,
  \item Model: \emphase{Response $Y_{Trt, Ind, Red}$ is Poisson}
    distributed with mean
    \begin{eqnarray*}
    \mu_{Trt, Ind} & = & \exp(\alpha_{Trt} + A_{Trt, Ind}) \\
    \text{or} \qquad \mu_{Trt, Ind} & = & \exp(\alpha_{Trt}) \times
    G_{Trt, Ind} 
    \end{eqnarray*}
  \end{itemize}
  
  \Pause \bigskip
  \paragraph{Distribution of the random effects.} Chosen to combine
  well with the distribution of the response, e.g.:
  \begin{itemize}
  \item Gaussian response: Gaussian (additive) random effect,
  \item Poisson response: Gamma (multiplicative) random effect,
  \end{itemize}
  }

%====================================================================
\frame{\frametitle{Random effects}

  As in Gaussian linear mixed models, a random effect induces a
  \emphase{within individual correlation} and an
  \emphase{over-dispersion} of the data.
  $$
  \begin{tabular}{ccc}
    Poisson & \multicolumn{2}{c}{Poisson-Gamma = Negative Binomial} \\
    \includegraphics[clip=, width=0.3\textwidth,
    height=0.35\textheight, bbllx=0, bblly=210, bburx=594, bbury=630
    ]{../FIGURES/MICA-Poisson.eps} 
    & 
    \includegraphics[clip=, width=0.3\textwidth,
    height=0.35\textheight, bbllx=0, bblly=210, bburx=594,
    bbury=630]{../FIGURES/MICA-PoissonGamma.eps} 
    & 
    \includegraphics[clip=, width=0.3\textwidth,
    height=0.35\textheight, bbllx=0, bblly=210, bburx=594,
    bbury=630]{../FIGURES/MICA-PoissonGammaBis.eps} 
    \end{tabular}
  $$
  \vspace{-.5cm}
  \begin{itemize}
    \item \Pause \emphase{NGS data} often display over-dispersion and
    Negative Binomial is often used to account for it.      
    \item \Pause \emphase{Mixed model interpretation}: over-dispersion
    is due to the addition of \emphase{several sources of variations}
    (that could possibly be investigated).
  \end{itemize} 
  }

%====================================================================
\frame{\frametitle{Inference}
  
  The inference of generalised mixed linear models suffers the
  difficulty of all models presented above.

  \bigskip \Pause
  \paragraph{Parameter estimation.}
  \begin{itemize}
  \item No explicite estimates \ra numerical optimisation \\
    with few guaranty about the convergence of the algorithm \\
    and potentially long computational times.
  \item 'Scale' parameters ($\sim$ variances) are often difficult to
    estimate \\
    \ra preprocessing to get a prior estimate.
  \end{itemize}
  
  \bigskip \Pause
  \paragraph{Tests.}
  \begin{itemize}
  \item No exact tests \ra asymptotic approximations
  \item Standard approximations used in the Gaussian framework are not
    valid anymore
  \end{itemize}

%   \bigskip \Pause
%   \paragraph{Modelling flexibility:} 
%   \begin{itemize}
%   \item Choice of the link function
%   \item Choice of the distribution of the random effects.
%   \end{itemize}
  }

%====================================================================
\frame[containsverbatim]{\frametitle{Practical implementation}
  
  \paragraph{SAS:} \\
{\tt
proc \emphase{GLiMMix}; \\
\qquad class Trt Indiv; \\
\qquad model Response = Trt / \emphase{dist = Poisson link = log}; \\
\qquad \emphase{random Indiv(Trt)}; \\
%\qquad estimate 'Trt1-Trt2' Trt 1 -1 0; \\
} 

\bigskip %\Pause 
\paragraph{Results with SAS:} 
\begin{verbatim}
        Convergence criterion (PCONV=1.11022E-8) satisfied.
\end{verbatim}
\begin{verbatim}
             Parm cov       Estimation    Ecart type
             Indiv             0.07947       0.07574
\end{verbatim}
\begin{verbatim}
   Effet   Num DDL   Den DDL          F    Pr > F
   Trt           2         3      18.50    0.0205
\end{verbatim}
\begin{verbatim}
Param�tre   Estimat   Ecart type   Ddl   Test t    Pr > |t|
Trt1-Trt2   -1.3348       0.3451     3    -3.87      0.0306
\end{verbatim}
  }

%====================================================================
%====================================================================
\section{Multiple analyses}
\frame{\frametitle{Pangenomic analysis} \Pause

  \paragraph{High-throughput technologies} often address the same
  question for a \emphase{large number of entities:}
  \begin{itemize}
  \item differential expression of \emphase{all genes} from a
    given species,
  \item differential abundance of \emphase{all species} in a given
    medium.
  \end{itemize}

  \bigskip \Pause
  \paragraph{The same design and the same (linear) model} is then used
  for each entity (e.g. species), with specific parameter:
  % $$
  % \mu(\xbf_{Condition}, \thetabf^{\emphase{Species}})
  % $$
  % $$
  % \text{e.g.} \quad \mu(\xbf_{Condition}, \thetabf^{\emphase{Species}}) =
  % \alpha^{\emphase{Species}}_{Medium} + \beta^{\emphase{Species}}_{Dose} +
  % \gamma^{\emphase{Species}}_{Medium, Dose} + \delta^{\emphase{Species}}_{Prep}
  % $$
  $$
  \mu(\xbf_{Condition}, \thetabf^{\emphase{Gene}})
  $$
  $$
  \text{e.g.} \quad \mu(\xbf_{Condition}, \thetabf^{\emphase{Gene}}) =
  \alpha^{\emphase{Gene}}_{Medium} + \beta^{\emphase{Gene}}_{Dose} +
  \gamma^{\emphase{Gene}}_{Medium, Dose} + \delta^{\emphase{Gene}}_{Prep}
  $$

  \bigskip \Pause
  % \ra The same analysis is achieved '\emphase{species by species}'.
  \ra The same analysis is achieved '\emphase{gene by gene}'.
  }

%====================================================================
%====================================================================
\subsection*{Sharing information}
\frame{\frametitle{Sharing information}
  
  One may take advantage of the dimension to \emphase{better estimate
    'common' parameters}, especially when the number of experiments
  (arrays, runs) is small.

  \bigskip 
  \begin{description}
  \item[Normalisation.]  
    % \Pause E.g. $\delta^{\emphase{Species}}_{Prep}$ can be supposed to
    % be \emphase{the same for all species} or \emphase{smoothed}
    % (e.g. 'lowess' for microarray data). \\ 
    \Pause E.g. $\delta^{\emphase{Gene}}_{Prep}$ can be supposed to
    be \emphase{the same for all genes} or \emphase{smoothed}
    (e.g. 'lowess' for microarray data). \\ 
    \medskip
  \item[Variance modelling.] \Pause Idem for the residual variance
    $\sigma^2$ in differential gene expression studies:
    %  $
    %  \quad 
    %  \widehat{\sigma}^{\emphase{Cst}} \quad \text{or} \quad 
    %  \widehat{\sigma}^{\emphase{Species~group}} \quad \text{or} \quad 
    %  \widehat{\sigma}^{\emphase{Species}}.
    %  $ \\
      $
      \quad 
      \widehat{\sigma}^{\emphase{Cst}} \quad \text{or} \quad 
      \widehat{\sigma}^{\emphase{Gene\; group}} \quad \text{or} \quad 
      \widehat{\sigma}^{\emphase{Gene}}.
      $ \\
    \medskip
    \item[Dispersion parameter for Negative Binomial.] \Pause 
    %  $
    %  \widehat{\phi}^{\emphase{Cst}} \text{ or } 
    %  \widehat{\phi}^{\emphase{Species~group}} \text{ or }
    %  \widehat{\phi}^{\emphase{Species}}.  
    %  $
      $
      \widehat{\phi}^{\emphase{Cst}} \text{ or } 
      \widehat{\phi}^{\emphase{Gene\; group}} \text{ or }
      \widehat{\phi}^{\emphase{Gene}}.  
      $
  \end{description}

  \bigskip \Pause 
  \begin{itemize}
  \item Scale parameters (e.g. variance) are often difficult to
    estimate with few observations.
  \item The \emphase{power of the tests} in often improved,
    when regularising 'scale' parameters.
  \item \paragraph{Bayesian inference} typically encodes such a
    \emphase{regularisation} in the prior.  
  \end{itemize}
}

%====================================================================
%====================================================================
\subsection*{Multiple testing}
\frame{\frametitle{Multiple testing}

  \paragraph{General problem.} When testing a given contrast for each
  gene or species, one wants to control the number of \emphase{false
    positives}. 
  $$
  \begin{tabular}{cc}
    \hspace{-.5cm} \Pause
    \begin{tabular}{p{.45\textwidth}}
      \paragraph{Bonferroni, False Discovery Rate (FDR) procedures}
      rely on several statistical assumptions:
      \begin{itemize}
      \item model goodness-of-fit, 
      \item independence between the tests, 
      \item etc.
      \end{itemize}  
    \end{tabular}
    & 
    \hspace{-.5cm} \Pause
    \begin{tabular}{p{.5\textwidth}}
      \begin{tabular}{l}
        \quad Histogram of the $p$-values \\
        \\
        \includegraphics[clip=, width=0.5\textwidth,
        height=0.35\textheight, bbllx=70, bblly=210, bburx=530, bbury=630
        ]{../FIGURES/MICA-PvalIndep.eps} 
      \end{tabular}
    \end{tabular}
  \end{tabular}
  $$
  These assumptions must be \emphase{checked beforehand}.
  }

%====================================================================
\frame{\frametitle{Multiple testing under dependence}

  $$
  \begin{tabular}{cc}
    \hspace{-.5cm} 
    \begin{tabular}{p{.45\textwidth}}    
      \paragraph{Interactions between the genes or species} do cause 
      correlations between the tests.
      
      \bigskip \Pause
      \paragraph{Correlations} between the test can dramatically alter
      the performances of FDR-like procedures. 
    \end{tabular}
    & 
    \hspace{-.5cm} \Pause
    \begin{tabular}{p{.5\textwidth}}
      \begin{tabular}{l}
        Histogram of 
        correlated $p$-values \\
        \\
        \includegraphics[clip=, width=0.5\textwidth,
        height=0.35\textheight, bbllx=70, bblly=210, bburx=530, bbury=630
        ]{../FIGURES/MICA-PvalDep.eps} 
      \end{tabular}
    \end{tabular}
  \end{tabular}
  $$
  
  \Pause
  Such correlations (must and) can be accounted for, either in a
  '\emphase{black-box}' way, or based on \emphase{biological
    information} (e.g. KEGG).  }

%====================================================================
%====================================================================
\section{Segmentation}
\frame{\frametitle{Looking for change-points along the genome} \Pause 

  The question is not always 
  $$
  \text{'How does the response vary between conditions ?'}
  $$
  but sometimes
  $$
  \text{'How is the signal organized along the genome ?'}
  $$

  \Pause\bigskip\bigskip
  \paragraph{Re-annotation:} RNAseq experiments (or tiling arrays)
  \begin{itemize}
  \item give access to a high-resolution mapping of the transcribed
    regions,
  \item which allows to reconsider the existing annotation.
  \end{itemize}

  \Pause\bigskip
  \paragraph{Copy number variation (CNV):} DNAseq experiments (or SNP
  arrays)
  \begin{itemize}
  \item give access to a high-resolution quantification of the copy number,
  \item which allows to precisely determined the location of genomic
    alterations.
  \end{itemize}
}

%====================================================================
\frame{\frametitle{Segmentation} 

  \vspace{-.25cm}
  The problem can be stated in similar terms, whatever the
  technology.

  \smallskip
  \begin{tabular}{cc}
    \hspace{-.5cm}
    \begin{tabular}{p{.5\textwidth}}
      \Pause
      \paragraph{Tiling array data:} \\
      \includegraphics[clip=, bbllx=0, bblly=360, bburx=435, bbury=600,
      width=.35\textwidth, height=.3\textheight]{../FIGURES/CBerard-Evry-25}
    \end{tabular}
    & 
    \hspace{-.5cm}
    \begin{tabular}{p{.5\textwidth}}
      \Pause
      \paragraph{RNA-seq data: } \\
      \includegraphics[clip=, width=.35\textwidth,
      height=.3\textheight]{../FIGURES/geneYAL030W-ypdrep-1400-2500-raw} 
    \end{tabular}
  \end{tabular}
  
  \Pause \bigskip
  \paragraph{We have in mind that}
  \begin{itemize}
  \item the 'theoretical' signal $\lambda_t$ is affected by abrupt
    changes:
    $$
    \includegraphics[clip=, bbllx=90, bblly=300, bburx=540,
    bbury=400,
    width=.6\textwidth]{../../../RUPTURES/Exposes/Figures/FigSeg_Intro.eps}  
    $$
  \item the observed signal $Y_t$ is noisy, with 'mean' $\mu_k$ within
    segment $I_k$.
  \end{itemize}  
  }


%====================================================================
%====================================================================
\subsection*{Statistical model}
\frame{\frametitle{Statistical model (... as usual)} \Pause

  \paragraph{Data.}
  \begin{itemize}
  \item $t =$ position (nucleotide with NGS / probe with arrays);
  \item $Y_t =$ observed signal at position $t$
  \end{itemize}

  \Pause \bigskip
  \paragraph{Parameters.}
  \begin{itemize}
  \item $K = $ number of segments;
  \item $\tau_1, \tau_2, ..., \tau_{K-1} = $ change-points positions, \\
    $I_k = [\tau_{k-1}+1; \tau_k] = k$-th segment;
  \item $\mu_k =$ 'theoretical' (or 'mean') signal in segment $I_k$.
  \end{itemize}

  \Pause \bigskip
  \paragraph{(Proposed) Model.}
  \begin{itemize}
  \item $\{Y_t\}$ are independent (!!!);
  \item if position $t$ is in segment $I_k$:
    $$
    Y_t \sim \Fcal(\mu_k) \qquad \mbox{e.g.} \quad 
    \left\{
      \begin{array}{rclll}
        Y__t & \sim & \Pcal(\mu_k) & & \text{for NGS}, \\
        Y__t & \sim & \Ncal(\mu_k, \sigma^2) & & \text{for arrays}.
      \end{array}
    \right.
    $$
  \end{itemize}
   }

%====================================================================
%====================================================================
\frame{\frametitle{Segmentation as a linear model} 

  The segmentation model can be written as a linear model:
  $$
  \Esp(\Ybf) = \Tbf \mubf,  \Pause \quad
  \Tbf = \left[ \tiny{
    \begin{array}{rrrr}
      I_1 \left\{\vecOne\right.&\vecZero&&\vecZero\Pause\\ 
      \hdashline
      \vecZero&I_2 \left\{\vecOne\right.&&\vecZero\Pause\\ 
      \hdashline
      && \ddots & \\
      \hdashline
      \vecZero&\vecZero&&I_K \left\{\vecOne\right.\\
    \end{array} }
    % \begin{array}{cccc}
    %   I_1 \tiny{\left\{\begin{array}{c} 1\\ \vdots\\ 1\end{array}\right.}&&\\
    %   &I_2 \tiny{\left\{\begin{array}{c} 1\\ \vdots\\ 1\end{array}\right.}&\\
    %   && \ddots & \\
    %   &&&I_K \tiny{\left\{\begin{array}{c} 1\\ \vdots\\ 1\end{array}\right.}\\
    % \end{array} 
  \right], 
  \quad
  \mubf = \left[ \begin{array}{c}
      \mu_1 \\ \mu_2 \\ \vdots \\ \mu_K
      \end{array} \right] 
  $$

  \Pause 
  ...except that we do not know the matrix $\Tbf$, that is
  \begin{itemize}
  \item where the blocks $I_k$ start and end \ra algorithmic issue,
  \item how many blocks $I_k$ there are \ra statistical issue.
  \end{itemize}
  }

%====================================================================
\subsection*{Inference}
\frame{\frametitle{Algorithmic issue} \Pause 

  \paragraph{Statistical inference.}
  \begin{itemize}
  \item Estimating the parameters $\mubf$ for given segmentation $\Tbf$;
  \item Estimating the segmentation $\Tbf$ (i.e. the $\tau_k$'s) is the
    main issue.
  \end{itemize}

  \Pause \bigskip
  \paragraph{Segmentation.}
  \begin{itemize}
  \item We look for $\widehat{\Tbf}$ that provide the best fit (e.g. maximum
    likelihood);
  \item We can not explore all possible segmentations $\Tbf$:
    $$
    n=1\,000, \quad K = 20 \quad \longrightarrow \quad
    \binom{n-1}{K-1} \simeq 10^{40} \text{ possibilities}.
    $$
  \item \Pause Well known algorithms (dynamic programming) retrieve
    the \emphase{optimal segmentation $\widehat{\Tbf}$ with complexity
      $\Ocal(K n^2)$}.
  \item \Pause $\Ocal(K n^2)$ is \emphase{still too much} for NGS where
    $n$ ranges from $10^4$ to $10^8$.
  \end{itemize}
  }

%====================================================================
\frame{\frametitle{Statistical issue} 

  \begin{tabular}{cc}
    \hspace{-.5cm}
    \begin{tabular}{p{.38\textwidth}}
      \paragraph{How many segments?} \\
      The fit increases with $K$. 
      \begin{overprint}
        \onslide<2>
        \includegraphics[clip=, width=.3\textwidth,
        height=.4\textheight]
        {../FIGURES/geneYAL030W-ypdrep-1400-2500-J}
        \onslide<3->
        \includegraphics[clip=, width=.3\textwidth,
        height=.4\textheight]
        {../FIGURES/geneYAL030W-ypdrep-1400-2500-J-K}
      \end{overprint}
      \onslide+<3->{
        \paragraph{Penalized constrast:} add a penalty term to account
        for the complexity of the model: \\
        \smallskip
        $
        - \Lcal(\widehat{\Tbf}_K, \widehat{\mubf}_K) + c \times
        \text{pen}(K) 
        $
      }
    \end{tabular}
    & 
    \hspace{-.75cm}
    \onslide+<4>{
      \begin{tabular}{p{.58\textwidth}}
        The result \emphase{strongly depends} on the choice of
        $\widehat{K}$ \\
        \includegraphics[clip=, width=.6\textwidth,
        height=.6\textheight]
        {../FIGURES/geneYAL030W-ypdrep-1400-2500-seg} \\
        but also one the choice of distribution $\Fcal$. 
      \end{tabular}
    }  
  \end{tabular} 
  }

%====================================================================
%====================================================================
\subsection*{Extensions}
\frame{\frametitle{Extensions} 

  \paragraph{Accounting for other effects.} Thanks to the linear model
  formulation, segmentation can be combined with other linear effects:
  $$
  \Esp(\Ybf) = \Tbf \mubf \Pause + \Xbf \thetabf + \Zbf \Ubf \Pause
  $$
  to account for, say,
  \begin{itemize}
  \item fix local GC-contents effect, 
  \item random probe (tiling array) or patient effect, ...
  \end{itemize}

  \Pause \bigskip
  \paragraph{CGHseg R package:}   
  \begin{itemize}
  \item available at {\tt cran.r-project.org/web/packages/cghseg},
  \item segmentation of multiple profiles,
  \item with fixed and random effects,
  \item designed for microarrays (... but may deserve to be tried on
    NGS).
  \end{itemize}
  }


% %====================================================================
% %====================================================================
% \section{Conclusion}
% \frame{\frametitle{Linear models} \Pause

%   \paragraph{Linear models} provide a general framework to both plan the
%   experiments and analyse their results for a large category of
%   data-type and dependency structures.

%   \bigskip \Pause
%   \paragraph{Commercial and open-source softwares} allows one to
%   perform such analyses for a reasonable investment (time, money,
%   brain, ...) for count data.

%   \bigskip \Pause
%   \paragraph{... but} generalised linear model inference still raises
%   \emphase{several open problems}
%   \begin{itemize}
%   \item \emphase{efficient algorithms} for repeated inference high
%     throughput technology,
%   \item \emphase{accurate tests procedures} especially for complex
%     dependency structure (e.g. time-course, spatial data),
%   \item \emphase{models for between-test} dependence,
%   \item ...
%   \end{itemize} 
%   }

% %====================================================================
% \frame{\frametitle{NGS data} 

%   \paragraph{Species abundance, gene expression} analyses require the
%   prior knowledge of the species / genes  \\
%   \ra one measurement for each species/gene in each condition \\
%   \ra fits into the \emphase{linear model framework}

%   \Pause \bigskip
%   \paragraph{NGS with (few) prior knowledge:} looking for new genes
%   (RNA-seq), for protein-DNA interaction loci (ChIP-seq) \\
%   \ra distribution of reads along the genome \\
%   \ra \emphase{'point process'} $\notin$ classical linear model
%   frameworks

%   }

%====================================================================
%====================================================================
\section*{Announcement}
\subsection*{}
\frame{\frametitle{SMPGD 2012}
  
  \paragraph{Statistical Methods for Post-Genomic Data  - SMPGD'12:} 
  $$
  \begin{tabular}{c}
    26-27 January 2012,\\
    Universit� Lyon - Claude Bernard: \\
    \url{smpgd2012.univ-lyon1.fr}
  \end{tabular}
  $$

  \bigskip
  Sessions:
  \begin{itemize}
  \item NGS data analysis
  \item Ecology \& genomics
  \item Data integration
  \end{itemize} 
}

%====================================================================
{\tiny
  \bibliography{/Biblio/ARC,/Biblio/AST,/Biblio/SSB}
  \bibliographystyle{/media/donnees/LATEX/astats}
  %\bibliographystyle{plain}
  }

%====================================================================
%====================================================================
\end{document}
%====================================================================
%====================================================================

%====================================================================
\frame{\frametitle{}
  }
%====================================================================

  \begin{tabular}{cc}
    \hspace{-.5cm}
    \begin{tabular}{p{.5\textwidth}}
    \end{tabular}
    & 
    \hspace{-.5cm}
    \begin{tabular}{p{.5\textwidth}}
    \end{tabular}
  \end{tabular}
