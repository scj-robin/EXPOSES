\documentclass[9pt]{beamer}

% Beamer style
%\usetheme[secheader]{Madrid}
% \usetheme{CambridgeUS}
\useoutertheme{infolines}
\usecolortheme[rgb={0.65,0.15,0.25}]{structure}
% \usefonttheme[onlymath]{serif}
\beamertemplatenavigationsymbolsempty
%\AtBeginSubsection

% Packages
%\usepackage[french]{babel}
\usepackage[latin1]{inputenc}
\usepackage{color}
\usepackage{xspace}
\usepackage{dsfont, stmaryrd}
\usepackage{amsmath, amsfonts, amssymb, stmaryrd}
\usepackage{epsfig}
\usepackage{tikz}
\usepackage{url}
% \usepackage{ulem}
\usepackage{/home/robin/LATEX/Biblio/astats}
%\usepackage[all]{xy}
\usepackage{graphicx}
\usepackage{xspace}

\input{/home/robin/RECHERCHE/EXPOSES/LATEX/Commands}
\newcommand{\CTSBM}{{\sl ct}-SBM\xspace}
\newcommand{\DTSBM}{{\sl dt}-SBM\xspace}

% Directory
\newcommand{\fignet}{/home/robin/Bureau/RECHERCHE/RESEAUX/EXPOSES/FIGURES}
\newcommand{\figchp}{/home/robin/Bureau/RECHERCHE/RUPTURES/EXPOSES/FIGURES}
% \newcommand{\figfig}{../figs}
\newcommand{\figCMR}{/home/robin/Bureau/RECHERCHE/ECOLOGIE/CountPCA/sparsepca/Article/Network_JCGS/trunk/figs}
\newcommand{\figDoR}{/home/robin/Bureau/RECHERCHE/BAYES/VBEM-IS/VBEM-IS.git/Data/Tree/Fig}


%====================================================================
%====================================================================

%====================================================================
%====================================================================
\begin{document}
%====================================================================
%====================================================================

%====================================================================
\title[Network reconstruction with the PLN model]{Ecological network reconstruction from abundance data \\ using the Poisson log-normal model}

\author[S. Robin]{S. Robin \\ ~ \\
  {\sl joint work with J. Chiquet and M. Mariadassou}}

\institute[]{INRA / AgroParisTech /univ. Paris-Saclay}

\date[BCMicrobiome, Paris]{R�union BCMicrobiome, Paris, Oct. 2018}

%====================================================================
%====================================================================
\maketitle
%====================================================================

%====================================================================
\section*{Ecological network inference}
%====================================================================
\frame{\frametitle{Typical dataset}

\paragraph{Barents fish \refer{FNA06}:} $n = 89$ sites, $m = 30$ species, $d = 4$ covariates

\bigskip \pause
\begin{tabular}{c|c|c}
  \onslide+<2->{\paragraph{Abundances $Y$:} $n \times m$}
  & 
  \onslide+<3->{\paragraph{Covariates $X$:} $n \times d$}
  & 
  \onslide+<4>{\paragraph{Network $G$:} $m \times m$ }
  \\
  \hspace{-.02\textwidth} 
  \begin{tabular}{p{.3\textwidth}}
    \onslide+<2->{\begin{tabular}{rrr}
      {\sl Hi.pl} & {\sl An.lu} & {\sl Me.ae} 
      \footnote{{\sl Hi.pl}: Long rough dab, {\sl An.lu}: Atlantic wolffish, {\sl Me.ae}: Haddock} \\ 
%       Dab & Wolffish & Haddock \\ 
      \hline
      31  &   0  & 108 \\
       4  &   0  & 110 \\
      27  &   0  & 788 \\
      13  &   0  & 295 \\
      23  &   0  &  13 \\
      20  &   0  &  97 \\
      \vdots & \vdots & \vdots 
    \end{tabular}} 
  \end{tabular}
  & 
  \begin{tabular}{p{.3\textwidth}}
    \onslide+<3->{\begin{tabular}{rrr}
      Lat. & Long. & Depth \\ \hline
      71.10 & 22.43 & 349 \\
      71.32 & 23.68 & 382 \\
      71.60 & 24.90 & 294 \\
      71.27 & 25.88 & 304 \\
      71.52 & 28.12 & 384 \\
      71.48 & 29.10 & 344 \\
      \vdots & \vdots & \vdots 
    \end{tabular}} 
  \end{tabular}
  & 
  \hspace{-.02\textwidth} \pause
  \begin{tabular}{p{.3\textwidth}}
    \onslide+<4>{\hspace{-.1\textwidth} 
      \begin{tabular}{c}
        \includegraphics[width=.35\textwidth]{\figCMR/network_BarentsFish_Gfull_full60edges}
      \end{tabular}}
  \end{tabular}
\end{tabular}

\bigskip \bigskip 
\onslide+<4>{\paragraph{Goal:} infer $G$ based on $X$ and $Y$.}
}
  
%====================================================================
\section{Graphical models}
\frame{\frametitle{Outline} \tableofcontents[currentsection]}
%====================================================================
\frame{\frametitle{Network inference: a framework} 

  \bigskip
  \paragraph{'Interaction':} vague concept 
  \ra need for a probabilistic / statistical definition
  
  \pause\bigskip\bigskip
  \paragraph{Graphical models \refer{Lau96}:} The distribution $p(U) = p(U_{1}, \dots, U_m)$ is faithful to $G$ iff
  $$
  p(U) \propto \prod_{C \in \Ccal(G)} \psi_C(U_{C}),
  $$
  where $U_{C} = \{U_{j}: j \in C\}$ and $\Ccal(G) =$ set of maximal cliques of $G$.

  \pause\vspace{.1\textheight}
  \begin{tabular}{cc}
    \begin{tabular}{p{.35\textwidth}}
	 \begin{tikzpicture}
\node[observed] (U3) at (0*\edgeunit, 0*\edgeunit) {$U_3$};
\node[observed] (U1) at (-0.65*\edgeunit, 1.3*\edgeunit) {$U_1$};
\node[observed] (U2) at (.65*\edgeunit, 1.3*\edgeunit) {$U_2$};
\node[observed] (U4) at (1.5*\edgeunit, 0*\edgeunit) {$U_4$};
\draw[edge] (U1) to (U2); \draw[edge] (U1) to (U3); \draw[edge] (U2) to (U3); \draw[edge] (U3) to (U4);
\end{tikzpicture}


    \end{tabular}
    & 
%     \hspace{-.15\textwidth}
    \begin{tabular}{p{.55\textwidth}}
    $p(U) \propto \psi_1(U_1, U_2, U_3) \; \psi_2(U_3, U_4)$ \\~
	 \begin{itemize}
	 \item Connected graph: %\\
	 all variables are dependent \\~
	 \item $U_3 =$ separator: $U_4 \perp (U_1, U_2) \gv U_3$ \\~ \\~
	 \end{itemize}
    \end{tabular}
  \end{tabular} 
  }

%====================================================================
\frame{\frametitle{Gaussian graphical model (GGM)} 

  \paragraph{Gaussian setting:} 
  $$
  U \sim \Ncal_m(0, \Sigma)
  $$
  
  \bigskip
  \paragraph{Property:} the edges of $G$ correspond to the \emphase{non-zero terms of the precision matrix}
  $$
  \Sigma^{-1} = \Omega = [\omega_{jk}]_{j, k}
  \qquad \Rightarrow \qquad
  p(U) 
  \propto \exp\left(-\frac12 \sum_{j, k} \omega_{jk} U_jU_k\right)
  = \prod_{j, k} e^{-\omega_{jk} U_jU_k/2}
  $$
  
  \pause 
  \begin{tabular}{cc}
    \begin{tabular}{p{.35\textwidth}}
	 \begin{tikzpicture}
\node[observed] (U3) at (0*\edgeunit, 0*\edgeunit) {$U_3$};
\node[observed] (U1) at (-0.65*\edgeunit, 1.3*\edgeunit) {$U_1$};
\node[observed] (U2) at (.65*\edgeunit, 1.3*\edgeunit) {$U_2$};
\node[observed] (U4) at (1.5*\edgeunit, 0*\edgeunit) {$U_4$};
\draw[edge] (U1) to (U2); \draw[edge] (U1) to (U3); \draw[edge] (U2) to (U3); \draw[edge] (U3) to (U4);
\end{tikzpicture}


    \end{tabular}
    & 
    \hspace{-.05\textwidth}
    \begin{tabular}{p{.55\textwidth}}
      $$
      \text{Covariance} = \left( \begin{array}{cccc}
                      * & * & * & * \\
                      * & * & * & * \\
                      * & * & * & * \\ 
                      * & * & * & * 
                      \end{array} \right)
      $$
      $$
      \text{Precision} = \left( \begin{array}{cccc}
                        * & * & * & \emphase{0} \\
                        * & * & * & \emphase{0} \\
                        * & * & * & * \\ 
                        \emphase{0} & \emphase{0} & * & * 
                        \end{array} \right)
      $$
    \end{tabular} \\~ % \\~
  \end{tabular} 
  
  \pause %\bigskip
  \paragraph{Graphical lasso (Glasso) \refer{FHT08}:} penalized likelihood, forcing $\Omega$ to have many zeros
  }
  
%====================================================================
\section{Poisson log-normal (PLN) model}
\frame{\frametitle{Outline} \tableofcontents[currentsection]}
%====================================================================
\frame{\frametitle{A generic model for multivariate count data}

  Need for a model accounting for correlations between count data
  
  \pause \bigskip 
  \paragraph{Observed data} for site $i$
  \begin{itemize}
   \item $Y_{ij}$: abundance of species $j$ ($j = 1 \dots m$)
   \item $o_{ij}$: sampling effort for species $j$ (offset)
   \item $x_i$: vector of covariates
  \end{itemize}
  
  \pause \bigskip 
  \paragraph{Poisson log-normal (PLN) model:} for each site $i$
  \begin{align*}
    \text{latent:} & & Z_i & \sim \Ncal_m(0, \Sigma) \\
    \text{observed:} & & Y_{ij} \gv Z_{ij} & \sim \Pcal\left(\exp\left(\underset{\text{offset}}{\underbrace{o_{ij}}} + \underset{\text{covariates}}{\underbrace{x_i^\intercal \beta_j}} +  \underset{\text{random effect}}{\underbrace{Z_{ij}}}\right)\right)
  \end{align*}
  \pause
  \begin{itemize}
   \item $\Sigma$: dependency structure
   \item $\beta_j$: effects of the covariates on species $j$
  \end{itemize}

}
  
%====================================================================
\frame{\frametitle{PLN-network model}

  \paragraph{Constraint on $\Omega$:} 
  \begin{align*}
    Z_i & \sim \Ncal_m(0, \Omega^{-1}), \qquad \qquad \text{\emphase{$\Omega$ being sparse}}\\
    Y_{ij} \gv Z_{ij} & \sim \Pcal(e^{o_{ij} + x_i^\intercal \beta_j + Z_{ij}})
  \end{align*}
  
  \pause\bigskip\bigskip
  \paragraph{Remark.} Actually models the dependecy structure in the \emphase{latent} layer:
  \begin{overprint}
    \onslide<3>
    $$
    \begin{array}{ccc}
      \begin{tikzpicture}[scale=.8]
  \node[hidden] (Z1) at ( 0.95*\edgeunit,  0.31*\edgeunit) {$Z_1$};
  \node[hidden] (Z2) at (-0.00*\edgeunit,  1.00*\edgeunit) {$Z_2$};
  \node[hidden] (Z3) at (-0.95*\edgeunit,  0.31*\edgeunit) {$Z_3$};
  \node[hidden] (Z4) at (-0.59*\edgeunit, -0.81*\edgeunit) {$Z_4$};
  \node[hidden] (Z5) at ( 0.59*\edgeunit, -0.81*\edgeunit) {$Z_5$};
  
  \draw[edge] (Z1) to (Z5);  \draw[edge] (Z2) to (Z3);  
  \draw[edge] (Z2) to (Z4);  \draw[edge] (Z3) to (Z4); 

  \node[observed] (Y1) at ( 1.05*\edgeunit, -0.39*\edgeunit) {$Y_1$};
  \node[observed] (Y2) at (-0.00*\edgeunit,  0.30*\edgeunit) {$Y_2$};
  \node[observed] (Y3) at (-1.05*\edgeunit, -0.39*\edgeunit) {$Y_3$};
  \node[observed] (Y4) at (-0.59*\edgeunit, -1.51*\edgeunit) {$Y_4$};
  \node[observed] (Y5) at ( 0.59*\edgeunit, -1.51*\edgeunit) {$Y_5$};
  
  \draw[arrow] (Z1) to (Y1); 
  \draw[arrow] (Z2) to (Y2);
  \draw[arrow] (Z3) to (Y3);
  \draw[arrow] (Z4) to (Y4);
  \draw[arrow] (Z5) to (Y5);
  \end{tikzpicture}

    & \qquad \qquad &
      \begin{tikzpicture}[scale=.8]
  \node[observed] (Y1) at ( 0.95*\edgeunit,  0.31*\edgeunit) {$Y_1$};
  \node[observed] (Y2) at (-0.00*\edgeunit,  1.00*\edgeunit) {$Y_2$};
  \node[observed] (Y3) at (-0.95*\edgeunit,  0.31*\edgeunit) {$Y_3$};
  \node[observed] (Y4) at (-0.59*\edgeunit, -0.81*\edgeunit) {$Y_4$};
  \node[observed] (Y5) at ( 0.59*\edgeunit, -0.81*\edgeunit) {$Y_5$};
  \node[empty] (YY) at ( 0.59*\edgeunit, -1.51*\edgeunit) {};

  \draw[edge] (Y1) to (Y5);  \draw[edge] (Y2) to (Y3);  
  \draw[edge] (Y2) to (Y4);  \draw[edge] (Y3) to (Y4);  
  
  \end{tikzpicture}

    \end{array}
    $$
    \onslide<4>
    $$
    \begin{array}{ccc}
      \begin{tikzpicture}[scale=.8]
    \node[hidden] (Z1) at ( 0.95*\edgeunit,  0.31*\edgeunit) {$Z_1$};
    \node[hidden] (Z2) at (-0.00*\edgeunit,  1.00*\edgeunit) {$Z_2$};
    \node[hidden] (Z3) at (-0.95*\edgeunit,  0.31*\edgeunit) {$Z_3$};
    \node[hidden] (Z4) at (-0.59*\edgeunit, -0.81*\edgeunit) {$Z_4$};
    \node[hidden] (Z5) at ( 0.59*\edgeunit, -0.81*\edgeunit) {$Z_5$};
    
    \draw[edge] (Z1) to (Z2);  \draw[edge] (Z1) to (Z4);  
    \draw[edge] (Z1) to (Z5);  \draw[edge] (Z2) to (Z3);  
    \draw[edge] (Z2) to (Z4);  \draw[edge] (Z3) to (Z4); 

    \node[observed] (Y1) at ( 1.05*\edgeunit, -0.39*\edgeunit) {$Y_1$};
    \node[observed] (Y2) at (-0.00*\edgeunit,  0.30*\edgeunit) {$Y_2$};
    \node[observed] (Y3) at (-1.05*\edgeunit, -0.39*\edgeunit) {$Y_3$};
    \node[observed] (Y4) at (-0.59*\edgeunit, -1.51*\edgeunit) {$Y_4$};
    \node[observed] (Y5) at ( 0.59*\edgeunit, -1.51*\edgeunit) {$Y_5$};

    \draw[arrow] (Z1) to (Y1); 
    \draw[arrow] (Z2) to (Y2);
    \draw[arrow] (Z3) to (Y3);
    \draw[arrow] (Z4) to (Y4);
    \draw[arrow] (Z5) to (Y5);
  \end{tikzpicture}

    & \qquad \qquad &
      \begin{tikzpicture}[scale=.8]
  \node[observed] (Y1) at ( 0.95*\edgeunit,  0.31*\edgeunit) {$Y_1$};
  \node[observed] (Y2) at (-0.00*\edgeunit,  1.00*\edgeunit) {$Y_2$};
  \node[observed] (Y3) at (-0.95*\edgeunit,  0.31*\edgeunit) {$Y_3$};
  \node[observed] (Y4) at (-0.59*\edgeunit, -0.81*\edgeunit) {$Y_4$};
  \node[observed] (Y5) at ( 0.59*\edgeunit, -0.81*\edgeunit) {$Y_5$};
  \node[empty] (YY) at ( 0.59*\edgeunit, -1.51*\edgeunit) {};

  \draw[edge] (Y1) to (Y2);  \draw[edge] (Y1) to (Y3);  
  \draw[edge] (Y1) to (Y4);  \draw[edge] (Y1) to (Y5);  
  \draw[edge] (Y2) to (Y3);  \draw[edge] (Y2) to (Y4); 
  \draw[edge] (Y2) to (Y5);  \draw[edge] (Y3) to (Y4);  
  \draw[edge] (Y3) to (Y5);  \draw[edge] (Y4) to (Y5);  
  \end{tikzpicture}

    \end{array}
    $$
  \end{overprint}
}
  
%====================================================================
\frame{\frametitle{PLN-network inference}

  \bigskip
  \paragraph{Inference \refer{CMR18b}.} Combines a variational approximation \refer{WaJ08} with the graphical lasso:
  $$
  \max_{\theta, q} J_\lambda(\theta, q) 
   = \underset{\text{max. likelihood}}{\underbrace{\log p_\theta(Y)}} 
   - \underset{\text{variational approximation}}{\underbrace{KL\left(q(Z) \;||\; p_\theta(Z \gv Y)\right)}} 
   - \underset{\text{sparsity}}{\underbrace{\emphase{\lambda \|\Omega\|_1}}}
  $$
  \begin{itemize}
   \item Variational approximation: $Z_i \gv Y_i \approx \Ncal(\mt_i, \St_i)$
   \item $J_\lambda(\theta, q)$ bi-convex $\rightarrow$ efficient gradient ascent algorithm
  \end{itemize}

  \pause \bigskip \bigskip 
  \paragraph{R package {\tt PLNmodels}\footnote{\onslide+<2->{many thanks to Charlie Pauvert for beta-testing}}:} 
  \begin{itemize}
   \item available at \textcolor{blue}{\url{https://github.com/jchiquet/PLNmodels}}
   \item syntax: 
   $$
   \text{\tt PLNnetwork(Y \url{~} X + offset(O))}
   $$
  \end{itemize}
}
  
%====================================================================
\section{Illustrations}
\frame{\frametitle{Outline} \tableofcontents[currentsection]}
%====================================================================
\frame{\frametitle{Barents fish} 

  \begin{center}
  \begin{tabular}{lccc}
    & no covariate & \textcolor{blue}{temp. \& depth} & \textcolor{red}{all covariates} \\
    \hline
    \rotatebox{90}{$\qquad\quad\lambda=.20$} &
    \includegraphics[width=.22\textwidth]{\figCMR/network_BarentsFish_Gnull_full60edges} &
    \includegraphics[width=.22\textwidth]{\figCMR/network_BarentsFish_Gsel_full60edges} &
    \includegraphics[width=.22\textwidth]{\figCMR/network_BarentsFish_Gfull_full60edges} 
    \vspace{-0.05\textheight} \\ \hline
    %
    \rotatebox{90}{$\qquad\quad\lambda=.28$} &
    \includegraphics[width=.22\textwidth]{\figCMR/network_BarentsFish_Gnull_sel60edges} & \includegraphics[width=.22\textwidth]{\figCMR/network_BarentsFish_Gsel_sel60edges} &
    \includegraphics[width=.22\textwidth]{\figCMR/network_BarentsFish_Gfull_sel60edges} 
    \vspace{-0.05\textheight} \\ \hline
    %
    \rotatebox{90}{$\qquad\quad\lambda=.84$} &
    \includegraphics[width=.22\textwidth]{\figCMR/network_BarentsFish_Gnull_null60edges} &
    \includegraphics[width=.22\textwidth]{\figCMR/network_BarentsFish_Gsel_null60edges} &
    \includegraphics[width=.22\textwidth]{\figCMR/network_BarentsFish_density} 
%     \includegraphics[width=.22\textwidth]{\figCMR/network_BarentsFish_Gfull_null60edges}  
  \end{tabular}
  \end{center}

}

%====================================================================
\frame{\frametitle{Oak mildew}

  \paragraph{Aim:}
  \begin{itemize}
   \item Understanding microbial interactions within communities living on oak leaves
   \item Focus on {\sl E. alphitoides}: fungal pathogene responsible for oak mildew
  \end{itemize}

  \bigskip \pause
  \paragraph{Data:}
  \begin{itemize}
   \item Metabarcoding data (sequencing)
   \item Different sampling effort for bacteria ('{\tt b}') and fungi ('{\tt f}')
   \item $n_1 = 39$ samples resistant to mildew, $n_2 = 39$ susceptible samples
  \end{itemize}

}
    
%====================================================================
\frame{\frametitle{Oak mildew}

  $$
  \begin{tabular}{cc}
    {\small resistant samples} & {\small susceptible samples} \\ 
    \includegraphics[width=.28\textwidth]{\figCMR/{network_oaks_resistant_stability_0.995}.pdf}
  & \includegraphics[width=.28\textwidth]{\figCMR/{network_oaks_susceptible_stability_0.995}.pdf} \\
   {\small samples from both origins} & {\small regression coef. for orientation} \\
    \includegraphics[width=.28\textwidth]{\figCMR/{network_oaks_both_stability_0.995}.pdf}
  &   \includegraphics[width=.28\textwidth]{\figCMR/mildew_stat_desc} \\     
  \end{tabular}
  $$
}

%====================================================================
\section{Other of PLN}
\frame{\frametitle{Outline} \tableofcontents[currentsection]}
%====================================================================
\frame{\frametitle{Other use of the PLN model}

  \paragraph{Reminder:} for each site $i$
  \begin{align*}
    \text{latent:} & & Z_i & \sim \Ncal_m(0, \Sigma) \\
    \text{observed:} & & Y_{ij} \gv Z_{ij} & \sim \Pcal\left(\exp\left(o_{ij} + x_i^\intercal \beta_j +  Z_{ij}\right)\right)
  \end{align*}

  \pause \bigskip \bigskip
  \paragraph{Regression coefficients $\beta$:} effects of the covariates on species abundances
  
  \bigskip
  \paragraph{Unstructured $\Sigma$:} 'correlation' between species abundance
  
  \bigskip
  \paragraph{Sparse $\Sigma^{-1}$:} network inference
  
  \bigskip
  \paragraph{Low rank $\Sigma$:} dimension reduction (PCA \refer{CMR17b}) \ra rank = 'effective species number'
}

%====================================================================
\section{Conclusion}
%====================================================================
\frame{\frametitle{Conclusion \& Future works} 

  \paragraph{Summary:}
  \begin{itemize}
   \item PLN: a flexible model for multivariate count data analysis
   \item Graphical models: a probabilistic framework for network inference
   \item PLN-network: builds on developments for network inference in GGM
  \end{itemize}
  
  \pause \bigskip 
  \paragraph{R package} available at \textcolor{blue}{\url{https://github.com/jchiquet/PLNmodels}}  
  \begin{itemize}
   \item PLNnetwork, PLN-PCA, ...
   \item Clustering, discriminant analysis, ...
  \end{itemize}
  
  \pause \bigskip 
  \paragraph{On-going work:}
  \begin{itemize}
   \item Model selection: still delicate
   \item Alternative graphical models (tree-shaped)
   \item Looking for missing actors \refer{RAR18}
  \end{itemize}

}

%====================================================================
\backupbegin
%====================================================================

%====================================================================
\frame[allowframebreaks]{ \frametitle{References}
  {%\footnotesize
   \tiny
   \bibliography{/home/robin/Biblio/BibGene}
%    \bibliographystyle{/home/robin/LATEX/Biblio/astats}
   \bibliographystyle{alpha}
  }
}

%====================================================================
\appendix 
\section*{Appendix}

%====================================================================
\frame{\frametitle{Infering ecological networks}

  \paragraph{Ecological network:} set of interactions between species living in a common area
  
  \bigskip \bigskip
  \paragraph{Aim:} based on abundance data, try to infer the ecological network
  
  \bigskip \bigskip
  \paragraph{Abundance data:} counts of individuals from each species, read counts mapped on each species genome, ...
  
  
  \bigskip \bigskip \pause
  \paragraph{Needs:}
  \begin{itemize}
   \item Account for the specificities of abundance data (counts, overdispersion, ...), \\~
   \item Account for the potential effects of environmental covariates, \\~
   \item Provide (statistical) definition of a network
  \end{itemize}

}
  
%====================================================================
\frame{\frametitle{Network inference: some strategies} 

  \bigskip
  \paragraph{Typical setting.} 
  $$
  \{Y_i\}_i \text{ iid}, \quad Y_i \sim p_G 
  \qquad \text{e.g. } p_G = \Ncal_m\left(0, \Omega^{-1}_G\right)
  $$
  
  \paragraph{First idea.}
  $$
  \widehat{G} = \arg\max_{G \in \Gcal} \; \log p_G(Y)
  $$
  \ra combinatorial issue: $|\Gcal| = 2^{m(m-1)/2}$.
  
  \bigskip\bigskip
  \paragraph{Some strategies.}
  \begin{enumerate}
   \item Exhaustive / stochastic \refer{NPK11} / heuristic search over $\Gcal$ \\ ~
   \item Regularization based on a sparsity assumption for $G$ (\refer{MeB06,FHT08}: Gaussian)
   $$
  \widehat{G} = \arg\max_G \; \log p_G(Y) - \lambda \|\Omega_G\|_1
   $$ \\~
   \item Averaging over a subset $\Tcal$ of graphs (e.g. spanning trees \refer{Cha82,MeJ06,Kir07})
   $$
   p(Y) = \sum_{T \in \Tcal} \pi_T p_T(Y)
   $$
  \end{enumerate}
  }

%====================================================================
\frame{\frametitle{How many edges?}

  \bigskip
  \paragraph{EBIC \refer{ChC08,FoD10}.} Network density controlled by $\lambda$ \ra need for a model selection criterion 
  $$
  EBIC_\gamma(\lambda) 
  = -2 J_\lambda(\widehat{\theta}, \widehat{q}) 
  + \log n \underset{\text{\# parameters}}{\underbrace{\left(|\Ecal_\lambda| + pd\right)}} 
  + \gamma \log \underset{\text{\# models}}{\underbrace{\binom{p(p-1)/2}{|\Ecal_\lambda|}}}
  $$
  
  \bigskip \pause
  \paragraph{Stability selection \refer{LRW10}.} Resampling-based approach to select consistently selected edges.
}

%====================================================================
\frame{\frametitle{PLN-PCA for oak mildew: First 2 PCs}

  $$
  \includegraphics[width=.9\textwidth]{../FIGURES/CMR17-Fig1c_IndMap.pdf}
  $$
  $M_0$ (left): offset only; $M_1$ (right): offset + covariates, 
}

%====================================================================
\backupend

%====================================================================
%====================================================================
\end{document}
%====================================================================
%====================================================================

  \begin{tabular}{cc}
    \begin{tabular}{p{.5\textwidth}}
    \end{tabular}
    & 
    \hspace{-.02\textwidth}
    \begin{tabular}{p{.5\textwidth}}
    \end{tabular}
  \end{tabular}

