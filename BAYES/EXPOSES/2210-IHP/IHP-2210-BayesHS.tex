\documentclass[8pt]{beamer}

% Beamer style
%\usetheme[secheader]{Madrid}
% \usetheme{CambridgeUS}
\useoutertheme{infolines}
\usecolortheme[rgb={0.65,0.15,0.25}]{structure}
% \usefonttheme[onlymath]{serif}
\beamertemplatenavigationsymbolsempty
%\AtBeginSubsection

% Packages
%\usepackage[french]{babel}
\usepackage[latin1]{inputenc}
\usepackage{color}
% \usepackage[dvipsnames]{xcolor}
\usepackage{xspace}
\usepackage{dsfont, stmaryrd}
\usepackage{amsmath, amsfonts, amssymb, stmaryrd, mathabx}
\usepackage{epsfig}
\usepackage{tikz}
\usepackage{url}
% \usepackage{ulem}
\usepackage{/home/robin/LATEX/Biblio/astats}
%\usepackage[all]{xy}
\usepackage{graphicx}

% Maths
% \newtheorem{theorem}{Theorem}
% \newtheorem{definition}{Definition}
\newtheorem{proposition}{Proposition}
% \newtheorem{assumption}{Assumption}
% \newtheorem{algorithm}{Algorithm}
% \newtheorem{lemma}{Lemma}
% \newtheorem{remark}{Remark}
% \newtheorem{exercise}{Exercise}
% \newcommand{\propname}{Prop.}
% \newcommand{\proof}{\noindent{\sl Proof:}\quad}
% \newcommand{\eproof}{$\blacksquare$}

% \setcounter{secnumdepth}{3}
% \setcounter{tocdepth}{3}
\newcommand{\pref}[1]{\ref{#1} p.\pageref{#1}}
\newcommand{\qref}[1]{\eqref{#1} p.\pageref{#1}}

% Colors : http://latexcolor.com/
\definecolor{darkred}{rgb}{0.65,0.15,0.25}
\definecolor{darkgreen}{rgb}{0,0.4,0}
\definecolor{darkred}{rgb}{0.65,0.15,0.25}
\definecolor{amethyst}{rgb}{0.6, 0.4, 0.8}
\definecolor{asparagus}{rgb}{0.53, 0.66, 0.42}
\definecolor{applegreen}{rgb}{0.55, 0.71, 0.0}
\definecolor{awesome}{rgb}{1.0, 0.13, 0.32}
\definecolor{blue-green}{rgb}{0.0, 0.87, 0.87}
\definecolor{red-ggplot}{rgb}{0.52, 0.25, 0.23}
\definecolor{green-ggplot}{rgb}{0.42, 0.58, 0.00}
\definecolor{purple-ggplot}{rgb}{0.34, 0.21, 0.44}
\definecolor{blue-ggplot}{rgb}{0.00, 0.49, 0.51}

% Commands
\newcommand{\backupbegin}{
   \newcounter{finalframe}
   \setcounter{finalframe}{\value{framenumber}}
}
\newcommand{\backupend}{
   \setcounter{framenumber}{\value{finalframe}}
}
\newcommand{\emphase}[1]{\textcolor{darkred}{#1}}
\newcommand{\comment}[1]{\textcolor{gray}{#1}}
\newcommand{\paragraph}[1]{\textcolor{darkred}{#1}}
\newcommand{\refer}[1]{{\small{\textcolor{gray}{{\cite{#1}}}}}}
\newcommand{\Refer}[1]{{\small{\textcolor{gray}{{[#1]}}}}}
\newcommand{\goto}[1]{{\small{\textcolor{blue}{[\#\ref{#1}]}}}}
\renewcommand{\newblock}{}

\newcommand{\tabequation}[1]{{\medskip \centerline{#1} \medskip}}
% \renewcommand{\binom}[2]{{\left(\begin{array}{c} #1 \\ #2 \end{array}\right)}}

% Variables 
\newcommand{\Abf}{{\bf A}}
\newcommand{\Beta}{\text{B}}
\newcommand{\Bcal}{\mathcal{B}}
\newcommand{\Bias}{\xspace\mathbb B}
\newcommand{\Cor}{{\mathbb C}\text{or}}
\newcommand{\Cov}{{\mathbb C}\text{ov}}
\newcommand{\cl}{\text{\it c}\ell}
\newcommand{\Ccal}{\mathcal{C}}
\newcommand{\cst}{\text{cst}}
\newcommand{\Dcal}{\mathcal{D}}
\newcommand{\Ecal}{\mathcal{E}}
\newcommand{\Esp}{\xspace\mathbb E}
\newcommand{\Espt}{\widetilde{\Esp}}
\newcommand{\Covt}{\widetilde{\Cov}}
\newcommand{\Ibb}{\mathbb I}
\newcommand{\Fcal}{\mathcal{F}}
\newcommand{\Gcal}{\mathcal{G}}
\newcommand{\Gam}{\mathcal{G}\text{am}}
\newcommand{\Hcal}{\mathcal{H}}
\newcommand{\Jcal}{\mathcal{J}}
\newcommand{\Lcal}{\mathcal{L}}
\newcommand{\Mt}{\widetilde{M}}
\newcommand{\mt}{\widetilde{m}}
\newcommand{\Nbb}{\mathbb{N}}
\newcommand{\Mcal}{\mathcal{M}}
\newcommand{\Ncal}{\mathcal{N}}
\newcommand{\Ocal}{\mathcal{O}}
\newcommand{\pt}{\widetilde{p}}
\newcommand{\Pt}{\widetilde{P}}
\newcommand{\Pbb}{\mathbb{P}}
\newcommand{\Pcal}{\mathcal{P}}
\newcommand{\Qcal}{\mathcal{Q}}
\newcommand{\qt}{\widetilde{q}}
\newcommand{\Rbb}{\mathbb{R}}
\newcommand{\Sbb}{\mathbb{S}}
\newcommand{\Scal}{\mathcal{S}}
\newcommand{\st}{\widetilde{s}}
\newcommand{\St}{\widetilde{S}}
\newcommand{\Tcal}{\mathcal{T}}
\newcommand{\todo}{\textcolor{red}{TO DO}}
\newcommand{\Ucal}{\mathcal{U}}
\newcommand{\Un}{\math{1}}
\newcommand{\Vcal}{\mathcal{V}}
\newcommand{\Var}{\mathbb V}
\newcommand{\Vart}{\widetilde{\Var}}
\newcommand{\Zcal}{\mathcal{Z}}

% Symboles & notations
\newcommand\independent{\protect\mathpalette{\protect\independenT}{\perp}}\def\independenT#1#2{\mathrel{\rlap{$#1#2$}\mkern2mu{#1#2}}} 
\renewcommand{\d}{\text{\xspace d}}
\newcommand{\gv}{\mid}
\newcommand{\ggv}{\, \| \, }
% \newcommand{\diag}{\text{diag}}
\newcommand{\card}[1]{\text{card}\left(#1\right)}
\newcommand{\trace}[1]{\text{tr}\left(#1\right)}
\newcommand{\matr}[1]{\boldsymbol{#1}}
\newcommand{\matrbf}[1]{\mathbf{#1}}
\newcommand{\vect}[1]{\matr{#1}} %% un peu inutile
\newcommand{\vectbf}[1]{\matrbf{#1}} %% un peu inutile
\newcommand{\trans}{\intercal}
\newcommand{\transpose}[1]{\matr{#1}^\trans}
\newcommand{\crossprod}[2]{\transpose{#1} \matr{#2}}
\newcommand{\tcrossprod}[2]{\matr{#1} \transpose{#2}}
\newcommand{\matprod}[2]{\matr{#1} \matr{#2}}
\DeclareMathOperator*{\argmin}{arg\,min}
\DeclareMathOperator*{\argmax}{arg\,max}
\DeclareMathOperator{\sign}{sign}
\DeclareMathOperator{\tr}{tr}
\newcommand{\ra}{\emphase{$\rightarrow$} \xspace}

% Hadamard, Kronecker and vec operators
\DeclareMathOperator{\Diag}{Diag} % matrix diagonal
\DeclareMathOperator{\diag}{diag} % vector diagonal
\DeclareMathOperator{\mtov}{vec} % matrix to vector
\newcommand{\kro}{\otimes} % Kronecker product
\newcommand{\had}{\odot}   % Hadamard product

% TikZ
\newcommand{\nodesize}{2em}
\newcommand{\edgeunit}{2.5*\nodesize}
\newcommand{\edgewidth}{1pt}
\tikzstyle{node}=[draw, circle, fill=black, minimum width=.75\nodesize, inner sep=0]
\tikzstyle{square}=[rectangle, draw]
\tikzstyle{param}=[draw, rectangle, fill=gray!50, minimum width=\nodesize, minimum height=\nodesize, inner sep=0]
\tikzstyle{hidden}=[draw, circle, fill=gray!50, minimum width=\nodesize, inner sep=0]
\tikzstyle{hiddenred}=[draw, circle, color=red, fill=gray!50, minimum width=\nodesize, inner sep=0]
\tikzstyle{observed}=[draw, circle, minimum width=\nodesize, inner sep=0]
\tikzstyle{observedred}=[draw, circle, minimum width=\nodesize, color=red, inner sep=0]
\tikzstyle{eliminated}=[draw, circle, minimum width=\nodesize, color=gray!50, inner sep=0]
\tikzstyle{empty}=[draw, circle, minimum width=\nodesize, color=white, inner sep=0]
\tikzstyle{blank}=[color=white]
\tikzstyle{nocircle}=[minimum width=\nodesize, inner sep=0]

\tikzstyle{edge}=[-, line width=\edgewidth]
\tikzstyle{edgebendleft}=[-, >=latex, line width=\edgewidth, bend left]
\tikzstyle{edgebendright}=[-, >=latex, line width=\edgewidth, bend right]
\tikzstyle{lightedge}=[-, line width=\edgewidth, color=gray!50]
\tikzstyle{lightedgebendleft}=[-, >=latex, line width=\edgewidth, bend left, color=gray!50]
\tikzstyle{lightedgebendright}=[-, >=latex, line width=\edgewidth, bend right, color=gray!50]
\tikzstyle{edgered}=[-, line width=\edgewidth, color=red]
\tikzstyle{edgebendleftred}=[-, >=latex, line width=\edgewidth, bend left, color=red]
\tikzstyle{edgebendrightred}=[-, >=latex, line width=\edgewidth, bend right, color=red]

\tikzstyle{arrow}=[->, >=latex, line width=\edgewidth]
\tikzstyle{arrowbendleft}=[->, >=latex, line width=\edgewidth, bend left]
\tikzstyle{arrowbendright}=[->, >=latex, line width=\edgewidth, bend right]
\tikzstyle{arrowred}=[->, >=latex, line width=\edgewidth, color=red]
\tikzstyle{arrowbendleftred}=[->, >=latex, line width=\edgewidth, bend left, color=red]
\tikzstyle{arrowbendrightred}=[->, >=latex, line width=\edgewidth, bend right, color=red]
\tikzstyle{arrowblue}=[->, >=latex, line width=\edgewidth, color=blue]
\tikzstyle{dashedarrow}=[->, >=latex, dashed, line width=\edgewidth]
\tikzstyle{dashededge}=[-, >=latex, dashed, line width=\edgewidth]
\tikzstyle{dashededgebendleft}=[-, >=latex, dashed, line width=\edgewidth, bend left]
\tikzstyle{lightarrow}=[->, >=latex, line width=\edgewidth, color=gray!50]


% Directory
\newcommand{\ts}{{\theta^{*}}}
\newcommand{\kl}{{(k\ell)}}
\newcommand{\figCMR}{/home/robin/Bureau/RECHERCHE/ECOLOGIE/CountPCA/sparsepca/Article/Network_JCGS/trunk/figs}
\newcommand{\fignet}{/home/robin/Bureau/RECHERCHE/RESEAUX/EXPOSES/FIGURES}
\newcommand{\figchp}{/home/robin/Bureau/RECHERCHE/RUPTURES/EXPOSES/FIGURES}
\newcommand{\figtree}{/home/robin/RECHERCHE/BAYES/VBEM-IS/VBEM-IS.git/Data/Tree/Fig}
\newcommand{\figzebra}{/home/robin/RECHERCHE/BAYES/VBEM-IS/VBEM-IS.git/Data/Zebra/Fig}

%====================================================================
%====================================================================

%====================================================================
%====================================================================
\begin{document}
%====================================================================
%====================================================================

%====================================================================
\title[Tree-based mixtures]{Tree-based mixtures: Application in epidemiology}

\author[S. Robin]{St\'ephane Robin \\ ~ \\
joint work with P. Barbillon, L. Schwaller}

\institute{Sorbonne Universit\'e / LPSM}

\date[IHP, Oct. 2022]{Bayesian Methods for the Social Sciences, IHP, Oct. 2022}

%====================================================================
%====================================================================
\maketitle
%====================================================================

%====================================================================
%====================================================================
\section*{Introduction}
%====================================================================
%====================================================================
\frame{\frametitle{Spred of an 'epidemics' \refer{BSR19}} 

  \paragraph{Data:} $n$ times, $p$ individuals,
  \begin{align*}
     Y_{tj} & = \text{status ('sick' / 'healthy') of individual $j$ at time $t$}
  \end{align*}
  
  $$
  \includegraphics[width=.6\textwidth]{\fignet/BSR18-Fig1}
  $$
  
  \bigskip \pause
  \paragraph{Assumption:} 'Contamination' spreads according to some 'social' network (not anybody can be 'contaminated' by anybody)
  
  \bigskip \pause
  \paragraph{Final aim:} Learn something about this network.
  
  \bigskip 
  \paragraph{For now:} Observe that the path of the 'epidemics' is a tree.
}

% %====================================================================
% \frame{\frametitle{Outline} \tableofcontents}
% 
%====================================================================
%====================================================================
\section{Reminder on graphical models and trees}
\frame{\frametitle{Outline} \tableofcontents[currentsection]}
%====================================================================
%====================================================================
\frame{\frametitle{Graphical models} 

  \paragraph{Undirected graphical model \refer{Lau96}:} the multivariate distribution $p$ is {\em faithful} to the graph $G$ iff
  $$
  p(Y) \; \propto \; \prod_{C \in \Ccal(G)} \psi_C(Y^C)
  $$
  (i.e., iff $p$ can be factorized according to the set  $\Ccal(G)$ of maximal cliques of $G$)
%   \ra Markov property: $G$ encodes the conditional independences of $p$:
%   $$
%   j \nsim k \qquad \Leftrightarrow \qquad Y_j \perp Y_k \mid Y_{\setminus \{j, k\}}
%   $$
%   $$
%   j \sim k \qquad \Leftrightarrow \qquad Y_j \notperp Y_k \mid Y_{\setminus \{j, k\}}
%   $$
  
  \bigskip \bigskip \bigskip \pause
  \paragraph{Example:} ~ \\ ~ \\
  \begin{tabular}{cc}
    \begin{tabular}{p{.3\textwidth}}
	 \begin{tikzpicture}
\node[observed] (Y3) at (0*\edgeunit, 0*\edgeunit) {$Y_3$};
\node[observed] (Y1) at (-0.5*\edgeunit, 1*\edgeunit) {$Y_1$};
\node[observed] (Y2) at (.5*\edgeunit, 1*\edgeunit) {$Y_2$};
\node[observed] (Y4) at (1*\edgeunit, 0*\edgeunit) {$Y_4$};
\draw[edge] (Y1) to (Y2); \draw[edge] (Y1) to (Y3); \draw[edge] (Y2) to (Y3); \draw[edge] (Y3) to (Y4);
\end{tikzpicture}


    \end{tabular}
    & 
    \begin{tabular}{p{.65\textwidth}}
    $p(Y) \; \propto \; \psi_1(Y_1, Y_2, Y_3) \; \psi_2(Y_3, Y_4)$ \\~
	 \begin{itemize}
	 \item Connected graph: all variables are dependent \\~
	 \item $Y_3 =$ separator: $\left(Y_4 \perp (Y_1, Y_2)\right) \mid Y_3$ 
	 \end{itemize}
    \end{tabular}
  \end{tabular} 

}

%====================================================================
\frame{\frametitle{Tree-shaped graphical model} 

  \paragraph{Spanning tree =} acyclic graph connecting all the nodes
  
  \medskip
  \begin{centering}
    \begin{tabular}{cc|c|c}
      \multicolumn{2}{c|}{spanning trees} & not spanning & not a tree \\
      & & & \\
      \begin{tabular}{p{.18\textwidth}}
        \begin{tikzpicture}
\node[node] (Y1) at (0*\edgeunit, 0*\edgeunit) {};
\node[node] (Y2) at (0*\edgeunit, 1*\edgeunit) {};
\node[node] (Y3) at (1*\edgeunit, 1*\edgeunit) {};
\node[node] (Y4) at (1*\edgeunit, 0*\edgeunit) {};
\node[node] (Y5) at (.5*\edgeunit, .5*\edgeunit) {};
\draw[edge] (Y1) to (Y2); \draw[edge] (Y2) to (Y3); \draw[edge] (Y3) to (Y4); \draw[edge] (Y4) to (Y5);
\end{tikzpicture}


      \end{tabular}
      & 
      \begin{tabular}{p{.18\textwidth}}
        \begin{tikzpicture}
\node[node] (Y1) at (0*\edgeunit, 0*\edgeunit) {};
\node[node] (Y2) at (0*\edgeunit, 1*\edgeunit) {};
\node[node] (Y3) at (1*\edgeunit, 1*\edgeunit) {};
\node[node] (Y4) at (1*\edgeunit, 0*\edgeunit) {};
\node[node] (Y5) at (.5*\edgeunit, .5*\edgeunit) {};
\draw[edge] (Y1) to (Y2); \draw[edge] (Y2) to (Y5); \draw[edge] (Y5) to (Y3); \draw[edge] (Y4) to (Y5);
\end{tikzpicture}


      \end{tabular}
      & \begin{tabular}{p{.18\textwidth}}
        \begin{tikzpicture}
\node[node] (Y1) at (0*\edgeunit, 0*\edgeunit) {};
\node[node] (Y2) at (0*\edgeunit, 1*\edgeunit) {};
\node[node] (Y3) at (1*\edgeunit, 1*\edgeunit) {};
\node[node] (Y4) at (1*\edgeunit, 0*\edgeunit) {};
\node[node] (Y5) at (.5*\edgeunit, .5*\edgeunit) {};
\draw[edge] (Y1) to (Y2); \draw[edge] (Y2) to (Y5); \draw[edge] (Y5) to (Y3); 
\end{tikzpicture}

 
      \end{tabular}
      & \begin{tabular}{p{.18\textwidth}}
        \begin{tikzpicture}
\node[node] (Y1) at (0*\edgeunit, 0*\edgeunit) {};
\node[node] (Y2) at (0*\edgeunit, 1*\edgeunit) {};
\node[node] (Y3) at (1*\edgeunit, 1*\edgeunit) {};
\node[node] (Y4) at (1*\edgeunit, 0*\edgeunit) {};
\node[node] (Y5) at (.5*\edgeunit, .5*\edgeunit) {};
\draw[edge] (Y1) to (Y2); \draw[edge] (Y2) to (Y5); \draw[edge] (Y5) to (Y3); \draw[edge] (Y5) to (Y4); \draw[edge] (Y4) to (Y3); 
\end{tikzpicture}


      \end{tabular}
    \end{tabular} 
  \end{centering}

  \bigskip \bigskip \pause
  \paragraph{Tree-shaped graphical model:} $p$ faithful to the spanning tree $T$
  $$
  \Leftrightarrow \qquad 
  p(Y) \; \propto \; \prod_{(j, k) \in T} \psi_{jk}(Y_j, Y_k)
  $$
  (cliques are only edges) 
  
  \bigskip
  \ra All variables are dependent ({\sl spanning}) but few are conditionnaly dependent ({\sl tree})
}

%====================================================================
\frame{\frametitle{Maximum likelihood inference} 

  \paragraph{Consider} $Y \sim p$, faithful to $T$:
  $$
  \log p_T(Y) = \sum_{(j, k) \in T} \log \psi_{jk}(Y_j, Y_k) + \cst
  $$
  
  \bigskip \pause
  Suppose each $\psi_{jk}$ has a parametric form and that MLEs $\widehat{\psi}_{jk}$ is available.
  
  \bigskip \bigskip \pause
  \paragraph{Tree MLE.} Finding 
  $$
  \widehat{T} = \argmax_{T \in \Tcal} \; \log p_T(Y)
  $$
  amounts to solve a \emphase{maximum spanning tree} problem, where edge $(j, k)$ has weight
  $$
  \log \widehat{\psi}_{jk}(Y_j, Y_k)
  $$
  (Chow \& Liu algorithm\footnote{Actually $\psi_{jk}(Y_j, Y_k) = p(Y_j, Y_k) / \left(p(Y_j) p(Y_k)\right)$, so edge weight = mutual information}: \refer{ChL68})
}

%====================================================================
%====================================================================
\section{Tree shaped distributions and mixtures}
\frame{\frametitle{Outline} \tableofcontents[currentsection]}
%====================================================================
\frame{\frametitle{Tree-based mixture} 

  \paragraph{Mixture of tree-shaped distributions \refer{MeJ06,Kir07}.}
  \begin{itemize}
    \setlength{\itemsep}{1.25\baselineskip}
    \item {Tree marginal distribution:} product form
    $$
    p(T) = B^{-1} \prod_{(j, k) \in T} \beta_{jk}, 
    \qquad 
    B = \emphase{\sum_{T \in \Tcal}} \prod_{(j, k) \in T} \beta_{jk}.
    $$
    \item \pause {Data conditional distribution (likelihood):} 
    $$
    p(Y \mid T) \; \propto \prod_{(j, k) \in T} \psi_{jk}(Y_j, Y_k).
    $$ 
    \item \pause {Data marginal distribution:} 
    \begin{align*}
    p(Y) 
    = \emphase{\sum_{T \in \Tcal}} p(T) p(Y \mid T) 
    \; \propto \; \emphase{\sum_{T \in \Tcal}} \prod_{(j, k) \in T} \underset{w_{jk}(Y)}{\underbrace{\beta_{jk} \psi_{jk}(Y_j, Y_k)}}.
    \end{align*} 
    \item \pause {Tree conditional distribution:} product form
    $$
    p(T \mid Y) = C^{-1} \prod_{(j, k) \in T} w_{jk}(Y), 
    \qquad
    C = \emphase{\sum_{T \in \Tcal}} \prod_{(j, k) \in T} w_{jk}(Y).    
    $$
  \end{itemize} 
}

%====================================================================
\frame{\frametitle{Summing over trees} 

  We need to sum over the set $\Tcal$ of spanning trees, but
  $$
  \card{\Tcal} = p^{p-2}
  $$

  \bigskip \bigskip \pause
  \paragraph{Matrix-Tree Theorem (Kirchhoff) \refer{Cha82}.} Let $W = [w_{jk}]$ be a symmetric matrix with null diagonal and $\Delta(W)$, its Laplacian:
  $$
  \Delta(W)_{jk} = \left\{
    \begin{array}{rl}
      \sum_k w_{jk} & \text{if } j = k \\
      -w_{jk} & \text{otherwise}
    \end{array}
  \right.
  $$
  then 
  \begin{enumerate}
    \setlength{\itemsep}{1.25\baselineskip}
    \item \pause all the cofactors $[\Delta(W)]^{uv}$ of $\Delta(W)$ are equal 
    \item \pause and
    $$
    [\Delta(W)]^{uv} = \emphase{\sum_{T \in \Tcal}} \prod_{(j, k) \in T} w_{jk}
    $$
    (i.e.: $\sum_{T \in \Tcal}$ can be computed at the price of a determinant: \emphase{$O(p^3)$})
  \end{enumerate}
  
}

%====================================================================
\frame{\frametitle{Edge probability} 

  We may not be interested in the whole (conditional) distribution of $T$, but rather on edge probabilities:
  $$
  \Pr\{(j, k) \in T\} 
  \qquad \text{or} \qquad 
  \Pr\{(j, k) \in T \mid Y\}.
  $$
  
  \bigskip \bigskip \pause
  \paragraph{Theorem \refer{Kir07}.} Denote $W^{ab}$ the same matrix as $W$ but setting $w_{ab} = w_{ba} = 0$, then
  $$
  [\Delta(W^{ab})]^{ab} = \sum_{\emphase{T \in \Tcal: (a, b) \notin T}} \prod_{(j, k) \in T} w_{jk}
  $$
  
  \bigskip \pause
  \begin{itemize}
    \setlength{\itemsep}{1.25\baselineskip}
    \item Gives access to $\Pr\{(j, k) \notin T\}$ or $\Pr\{(j, k) \notin T\}$
    \item All $[\Delta(W^{ab})]^{ab}$ can be computed at the price of one matrix inversion: \emphase{$O(p^3)$}
  \end{itemize}
}

%====================================================================
%====================================================================
\section{Spread of an epidemics}
\frame{\frametitle{Outline} \tableofcontents[currentsection]}
%====================================================================
\frame{\frametitle{Susceptible-Infected-Susceptible (SIS) model} 

  \paragraph{Data.} $n$ times, $p$ individuals
  $$
  Y_{tj} = 
  \left\{ \begin{array}{rl}
  1 & \text{if individual $j$ is infected at time $t$} \\
  0 & \text{otherwise}
    \end{array} \right.
  $$
  \ra Complete observations
  
  \bigskip \bigskip \pause 
  \paragraph{Model.} At each time $t$, each node $k$ 
  \begin{itemize}
    \item picks up a parent $j$ at time $t-1$ with probability $\propto \; \beta_{jk}$ and, 
    \item denoting $\psi_{jk}^t = \Pr\{Y_{t+1, k} \mid Y_{t, k}, Y_{t, j}\}$, it evolves according to
    $$
    \begin{array}{cc|cc}
      \multicolumn{2}{c|}{\psi_{jk}^t} & Y_{t+1, k} = 0 & Y_{t+1, k} = 1 \\
      \hline
      Y_{t, k} = 1 & Y_{t, j} = 1 & e & \textcolor{gray}{1-e} \\
      Y_{t, k} = 1 & Y_{t, j} = 0 & e & \textcolor{gray}{1-e} \\
      Y_{t, k} = 0 & Y_{t, j} = 1 & \textcolor{gray}{1-c} & c \\
      Y_{t, k} = 0 & Y_{t, j} = 0 & \textcolor{gray}{1} & 0 
    \end{array}
    $$ 
  \end{itemize}
  \begin{itemize}
   \item $c =$ contamination rate
   \item $e =$ extinction rate (become susceptible again)
  \end{itemize}

}

%====================================================================
\frame{\frametitle{A tree-shaped path} 

  Adding a fictitious root $\Delta$ (at time 0), a path is a tree $T = (T^1, \dots, T^{n-1})$:
  $$
  \includegraphics[width=.6\textwidth]{\fignet/BSR19-Fig1}
  $$
  
  \bigskip \pause
  Define $\Tcal_\Delta$ the set of oriented spanning tree
  \begin{itemize}
   \item over the nodes $\Delta \cup \{(t, k): 1 \leq t \leq n, 1 \leq k \leq p\}$,
   \item rooted in $\Delta$,
   \item with edges connecting only time-adjacent nodes $(j \neq k)$
  \end{itemize}
  
  \bigskip
  \ra $\Tcal_\Delta =$ set of spanning trees going 'forward' in time ($|\Tcal_\Delta| = (p-1)^{p(n-1)}$)
}

%====================================================================
\frame{\frametitle{Inference} 

  \paragraph{EM algorithm.} Denote 
  $$\theta = (\beta = [\beta_{jk}], e, c)$$
  \begin{description}
    \setlength{\itemsep}{1.25\baselineskip}
    \item[E Step:] \pause Compute the conditional distribution
    $$
    p(T_\Delta = (T^1, \dots T^{n-1}) \mid Y)
%     =
%     \frac{\prod_{t=1}^{n-1} \prod_{(j, k) \in T^t} \beta_{jk} \psi_{jk}^t}{\sum_{T \in \Tcal_\Delta} \prod_{t=1}^{n-1} \prod_{(j, k) \in T^t} \underset{w^t_{jk}}{\underbrace{\beta_{jk} \psi_{jk}^t}}}
    =
    \frac{\prod_{t=1}^{n-1} \prod_{(j, k) \in T^t} w^t_{jk}}{\emphase{\sum_{T \in \Tcal_\Delta}} \prod_{t=1}^{n-1} \prod_{(j, k) \in T^t} w^t_{jk}}, 
    \qquad
    w^t_{jk} = \beta_{jk} \psi_{jk}^t
    $$
    \item[M Step:] \pause Update the parameters as
    $$
    \widehat{\theta} = \argmax \Esp\left[\log p_\theta(Y, T) \mid Y\right]
    $$
  \end{description}

  \bigskip \bigskip \pause
  \paragraph{Critical step = E step.} \\
  \medskip
  \begin{itemize}
    \setlength{\itemsep}{1.25\baselineskip}
    \item We now need to sum over the (huge) set of rooted oriented trees $\Tcal_\Delta$.
    \item Hopefully, a alternative version of the matrix-tree theorem enables to sum over all {\sl directed} trees ($W$ asymmetric) with given root \refer{Cha82}.
  \end{itemize}
}

%====================================================================
\frame{\frametitle{An easy situation} 

  \begin{tabular}{lll}
   \hspace{-.05\textwidth}
    \begin{tabular}{p{.015\textwidth}}
      $W =$
    \end{tabular}
    &
    \begin{tabular}{l}
      \includegraphics[width=.5\textwidth]{\fignet/BSR19-Fig2p}
    \end{tabular}
   &
   \hspace{-.05\textwidth}
   \begin{tabular}{p{.45\textwidth}}
    \paragraph{$W = [w^t_{jk}]$} where $w^t_{jk} = \beta_{jk} \psi_{jk}^t$: \\
    Edges connect only time-adjacent \\
    nodes
    
    \bigskip \bigskip \pause
    \paragraph{\ra Laplacian $\Delta(W)$} \\ is upper triangular 
    
    \bigskip \bigskip 
    \paragraph{\ra Matrix-tree theorem:}
    $$
    \Delta(W)^{00} = \prod_{t, j} \left(\sum_k \beta_{jk} \psi_{jk}^t\right)
    $$
    \ra computable in $O(np^2)$

   \end{tabular}
  \end{tabular}
}

%====================================================================
\frame{\frametitle{In practice} 

  \paragraph{Edge probabilities:}
  \begin{align*}
    \text{At a given time $t$:} & \qquad \Pr\{(j, k) \in T^t \mid Y\} \\
    \text{At least in one time:} & \qquad \Pr\{\exists t: (j, k) \in T^t \mid Y\}
  \end{align*}

  \bigskip \bigskip \pause
  \paragraph{Alternatives.}
  \begin{itemize}
    \setlength{\itemsep}{1.25\baselineskip}
    \item Bayesian inference can be carried out for $e$ an $c$
    \item Iterating the EM steps does not improve the performances very much
    \item Observing multiple waves of the epidemics (even over a shortest time-range) improves the accuracy (see next)
  \end{itemize} 
}

%====================================================================
\frame{\frametitle{Simulation study} 

  \paragraph{Design:} Consider a graph $G$ and launch the 'epidemics' along its edges.
  
  \bigskip 
  \paragraph{Method:} Predict if $(j, k) \in G$ based on edge probability $\Pr\{\exists t: (j, k) \in T^t \mid Y\}$.
  
  \bigskip \pause
  \begin{tabular}{cc}
    \hspace{-.04\textwidth}
    \begin{tabular}{p{.5\textwidth}}
      \paragraph{Parameters:} $p = 20$ nodes \\
      ~ \\ \pause
      $G =$ ER: Erd\"os, \\
      $G =$ PA: preferential attachment \\
      ~\\ \pause
      $d =$ network density \\
      $e = .05$ \\
      ~\\ \pause
      \emphase{uW:} one wave ($n = 200$) \\
      \emphase{mW:} 10 waves ($n = 20$) \\
      ~ \\ \pause
      \emphase{s}: forcing edge probabilities to be symmetric \\
      ~ \\ ~ \\ 
    \end{tabular} \pause
    &
    \hspace{-.05\textwidth}
    \begin{tabular}{l}
      \paragraph{AUC:} \\
      \includegraphics[width=.35\textwidth, trim=0 0 255 300, clip=]{\fignet/BSR19-Fig3}
    \end{tabular}   
  \end{tabular}

}

%====================================================================
\frame{\frametitle{Illustration: Seed exchange network} 

  \begin{tabular}{cc}
    \hspace{-.04\textwidth}
    \begin{tabular}{p{.5\textwidth}}
      \paragraph{Question:} decipher the social structure underlying seed exchanges between farmers 
      
      \bigskip
      \paragraph{Telangana region (India) data:}
      \begin{itemize}
        \item  $p = 127$ farmers
        \item  $n = 3$ years
        \item  14 seed varieties (waves) 
      \end{itemize}
%       $p = 127$ farmers \\
%       $n = 3$ years \\
%       14 seed varieties (waves) 
      
      \bigskip
      $Y_{ti}^h =1$ if farmer $i$ holds variety $h$ at time $t$.
      
      \bigskip
      No symmetry assumption.
      
      \bigskip
      \paragraph{More exchanges}
      \begin{itemize}
        \item within the same caste
        \item within the same village
        \item from younger to older
      \end{itemize}
%       \ra within the same caste \\
%       \ra within the same village \\
%       \ra from younger to older
    \end{tabular}
    & 
    \hspace{-.05\textwidth}
    \begin{tabular}{c}
      \includegraphics[width=.45\textwidth]{\fignet/BSR19-Fig4} \\
      Most probable donor for each farmer
    \end{tabular}
  \end{tabular}
}

%====================================================================
%====================================================================
\section*{Concluding remarks}
%====================================================================
\frame{\frametitle{Conclusion} 

  \paragraph{Tree shaped mixtures}
  \begin{itemize}
   \item Flexible model for multivariate distributions
   \item Base on a mixture with exponentially many components ($p^{p-2}$)
   \item But giving access to edge probabilities at a low computational cost
  \end{itemize}
  
  \bigskip \bigskip \pause
  \paragraph{Extensions}
  \begin{itemize}    
    \item 'Network' inference (= structure inference) \refer{SRS19}
    \item Network comparison or network changes along time \refer{ScR17}
    \item Accounting for missing nodes ('actors') \refer{RAR18,MRA20,MRA21}
    \item S-I-S model can be extended to more that two states (e.g. S-I-R models)
  \end{itemize}
  
  \bigskip \bigskip \pause
  \paragraph{Some questions}
  \begin{itemize}
    \item Theoretical guaranties (e.g.: consistency of the estimated graph)?
%     \item Which distributions are well approximated by tree-shaped mixtures?
    \item Numerical issues arising for large $p$ or $n$ (use tempering?)
  \end{itemize}
}

%====================================================================
\frame[allowframebreaks]{ \frametitle{References}
  {%\footnotesize
   \tiny
   \bibliography{/home/robin/Biblio/BibGene}
%    \bibliographystyle{/home/robin/LATEX/Biblio/astats}
   \bibliographystyle{alpha}
  }
}

%====================================================================
\backupbegin
%====================================================================
%====================================================================
\frame{\frametitle{'Tree-averaging' principle} 

%   \renewcommand{\nodesize}{2em}
  \renewcommand{\nodesize}{1.5em}
  
  \hspace{-.025\textwidth}
  \begin{tabular}{cc}
    \begin{tabular}{p{.45\textwidth}}
      \paragraph{Tree conditional distribution.} $p(T | Y)$ \\
      ~ \\
      \begin{tabular}{cc}
%         \multicolumn{2}{c}{$p(T_1 | Y)$} \\
%         \hline
        $2.1\%$ & $3.5\%$ \\
          \begin{tikzpicture}
  \node[observed] (1) at (0*\edgeunit, 0*\edgeunit) {$1$};
  \node[observed] (2) at (0*\edgeunit, 1*\edgeunit) {$2$};
  \node[observed] (3) at (1*\edgeunit, 1*\edgeunit) {$3$};
  \node[observed] (4) at (1*\edgeunit, 0*\edgeunit) {$4$};

  \draw[edge] (1) to (4); \draw[edge] (2) to (3); \draw[edge] (2) to (4);
  \end{tikzpicture}
 &
          \begin{tikzpicture}
  \node[observed] (1) at (0*\edgeunit, 0*\edgeunit) {$1$};
  \node[observed] (2) at (0*\edgeunit, 1*\edgeunit) {$2$};
  \node[observed] (3) at (1*\edgeunit, 1*\edgeunit) {$3$};
  \node[observed] (4) at (1*\edgeunit, 0*\edgeunit) {$4$};
  \draw[edge] (1) to (3); \draw[edge] (2) to (3); \draw[edge] (2) to (4);
  \end{tikzpicture}
 \\
        \hline
        $34.1\%$ & $15.6\%$ \\
          \begin{tikzpicture}
  \node[observed] (1) at (0*\edgeunit, 0*\edgeunit) {$1$};
  \node[observed] (2) at (0*\edgeunit, 1*\edgeunit) {$2$};
  \node[observed] (3) at (1*\edgeunit, 1*\edgeunit) {$3$};
  \node[observed] (4) at (1*\edgeunit, 0*\edgeunit) {$4$};
  \draw[edge] (1) to (2); \draw[edge] (1) to (3); \draw[edge] (2) to (4);
  \end{tikzpicture}
 &
          \begin{tikzpicture}
  \node[observed] (1) at (0*\edgeunit, 0*\edgeunit) {$1$};
  \node[observed] (2) at (0*\edgeunit, 1*\edgeunit) {$2$};
  \node[observed] (3) at (1*\edgeunit, 1*\edgeunit) {$3$};
  \node[observed] (4) at (1*\edgeunit, 0*\edgeunit) {$4$};
  \draw[edge] (1) to (2); \draw[edge] (1) to (3); \draw[edge] (1) to (4);
  \end{tikzpicture}
 \\
        \hline
        $ < .1\%$ & \\ 
          \begin{tikzpicture}
  \node[observed] (1) at (0*\edgeunit, 0*\edgeunit) {$1$};
  \node[observed] (2) at (0*\edgeunit, 1*\edgeunit) {$2$};
  \node[observed] (3) at (1*\edgeunit, 1*\edgeunit) {$3$};
  \node[observed] (4) at (1*\edgeunit, 0*\edgeunit) {$4$};
  \draw[edge] (1) to (3); \draw[edge] (2) to (4); \draw[edge] (3) to (4);
  \end{tikzpicture}
 &
        \dots \\
      \end{tabular}
    \end{tabular}
    & 
    \hspace{-.02\textwidth}
    \begin{tabular}{p{.5\textwidth}}
      \pause
      \paragraph{Edge probabilities.} $\Pr\{(j, k) \in T \mid Y\}$ 
      \begin{center}
        \begin{tikzpicture}
  \node[observed] (1) at (0*\edgeunit, 0*\edgeunit) {$1$};
  \node[observed] (2) at (0*\edgeunit, 1*\edgeunit) {$2$};
  \node[observed] (3) at (1*\edgeunit, 1*\edgeunit) {$3$};
  \node[observed] (4) at (1*\edgeunit, 0*\edgeunit) {$4$};

  \draw[-, line width=2pt, color=black] (1) to (2); 
  \draw[-, line width=4pt, color=black] (1) to (3); 
  \draw[-, line width=2pt, color=black] (1) to (4);
  \draw[-, line width=.5pt, color=black] (2) to (3); 
  \draw[-, line width=4pt, color=black] (2) to (4); 
  \draw[-, line width=.1pt, color=black] (3) to (4);
  \end{tikzpicture}

% > round(50*Pedge)/10
%      [,1] [,2] [,3] [,4]
% [1,]  0.0  2.6  4.8  3.0
% [2,]  2.6  0.0  0.5  4.1
% [3,]  4.8  0.5  0.0  0.0
% [4,]  3.0  4.1  0.0  0.0
% > Pedge
%           [,1]       [,2]         [,3]         [,4]
% [1,] 0.0000000 0.52246119 9.540435e-01 6.057683e-01
% [2,] 0.5224612 0.00000000 9.751246e-02 8.201690e-01
% [3,] 0.9540435 0.09751246 0.000000e+00 4.565667e-05
% [4,] 0.6057683 0.82016896 4.565667e-05 0.000000e+00
 
      \end{center}
      
      \bigskip \bigskip \bigskip \pause
      \paragraph{Most probable edges.} 
      \begin{center}
        \begin{tikzpicture}
  \node[observed] (1) at (0*\edgeunit, 0*\edgeunit) {$1$};
  \node[observed] (2) at (0*\edgeunit, 1*\edgeunit) {$2$};
  \node[observed] (3) at (1*\edgeunit, 1*\edgeunit) {$3$};
  \node[observed] (4) at (1*\edgeunit, 0*\edgeunit) {$4$};

  \draw[edge] (1) to (2); \draw[edge] (1) to (3); \draw[edge] (1) to (4); \draw[edge] (2) to (4);
  \end{tikzpicture}

      \end{center}
      (not a tree) \refer{SRS19}
   \end{tabular}
  \end{tabular}
}

%====================================================================
\backupend
%====================================================================

%====================================================================
%====================================================================
\end{document}
%====================================================================
%====================================================================
  
  \hspace{-.025\textwidth}
  \begin{tabular}{cc}
    \begin{tabular}{p{.5\textwidth}}
    \end{tabular}
    & 
    \hspace{-.02\textwidth}
    \begin{tabular}{p{.5\textwidth}}
    \end{tabular}
  \end{tabular}

