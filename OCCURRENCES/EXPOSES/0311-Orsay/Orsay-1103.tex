\documentclass[dvips]{slides}
\usepackage{lscape}
\textwidth 19cm
\textheight 23cm 
\topmargin 0 cm 
\oddsidemargin  -1cm 
\evensidemargin  -1cm
%\renewcommand{\baselinestretch}{1.1}     

%\documentclass[dvips, 12pt]{d:/Latex/foiltex/foils}
%\LogoOff
%\usepackage{geometry}
%\geometry{landscape,a4paper,tmargin=1cm,bmargin=1cm,lmargin=2.5cm,rmargin=2.5cm}


                                % %%%%%%%%%%%%%%%%%%%%%%%%%%%%%%%%%%%%%%%%
                                % Graphics Gallery :                     %
                                % http://www.accessexcellence.com/AB/GG/ %
                                % %%%%%%%%%%%%%%%%%%%%%%%%%%%%%%%%%%%%%%%%


% Maths
\usepackage{amsfonts}
\usepackage{amsmath}
\usepackage{amssymb}
%\usepackage{c:/Stephane/Tex/AjoutSR/astats} 
%\usepackage{d:/Latex/astats} 
\newcommand{\Acal}{\mathcal{A}}
\newcommand{\Ccal}{\mathcal{C}}
\newcommand{\Ecal}{\mathcal{E}}
\newcommand{\Pcal}{\mathcal{P}}
\newcommand{\Wcal}{\mathcal{W}}
\newcommand{\att}{{\tt a}}
\newcommand{\ctt}{{\tt c}}
\newcommand{\gtt}{{\tt g}}
\newcommand{\ttt}{{\tt t}}
\newcommand{\mbf}{{\bf m}}
\newcommand{\Phibf}{\mbox{\mathversion{bold}{$\Phi$}}}
\newcommand{\Pibf}{\mbox{\mathversion{bold}{$\Pi$}}}
\newcommand{\Sbf}{{\bf S}}
\newcommand{\bps}{\mbox{bps}}
\newcommand{\vbf}{{\bf v}}
\newcommand{\wbf}{{\bf w}}
\newcommand{\Esp}{{\mathbb E}}
\newcommand{\Var}{{\mathbb V}}
\newcommand{\Indic}{{\mathbb I}}

% Couleur et graphiques
\usepackage{color}
\usepackage{graphics}
\usepackage{epsfig} 
\usepackage{pstcol}

% Texte
%\usepackage{d:/LATEX/fancyheadings}
%\usepackage{multirow}
\usepackage{enumerate}
\usepackage[french]{babel}
\usepackage[latin1]{inputenc}
%\definecolor{darkgreen}{cmyk}{0.5, 0, 0.5, 0.4}
\definecolor{darkgreen}{cmyk}{0.5, 0, 0.5, 0.4}
\definecolor{orange}{cmyk}{0, 0.6, 0.8, 0}
\definecolor{jaune}{cmyk}{0, 0.5, 0.5, 0}
\newcommand{\textblue}[1]{\textcolor{blue}{\bf #1}}
\newcommand{\textred}[1]{\textcolor{red}{\bf #1}}
%\newcommand{\textgreen}[1]{\textcolor{darkgreen}{\bf #1}}
\newcommand{\textgreen}[1]{\textcolor{green}{\bf #1}}
\newcommand{\textorange}[1]{\textcolor{orange}{\bf #1}}
\newcommand{\textyellow}[1]{\textcolor{yellow}{\bf #1}}
\newcommand{\emphase}[1]{{\bf #1}}
% Sections
\newcommand{\chapter}[1]{\centerline{\LARGE \bf \textgreen{#1}}}
\newcommand{\section}[1]{\centerline{\Large \bf \textyellow{#1}}}
\newcommand{\subsection}[1]{\noindent{\large \bf \textyellow{#1}}}
\newcommand{\paragraph}[1]{{\textgreen{#1}}}

% D�finition de dessins
\newcommand{\reddot}{\pscircle*[linecolor=red, fillcolor=red]}
\newcommand{\redcircle}{\pscircle[linecolor=red, linewidth=.05]}
\newcommand{\bluedot}{\pscircle*[linecolor=blue, fillcolor=blue]}
\newcommand{\bluesquare}{\psframe[linecolor=blue, fillcolor=blue, 
  linewidth=0.1]}
\newcommand{\greendot}{\pscircle*[linecolor=green, fillcolor=green]}
\newcommand{\greensquare}{\psframe*[linecolor=green, fillcolor=blue]}
\newcommand{\losange}{\psdots[dotstyle=diamond*, dotscale=4 4]}
\newcommand{\arrow}{\psline[linestyle=dotted, linewidth=.05]{->}}

%%%%%%%%%%%%%%%%%%%%%%%%%%%%%%%%%%%%%%%%%%%%%%%%%%%%%%%%%%%%%%%%%%%%%%
%%%%%%%%%%%%%%%%%%%%%%%%%%%%%%%%%%%%%%%%%%%%%%%%%%%%%%%%%%%%%%%%%%%%%%
\begin{document}
\landscape
\pagecolor{blue}
\color{white}
%%%%%%%%%%%%%%%%%%%%%%%%%%%%%%%%%%%%%%%%%%%%%%%%%%%%%%%%%%%%%%%%%%%%%%
%%%%%%%%%%%%%%%%%%%%%%%%%%%%%%%%%%%%%%%%%%%%%%%%%%%%%%%%%%%%%%%%%%%%%%

\begin{center}
  \textgreen{\Large MODELING MOTIF DISTRIBUTION}
  
  \textgreen{\Large IN BIOLOGICAL SEQUENCES}
  
  {\large St�phane ROBIN} \\
  {\tt robin@inapg.inra.fr}
  
  {UMR INA-PG / INRA, Paris} \\
  {Biom�trie \& Intelligence Artificielle}
  
  {Statistiques des s�quences biologiques (SSB): {\tt
      www-mig.jouy.inra.fr/ssb/}} 
  \vspace{1cm}

  {S�minaire ``Langage du g�nome'', Orsay, 6 / 11 / 03}
\end{center}

%%%%%%%%%%%%%%%%%%%%%%%%%%%%%%%%%%%%%%%%%%%%%%%%%%%%%%%%%%%%%%%%%%%%%%
%%%%%%%%%%%%%%%%%%%%%%%%%%%%%%%%%%%%%%%%%%%%%%%%%%%%%%%%%%%%%%%%%%%%%%
%%%%%%%%%%%%%%%%%%%%%%%%%%%%%%%%%%%%%%%%%%%%%%%%%%%%%%%%%%%%%%%%%%%%%%
%%%%%%%%%%%%%%%%%%%%%%%%%%%%%%%%%%%%%%%%%%%%%%%%%%%%%%%%%%%%%%%%%%%%%%
\newpage
% \section{Contents}
% %%%%%%%%%%%%%%%%%%%%%%%%%%%%%%%%%%%%%%%%%%%%%%%%%%%%%%%%%%%%%%%%%%%%%%
% \begin{enumerate}
% \item Biological interest of motif statistics \\
% \item A model: what for ? \\
% \item Motifs occurrences in Markov chains \\
% \item Compound Poisson model \\
% \item Motifs distribution along a sequence
% \end{enumerate}

%%%%%%%%%%%%%%%%%%%%%%%%%%%%%%%%%%%%%%%%%%%%%%%%%%%%%%%%%%%%%%%%%%%%%%
%%%%%%%%%%%%%%%%%%%%%%%%%%%%%%%%%%%%%%%%%%%%%%%%%%%%%%%%%%%%%%%%%%%%%%
%%%%%%%%%%%%%%%%%%%%%%%%%%%%%%%%%%%%%%%%%%%%%%%%%%%%%%%%%%%%%%%%%%%%%%
%%%%%%%%%%%%%%%%%%%%%%%%%%%%%%%%%%%%%%%%%%%%%%%%%%%%%%%%%%%%%%%%%%%%%%
\newpage
\chapter{Biological interest of motif statistics}
\chapter{~}
\section{Vocabulary}
%%%%%%%%%%%%%%%%%%%%%%%%%%%%%%%%%%%%%%%%%%%%%%%%%%%%%%%%%%%%%%%%%%%%%%
\paragraph{Letter} = nucleotide $\in \{\att, \ctt ,\gtt,
\ttt\}$ (= base pair)

\paragraph{Word} = short, exact sequence of letters:
$$
\wbf = \gtt\ctt\ttt\gtt\gtt\ttt\gtt\gtt
$$
\paragraph{Motif} = set of words:
$$
\begin{array}{rclcll}
  \mbf & = & \{\gtt{\tt N}\ttt\gtt\gtt\ttt\gtt\gtt\} 
  & = \{ & \wbf_1 = \gtt \emphase{\tt a} \ttt\gtt\gtt\ttt\gtt\gtt, \\
  & & &  & \wbf_2 = \gtt \emphase{\tt c} \ttt\gtt\gtt\ttt\gtt\gtt, \\
  & & &  & \wbf_3 = \gtt \emphase{\tt g} \ttt\gtt\gtt\ttt\gtt\gtt, \\
  & & &  & \wbf_4 = \gtt \emphase{\tt t} \ttt\gtt\gtt\ttt\gtt\gtt & \} \\
\end{array}
$$

%%%%%%%%%%%%%%%%%%%%%%%%%%%%%%%%%%%%%%%%%%%%%%%%%%%%%%%%%%%%%%%%%%%%%%
\newpage
\section{Three examples}
%%%%%%%%%%%%%%%%%%%%%%%%%%%%%%%%%%%%%%%%%%%%%%%%%%%%%%%%%%%%%%%%%%%%%%
\subsection{Ex 1: Promoter motifs} = structured motifs where
polymerase binds to DNA
$$
\begin{pspicture}(20, 2.5)(0, -2.5)
  \psline[linewidth=0.05, linestyle=dashed, linecolor=white]{<-|}(0,
  1.2)(14.2, 1.2) 
  \rput[B]{0}(6.85, 1.4){$\simeq$ 100 bps}

  \psline[linewidth=0.1, linecolor=white](0, 0)(3.3, 0)
  \rput[B]{0}(4, -0.1){\fbox{\rule[-0.2cm]{0cm}{0.8cm}$\;\vbf\;$}}
  \psline[linewidth=0.1, linecolor=white](4.7, 0)(7.3, 0)
  \rput[B]{0}(8, -0.1){\fbox{\rule[-0.2cm]{0cm}{0.8cm}$\;\wbf\;$}}
  \psline[linewidth=0.1, linecolor=white](8.7, 0)(14.2, 0)
  \rput[B]{0}(17, -0.1){\fbox{\rule[-0.2cm]{0cm}{0.8cm}\qquad gene\qquad}}
  
  \psline[linewidth=0.05, linestyle=dashed, linecolor=white]{<->}(4.7,
  -1)(7.2, -1) 
  \rput[B]{0}(5.95, -2){16 bps $\leq d \leq$ 18 bps}
\end{pspicture}
$$
Which structured motifs are \emphase{unexpectedly frequent} in
upstream regions of the genes of a given species?

%%%%%%%%%%%%%%%%%%%%%%%%%%%%%%%%%%%%%%%%%%%%%%%%%%%%%%%%%%%%%%%%%%%%%%
\newpage
\subsection{Ex 2: CHI motifs in bacterial genomes} 

{\sl Crossover Hot-spot Initiator}: \emphase{defense function} of the
genome against the degradation activity of an enzyme 

Exists in {\sl E. coli} = {\gtt\ctt\ttt\gtt\gtt\ttt\gtt\gtt} and {\sl H.
  influenza} = \gtt{\tt N}\ttt\gtt\gtt\ttt\gtt\gtt

Is this motif \emphase{unexpectedly frequent} in some regions of the
genome?

If so, these regions may contain crucial functions

%%%%%%%%%%%%%%%%%%%%%%%%%%%%%%%%%%%%%%%%%%%%%%%%%%%%%%%%%%%%%%%%%%%%%%
\newpage
\subsection{Ex 3: Palindromes} = self-complementary words 
$$
  \begin{pspicture}(10, 4.5)
    \psline[linewidth=0.1, linecolor=white]{->}(2, 3.5)(8, 3.5)
    \rput[B]{0}(5, 1.5){$
      \begin{array}{cccccc}
        \gtt  & \ttt & \ttt & \att & \att & \ctt \\
        |  & |  & |  & |  & |  & | \\
        \ctt & \att & \att & \ttt & \ttt & \gtt 
      \end{array}
      $}
    \psline[linewidth=0.1, linecolor=white]{<-}(2, 0)(8, 0)
  \end{pspicture}
$$

Palindromes of length 6 are restriction sites (i.e. \emphase{frailty
  sites}) of the genome of {\sl E. coli}

If they are \emphase{especially avoided} in some regions, these
regions may be of major importance for the organism

%%%%%%%%%%%%%%%%%%%%%%%%%%%%%%%%%%%%%%%%%%%%%%%%%%%%%%%%%%%%%%%%%%%%%%
%%%%%%%%%%%%%%%%%%%%%%%%%%%%%%%%%%%%%%%%%%%%%%%%%%%%%%%%%%%%%%%%%%%%%%
%%%%%%%%%%%%%%%%%%%%%%%%%%%%%%%%%%%%%%%%%%%%%%%%%%%%%%%%%%%%%%%%%%%%%%
%%%%%%%%%%%%%%%%%%%%%%%%%%%%%%%%%%%%%%%%%%%%%%%%%%%%%%%%%%%%%%%%%%%%%%
\newpage
\chapter{A model: what for ?}
\chapter{~}
\subsection{Model = Reference} \\
To be able to decide if something is \emphase{unexpected}, one
first need to know \emphase{what to expect} 
\begin{enumerate}[$\bullet$]
\item The model must be \emphase{well fitted} to the data to avoid artifacts and
  non-interesting discoveries
\item but \emphase{not too well} fitted to let unexpected events
  happen 
\end{enumerate}

\emphase{Interesting} cases are revealed when the model is (very)
\emphase{wrong}

The choice of the model depends on the problem
 
%%%%%%%%%%%%%%%%%%%%%%%%%%%%%%%%%%%%%%%%%%%%%%%%%%%%%%%%%%%%%%%%%%%%%%
\newpage
\subsection{Overlapping structure of the word} \\
Some words can \emphase{overlap themselves} \dots
$$
\begin{tabular}{lcccccccc|ccccccc}
  1 letter : &  &  &  &  &  &  &  & \emphase{g}& \gtt & \ttt & \gtt
  & \gtt & \ttt & \gtt & \gtt \\
  \hline
  \emphase{w} & \emphase{g} & \emphase{g} &  \emphase{t} 
  & \emphase{g} & \emphase{g} & \emphase{t} & \emphase{g} 
  & \emphase{g} &  &  &  &  &  &  & \\
  \hline
  5 letters : &  &  &  & \emphase{g} & \emphase{g} &
  \emphase{t} & \emphase{g} & \emphase{g} & \ttt & \gtt &
  \gtt &  &  &  & \\
  2 letters :  &  &  &  &  &  &  & \emphase{g} & \emphase{g} 
  & \ttt & \gtt & \gtt & \ttt & \gtt & \gtt & 
\end{tabular}
$$ {\sl Conway (Gardner, 74); Guibas \& Odlyzko, 81}

 \dots and tend therefore to occur in \emphase{clumps}
$$
\begin{tabular}{cc}
  $\wbf=({\tt gatc})$ & $\wbf=({\tt aaaa})$ \\ 
  \epsfig{file = ../RSS-03/RoD99-JAP-Fig1-1.eps, height = 6cm, width = 10cm} & 
  \epsfig{file = ../RSS-03/RoD99-JAP-Fig1-2.eps, height = 6cm, width = 10cm} \\
  $\begin{array}{ll}
    \mbox{M00}: & \Esp(Y) = 256\;\bps \\
    & \Var(Y) = (256.2\;\bps)^2
  \end{array}$
  &
  $\begin{array}{ll}
    \mbox{M00}: & \Esp(Y) = 256\;\bps \\
    & \Var(Y) = (326.7\;\bps)^2
  \end{array}$
\end{tabular}
$$

%%%%%%%%%%%%%%%%%%%%%%%%%%%%%%%%%%%%%%%%%%%%%%%%%%%%%%%%%%%%%%%%%%%%%%
\newpage
\subsection{Statistical issues}
\begin{enumerate}[$\bullet$]
\item \emphase{Probability for a motif to occur} in a
  sequence \\
  $\longrightarrow$ promoter motifs
\item Distribution of the \emphase{number} of occurrences ({\sl Prum
    \& al, Schbath, 95; Nicod�me \& al, 99; Regnier, 00; Nuel, 01})
\item \emphase{Distribution} of the occurrences along the
  sequence \\
  $\longrightarrow$ CHI motifs, palindromes
\end{enumerate}

\vspace{-0.5cm}
\paragraph{Positions, distances, counts} 
$$
                                %FigUnMot.ps
\begin{pspicture}(25, 8.1)
  \rput[bl](-1.5, 0){  \colorbox{white}{
      \psfig{file = ../RSS-03/FigUnMot.ps,
        height=8.1cm, width=25cm, bbllx=75 , bblly=525 , bburx=520 ,
        bbury=670 , clip=} } }
  \rput[Br](24.5, 6.5){\colorbox{white}{\textcolor{black}{$N(\wbf) = 6$}}}
\end{pspicture}
$$

%%%%%%%%%%%%%%%%%%%%%%%%%%%%%%%%%%%%%%%%%%%%%%%%%%%%%%%%%%%%%%%%%%%%%%
%%%%%%%%%%%%%%%%%%%%%%%%%%%%%%%%%%%%%%%%%%%%%%%%%%%%%%%%%%%%%%%%%%%%%%
%%%%%%%%%%%%%%%%%%%%%%%%%%%%%%%%%%%%%%%%%%%%%%%%%%%%%%%%%%%%%%%%%%%%%%
%%%%%%%%%%%%%%%%%%%%%%%%%%%%%%%%%%%%%%%%%%%%%%%%%%%%%%%%%%%%%%%%%%%%%%
\newpage
\chapter{Motifs occurrences in Markov chains}
\chapter{~}
\section{Markov chains = Discrete modeling}
%%%%%%%%%%%%%%%%%%%%%%%%%%%%%%%%%%%%%%%%%%%%%%%%%%%%%%%%%%%%%%%%%%%%%
$\Sbf = (S_1, \dots, S_{\ell})$ is an homogeneous stationary Markov
chain 
\vspace{-1cm}
\begin{enumerate}[$\bullet$]
\item of \emphase{order} $m$ (M$m$ model)
\item with \emphase{transition probabilities} $\pi(s_1, \dots,
  s_m; s_{m+1}) =$
  $$
  \Pr\{S_x = s_{m+1} | S_{x-m}=s_1, \dots, S_{x-1} = s_m\} 
  $$
\end{enumerate}
M$m$ model is fitted to the frequencies of \emphase{all the words of
  length $(m+1)$}
$$
\widehat{\pi}(s_1, \dots, s_m; s_{m+1}) = \frac{N(s_1 \dots s_m
  s_{m+1})}{N(s_1 \dots s_m)}
$$

\emphase{Theoretically}, properties derived under M1 can be
generalized to M$m$: M2 is equivalent to M1 on the alphabet $\Acal^2 =
\{\att\att, \att\ctt, \dots, \ttt\ttt\}$

%%%%%%%%%%%%%%%%%%%%%%%%%%%%%%%%%%%%%%%%%%%%%%%%%%%%%%%%%%%%%%%%%%%%%%
\newpage
\subsection{Principle for one word under M1}

\vspace{-1cm} \begin{flushright} 
  {\sl Blom \& Thorburn, 82} (M0); {\sl R. \& Daudin, 99} (M1) 
\end{flushright} \vspace{-1cm}


Distribution of the distance $Y$
$$
p(y) =\Pr \{ Y=y \}
$$
\begin{enumerate}
\item \emphase{Recursive formula} of order $y-1$ $(\emphase{O(y^2)})$
  $$
  p(y) = f [ p(1), \dots, p(y-1) ]
  $$
\item Probability \emphase{generating function}
  $$
  \phi_Y (t) = \sum_{y\geq 1}p(y) t^{y} = U_Y(t) / V_Y(t)
  $$
\item \emphase{Taylor expansion} of $\phi_Y$
  $$
  p(y) = g [ p(y-|\wbf|), \dots, p(y-1)]
  $$
  $\longrightarrow$ recurrence of order $|\wbf|$ $(\emphase{O(y)})$
\end{enumerate}

%%%%%%%%%%%%%%%%%%%%%%%%%%%%%%%%%%%%%%%%%%%%%%%%%%%%%%%%%%%%%%%%%%%%%%
\newpage
\subsection{Principle for a motif} 
\qquad \qquad {\sl R. \& Daudin, 01} (M1) 

Consider the distribution of the occurrences of the motif 
$$
\mbf = \{\wbf_1, \dots, \wbf_I\}
$$
The distribution of the distances \emphase{depends on the
  words} themselves 
$$
                                %FigMultiMot.ps  
  \hspace{-1.5cm}
  \colorbox{white}{
    \psfig{file = ../RSS-03/FigMultiMot.ps, height=4.8cm,
      width=25cm, bbllx=75, bblly=255 , bburx=520 , bbury=350 , clip=}
    }
$$

Steps 1, 2, 3 follow the same principle as for one word but involve
\emphase{generating matrices}

%%%%%%%%%%%%%%%%%%%%%%%%%%%%%%%%%%%%%%%%%%%%%%%%%%%%%%%%%%%%%%%%%%%%%%
\newpage
Denoting $\phi_{ij}(t) = \phi_{Y_{ij}}(t)$, ($i, j = 1..I$)
$$
\underset{I \times I}{\Phibf(t)} = \left[ 
  \begin{array}{ccc}
    \phi_{11}(t) & \dots & \phi_{1I}(t) \\
    \vdots       &       & \vdots       \\
    \phi_{I1}(t) & \dots & \phi_{II}(t) \\
  \end{array}
\right], 
\qquad \phi_{ij}(t) = \frac{U_{ij}(t)}{V_{ij}(t)}
$$

Step 2 requires the \emphase{inversion} of a generating matrix:
$$
\Phibf(t) = {\bf F}(t)[{\bf I} - {\bf F}(t)]^{-1}
$$
\vspace{0.5cm}
\paragraph{Limitations:}
\vspace{-1cm}
\begin{enumerate}[$\bullet$] 
\item \emphase{Complexity} of this last step: \emphase{$O(I^3
    |\mbf|)$}
\item \emphase{Numerical instability} except if $[{\bf I} -
  {\bf F}(t)]$ is inverted formally \\
  $\Longrightarrow$ small set of short words (small $I$ and
  $|\mbf|$)
\end{enumerate}
\paragraph{Other approaches} 
\vspace{-1cm}
\begin{enumerate}[$\bullet$]
\item grammars: {\sl Nicod�me, 00}
\item embedded Markov chain: {\sl Fu \& Koutras, 94};
{\sl Koutras, 97} 
\item porperties of the exponential family: {\sl Stefanov \& Pakes,
    99}
\end{enumerate}

%%%%%%%%%%%%%%%%%%%%%%%%%%%%%%%%%%%%%%%%%%%%%%%%%%%%%%%%%%%%%%%%%%%%%%
\newpage
\section{Application to structured motifs} 
%%%%%%%%%%%%%%%%%%%%%%%%%%%%%%%%%%%%%%%%%%%%%%%%%%%%%%%%%%%%%%%%%%%%%%
\paragraph{Difficulty:} Complexity of the overlapping structure of structured
motif 
$$
\begin{pspicture}(11, 1.75)(0, 0.75)
  \rput[B]{0}(1, 1){$\mbf = \framebox{\quad$\vbf$\quad}$}
  \psline[linewidth=0.1, linecolor=white]{<->}(3.2, 1.1)(8.7, 1.1)
  \rput[B]{0}(10, 1){\framebox{\quad$\wbf$\quad}}
  \rput[B](6, 1.3){$d$}
\end{pspicture}
$$
$\Longrightarrow$ (almost) impossible to calculate the exact distribution of
$X_1(\mbf)$

\paragraph{Approximation} ({\sl R. \& al, 02})
\vspace{-0.5cm}
\begin{enumerate}
\item Probability for $\mbf$ \emphase{to occur at a given
    position} (using the distribution of the distances): $ \mu(\mbf)$ \\
\item Approximation of \emphase{order 0} (geometric): 
  $$
  \Pr \left\{ N(\wbf) \geq 1 \right\}
  \approx
  1 - [1 - \mu(\mbf)]^{\ell - |\mbf| + 1} 
  $$
  does not work (\emphase{simulations}, next slide) \\
\item Approximation of \emphase{order 1}: 
  $$
  \Pr \left\{ N(\wbf) \geq 1 \right\}
  \approx
  1 - [1 - \mu(\mbf)] [1 - \mu_1(\mbf)]^{\ell-|\mbf|}
  $$
  where 
  $
  \mu_1(\mbf) = \Pr\{\mbf \mbox{ at }x | \mbf \mbox{ not at } x-1\}
  $ 
\end{enumerate}

%%%%%%%%%%%%%%%%%%%%%%%%%%%%%%%%%%%%%%%%%%%%%%%%%%%%%%%%%%%%%%%%%%%%%%
\newpage 
\paragraph{Comparison of approximations:} (order 0 left, order 1 right)
$$
\hspace{-2cm}
\begin{tabular}{lc}
%  \mbox{order 0} & \mbox{order 1} \\
  \begin{tabular}{l}
    $x$: \\
    simulations \\
    \\
    \\
    $y$: \\
    approximation \\
    \\
    \\
    dotted line: \\
    95\% conf. lim.
  \end{tabular}
  & 
  \hspace{-1.5cm}
  \begin{tabular}{cc}
    \epsfig{file = RDRSS02-FigSub0.eps, height = 12cm, width = 10cm} 
    & \epsfig{file = RDRSS02-FigSub1.eps, height = 12cm, width = 10cm}
  \end{tabular}
\end{tabular}
$$
Structure motifs for which order 0 works bad ($\circ$) are those
for which $\vbf$ is very overlapping (eg $\vbf =
\ttt\ttt\ttt\ttt\ttt\ttt$).
%%%%%%%%%%%%%%%%%%%%%%%%%%%%%%%%%%%%%%%%%%%%%%%%%%%%%%%%%%%%%%%%%%%%%%
\newpage
\hspace{-3.5cm}
\begin{tabular}{lc}
  \begin{tabular}{l}
    \textgreen{\Large Promoters} \\
    \textgreen{\Large in} \\
    \textgreen{\Large {\sl B. subtilis}:} \\
    \\
    131 upstream \\
    regions \\
    of 100 bps \\
    \\
    $p$-value \\
    $< 10^{-16}$ \\
    \\
    (putative \\
    alignment) \\
  \end{tabular}
  &
  {\small
    \begin{tabular}{lclccc}
%      $\qquad\vbf$ & $(d_1:d_2)$ & $\qquad\wbf$ & $N_{\mbox{obs}}$& $\Esp_1(N)$ \\
%      $\qquad\vbf$ & $(d_1:d_2)$ & $\qquad\wbf$ 
      \multicolumn{3}{c}{$\begin{array}{c}
          \mbf \\
          \overbrace{\vbf \hspace{2cm} (d_1:d_2) \hspace{2cm} \wbf} \\
        \end{array}$
        }
      & \begin{tabular}{c} number of\\ regions\\ containing
      $\mbf$\end{tabular} 
      & \begin{tabular}{c} expected\\ number\end{tabular}\\
      \hline 
      {\tt\ \ \ gttgaca\ } & $(16:18)$ & {\tt atataat}     &  7 &   2.43 $10^{-2}$ \\  
      {\tt\ \ \ gttgaca\ } & $(16:18)$ & {\tt \ tataata}   &  8 &   2.23 $10^{-2}$ \\  
      {\tt\ \ tgttgac\ \ } & $(16:18)$ & {\tt \ tataata}   & 10 &   2.12 $10^{-2}$ \\
      {\tt\ \ \ \ ttgacaa} & $(16:18)$ & {\tt \ tacaat}    &  9 &   9.82 $10^{-2}$ \\
      {\tt\ \ \ \ ttgacaa} & $(16:18)$ & {\tt \ tataata}   & 10 &   5.07 $10^{-2}$ \\
      {\tt\ \ \ \ ttgacag} & $(16:18)$ & {\tt \ tataat}    &  9 &   7.12 $10^{-2}$ \\
      {\tt\ \ \ \ ttgacaa} & $(17:19)$ & {\tt \ \ ataataa} &  9 &   6.97 $10^{-2}$ \\
      {\tt \ ttgttga\ \ } & $(17:19)$ & {\tt \ tataata}    &  8 &   5.17 $10^{-2}$ \\
      {\tt\ \ \ gttgaca\ } & $(17:19)$ & {\tt \ \ ataataa} &  8 &   3.09 $10^{-2}$ \\
      {\tt\ \ \ gttgaca\ } & $(17:19)$ & {\tt \ tataata}   &  8 &   2.19 $10^{-2}$ \\
      {\tt\ \ \ cttgaca\ } & $(17:19)$ & {\tt \ tataat}    &  8 &   6.04 $10^{-2}$ \\
      {\tt\ \ tgttgac\ \ } & $(17:19)$ & {\tt \ tataata}   & 12 &   2.09 $10^{-2}$ \\
      {\tt\ \ tgttgac\ \ } & $(17:19)$ & {\tt atataat}     &  7 &   2.29 $10^{-2}$ \\
      {\tt\ ttgttga\ \ \ } & $(18:20)$ & {\tt \ tataata}   &  8 &   5.09 $10^{-2}$ \\
      {\tt\ \ \ gttgaca\ } & $(18:20)$ & {\tt \ \ ataatga} &  7 &   1.79 $10^{-2}$ \\
      {\tt gttgttg\ \ \ \ } & $(18:20)$ & {\tt \ tataata}  &  7 &   2.53 $10^{-2}$ \\
      {\tt\ \ tgttgac\ \ } & $(18:20)$ & {\tt \ \ ataataa} & 10 &   2.90 $10^{-2}$ \\
      {\tt\ \ tgttgac\ \ } & $(18:20)$ & {\tt \ \ atacta}  &  7 &   2.77 $10^{-2}$ \\
      {\tt\ \ tgttgac\ \ } & $(19:21)$ & {\tt \ \ ataataa} & 10 &   2.86 $10^{-2}$ \\
      {\tt\ \ tgttgac\ \ } & $(19:21)$ & {\tt \ \ atacta}  &  7 &   2.73 $10^{-2}$ \\
      {\tt\ \ tgttgac\ \ } & $(19:21)$ & {\tt \ \ \ \ tataat}    & 10 &   6.53 $10^{-2}$ \\
      {\tt\ \ \ gttgact\ } & $(19:21)$ & {\tt \ \ \ \ \ ataata}  &  8 &   6.25 $10^{-2}$ \\
    \end{tabular}
    }
\end{tabular}

%%%%%%%%%%%%%%%%%%%%%%%%%%%%%%%%%%%%%%%%%%%%%%%%%%%%%%%%%%%%%%%%%%%%%%
%%%%%%%%%%%%%%%%%%%%%%%%%%%%%%%%%%%%%%%%%%%%%%%%%%%%%%%%%%%%%%%%%%%%%%
%%%%%%%%%%%%%%%%%%%%%%%%%%%%%%%%%%%%%%%%%%%%%%%%%%%%%%%%%%%%%%%%%%%%%%
%%%%%%%%%%%%%%%%%%%%%%%%%%%%%%%%%%%%%%%%%%%%%%%%%%%%%%%%%%%%%%%%%%%%%%
\newpage
\chapter{Compound Poisson model}
\chapter{~}
\section{Compound Poisson process}
\section{= Continuous modeling} 
%%%%%%%%%%%%%%%%%%%%%%%%%%%%%%%%%%%%%%%%%%%%%%%%%%%%%%%%%%%%%%%%%%%%%%
For rare words, the sequence $\Sbf$ can be viewed as a
\emphase{continuous line} $[0; \ell]$ 
$$
\hspace{-1.5cm}
\begin{tabular}{c}
  Real occurrences  \\
                                %FigPoissonComp.ps
  \colorbox{white}{      
    \psfig{file = ../RSS-03/FigPoissonComp.ps,
      height=2cm, width=25cm, bbllx=80 , bblly=370 , bburx=525 ,
      bbury=405, clip=}
    }  \\
  \\
  Compound Poisson modeling \\
                                %FigPoissonComp.ps
  \colorbox{white}{
    \psfig{file = ../RSS-03/FigPoissonComp.ps,
      height=3.9cm, width=25cm, bbllx=80 , bblly=260 , bburx=525 ,
      bbury=330, clip=}
    }  \\
\end{tabular}
$$
%%%%%%%%%%%%%%%%%%%%%%%%%%%%%%%%%%%%%%%%%%%%%%%%%%%%%%%%%%%%%%%%%%%%%%
\newpage
\paragraph{Clump process} $\{C(x)\} =$ Poisson process
$\Pcal(\lambda x)$

\paragraph{Clump sizes} $\{K_1, K_2, \dots\}$ are i.i.d. 
$$
Pr\{K = k\} = g(k)
$$

\paragraph{Counting process} of the occurrences $\{N(x)\} = $
compound Poisson process:
$$
N(x) = \sum_{c = 1}^{C(x)} K_c
$$

\emphase{Non overlapping} word $\Longrightarrow$ \emphase{simple Poisson}
process

\paragraph{Interpretation:} Poisson modeling implies that the clump are
  \emphase{uniformly distributed} along the genome \\
  \centerline{$\longrightarrow$ \emphase{Null hypothesis} of the next
    part}

%%%%%%%%%%%%%%%%%%%%%%%%%%%%%%%%%%%%%%%%%%%%%%%%%%%%%%%%%%%%%%%%%%%%%%
\newpage
\subsection{P\'olya-Aeppli model} \\
When considering one single word $\wbf$, the clump sizes have a
\emphase{geometric distribution}$$
g(k) = a^{k-1} (1-a)
\quad \Longrightarrow \quad
\Esp(K) = 1 / (1-a)
$$
where $a$ is the overlapping probability of $\wbf$

\paragraph{Parameter estimates:} In a sequence of length $\ell$
\begin{enumerate}[$\bullet$]
\item $\widehat{\lambda}$ is the empirical frequency of the clumps:
  $$
  \widehat{\lambda} = \frac{C(\ell)}{\ell}
  $$
\item $\widehat{a}$ is the proportion of overlapped occurrences:
  $$
  \widehat{a} = \frac{N(\ell) - C(\ell)}{C(\ell)}
  $$
\end{enumerate}


%%%%%%%%%%%%%%%%%%%%%%%%%%%%%%%%%%%%%%%%%%%%%%%%%%%%%%%%%%%%%%%%%%%%%%
\newpage
\paragraph{Properties}
\begin{enumerate}[$\bullet$]
\item In the P\'olya-Aeppli model, distances $Y$ are
  \emphase{i.i.d.} with mixture distribution
  $$
  \begin{array}{rclll}
    Y & = & 0 & \mbox{with probability }a & \mbox{(overlap)} \\
    & \sim & \Ecal(\lambda) & \mbox{with probability }1-a & \mbox{(no overlap)}
  \end{array}
  $$ 
  \smallskip
\item P\'olya-Aeppli is the \emphase{best approximation} of the
  distribution of the word count in the Markov
  model \\
\item $\Esp[N(\ell)] = \ell \times \lambda \times \Esp(K)$ 
  $$
  \Longrightarrow \quad
  \widehat{\Esp}N(\ell) = \ell \widehat{\lambda} / (1-\widehat{a}) = N(\ell)
  $$
  \centerline{$\Longrightarrow \quad$ \emphase{no word has an ``unexpected''
      count}}
\end{enumerate}

%%%%%%%%%%%%%%%%%%%%%%%%%%%%%%%%%%%%%%%%%%%%%%%%%%%%%%%%%%%%%%%%%%%%%%
\newpage
\section{Clump size modeling}
%%%%%%%%%%%%%%%%%%%%%%%%%%%%%%%%%%%%%%%%%%%%%%%%%%%%%%%%%%%%%%%%%%%%%%
\vspace{-1cm} \begin{flushright} {\sl R., 02} \end{flushright} \vspace{-1cm}

In the general case (e.g. \emphase{motif} $\mbf =\{\wbf_1$, $\wbf_2$,
$\dots$\}), the clump sizes do not have a geometric distribution

One may use
\vspace{-1cm}
\begin{enumerate}[$\bullet$]
\item empirical estimates of an \emphase{arbitrary} distribution $g(k)$ 
\item empirical estimates of the \emphase{overlapping
    probabilities} between words $\wbf_1$, $\wbf_2$, $\dots$
\item \emphase{Markov estimates} of the overlapping probabilities
\end{enumerate}

However, distances $Y$ between words are \emphase{not i.i.d.}
%%%%%%%%%%%%%%%%%%%%%%%%%%%%%%%%%%%%%%%%%%%%%%%%%%%%%%%%%%%%%%%%%%%%%%
%%%%%%%%%%%%%%%%%%%%%%%%%%%%%%%%%%%%%%%%%%%%%%%%%%%%%%%%%%%%%%%%%%%%%%
%%%%%%%%%%%%%%%%%%%%%%%%%%%%%%%%%%%%%%%%%%%%%%%%%%%%%%%%%%%%%%%%%%%%%%
%%%%%%%%%%%%%%%%%%%%%%%%%%%%%%%%%%%%%%%%%%%%%%%%%%%%%%%%%%%%%%%%%%%%%%
\newpage
\chapter{Motifs distribution along a sequence}
\chapter{~}
\section{Two statistics}
%%%%%%%%%%%%%%%%%%%%%%%%%%%%%%%%%%%%%%%%%%%%%%%%%%%%%%%%%%%%%%%%%%%%%%
We aim to detect \emphase{poor} or \emphase{rich} regions in terms of
occurrences of a given motif

A natural criterion for a given region is the \emphase{ratio}
$$
\frac{\mbox{\emphase{number of occurrences} in the
    region}}{\mbox{\emphase{size} of the region}}
$$

%%%%%%%%%%%%%%%%%%%%%%%%%%%%%%%%%%%%%%%%%%%%%%%%%%%%%%%%%%%%%%%%%%%%%%
\newpage
\paragraph{Cumulated distances:} 
\begin{tabular}{c}
  fixed numerator \emphase{$r$} \\
  \hline
  random denominator \emphase{$Y^r$}
\end{tabular}\\ 
\vspace{-1.25cm} \begin{flushright} 
  {\sl Karlin \& Macken, 91}: $r$-scans
\end{flushright} \vspace{-1cm}

 
$$
\hspace{-1.5cm}
                                %FigHomogene.ps
\colorbox{white}{
  \psfig{file = ../RSS-03/FigHomogene.ps, height=7.6cm,
    width=25cm, bbllx=70 , bblly=590 , bburx=515 , bbury=725 , clip=}  
  }
$$

\paragraph{Moving windows:} 
\begin{tabular}{c}
  random numerator \emphase{$\Delta N$} \\
  \hline
  fixed denominator \emphase{$y$}
\end{tabular} 

%%%%%%%%%%%%%%%%%%%%%%%%%%%%%%%%%%%%%%%%%%%%%%%%%%%%%%%%%%%%%%%%%%%%%%
\newpage
\subsection{Distribution of the statistics}
\qquad \qquad \qquad {\sl R., 02}

\paragraph{Cumulated distance:} the distribution of 
$$
Y_i^r = \sum_{j=i}^{i+r-1} Y_j  = X_{i+r} - X_i
$$
is known \emphase{when the distances $Y_i$ are i.i.d.} (e.g. in the
one word case) for Markov and compound Poisson models

\paragraph{Moving window:} the distribution of
$$
\Delta N(x) =  N(x) - N(x-y)
$$
is known for Markov and compound Poisson models ({\sl Barbour \&
  al, 92})

%%%%%%%%%%%%%%%%%%%%%%%%%%%%%%%%%%%%%%%%%%%%%%%%%%%%%%%%%%%%%%%%%%%%%%
\newpage
\section{Extremal statistics}
We are interested in the \emphase{richest} region, i.e.
$$
Y^r_{\min} = \min_i \{Y_i^r\}\qquad \mbox{or} \qquad \Delta N_{\sup} = \sup_x \{\Delta N(x)\}
$$

\subsection{Poisson approximation} \\
If the $\{Y_i^r\}$ or the $\{\Delta N(x)\}$ were \emphase{independent}
$$
\Pr\{Y^r_{\min} \leq y\} \underset{n \rightarrow
  \infty}{\longrightarrow} \exp[-(n-r)\Pr\{Y^r \leq y\}]
$$
$$
\Pr\{\Delta N_{\sup} > n\} \underset{\ell \rightarrow
  \infty}{\longrightarrow} \exp[-(\ell - y)\Pr\{\Delta N > n\}]
$$

%%%%%%%%%%%%%%%%%%%%%%%%%%%%%%%%%%%%%%%%%%%%%%%%%%%%%%%%%%%%%%%%%%%%%%
\newpage
\subsection{Chen-Stein method}\\
$\{Y_i^r\}$ and $\{\Delta N(x)\}$ are \emphase{not independent} but
the quality of the Poisson approximation can be controlled in total
variation distance ({\sl Arratia \& al, 89}):
$$
\max_y \left| \Pr\{Y^r_{\min} \leq y\} - e^{-(n-r)\Pr\{Y^r \leq y\}}\right|
\leq \mbox{bound}
$$

\paragraph{Cumulated distances:} an \emphase{explicit bound} can be calculated 
({\sl Dembo \& Karlin, 92})

\paragraph{Moving windows:} \emphase{no explicit bound} can be
derived, but this approximation is \emphase{optimal} ({\sl Barbour \&
  Brown, 92})
 
%%%%%%%%%%%%%%%%%%%%%%%%%%%%%%%%%%%%%%%%%%%%%%%%%%%%%%%%%%%%%%%%%%%%%%
\newpage 
\section{Applications}
%%%%%%%%%%%%%%%%%%%%%%%%%%%%%%%%%%%%%%%%%%%%%%%%%%%%%%%%%%%%%%%%%%%%%%
\subsection{CHI motif in {\sl H. influenza}} \\
In terms of overlap, $\mbf = (\gtt{\tt N}\ttt\gtt\gtt\ttt\gtt\gtt)$
behaves as one single word 

$\longrightarrow$ \emphase{cumulated distances} can be used 

\paragraph{Number of occurrences:} $\ell = 1\;903\;356$ bps
$$
\begin{tabular}{lcl}
  observed number of occurrences & = & 223 \\
  expected under Markov (M1) & = & \emphase{58.5} \\
  expected under compound Poisson & = & 223 \\
\end{tabular}
$$

\paragraph{Significancy thresholds:} for $\alpha = 5\%$ 
$$
\begin{tabular}{lr}
  for $Y^r$: & 6 312 bps \\
  for $\displaystyle{\min_{i=1...222}Y_i^r}$: & 238 bps 
\end{tabular}
$$
%%%%%%%%%%%%%%%%%%%%%%%%%%%%%%%%%%%%%%%%%%%%%%%%%%%%%%%%%%%%%%%%%%%%%%
\newpage
\paragraph{Distribution:} cumulated distances of order $r=3$ 

$$
\hspace{-1.5cm}
\begin{tabular}{cc}
  \multicolumn{2}{c}{plot of the \emphase{ratio} $3/Y^3$ ($\times
    10^{-3}$) versus the 
    \emphase{position} $x$} \\
  \\
  \psfig{file = ../RSS-03/Rob02-JRSSC-Fig2-1.eps, height=8cm, width=12cm} &
  \psfig{file = ../RSS-03/Rob02-JRSSC-Fig2-2.eps, height=8cm, width=12cm} \\
  Markov (M1) & compound Poisson  \\
  \emphase{overall bias} &  \emphase{no significant peak} \\
\end{tabular}
$$

%%%%%%%%%%%%%%%%%%%%%%%%%%%%%%%%%%%%%%%%%%%%%%%%%%%%%%%%%%%%%%%%%%%%%%
\newpage 
\paragraph{Remarks:}
\begin{enumerate}
\item[$\bullet$] \emphase{Markov model M7} would be unbiased (since
  $|\mbf| = 8$) but involves more than \emphase{$12\;000$ parameters} \\
\item[$\bullet$] In the compound Poisson model, the peak around 1.0 Mb
  (replication termination) is significant \emphase{on its own}:
  \begin{eqnarray*}
  \Pr\{Y^3 \leq 208\} = 1.6 10^{-4} \\
  \\
  \Pr\left\{\min_{i=1..220}\left(Y_i^3\right) \leq 208\right\} > 0.05
  \end{eqnarray*}
\end{enumerate}

%%%%%%%%%%%%%%%%%%%%%%%%%%%%%%%%%%%%%%%%%%%%%%%%%%%%%%%%%%%%%%%%%%%%%%
\newpage
\subsection{Palindromes in {\sl E. coli}} ({$\ell = 4\;638\;868$)\\
There are 64 palindromes of length 6\\
They occur 54\;724 times in 50\;941 clumps

\paragraph{Clump size:} Because of their overlapping structure, clumps 
can not be considered as geometric \\
$\longrightarrow$ \emphase{Moving windows}

We use a \emphase{parsimonious modeling} based the overlapping
probabilities given by the \emphase{M0 model} (4 parameters)

\paragraph{Results: } Moving windows of width $y = 10\;000$ bps
\begin{enumerate}
\item[$\bullet$] \emphase{Poorest region}: 73 occurrences ($p$-value $>
  10\%$) \\
  \centerline{\emphase{non significant}}
\item[$\bullet$] \emphase{Richest region}: 185 occurrences ($p$-value
  $< 5\%$) \\
  \centerline{\emphase{[2\;460\;567 bps; 2\;461\;566 bps]}}
\end{enumerate}
... interpretation: horizontal transfert?
%%%%%%%%%%%%%%%%%%%%%%%%%%%%%%%%%%%%%%%%%%%%%%%%%%%%%%%%%%%%%%%%%%%%%%
\newpage 
\section{Distribution in heterogeneous sequences} 
%%%%%%%%%%%%%%%%%%%%%%%%%%%%%%%%%%%%%%%%%%%%%%%%%%%%%%%%%%%%%%%%%%%%%%
Biological sequences are generally heterogeneous because of succession
of different ``states'' (coding / non-coding, $\gtt$-$\ttt$ content,
etc.)

\paragraph{Aim:} Account for this heterogeneity to avoid statistical
biases. 

\paragraph{Example:} $\gtt$-$\ttt$ rich motif are more frequent in
$\gtt$-$\ttt$ rich regions.

\paragraph{Prior information:} 
The segmentation of the sequence related to this heterogeneity is
often known \\
$\bullet \quad$ sequence annotations give coding / non-coding regions
\\
$\bullet \quad$ HMM provide a probabilistic segmentation based on the
sequence composition \\

\centerline{$\Longrightarrow \quad \pi_s(x) = \Pr\{\mbox{position } x \in \mbox{state
  } s\}$ \emphase{known a priori}}

%%%%%%%%%%%%%%%%%%%%%%%%%%%%%%%%%%%%%%%%%%%%%%%%%%%%%%%%%%%%%%%%%%%%%%
\newpage
\paragraph{Model:} \hspace{14cm} {\sl Ledent \& R., 03} 

$\bullet \quad$ Clump process $\{C(x)\}$ is an \emphase{heterogeneous Poisson} process with intensity
$$
\lambda(x) = \sum_s \lambda_s \pi_s(x)
$$
$\bullet \quad$ Clump sizes $\{K_c\}$ have heterogeneous
distributions:
$$
g(k \;|\; x) = \sum_s \pi_s(x) g_s(k)
$$
typically
$$
g(k \;|\; x) = \sum_s \pi_s(x) a_s^{k-1} (1-a_s)
$$

Once the parameters are estimated, the occurrences processes can be
\emphase{homogenized} and former techniques can be applied to the
homogenized data

%%%%%%%%%%%%%%%%%%%%%%%%%%%%%%%%%%%%%%%%%%%%%%%%%%%%%%%%%%%%%%%%%%%%%%
\newpage
\paragraph{3 steps analysis:} 

\hspace{-4cm}
\begin{tabular}{rl}
  \colorbox{white}{
    \epsfig{file = ../RSS-03/lambda-aatt-hetero.ps, height = 9cm, width =
      12cm} 
    } & 
  \colorbox{white}{
    \epsfig{file = ../RSS-03/lambda-aatt-homo.ps, height = 9cm, width =
      12cm} 
    } \\
  $\left.
    \begin{tabular}{r}
      Estimate an \emphase{heterogeneous} \emphase{Poisson} \\
      \emphase{process} with some \emphase{prior heterogeneity} \\
      (coding / non-coding, $\gtt$-$\ctt$ content, etc)
    \end{tabular}
  \right\}$
  & {\bf 1} \\
  {\bf 2} &
  $\left\{
    \begin{tabular}{l}
      ``\emphase{Homogenize}'' the scan process \\
      and calculate thresholds 
    \end{tabular}
  \right.$ \\
  $\left.
    \begin{tabular}{r}
      come back to the original process
    \end{tabular}
  \right\}$
  &
  {\bf 3}
\end{tabular}

%%%%%%%%%%%%%%%%%%%%%%%%%%%%%%%%%%%%%%%%%%%%%%%%%%%%%%%%%%%%%%%%%%%%%%
%%%%%%%%%%%%%%%%%%%%%%%%%%%%%%%%%%%%%%%%%%%%%%%%%%%%%%%%%%%%%%%%%%%%%%
\newpage
\chapter{Work in progress}

\paragraph{Comparing exceptionalities:} \hspace{6cm}({\sl with
  S. Schbath}) 

The word $\wbf$ is unexpectedly frequent in sequence $\Sbf_1$ and
$\Sbf_2$ (with different lengths and compositions) \\
\centerline{Is $\wbf$ more exceptional in $\Sbf_1$ than in $\Sbf_2$?} \\

\paragraph{Refining results about structured motifs:}
\begin{flushright}
  \vspace{-1cm}
  ({\sl with S. Schbath and V. Stefanov}) 
\end{flushright}

$\bullet \quad$ exact results on the distribution of structured motifs \\
$\bullet \quad$ extension to more complex motifs (3 boxes ?)

%%%%%%%%%%%%%%%%%%%%%%%%%%%%%%%%%%%%%%%%%%%%%%%%%%%%%%%%%%%%%%%%%%%%%%
%%%%%%%%%%%%%%%%%%%%%%%%%%%%%%%%%%%%%%%%%%%%%%%%%%%%%%%%%%%%%%%%%%%%%%

%%%%%%%%%%%%%%%%%%%%%%%%%%%%%%%%%%%%%%%%%%%%%%%%%%%%%%%%%%%%%%%%%%%%%%
%%%%%%%%%%%%%%%%%%%%%%%%%%%%%%%%%%%%%%%%%%%%%%%%%%%%%%%%%%%%%%%%%%%%%%
%%%%%%%%%%%%%%%%%%%%%%%%%%%%%%%%%%%%%%%%%%%%%%%%%%%%%%%%%%%%%%%%%%%%%%
%%%%%%%%%%%%%%%%%%%%%%%%%%%%%%%%%%%%%%%%%%%%%%%%%%%%%%%%%%%%%%%%%%%%%%
\end{document}
%%%%%%%%%%%%%%%%%%%%%%%%%%%%%%%%%%%%%%%%%%%%%%%%%%%%%%%%%%%%%%%%%%%%%%
%%%%%%%%%%%%%%%%%%%%%%%%%%%%%%%%%%%%%%%%%%%%%%%%%%%%%%%%%%%%%%%%%%%%%%
%%%%%%%%%%%%%%%%%%%%%%%%%%%%%%%%%%%%%%%%%%%%%%%%%%%%%%%%%%%%%%%%%%%%%%
%%%%%%%%%%%%%%%%%%%%%%%%%%%%%%%%%%%%%%%%%%%%%%%%%%%%%%%%%%%%%%%%%%%%%%

