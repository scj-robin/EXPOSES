\documentclass[11pt,a4paper,french]{article}

%%%%%%%%%%%%%%%%%%%%%%%%%%%%%%%
% COPIE DU RESUME POUR LA RSS %
%%%%%%%%%%%%%%%%%%%%%%%%%%%%%%%


\usepackage{enumerate}
\usepackage{amsmath}

\textwidth=16cm
\textheight=22.5cm
\oddsidemargin=0cm
\evensidemargin 0 cm
\topmargin=0cm
%\makeatother

%%%%%%%%%%%%%%%%%%%%%%%%%%%%%%%%%%%%%%%%%%%%%%%%%%%%%%%%%%%
\begin{document}
%%%%%%%%%%%%%%%%%%%%%%%%%%%%%%%%%%%%%%%%%%%%%%%%%%%%%%%%%%%

\begin{center}
  {\Large \bf Motif distribution in DNA sequences}

  {\large St\'ephane {\sc Robin} \\
  INA P-G / INRA, Biom\'etrie \\
  16, rue Claude Bernard, 75005 Paris, {\sc France} \\
  {\tt robin@inapg.inra.fr}}

\end{center}

Motifs statistics are part of basic tools in DNA sequence analysis.
In this presentation, we define a motif (or pattern) as a short (a
dozen at most) sequence of nucleotides {\tt a}, {\tt c}, {\tt g} or
{\tt t}.  A motif can be defined exactly (it is then called a word) or
admit few errors, insertions or deletions. The underlying idea of the
statistical analysis of a motif is that its presence, frequency or
distribution provides informations regarding its biological function.

%%%%%%%%%%%%%%%%%%%%%%%%%%%%%%%%%%%%%%%%%%%%%%%%%%%%%%%
\section*{Three examples}

\paragraph{Promotor sites.}
A promotor is a motif appearing upstream from a gene and to which the
enzyme responsible of the transcription binds. These motifs are
generally unknown but the structure of the enzyme provides some
information about their structure. We aim to discover promotor motifs
by looking for motifs having the right structure and being frequently
present upstream from known genes.

\paragraph{CHI sites and palindromes.}
Some motifs, like CHI in several bacteria, protect the genome against
its degradation by some specific enzyme. Some others, like palindromes
in {\em E. coli}, are frailty sites. The former must have been
selected by evolution while the latter must have been avoided. Motifs
of both kinds may be detected by looking at motifs with unexpectedly
high or unexpectedly low frequencies.

\paragraph{Rich regions in CHI occurrences.}
Following the last example, one may also think that evolution may have
concentrated occurrences of CHI (resp. avoided ocurrences of
palindromes) in regions that are essential for the genome. One may
look for such crucial regions by searching regions that are
unexpectedly rich (resp. poor) in certain motifs. Such an analysis is
based on the distribution of these motifs along the genome.

%%%%%%%%%%%%%%%%%%%%%%%%%%%%%%%%%%%%%%%%%%%%%%%%%%%%%%%
\section*{Two models}

In the three preceding examples, we are considering motifs having an
unexpected behaviour. The need for a statistical modelling raises
immediately: to know if something is unexpected, one first needs to
define what is expected.  In these analyses, the statistical model
provides a reference in terms of probability of presence (ex. 1),
expected frequency (ex. 2) or expected distribution (ex. 3). Observed
data are then compared to this reference to decide whether what we
observe is significant or not.

\paragraph{Markov chains.}
The most popular models in genomics are Markov chains that consider
DNA as a sequence of random letters, the distribution of each letter
depending on the few preceding ones.  We present here some theoretical
results about the waiting time till a motif occur and the distance
between occurrences in first order Markov chains. These results are
obtained through the generating function of the associated random
values. In particular, we show that the distribution of the distance
between occurrences strongly depends on the overlapping structure of
the motif.  Knowing the distribution of the first occurrence and of
the distance between occurrences, we derive the exact distributions of
($i$) the position of any occurrence (second, third, etc.), ($ii$) the
number of occurrences in the sequence, ($iii$) and the cumulated
distances ({\em scans}). \\
These results are used to discover promotor motifs (ex. 1) and study
the distribution of CHI sites (ex. 3). The latter analysis is based on
cumulated distances and bounds for the associated $p$-value are
obtained thanks to a Chen-stein type Poisson approximation.

This generating function approach appears to lead to heavy
computations that become prohibitive for high order Markov models.
This is a strong limitation since low order Markov models induce
strong biases when applied to long motifs.

\paragraph{Compound Poisson process.}  
We propose an alternative modelling in which the DNA sequence is
considered as a continuous line along which motifs occur as points.
Simple Poisson process is not satisfying in the general case because
of possible of overlaps between occurrences. We use a compound Poisson
process that assumes that the occurrences of the motif appear in
clumps (that occur according to a simple Poisson process) with random
sizes. \\
We pay a special attention to the modelling of clump size, and
especially to the geometric case. In this case --always valid for a
single word--, distances between occurrences are iid and cumulated
distances can be used to study the distribution of the occurrences
along the sequence (ex. 3, CHI motif). When the geometric hypothesis
does not hold, the analysis of the distribution can be performed using
sliding windows, but the theoretical results are harder to derive.


%\nocite{RoD99} \nocite{RoD01} \nocite{RoS01} \nocite{Rob02} \nocite{RDR02}
%\bibliography{../../SSB}
%\bibliographystyle{c:/Stephane/Tex/AjoutSR/astats}

%%%%%%%%%%%%%%%%%%%%%%%%%%%%%%%%%%%%%%%%%%%%%%%%%%%%%%%
\section*{References}
\begin{description}
\item[{\sc Robin, S.}]  \newblock (2002).  \newblock A compound
  {P}oisson model for words occurrences in {DNA} sequences.  \newblock
  {\em J. R. Statist. Soc. C}.  \newblock {\bf 51} 437--451.
  
\item[{\sc Robin, S.} \textnormal{and} {\sc Daudin, J.-J.}]  \newblock
  (1999).  \newblock Exact distribution of word occurrences in a
  random sequence of letters.  \newblock {\em J. Appl. Prob.}
  \newblock {\bf 36} 179--193.

\item[{\sc Robin, S.} \textnormal{and} {\sc Daudin, J.-J.}]  \newblock
  (2001).  \newblock Exact distribution of the distances between any
  occurrences of a set of words.  \newblock {\em Ann. Inst. Statist.
    Math.}  \newblock {\bf 36}~{\bf (4)} 895--905.

\item[{\sc Robin, S.}, {\sc Daudin, J.-J.}, {\sc Richard, H.}, {\sc
    Sagot, M.-F.} \textnormal{and} {\sc Schbath, S.}]  \newblock
  (2002).  \newblock Occurrence probability of structured motifs in
  random sequences.  \newblock {\em J. Comp. Biol.}  \newblock
  761--773.

\item[{\sc Robin, S.} \textnormal{and} {\sc Schbath, S.}]  \newblock
  (2001).  \newblock Numerical comparison of several approximations of
  the word count distribution in random sequences.  \newblock {\em J.
    Comp. Biol.}  \newblock {\bf 8}~{\bf (4)} 349--359.

\end{description}

%%%%%%%%%%%%%%%%%%%%%%%%%%%%%%%%%%%%%%%%%%%%%%%%%%%%%%% 
\end{document}
%%%%%%%%%%%%%%%%%%%%%%%%%%%%%%%%%%%%%%%%%%%%%%%%%%%%%%%

