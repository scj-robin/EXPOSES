\documentclass[dvips]{slides}
\textwidth 19cm
\textheight 23cm 
\topmargin 0 cm 
\oddsidemargin  -2cm 
\evensidemargin  -2cm

% Maths
\usepackage{amsfonts, amsmath, amssymb}
\newcommand{\Acal}{\mathcal{A}}
\newcommand{\Ccal}{\mathcal{C}}
\newcommand{\Ecal}{\mathcal{E}}
\newcommand{\Ncal}{\mathcal{N}}
\newcommand{\Ocal}{\mathcal{O}}
\newcommand{\Pcal}{\mathcal{P}}
\newcommand{\Wcal}{\mathcal{W}}
\newcommand{\att}{{\tt a}}
\newcommand{\ctt}{{\tt c}}
\newcommand{\gtt}{{\tt g}}
\newcommand{\ttt}{{\tt t}}
\newcommand{\mbf}{{\bf m}}
\newcommand{\Phibf}{\mbox{\mathversion{bold}{$\Phi$}}}
\newcommand{\Sbf}{{\bf S}}
\newcommand{\bps}{\mbox{bps}}
\newcommand{\vbf}{{\bf v}}
\newcommand{\wbf}{{\bf w}}
\newcommand{\Esp}{{\mathbb E}}
\newcommand{\Var}{{\mathbb V}}
\newcommand{\Indic}{{\mathbb I}}

% Couleur et graphiques
\usepackage{color}
\usepackage{graphics}
\usepackage{epsfig} 
\usepackage{pstcol}

% Texte
\usepackage{lscape, ../../../fancyheadings, rotating, enumerate}
\usepackage[french]{babel}
\usepackage[latin1]{inputenc}
\definecolor{darkgreen}{cmyk}{0.5, 0, 0.5, 0.5}
\definecolor{orange}{cmyk}{0, 0.6, 0.8, 0}
\definecolor{jaune}{cmyk}{0, 0.5, 0.5, 0}
\newcommand{\textblue}[1]{\textcolor{blue}{#1}}
\newcommand{\textred}[1]{\textcolor{red}{#1}}
%\newcommand{\textgreen}[1]{\textcolor{green}{\bf #1}}
\newcommand{\textgreen}[1]{\textcolor{darkgreen}{\bf #1}}
\newcommand{\textorange}[1]{\textcolor{orange}{\bf #1}}
\newcommand{\textyellow}[1]{\textcolor{yellow}{\bf #1}}

% Sections
\newcommand{\chapter}[1]{\centerline{\LARGE \textred{#1}}}
\newcommand{\section}[1]{\centerline{\Large \textblue{#1}}}
\newcommand{\subsection}[1]{\noindent{\large \textred{#1}}}
\newcommand{\paragraph}[1]{{\textblue{#1}}}

%%%%%%%%%%%%%%%%%%%%%%%%%%%%%%%%%%%%%%%%%%%%%%%%%%%%%%%%%%%%%%%%%%%%%%
%%%%%%%%%%%%%%%%%%%%%%%%%%%%%%%%%%%%%%%%%%%%%%%%%%%%%%%%%%%%%%%%%%%%%%
\begin{document}
\landscape
\headrulewidth 0pt 
\pagestyle{fancy} 
\cfoot{}
\rfoot{\begin{rotate}{90}{
      \vspace{0.5cm}
      \hspace{-1.5cm} \tiny S. Robin (Motif statistics in DNA)
      }\end{rotate}}
\rhead{\begin{rotate}{90}{
      \hspace{.5cm} \tiny \thepage
      }\end{rotate}}
%%%%%%%%%%%%%%%%%%%%%%%%%%%%%%%%%%%%%%%%%%%%%%%%%%%%%%%%%%%%%%%%%%%%%%
%%%%%%%%%%%%%%%%%%%%%%%%%%%%%%%%%%%%%%%%%%%%%%%%%%%%%%%%%%%%%%%%%%%%%%

\begin{center}
   \textblue{\LARGE MOTIFS DISTRIBUTION}

   \textblue{\LARGE IN DNA SEQUENCES}

   {\large St�phane ROBIN} \\
   robin@inapg.inra.fr

   {UMR INA-PG / INRA, Paris} \\
   {Math�matique et Informatique Appliqu�es}
   
   \vspace{2cm}
   {Bio-Info-Math Workshop, Tehran, April 2005}
\end{center}

%%%%%%%%%%%%%%%%%%%%%%%%%%%%%%%%%%%%%%%%%%%%%%%%%%%%%%%%%%%%%%%%%%%%%%
%%%%%%%%%%%%%%%%%%%%%%%%%%%%%%%%%%%%%%%%%%%%%%%%%%%%%%%%%%%%%%%%%%%%%%
%%%%%%%%%%%%%%%%%%%%%%%%%%%%%%%%%%%%%%%%%%%%%%%%%%%%%%%%%%%%%%%%%%%%%%
%%%%%%%%%%%%%%%%%%%%%%%%%%%%%%%%%%%%%%%%%%%%%%%%%%%%%%%%%%%%%%%%%%%%%%
\newpage
\chapter{Biological interest of motif statistics}
\chapter{~}
\section{Four examples}
%%%%%%%%%%%%%%%%%%%%%%%%%%%%%%%%%%%%%%%%%%%%%%%%%%%%%%%%%%%%%%%%%%%%%%
\subsection{Ex 1: Promoter motifs} = structured motifs where
polymerase binds to DNA
$$
\begin{pspicture}(20, 2.5)(0, -2.5)
  \psline[linewidth=0.05, linestyle=dashed, linecolor=black]{<-|}(0,
  1.2)(14.2, 1.2) 
  \rput[B]{0}(6.85, 1.4){$\simeq$ 100 bps}

  \psline[linewidth=0.1, linecolor=black](0, 0)(3.3, 0)
  \rput[B]{0}(4, -0.1){\fbox{\rule[-0.2cm]{0cm}{0.8cm}$\;\vbf\;$}}
  \psline[linewidth=0.1, linecolor=black](4.7, 0)(7.3, 0)
  \rput[B]{0}(8, -0.1){\fbox{\rule[-0.2cm]{0cm}{0.8cm}$\;\wbf\;$}}
  \psline[linewidth=0.1, linecolor=black](8.7, 0)(14.2, 0)
  \rput[B]{0}(17, -0.1){\fbox{\rule[-0.2cm]{0cm}{0.8cm}\qquad gene\qquad}}
  
  \psline[linewidth=0.05, linestyle=dashed, linecolor=black]{<->}(4.7,
  -1)(7.2, -1) 
  \rput[B]{0}(5.95, -2){16 bps $\leq d \leq$ 18 bps}
\end{pspicture}
$$
Which structured motifs occur almost ({\sl too} ?) systematically
in upstream regions of the genes of a given species?

%%%%%%%%%%%%%%%%%%%%%%%%%%%%%%%%%%%%%%%%%%%%%%%%%%%%%%%%%%%%%%%%%%%%%%
\newpage
\subsection{Ex 2: CHI motifs in bacterial genomes} 

{\em Crossover Hot-spot Initiator}: defense function of the genome
against the degradation activity of an enzyme

\begin{tabular}{ll}
  \begin{tabular}{p{10cm}}
    Known in several bacterial genomes:
    \\
    \\
    {\em E. coli}: {\gtt\ctt\ttt\gtt\gtt\ttt\gtt\gtt} \\
    \\
    {\em H.   influenza}: \gtt{\tt N}\ttt\gtt\gtt\ttt\gtt\gtt \\
    \\
    ({\sl Figure: Schbath, 95})
    \end{tabular}
    &
    \begin{tabular}{c}
      \epsfig{file=../Figures/FigSCSChi.ps, width=10cm, height=8cm,
      clip=, bbllx=178, bblly=479, bburx=482, bbury=730} 
    \end{tabular}
\end{tabular}

Is this motif {\em unexpectedly frequent} in some regions of the
genome? \\
If so, these regions may contain crucial functions.

%%%%%%%%%%%%%%%%%%%%%%%%%%%%%%%%%%%%%%%%%%%%%%%%%%%%%%%%%%%%%%%%%%%%%%
\newpage
\subsection{Ex 3: Palindromes} = self-complementary words 
$$
  \begin{pspicture}(10, 4.5)
    \psline[linewidth=0.1, linecolor=black]{->}(2, 3.5)(8, 3.5)
    \rput[B]{0}(5, 1.5){$
      \begin{array}{cccccc}
        \gtt  & \ttt & \ttt & \att & \att & \ctt \\
        |  & |  & |  & |  & |  & | \\
        \ctt & \att & \att & \ttt & \ttt & \gtt 
      \end{array}
      $}
    \psline[linewidth=0.1, linecolor=black]{<-}(2, 0)(8, 0)
  \end{pspicture}
$$

Palindromes of length 6 are restriction sites (i.e. frailty sites) of
the genome of {\em E. coli}.

If they are {\em especially avoided} in some regions, these
regions may be of major importance for the organism.

%%%%%%%%%%%%%%%%%%%%%%%%%%%%%%%%%%%%%%%%%%%%%%%%%%%%%%%%%%%%%%%%%%%%%%
\newpage
\subsection{Ex 4: Detection of unknown motifs}

\begin{itemize}
\item Motifs with favorable functions should be {\sl unexpectedly
    frequent},
\item Motifs with damaging functions should be {\sl unexpectedly
    rare}
\end{itemize}

Even when we know nothing about them (except their length) , such
motifs may be detected only because they have {\sl unexpected
  frequencies}

%%%%%%%%%%%%%%%%%%%%%%%%%%%%%%%%%%%%%%%%%%%%%%%%%%%%%%%%%%%%%%%%%%%%%%
%%%%%%%%%%%%%%%%%%%%%%%%%%%%%%%%%%%%%%%%%%%%%%%%%%%%%%%%%%%%%%%%%%%%%%
%%%%%%%%%%%%%%%%%%%%%%%%%%%%%%%%%%%%%%%%%%%%%%%%%%%%%%%%%%%%%%%%%%%%%%
%%%%%%%%%%%%%%%%%%%%%%%%%%%%%%%%%%%%%%%%%%%%%%%%%%%%%%%%%%%%%%%%%%%%%%
\newpage
\chapter{A model: what for ?}
\chapter{~}
\subsection{Model = Reference} \\
{\em To be able to decide if something is unexpected, we first
  need to know what to expect.}

To avoid artifacts, the model should typically account for
\begin{enumerate}[$\bullet$]
\item the frequencies of nucleotides, or di-, or tri-nucleotides in
  the sequence,
\item the overlapping structure of the word,
\item eventually, the overall frequency of the word in the sequence
\end{enumerate}

The choice of the model (Markov chain / compound Poisson process)
depends on the question.

({\sl R., Rodolphe \& Schbath; 05})
 
%%%%%%%%%%%%%%%%%%%%%%%%%%%%%%%%%%%%%%%%%%%%%%%%%%%%%%%%%%%%%%%%%%%%%%
\newpage
\paragraph{Overlapping structure of the word} \\
Some words can overlap themselves (see {\em Conway (Gardner, 74);
  Guibas \& Odlyzko, 81}).

Such words tend to occur in {\em clumps} and have a less regular
distribution along the sequence.

Cdf of the distance between two occurrences under model M00:
$$
\begin{tabular}{cc}
  $\wbf=({\tt gatc})$ & $\wbf=({\tt aaaa})$ \\ 
  \epsfig{file = ../FIGURES/RoD99-JAP-Fig1-1.eps, height = 6cm, width
    = 10cm} & 
  \epsfig{file = ../FIGURES/RoD99-JAP-Fig1-2.eps, height = 6cm, width
    = 10cm} \\
  $\Esp(Y) = 256\;\bps$ & $\Esp(Y) = 256\;\bps$ \\
  $\Var(Y) = (256.2\;\bps)^2$ & $\Var(Y) = (326.7\;\bps)^2$
\end{tabular}
$$

%%%%%%%%%%%%%%%%%%%%%%%%%%%%%%%%%%%%%%%%%%%%%%%%%%%%%%%%%%%%%%%%%%%%%%
\newpage
\subsection{Probabilities and distributions of interest} 

%\vspace{-0.5cm}
\paragraph{Positions, distances, counts} 
$$
                                %FigUnMot.ps
\begin{pspicture}(22, 7)
  \rput[bl](-1.5, 0){  \colorbox{white}{
      \psfig{file = ../FIGURES/FigUnMot.ps,
        height=7cm, width=22cm, bbllx=75 , bblly=525 , bburx=520 ,
        bbury=670 , clip=} } }
  \rput[Br](21.5, 6.5){\colorbox{white}{\textcolor{black}{$N(\wbf) = 6$}}}
\end{pspicture}
$$

\vspace{-1.5cm}
\begin{enumerate}[$\bullet$]
\item Probability for a motif to occur in a sequence: $X_1$ \\
  $\longrightarrow$ promoter motifs 
\item Distribution of the number of occurrences: $N$ 
\item Distribution of the occurrences along the sequence: $Y^r, N(x) -
  N(x-y)$ %\
  $\longrightarrow$ CHI motifs, palindromes
\end{enumerate}


%%%%%%%%%%%%%%%%%%%%%%%%%%%%%%%%%%%%%%%%%%%%%%%%%%%%%%%%%%%%%%%%%%%%%%
%%%%%%%%%%%%%%%%%%%%%%%%%%%%%%%%%%%%%%%%%%%%%%%%%%%%%%%%%%%%%%%%%%%%%%
%%%%%%%%%%%%%%%%%%%%%%%%%%%%%%%%%%%%%%%%%%%%%%%%%%%%%%%%%%%%%%%%%%%%%%
%%%%%%%%%%%%%%%%%%%%%%%%%%%%%%%%%%%%%%%%%%%%%%%%%%%%%%%%%%%%%%%%%%%%%%
\newpage
\chapter{Motifs occurrences in Markov chains}
\chapter{~}
\section{Markov chains = Discrete modeling}
%%%%%%%%%%%%%%%%%%%%%%%%%%%%%%%%%%%%%%%%%%%%%%%%%%%%%%%%%%%%%%%%%%%%%
$\Sbf = (S_1, \dots, S_{\ell})$ is an homogeneous stationary Markov
chain 
%\vspace{-1cm}
\begin{enumerate}[$\bullet$]
\item of order $m$ (M$m$ model) over the alphabet $\Acal = \{\att,
  \ctt, \gtt, \ttt\}$
\item with transition probabilities $\pi(s_1, \dots, s_m; s_{m+1})$.
  $$
  \begin{tabular}{c}
    \psfig{file = ../Figures/FigChaineMarkov.ps, bbllx=120, bblly=283, bburx=430,
      bbury=295, clip=, width=15cm}
    \\
    \psfig{file = ../Figures/FigChaineMarkov.ps, bbllx=120, bblly=235, bburx=430,
      bbury=263, clip=, width=15cm}
  \end{tabular}
  $$
\end{enumerate}

The M$m$ model is fitted to the frequencies of all the words of length
$(m+1)$
$$
\widehat{\pi}(s_1, \dots, s_m; s_{m+1}) = \frac{N(s_1 \dots s_m
  s_{m+1})}{N(s_1 \dots s_m)}
$$
Theoretically, properties derived under M1 can be generalized to M$m$:
M2 is equivalent to M1 on the alphabet $\Acal^2 = \{\att\att,
\att\ctt, \dots, \ttt\ttt\}$

%%%%%%%%%%%%%%%%%%%%%%%%%%%%%%%%%%%%%%%%%%%%%%%%%%%%%%%%%%%%%%%%%%%%%%
\newpage
\subsection{Distribution of the count}

The (ficticious) word $\wbf = \gtt\ctt\ttt\ttt$ occurs 56 times in a
given genome, is it significantly high ?

\paragraph{M1 model.} Occurrence probability (at any position):
$$
\mu(\wbf) = \mu(w1) \times \pi(w_1, w_2) \times \dots \times
\mu(w_{|\wbf|-1}, w_{|\wbf|})
$$

\paragraph{Expected count} (sequence of length $\ell$):
$\Esp N(\wbf) = (\ell - k +1) \mu(\wbf)$
{\em Kleffe \& Borodowsky, 92}: $\Esp N(\wbf), \Var N(\wbf)$

%{\sl; Prum \& al, Schbath, 95; Nicod�me \& al, 99; Regnier, 00; Nuel, 01})



\paragraph{Distribution of the count.} The exceptionality of the
observed frequency is measured by the $p$-value
$$
\Pr_{M1}\{N(\wbf) \geq n_{obs}(\wbf)\}
=
\Pr_{M1}\{N(\gtt\ctt\ttt\ttt) \geq 56\}
$$

%%%%%%%%%%%%%%%%%%%%%%%%%%%%%%%%%%%%%%%%%%%%%%%%%%%%%%%%%%%%%%%%%%%%%%
\newpage
\paragraph{Gaussian approximation.} If $\wbf$ is ``frequent'', $\Esp
N(\wbf) = \Ocal(\ell)$ ({\sl Prum \& al, 95}),
$$
U(\wbf) = \frac{N(\wbf) - \widehat{\Esp} N(\wbf)}{\sqrt{\widehat{\Var}
    N(\wbf)}} \approx \Ncal(0, 1)
$$ 

\paragraph{Poisson approximation.} If $\wbf$ is ``rare'': $\Esp N(\wbf) = \Ocal(\log
\ell)$ ({\sl Schbath, 95}),
$$
N(\wbf) \approx \Pcal[\Esp N(\wbf)]
$$
For overlapping words: compound Poisson approximation.

\paragraph{Binomial approximation:} {\sl Van Helden, 99}

\paragraph{Exact distribution of $N(\wbf)$:} {\sl R. \& Daudin, 99; 
Nicod�me \& al, 99; Regnier, 00}

\paragraph{Large deviation:} {\sl Nuel, 01}

%%%%%%%%%%%%%%%%%%%%%%%%%%%%%%%%%%%%%%%%%%%%%%%%%%%%%%%%%%%%%%%%%%%%%%
\newpage
\paragraph{Quality of the approximations.} The (compound) Poisson
approximation turns out to perform very well, in many situations ({\sl
  R. \& Schbath, 01}):
$$
\epsfig{file=../FIGURES/RoS01-Fig1.ps, width=24cm, height=8cm, clip=,
  bbllx=85, bblly=523, bburx=520, bbury=642}
$$
Even for rather frequent words.

CP approximation fails for frequent and short words.

%%%%%%%%%%%%%%%%%%%%%%%%%%%%%%%%%%%%%%%%%%%%%%%%%%%%%%%%%%%%%%%%%%%%%%
\newpage

\paragraph{Influence of the order of the Markov chain.}
The exceptionality of a word's frequency strongly depends on the
chosen model:
$$
\epsfig{file = ../FIGURES/FigRmesStat.ps, width=22cm, height=9cm, clip=,
  bbllx=123, bblly=427, bburx=475, bbury=589}
$$
(R'mes software: {\sl Bouvier \& al., 99})

The CHI motif $\wbf =\gtt\ctt\ttt\gtt\gtt\ttt\gtt\gtt$ appears at the
top of the list for almost all orders.

%%%%%%%%%%%%%%%%%%%%%%%%%%%%%%%%%%%%%%%%%%%%%%%%%%%%%%%%%%%%%%%%%%%%%%
\newpage

\hspace{-1cm}
\begin{tabular}{ll}
  \begin{tabular}{p{9cm}}
    \paragraph{Palindromes of length 6:} \\
    \\
    In both model M3 and M4, most of them seem to be avoided in the genome
    of {\sl E. coli} \\
    \\
    Most of them are restriction sites: possible defense system of
    {\sl E. coli}'s genome. \\
    \\
    \\
  \end{tabular}
  &
  \begin{tabular}{cc}
    \begin{rotate}{90}{\hspace{5cm}$U_{M4}(\wbf)$} \end{rotate} & \epsfig{file =
        ../FIGURES/FigRmesPalindromes.ps, width=12cm, height=12cm, clip=,
        bbllx=125, bblly=265, bburx=475, bbury=615} \\
      & 
      $U_{M3}(\wbf)$
  \end{tabular}
\end{tabular}

%%%%%%%%%%%%%%%%%%%%%%%%%%%%%%%%%%%%%%%%%%%%%%%%%%%%%%%%%%%%%%%%%%%%%%
\newpage
\subsection{Distribution of the distance: one word}

\vspace{-1cm} \begin{flushright} 
  {\em Blom \& Thorburn, 82} (M0); {\em R. \& Daudin, 99} (M1) 
\end{flushright} \vspace{-0.5cm}

Distribution of the distance $Y$
$$
p(y) =\Pr \{ Y=y \}
$$
\begin{enumerate}
\item Linear recursive formula of order $y-1$ (complexity = $O(y^2)$)
  $$
  p(y) = \sum_{z=1}^{y-1} c_{z} p(y-z)
  $$
\item Derive the probability generating function
  $$
  \phi_Y (t) = \sum_{y\geq 1}p(y) t^{y} = U_Y(t) / V_Y(t)
  $$
\item Taylor expansion of $\phi_Y$ with a {\em new} linear
  recursive formula of order $|\wbf|$ (complexity =
  $O(y)$)
  $$
  p(y) = \sum_{z=1}^{|\wbf|} c'_{k} p(y-z)
  $$
\end{enumerate}

%%%%%%%%%%%%%%%%%%%%%%%%%%%%%%%%%%%%%%%%%%%%%%%%%%%%%%%%%%%%%%%%%%%%%%
\newpage
\subsection{Principle for a set of words} 
\hspace{3cm} 
{\em R. \& Daudin, 01} (M1) 

Consider the distribution of the occurrences of the motif 
$$
\mbf = \{\wbf_1, \dots, \wbf_I\}
$$
The distribution of the distances depends on the words themselves
(semi-Markov process)
$$
                                %FigMultiMot.ps  
  \hspace{-1.5cm}
  \colorbox{white}{
    \psfig{file = ../FIGURES/FigMultiMot.ps, height=4.8cm,
      width=25cm, bbllx=75, bblly=255 , bburx=520 , bbury=350 , clip=}
    }
$$

Steps 1, 2, 3 follow the same principle as for one word but involve
{\em generating matrices}

%%%%%%%%%%%%%%%%%%%%%%%%%%%%%%%%%%%%%%%%%%%%%%%%%%%%%%%%%%%%%%%%%%%%%%
\newpage
Denoting $\phi_{ij}(t) = \phi_{Y_{ij}}(t)$, ($i, j = 1..I$)
$$
\underset{I \times I}{\Phibf(t)} = \left[ 
  \begin{array}{ccc}
    \phi_{11}(t) & \dots & \phi_{1I}(t) \\
    \vdots       &       & \vdots       \\
    \phi_{I1}(t) & \dots & \phi_{II}(t) \\
  \end{array}
\right], 
\qquad \phi_{ij}(t) = \frac{U_{ij}(t)}{V_{ij}(t)}
$$

Step 2 requires the inversion of a generating matrix:
$$
\Phibf(t) = {\bf F}(t)[{\bf I} - {\bf F}(t)]^{-1}
$$
\paragraph{Limitations:}
%\vspace{-1cm}
\begin{enumerate}[$\bullet$] 
\item Complexity of this last step: $O(I^3 |\mbf|)$
\item Numerical instability except if $[{\bf I} -
  {\bf F}(t)]$ is inverted formally \\
  $\Longrightarrow$ small set of short words (small $I$ and
  $|\mbf|$)
\end{enumerate}
\paragraph{Other approaches:} algorithmic ({\em Nicod�me, 00}),
embedded Markov chain ({\em Fu \& Koutras, 94}, {\em Koutras, 97}),
properties of the exponential family ({\em Stefanov \& Pakes, 99}), etc.

%%%%%%%%%%%%%%%%%%%%%%%%%%%%%%%%%%%%%%%%%%%%%%%%%%%%%%%%%%%%%%%%%%%%%%
\newpage
\section{Application to structured motifs} 
%%%%%%%%%%%%%%%%%%%%%%%%%%%%%%%%%%%%%%%%%%%%%%%%%%%%%%%%%%%%%%%%%%%%%%
\paragraph{Difficulty:} Complexity of the overlapping structure of structured
motif 
$$
\begin{pspicture}(11, 1.75)(0, 0.75)
  \rput[B]{0}(1, 1){$\mbf = \framebox{\quad$\vbf$\quad}$}
  \psline[linewidth=0.1, linecolor=black]{<->}(3.2, 1.1)(8.7, 1.1)
  \rput[B]{0}(10, 1){\framebox{\quad$\wbf$\quad}}
  \rput[B](6, 1.3){$d$}
\end{pspicture}
$$
\begin{tabular}{ll}
  $\Longrightarrow$ & impossible to calculate the exact distribution of
  $X_1(\mbf)$ with \\
  & the method presented above
\end{tabular}

\paragraph{Approximation} ({\em R. \& al, 02})
%\vspace{-1cm}
\begin{enumerate}
\item Probability for $\mbf$ to occur at a given position (using the
  distribution of the distances): $ \mu(\mbf)$ 
\item Approximation of order 0 (geometric) does not work (simulations) :
  $$
  \Pr \left\{ N(\wbf) \geq 1 \right\}
  \approx
  1 - [1 - \mu(\mbf)]^{\ell - |\mbf| + 1}.
  $$
\item Approximation of order 1 $\left(\mu_1(\mbf) = \Pr\{\mbf \mbox{
      at }x | \mbf \mbox{ not at } x-1\}\right)$:
  $$
  \Pr \left\{ N(\wbf) \geq 1 \right\}
  \approx
  1 - [1 - \mu(\mbf)] [1 - \mu_1(\mbf)]^{\ell-|\mbf|}
  $$
\end{enumerate}

%%%%%%%%%%%%%%%%%%%%%%%%%%%%%%%%%%%%%%%%%%%%%%%%%%%%%%%%%%%%%%%%%%%%%%
\newpage
\hspace{-3.5cm}
\begin{tabular}{lc}
  \begin{tabular}{l}
    \textred{\large Promoters} \\
    \textred{\large in} \\
    \textred{\large {\em B. subtilis}:} \\
    \\
    131 upstream \\
    regions \\
    of 100 bps \\
    \\
    $p$-value \\
    $< 10^{-16}$ \\
    \\
    (putative \\
    alignment) \\
  \end{tabular}
  &
  {\small
    \begin{tabular}{lclccc}
%      $\qquad\vbf$ & $(d_1:d_2)$ & $\qquad\wbf$ & $N_{\mbox{obs}}$& $\Esp_1(N)$ \\
%      $\qquad\vbf$ & $(d_1:d_2)$ & $\qquad\wbf$ 
      \multicolumn{3}{c}{$\begin{array}{c}
          \mbf \\
          \overbrace{\vbf \hspace{2cm} (d_1:d_2) \hspace{2cm} \wbf} \\
        \end{array}$
        }
      & \begin{tabular}{c} number of\\ regions\\ containing
      $\mbf$\end{tabular} 
      & \begin{tabular}{c} expected\\ number\end{tabular}\\
      \hline 
      {\tt\ \ \ gttgaca\ } & $(16:18)$ & {\tt atataat}     &  7 &   2.43 $10^{-2}$ \\  
      {\tt\ \ \ gttgaca\ } & $(16:18)$ & {\tt \ tataata}   &  8 &   2.23 $10^{-2}$ \\  
      {\tt\ \ tgttgac\ \ } & $(16:18)$ & {\tt \ tataata}   & 10 &   2.12 $10^{-2}$ \\
      {\tt\ \ \ \ ttgacaa} & $(16:18)$ & {\tt \ tacaat}    &  9 &   9.82 $10^{-2}$ \\
      {\tt\ \ \ \ ttgacaa} & $(16:18)$ & {\tt \ tataata}   & 10 &   5.07 $10^{-2}$ \\
      {\tt\ \ \ \ ttgacag} & $(16:18)$ & {\tt \ tataat}    &  9 &   7.12 $10^{-2}$ \\
      {\tt\ \ \ \ ttgacaa} & $(17:19)$ & {\tt \ \ ataataa} &  9 &   6.97 $10^{-2}$ \\
      {\tt \ ttgttga\ \ } & $(17:19)$ & {\tt \ tataata}    &  8 &   5.17 $10^{-2}$ \\
      {\tt\ \ \ gttgaca\ } & $(17:19)$ & {\tt \ \ ataataa} &  8 &   3.09 $10^{-2}$ \\
      {\tt\ \ \ gttgaca\ } & $(17:19)$ & {\tt \ tataata}   &  8 &   2.19 $10^{-2}$ \\
      {\tt\ \ \ cttgaca\ } & $(17:19)$ & {\tt \ tataat}    &  8 &   6.04 $10^{-2}$ \\
      {\tt\ \ tgttgac\ \ } & $(17:19)$ & {\tt \ tataata}   & 12 &   2.09 $10^{-2}$ \\
      {\tt\ \ tgttgac\ \ } & $(17:19)$ & {\tt atataat}     &  7 &   2.29 $10^{-2}$ \\
      {\tt\ ttgttga\ \ \ } & $(18:20)$ & {\tt \ tataata}   &  8 &   5.09 $10^{-2}$ \\
      {\tt\ \ \ gttgaca\ } & $(18:20)$ & {\tt \ \ ataatga} &  7 &   1.79 $10^{-2}$ \\
      {\tt gttgttg\ \ \ \ } & $(18:20)$ & {\tt \ tataata}  &  7 &   2.53 $10^{-2}$ \\
      {\tt\ \ tgttgac\ \ } & $(18:20)$ & {\tt \ \ ataataa} & 10 &   2.90 $10^{-2}$ \\
      {\tt\ \ tgttgac\ \ } & $(18:20)$ & {\tt \ \ atacta}  &  7 &   2.77 $10^{-2}$ \\
      {\tt\ \ tgttgac\ \ } & $(19:21)$ & {\tt \ \ ataataa} & 10 &   2.86 $10^{-2}$ \\
      {\tt\ \ tgttgac\ \ } & $(19:21)$ & {\tt \ \ atacta}  &  7 &   2.73 $10^{-2}$ \\
      {\tt\ \ tgttgac\ \ } & $(19:21)$ & {\tt \ \ \ \ tataat}    & 10 &   6.53 $10^{-2}$ \\
      {\tt\ \ \ gttgact\ } & $(19:21)$ & {\tt \ \ \ \ \ ataata}  &  8 &   6.25 $10^{-2}$ \\
    \end{tabular}
    }
\end{tabular}

%%%%%%%%%%%%%%%%%%%%%%%%%%%%%%%%%%%%%%%%%%%%%%%%%%%%%%%%%%%%%%%%%%%%%%
%%%%%%%%%%%%%%%%%%%%%%%%%%%%%%%%%%%%%%%%%%%%%%%%%%%%%%%%%%%%%%%%%%%%%%
%%%%%%%%%%%%%%%%%%%%%%%%%%%%%%%%%%%%%%%%%%%%%%%%%%%%%%%%%%%%%%%%%%%%%%
%%%%%%%%%%%%%%%%%%%%%%%%%%%%%%%%%%%%%%%%%%%%%%%%%%%%%%%%%%%%%%%%%%%%%%
\newpage
\chapter{Compound Poisson model}
\chapter{~}
\section{Compound Poisson process}
\section{= Continuous modeling} 
%%%%%%%%%%%%%%%%%%%%%%%%%%%%%%%%%%%%%%%%%%%%%%%%%%%%%%%%%%%%%%%%%%%%%%
For rare words, the sequence $\Sbf$ can be viewed as a continuous line
$[0; \ell]$
$$
\hspace{-1.5cm}
\begin{tabular}{c}
  Real occurrences  \\
                                %FigPoissonComp.ps
  \colorbox{white}{      
    \psfig{file = ../FIGURES/FigPoissonComp.ps,
      height=2cm, width=25cm, bbllx=80 , bblly=370 , bburx=525 ,
      bbury=405, clip=}
    }  \\
  \\
  Compound Poisson modeling \\
                                %FigPoissonComp.ps
  \colorbox{white}{
    \psfig{file = ../FIGURES/FigPoissonComp.ps,
      height=3.9cm, width=25cm, bbllx=80 , bblly=260 , bburx=525 ,
      bbury=330, clip=}
    }  \\
\end{tabular}
$$
%%%%%%%%%%%%%%%%%%%%%%%%%%%%%%%%%%%%%%%%%%%%%%%%%%%%%%%%%%%%%%%%%%%%%%
\newpage
\paragraph{Clump process} $\{C(x)\} =$ Poisson process
with intensity $\equiv \lambda$

\paragraph{Clump sizes} $\{K_1, K_2, \dots\}$ are iid 
$$
Pr\{K = k\} = g(k)
$$

\paragraph{Counting process} of the occurrences $\{N(x)\} = $
compound Poisson process:
$$
N(x) = \sum_{c = 1}^{C(x)} K_c
$$

Non overlapping word $\Longrightarrow$ simple Poisson process

\paragraph{Interpretation:} Poisson modeling implies that the clumps are
{\em uniformly distributed} along the genome \\
\centerline{$\longrightarrow$ {\em Null hypothesis} of the next part}

%%%%%%%%%%%%%%%%%%%%%%%%%%%%%%%%%%%%%%%%%%%%%%%%%%%%%%%%%%%%%%%%%%%%%%
\newpage
\subsection{P\'olya-Aeppli model} \\
When considering one single word $\wbf$, the clump size has a
geometric distribution
$$
g(k) = a^{k-1} (1-a) \quad \Longrightarrow
\quad \Esp(K) = 1 / (1-a)
$$
where $a$ is the overlapping probability of $\wbf$

\paragraph{Parameter estimates:} In a sequence of length $\ell$ 
%\vspace{-1cm}
\begin{enumerate}[$\bullet$]
\item $\widehat{\lambda}$ is the empirical frequency of the
  clumps: $ \widehat{\lambda} = C(\ell) / \ell $
\item $\widehat{a}$ is the proportion of overlapped occurrences:
  $\widehat{a} = \frac{N(\ell) - C(\ell)}{N(\ell)}$
\end{enumerate}

\paragraph{Properties}
%\vspace{-1cm}
\begin{enumerate}[$\bullet$]
\item P\'olya-Aeppli is the best approximation of the distribution of
  the word count in the Markov model ({\em R. \& Schbath, 01})
\item $\Esp[N(\ell)] = \ell \times \lambda \times \Esp(K) \quad
  \Longrightarrow \quad \widehat{\Esp}N(\ell) = \ell
  \widehat{\lambda} / (1-\widehat{a}) = N(\ell)$ \\
  \centerline{$\Longrightarrow \quad$ {\em no word has an
      ``unexpected'' count}}
\end{enumerate}

%%%%%%%%%%%%%%%%%%%%%%%%%%%%%%%%%%%%%%%%%%%%%%%%%%%%%%%%%%%%%%%%%%%%%%
\newpage
\section{Clump size modeling}
%%%%%%%%%%%%%%%%%%%%%%%%%%%%%%%%%%%%%%%%%%%%%%%%%%%%%%%%%%%%%%%%%%%%%%
\vspace{-2cm} \begin{flushright} {\em R., 02} \end{flushright} \vspace{-1cm}

In the general case (e.g. motif $\mbf =\{\wbf_1$, $\wbf_2$,
$\dots$\}), the clump size does not have a geometric distribution

We may use
%\vspace{-1cm}
\begin{enumerate}[$\bullet$]
\item empirical estimates of an arbitrary distribution $g(k)$ \\
\item empirical estimates of the overlapping probabilities between
  words $\wbf_1$, $\wbf_2$, $\dots \Longrightarrow I^2$ parameters to
  be estimated \\
\item Markov estimates of the overlapping probabilities
  $\longrightarrow$ even M00 may provide a good fit
\end{enumerate}

However, distances $Y$ between words are {\em not iid}
%%%%%%%%%%%%%%%%%%%%%%%%%%%%%%%%%%%%%%%%%%%%%%%%%%%%%%%%%%%%%%%%%%%%%%
%%%%%%%%%%%%%%%%%%%%%%%%%%%%%%%%%%%%%%%%%%%%%%%%%%%%%%%%%%%%%%%%%%%%%%
%%%%%%%%%%%%%%%%%%%%%%%%%%%%%%%%%%%%%%%%%%%%%%%%%%%%%%%%%%%%%%%%%%%%%%
%%%%%%%%%%%%%%%%%%%%%%%%%%%%%%%%%%%%%%%%%%%%%%%%%%%%%%%%%%%%%%%%%%%%%%
\newpage
\chapter{Motifs distribution along a sequence}
\chapter{~}
\section{Two statistics}
%%%%%%%%%%%%%%%%%%%%%%%%%%%%%%%%%%%%%%%%%%%%%%%%%%%%%%%%%%%%%%%%%%%%%%
We aim to detect poor or rich regions in terms of occurrences of a
given motif

A natural criterion for a given region is the ratio
$$
\frac{\mbox{number of occurrences in the
    region}}{\mbox{size of the region}}
$$

\paragraph{Cumulated distances} of order $r$: 
\begin{tabular}{c}
  fixed numerator $r$ \\
  \hline
  random denominator $Y^r$
\end{tabular} 

\paragraph{Local counts} in a window of width $y$:
\begin{tabular}{c}
  random numerator $\Delta N$ \\
  \hline
  fixed denominator $y$
\end{tabular} 

%%%%%%%%%%%%%%%%%%%%%%%%%%%%%%%%%%%%%%%%%%%%%%%%%%%%%%%%%%%%%%%%%%%%%%
\newpage
\subsection{Distribution of the statistics}
\hspace{7cm} {\em R., 02}

\paragraph{Cumulated distance:} the distribution of 
$$
Y_i^r = \sum_{j=i}^{i+r-1} Y_j  = X_{i+r} - X_i
$$
is known {\em when the distances $Y_i$ are iid} (e.g. in the
one word case) for Markov and compound Poisson models

\paragraph{Local count:} the distribution of the count
$$
\Delta N(x) =  N(x) - N(x-y)
$$
is known for Markov and compound Poisson models ({\em Barbour \&
  al, 92})

%%%%%%%%%%%%%%%%%%%%%%%%%%%%%%%%%%%%%%%%%%%%%%%%%%%%%%%%%%%%%%%%%%%%%%
\newpage
\section{Extremal statistics}
We are interested in the richest region, i.e.
$$
Y^r_{\min} = \min_i \{Y_i^r\}\qquad \mbox{or} \qquad \Delta N_{\sup} = \sup_x \{\Delta N(x)\}
$$

\subsection{Chen-Stein approximation}  
\hspace{4cm} ({\em Arratia \& al, 89})
%\vspace{-1cm}

\paragraph{Cumulated distances:} 
an explicit bound distance can be calculated ({\em Dembo \&
  Karlin, 92}) for the distribution of $Y^r_{\min}$:
$$
\max_y \left| \Pr\{Y^r_{\min} \leq y\} - e^{-(n-r)\Pr\{Y^r \leq
    y\}}\right| \leq \mbox{bound}
$$

\paragraph{Local counts:} no explicit bound can be
derived, but this approximation
$$
\Pr\{\Delta N_{\sup} > n\} \simeq \exp[-(\ell - y)\Pr\{\Delta N >
n\}]
$$
is optimal ({\em Barbour \& Brown, 92})

%%%%%%%%%%%%%%%%%%%%%%%%%%%%%%%%%%%%%%%%%%%%%%%%%%%%%%%%%%%%%%%%%%%%%%
\newpage 
\section{Applications}
%%%%%%%%%%%%%%%%%%%%%%%%%%%%%%%%%%%%%%%%%%%%%%%%%%%%%%%%%%%%%%%%%%%%%%
\subsection{CHI motif in {\em H. influenza}} \\
In terms of overlap, $\mbf = (\gtt{\tt N}\ttt\gtt\gtt\ttt\gtt\gtt)$
behaves as one single word 

$\Longrightarrow$ cumulated distances can be used 

\paragraph{Number of occurrences:} $\ell = 1\;903\;356$ bps
$$
\begin{tabular}{lcl}
  observed number of occurrences & = & 223 \\
  expected under Markov (M1) & = & {\em 58.5} \\
  expected under compound Poisson & = & 223 \\
\end{tabular}
$$

\paragraph{Significancy thresholds:} for $\alpha = 5\%$ 
$$
\begin{tabular}{lr}
  for $Y^r$: & 6 312 bps \\
  for $\displaystyle{\min_{i=1...222}Y_i^r}$: & 238 bps 
\end{tabular}
$$
%%%%%%%%%%%%%%%%%%%%%%%%%%%%%%%%%%%%%%%%%%%%%%%%%%%%%%%%%%%%%%%%%%%%%%
\newpage
\paragraph{Distribution:} cumulated distances of order $r=3$ 

$$
\hspace{-1.5cm}
\begin{tabular}{cc}
  \multicolumn{2}{c}{plot of the ratio $3/Y^3$ ($\times
    10^{-3}$) versus the position $x$} \\
  \\
  \psfig{file = ../FIGURES/Rob02-JRSSC-Fig2-1.eps, height=8cm, width=12cm} &
  \psfig{file = ../FIGURES/Rob02-JRSSC-Fig2-2.eps, height=8cm, width=12cm} \\
  Markov (M1) & compound Poisson  \\
  overall bias &  no significant peak \\
\end{tabular}
$$

%%%%%%%%%%%%%%%%%%%%%%%%%%%%%%%%%%%%%%%%%%%%%%%%%%%%%%%%%%%%%%%%%%%%%%
\newpage 
\paragraph{Remarks:}
\begin{enumerate}
\item[$\bullet$] Markov model M7 would be unbiased (since $|\mbf| =
  8$) but involves more than $12\;000$ parameters
  \\
  \\
  The compound Poisson model has a better fit with much less
  parameters \\
\item[$\bullet$] In the compound Poisson model, the peak around 1.0 Mb
  (replication termination) is significant {\em on its own}:
  \begin{eqnarray*}
  \Pr\{Y^3 \leq 208\} = 1.6 10^{-4} \\
  \\
  \Pr\left\{\min_{i=1..220}\left(Y_i^3\right) \leq 208\right\} > 0.05
  \end{eqnarray*}
\end{enumerate}

%%%%%%%%%%%%%%%%%%%%%%%%%%%%%%%%%%%%%%%%%%%%%%%%%%%%%%%%%%%%%%%%%%%%%%
\newpage
\subsection{Palindromes in {\em E. coli}} ({$\ell = 4\;638\;868$)\\
There are 64 palindromes of length 6\\
They occur 54\;724 times in 50\;941 clumps

\paragraph{Clump size:} Because of their overlapping structure, clumps 
can not be considered as geometric \\
$\Longrightarrow$ Local counts should be used

We use a parsimonious modeling of $g(k)$ based the overlapping
probabilities given by the M0 model (4 parameters)

\paragraph{Results: } Windows of width $y = 10\;000$ bps
\begin{enumerate}
%\vspace{-1cm}
\item[$\bullet$] Poorest region: 73 occurrences ($p$-value $> 10\%$): \\
  {non significant}
\item[$\bullet$] Richest region: 185 occurrences ($p$-value
  $< 5\%$) \\
  \centerline{[2\;460\;567 bps; 2\;461\;566 bps]}
  ... interpretation: horizontal transfer?
\end{enumerate}

%%%%%%%%%%%%%%%%%%%%%%%%%%%%%%%%%%%%%%%%%%%%%%%%%%%%%%%%%%%%%%%%%%%%%%
\newpage 
%\section{Present ...}
%%%%%%%%%%%%%%%%%%%%%%%%%%%%%%%%%%%%%%%%%%%%%%%%%%%%%%%%%%%%%%%%%%%%%%
\subsection{Distribution in heterogeneous sequences} 
%\vspace{-1cm} 
\begin{flushright} 
  {\em Ledent \& R., 04}
\end{flushright} %\vspace{-1cm}

An exogenous information about the heterogeneity of the
sequence is sometimes available. 

It can be summarize in the quantity $\pi_s(x) =$
%\vspace{-1cm}
\begin{enumerate}
\item[$\bullet$] binary (0/1) variable indicating if position $x$
  belongs to state $s$, where states can be: coding / non coding, 
\item[$\bullet$] posterior probability of being in state $s$ at
  position $x$ provided by an HMM model
\end{enumerate}

The intensity $\lambda(x)$ can be modeled according to this information:
$$
\lambda(x) = \sum_s \lambda_s \pi_s(x),
$$
so does the distribution of the clump.

%%%%%%%%%%%%%%%%%%%%%%%%%%%%%%%%%%%%%%%%%%%%%%%%%%%%%%%%%%%%%%%%%%%%%%
\newpage
\subsection{Three steps estimation procedure.} 
Occurrences of $\att\att\ttt\ttt$ in the genome of phage {\em Lambda}
($\ell= 48\;500$ bps) \\
\begin{tabular}{rl}
  \hspace{-1cm}
  \epsfig{file = ../FIGURES/lambda-aatt-hetero.ps, height = 9cm, width =
    12cm} 
  & 
  \hspace{-1cm}
  \epsfig{file = ../FIGURES/lambda-aatt-homo.ps, height = 9cm, width =
    12cm} 
\end{tabular}
\paragraph{3 steps:} 
%\vspace{-1cm}
\begin{enumerate}
\item Estimate the intensity $\lambda(x)$ (left: green line)
\item ``Homogenize'' the clump process and calculate thresholds
  (right: red line + blue lines for the bounds)
\item come back to the original process (left)
\end{enumerate}

%%%%%%%%%%%%%%%%%%%%%%%%%%%%%%%%%%%%%%%%%%%%%%%%%%%%%%%%%%%%%%%%%%%%%%
%%%%%%%%%%%%%%%%%%%%%%%%%%%%%%%%%%%%%%%%%%%%%%%%%%%%%%%%%%%%%%%%%%%%%%
%%%%%%%%%%%%%%%%%%%%%%%%%%%%%%%%%%%%%%%%%%%%%%%%%%%%%%%%%%%%%%%%%%%%%%
%%%%%%%%%%%%%%%%%%%%%%%%%%%%%%%%%%%%%%%%%%%%%%%%%%%%%%%%%%%%%%%%%%%%%%
\end{document}
%%%%%%%%%%%%%%%%%%%%%%%%%%%%%%%%%%%%%%%%%%%%%%%%%%%%%%%%%%%%%%%%%%%%%%
%%%%%%%%%%%%%%%%%%%%%%%%%%%%%%%%%%%%%%%%%%%%%%%%%%%%%%%%%%%%%%%%%%%%%%
%%%%%%%%%%%%%%%%%%%%%%%%%%%%%%%%%%%%%%%%%%%%%%%%%%%%%%%%%%%%%%%%%%%%%%
%%%%%%%%%%%%%%%%%%%%%%%%%%%%%%%%%%%%%%%%%%%%%%%%%%%%%%%%%%%%%%%%%%%%%%

