\documentclass[dvips]{slides}
\usepackage{lscape}
\textwidth 19cm
\textheight 23cm 
\topmargin 0 cm 
\oddsidemargin  -1cm 
\evensidemargin  -1cm

% Maths
\usepackage{amsfonts}
\usepackage{amsmath}
\usepackage{amssymb}
\newcommand{\Acal}{\mathcal{A}}
\newcommand{\Ccal}{\mathcal{C}}
\newcommand{\Ecal}{\mathcal{E}}
\newcommand{\Pcal}{\mathcal{P}}
\newcommand{\Wcal}{\mathcal{W}}
\newcommand{\D}{\mbox{d}}
\newcommand{\Ctilde}{\widetilde{C}}
\newcommand{\elltilde}{\tilde{\ell}}
\newcommand{\Ytilde}{\widetilde{Y}}
\newcommand{\Ztilde}{\widetilde{Z}}
\newcommand{\att}{{\tt a}}
\newcommand{\ctt}{{\tt c}}
\newcommand{\gtt}{{\tt g}}
\newcommand{\ttt}{{\tt t}}
\newcommand{\mbf}{{\bf m}}
\newcommand{\Phibf}{\mbox{\mathversion{bold}{$\Phi$}}}
\newcommand{\Sbf}{{\bf S}}
\newcommand{\bps}{\mbox{bps}}
\newcommand{\vbf}{{\bf v}}
\newcommand{\wbf}{{\bf w}}
\newcommand{\Esp}{{\mathbb E}}
\newcommand{\Var}{{\mathbb V}}
\newcommand{\Indic}{{\mathbb I}}

% Couleur et graphiques
\usepackage{graphics}
\usepackage{epsfig} 
\usepackage{pstcol}

% Texte
\usepackage{lscape, ../../../fancyheadings, rotating, enumerate}
\usepackage[french]{babel}
\usepackage[latin1]{inputenc}
\definecolor{darkgreen}{cmyk}{0.5, 0, 0.5, 0.5}
\definecolor{orange}{cmyk}{0, 0.6, 0.8, 0}
\definecolor{jaune}{cmyk}{0, 0.5, 0.5, 0}
\newcommand{\textblue}[1]{\textcolor{blue}{\bf #1}}
\newcommand{\textred}[1]{\textcolor{red}{\bf #1}}
\newcommand{\textgreen}[1]{\textcolor{darkgreen}{\bf #1}}
%\newcommand{\textgreen}[1]{\textcolor{green}{\bf #1}}
\newcommand{\textorange}[1]{\textcolor{orange}{\bf #1}}
\newcommand{\textyellow}[1]{\textcolor{yellow}{\bf #1}}

% Sections
\newcommand{\chapter}[1]{\centerline{\LARGE \bf \textred{#1}}}
\newcommand{\section}[1]{\centerline{\Large \bf \textblue{#1}}}
\newcommand{\subsection}[1]{\noindent{\large \bf \textblue{#1}}}
\newcommand{\paragraph}[1]{{\textgreen{#1}}}

%%%%%%%%%%%%%%%%%%%%%%%%%%%%%%%%%%%%%%%%%%%%%%%%%%%%%%%%%%%%%%%%%%%%%%
%%%%%%%%%%%%%%%%%%%%%%%%%%%%%%%%%%%%%%%%%%%%%%%%%%%%%%%%%%%%%%%%%%%%%%
\begin{document}
\landscape
\headrulewidth 0pt 
\pagestyle{fancy} 
\cfoot{}
\rfoot{\begin{rotate}{90}{
      \vspace{0.5cm}
      \hspace{-1.5cm} \tiny S. Ledent \& S. Robin (RECOMB 2004)
      }\end{rotate}}
\rhead{\begin{rotate}{90}{
      \hspace{1.5cm} \tiny \thepage
      }\end{rotate}}
%%%%%%%%%%%%%%%%%%%%%%%%%%%%%%%%%%%%%%%%%%%%%%%%%%%%%%%%%%%%%%%%%%%%%%
%%%%%%%%%%%%%%%%%%%%%%%%%%%%%%%%%%%%%%%%%%%%%%%%%%%%%%%%%%%%%%%%%%%%%%

\begin{center}
  \textblue{\Large CHECKING HOMOGENEITY}

  \textblue{\Large OF MOTIFS' DISTRIBUTION }

  \textblue{\Large IN HETEROGENOUS SEQUENCES}

   {\large Sabrina LEDENT$^1$ \& St�phane ROBIN$^{1, 2}$} \\
%   robin@inapg.inra.fr
\end{center}
{\small
  {$(^1)$ INRA Math�matique, Informatique et G�nome, Jouy-en-Josas, FRANCE} \\
  {$(^2)$ INA-PG / INRA Math�matique et Informatique Appliqu�es,
    Paris, FRANCE} }

\centerline{\large RECOMB 2004, San Diego}

%%%%%%%%%%%%%%%%%%%%%%%%%%%%%%%%%%%%%%%%%%%%%%%%%%%%%%%%%%%%%%%%%%%%%%
%%%%%%%%%%%%%%%%%%%%%%%%%%%%%%%%%%%%%%%%%%%%%%%%%%%%%%%%%%%%%%%%%%%%%%
%%%%%%%%%%%%%%%%%%%%%%%%%%%%%%%%%%%%%%%%%%%%%%%%%%%%%%%%%%%%%%%%%%%%%%
%%%%%%%%%%%%%%%%%%%%%%%%%%%%%%%%%%%%%%%%%%%%%%%%%%%%%%%%%%%%%%%%%%%%%%

\newpage
\chapter{Contents}

\subsection{Modeling motifs' occurrences}

\subsection{Homogenous compound Poisson model}

\subsection{Heterogenous compound Poisson model}

\subsection{Checking homogeneity distribution}

\subsection{Applications} 

%%%%%%%%%%%%%%%%%%%%%%%%%%%%%%%%%%%%%%%%%%%%%%%%%%%%%%%%%%%%%%%%%%%%%%
%%%%%%%%%%%%%%%%%%%%%%%%%%%%%%%%%%%%%%%%%%%%%%%%%%%%%%%%%%%%%%%%%%%%%%
\newpage
\chapter{Modeling motifs' occurrences}

\paragraph{Motif statistics} aim at detecting words with unexpected
frequency, regions where a motif is unexpectedly frequent or rare,
etc.

\paragraph{A model is needed} because {\em to decide whether something
  is unexpected or not, we first need to know what to expect.}

\paragraph{Model = Reference}\\
To avoid artifacts, the model should account for 
\vspace{-1cm}
\begin{enumerate}[$\bullet$]
\item the frequencies of nucleotides, or di-, or tri-nucleotides in
  the sequence,
\item the overlapping structure of the word,
\item eventually, the overall frequency of the word in the sequence,
\item the potential heterogeneity of the sequence.
\end{enumerate}

%%%%%%%%%%%%%%%%%%%%%%%%%%%%%%%%%%%%%%%%%%%%%%%%%%%%%%%%%%%%%%%%%%%%%%
%%%%%%%%%%%%%%%%%%%%%%%%%%%%%%%%%%%%%%%%%%%%%%%%%%%%%%%%%%%%%%%%%%%%%%
\newpage
\section{Markov chains (M$m$) = Discrete modeling}
%%%%%%%%%%%%%%%%%%%%%%%%%%%%%%%%%%%%%%%%%%%%%%%%%%%%%%%%%%%%%%%%%%%%%
$\Sbf = (S_1, \dots, S_{\ell})$ is an homogenous stationary Markov
chain \vspace{-1cm}
\begin{enumerate}[$\bullet$]
\item of order $m$ (M$m$ model) over the alphabet $\Acal =
  \{\att, \ctt, \gtt, \ttt\}$
\item with transition probabilities $\pi(s_1, \dots,
  s_m; s_{m+1}) =$
  $$
  \Pr\{S_x = s_{m+1} | S_{x-m}=s_1, \dots, S_{x-1} = s_m\} 
  $$
\end{enumerate}
M$m$ model is fitted to the frequencies of all the words of length
$(m+1)$
$$
\widehat{\pi}(s_1, \dots, s_m; s_{m+1}) = \frac{N(s_1 \dots s_m
  s_{m+1})}{N(s_1 \dots s_m)}
$$

%%%%%%%%%%%%%%%%%%%%%%%%%%%%%%%%%%%%%%%%%%%%%%%%%%%%%%%%%%%%%%%%%%%%%%
\newpage
\section{Compound Poisson process (CP)}
\section{= Continuous modeling} 

For rare words, the sequence $\Sbf$ can be viewed as a
continuous line $[0; \ell]$ 
$$
\hspace{-2cm}
\begin{tabular}{c}
  Real occurrences  \\
                                %FigPoissonComp.ps
  \colorbox{white}{      
    \psfig{file = ../FIGURES/FigPoissonComp.ps,
      height=2cm, width=25cm, bbllx=80 , bblly=370 , bburx=525 ,
      bbury=405, clip=}
    }  \\
  \\
  Compound Poisson modeling \\
                                %FigPoissonComp.ps
  \colorbox{white}{
    \psfig{file = ../FIGURES/FigPoissonComp.ps,
      height=3.9cm, width=25cm, bbllx=80 , bblly=260 , bburx=525 ,
      bbury=330, clip=}
    }  \\
\end{tabular}
$$

%%%%%%%%%%%%%%%%%%%%%%%%%%%%%%%%%%%%%%%%%%%%%%%%%%%%%%%%%%%%%%%%%%%%%%
%%%%%%%%%%%%%%%%%%%%%%%%%%%%%%%%%%%%%%%%%%%%%%%%%%%%%%%%%%%%%%%%%%%%%%
\newpage
\chapter{Homogenous compound Poisson model}
\vspace{-2cm} \begin{flushright} {\em R., 02} \end{flushright}
\vspace{-1cm}

\paragraph{Clump process} $\{C(x)\} =$ Poisson process
with intensity $\equiv \lambda$

\paragraph{Clump sizes} $\{K_1, K_2, \dots\}$ are iid: $\Pr\{K = k\} =
g(k)$

\paragraph{Counting process} of the occurrences $\{N(x)\} = $
compound Poisson process:
$$
N(x) = \sum_{c = 1}^{C(x)} K_c
$$
\paragraph{Remarks:}
\vspace{-1cm}
\begin{enumerate}[$\bullet$]
\item Poisson modeling implies that the clumps are
  {\em uniformly distributed} along the genome: 
  $\longrightarrow$ {\em Null hypothesis} of the
  analysis
\item The simple Poisson process ({\em Karlin \& Macken, 91; Churchill
    \& al., 89}) is the special case where $K_c \equiv 1$.
\end{enumerate}

%%%%%%%%%%%%%%%%%%%%%%%%%%%%%%%%%%%%%%%%%%%%%%%%%%%%%%%%%%%%%%%%%%%%%%
\newpage
\subsection{P\'olya-Aeppli model} \\
When considering one single word, the clump size has a geometric
distribution with parameter $(1-a)$
$$
p(k) = a^{k-1} (1-a) \quad \Longrightarrow
\quad \Esp(K) = 1 / (1-a)
$$
where $a$ is the overlapping probability of the word

\paragraph{Parameter estimates:} In a sequence of length $\ell$ 
\vspace{-1cm}
\begin{enumerate}[$\bullet$]
\item $\widehat{\lambda}$ is the empirical frequency of the
  clumps: $ \widehat{\lambda} = C(\ell) / \ell $
\item $\widehat{a}$ is the proportion of overlapped occurrences:
  $\widehat{a} = 1 - C(\ell) / N(\ell)$
\end{enumerate}

\paragraph{Properties}
\vspace{-1cm}
\begin{enumerate}[$\bullet$]
\item P\'olya-Aeppli is the best approximation of the distribution of
  the word count in the Markov model ({\em R. \& Schbath, 01})
\item $\Esp[N(\ell)] = \ell \times \lambda \times \Esp(K) \quad
  \Longrightarrow \quad \widehat{\Esp}N(\ell) = \ell
  \widehat{\lambda} / (1-\widehat{a}) = N(\ell)$ \\
  \centerline{$\Longrightarrow \quad$ {\em no word has an
      ``unexpected'' count}}
\end{enumerate}

%%%%%%%%%%%%%%%%%%%%%%%%%%%%%%%%%%%%%%%%%%%%%%%%%%%%%%%%%%%%%%%%%%%%%%
%%%%%%%%%%%%%%%%%%%%%%%%%%%%%%%%%%%%%%%%%%%%%%%%%%%%%%%%%%%%%%%%%%%%%%
\newpage
\chapter{Heterogenous compound Poisson model}
\section{~}
\section{Exogenous information}

\vspace{-0.5 cm}
\paragraph{Assumptions:}
\vspace{-1cm}
\begin{enumerate}[$\bullet$]
\item The heterogeneity of the sequence can be described in
  terms of $S$ underlying states;
\item We know functions $\pi_s(x) \in [0; 1]$, {\em independently from
    the occurrences of the motif}, that relates position $x$ to state
  $s$.
\end{enumerate}
\paragraph{Example:} States = $\{$CDS+, CDS-, inter-genic$\}$, \
\vspace{-1cm}
\begin{enumerate}[$\bullet$]
\item genome annotation provides a segmentation
  \begin{eqnarray*}
    \pi_s(x) & = & 1 \qquad \mbox{if position $x$ belongs to state
      $s$} \\
    & = & 0 \qquad \mbox{otherwise}
    \end{eqnarray*}
  \item HMM modeling provides
    $$\pi_s(x) = \Pr\{\mbox{position $x$ belongs
      to state $s$}\}
    $$
\end{enumerate}

%%%%%%%%%%%%%%%%%%%%%%%%%%%%%%%%%%%%%%%%%%%%%%%%%%%%%%%%%%%%%%%%%%%%%%
\newpage
\section{Heterogeneity modeling}

\paragraph{Intensity of the clump process:}
$\lambda(x)$ is a mixture of $S$ constant intensities $\lambda_1,
\dots \lambda_S$
$$
\lambda(x) = \sum_s \lambda_s \pi_s(x).
$$

\paragraph{Distribution of the clump size:}
$p(k \;|\; x)$ is a mixture of $S$ distributions $p_1(k), \dots,
p_S(k)$:
$$
p(k \;|\; x) = \sum_s p_s(k) \lambda_s \pi_s(x) \left/ \sum_s
\lambda_s \pi_s(x) \right..
$$
If all the $p_s$'s are geometric with respective parameters
$(1-a_s)$:
$$
p(k \;|\; x) = \sum_s a_s^{k-1} (1-a_s) \lambda_s
  \pi_s(x) \left/ \sum_s \lambda_s \pi_s(x) \right..
$$

%%%%%%%%%%%%%%%%%%%%%%%%%%%%%%%%%%%%%%%%%%%%%%%%%%%%%%%%%%%%%%%%%%%%%%
\newpage
\section{Parameter estimates}

\paragraph{Strict segmentation:} $\pi_s(x) = 0$ or 1 (genome
annotation) \\
Parameters ($\lambda_s$, $p_s$ or $a_s$) are estimated state by state,
as in the homogenous CP model

\paragraph{General case:} $\pi_s(x) \in [0; 1]$ (HMM) \\
Maximum likelihood estimates can be obtained, using the
Newton-Raphston algorithm.

The estimates of the segmentation can be adapted to provide good
starting values for the algorithm.

\paragraph{Semi-heterogenous models:} \\
For some of them, explicit estimates can be derived.

%%%%%%%%%%%%%%%%%%%%%%%%%%%%%%%%%%%%%%%%%%%%%%%%%%%%%%%%%%%%%%%%%%%%%%
\newpage
\section{Semi-heterogenous models} 
$\begin{array}{l|rl|rl}
  \mbox{model: \# param.} 
  & \multicolumn{2}{c|}{\lambda(x) \equiv \lambda} & 
  \multicolumn{2}{c}{\lambda(x) = \sum_s \pi_s(x) \lambda_s} \\
  & & & \\
  \hline 
  & & & \\
  a^{k-1} (1-a)                     & P_0A_0: & 2       & PA_0: & S+1 \\
  & & & \\
  \sum_s \pi_s(x) a_s^{k-1} (1-a_s) & P_0A: & S+1       & PA: & 2S \\
  & & & \\
  p(k)                              & P_0G_0: & k_{\max}     & PG_0: & S+k_{\max}-1 \\
  & & & \\
  \sum_s \pi_s(x) p_s(k)            & P_0G: & S(k_{\max}-1)+1  & PG: & Sk_{\max} \\
\end{array}$

\paragraph{The fits of these different models} can be compared with AIC

\paragraph{Restrictive hypotheses} can be tested with the likelihood
ratio test (LRT), or with the chi square test

%%%%%%%%%%%%%%%%%%%%%%%%%%%%%%%%%%%%%%%%%%%%%%%%%%%%%%%%%%%%%%%%%%%%%%
%%%%%%%%%%%%%%%%%%%%%%%%%%%%%%%%%%%%%%%%%%%%%%%%%%%%%%%%%%%%%%%%%%%%%%
\newpage
\chapter{Checking homogeneity distribution}
\section{~}
\section{Two statistics} 
We are looking for poor or rich regions in terms of occurrences of a
given motif. \\
A natural criterion for a given region is the ratio
$$
\frac{\mbox{number of occurrences in the
    region}}{\mbox{size of the region}}
$$
$$
\hspace{-1.5cm}
\psfig{file = ../FIGURES/FigHomogene.ps, height=7.6cm,
  width=25cm, bbllx=70 , bblly=590 , bburx=515 , bbury=725 , clip=}  
$$

%%%%%%%%%%%%%%%%%%%%%%%%%%%%%%%%%%%%%%%%%%%%%%%%%%%%%%%%%%%%%%%%%%%%%%
\newpage
\subsection{$Y^r = $ cumulated distances} of order $r$:
$$
\frac{\mbox{fixed numerator $r$}}{\mbox{random
    denominator $Y^r$}}
$$
where
$$
Y_i^r = \sum_{j=i}^{i+r-1} Y_j  
= \left( 
  \mbox{\begin{tabular}{c}
      distance between occurrence number $i$ \\
      and occurrence number $i+r$
    \end{tabular}}
\right)
$$
  ({\em Karlin \& Macken, 91}: $r$-scans).

\paragraph{Distribution under homogenous PC:} if the simple distances are iid
(P\'olya-Aeppli) 
\vspace{-1cm}
\begin{enumerate}[$\bullet$]
\item the distribution of $Y^r$ is known: {\em R. (02)};
\item $\Pr\{\min_i Y_i^r \leq y\} \simeq e^{-[N(\ell)-r] \Pr\{Y^r
  \leq y\}}$ \\
  + Chen-Stein bound for the error : {\em Arratia \& al (89), Dembo \&
    Karlin (92), R. (02)}.
\end{enumerate}

%%%%%%%%%%%%%%%%%%%%%%%%%%%%%%%%%%%%%%%%%%%%%%%%%%%%%%%%%%%%%%%%%%%%%%
\newpage
\subsection{$\Delta N =$ local count} in a window of width $y$: 
$$
\frac{\mbox{random numerator $\Delta N$}}{\mbox{fixed denominator
    $y$}} 
$$
where
$$
\Delta N(x) = N(x) -  N(x-y)
= \left( 
  \mbox{\begin{tabular}{c}
      number of occurrences  \\
       between $x-y$ and $x$
    \end{tabular}}
\right)
$$

\paragraph{Distribution under homogenous PC:} 
\vspace{-1cm}
\begin{enumerate}[$\bullet$]
\item the distribution of $\Delta N(x)$ is known: {\em Barbour \& al. (92)};
\item the approximation
  $$
  \Pr\{\max_x \Delta N(x) \geq n\} \simeq e^{-[\ell - y]
    \Pr\{\Delta N \geq n\}}
  $$
  is optimal but no bound is known : {\em Barbour \& al. (92), R.
    (02)}.
\end{enumerate}

%%%%%%%%%%%%%%%%%%%%%%%%%%%%%%%%%%%%%%%%%%%%%%%%%%%%%%%%%%%%%%%%%%%%%%
\newpage
\section{Homogenization technique} 

Once the $\lambda_s$'s have been estimated, the heterogenous Poisson
clump process $C(x)$ can be homogenized into a {\em fictitious
  homogenous} Poisson process $\Ctilde(x)$.

Denoting $Z_c$ the true positions of clump number $c$ ($Z_0 = 0$), its
`homogenized' position $\Ztilde_c$ is defined by
$$
\Ztilde_c = \sum_s \widehat{\lambda}_s \int_0^{Z_c} \pi_s(x) \D x.
$$

\paragraph{Properties:}
\vspace{-1cm}
\begin{enumerate}[$\bullet$]
\item This procedure does not homogenize the clump size.
\item If the clump size distribution is homogenous ($PA_0$ and $PG_0$
  models), techniques developed for the homogenous PC model can be
  applied to the homogenized process.
\end{enumerate}

%%%%%%%%%%%%%%%%%%%%%%%%%%%%%%%%%%%%%%%%%%%%%%%%%%%%%%%%%%%%%%%%%%%%%%
\newpage
\subsection{Illustration} \\
Occurrences of $\att\att\ttt\ttt$ in the genome of phage {\em Lambda}
($\ell
= 48\;500$ bps) \\
\begin{tabular}{rl}
  \hspace{-1cm}
  \epsfig{file = ../FIGURES/lambda-aatt-hetero.ps, height = 9cm, width =
    12cm} 
  & 
  \hspace{-1cm}
  \epsfig{file = ../FIGURES/lambda-aatt-homo.ps, height = 9cm, width =
    12cm} 
\end{tabular}
\paragraph{3 steps:} 
\vspace{-1cm}
\begin{enumerate}
\item Estimate the intensity $\lambda(x)$ (left: green line)
\item ``Homogenize'' the clump process and calculate thresholds
  (right: red line + blue lines for the bounds)
\item come back to the original process (left)
\end{enumerate}

%%%%%%%%%%%%%%%%%%%%%%%%%%%%%%%%%%%%%%%%%%%%%%%%%%%%%%%%%%%%%%%%%%%%%%
\newpage
\chapter{Applications} 
\section{~}
\section{CHI motif of {\em H. influenzae}}

$\gtt{\tt N}\ttt\gtt\gtt\ttt\gtt\gtt$ occurs $N(\ell) = 223$ times, in
$C(\ell) = 215$ clumps, in the genome ($\ell = 1\;830\;140$ bps). 

\paragraph{Heterogeneity} is modeled through posterior probabilities
of an HMM with 3 hidden states corresponding to CDS-, CDS+ and
inter-genic regions ({\em source: P. Nicolas}).

\paragraph{Model $PA_0$} provides the best fit (LRT + AIC): 
$\widehat{a} = 3.59 \; 10^{-2} $, $
\widehat{\lambda}_{\mbox{\footnotesize CDS-}} = 0.51 \; 10^{-4}$,
$\widehat{\lambda}_{\mbox{\footnotesize CDS+}} = 2.14 \; 10^{-4}$,
$\widehat{\lambda}_{\mbox{\footnotesize inter}} = 0.05 \; 10^{-4}$.

\paragraph{The shortest $Y^3$} is 625 bps long. \\
The corresponding homogenized distance, $\Ytilde^3 = 0.132$, is non
significant.

%%%%%%%%%%%%%%%%%%%%%%%%%%%%%%%%%%%%%%%%%%%%%%%%%%%%%%%%%%%%%%%%%%%%%%
\newpage
\section{Palindromes in {\em E. coli}'s genome}

The 64 palindromes of length 6 occur 54\;724 times, in 50\;941 clumps,
in the genome ($\ell = 4\;638\;868$ bps).

\paragraph{Heterogeneity} is modeled in the same way as in {\em
  H. influenzae}.

\paragraph{Model $PG_0$} appears to be the most parsimonious (best
AIC), although the LR test $PG$ / $PG_0$ is significant (p-value = 2\%).

\paragraph{The richest windows} of width $y = 10\;000$ bps contains
185 occurrences and is significantly rich in the $P_0G_0$ model.

The richest homogenized window contains 171 occurrences:
$$
\begin{array}{rclcl}
  \Pr\{\Delta N(x) \geq 171 \} & = & 4.5 \; 10^{-5} 
  & \Rightarrow & \mbox{significant} \\
  \mbox{but} \quad   \Pr\{\sup_x \Delta N(x) \geq 171 \} & \gg & 5 \; 10^{-2} 
  & \Rightarrow & \mbox{non significant}
\end{array}
$$

%%%%%%%%%%%%%%%%%%%%%%%%%%%%%%%%%%%%%%%%%%%%%%%%%%%%%%%%%%%%%%%%%%%%%%
\newpage
\section{CHI motif of {\em E. coli}}

The Chi motif occurs 761 in the genome of {\em E. coli} K12. \\
This genome can be segmented between backbone and loops ({\em El
  Karoui \& al., 03)}


\paragraph{Homogenous Poisson model:}
According to cumulated distances of order 4, 2 regions are found
significantly rich \vspace{-1cm}
\begin{enumerate}[($a$)]
\item $Y^4 = 1495$ bps, around position 366\;123 ($p$-value =
  $1.24\;10^{-4}$),
\item $Y^4 = 1204$ bps, around position 4\;264\;495 ($p$-value =
  $0.51\;10^{-4}$).
\end{enumerate}

\paragraph{Heterogenous Poisson model:} Chi is more abundant in
backbone regions than in loops
($\widehat{\lambda}_{\mbox{\footnotesize backbone}} = 1.8\;10^{-4},
\widehat{\lambda}_{\mbox{\footnotesize loop}} = 0.9\;10^{-4}$)
    
Only region ($b$) is still significantly rich, with $p$-value =
$0.8\;10^{-4}$.

%%%%%%%%%%%%%%%%%%%%%%%%%%%%%%%%%%%%%%%%%%%%%%%%%%%%%%%%%%%%%%%%%%%%%%
\newpage
\chapter{Conclusions} 

\paragraph{The heterogenous compound Poisson model}
accounts for critical informations such as 
\vspace{-1cm}
\begin{enumerate}[$\bullet$]
\item the frequency and overlapping probability of the motif,
\item the heterogeneity of the sequence.
\end{enumerate}
\vspace{-0.5cm}
\paragraph{The homogenization technique}
\vspace{-1cm}
\begin{enumerate}[$\bullet$]
\item provides an adaptive threshold for both cumulated distances and
  local counts
\item but is not really efficient for heterogeneity of the clump size.
\end{enumerate}
\vspace{-0.5cm}
\paragraph{Results seem to become less significant} when going from
homogenous to heterogenous modeling. \\
Further theoretical statistical properties or simulations are needed
to know if this loss of power is a general property of this approach.

%%%%%%%%%%%%%%%%%%%%%%%%%%%%%%%%%%%%%%%%%%%%%%%%%%%%%%%%%%%%%%%%%%%%%%
%%%%%%%%%%%%%%%%%%%%%%%%%%%%%%%%%%%%%%%%%%%%%%%%%%%%%%%%%%%%%%%%%%%%%%
%%%%%%%%%%%%%%%%%%%%%%%%%%%%%%%%%%%%%%%%%%%%%%%%%%%%%%%%%%%%%%%%%%%%%%
%%%%%%%%%%%%%%%%%%%%%%%%%%%%%%%%%%%%%%%%%%%%%%%%%%%%%%%%%%%%%%%%%%%%%%
\end{document}
%%%%%%%%%%%%%%%%%%%%%%%%%%%%%%%%%%%%%%%%%%%%%%%%%%%%%%%%%%%%%%%%%%%%%%
%%%%%%%%%%%%%%%%%%%%%%%%%%%%%%%%%%%%%%%%%%%%%%%%%%%%%%%%%%%%%%%%%%%%%%
%%%%%%%%%%%%%%%%%%%%%%%%%%%%%%%%%%%%%%%%%%%%%%%%%%%%%%%%%%%%%%%%%%%%%%
%%%%%%%%%%%%%%%%%%%%%%%%%%%%%%%%%%%%%%%%%%%%%%%%%%%%%%%%%%%%%%%%%%%%%%

