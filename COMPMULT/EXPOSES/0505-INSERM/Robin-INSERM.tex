\documentclass[10pt]{article}
\usepackage{amssymb}
\usepackage{amsfonts, amsmath}
\usepackage{epsfig, color}
\usepackage{enumerate}
\usepackage{/Latex/astats}
%\usepackage[latin1]{inputenc}

\newcommand{\fbf}{{\bf f}}
\newcommand{\gbf}{{\bf g}}
\newcommand{\Kbf}{{\bf K}}
\newcommand{\Unbf}{{\bf 1}}
\newcommand{\ubf}{{\bf u}}
\newcommand{\vbf}{{\bf v}}
\newcommand{\wbf}{{\bf w}}
\newcommand{\alphabf}{\mbox{\mathversion{bold}{$\alpha$}}}
\newcommand{\betabf}{\mbox{\mathversion{bold}{$\beta$}}}
\newcommand{\gammabf}{\mbox{\mathversion{bold}{$\gamma$}}}
\newcommand{\phibf}{\mbox{\mathversion{bold}{$\phi$}}}
\newcommand{\psibf}{\mbox{\mathversion{bold}{$\psi$}}}
\newcommand{\taubf}{\mbox{\mathversion{bold}{$\tau$}}}

\newcommand{\Rbb}{{\mathbb R}}

\newcommand{\Bcal}{{\mathcal B}}
\newcommand{\Ecal}{{\mathcal E}}
\newcommand{\Fcal}{{\mathcal F}}
\newcommand{\Lcal}{{\mathcal L}}
\newcommand{\Ncal}{{\mathcal N}}
\newcommand{\Ucal}{{\mathcal U}}
\newcommand{\Xcal}{{\mathcal X}}
\newcommand{\Ycal}{{\mathcal Y}}

\newcommand{\Esp}{{\mathbb E}}
\newcommand{\Var}{{\mathbb V}}
\newcommand{\lFDR}{\ell FDR}

\newtheorem{theorem}{Theorem}
\newtheorem{lemma}[theorem]{Lemma}
\newtheorem{corollary}[theorem]{Corollary}
\newtheorem{algorithm}[theorem]{Algorithm}

\textwidth 16cm
\textheight 22cm
\topmargin 0 cm
\oddsidemargin 0cm
\evensidemargin 0cm
\renewcommand{\baselinestretch}{1.2}    

%%%%%%%%%%%%%%%%%%%%%%%%%%%%%%%%%%%%%%%%%%%%%%%%%%%%%%%%%%%%%%%%%%
%%%%%%%%%%%%%%%%%%%%%%%%%%%%%%%%%%%%%%%%%%%%%%%%%%%%%%%%%%%%%%%%%%
\begin{document}
%%%%%%%%%%%%%%%%%%%%%%%%%%%%%%%%%%%%%%%%%%%%%%%%%%%%%%%%%%%%%%%%%%
%%%%%%%%%%%%%%%%%%%%%%%%%%%%%%%%%%%%%%%%%%%%%%%%%%%%%%%%%%%%%%%%%%
\pagestyle{empty}
\title{\bf A semi-parametric approach for mixture models:
  Application to  local FDR estimation }

% \author{St\'{e}phane Robin, Avner Bar-Hen, Jean-Jacques Daudin\\INA-PG / INRA Biom\'{e}trie, 16 rue Claude Bernard,
%     75005 Paris, {\sc France}\\ Laurent Pierre\\ Universit\'{e} Paris X,  200 avenue de la R\'{e}publique, 92001  Nanterre Cedex}
\author{St\'{e}phane Robin, Avner Bar-Hen, Jean-Jacques Daudin,
  Laurent Pierre} 
\date{ }
\maketitle

\pagestyle{empty}

\newpage

% \begin{abstract}
% In this article we propose a procedure to estimate a two-components mixture model where one component is known. The unknown part is estimated with a weighted
% kernel function. The weights are defined in an adaptative way.  We prove the  convergence and unicity of our estimation procedure. This results are applied to
% multiple testing procedure so as to estimate the posterior population probabilities and the local FDR.
% \bigskip

% {\em Key words:} FDR, Mixture model,Multiple testing procedure, Semi-parametric density estimation
% \end{abstract}


%%%%%%%%%%%%%%%%%%%%%%%%%%%%%%%%%%%%%%%%%%%%%%%%%%%%%%%%%%%%%%%%%%
%%%%%%%%%%%%%%%%%%%%%%%%%%%%%%%%%%%%%%%%%%%%%%%%%%%%%%%%%%%%%%%%%%
\section{Introduction}
%%%%%%%%%%%%%%%%%%%%%%%%%%%%%%%%%%%%%%%%%%%%%%%%%%%%%%%%%%%%%%%%%%



Probability density estimation is a classical approach to classification. The population densities are estimated  and combined with the estimated or known
prior population probabilities to construct discrimination rules that perform optimally, if the density estimates are accurate. While it is true that optimal
density estimates are not necessary for good discrimination rules, it is sometimes better to estimate the population posterior probabilities directly. Local
FDR is a classical example and will developed in section 5

Let consider the two population mixture model. The density of the data can be written as\[g(x)=af(x)+(1-a)\phi(x)\] where $f$ and $\phi$ are the population
densities and $a$ is the prior population probability.

In this model, the posterior probability for the observation $i$ to be issued from the distribution $f$ given the observed value $x_i$ is
\begin{equation} \label{Eq:Tau}
\tau_i = a f(x_i) \left/ g(x_i) \right..
\end{equation}

In many situations, the two populations are not equivalent and the amount of information on the population density is very different. For example, if $f$ is a
contamination of $\phi$, it is often reasonable to make assumptions about the density of $\phi$ while $f$ is completely unknown. The same situation arises in
microarrays analysis, neuro-imaging or any multiple testing situation: the distribution of the $p$-value under $H_0$ is uniformly distributed on $[0,1]$ (as
soon as the distribution of test statistic is valid) while the distribution of the $p$-values associated to differentially expressed genes is unknown.


The two classical approaches to probability density estimation are the parametric and the non parametric ones. In parametric estimation, the density underlying
the data is assumed to belong to some rather restricted family of functions $f(\theta)$. Such approach is very efficient as soon as the family $f(\theta)$ is
close to $f$. Therefore this approach can be recommended in case of small amount of data or when assumptions are realistic.

In the framework of multiple testing, mixture model (see, for example, McLahan and Peel \cite{MaP00} ) have been proposed.
% with hypothesis on the
% modified genes distribution, frequentist and bayesian approaches
A key point is to note that the distribution of the $p$-values is uniform on the interval [0,1] regardless of the statistical test used (provided that test is
valid) and regardless of the sample size. In contrast, under the alternative hypothesis, the distribution of $p$-values is concentrated near zero and null near
one. Allison et al. \cite{AGH02} proposed the use of finite mixture models of beta laws (since a beta(1,1) is a uniform distribution). They estimated the
number of component as well as the parameters. Pounds and Morris (\cite{POM03}) is a restricted version of this work. This work has been extended to the
mixture of a uniform and a beta distribution by Liao et al. \cite{LLS04}. Other approaches based on mixture models in a Bayesian framework have also been
proposed (see for example Bro\"{e}t et al. \cite{BLR04} or Newton et al. \cite{NNS04}).

In nonparametric density estimation, no parametric assumption about $f$ is made except some smoothness properties. The shape of the density  is determined by
the data and arbitrary densities $f$ can be estimated accurately. A classical nonparametric estimator is the kernel estimator based on local smoothing of the
data. In practice it requires large amount of data and heavy smoothing can lead to large bias. It is also computationally expensive.


In this paper we propose a non parametric estimate of $\tau_i$ that
use information on $\phi$. The difficulty arises from the fact that
some observations are coming from the unknown density $f$ while some
observations are coming from the parametric density $\phi$.


The idea to mix parametric and nonparametric estimate is not new. Olkin and Spiegelman \cite{OlS87} proposed to use a linear combination of a parametric
estimate and a nonparametric estimates (there is a huge literature on aggregation of classifiers, see Acuna and Rojas \cite{AcR01} for aggregation of
classifier). Hjort and Glad \cite{HjG95} proposed to update parametric estimate by nonparametric correction functions. The reverse idea using properties of the
exponential family was developed by Efron and Tibshirani \cite{EfT96}. Priebe and Marchette \cite{PrM00} and Di Marzio and Taylor \cite{DMT04} proposed to use
parametric estimates for the weights of kernel density estimation. Using projection pursuit density estimation framework, Hoti and Holmstr\"om \cite{HoH04}
proposed to estimate a mixture of normal densities with kernels functions.

The general approach of our work and the main result are presented in Section 2. Practical issues are discussed in Section 3. Application to the multiple
testing procedure, estimation of FDR, FNR and local FDR is studied in Section 4. The last section is devoted to the application of the proposed procedure to
the classical Hedenfalk  dataset \cite{HDC01}.

%\paragraph{Motivation.} Multiple testing problems in large datasets,
%FDR and local FDR estimation
%
%\paragraph{Biblio.}
%
%\paragraph{Approach.} Semi parametric mixture model

%%%%%%%%%%%%%%%%%%%%%%%%%%%%%%%%%%%%%%%%%%%%%%%%%%%%%%%%%%%%%%%%%%%
%%%%%%%%%%%%%%%%%%%%%%%%%%%%%%%%%%%%%%%%%%%%%%%%%%%%%%%%%%%%%%%%%%%
%\section{Mixture model}
%%%%%%%%%%%%%%%%%%%%%%%%%%%%%%%%%%%%%%%%%%%%%%%%%%%%%%%%%%%%%%%%%%%
%We consider the mixture density $g$ defined as
%\begin{equation} \label{Eq:MixtureModel}
%g(x) = a f(x) + (1-a) \phi(x)
%\end{equation}
%where the proportion $a$ and the density $f$ are unknown, while the
%density $\phi$ is completely specified. Typically, $\phi$ is the
%standard normal probability density function (pdf) or the uniform pdf
%over $(0, 1)$. In the following, we denote $G$, $F$ and $\Phi$ the
%cumulative distribution functions (cdf) of $g$, $f$ and $\phi$,
%respectively.
%
%In this model, the posterior probability for the observation $i$ to be
%issued from the distribution $f$ given the observed value $x_i$ is
%\begin{equation} \label{Eq:Tau}
%\tau_i = a f(x_i) \left/ g(x_i) \right..
%\end{equation}
%% In the multiple testing framework described above, we expect
%% $\tau_i$ to decrease with $x_i$. This implies that
%% $$
%% \frac1{\tau_i} = 1 + \frac{(1-a)}a  \frac{\phi(x_i)}{f(x_i)}
%% $$
%% increases with $x_i$, so $f(x) / \phi(x)$ should be a decreasing
%% function of $x$.
%
%%%%%%%%%%%%%%%%%%%%%%%%%%%%%%%%%%%%%%%%%%%%%%%%%%%%%%%%%%%%%%%%%%
%%%%%%%%%%%%%%%%%%%%%%%%%%%%%%%%%%%%%%%%%%%%%%%%%%%%%%%%%%%%%%%%%%
\section{Estimation of the unknown density}
%%%%%%%%%%%%%%%%%%%%%%%%%%%%%%%%%%%%%%%%%%%%%%%%%%%%%%%%%%%%%%%%%%

%%%%%%%%%%%%%%%%%%%%%%%%%%%%%%%%%%%%%%%%%%%%%%%%%%%%%%%%%%%%%%%%%%
\subsection{Kernel estimate}
%%%%%%%%%%%%%%%%%%%%%%%%%%%%%%%%%%%%%%%%%%%%%%%%%%%%%%%%%%%%%%%%%%
Since $f$ is completely unspecified, it has to be estimated in a
non-parametric way. Let $k$ denote an arbitrary cdf (called kernel
function), the standard kernel estimate of $f$ is
$$
\widehat{f}(x) = \left[\sum_i Z_i k_i(x)\right] \left/ \sum_i Z_i \right..
$$
where $k_i(x) = k[(x-x_i)/h]/h$, $h$ is the bandwidth of the kernel and $Z_i$ is one if the data $x_i$ comes from $f$ and 0 otherwise.

This estimate can not be directly used since the $\{Z_i\}$ are unknown. We propose to replace them with their conditional expectation given the data $\{x_i\}$
that are equal to the posterior probabilities: $\Esp(Z_i \;|\; x_i) = \tau_i$ as defined in equation~\ref{Eq:Tau}. We get the following estimate for $f$:
\begin{equation} \label{Eq:WeightedKernel}
\widehat{f}(x) = \left(\sum_i \tau_i k_i(x)\right)
\left/ \sum_i \tau_i \right..
\end{equation}
This estimates is a {\em weighted kernel estimates} where each observation is weighted according to its posterior probability to be issued from $f$.


%%%%%%%%%%%%%%%%%%%%%%%%%%%%%%%%%%%%%%%%%%%%%%%%%%%%%%%%%%%%%%%%%%
\subsection{Estimation of the posterior probabilities}
%%%%%%%%%%%%%%%%%%%%%%%%%%%%%%%%%%%%%%%%%%%%%%%%%%%%%%%%%%%%%%%%%%

The conjunction of (\ref{Eq:WeightedKernel}) and (\ref{Eq:Tau})
implies that the vector $\widehat{\taubf}$ containing the estimated
posterior probabilities $\tau_i$ must satisfy
$$
\widehat{\taubf} = \psibf(\widehat{\taubf})
$$
where $\psibf$ maps $\Rbb^n$ into $\Rbb^n$:
\begin{equation} \label{Eq:Psi}
  \mbox{ For all } \ubf  = (u_1 \dots u_n) \in \Rbb^n: \psi_j(\ubf) =
  \frac{\sum_i u_i b_{ij}}{\sum_i u_i b_{ij}+ \sum_i u_i},
  \qquad \mbox{with } b_{ij} = \frac{a}{1-a} \frac{k_i(x_j)}{\phi(x_j)}.
\end{equation}
$\widehat{\taubf}$ must therefore be a fix point of the function
$\psibf$.

\begin{theorem} \label{Thm:Contract}
  If all coefficients $b_{ij}$ are positive, the function $\psibf$ has
  a unique fix point $\ubf^*$ and the sequence $\ubf^{\ell+1} =
  \psibf(\ubf^{\ell})$ converges towards it for any initial value
  $\ubf^0$.
\end{theorem}

The proof of this theorem is based on the decomposition of $\psibf$ as
$\psibf = \alphabf \circ \betabf \circ \gammabf$ where $\alphabf$,
$\betabf$ and $\gammabf$ are functions mapping from $\Rbb^n$ into
$\Rbb^n$:
$$
\alpha_j(\ubf) = \frac{u_j}{u_j + 1},
\qquad
\beta_j(\ubf) = \sum_i b_{ij} u_i,
\qquad
\gamma_j(\ubf) = \frac{u_j}{\sum_i u_i}.
$$
$\gammabf$ is actually the projection onto the simplex
$\Ecal = \{\ubf \in \Rbb^n: \sum_i u_i = 1\}$. \\
The proof requires the three following lemmas, the proofs of which are
given in Appendix.

\begin{lemma} \label{Lem:Simplex}
  $\ubf^*$ is a fix point of $\psibf$ if and only if $\vbf^* = \gammabf(\ubf^*)$ is
  a fix point of $\gammabf \circ \psibf$.
\end{lemma}

\begin{lemma} \label{Lem:DefDistance}
  Consider the interior $\Fcal$ of the simplex $\Ecal$: $\Fcal =
  \{\ubf \in \Ecal: \mbox{ For all } i, u_i > 0\}$. The function $d$ mapping
  $\Fcal \times \Fcal$ into $\Rbb+$:
  $$
  d(\ubf, \vbf) = \ln \left[ \max_i \left(\frac{u_i}{v_i}\right)
    \left/ \min_i \left(\frac{u_i}{v_i}\right) \right. \right]
  $$
  is a distance.
\end{lemma}

\begin{lemma} \label{Lem:Decrease}
  For any $\vbf$ and $\wbf$ in $\Fcal$, we have
  $$
  d(\gammabf \circ \psibf(\vbf), \gammabf \circ \psibf(\wbf))  <
  d(\vbf, \wbf)
  $$
  if $\vbf \neq \wbf$, and $d(\gammabf \circ \psibf(\vbf), \gammabf
  \circ \psibf(\wbf)) = d(\vbf, \wbf) = 0$ otherwise.
\end{lemma}

\paragraph{Proof of Theorem {\bf \ref{Thm:Contract}}.}
Thanks to Lemma {\bf \ref{Lem:Simplex}}, we can restrict the proof to
the study of the convergence of the sequence $\vbf^{\ell+1} = \gammabf
\circ \psibf(\vbf^{\ell})$ in the simplex $\Ecal$. Since $\Ecal$ is a
compact and $\gammabf \circ \psibf$ is continuous, Brouwer's theorem
insures that $\gammabf \circ \psibf$ admits at least one fix point in
$\Ecal$.

Furthermore, since every $b_{ij}$ is strictly greater than zero, the
image $\gammabf \circ \psibf(\vbf)$ of any element $\vbf$ of $\Ecal$
can not have any null coordinate. That is: the function $\gammabf
\circ \psibf$ sends the elements of the border of $\Ecal$ into its
interior $\Fcal$. So the fix points of $\gammabf \circ \psibf$
necessarily belong to $\Fcal$.

Lemma {\bf \ref{Lem:Decrease}} proves that $\gammabf \circ \psibf$ admits at most one fix point since the 2 fix points case would contradict the lemma for
$\vbf \neq \wbf$. This implies that there exist a unique fix point. Finally, Lemma {\bf \ref{Lem:Decrease}} says that the distance $d$ (Lemma {\bf
\ref{Lem:DefDistance}}) strictly decreases when the function $\gammabf \circ \psibf$ is applied. This shows that the iteration of the function $\gammabf \circ
\psibf$ necessarily converges toward its unique fix point and achieves the proof.  $\blacksquare$

\paragraph{Hypothesis on the $b_{ij}.$}
This hypothesis may be relaxed. The second argument of the proof still
holds if the function $\psibf$ sends any element of the border of
$\Ecal$ into the interior in a finite number of iterations.  In terms
of kernel estimate, the convergence is therefore guarantied for kernel
with non-compact support (such as the Gaussian kernel), or if no
observation $x_i$ is isolated within its kernel, i.e. if
$$
\mbox{ For all } i: \min_{j \neq i} |x_i - x_j| < h.
$$


%%%%%%%%%%%%%%%%%%%%%%%%%%%%%%%%%%%%%%%%%%%%%%%%%%%%%%%%%%%%%%%%%%
\subsection{Estimation algorithm}
%%%%%%%%%%%%%%%%%%%%%%%%%%%%%%%%%%%%%%%%%%%%%%%%%%%%%%%%%%%%%%%%%%

The iteration of the function $\psibf$ can be decomposed in 3 sub-steps, as exposed in Algorithm \ref{Algo:Estim}.

\begin{algorithm} \label{Algo:Estim}
  Let $\widehat{f}^{(h)}$, $\widehat{g}^{(h)}$ and
  $\widehat{\tau}_i^{(h)}$ denote the estimates of $f$, $g$ and
  $\tau_i$ after step $h$.
  $$
  \fbox{
    \begin{tabular}{p{2cm}p{3cm}p{9cm}}
      \multicolumn{2}{p{5cm}}{\bf Initialization:} & Set $\widehat{\tau}^0_i$
      to 1 for the proportion $a$ of the smallest $x_i$ and to 0 for
      the remaining. \\
      \\
      {\bf Step $l$} & estimation of $f$: &  $\widehat{f}^{(l)}(x) = \sum_i
      \widehat{\tau}_i^{(l-1)} k_i(x) \left/ \sum_i \widehat{\tau}_i^{(l-1)}
      \right. $\\
      \\
      & estimation of $g$: & $\widehat{g}^{(l)}(x) = a \widehat{f}^{(l)}(x) +
      (1-a) \phi(x)$ \\
      \\
      & update of $\{\tau_i\}$: & $\widehat{\tau}_i^{(l)} = a
      \widehat{f}^{(l)}(x_i) \left/ \widehat{g}^{(l)}(x_i) \right. $  \\
      \\
      \multicolumn{2}{p{5cm}}{\bf Stopping rule} & Stop when $\max_i
      |\widehat{\tau}^{(l)}_i - \widehat{\tau}^{(l-1)}_i| /
      \widehat{\tau}^{(l-1)}_i < \varepsilon$.
    \end{tabular}
    }
  $$
\end{algorithm}

\paragraph{Connexion with the E-M algorithm.}
Algorithm \ref{Algo:Estim} has some Expectation-Maxi\-mi\-za\-tion
(E-M) flavor.  Actually the updating of the $\widehat{\tau}_i$ is
equivalent to the E step. Moreover, considering the $\{k_i(x)\}$ as
data, $\widehat{f}(x)$ can be seen as an average of them, so the
updating of $\widehat{f}$ may look like an M step.  \\
However, this comparison is not valid since kernel estimates do not
aim at maximizing the likelihood of the data (like E-M does), but
typically to minimize the norm of $(\widehat{f} - f)$.  Therefore,
Algorithm \ref{Algo:Estim} can not be justified in the standard E-M
framework for mixture models.

%%%%%%%%%%%%%%%%%%%%%%%%%%%%%%%%%%%%%%%%%%%%%%%%%%%%%%%%%%%%%%%%%%
%%%%%%%%%%%%%%%%%%%%%%%%%%%%%%%%%%%%%%%%%%%%%%%%%%%%%%%%%%%%%%%%%%
\section{Estimation of the proportion and bandwidth}
%%%%%%%%%%%%%%%%%%%%%%%%%%%%%%%%%%%%%%%%%%%%%%%%%%%%%%%%%%%%%%%%%%

%%%%%%%%%%%%%%%%%%%%%%%%%%%%%%%%%%%%%%%%%%%%%%%%%%%%%%%%%%%%%%%%%%
\subsection{Simultaneous estimation $a$ and $f$}
%%%%%%%%%%%%%%%%%%%%%%%%%%%%%%%%%%%%%%%%%%%%%%%%%%%%%%%%%%%%%%%%%%
The analogy with the E-M algorithm suggests to estimate $a$ using a
modified version of Algorithm \ref{Algo:Estim}, where
$\widehat{a}$ is updated at each step:
$$
\widehat{a}^{(l)} = \frac1n \sum_i \widehat{\tau}_i^{(l)}.
$$
However, it can be easily seen that the solution $\widehat{a} = 1$ and $\widehat{\tau}_i = 1$, for all $i$, is a fixed point of this modified algorithm. This
solution corresponds to the standard kernel estimate of $g$ (not of $f$). This property can be interpreted as an over-fitting trend: $\widehat{a}=1$ will
always provide a better fit than any other value since $\phi$ itself can be estimate using a kernel estimate.

%%%%%%%%%%%%%%%%%%%%%%%%%%%%%%%%%%%%%%%%%%%%%%%%%%%%%%%%%%%%%%%%%%
\subsection{Estimation of $a$} \label{Subsec:EstimA}
%%%%%%%%%%%%%%%%%%%%%%%%%%%%%%%%%%%%%%%%%%%%%%%%%%%%%%%%%%%%%%%%%%
The estimation of $a$ is a difficult task that can not be achieved by the algorithm proposed in the preceding section. In the case were the support of the
distribution $f$ is compact (typically, $(-\infty, \lambda]$), two unbiased estimates of $a$ can be proposed. Both come from the observation that, for $x >
\lambda$, if $F(x)=1$, the mixture cdf becomes
$$
G(x) = a + (1-a)\Phi(x),
$$
where $G$ and $\Phi$ are the respective cdfs of $g$ and $\phi$. In the framework of FDR control, Storey et al. \cite{STS04} proposes
$$
\widehat{a} = \frac{\widehat{G}(\lambda)-\Phi(\lambda)}{1-\Phi(\lambda)}
$$
where $\widehat{G}$ is the empirical cdf of $X$. The authors discuss the performances of these estimates and its sensitivity to the choice of $\lambda$.
Following the same principle, $a$ can be estimated using a linear least square fit of $\widehat{G}(X_i)$ to $\Phi(X_i)$, that is
$$
\widehat{a} = \arg\min_a \sum_{i:X_i > \lambda} (\widehat{G}(X_i) - b-(1-a)\Phi(X_i))^2.
$$
where $b$ is a constant. Provided $\lambda$ exists and is known, both estimates are unbiased. However, they both rely on the existence of some additional
information about the relative positions of distributions $f$ and $\phi$.

%%%%%%%%%%%%%%%%%%%%%%%%%%%%%%%%%%%%%%%%%%%%%%%%%%%%%%%%%%%%%%%%%%
\subsection{Estimation of $h$}
%%%%%%%%%%%%%%%%%%%%%%%%%%%%%%%%%%%%%%%%%%%%%%%%%%%%%%%%%%%%%%%%%%
To estimate the bandwidth $h$, we propose to use the standard approach (Scott, \ref{Sco92}) based on $V$-fold cross-validation. We split randomly the data set
$\{x_i\}_{i=1..n}$ into $V$ non-overlapping subsets $\Ycal_1, \dots, \Ycal_V$, each of size $n/V$: $\cup_v \Ycal_v = \{x_i\}_{i=1..n}$.  For each $v=1\ldots
V$, we define $\Xcal_v = \cup_{u
  \neq v} \Ycal_u$ as the training set, and $\Ycal_v$ as the test set.
We denote $\Lcal(\Ycal_v; h)$ the log-likelihood of the subset
$\Ycal_v$:
$$
\Lcal(\Ycal_v; h) = \sum_{x_j \in \Ycal_v} \ln \widehat{g}_v(x_j; h)
$$
where $\widehat{g}_v$ is estimated with Algorithm \ref{Algo:Estim} on the training set $\Xcal_v$ with the given window width $h$. We define the $V$-fold cross
validation log-likelihood as
$$
\Lcal_{CV}(h) = \frac1V \sum_v \Lcal(\Ycal_v; h).
$$
$n^{-1} \left(\sum_{i=1}^{n} \ln g(x_i) - V \Lcal_{CV}(h)\right)$ is an estimate of the Kullback-Leibler divergence between $\widehat{g}$ and $g$. This
estimated divergence between $\widehat{g}$ and $g$ is minimized when the cross-validation likelihood is maximized, that is for
$$
\widehat{h} = \arg \max_{h} \Lcal_{CV}((h).
$$
This optimization can be performed numerically.

 %In practice we
%observe the $\Lcal_{CV}((h)$ is more sensitive to the variations of
%$a$ than to the variations of $h$ (see Section
%\ref{Sec:Applications}).

%%%%%%%%%%%%%%%%%%%%%%%%%%%%%%%%%%%%%%%%%%%%%%%%%%%%%%%%%%%%%%%%%%
%%%%%%%%%%%%%%%%%%%%%%%%%%%%%%%%%%%%%%%%%%%%%%%%%%%%%%%%%%%%%%%%%%
\section{False positive and negative rates}
%%%%%%%%%%%%%%%%%%%%%%%%%%%%%%%%%%%%%%%%%%%%%%%%%%%%%%%%%%%%%%%%%%

%%%%%%%%%%%%%%%%%%%%%%%%%%%%%%%%%%%%%%%%%%%%%%%%%%%%%%%%%%%%%%%%%%
\subsection{Presentation and definitions}
%%%%%%%%%%%%%%%%%%%%%%%%%%%%%%%%%%%%%%%%%%%%%%%%%%%%%%%%%%%%%%%%%%
Multiple testing is a classical problem for many high-dimensional data sets. The breakthrough of technology for image analysis or genomic has given a new
interest for this question.
% A classical application is microarrays. This technology is part of a
% new class of biotechnologies that allow the monitoring of the
% expression level of thousands of genes simultaneously. Among the
% applications of microarrays, an important task is the identification
% of differentially expressed genes i.e., genes whose expression is
% associated with the status of patients (treatment/control for
% example). The biological question of identifying of differentially
% expressed genes can be restated as a two-sample hypothesis testing
% procedure: is the gene differentially expressed between the two
% situations. However, when thousands of genes in a microarray data set
% are evaluated simultaneously by fold changes and significance tests
% approach, multiple testing problems immediately arise and lead to many
% false positive genes. In this "one-by-one gene" the probability of
% detecting false positives rises sharply.
% Another classical application is neuroimaging data set. At each vowel,
% a test statistic is computed related to the null hypothesis of no
% difference between two images. The voxel for which the test exceed the
% threshold are then classify as different between the two images under
% study. The main question is to choose the threshold. When using the
% nominal level of the test, the probability of type 1 error becomes
% high.
% There is other application as genetics or astronomy.
A central problem in multiple testing problems is the control of type
I (i.e. false positive) and type II (i.e. false negative) errors. For
a given threshold $t$, we denote
$$
\begin{tabular}{rcll}
  $P(t)$ & $=$ & $\#\{j: X_j < t\}$ & the number of positives; \\
  $FP(t)$ & $=$ & $\#\{j: (X_j < t) \cap (Z_j=0)\}$ & the number of false
  positives; \\
  $N(t)$ & $=$ & $\#\{j: X_j \geq t\}$ & the number of negatives; \\
  $FN(t)$ & $=$ & $\#\{j: (X_j \leq t) \cap (Z_j=1)\}$ & the number of false
  positives.
\end{tabular}
$$

% However, as argued by Benjamini and Hochberg \cite{BeH95}, controlling
% the FWER in multiple testing settings may not always be appropriate.
% As an alternative and less stringent concept of error control they
% introduced the false discovery rate (FDR). The main interest of
% controlling the FDR is that it is an appealing quantity that leads to
% more powerful procedures than those relying on the FWER. The False
% Discovery Rate (FDR) is defined as the expected proportion of false
% positives among the total number of positives:

% \paragraph{Control of the type I error.}
The most popular criterion regarding type I errors is
% are the
% Family-Wise Error Rate ($FWER$):
% $$
% FWER(t) = \Pr\{ FP(t) \leq 1\}
% $$
% and
the False Discovery Rate ($FDR$: Benjamini and Hochberg  \cite{BeH95}):
%\begin{equation} \label{Eq:FDR}
$$
FDR(t) = \Esp\left[ FP(t) / \max\{P(t), 1\}\right].
$$
%\end{equation}
Broadly speaking, the FDR corresponds to the proportion of rejections
that are incorrect.{}
%\paragraph{Control of the type II error.}
It is worth noting that a dual quantity of the FDR is the FNR (false
non-discovery rate) as defined by Genovese and Wasserman~\cite{GeW02}:
% To take into account the sensitivity of a procedure we also have to care about the False Negative Rate ($FNR$):
$$
FNR(t) = \Esp\left( FN(t) / \max(N(t), 1) \right).
$$
In the mixture model framework $FDR$ and $FNR$ play symmetric
roles.

% In their seminal paper, Benjamini and Hochberg \cite{BeH95} discussed
% another criterion, later called the positive False Discovery Rate
% ($pFDR$) by Storey \cite{storey04} and defined as $pFDR=E\left(
%   \frac{V}{R} \;|\; R>0\right) $. However, Benjamini and Hochberg
% \cite{BeH95} did not consider this criterion because it cannot be
% controlled since under the complete null hypothesis (all null
% hypotheses tested are true), all significant results (if there are
% significant ones) are necessary false discoveries. So, it is
% impossible to insure that $pFDR<\alpha $ if $\alpha < 1$.

% The FDR seems particularly well-suited for exploratory analysis. In
% practice, the FDR leads to more powerful procedures than those relying
% on the FWER and gives an idea of the proportion of false positive
% hypothesis that a practitioner can expect if the experiment is done an
% infinite number of time. However, being an expectation, it gives very
% little information about the proportion of false discovery hypothesis
% in a given experiment. It also gives little information about

% Storey {\em et al.}, 2003 have stressed the importance of assessing to each feature its own measure of significance. They proposed to use the $q$-value,
% \[\frac{\hat{m_0}P_i}{R_i},\]
% where $P_i$ is the $p$-value of the ordered gene $i$ and $R_i$ is
% the total number of rejected genes with the threshold $t=P_i$. \\
% The $q$-value is appealing because it gives a measure of significance
% that can be attached to each gene, but it must be stressed that it is
% not an estimate of the probability for the gene to be a false
% positive.  The $q$-value is generally lower than the latter because it
% is computed using all the genes that are more significant than gene
% $i$.  Obviously a gene whose $p$-value is near to the threshold $t$
% does not have the same probability to be differentially expressed than
% a gene whose $p$-value is close to zero. Therefore the $q$-value gives
% a too optimistic view of the probability for the gene to be a false
% positive.

% \paragraph{Local FDR.}
More recently, it has been pointed out that, in many multiple testing framework, we need an information at the individual level about the probability for a
given observation to be a false positive (Aubert et al. \cite{ABD04}). This motivated the work of Storey and Tibshirani \cite{StT03} regarding the $q$-value,
which is actually not completely individual. In a mixture framework, a natural way to define a 'local FDR' ($\lFDR$: Efron et al. \cite{ETS01}) is to consider
the posterior probability
$$
\lFDR(x) = \Pr\{Z_i = 0 \;|\; X_i = x\}.
$$

%%%%%%%%%%%%%%%%%%%%%%%%%%%%%%%%%%%%%%%%%%%%%%%%%%%%%%%%%%%%%%%%%%
\subsection{Estimation}
%%%%%%%%%%%%%%%%%%%%%%%%%%%%%%%%%%%%%%%%%%%%%%%%%%%%%%%%%%%%%%%%%%

\paragraph{False positive and negative rates.}
In many practical situations, given the ordered observations $x_1 \leq \dots \leq x_i \leq \dots \leq x_n$, the question is to decide where to stop in that
list. The problem is therefore to estimate $FDR$ and $FNR$ when the first $i$ observations are declared positive. Following Genovese and Wasserman
\cite{GeW02,GeW04}, we define conditional versions of all the variables defined at the beginning of this section:
$$
P_i = i, \quad
FP_i = \#\{j \leq i: Z_j = 0\}, \quad
N_i = n-i, \quad
FN_i = \#\{j > i: Z_j = 1\}
$$
All these quantities are defined conditionally to the observed
values $(x_1, \dots, x_n)$ and correspond to any threshold $t$ such as
$x_i < t \leq x_{i+1}$.  The conditional rates follow:
\begin{equation} \label{Eq:FDRiFNRi}
  FDR_i = \Esp(FP_i) / i,
  \qquad
  FNR_i = \Esp(FN_i) / (n-i).
\end{equation}
Both expectations and variances of $FP$ and $FN$ can be calculated in
the mixture model framework:
$$
\Esp(FP_i) = \sum_{j=1}^i \Pr\{Z_j = 0 \;|\; x_j\} = \sum_{j=1}^i (1-\tau_j), \qquad \Esp(FN_i) = \sum_{j=i+1}^n \tau_j,
$$
$$
\Var(FP_i) = \sum_{j=1}^i \Pr\{Z_j = 0 \;|\; x_j\} = \sum_{j=1}^i (1-\tau_j) \tau_j, \qquad \Var(FN_i) = \sum_{j=i+1}^n \tau_j (1- \tau_j).
$$
Finally, we get two natural estimates:
$$
\widehat{FDR}_i = \frac1i \sum_{j=1}^i (1-\widehat{\tau}_j),
\qquad
\widehat{FNR}_i = \frac1{n-i} \sum_{j=i+1}^n \widehat{\tau}_j
$$
which are unbiased if the posterior probability
estimates are unbiased. Since the estimate of $\tau_j$ is proportional
to the estimate of $a$, underestimating $a$ leads to overestimate
$FDR$ and underestimate $FNR$.

\paragraph{Local $FDR$.} According to the definition given above, the
local $FDR$ for observation $i$ is
$$
\lFDR(x_i) = 1 - \tau_i
$$
so its natural estimate is $\widehat{\lFDR}_i = 1 - \widehat{\tau}_i$. Remark that the estimates $\widehat{\lFDR}$ and $\widehat{FDR}$ are consistent with the
definition of $\lFDR$ in terms of derivative of $FDR$ proposed by Aubert et al. \cite{ABD04}.

% In the framework of mixtu\widehat{\tau}re model (\ref{Eq:MixtureModel}), $DFR(t)$
% can be restated as
% \begin{equation} \label{Eq:FDRMixture}
%   \begin{array}{rcl}
%     FDR(t) & = & \displaystyle{\Pr\left\{Z = 0 \;|\; X \leq t\right\}
%       = \frac{\Pr\{(Z=0) \cap (X \leq t)\}}{\Pr\{X \leq t\}}} \\
%     & = & \displaystyle{\frac{\int_{-\infty}^{t} (1-a) \phi(x)
%         \mbox{d}x}{G(t)} = \frac{(1-a) \Phi(t)}{G(t)}}.
%     \end{array}
% \end{equation}

% Two estimates of $FDR(t)$ can be proposed.
% \begin{description}
% \item[Plug-in:] Denoting $K(x)$ the cdf associated to the kernel
%   function $k(x)$ and $K_i(x) = K[(x-x_i)/h]$, the cdfs $F(x)$ and
%   $G(x)$ can be estimated\widehat{\tau} by
%   $$
%   \widehat{F}(x) = \sum_i \widehat{\tau}_i K_i(x) \left/ \sum_i
%     \widehat{\tau}_i \right.,
%   \qquad
%   \widehat{G}(x) = \widehat{a} \widehat{F}(x) + (1-\widehat{a})
%   \Phi(x).
%   $$
%   A simple plug-in of these estimates in (\ref{Eq:FDRMixture}) provides
%   $$
%   \widehat{FDR}_1(t) = (1-\widehat{a}) \Phi(t) / \widehat{G}(t).
%   $$
% \item[Empirical:] A simpl\widehat{\tau}er estimate is based on the empirical version
%   of the expectation of definition (\ref{Eq:FDR}):
%   $$
%   \widehat{FDR}_2(t) = \frac{\sum_{i: x_i \leq t}
%     \widehat{\tau}_i}{\# \{i: x_i \leq t\} }.
%   $$
% \end{description}

% %%%%%%%%%%%%%%%%%%%%%%%%%%%%%%%%%%%%%%%%%%%%%%%%%%%%%%%%%%%%%%%%%%
% \subsection{Local FDR}
% %%%%%%%%%%%%%%%%%%%%%%%%%%%%%%%%%%%%%%%%%%%%%%%%%%%%%%%%%%%%%%%%%%

% With the notations of the present paper, it equals $1 - \tau_i$.

% The main interest of the mixture model (\ref{Eq:MixtureModel}) lies in
% the estimation of the local FDR ($\lFDR$). Indeed, in this
% framework, the local FDR associate to a given value $x$ can be
% precisely defined as the posterior probability
% $$
% \lFDR(x) = \Pr\{Z = 0 \;|\; X=x\} = \frac{a f(x)}{g(x)}.
% $$
% The $\{\widehat{\tau}_i\}$ provided by Algorithm \ref{Algo:Estim}
% are therefore the estimates of the local FDR associated to all the
% observed values $\{x_i\}$\widehat{\tau}:
% $$
% \widehat{\lFDR}(x_i) = \widehat{\tau}_i.
% $$

%%%%%%%%%%%%%%%%%%%%%%%%%%%%%%%%%%%%%%%%%%%%%%%%%%%%%%%%%%%%%%%%%%
%%%%%%%%%%%%%%%%%%%%%%%%%%%%%%%%%%%%%%%%%%%%%%%%%%%%%%%%%%%%%%%%%%
\section{Application: Hedenfalk data} \label{Sec:Applications}
%%%%%%%%%%%%%%%%%%%%%%%%%%%%%%%%%%%%%%%%%%%%%%%%%%%%%%%%%%%%%%%%%%

Hedenfalk et al. \cite{HDC01} compare the gene expression levels measured on patients with two different breast cancer. The dataset consist in 7 BRCA1 patients
and 8 BRCA2 patients; the total number of genes is $n=3226$.

\paragraph{Differential analysis.} We used t-test to detect
differentially expressed genes. Because of the small number of replicates, the estimate of the within group variability appears to very quite poor. This bad
estimation is known to have strong consequences on the conclusion. To circumvent this problem, we used test statistics and $p$ values $P_i$ computed under the
two following hypothesis regarding the variance.
\begin{description}
\item[{\sl (i) Homogenous variance.}] The variance of all the genes are
  all equal to a same variance~$\sigma^2$.
\item[{\sl (ii) Mixture model.}] Genes are spread into $K$ groups of
  variance, the variance and proportion of which can estimated using  a
  mixture model  (Delmar et al. \cite{DRL05}).
\end{description}
Other variance modeling have been proposed: see Rudemo et al. \cite{RLM02} for a general discussion.

\paragraph{Semi-parametric modeling.} In this situation, the
$p$-values are expected to have a mixture distribution $$p_i\sim a F + (1-a) \Ucal_{[0; 1]}$$. This mixture is hard to identify because of a high number of
$p$-values very close to 0. Therefore, we used the probit transform suggested by \cite{Efr03}, and considered the mixture model on the transformed $p$-values:
$$
X_i = \Phi^{-1}(p_i) \sim a F + (1-a) \Phi
$$
were $\Phi$ is the cdf of the standard Gaussian distribution.

\paragraph{Results.} Figure \ref{Fig:MixtureX} presents the fit of the
semi-parametric mixture model to the histogram of the transformed
$p$-values. We see that in both cases, the distribution $f$ and $\phi$
strongly overlap. In both case, we used the least-square estimate of
$a$ presented in Section \ref{Subsec:EstimA}. It resulted in
$\widehat{a} = 20.6 \%$ in the homogenous variances case ,and in
$\widehat{a} = 30.5 \%$ in the mixture variance case.

\begin{figure}
  \begin{center}
    \begin{tabular}{cc}
    \epsfig{
      file=/RECHERCHE/EXPRESSION/EXEMPLES/HEDENFALK/Ainit/Asup-1.0/Homogen-Gaus.eps, 
%      file=Homogen-Gaus.eps, 
      height=7cm, width=4cm, bbllx=66, bblly=510, bburx=283,
      bbury=691, clip=, angle=90}
    &
    \epsfig{
      file=/RECHERCHE/EXPRESSION/EXEMPLES/HEDENFALK/Ainit/Asup-1.0/Varmixt-Gaus.eps,
%      file=Varmixt-Gaus.eps,
      height=7cm, width=4cm, bbllx=66, bblly=510, bburx=283,
      bbury=691, clip=, angle=90}
    \end{tabular}
    \caption{Fit of the semi-parametric mixture model to the
      transformed $p$-values for homogenous (left) and mixture (right)
      gene variances. {\bf --}: histogram, \textcolor{blue}{\bf --}:
      mixture density, \textcolor{red}{\bf -~-}: $\widehat{a}
      \widehat{f}$, \textcolor{green}{\bf -~-}: $(1-\widehat{a})
    \phi$.}
    \label{Fig:MixtureX}
  \end{center}
\end{figure}

Figures \ref{Fig:TauX} and \ref{Fig:TauP} present the estimated posterior probabilities $\widehat{\tau}_i$ and $\widehat{FDR}_i$ as a function of $X_i$ and
$P_i$ respectively. In the homogeneous variance case, we see that the posterior probabilities first decrease (as expected), and then re-increase on the right
part of the plot, which is unexpected. The explanation is that the non-parametric part of the mixture model actually capture a lack of fit of the true
distribution of the test statistic to the theoretical distribution under the null hypothesis.  In this situation, estimation procedures of the proportion $a$
based on this theoretical distribution under $H_0$ are probably biased.  This phenomena is strongly reduced by the mixture model for the variances.

\begin{figure}
  \begin{center}
    \begin{tabular}{cc}
    \epsfig{
      file=/RECHERCHE/EXPRESSION/EXEMPLES/HEDENFALK/Ainit/Asup-1.0/Homogen-Gaus.eps, 
%      file=Homogen-Gaus.eps,
      height=7cm, width=4cm, bbllx=66, bblly=295, bburx=283,
      bbury=480, clip=, angle=90}
    &
    \epsfig{
      file=/RECHERCHE/EXPRESSION/EXEMPLES/HEDENFALK/Ainit/Asup-1.0/Varmixt-Gaus.eps,
%      file=Varmixt-Gaus.eps,
      height=7cm, width=4cm, bbllx=66, bblly=295, bburx=283,
      bbury=480, clip=, angle=90}
    \end{tabular}
    \caption{Posterior probabilities $\widehat{\tau}$ (\textcolor{blue}{\bf --}) and false
      discovery rate $\widehat{FDR}_i$ (\textcolor{red}{\bf -~-}) as a
      function of the transformed $p$-values $X_i$. Left: homogenous gene
      variances, right: mixture gene variances.}
    \label{Fig:TauX}
  \end{center}
\end{figure}

\begin{figure}
  \begin{center}
    \begin{tabular}{cc}
      \epsfig{
      file=/RECHERCHE/EXPRESSION/EXEMPLES/HEDENFALK/Ainit/Asup-1.0/Homogen-Gaus.eps,
%        file=Homogen-Gaus.eps,
        height=7cm, width=4cm, bbllx=66, bblly=85, bburx=283,
        bbury=265, clip=, angle=90}
      &
      \epsfig{
      file=/RECHERCHE/EXPRESSION/EXEMPLES/HEDENFALK/Ainit/Asup-1.0/Varmixt-Gaus.eps,
%        file=Varmixt-Gaus.eps,
        height=7cm, width=4cm, bbllx=66, bblly=85, bburx=283,
        bbury=265, clip=, angle=90}
    \end{tabular}
    \caption{Posterior probabilities $\widehat{\tau}$ (\textcolor{blue}{\bf --}) and false
      discovery rate $\widehat{FDR}_i$ (\textcolor{red}{\bf -~-}) as a
      function of the $p$-values $P_i$. Left: homogenous gene
      variances, right: mixture gene variances.}
    \label{Fig:TauP}
  \end{center}
\end{figure}

Table \ref{Tab:FDR} gives the number of positive genes for some pre-specified values of the FDR. We see that, for small FDR, the minimal posterior probability
is still high, which means that all the positive genes can be trusted. We also see that FNR slowly decreases. The estimated FDR and FNR are equal ($19.7 \%$)
for $i = 633$ positive genes: the corresponding $p$-value is $P_{[i]} = 5.4 \%$, the posterior probability is $\widehat{\tau}_{(i)} = 43.5 \%$. This means
that, at this point, some of the positive genes are really questionable.

\begin{table}
  \begin{center}
    $$
    \begin{array}{ccccc}
      \quad \widehat{FDR}_{(i)} \quad & \qquad i \qquad & \quad P_{(i)}
      \quad & \quad \widehat{\tau}_{(i)} \quad & \quad
      \widehat{FNR}_{(i)} \quad \\
      \hline
      1\% & 4 & 2.5\;10^{-5} & 0.988 & 31.5 \% \\
      5\% & 142 & 3.1\;10^{-3} & 0.914 & 28.7 \% \\
      10\% & 296 & 1.3\;10^{-2} & 0.798 & 25.7 \% \\
    \end{array}
    $$
    \caption{}
    \label{Tab:FDR}
  \end{center}
\end{table}


%%%%%%%%%%%%%%%%%%%%%%%%%%%%%%%%%%%%%%%%%%%%%%%%%%%%%%%%%%%%%%%%%%
%%%%%%%%%%%%%%%%%%%%%%%%%%%%%%%%%%%%%%%%%%%%%%%%%%%%%%%%%%%%%%%%%%
\bibliographystyle{/Latex/astats}
%\bibliographystyle{astats}
\bibliography{/Biblio/AST}
% \begin{itemize}
% \item Olkin, I., and Spiegelman, C. (1987), A Semiparametric Approach
%   to Density Estimation, Journal of the American Statistical
%   Association, 82, 858-865.
% \item Hjort, N.L. and Glad, I.K., 'Nonparametric density estimation
%   with a parametric start', Annals of Statistics, Vol.23, No.3 1995.
%   882-904
% \item F. Hoti and L. Holmstr\"{o}m. A semiparametric density
%   estimation approach to pattern classification. Pattern Recognition,
%   37:409-419, 2004.
% \item C.E. Priebe and D.J. Marchette, "Alternating Kernel and Mixture
%   Density Estimates," Computational Statistics and Data Analysis, Vol.
%   35, No. 1, pp. 43--65, 2000.
% \item Marco Di Marzio and Charles C. Taylor Boosting kernel density
%   estimates: A bias reduction technique?  Biometrika 2004 91: 226-233;
% \item B. Efron and R. Tibshirani Using specially designed exponential
%   families for density estimation Ann. Statist.  24 (1996), no. 6,
%   2431--2461
% \item Acuna, E. and Rojas, A. (2001) Bagging classifiers based on
%   kernel density estimators. Proceedings of the International
%   Conference on New Trends in Computational Statistics with Biomedical
%   Applications. Osaka University, Japan. 343-350.
% \item{allison} Allison, D.B.; Gadbury, G.; Heo, M; Fernandez, J., Lee,
%   C.-K.; Prolla, T.A.; Weindruch, R. A Mixture Model Approach For The
%   Analysis Of Microarray Gene Expression Data. Computational
%   Statistics \& Data Analysis {\bf 2002}, 39, 1-20.
% \item{broet} Bro\"{e}t, P.; Lewin, A.; Richardson, S.; Dalmasso, C.;
%   Magdelenat H.  a mixture model-based strategy for selecting sets of
%   genes in multiclass response microarray experiments. Bioinformatics
%   {\bf 2004}, {20}, 2562-2571
% \item{liao} Liao, J.G.; Lin, Y. ; Selvanayagam, Z.E.; Weichung J.S.  A
%   mixture model for estimating the local false discovery rate in DNA
%   microarray analysis. {Bioinformatics} {\bf 2004}, {20(16)},
%   2694-2701.
% \item{newton} Newton, M.A.; Noueiry, A.; Sarkar, D.; Ahlquist, P.
%   Detecting differential gene expression with a semiparametric
%   hierarchical mixture method. {Biostatistics} {\bf 2004}, {5},
%   155-176
% \item{pounds2} Pounds, S.; Morris, S.W.  Estimating the occurrence of
%   false positives and false negatives in microarray studies by
%   approximating and partitioning the empirical distribution of
%   $p$-values. Bioinformatics {\bf 2003}, {19}, 1236-42.
% \end{itemize}

%%%%%%%%%%%%%%%%%%%%%%%%%%%%%%%%%%%%%%%%%%%%%%%%%%%%%%%%%%%%%%%%%%
%%%%%%%%%%%%%%%%%%%%%%%%%%%%%%%%%%%%%%%%%%%%%%%%%%%%%%%%%%%%%%%%%%
\appendix
\section*{Appendix}
%%%%%%%%%%%%%%%%%%%%%%%%%%%%%%%%%%%%%%%%%%%%%%%%%%%%%%%%%%%%%%%%%%

\paragraph{Proof of Lemma {\bf \ref{Lem:Simplex}}.}  We have $\psibf = \alphabf \circ \betabf \circ \gammabf$. Since $\gammabf$
is a projection, we have $\psibf \circ \gammabf = \psibf$. So, for $\vbf^* = \gammabf(\ubf^*)$, we have $\psibf(\ubf^*) = \psibf(\vbf^*)$ and
$$
\ubf^* = \psibf(\ubf^*) \quad \Rightarrow \quad \vbf^* = \gammabf
\circ \psibf(\ubf^*) = \gammabf \circ \psibf \circ \gammabf (\ubf^*) =
\gammabf \circ \psibf(\vbf^*).
$$
Conversely, let $\vbf^*$ denote a fix point of $\gammabf \circ
\psibf$ and $\ubf^* = \alphabf \circ \betabf(\vbf^*)$. Since $\vbf^*$
belongs to $\Ecal$, we have $\vbf^* = \gammabf(\vbf^*)$ so $\ubf^* =
\psibf(\vbf^*)$ and $\psibf(\ubf^*) = \psibf \circ \psibf(\vbf^*)$.
Remarking that
$$
\psibf \circ \psibf(\vbf^*) = \psibf \circ \gammabf \circ
\psibf(\vbf^*) = \psibf(\vbf^*)
$$
we get $\psibf(\ubf^*) = \psibf(\vbf^*) = \ubf^*$.  $\blacksquare$

\paragraph{Proof of Lemma {\bf \ref{Lem:DefDistance}}.} $d$ is a distance
iff, for all $\ubf$, $\vbf$ and $\wbf$ in $\Fcal$, ($i$) $d(\ubf,
\vbf) \geq 0$; ($ii$) $d(\ubf, \vbf) = d(\vbf, \ubf)$; ($iii$)
$\{d(\ubf, \vbf) = 0\} \Leftrightarrow \{\ubf = \vbf\}$; ($iv$)
$d(\ubf, \wbf) \leq d(\ubf, \vbf) + d(\vbf, \wbf)$. ($i$), ($ii$) and
($iii$) are straightforward. ($iv$) is due to
$$
\max_i \left(\frac{u_i}{w_i}\right)
=
\max_i \left(\frac{u_i}{v_i}\frac{v_i}{w_i}\right)
\leq
\max_i \left(\frac{u_i}{v_i}\right) \max_i \left(\frac{v_i}{w_i}\right)
$$
and conversely for the $\min$. $\blacksquare$

\paragraph{Proof of Lemma {\bf \ref{Lem:Decrease}}.}  The second case
is obvious since $d$ is a distance. So we concentrate on the proof of the first one. The main idea is to prove that  $\betabf$ can not increase the distance
$d$ and that $\alphabf\circ\betabf$ necessarily reduces it.
\begin{description}
\item[$\betabf$:] For $\vbf \neq \wbf$ we define $c_1 = \min_i
  (w_i/v_i)$ and $c_2 = \max_i(w_i/v_i)$. Remark that $c_1 < 1 < c_2$,
  $d(\vbf, \wbf) = \ln(c_2/c_1)$ and
  \begin{equation} \label{Eq:Ineq_ab}
    \mbox{For all } i: c_1 v_i \leq w_i \leq c_2 v_i.
  \end{equation}
  Denote $v'_j = \beta_j(\vbf)$ and $w'_j = \beta_j(\wbf)$. Since all
  the $b_{ij}$ are positive, (\ref{Eq:Ineq_ab}) implies that $c_1 v'_j
  \leq w'_j \leq c_2 v'_j$ for all $j$, which means that $\betabf$
  does not increase $d$.
\item[$\alphabf\circ\betabf$:] Denote now $v''_j =
  \alpha_j[\betabf(\vbf)] = v'_j/(1+v'_j)$ and $w''_j =
  \alpha_j[\betabf(\wbf)] = w'_j/(1+w'_j)$.  Remarking that the
  transformation $t \rightarrow t/(1+t)$ is increasing, we derive that
  if $w'_i \geq v'_i$ then $w''_i \geq v''_i > c_1 v''_i$, and if $
  w'_i < v'_i$ then
  $$
  w''_i = \frac{w'_i}{1+w'_i} > \frac{w'_i}{1+v'_i} \geq c_1
  \frac{v'_i}{1+v'_i} = c_1 v''_i.
  $$
  So, in both cases, we have $w''_i > c_1 v''_i$ for every $i$.\\
  Conversely, if $w'_i \leq v'_i$ then $w''_i \leq v''_i < c_2 v''_i$,
  and if $w'_i > v'_i$ then
  $$
  w''_i = \frac{w'_i}{1+w'_i} < \frac{w'_i}{1+v'_i} \leq c_2
  \frac{v'_i}{1+v'_i} = c_2 v''_i.
  $$
  So, in both cases, we have $w''_i < c_2 v''_i$ for every $i$.\\
\end{description}
This shows that, for every $i$, $c_1 v''_i < w''_i < c_2 v''_i$. So,
denoting $c''_1 = \min_i (w''_i / v''_i)$ and $c''_2 = \max_i (w''_i /
v''_i)$ we have $c_1 < c''_1$ and $c_2 > c''_2$, which implies that
$$
d(\gammabf\circ\psibf(\wbf), \gammabf\circ\psibf(\vbf)) = \ln(c''_2/c''_1) < \ln(c_2/c_1) = d(\wbf, \vbf).
$$
We conclude that $\alphabf\circ\betabf$ strictly reduces $d$.
$\blacksquare$

%%%%%%%%%%%%%%%%%%%%%%%%%%%%%%%%%%%%%%%%%%%%%%%%%%%%%%%%%%%%%%%%%%
%%%%%%%%%%%%%%%%%%%%%%%%%%%%%%%%%%%%%%%%%%%%%%%%%%%%%%%%%%%%%%%%%%
\end{document}
%%%%%%%%%%%%%%%%%%%%%%%%%%%%%%%%%%%%%%%%%%%%%%%%%%%%%%%%%%%%%%%%%%
%%%%%%%%%%%%%%%%%%%%%%%%%%%%%%%%%%%%%%%%%%%%%%%%%%%%%%%%%%%%%%%%%%
\bibitem{storey04} Storey JD, Taylor JE, and Siegmund D. (2004) Strong control,
conservative point estimation, and simultaneous conservative consistency of false discovery rates: A unified approach. {\em Journal of the Royal Statistical
Society, Series B}, {\bf 66} 187-205.

Scott, D. W. (1992) Multivariate Density Estimation. Theory, Practice and Visualization. New York: Wiley.

Silverman, B. W. (1986) Density Estimation. London:Chapman and Hall.
