\documentclass{beamer}

% Beamer style
%\usetheme[secheader]{Madrid}
\usetheme{CambridgeUS}
\usecolortheme[rgb={0.65,0.15,0.25}]{structure}
%\usefonttheme[onlymath]{serif}
\beamertemplatenavigationsymbolsempty
%\AtBeginSubsection

% Packages
%\usepackage[french]{babel}
\usepackage[latin1]{inputenc}
\usepackage{color}
%\usepackage{dsfont, stmaryrd}
\usepackage{amsmath, amsfonts, amssymb}
%\usepackage{stmaryrd}
\usepackage{epsfig}
\usepackage{../../../../LATEX/astats}
%\usepackage[all]{xy}
\usepackage{graphicx}

% Commands
\definecolor{darkred}{rgb}{0.65,0.15,0.25}
\definecolor{darkgreen}{rgb}{0,0.4,0}
%\newcommand{\emphase}[1]{\textcolor{darkred}{#1}}
\newcommand{\emphase}[1]{{#1}}
\newcommand{\paragraph}[1]{\textcolor{darkred}{#1}}
\newcommand{\refer}[1]{\textcolor{gray}{\sl \cite{#1}}}
\newcommand{\Refer}[1]{\textcolor{gray}{\sl #1}}
\newcommand{\newblock}{}
\newcommand{\ra}{$\emphase{\rightarrow}$}

% Symbols


%====================================================================
\title[Joint CNV-LOH analysis]{Joint CNV-LOH analysis \\ using HMM with mixtures as emission distributions}

\author[S. Robin]{C. B�rard$^1$, S. Robin$^2$, S. Volant$^3$}

\institute[AgroParisTech / INRA]{
  ($^1$)univ. Rouen, \quad ($^2$) INRA / Agroparistech, \quad ($^3$) XL-stat \\
  \bigskip
 \begin{tabular}{ccccc}
    \includegraphics[width=.2\textwidth]{../Figures/LogoINRA-Couleur} & 
    \hspace{.02\textwidth} &
    \includegraphics[width=.3\textwidth]{../Figures/logagroptechsolo} & 
    \hspace{.02\textwidth} &
    \includegraphics[width=.2\textwidth]{../Figures/logo-ssb} \\ 
  \end{tabular} \\
  \bigskip
  }

  \date[SSB]{SSB, May 2013}

%====================================================================

%====================================================================
%====================================================================
\begin{document}
%====================================================================
%====================================================================

%====================================================================
\frame{\titlepage
  }

%====================================================================
%====================================================================
\section{CNV / LOH}
\frame{\frametitle{Copy number variation (CNV) / Loss of heterozygosity (LOH)}

  \paragraph{A hot topic:} In last SSB meetings
  \begin{itemize}
  \item 12/2/13, Cyril Dalmasso: Etiquetage des r\'egions issues d'une segmentation dans le cadre de la d\'etection d'alt\'erations g\'enomiques en oncologie clinique
  \item 16/4/13, Morgane Pierre-Jean: Comparaison des m\'ethodes de segmentation du nombre de copies d'ADN et de la fraction d'all\`ele B.
  \end{itemize}
  
  \bigskip\bigskip \pause
  \paragraph{Data at hand:} SNP array data
  \begin{itemize}
   \item $t = 1..n$ single nucleotide polymorphism (SNP) loci;
   \item $A_t =$ signal corresponding to allele A at loci $t$;
   \item $B_t =$ signal corresponding to allele B at loci $t$.
  \end{itemize}
}

%====================================================================
\newcommand{\figdir}{/home/robin/RECHERCHE/RUPTURES/LOH/Data/NA06991}
\newcommand{\figname}{NA06991-Chr6}

%====================================================================
\frame{\frametitle{An example (from \refer{SLV08})}

  \begin{tabular}{cc}
    \hspace{-0.5cm}
    \begin{tabular}{p{.45\textwidth}}
      The relative ratio 
      $$RR_t = (A_t + B_t) / 2$$
      reveals copy number variations.
    \end{tabular}
    &
    \hspace{-1cm}
    \begin{tabular}{p{.5\textwidth}}
      \includegraphics[width=.5\textwidth, height=.45\textheight]{\figdir/\figname-LRR}
    \end{tabular} \\ \pause
    \hspace{-0.5cm}
    \begin{tabular}{p{.5\textwidth}}
      \includegraphics[width=.5\textwidth, height=.45\textheight]{\figdir/\figname-BAF}
    \end{tabular}
    &
    \hspace{-1cm}
    \begin{tabular}{p{.5\textwidth}}
      The B allele frequency 
      $$BAF_t = B_t / (A_t + B_t)$$
      reveals loss of heterozygosity.
    \end{tabular} \\    
  \end{tabular}
}


%====================================================================
\renewcommand{\figdir}{/home/robin/RECHERCHE/RUPTURES/LOH/Simul}
\renewcommand{\figname}{SimulAB_gamma_s0.3}

%====================================================================
\frame{\frametitle{Data: 3 representations}

  \begin{tabular}{cc}
    \hspace{-0.5cm}
    \begin{tabular}{p{.45\textwidth}}
      \paragraph{Synthetic data.}
      \begin{itemize}
      \item \onslide+<2->{$(A_t + B_t)$ $\times$  position $t$ (CNV)}
      \item \onslide+<3->{$B_t/(A_t + B_t)$ $\times$  $t$ (LOH)}
      \item \onslide+<4->{$A_t$ $\times$ $B_t$ (genotype)}
      \end{itemize}
    \end{tabular}
    &
    \hspace{-1cm}
    \begin{tabular}{p{.5\textwidth}}
      \vspace{-.35cm}
      \onslide+<2->{
      \includegraphics[width=.5\textwidth, height=.45\textheight]{\figdir/\figname-LRR}
      }
    \end{tabular} \\
    \hspace{-0.5cm}
    \begin{tabular}{p{.5\textwidth}}
      \vspace{-.35cm}
      \onslide+<3->{
      \includegraphics[width=.5\textwidth, height=.45\textheight]{\figdir/\figname-BAF}
      }
    \end{tabular}
    &
    \hspace{-1cm}
    \begin{tabular}{p{.5\textwidth}}
      \vspace{-.35cm}
      \onslide+<4->{
      \includegraphics[width=.5\textwidth, height=.45\textheight]{\figdir/\figname-AB}
      }
    \end{tabular} \\    
  \end{tabular}
}

%====================================================================
\renewcommand{\figname}{SimulAB_gamma_s0.3-c2-3}

%====================================================================
\frame{\frametitle{A classification problem}

  \begin{tabular}{cc}
    \hspace{-0.5cm}
    \begin{tabular}{p{.45\textwidth}}
      \paragraph{Aim:} retrieve both CNV and LOH status. 
      
    \end{tabular}
    &
    \hspace{-1cm}
    \begin{tabular}{p{.5\textwidth}}
      \includegraphics[width=.5\textwidth, height=.45\textheight]{\figdir/\figname-LRR}
    \end{tabular} \\
    \hspace{-0.5cm}
    \begin{tabular}{p{.5\textwidth}}
      \includegraphics[width=.5\textwidth, height=.45\textheight]{\figdir/\figname-BAF}
    \end{tabular}
    &
    \hspace{-1cm}
    \begin{tabular}{p{.5\textwidth}}
      \includegraphics[width=.5\textwidth, height=.45\textheight]{\figdir/\figname-AB}
    \end{tabular} \\    
  \end{tabular}
}

%====================================================================
%====================================================================
\section{Model}
\subsection{CNV/LOH status}
\frame{\frametitle{CNV/LOH status}

  \paragraph{Principle.}
  \begin{itemize}
  \item CNV refers to the local number of copies.
  \item LOH refers to the (relative) abundance of each copies.
  \end{itemize}

  \bigskip \bigskip \pause
  \paragraph{Remark.} we are not interested by the genotype $(a, b)$ at each locus, denoted:
  $$
  \text{AA} = (1, 1), \quad
  \text{AAB} = (2, 1), \quad
  \text{ABBB} = (1, 3).  
  $$

  \bigskip \bigskip \pause
  \paragraph{State:} couple $s = (c, m)$
  \begin{itemize}
  \item $c =$ total copy number
  \item $m =$ number of copies of the minority chromosome
  \end{itemize}
  $$
  \mathcal S = \left\{(c, m): c_{\min} \leq c \leq c_{\max}, 
  \; 0 \leq m \leq \lfloor \frac{c}2 \rfloor \right\}
  $$

}

%====================================================================
\renewcommand{\figdir}{/home/robin/RECHERCHE/RUPTURES/LOH/Fig}

%===================================================================
\renewcommand{\figname}{FigLOH}
\newcommand{\cparm}{2} \newcommand{\mparm}{1}
\newcommand{\statename}{Normal state}
\newcommand{\statedesc}{Two different copies}
\newcommand{\stategeno}{AA, AB, BB}
\frame{\frametitle{State $c = \cparm$, $m = \mparm$}

  %\vspace{-0.5cm}
  \begin{tabular}{cc}
    \hspace{-0.5cm}
    \begin{tabular}{p{.5\textwidth}}
    \paragraph{\statename:} \\
    \statedesc
    
    \bigskip
    Possible genotypes: \\
    \centerline{\stategeno}
    \end{tabular}
    &
    \hspace{-1cm}
    \begin{tabular}{p{.5\textwidth}}
      \includegraphics[width=.5\textwidth, height=.45\textheight]{\figdir/\figname-c\cparm-m\mparm-LRR}
    \end{tabular} \\
    \hspace{-0.5cm}
    \begin{tabular}{p{.5\textwidth}}
      \includegraphics[width=.5\textwidth, height=.45\textheight]{\figdir/\figname-c\cparm-m\mparm-BAF}
    \end{tabular}
    &
    \hspace{-1cm}
    \begin{tabular}{p{.5\textwidth}}
      \includegraphics[width=.5\textwidth, height=.45\textheight]{\figdir/\figname-c\cparm-m\mparm-AB}
    \end{tabular} \\    
  \end{tabular}
}


\renewcommand{\cparm}{2} \renewcommand{\mparm}{0}
\renewcommand{\statename}{Regular LOH}
\renewcommand{\statedesc}{Loss of one copy}
\renewcommand{\stategeno}{AA, BB}
\frame{\frametitle{State $c = \cparm$, $m = \mparm$}

  %\vspace{-0.5cm}
  \begin{tabular}{cc}
    \hspace{-0.5cm}
    \begin{tabular}{p{.5\textwidth}}
    \paragraph{\statename:} \\
    \statedesc
    
    \bigskip
    Possible genotypes: \\
    \centerline{\stategeno}
    \end{tabular}
    &
    \hspace{-1cm}
    \begin{tabular}{p{.5\textwidth}}
      \includegraphics[width=.5\textwidth, height=.45\textheight]{\figdir/\figname-c\cparm-m\mparm-LRR}
    \end{tabular} \\
    \hspace{-0.5cm}
    \begin{tabular}{p{.5\textwidth}}
      \includegraphics[width=.5\textwidth, height=.45\textheight]{\figdir/\figname-c\cparm-m\mparm-BAF}
    \end{tabular}
    &
    \hspace{-1cm}
    \begin{tabular}{p{.5\textwidth}}
      \includegraphics[width=.5\textwidth, height=.45\textheight]{\figdir/\figname-c\cparm-m\mparm-AB}
    \end{tabular} \\    
  \end{tabular}
}


\renewcommand{\cparm}{0} \renewcommand{\mparm}{0}
\renewcommand{\statename}{Homozygous loss}
\renewcommand{\statedesc}{Loss of the two copies}
\renewcommand{\stategeno}{-}
\frame{\frametitle{State $c = \cparm$, $m = \mparm$}

  %\vspace{-0.5cm}
  \begin{tabular}{cc}
    \hspace{-0.5cm}
    \begin{tabular}{p{.5\textwidth}}
    \paragraph{\statename:} \\
    \statedesc
    
    \bigskip
    Possible genotypes: \\
    \centerline{\stategeno}
    \end{tabular}
    &
    \hspace{-1cm}
    \begin{tabular}{p{.5\textwidth}}
      \includegraphics[width=.5\textwidth, height=.45\textheight]{\figdir/\figname-c\cparm-m\mparm-LRR}
    \end{tabular} \\
    \hspace{-0.5cm}
    \begin{tabular}{p{.5\textwidth}}
      \includegraphics[width=.5\textwidth, height=.45\textheight]{\figdir/\figname-c\cparm-m\mparm-BAF}
    \end{tabular}
    &
    \hspace{-1cm}
    \begin{tabular}{p{.5\textwidth}}
      \includegraphics[width=.5\textwidth, height=.45\textheight]{\figdir/\figname-c\cparm-m\mparm-AB}
    \end{tabular} \\    
  \end{tabular}
}


\renewcommand{\cparm}{1} \renewcommand{\mparm}{0}
\renewcommand{\statename}{Heterozygous loss}
\renewcommand{\statedesc}{Loss of one copy}
\renewcommand{\stategeno}{A, B}
\frame{\frametitle{State $c = \cparm$, $m = \mparm$}

  %\vspace{-0.5cm}
  \begin{tabular}{cc}
    \hspace{-0.5cm}
    \begin{tabular}{p{.5\textwidth}}
    \paragraph{\statename:} \\
    \statedesc
    
    \bigskip
    Possible genotypes: \\
    \centerline{\stategeno}
    \end{tabular}
    &
    \hspace{-1cm}
    \begin{tabular}{p{.5\textwidth}}
      \includegraphics[width=.5\textwidth, height=.45\textheight]{\figdir/\figname-c\cparm-m\mparm-LRR}
    \end{tabular} \\
    \hspace{-0.5cm}
    \begin{tabular}{p{.5\textwidth}}
      \includegraphics[width=.5\textwidth, height=.45\textheight]{\figdir/\figname-c\cparm-m\mparm-BAF}
    \end{tabular}
    &
    \hspace{-1cm}
    \begin{tabular}{p{.5\textwidth}}
      \includegraphics[width=.5\textwidth, height=.45\textheight]{\figdir/\figname-c\cparm-m\mparm-AB}
    \end{tabular} \\    
  \end{tabular}
}


\renewcommand{\cparm}{3} \renewcommand{\mparm}{0}
\renewcommand{\statename}{Three identical copies}
\renewcommand{\statedesc}{Loss of one copy + triplication}
\renewcommand{\stategeno}{AAA, BBB}
\frame{\frametitle{State $c = \cparm$, $m = \mparm$}

  %\vspace{-0.5cm}
  \begin{tabular}{cc}
    \hspace{-0.5cm}
    \begin{tabular}{p{.5\textwidth}}
    \paragraph{\statename:} \\
    \statedesc
    
    \bigskip
    Possible genotypes: \\
    \centerline{\stategeno}
    \end{tabular}
    &
    \hspace{-1cm}
    \begin{tabular}{p{.5\textwidth}}
      \includegraphics[width=.5\textwidth, height=.45\textheight]{\figdir/\figname-c\cparm-m\mparm-LRR}
    \end{tabular} \\
    \hspace{-0.5cm}
    \begin{tabular}{p{.5\textwidth}}
      \includegraphics[width=.5\textwidth, height=.45\textheight]{\figdir/\figname-c\cparm-m\mparm-BAF}
    \end{tabular}
    &
    \hspace{-1cm}
    \begin{tabular}{p{.5\textwidth}}
      \includegraphics[width=.5\textwidth, height=.45\textheight]{\figdir/\figname-c\cparm-m\mparm-AB}
    \end{tabular} \\    
  \end{tabular}
}


\renewcommand{\cparm}{3} \renewcommand{\mparm}{1}
\renewcommand{\statename}{Duplication}
\renewcommand{\statedesc}{One copy amplified}
\renewcommand{\stategeno}{AAA, AAB, ABB, BBB}
\frame{\frametitle{State $c = \cparm$, $m = \mparm$}

  %\vspace{-0.5cm}
  \begin{tabular}{cc}
    \hspace{-0.5cm}
    \begin{tabular}{p{.5\textwidth}}
    \paragraph{\statename:} \\
    \statedesc
    
    \bigskip
    Possible genotypes: \\
    \centerline{\stategeno}
    \end{tabular}
    &
    \hspace{-1cm}
    \begin{tabular}{p{.5\textwidth}}
      \includegraphics[width=.5\textwidth, height=.45\textheight]{\figdir/\figname-c\cparm-m\mparm-LRR}
    \end{tabular} \\
    \hspace{-0.5cm}
    \begin{tabular}{p{.5\textwidth}}
      \includegraphics[width=.5\textwidth, height=.45\textheight]{\figdir/\figname-c\cparm-m\mparm-BAF}
    \end{tabular}
    &
    \hspace{-1cm}
    \begin{tabular}{p{.5\textwidth}}
      \includegraphics[width=.5\textwidth, height=.45\textheight]{\figdir/\figname-c\cparm-m\mparm-AB}
    \end{tabular} \\    
  \end{tabular}
}


\renewcommand{\cparm}{4} \renewcommand{\mparm}{0}
\renewcommand{\statename}{Four identical copies}
\renewcommand{\statedesc}{Loss of one copy + quadruplication}
\renewcommand{\stategeno}{AAAA, BBBB}
\frame{\frametitle{State $c = \cparm$, $m = \mparm$}

  %\vspace{-0.5cm}
  \begin{tabular}{cc}
    \hspace{-0.5cm}
    \begin{tabular}{p{.5\textwidth}}
    \paragraph{\statename:} \\
    \statedesc
    
    \bigskip
    Possible genotypes: \\
    \centerline{\stategeno}
    \end{tabular}
    &
    \hspace{-1cm}
    \begin{tabular}{p{.5\textwidth}}
      \includegraphics[width=.5\textwidth, height=.45\textheight]{\figdir/\figname-c\cparm-m\mparm-LRR}
    \end{tabular} \\
    \hspace{-0.5cm}
    \begin{tabular}{p{.5\textwidth}}
      \includegraphics[width=.5\textwidth, height=.45\textheight]{\figdir/\figname-c\cparm-m\mparm-BAF}
    \end{tabular}
    &
    \hspace{-1cm}
    \begin{tabular}{p{.5\textwidth}}
      \includegraphics[width=.5\textwidth, height=.45\textheight]{\figdir/\figname-c\cparm-m\mparm-AB}
    \end{tabular} \\    
  \end{tabular}
}


\renewcommand{\cparm}{4} \renewcommand{\mparm}{1}
\renewcommand{\statename}{Triplication}
\renewcommand{\statedesc}{One copy amplified twice}
\renewcommand{\stategeno}{AAAA, AAAB, ABBB, BBBB}
\frame{\frametitle{State $c = \cparm$, $m = \mparm$}

  %\vspace{-0.5cm}
  \begin{tabular}{cc}
    \hspace{-0.5cm}
    \begin{tabular}{p{.5\textwidth}}
    \paragraph{\statename:} \\
    \statedesc
    
    \bigskip
    Possible genotypes: \\
    \centerline{\stategeno}
    \end{tabular}
    &
    \hspace{-1cm}
    \begin{tabular}{p{.5\textwidth}}
      \includegraphics[width=.5\textwidth, height=.45\textheight]{\figdir/\figname-c\cparm-m\mparm-LRR}
    \end{tabular} \\
    \hspace{-0.5cm}
    \begin{tabular}{p{.5\textwidth}}
      \includegraphics[width=.5\textwidth, height=.45\textheight]{\figdir/\figname-c\cparm-m\mparm-BAF}
    \end{tabular}
    &
    \hspace{-1cm}
    \begin{tabular}{p{.5\textwidth}}
      \includegraphics[width=.5\textwidth, height=.45\textheight]{\figdir/\figname-c\cparm-m\mparm-AB}
    \end{tabular} \\    
  \end{tabular}
}


\renewcommand{\cparm}{4} \renewcommand{\mparm}{2}
\renewcommand{\statename}{Double duplication}
\renewcommand{\statedesc}{Both copies amplified}
\renewcommand{\stategeno}{AAAA, AABB, BBBB}
\frame{\frametitle{State $c = \cparm$, $m = \mparm$}

  %\vspace{-0.5cm}
  \begin{tabular}{cc}
    \hspace{-0.5cm}
    \begin{tabular}{p{.5\textwidth}}
    \paragraph{\statename:} \\
    \statedesc
    
    \bigskip
    Possible genotypes: \\
    \centerline{\stategeno}
    \end{tabular}
    &
    \hspace{-1cm}
    \begin{tabular}{p{.5\textwidth}}
      \includegraphics[width=.5\textwidth, height=.45\textheight]{\figdir/\figname-c\cparm-m\mparm-LRR}
    \end{tabular} \\
    \hspace{-0.5cm}
    \begin{tabular}{p{.5\textwidth}}
      \includegraphics[width=.5\textwidth, height=.45\textheight]{\figdir/\figname-c\cparm-m\mparm-BAF}
    \end{tabular}
    &
    \hspace{-1cm}
    \begin{tabular}{p{.5\textwidth}}
      \includegraphics[width=.5\textwidth, height=.45\textheight]{\figdir/\figname-c\cparm-m\mparm-AB}
    \end{tabular} \\    
  \end{tabular}
}


%====================================================================
\subsection{Genotype}
\frame{\frametitle{Genotype}

  Denote $z = (a, b)$, $a = \#$ copies of allele A: $z = (2, 1) = AAB$.\begin{itemize}
  \item A same genotype $z$ can be found in several states $s = (c, m)$. \\ ~
  \item \pause Denoting $p$ the frequency of allele A at the current locus:
  \end{itemize}
  \footnotesize{
  $$
  \begin{array}{r|ccccccccccccccc}
  s=& \multicolumn{15}{c}{z = (a ,b)} \\
  (c, m) & 00 & 10 & 01 & 20 & 11 & 02 & 30 & 21 & 12 & 03 & 40 & 31 & 22 & 13 & 04 \\
  \hline
  (0, 0) & 1 \\
  (1, 0) & & p & q \\
  (2, 0) & & & & p &  & q\\
  (2, 1) & & & & p^2 & 2pq & q^2\\
  (3, 0) & & & & & & & p & & & q \\
  (3, 1) & & & & & & & p^2 & pq & pq & q^2 \\
  (4, 0) & & & & & & & & & & & p & & & & q \\
  (4, 1) & & & & & & & & & & & p^2 & pq & & pq & q^2 \\
  (4, 2) & & & & & & & & & & & p^2 & & 2pq & & q^2
  \end{array}
  $$}
  where $q = 1-p$.

}

%====================================================================
\subsection{Hidden Markov model}
\frame{\frametitle{HMM with mixture emissions}

  \begin{itemize}
   \item \pause CNV/LOH status $S_t =$ homogeneous Markov chain:
   $$
   (S_t)_t \sim MC(\pi);
   $$
   \item \pause Genotypes $(Z_t)_t$ independent conditional on $(S_t)_t$:
   $$
   (Z_t | S_t = s) \sim {\mathcal M}(1; \nu_s), 
   \qquad \nu_s = \nu_s(p_t, q_t);
   $$
   \item \pause SNP signal $Y_t = (A_t, B_t)$ independent conditional on $(Z_t)_t$:
   $$
   [Y_t | Z_t=z] \sim f_{z = (a, b)} = f(\mu_a, \mu_b, ...).
   $$
  \end{itemize}
  
  \pause
  \paragraph{Mixture emissions:}
   $$
   (Y_t | S_t=s) \sim \phi_z = \sum_{z} \nu_{sz} f_z.
   $$
}

%====================================================================
\frame{\frametitle{Identifiability}

  \paragraph{Theorem (Gassiat, 2013).} Hidden Markov models are identifiable as soon as the emission distributions are linearly independent.
  
  \bigskip
  {\sl Sketch of proof.} Consider the joint distribution of $(Y_{t-1}, Y_t, Y_{t+1})$. The rest follows from \refer{AMR09}

  \pause \bigskip \bigskip
  \paragraph{Corollary.} The proposed HMM for CNV/LOH analysis is identifiable as soon as the emission distributions $f_z$ are linearly independent.
  
  \bigskip
  {\sl Proof.} The matrix $\nu = [\nu_{sz}]$ has full rank.
}

%====================================================================
\section{Inference}
\subsection{EM}
\frame{\frametitle{Inference}

  \paragraph{Maximum likelihood:} use a regular EM algorithm.
  
  \pause \bigskip %\bigskip
  \paragraph{E step.}   $(S_t, Z_t)$ is a Markov chain with specific transition matrix 
  $$
  P = f(\pi, \nu).
  $$
  \ra~Forward-backward recursion directly applies.

  \pause \bigskip %\bigskip
  \paragraph{M step.} Some specificities:
  \begin{itemize}
   \item The emission parameters $\mu_a$, $\mu_b$ are supposed to be organized along a grid (see examples).
   \item Allelic frequencies $(p_t, q_t)$ are 
    \begin{itemize}
    \item either known (provided by Affymetrix) 
    \item or supposed constant ($p_t \equiv p$) 
    \item or simply fixed (e.g. $p = q = .5$)
    \end{itemize}
  \end{itemize}

  }

%====================================================================
\frame{\frametitle{Model selection}

  The performances may vary according to the choice of $c_{\min}$ and $c_{\max}$. 
  
  \pause \bigskip \bigskip
  \paragraph{Criterion.} As we are in a classification context, we use
  $$
  ICL 
  = \mathbb E(\log P(Y, H) | Y; \widehat{\theta}) - .5 d \log n 
  = BIC - \mathbb H(H | Y) 
  $$
  where $H$ stands for the hidden variables.
  
  \pause \bigskip \bigskip
  \paragraph{Dedicated version.} As we are only interested in $S$, we use
  \begin{eqnarray*}
  ICL = BIC - \mathbb H(S | Y) 
  \end{eqnarray*}
  which does not penalize for the posterior entropy of $Z$.
  }

%====================================================================
\section{Illustrations}
\subsection{Simulations}
%====================================================================
\frame{\frametitle{Some simulations}

  \paragraph{Simulation design:} According to the model with $n = 10^5$ and emission distributions
  $$
  f_{(a, b)} = \text{Gam}(.1+a, \sigma^2) \otimes
  \text{Gam}(.1+b, \sigma^2)
  $$
  denoting Gam(mean, variance).
  
  \pause \bigskip \bigskip
  \paragraph{Model fitted.}
   $$
   f_{(a, b)} =  \mathcal N \left(\left[\begin{array}{c} \mu_a \\ \mu_b \end{array} \right], \left[\begin{array}{cc} \sigma^2 & 0 \\ 0 & \sigma^2 \end{array} \right] \right)
   $$
  }

%====================================================================
\renewcommand{\figdir}{/home/robin/RECHERCHE/RUPTURES/LOH/Simul}
\renewcommand{\figname}{SimulAB_gamma_s0.1-c2-3}

%====================================================================
\frame{\frametitle{Low variance ($\sigma^2 = .1$)}

  \begin{tabular}{cc}
    \hspace{-0.5cm}
    \begin{tabular}{p{.45\textwidth}}
      \paragraph{Model choice:} 
      $$c_{\min} = 2, c_{\max} = 3$$
%       \paragraph{Rand index:} 99.8 \%. 
      
    \end{tabular}
    &
    \hspace{-1cm}
    \begin{tabular}{p{.5\textwidth}}
      \includegraphics[width=.5\textwidth, height=.45\textheight]{\figdir/\figname-LRR}
    \end{tabular} \\
    \hspace{-0.5cm}
    \begin{tabular}{p{.5\textwidth}}
      \includegraphics[width=.5\textwidth, height=.45\textheight]{\figdir/\figname-BAF}
    \end{tabular}
    &
    \hspace{-1cm}
    \begin{tabular}{p{.5\textwidth}}
      \includegraphics[width=.5\textwidth, height=.45\textheight]{\figdir/\figname-AB}
    \end{tabular} \\    
  \end{tabular}
}

%====================================================================
\renewcommand{\figname}{SimulAB_gamma_s0.5-c0-3}

%====================================================================
\frame{\frametitle{High variance ($\sigma^2 = .5$)}

  \begin{tabular}{cc}
    \hspace{-0.5cm}
    \begin{tabular}{p{.45\textwidth}}
      \paragraph{Model choice:} 
      $$c_{\min} = 0, c_{\max} = 3$$
%       \paragraph{Rand index:} 99.4 \%. 
      
    \end{tabular}
    &
    \hspace{-1cm}
    \begin{tabular}{p{.5\textwidth}}
      \includegraphics[width=.5\textwidth, height=.45\textheight]{\figdir/\figname-LRR}
    \end{tabular} \\
    \hspace{-0.5cm}
    \begin{tabular}{p{.5\textwidth}}
      \includegraphics[width=.5\textwidth, height=.45\textheight]{\figdir/\figname-BAF}
    \end{tabular}
    &
    \hspace{-1cm}
    \begin{tabular}{p{.5\textwidth}}
      \includegraphics[width=.5\textwidth, height=.45\textheight]{\figdir/\figname-AB}
    \end{tabular} \\    
  \end{tabular}
}

%====================================================================
\frame{\frametitle{Model choice matters}

  Series of combinations $(c_{\min}, c_{\max})$ with various variances $\sigma^2 = .1, \textcolor{red}{.2}, \textcolor{green}{.3}, \textcolor{blue}{.4}, \textcolor{cyan}{.5}$

  $$
  \includegraphics[width=.5\textwidth]{\figdir/randIndexBIC}
  $$
}

%====================================================================
\renewcommand{\figdir}{/home/robin/RECHERCHE/RUPTURES/LOH/Data/NA06991}
\renewcommand{\figname}{NA06991-BAF-R-Chr6-c0-3}

%====================================================================
\subsection{Examples}
\frame{\frametitle{Chromosome 6 from \refer{SLV08}}

  \begin{tabular}{cc}
    \hspace{-0.5cm}
    \begin{tabular}{p{.45\textwidth}}
      \paragraph{Normalization matters:} 
      $$c_{\min} = 0, c_{\max} = 3$$
       
    \end{tabular}
    &
    \hspace{-1cm}
    \begin{tabular}{p{.5\textwidth}}
      \includegraphics[width=.5\textwidth, height=.45\textheight]{\figdir/\figname-LRR}
    \end{tabular} \\
    \hspace{-0.5cm}
    \begin{tabular}{p{.5\textwidth}}
      \includegraphics[width=.5\textwidth, height=.45\textheight]{\figdir/\figname-BAF}
    \end{tabular}
    &
    \hspace{-1cm}
    \begin{tabular}{p{.5\textwidth}}
      \includegraphics[width=.5\textwidth, height=.45\textheight]{\figdir/\figname-AB}
    \end{tabular} \\    
  \end{tabular}
}

%====================================================================
\renewcommand{\figname}{NA06991-Chr6-c0-3}

%====================================================================
\frame{\frametitle{Chromosome 6 from \refer{SLV08}}

  \begin{tabular}{cc}
    \hspace{-0.5cm}
    \begin{tabular}{p{.45\textwidth}}
      \paragraph{Normalization matters:} 
      $$c_{\min} = 0, c_{\max} = 3$$
      
    \end{tabular}
    &
    \hspace{-1cm}
    \begin{tabular}{p{.5\textwidth}}
      \includegraphics[width=.5\textwidth, height=.45\textheight]{\figdir/\figname-LRR}
    \end{tabular} \\
    \hspace{-0.5cm}
    \begin{tabular}{p{.5\textwidth}}
      \includegraphics[width=.5\textwidth, height=.45\textheight]{\figdir/\figname-BAF}
    \end{tabular}
    &
    \hspace{-1cm}
    \begin{tabular}{p{.5\textwidth}}
      \includegraphics[width=.5\textwidth, height=.45\textheight]{\figdir/\figname-AB}
    \end{tabular} \\    
  \end{tabular}
}

%====================================================================
\renewcommand{\figname}{NA06991-Chr6-c2-2}

%====================================================================
\frame{\frametitle{Chromosome 6 from \refer{SLV08}}

  \begin{tabular}{cc}
    \hspace{-0.5cm}
    \begin{tabular}{p{.45\textwidth}}
      \paragraph{Model choice matters matters:} 
      $$
      \text{Plot:} \quad (c_{\min}, c_{\max}) = (2, 2)
      $$
      but
      $
      \arg\max ICL = (0, 3).
      $
      
    \end{tabular}
    &
    \hspace{-1cm}
    \begin{tabular}{p{.5\textwidth}}
      \includegraphics[width=.5\textwidth, height=.45\textheight]{\figdir/\figname-LRR}
    \end{tabular} \\
    \hspace{-0.5cm}
    \begin{tabular}{p{.5\textwidth}}
      \includegraphics[width=.5\textwidth, height=.45\textheight]{\figdir/\figname-BAF}
    \end{tabular}
    &
    \hspace{-1cm}
    \begin{tabular}{p{.5\textwidth}}
      \includegraphics[width=.5\textwidth, height=.45\textheight]{\figdir/\figname-AB}
    \end{tabular} \\    
  \end{tabular}
}

%====================================================================
\section{Conclusion}
\subsection{}
\frame{\frametitle{To be continued}

  \paragraph{Summary:}
  \begin{itemize}
   \item A (first?) joint modeling for CNV and LOH analysis.
   \item A comfortable framework for parameter inference and classification.
  \end{itemize}

  \bigskip \bigskip \pause
  \paragraph{Still to be done:}
  \begin{itemize}
   \item Complete simulation study 
   \item Improve model selection ?
   \item Consider alternative emission distribution (Gamma?) 
 %  \item Knowing the phase would still improve (neighbor genotypes are not independent)
   \item Accounting for contamination
  \end{itemize}


}

%====================================================================
%====================================================================
\section*{Appendix}
%====================================================================
{\tiny
  \bibliography{../../../../Biblio/ARC,../../../../Biblio/AST,../../../../Biblio/SSB}
  \bibliographystyle{../../../../LATEX/astats}
  %\bibliographystyle{plain}
  }

%====================================================================
%====================================================================
\end{document}
%====================================================================
%====================================================================


\frame{\frametitle{}
  }

  \vspace{-0.5cm}
  \begin{tabular}{cc}
    \hspace{-0.5cm}
    \begin{tabular}{p{.5\textwidth}}
    \end{tabular}
    &
    \hspace{-1cm}
    \begin{tabular}{p{.5\textwidth}}
    \end{tabular}
  \end{tabular}
