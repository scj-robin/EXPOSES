\documentclass[dvips, lscape]{foils}
%\documentclass[dvips, french]{slides}
\textwidth 18.5cm
\textheight 25cm 
\topmargin -1cm 
\oddsidemargin  -1cm 
\evensidemargin  -1cm

% Maths
\usepackage{amsfonts, amsmath, amssymb}

\newcommand{\coefbin}[2]{\left( 
    \begin{array}{c} #1 \\ #2 \end{array} 
  \right)}
\newcommand{\Bcal}{\mathcal{B}}
\newcommand{\Ccal}{\mathcal{C}}
\newcommand{\Dcal}{\mathcal{D}}
\newcommand{\Ecal}{\mathcal{E}}
\newcommand{\Gcal}{\mathcal{G}}
\newcommand{\Mcal}{\mathcal{M}}
\newcommand{\Ncal}{\mathcal{N}}
\newcommand{\Pcal}{\mathcal{P}}
\newcommand{\Rcal}{\mathcal{R}}
\newcommand{\Lcal}{\mathcal{L}}
\newcommand{\Tcal}{\mathcal{T}}
\newcommand{\Ucal}{\mathcal{U}}
\newcommand{\alphabf}{\mbox{\mathversion{bold}{$\alpha$}}}
\newcommand{\betabf}{\mbox{\mathversion{bold}{$\beta$}}}
\newcommand{\gammabf}{\mbox{\mathversion{bold}{$\gamma$}}}
\newcommand{\mubf}{\mbox{\mathversion{bold}{$\mu$}}}
\newcommand{\Pibf}{\mbox{\mathversion{bold}{$\Pi$}}}
\newcommand{\psibf}{\mbox{\mathversion{bold}{$\psi$}}}
\newcommand{\Sigmabf}{\mbox{\mathversion{bold}{$\Sigma$}}}
\newcommand{\taubf}{\mbox{\mathversion{bold}{$\tau$}}}
\newcommand{\Hbf}{{\bf H}}
\newcommand{\Ibf}{{\bf I}}
\newcommand{\Sbf}{{\bf S}}
\newcommand{\mbf}{{\bf m}}
\newcommand{\ubf}{{\bf u}}
\newcommand{\vbf}{{\bf v}}
\newcommand{\xbf}{{\bf x}}
\newcommand{\Xbf}{{\bf X}}
\newcommand{\Esp}{{\mathbb E}}
\newcommand{\Var}{{\mathbb V}}
\newcommand{\Cov}{{\mathbb C}\mbox{ov}}
\newcommand{\Ibb}{{\mathbb I}}
\newcommand{\Rbb}{\mathbb{R}}

% sommes
\newcommand{\sumk}{\sum_k}
\newcommand{\sumt}{\sum_{t \in I_k}}
\newcommand{\sumth}{\sum_{t=t_{k-1}^{(h)}+1}^{t_k^{(h)}}}
\newcommand{\sump}{\sum_{p=1}^{P}}
\newcommand{\suml}{\sum_{\ell=1}^{P}}
\newcommand{\sumtau}{\sum_k \hat{\tau}_{kp}}

% Couleur et graphiques
\usepackage{color}
\usepackage{graphics}
\usepackage{epsfig} 
\usepackage{pstcol}

% Texte
\usepackage{lscape}
\usepackage{../../../../Latex/fancyheadings, rotating, enumerate}
%\usepackage[french]{babel}
\usepackage[latin1]{inputenc}
\definecolor{darkgreen}{cmyk}{0.5, 0, 0.5, 0.5}
\definecolor{orange}{cmyk}{0, 0.6, 0.8, 0}
\definecolor{jaune}{cmyk}{0, 0.5, 0.5, 0}
\newcommand{\textblue}[1]{\textcolor{blue}{#1}}
\newcommand{\textred}[1]{\textcolor{red}{#1}}
\newcommand{\textgreen}[1]{\textcolor{green}{ #1}}
\newcommand{\textlightgreen}[1]{\textcolor{green}{#1}}
%\newcommand{\textgreen}[1]{\textcolor{darkgreen}{#1}}
\newcommand{\textorange}[1]{\textcolor{orange}{#1}}
\newcommand{\textyellow}[1]{\textcolor{yellow}{#1}}
\newcommand{\refer}[2]{{\sl #1}}

% Sections
%\newcommand{\chapter}[1]{\centerline{\LARGE \textblue{#1}}}
% \newcommand{\section}[1]{\centerline{\Large \textblue{#1}}}
% \newcommand{\subsection}[1]{\noindent{\Large \textblue{#1}}}
% \newcommand{\subsubsection}[1]{\noindent{\large \textblue{#1}}}
% \newcommand{\paragraph}[1]{\noindent {\textblue{#1}}}
% Sectionsred
\newcommand{\chapter}[1]{
  \addtocounter{chapter}{1}
  \setcounter{section}{0}
  \setcounter{subsection}{0}
  {\centerline{\LARGE \textblue{\arabic{chapter} - #1}}}
  }
\newcommand{\section}[1]{
  \addtocounter{section}{1}
  \setcounter{subsection}{0}
  {\centerline{\Large \textblue{\arabic{chapter}.\arabic{section} - #1}}}
  }
\newcommand{\subsection}[1]{
  \addtocounter{subsection}{1}
  {\noindent{\large \textblue{#1}}}
  }
% \newcommand{\subsection}[1]{
%   \addtocounter{subsection}{1}
%   {\noindent{\large \textblue{\arabic{chapter}.\arabic{section}.\arabic{subsection} - #1}}}
%   }
\newcommand{\paragraph}[1]{\noindent{\textblue{#1}}}
\newcommand{\emphase}[1]{\textblue{#1}}

%%%%%%%%%%%%%%%%%%%%%%%%%%%%%%%%%%%%%%%%%%%%%%%%%%%%%%%%%%%%%%%%%%%%%%
%%%%%%%%%%%%%%%%%%%%%%%%%%%%%%%%%%%%%%%%%%%%%%%%%%%%%%%%%%%%%%%%%%%%%%
%%%%%%%%%%%%%%%%%%%%%%%%%%%%%%%%%%%%%%%%%%%%%%%%%%%%%%%%%%%%%%%%%%%%%%
%%%%%%%%%%%%%%%%%%%%%%%%%%%%%%%%%%%%%%%%%%%%%%%%%%%%%%%%%%%%%%%%%%%%%%
\begin{document}
%%%%%%%%%%%%%%%%%%%%%%%%%%%%%%%%%%%%%%%%%%%%%%%%%%%%%%%%%%%%%%%%%%%%%%
%%%%%%%%%%%%%%%%%%%%%%%%%%%%%%%%%%%%%%%%%%%%%%%%%%%%%%%%%%%%%%%%%%%%%%
%%%%%%%%%%%%%%%%%%%%%%%%%%%%%%%%%%%%%%%%%%%%%%%%%%%%%%%%%%%%%%%%%%%%%%
%%%%%%%%%%%%%%%%%%%%%%%%%%%%%%%%%%%%%%%%%%%%%%%%%%%%%%%%%%%%%%%%%%%%%%
\landscape
\newcounter{chapter}
\newcounter{section}
\newcounter{subsection}
\setcounter{chapter}{0}
\headrulewidth 0pt 
\pagestyle{fancy} 
\cfoot{}
\rfoot{\begin{rotate}{90}{
      %\hspace{1cm} \tiny S. Robin: Segmentation-clustering for CGH
      }\end{rotate}}
\rhead{\begin{rotate}{90}{
      \hspace{-.5cm} \tiny \thepage
      }\end{rotate}}

%%%%%%%%%%%%%%%%%%%%%%%%%%%%%%%%%%%%%%%%%%%%%%%%%%%%%%%%%%%%%%%%%%%%%%
%%%%%%%%%%%%%%%%%%%%%%%%%%%%%%%%%%%%%%%%%%%%%%%%%%%%%%%%%%%%%%%%%%%%%%
\begin{center}
  \textblue{\LARGE Significance of Minimal Regions}
  
  \textblue{\LARGE  in Multiple CGH Profiles}
\end{center}

\paragraph{Working group:} 
\begin{itemize}
\item {\bf C. Rouveirol} (Institut Curie / Paris XI), 
\item R. Radvanyi (Institut Curie), 
\item E. Lebarbier (SSB, Paris), 
\item F. Picard (SSB, Paris, San Francisco, Evry, Lyon, Chateauroux),
\item {\bf S. Robin} (SSB, Paris)
\end{itemize}

% %%%%%%%%%%%%%%%%%%%%%%%%%%%%%%%%%%%%%%%%%%%%%%%%%%%%%%%%%
% %%%%%%%%%%%%%%%%%%%%%%%%%%%%%%%%%%%%%%%%%%%%%%%%%%%%%%%%%%%%
% \newpage
% \chapter{Detection of chromosomal aberrations}
% %%%%%%%%%%%%%%%%%%%%%%%%%%%%%%%%%%%%%%%%%%%%%%%%%%%%%%%%%%%%
% %%%%%%%%%%%%%%%%%%%%%%%%%%%%%%%%%%%%%%%%%%%%%%%%%%%%%%%%%

% %%%%%%%%%%%%%%%%%%%%%%%%%%%%%%%%%%%%%%%%%%%%%%%%%%%%%%%%%
% \vspace{1cm}
% \section{Aberration at the chromosomic scale}
% %%%%%%%%%%%%%%%%%%%%%%%%%%%%%%%%%%%%%%%%%%%%%%%%%%%%%%%%%

% Known effects of big size chromosomal aberrations  (ex:
% trisomy).

% \centerline{$\rightarrow$ experimental tool: \textblue{Karyotype}
%   (Resolution $\sim$ chromosome)} 

% $$
% \epsfig{file = ../Figures/Karyotype.ps, clip=,
%   bbllx=158, bblly=560, bburx=452, bbury=778, scale=1.2}
% $$

% %%%%%%%%%%%%%%%%%%%%%%%%%%%%%%%%%%%%%%%%%%%%%%%%%%%%%%%%%
% \newpage
% \section{Within chromosome aberration}
% %%%%%%%%%%%%%%%%%%%%%%%%%%%%%%%%%%%%%%%%%%%%%%%%%%%%%%%%%
% \begin{itemize}
% \item Change of scale: what are the effects of small size DNA
%   sequences deletions/amplifications?\\
%  \\
%   \centerline{$\rightarrow$ experimental tool:
%     \textblue{"conventional" CGH} (resolution $\sim$ 10Mb).}
% \item CGH = Comparative Genomic Hybridization: method for the
%   comparative measurement of relative DNA copy numbers between two
%   samples (normal/disease, test/reference).\\ 
%  \\
%   \centerline{$\rightarrow$ Application of the \textblue{microarray}
%     technology to CGH (resolution $\sim$ 100kb).}
%   $$
%   \epsfig{file = ../Figures/CGHarray.ps, clip=,
%     bbllx=113, bblly=564, bburx=497, bbury=778, scale=1.1}
%   $$
% \end{itemize}

% %%%%%%%%%%%%%%%%%%%%%%%%%%%%%%%%%%%%%%%%%%%%%%%%%%%%%%%%%%%%
% \newpage
% \section{Microarray technology in its principle }
% %%%%%%%%%%%%%%%%%%%%%%%%%%%%%%%%%%%%%%%%%%%%%%%%%%%%%%%%%%%%
% \vspace{-1cm}
% $$
% \epsfig{file = ../Figures/principe_CGH.eps, clip=,
%   bbllx=0, bblly=41, bburx=700, bbury=478, scale=0.9}
% $$

% %%%%%%%%%%%%%%%%%%%%%%%%%%%%%%%%%%%%%%%%%%%%%%%%%%%%%%%%%%
% \newpage
% \section{A segmentation / clustering problem}
% %%%%%%%%%%%%%%%%%%%%%%%%%%%%%%%%%%%%%%%%%%%%%%%%%%%%%%%%%%

% Several methods have been proposed to detect the breakpoint position
% (left) and to classify segments into status (deleted / normal /
% amplified: right).
% $$
%   \begin{tabular}{cc}
%     Segmentation & Segmentation/Clustering \\
%     \multicolumn{2}{c}{$K=4$ segments} \\
%     \epsfig{file = ../Figures/bt474_c9_seg_homo_K4, clip=, scale=0.7} 
%     & 
%     \epsfig{file = ../Figures/bt474_c9_segclas_homo_P3K4 , clip=, scale=0.7} 
%   \end{tabular}
% $$

%%%%%%%%%%%%%%%%%%%%%%%%%%%%%%%%%%%%%%%%%%%%%%%%%%%%%%%%%%%%%%%%%%%%%%
%%%%%%%%%%%%%%%%%%%%%%%%%%%%%%%%%%%%%%%%%%%%%%%%%%%%%%%%%%%%%%%%%%%%%%
\newpage
\chapter{Minimal region in (discretized) CGH profiles}
%%%%%%%%%%%%%%%%%%%%%%%%%%%%%%%%%%%%%%%%%%%%%%%%%%%%%%%%%%%%%%%%%%%%%%
%%%%%%%%%%%%%%%%%%%%%%%%%%%%%%%%%%%%%%%%%%%%%%%%%%%%%%%%%%%%%%%%%%%%%%

%%%%%%%%%%%%%%%%%%%%%%%%%%%%%%%%%%%%%%%%%%%%%%%%%%%%%%%%%%%%%%%%%%%%%%
\bigskip
\section{Discretized CGH profiles}
%%%%%%%%%%%%%%%%%%%%%%%%%%%%%%%%%%%%%%%%%%%%%%%%%%%%%%%%%%%%%%%%%%%%%%

\hspace{-2cm}
\begin{tabular}{cc}
  \begin{tabular}{p{7cm}}
    $i = 1..m$ patients \\
    ($m = 84$), \\
    \\
    $t = 1..n$ positions \\
    ($n = 2360$) \\
    \\
    $X_{it}$ = status of\\
    patient $i$ %\\
    at position $t$. \\
    \\
    \begin{tabular}{rcrl}
      $X_{it}$ & = & \textblue{--1} & \textblue{(loss)} \\
       & = & 0 & (normal) \\
       & = & \textred{+1} & \textred{(gain)}
    \end{tabular}
  \end{tabular}
  &
  \begin{tabular}{c}
    \epsfig{file=/RECHERCHE/RUPTURES/Curie/MinRegion/Data1/Profiles.eps}
  \end{tabular}
\end{tabular}

%%%%%%%%%%%%%%%%%%%%%%%%%%%%%%%%%%%%%%%%%%%%%%%%%%%%%%%%%%%%%%%%%%%%%%
\newpage
\section{Minimal region}
%%%%%%%%%%%%%%%%%%%%%%%%%%%%%%%%%%%%%%%%%%%%%%%%%%%%%%%%%%%%%%%%%%%%%%
$$
\begin{tabular}{cc}
  \begin{tabular}{p{15cm}}
    A minimal region $\Rcal$ is a sequence of \emphase{successive positions}
    for which the \emphase{same status} is observed in \emphase{large
    number} of patients at the same time. \\
    \\
    It is characterized by
    \begin{itemize}
    \item its position $t^*$, 
    \item its length $\ell$,
    \item its status $s$, 
    \item the number of patients $M^*$.
    \end{itemize}
  \end{tabular}
  &
  \begin{tabular}{c}
    \epsfig{file=/RECHERCHE/RUPTURES/Curie/MinRegion/Data1/ExMinRegion.eps,
      clip=, bbllx=270, bblly=209, bburx=400, bbury=593}
  \end{tabular}
\end{tabular}
$$

%%%%%%%%%%%%%%%%%%%%%%%%%%%%%%%%%%%%%%%%%%%%%%%%%%%%%%%%%%%%%%%%%%%%%%
\newpage
\section{Statistical question}
%%%%%%%%%%%%%%%%%%%%%%%%%%%%%%%%%%%%%%%%%%%%%%%%%%%%%%%%%%%%%%%%%%%%%%

\paragraph{Binary case.} Assume that only 2 status exist: 
$$
0 = \text{normal}, \quad 1 = \text{aberration}.
$$
\paragraph{Region occurrence.} $Y_{it} = Y_{it}(\Rcal)$ indicates the
occurrence of $\ell$ successive aberrations ending at position $t$ in
patient $i$:
$$
Y_{it}= \prod_{u=1}^{\ell} X_{i, t-u+1}.
$$
\paragraph{Simultaneous occurrences.} $Y_{+t} = Y_{+t}(\Rcal)$ counts
the number of patients for which $\Rcal$ occurs at position $t$:
$$
Y_{+t} = \sum_{i=1}^m Y_{it}.
$$
\paragraph{Significance of an observed minimal region.} We have to calculate
$$
\Pr\left\{\sup_{\ell \leq t \leq n}Y_{+t} \geq M^*\right\}.
$$

%%%%%%%%%%%%%%%%%%%%%%%%%%%%%%%%%%%%%%%%%%%%%%%%%%%%%%%%%%%%%%%%%%%%%%
%%%%%%%%%%%%%%%%%%%%%%%%%%%%%%%%%%%%%%%%%%%%%%%%%%%%%%%%%%%%%%%%%%%%%%
\newpage
\chapter{Significance}
%%%%%%%%%%%%%%%%%%%%%%%%%%%%%%%%%%%%%%%%%%%%%%%%%%%%%%%%%%%%%%%%%%%%%%
%%%%%%%%%%%%%%%%%%%%%%%%%%%%%%%%%%%%%%%%%%%%%%%%%%%%%%%%%%%%%%%%%%%%%%

%%%%%%%%%%%%%%%%%%%%%%%%%%%%%%%%%%%%%%%%%%%%%%%%%%%%%%%%%%%%%%%%%%%%%%
\bigskip
\section{Model}
%%%%%%%%%%%%%%%%%%%%%%%%%%%%%%%%%%%%%%%%%%%%%%%%%%%%%%%%%%%%%%%%%%%%%%

\paragraph{Markov chain.} 
Each profile $\{X_{it}\}_t$ is a 2-state stationary Markov chain (MC):
$$
\{X_{it}\}_t \sim \text{MC}(\Pibf, \mubf)
$$
where $\Pibf$ is the transition matrix and $\mubf$ the stationary
distribution ($\forall i: X_{i1} \sim \mubf$).

The patients (profiles) are supposed to be independent.

\bigskip
\paragraph{Estimated transition probabilities and stationary
  distributions.} 
$$
\text{Gain:} \qquad\qquad
\widehat{\Pibf}^+ = \left(\begin{array}{rr}
       99.74 &      0.26 \\
        1.98 &      98.02 \\
  \end{array}\right), 
\qquad \widehat{\mubf}^+ = (88.36 \quad      11.64), 
$$
$$
\text{Loss:} \qquad\qquad
\widehat{\Pibf}^- = \left(\begin{array}{rr}
        99.72 &      0.28 \\
       2.26 &       97.74 \\
  \end{array}\right), 
\qquad \widehat{\mubf}^- = (88.98 \quad      11.02).
$$
% $$
% \widehat{\Pibf} = \left(\begin{array}{rr}
%        99.7 & 0.3 \\
%        2.0 &  98.0 \\
%   \end{array}\right), 
% \qquad \widehat{\mubf} = (88.4 \quad  11.6).
% $$

%%%%%%%%%%%%%%%%%%%%%%%%%%%%%%%%%%%%%%%%%%%%%%%%%%%%%%%%%%%%%%%%%%%%%%
\newpage
\section{Exact calculation}
%%%%%%%%%%%%%%%%%%%%%%%%%%%%%%%%%%%%%%%%%%%%%%%%%%%%%%%%%%%%%%%%%%%%%%

\subsection{One patient} 

We are interested in patients for which $\Rcal$ occurs at some
position $t$. This depends on the number of successive aberrations
observed at the last $\ell$ positions.

\paragraph{Embedded Markov chain.} We define the embedded MC with
($\ell +1$) states $s = 0, \dots \ell$, where
$$
\text{\emphase{$s$ = number of '1' in the last $\ell$ positions.}}
$$
Denoting $p= \pi_{1, 1}$ and $q = \pi_{0, 0}$, the transition
matrix of the embedded MC is
$$
\left( 
\begin{array}{c|cccccc}
  & 0 & 1 & \dots  & s & \dots & \ell \\
  \hline
  0 & \pi_{0, 0} & \pi_{0, 1} & \\
  1 & \pi_{1, 0} & & \pi_{1, 1} & \\
  \vdots  & \vdots & & & \ddots \\
  s & \pi_{1, 0} & & & & \pi_{1, 1}  \\
  \vdots  & \vdots & & & & & \ddots \\
  \ell & \pi_{1, 0} & & & & & \pi_{1, 1}  
\end{array}
\right).
$$
\paragraph{$\Rcal$ occurs each time the MC visits state $\ell$.}

%%%%%%%%%%%%%%%%%%%%%%%%%%%%%%%%%%%%%%%%%%%%%%%%%%%%%%%%%%%%%%%%%%%%%%
\newpage

\subsection{$m$ patients} 

We are now interested in the number of patients for which $\Rcal$ occurs
at position $t$. This depends on the number of successive aberrations
observed in \emphase{all patients} at the last $\ell$ position.

\paragraph{Counts.} Denote 
\begin{description}
\item[$M^0_t = $] \# {normal patients
    at position $t$}: $M^0_t = {\sum_i (1-X_{it})}$ 
\item[$M^s_t = $] \# {patients with aberrations at (exactly) the last $s$
    positions} ($1 \leq s \leq \ell$): $M^s_t = {\sum_i (1-X_{i, t-s})
    \prod_{u=1}^s X_{i, t-s+u}}$ 
\item[$M^{\ell}_t =$]\# {patients with aberrations at the last $\ell$
    positions: $M^{\ell}_t = Y_{+t} = {\sum_i \prod_{u=1}^{\ell} X_{i,
        t-s+u}}$}
\end{description}

%%%%%%%%%%%%%%%%%%%%%%%%%%%%%%%%%%%%%%%%%%%%%%%%%%%%%%%%%%%%%%%%%%%%%%
\newpage
\paragraph{Multivariate binomial process.} The distribution of the
counts at position $t+1$ given the counts at position $t$ is defined
as follows:
\begin{eqnarray*}
  M^1_{t+1} & \sim & \Bcal(M^0_t, \pi_{0, 1}), \\
  M^{s+1}_{t+1} & \sim & \Bcal(M^s_t, \pi_{1, 1}), \qquad \qquad \text{for } 1 \leq
  s \leq \ell-2 \\
  M^{\ell}_{t+1} & \sim & \Bcal(M^{\ell-1}_t + M^{\ell}_t, \pi_{1,
  1}), \\
  M^0_{t+1} & = & M - \sum_{s=1}^{\ell} M^s_{t+1}. \\
\end{eqnarray*}

\vspace{-1cm}
\hspace{-2cm}
\begin{tabular}{cc}
  \begin{tabular}{p{12cm}}
    The process $\{(M^0_t, \dots, M^{\ell}_t)\}$ is a first order Markov
    chain. \textred{Its number of states} (\textgreen{$< (\ell+1)^m$}) is
    related to the \textblue{Stirling number of the second kind}:  
    $$
    m = 84, \ell = 15 \rightarrow
    8.3\;10^{86} \text{states}.
    $$
    The transition matrix is very sparse.
  \end{tabular}
    &
  \begin{tabular}{c}
    \epsfig{file=/RECHERCHE/RUPTURES/Curie/MinRegion/Res/NbState.eps,
    width=12cm, height=8cm}
  \end{tabular}
\end{tabular}


%%%%%%%%%%%%%%%%%%%%%%%%%%%%%%%%%%%%%%%%%%%%%%%%%%%%%%%%%%%%%%%%%%%%%%
\newpage
\section{Poisson approximation}
%%%%%%%%%%%%%%%%%%%%%%%%%%%%%%%%%%%%%%%%%%%%%%%%%%%%%%%%%%%%%%%%%%%%%%

\paragraph{Distribution of $Y_{it}$ and $Y_{+t}$.} Denoting the
occurrence probability of $\Rcal$ 
$$
\mu(\Rcal) = \mu_1 (\pi_{1, 1})^{\ell-1}
$$
and
$$
\forall i, t: Y_{it} \sim \Bcal[\mu(\Rcal)], 
\qquad
\forall t: Y_{+t} \sim \Bcal[m, \mu(\Rcal)], 
$$

\paragraph{Poisson approximation.} 
If $M^*$ is large, then $p(\Rcal) = \Pr\{Y_{+t} \geq M^*\}$ is small so (\emphase{?})
$$
\sum_{t=\ell}^n \Ibb\{Y_{+t} \geq M^*\} \overset{\text{\emphase{?}}}{\approx}
\Pcal[(n-\ell+1)p(\Rcal)]
$$
and (\emphase{?})
$$
\Pr\left\{\sup_t Y_{+t} \geq M^*\right\}
\overset{\text{\emphase{?}}}{\simeq}
1 - \exp[-(n-\ell+1)p(\Rcal)].
$$
\emphase{$\rightarrow$ does not work.}

%%%%%%%%%%%%%%%%%%%%%%%%%%%%%%%%%%%%%%%%%%%%%%%%%%%%%%%%%%%%%%%%%%%%%%
%%%%%%%%%%%%%%%%%%%%%%%%%%%%%%%%%%%%%%%%%%%%%%%%%%%%%%%%%%%%%%%%%%%%%%
\newpage
\section{Upper bound}
%%%%%%%%%%%%%%%%%%%%%%%%%%%%%%%%%%%%%%%%%%%%%%%%%%%%%%%%%%%%%%%%%%%%%%
%%%%%%%%%%%%%%%%%%%%%%%%%%%%%%%%%%%%%%%%%%%%%%%%%%%%%%%%%%%%%%%%%%%%%%

\paragraph{Numbered of altered patients.} Denoting $X_{+t} = \sum_i
X_{it}$ the number of altered patients at position $t$, we have
\begin{eqnarray*}
% \{Y_{+t} \geq M^*\} & \Rightarrow & \{X_{+t}, X_{+, t+1},\dots X_{+, t+\ell-1}
% \geq M^*\} \\ 
% \Rightarrow \qquad 
\Pr\{Y_{+t} \geq M^*\} & \leq & \Pr\{X_{+t}, X_{+, t+1},
\dots X_{+, t+\ell-1} \geq M^*\} \\
\Rightarrow \qquad \Pr\left\{\sup_t Y_{+t} \geq M^*\right\} & \leq &
\Pr\{\exists t: X_{+, t-\ell+1}, \dots, X_{+, t-1}, X_{+t} \geq M^*\} \\
\end{eqnarray*}

\paragraph{$\{X_{+t}\}$ as a Markov chain.} Given the number of altered
patients at position $t$ ($X_{+t}$), $X_{+, t+1}$ is
$$
X_{+, t+1} = A_{t+1} + B_{t+1}
$$
where
$$
\begin{array}{rcll}
  A_{t+1} & \sim & \Bcal(X_{+t}, \pi_{1,1})  & \text{number of altered at $t$
  still altered at $t+1$,} \\
  B_{t+1} & \sim & \Bcal(m-X_{+t}, \pi_{0,1})& \text{number of normal at $t$
  becoming altered et $t+1$.}
\end{array}
$$

%%%%%%%%%%%%%%%%%%%%%%%%%%%%%%%%%%%%%%%%%%%%%%%%%%%%%%%%%%%%%%%%%%%%%%
\newpage
\paragraph{Embedded MC $\{C^*_t\}$.} We can define the MC counting the
number of successive times where $X_{+t}$ exceeds $M^*$, with an
\emphase{absorbing state} when this number reaches $\ell$.

\paragraph{Example.} For $m = 84$ patients, $\ell = 5$ and $M^* = 31$,
the embedded MC has $(m+1) + (\ell-2)(m-M^*+1) + 1 = 248$ states; its
transition matrix has the following shape
$$
\vspace{-1cm}
\Pibf_{C^*} = \left(
  \begin{tabular}{c}
    \epsfig{file=/RECHERCHE/RUPTURES/Curie/MinRegion/Res/ExPiR.eps,
    width=10cm, clip=}
  \end{tabular}
  \right)
$$
\paragraph{$p$-value.}
The $p$-value is given by the distribution of $C^*_n$:
$\mubf^*_0 \left(\Pibf_{c^*}\right)^{n-1}$.

%%%%%%%%%%%%%%%%%%%%%%%%%%%%%%%%%%%%%%%%%%%%%%%%%%%%%%%%%%%%%%%%%%%%%%
%%%%%%%%%%%%%%%%%%%%%%%%%%%%%%%%%%%%%%%%%%%%%%%%%%%%%%%%%%%%%%%%%%%%%%
\newpage
\section{Application}
%%%%%%%%%%%%%%%%%%%%%%%%%%%%%%%%%%%%%%%%%%%%%%%%%%%%%%%%%%%%%%%%%%%%%%
%%%%%%%%%%%%%%%%%%%%%%%%%%%%%%%%%%%%%%%%%%%%%%%%%%%%%%%%%%%%%%%%%%%%%%

%%%%%%%%%%%%%%%%%%%%%%%%%%%%%%%%%%%%%%%%%%%%%%%%%%%%%%%%%%%%%%%%%%%%%%
\subsection{Multiple chromosomes}
%%%%%%%%%%%%%%%%%%%%%%%%%%%%%%%%%%%%%%%%%%%%%%%%%%%%%%%%%%%%%%%%%%%%%%

Profiles are actually spread into 24 distinct chromosomes (22 + X + Y) with
varying lengths.

\paragraph{One chromosome.}
Denoting $n_k$ the number of positions in chromosome $k$, the upper
bound of $\Pr\{\sup_{1 \leq t \leq n_k} Y_{+t} \geq M^*\}$ is 
$$
\Pr\left\{\sup_{1 \leq t \leq n_k} Y_{+t} \geq M^*\right\} \leq \mubf^*_0
(\Pibf^*)^{n_k-1} 
$$
where $\mubf^*_0$ denotes the binomial distribution $\Bcal(m,
\mu_1)$.

\paragraph{All chromosomes.} The global upper bound is given by 
$$
\Pr\left\{\sup_{1 \leq t \leq n} Y_{+t} \geq M^*\right\} \leq 1 -
\prod_{k=1}^K \left[1 - \mubf^*_0 (\Pibf^*)^{n_k-1} \right].
$$

%%%%%%%%%%%%%%%%%%%%%%%%%%%%%%%%%%%%%%%%%%%%%%%%%%%%%%%%%%%%%%%%%%%%%%
\newpage
\subsection{Abacus}
%%%%%%%%%%%%%%%%%%%%%%%%%%%%%%%%%%%%%%%%%%%%%%%%%%%%%%%%%%%%%%%%%%%%%%
For given number of patient $m$ and chromosomes lengths $\{n_k\}$, we
can give the abacus of the significance as a function of the length
$\ell$ and the count $M^*$.

\paragraph{Abacus for gains.} Upper bound of the $p$-value in
$\log_{10}$ scale.
$$
\epsfig{file=/RECHERCHE/RUPTURES/Curie/MinRegion/Data1/Abaque-gain.eps,
  width=16cm, clip=}
$$
Because of the very high value of $\pi_{1, 1}$, \emphase{the
  significance mainly depends on $M^*$}.

%%%%%%%%%%%%%%%%%%%%%%%%%%%%%%%%%%%%%%%%%%%%%%%%%%%%%%%%%%%%%%%%%%%%%%
\newpage
\subsection{Results for raw profiles}
%%%%%%%%%%%%%%%%%%%%%%%%%%%%%%%%%%%%%%%%%%%%%%%%%%%%%%%%%%%%%%%%%%%%%%

Significant regions at the (Bonferonni corrected) 10\% threshold.
{\small
$$
\hspace{-1.5cm}
\begin{tabular}{rrrrr|ccc|cc}
  \multicolumn{5}{c|}{Region} & \multicolumn{3}{c|}{Poisson approximation}
  & \multicolumn{2}{c}{Upper bound} \\
  \input{/RECHERCHE/RUPTURES/Curie/MinRegion/Data1/zones_curie_2_m10.TabRes.tex}
\end{tabular}
$$
}

%%%%%%%%%%%%%%%%%%%%%%%%%%%%%%%%%%%%%%%%%%%%%%%%%%%%%%%%%%%%%%%%%%%%%%
\newpage
\subsection{Results for smoothed  profiles}
%%%%%%%%%%%%%%%%%%%%%%%%%%%%%%%%%%%%%%%%%%%%%%%%%%%%%%%%%%%%%%%%%%%%%%

$$
\begin{array}{cc}
  \begin{array}{c}
    \epsfig{file=/RECHERCHE/RUPTURES/Curie/MinRegion/Data2/Profiles.eps}
  \end{array}
  & 
  \begin{array}{l}
    \Pibf^+ = \left(\begin{array}{cc}
        99.8   &       0.2 \\
        5.0     &      95.0
      \end{array}
    \right) \\ \\
    \mubf^+ = (96.1 \quad  3.9) \\
    \\ \\
    \Pibf^- = \left(\begin{array}{cc}
        99.8 &         0.2 \\
        4.5  &       95.5
      \end{array}
    \right) \\ \\
    \mubf^- = (95.2 \quad 4.8) \\
  \end{array}
\end{array}
$$

%%%%%%%%%%%%%%%%%%%%%%%%%%%%%%%%%%%%%%%%%%%%%%%%%%%%%%%%%%%%%%%%%%%%%%
\newpage
Significant regions at the (Bonferonni corrected) 10\% threshold.
{\small
$$
\hspace{-1.5cm}
\begin{tabular}{rrrrr|ccc|cc}
  \multicolumn{5}{c|}{Region} & \multicolumn{3}{c|}{Poisson approximation}
  & \multicolumn{2}{c}{Upper bound} \\
  \input{/RECHERCHE/RUPTURES/Curie/MinRegion/Data2/zones_curielisse_2_m10.TabRes.tex}
\end{tabular}
$$
}

%%%%%%%%%%%%%%%%%%%%%%%%%%%%%%%%%%%%%%%%%%%%%%%%%%%%%%%%%%%%%%%%%%%%%%
%%%%%%%%%%%%%%%%%%%%%%%%%%%%%%%%%%%%%%%%%%%%%%%%%%%%%%%%%%%%%%%%%%%%%%
%%%%%%%%%%%%%%%%%%%%%%%%%%%%%%%%%%%%%%%%%%%%%%%%%%%%%%%%%%%%%%%%%%%%%%
%%%%%%%%%%%%%%%%%%%%%%%%%%%%%%%%%%%%%%%%%%%%%%%%%%%%%%%%%%%%%%%%%%%%%%
\end{document}
%%%%%%%%%%%%%%%%%%%%%%%%%%%%%%%%%%%%%%%%%%%%%%%%%%%%%%%%%%%%%%%%%%%%%%
%%%%%%%%%%%%%%%%%%%%%%%%%%%%%%%%%%%%%%%%%%%%%%%%%%%%%%%%%%%%%%%%%%%%%%
%%%%%%%%%%%%%%%%%%%%%%%%%%%%%%%%%%%%%%%%%%%%%%%%%%%%%%%%%%%%%%%%%%%%%%
%%%%%%%%%%%%%%%%%%%%%%%%%%%%%%%%%%%%%%%%%%%%%%%%%%%%%%%%%%%%%%%%%%%%%%
