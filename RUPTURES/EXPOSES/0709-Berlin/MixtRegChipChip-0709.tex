\documentclass[dvips, lscape]{foils}
%\documentclass[dvips, french]{slides}
\textwidth 18.5cm
\textheight 25cm 
\topmargin -1cm 
\oddsidemargin  -1cm 
\evensidemargin  -1cm

% Maths
\usepackage{amsfonts, amsmath, amssymb}

\newcommand{\coefbin}[2]{\left( 
    \begin{array}{c} #1 \\ #2 \end{array} 
  \right)}
\newcommand{\Bcal}{\mathcal{B}}
\newcommand{\Ccal}{\mathcal{C}}
\newcommand{\Dcal}{\mathcal{D}}
\newcommand{\Ecal}{\mathcal{E}}
\newcommand{\Gcal}{\mathcal{G}}
\newcommand{\Mcal}{\mathcal{M}}
\newcommand{\Ncal}{\mathcal{N}}
\newcommand{\Pcal}{\mathcal{P}}
\newcommand{\Rcal}{\mathcal{R}}
\newcommand{\Lcal}{\mathcal{L}}
\newcommand{\Tcal}{\mathcal{T}}
\newcommand{\Ucal}{\mathcal{U}}
\newcommand{\alphabf}{\mbox{\mathversion{bold}{$\alpha$}}}
\newcommand{\betabf}{\mbox{\mathversion{bold}{$\beta$}}}
\newcommand{\gammabf}{\mbox{\mathversion{bold}{$\gamma$}}}
\newcommand{\mubf}{\mbox{\mathversion{bold}{$\mu$}}}
\newcommand{\Pibf}{\mbox{\mathversion{bold}{$\Pi$}}}
\newcommand{\psibf}{\mbox{\mathversion{bold}{$\psi$}}}
\newcommand{\Sigmabf}{\mbox{\mathversion{bold}{$\Sigma$}}}
\newcommand{\taubf}{\mbox{\mathversion{bold}{$\tau$}}}
\newcommand{\Hbf}{{\bf H}}
\newcommand{\Ibf}{{\bf I}}
\newcommand{\Sbf}{{\bf S}}
\newcommand{\mbf}{{\bf m}}
\newcommand{\ubf}{{\bf u}}
\newcommand{\vbf}{{\bf v}}
\newcommand{\xbf}{{\bf x}}
\newcommand{\Xbf}{{\bf X}}
\newcommand{\Esp}{{\mathbb E}}
\newcommand{\Var}{{\mathbb V}}
\newcommand{\Cov}{{\mathbb C}\mbox{ov}}
\newcommand{\Ibb}{{\mathbb I}}
\newcommand{\Rbb}{\mathbb{R}}

% sommes
\newcommand{\sumk}{\sum_k}
\newcommand{\sumt}{\sum_{t \in I_k}}
\newcommand{\sumth}{\sum_{t=t_{k-1}^{(h)}+1}^{t_k^{(h)}}}
\newcommand{\sump}{\sum_{p=1}^{P}}
\newcommand{\suml}{\sum_{\ell=1}^{P}}
\newcommand{\sumtau}{\sum_k \hat{\tau}_{kp}}

% Couleur et graphiques
\usepackage{color}
\usepackage{graphics}
\usepackage{epsfig} 
\usepackage{pstcol}
\newcommand{\Example}{tfl2_1_ResReg}

% Texte
\usepackage{lscape}
\usepackage{../../../../Latex/fancyheadings, rotating, enumerate}
%\usepackage[french]{babel}
\usepackage[latin1]{inputenc}
\definecolor{darkgreen}{cmyk}{0.5, 0, 0.5, 0.5}
\definecolor{orange}{cmyk}{0, 0.6, 0.8, 0}
\definecolor{jaune}{cmyk}{0, 0.5, 0.5, 0}
\newcommand{\textblue}[1]{\textcolor{blue}{#1}}
\newcommand{\textred}[1]{\textcolor{red}{#1}}
\newcommand{\textgreen}[1]{\textcolor{green}{ #1}}
\newcommand{\textlightgreen}[1]{\textcolor{green}{#1}}
%\newcommand{\textgreen}[1]{\textcolor{darkgreen}{#1}}
\newcommand{\textorange}[1]{\textcolor{orange}{#1}}
\newcommand{\textyellow}[1]{\textcolor{yellow}{#1}}
\newcommand{\refer}[2]{{\sl #1}}

% Sections
%\newcommand{\chapter}[1]{\centerline{\LARGE \textblue{#1}}}
% \newcommand{\section}[1]{\centerline{\Large \textblue{#1}}}
% \newcommand{\subsection}[1]{\noindent{\Large \textblue{#1}}}
% \newcommand{\subsubsection}[1]{\noindent{\large \textblue{#1}}}
% \newcommand{\paragraph}[1]{\noindent {\textblue{#1}}}
% Sectionsred
\newcommand{\chapter}[1]{
  \addtocounter{chapter}{1}
  \setcounter{section}{0}
  \setcounter{subsection}{0}
  {\centerline{\LARGE \textblue{\arabic{chapter} - #1}}}
  }
\newcommand{\section}[1]{
  \addtocounter{section}{1}
  \setcounter{subsection}{0}
  {\centerline{\Large \textblue{\arabic{chapter}.\arabic{section} - #1}}}
  }
\newcommand{\subsection}[1]{
  \addtocounter{subsection}{1}
  {\noindent{\large \textblue{#1}}}
  }
% \newcommand{\subsection}[1]{
%   \addtocounter{subsection}{1}
%   {\noindent{\large \textblue{\arabic{chapter}.\arabic{section}.\arabic{subsection} - #1}}}
%   }
\newcommand{\paragraph}[1]{\noindent{\textblue{#1}}}
\newcommand{\emphase}[1]{\textblue{#1}}

%%%%%%%%%%%%%%%%%%%%%%%%%%%%%%%%%%%%%%%%%%%%%%%%%%%%%%%%%%%%%%%%%%%%%%
%%%%%%%%%%%%%%%%%%%%%%%%%%%%%%%%%%%%%%%%%%%%%%%%%%%%%%%%%%%%%%%%%%%%%%
%%%%%%%%%%%%%%%%%%%%%%%%%%%%%%%%%%%%%%%%%%%%%%%%%%%%%%%%%%%%%%%%%%%%%%
%%%%%%%%%%%%%%%%%%%%%%%%%%%%%%%%%%%%%%%%%%%%%%%%%%%%%%%%%%%%%%%%%%%%%%
\begin{document}
%%%%%%%%%%%%%%%%%%%%%%%%%%%%%%%%%%%%%%%%%%%%%%%%%%%%%%%%%%%%%%%%%%%%%%
%%%%%%%%%%%%%%%%%%%%%%%%%%%%%%%%%%%%%%%%%%%%%%%%%%%%%%%%%%%%%%%%%%%%%%
%%%%%%%%%%%%%%%%%%%%%%%%%%%%%%%%%%%%%%%%%%%%%%%%%%%%%%%%%%%%%%%%%%%%%%
%%%%%%%%%%%%%%%%%%%%%%%%%%%%%%%%%%%%%%%%%%%%%%%%%%%%%%%%%%%%%%%%%%%%%%
\landscape
\newcounter{chapter}
\newcounter{section}
\newcounter{subsection}
\setcounter{chapter}{0}
\headrulewidth 0pt 
\pagestyle{fancy} 
\cfoot{}
\rfoot{\begin{rotate}{90}{
      %\hspace{1cm} \tiny S. Robin: Segmentation-clustering for CGH
      }\end{rotate}}
\rhead{\begin{rotate}{90}{
      \hspace{-.5cm} \tiny \thepage
      }\end{rotate}}

%%%%%%%%%%%%%%%%%%%%%%%%%%%%%%%%%%%%%%%%%%%%%%%%%%%%%%%%%%%%%%%%%%%%%%
%%%%%%%%%%%%%%%%%%%%%%%%%%%%%%%%%%%%%%%%%%%%%%%%%%%%%%%%%%%%%%%%%%%%%%
\begin{center}
  \textblue{\LARGE Regression Mixture Model }
  
  \textblue{\LARGE  for ChIP-chip Analysis}
  
  \vspace{1cm}
  {\large M.-L. Martin-Magniette$^{1,2}$, T. Mary-Huard$^{1}$,}
  
  {\large C.  B�rard$^{2}$, S. Robin$^{1}$}

  \vspace{1cm}
  ($^1$) UMR AgroParisTech / INRA, ($^2$) INRA URGV

\end{center}

%%%%%%%%%%%%%%%%%%%%%%%%%%%%%%%%%%%%%%%%%%%%%%%%%%%%%%%%%%%%
%%%%%%%%%%%%%%%%%%%%%%%%%%%%%%%%%%%%%%%%%%%%%%%%%%%%%%%%%%%%
\newpage
\chapter{Detecting Hybridization in CHip-Chip data}
%%%%%%%%%%%%%%%%%%%%%%%%%%%%%%%%%%%%%%%%%%%%%%%%%%%%%%%%%%%%
%%%%%%%%%%%%%%%%%%%%%%%%%%%%%%%%%%%%%%%%%%%%%%%%%%%%%%%%%%%%

%%%%%%%%%%%%%%%%%%%%%%%%%%%%%%%%%%%%%%%%%%%%%%%%%%%%%%%%%%%%
\bigskip
\section{CHip-Chip technology}
%%%%%%%%%%%%%%%%%%%%%%%%%%%%%%%%%%%%%%%%%%%%%%%%%%%%%%%%%%%%

%%%%%%%%%%%%%%%%%%%%%%%%%%%%%%%%%%%%%%%%%%%%%%%%%%%%%%%%%%%%
\newpage
\section{Unsupervised Classification Problem}
%%%%%%%%%%%%%%%%%%%%%%%%%%%%%%%%%%%%%%%%%%%%%%%%%%%%%%%%%%%%

\paragraph{Data:} For each probe $i$, we hence get 2 signals
$$
  X_i = \text{Input signal}, 
  \qquad 
  Y_i = \text{IP signal}.
$$

\bigskip
\paragraph{Problem:} According to these two signals, we \emphase{have to}  
\begin{itemize}
\item \vspace{-0.5cm} \emphase{Classify each probe} into the
  \emphase{enriched} group or into the \emphase{normal} group,
\item \vspace{-0.5cm} Account for the \emphase{link between the IP and
    Input} signals
\end{itemize}

\bigskip
\noindent In addition we \emphase{would like to}
\begin{itemize}
\item \vspace{-0.5cm} Evaluate the strength of the classification;
\item \vspace{-0.5cm} Avoid numerous false detections.
\end{itemize}

%%%%%%%%%%%%%%%%%%%%%%%%%%%%%%%%%%%%%%%%%%%%%%%%%%%%%%%%%%%%
%%%%%%%%%%%%%%%%%%%%%%%%%%%%%%%%%%%%%%%%%%%%%%%%%%%%%%%%%%%%
\newpage
\chapter{Mixture Model of Regressions}
%%%%%%%%%%%%%%%%%%%%%%%%%%%%%%%%%%%%%%%%%%%%%%%%%%%%%%%%%%%%
%%%%%%%%%%%%%%%%%%%%%%%%%%%%%%%%%%%%%%%%%%%%%%%%%%%%%%%%%%%%

%%%%%%%%%%%%%%%%%%%%%%%%%%%%%%%%%%%%%%%%%%%%%%%%%%%%%%%%%%%%
\bigskip
\section{Motivations}
%%%%%%%%%%%%%%%%%%%%%%%%%%%%%%%%%%%%%%%%%%%%%%%%%%%%%%%%%%%%
$$
\begin{tabular}{cc}
  \begin{tabular}{p{10cm}}
    \paragraph{Correlation IP / Input.} \\ \\
    A link between the two signal is observed in all studies. \\ \\
    This correlation suggest at least to consider the logratio
    IP/Input. \\
  \end{tabular}
  &
  \begin{tabular}{c}
    \epsfig{file = ../Figures/\Example-LogRatio.eps, width=12cm,
    height=12cm, clip=} 
%     \epsfig{file = ../Figures/NuageIPInput.ps, width=12cm,
%     height=12cm, clip=} 
  \end{tabular}
\end{tabular}
$$

%%%%%%%%%%%%%%%%%%%%%%%%%%%%%%%%%%%%%%%%%%%%%%%%%%%%%%%%%%%%
\newpage
$$
\begin{tabular}{cc}
  \begin{tabular}{p{10cm}}
    \paragraph{Mixture.} \\ \\
    A closer look shows that the relation between IP and Input
    may differ between the two groups 'normal' and 'enriched'.
  \end{tabular}
  &
  \begin{tabular}{c}
    \epsfig{file = ../Figures/\Example-RawIPInput.eps, width=12cm,
    height=12cm, clip=} 
  \end{tabular}
\end{tabular}
$$

%%%%%%%%%%%%%%%%%%%%%%%%%%%%%%%%%%%%%%%%%%%%%%%%%%%%%%%%%%%%
\newpage
\section{Mixture model}
%%%%%%%%%%%%%%%%%%%%%%%%%%%%%%%%%%%%%%%%%%%%%%%%%%%%%%%%%%%%

\bigskip
\paragraph{Model.} We assume that each probe $i$ has probability $\pi$
to be enriched:
$$
\Pr\{\text{Probe $i$ enriched}\} = \pi, 
\qquad
\Pr\{\text{Probe $i$ normal}\} = 1 - \pi, 
$$
and that the relation between log-IP ($Y_i$) and log-Input ($X_i$) depends on
the status of the probe:
$$
Y_i = \left\{ \begin{array}{ll}
    a_0 + b_0 X_i + E_i & \quad \text{if $i$ is normal} \\
    \\
    a_1 + b_1 X_i + E_i & \quad \text{if $i$ is enriched}
  \end{array} \right.
$$

\bigskip\bigskip
\paragraph{Comparison with the 'Logratio' analysis.} The standard
analysis based on the log(IP/Input) is the particular case of the
regression mixture model where
$$
b_0 = b_1 = 1.
$$

%%%%%%%%%%%%%%%%%%%%%%%%%%%%%%%%%%%%%%%%%%%%%%%%%%%%%%%%%%%%
\newpage
\section{Parameter Estimation and Probe Classification}
%%%%%%%%%%%%%%%%%%%%%%%%%%%%%%%%%%%%%%%%%%%%%%%%%%%%%%%%%%%%

\bigskip
\paragraph{Task.} We have to estimate
\begin{itemize}
\item \vspace{-0.5cm} the proportion of enriched probes: $\pi$;
\item \vspace{-0.5cm} the regression parameters: intercepts ($a_0$,
  $a_1$), slopes ($b_0$, $b_1$) and variance $\sigma^2$.
\end{itemize}

\bigskip\bigskip
\paragraph{Algorithm.} This can be done using the E-M algorithm which
alternates
\begin{description}
\item[E-step:] \vspace{-0.5cm} prediction of the probe status given
  the parameters;
\item[M-step:] \vspace{-0.5cm} estimation of the parameters given the
  (predicted) probe status.
\end{description}

\bigskip\bigskip
\paragraph{Posterior probability.} The status prediction is based on
the posterior probability $\tau_i$
$$
\textred{\tau_i = \Pr\{\text{$i$ enriched} \; |\; X_i, Y_i\}},
\qquad 1 - \tau_i = \Pr\{\text{$i$ normal} \; |\; X_i, Y_i\}.
$$
This probability provides a \emphase{probe classification rule}.

%%%%%%%%%%%%%%%%%%%%%%%%%%%%%%%%%%%%%%%%%%%%%%%%%%%%%%%%%%%%
\newpage
$$
\begin{tabular}{cc}
  \begin{tabular}{p{10cm}}
    \paragraph{Posterior probabilities:}
  \end{tabular}
  &
  \begin{tabular}{c}
%      \epsfig{file=../Figures/Graph_Regression_MoyDye_Rep1_chr4_degrade.ps,
%      width=12cm, height=12cm, angle=270, clip=} 
  \end{tabular}
\end{tabular}
$$

%%%%%%%%%%%%%%%%%%%%%%%%%%%%%%%%%%%%%%%%%%%%%%%%%%%%%%%%%%%%
%%%%%%%%%%%%%%%%%%%%%%%%%%%%%%%%%%%%%%%%%%%%%%%%%%%%%%%%%%%%
\newpage
\chapter{Limiting False Detections}
%%%%%%%%%%%%%%%%%%%%%%%%%%%%%%%%%%%%%%%%%%%%%%%%%%%%%%%%%%%%
%%%%%%%%%%%%%%%%%%%%%%%%%%%%%%%%%%%%%%%%%%%%%%%%%%%%%%%%%%%%

\bigskip
\paragraph{Maximum A Posteriori (MAP) rule.} Probes are usually classified
into their most probable class, using the 50\% threshold:
$$
\begin{array}{rclcl}
  \tau_i & \geq & 50\% & \Rightarrow & \text{$i$ classified as
  'enriched'}, \\ 
  \tau_i & < & 50\% & \Rightarrow & \text{$i$ classified as
  'normal'}. 
\end{array}
$$

\bigskip\bigskip
\paragraph{Controlling false detections.} We want to control the
probability for the $\tau_i$ of a normal probe to fall above the
classification threshold. 

For a \emphase{fixed risk $\alpha$} we calculate the threshold $s$
such that
$$
\textred{s: \qquad \Pr\{\tau_i > s \;|\; \text{$i$ normal}, \;
  \log(\text{Input})=X_i\} \leq \alpha}
$$
The threshold $s$ depends on both $\alpha$ and the log-Input $X_i$.

%%%%%%%%%%%%%%%%%%%%%%%%%%%%%%%%%%%%%%%%%%%%%%%%%%%%%%%%%%%%
\newpage
$$
\begin{tabular}{cc}
  \begin{tabular}{p{10cm}}
    \paragraph{Number of enriched probes.}
    Number enriched probes as a function of the risk $\alpha$.
  \end{tabular}
  &
  \begin{tabular}{c}
  \end{tabular}
\end{tabular}
$$


%%%%%%%%%%%%%%%%%%%%%%%%%%%%%%%%%%%%%%%%%%%%%%%%%%%%%%%%%%%%
\newpage
$$
\begin{tabular}{cc}
  \begin{tabular}{p{10cm}}
    \paragraph{Classification with 2\% error rate.}
  \end{tabular}
  &
  \begin{tabular}{c}
%     \epsfig{file= ../Figures/Graph_Regression_MoyDye_Rep1_chr4.ps,
%       width=12cm, height=12cm, angle=270, clip=} 
  \end{tabular}
\end{tabular}
$$

%%%%%%%%%%%%%%%%%%%%%%%%%%%%%%%%%%%%%%%%%%%%%%%%%%%%%%%%%%%%%%%%%%%%%%
%%%%%%%%%%%%%%%%%%%%%%%%%%%%%%%%%%%%%%%%%%%%%%%%%%%%%%%%%%%%%%%%%%%%%%
%%%%%%%%%%%%%%%%%%%%%%%%%%%%%%%%%%%%%%%%%%%%%%%%%%%%%%%%%%%%%%%%%%%%%%
%%%%%%%%%%%%%%%%%%%%%%%%%%%%%%%%%%%%%%%%%%%%%%%%%%%%%%%%%%%%%%%%%%%%%%
\end{document}
%%%%%%%%%%%%%%%%%%%%%%%%%%%%%%%%%%%%%%%%%%%%%%%%%%%%%%%%%%%%%%%%%%%%%%
%%%%%%%%%%%%%%%%%%%%%%%%%%%%%%%%%%%%%%%%%%%%%%%%%%%%%%%%%%%%%%%%%%%%%%
%%%%%%%%%%%%%%%%%%%%%%%%%%%%%%%%%%%%%%%%%%%%%%%%%%%%%%%%%%%%%%%%%%%%%%
%%%%%%%%%%%%%%%%%%%%%%%%%%%%%%%%%%%%%%%%%%%%%%%%%%%%%%%%%%%%%%%%%%%%%%
