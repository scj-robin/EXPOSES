\documentclass{beamer}
\usetheme{Madrid}
%\usetheme{CambridgeUS}

\usecolortheme[rgb={0.65,0.15,0.25}]{structure}

%\usefonttheme[onlymath]{serif}
\usepackage[french]{babel}
\usepackage[latin1]{inputenc}
\usepackage{color}
\usepackage{amsmath, amsfonts}
\usepackage{epsfig}
\usepackage{/Latex/astats}
%\usepackage[all]{xy}
%\usepackage{graphicx}
%\beamertemplatenavigationsymbolsempty

\newcommand{\emphase}[1]{\textcolor{blue}{#1}}
\newcommand{\paragraph}[1]{\noindent \emphase{#1}}

\AtBeginSubsection

%====================================================================
\title[CGH arrays]{UMR 518: CGH (\& ChIP)
  array analysis} 

%\subtitle[Visite AERES]{Visite du comit� d'�valuation de l'AERES}

\author[EL, SR, BT, MLM]{E. Lebarbier, S. Robin, B. Thiam +
  M.-L. Martin-Magniette} 

\institute[AgroParisTech / INRA]{AgroParisTech / INRA \\
  \bigskip
  \begin{tabular}{ccccc}
    \epsfig{file=../Figures/LogoINRA-Couleur.ps, width=2.5cm} &
    \hspace{.5cm} &
    \epsfig{file=../Figures/logagroptechsolo.eps, width=3.75cm} &
    \hspace{.5cm} &
    \epsfig{file=../Figures/Logo-SSB.eps, width=2.5cm} \\
  \end{tabular} \\
  \bigskip
  }

\date[Feb'09]{Saclay, February 2009}
%====================================================================

%====================================================================
%====================================================================
\begin{document}
%====================================================================
%====================================================================

%====================================================================
\frame{\titlepage}
%====================================================================

%====================================================================
\frame{ \frametitle{Outline}
%====================================================================
  
  \tableofcontents
%   \tableofcontents[pausesections]
  }

%====================================================================
\section{CGH arrays}
\frame{ \frametitle{CGH arrays}
%==================================================================== 
  \paragraph{Usual purpose}
  \begin{itemize}
  \item Evaluate the number of copy at each locus
  \item Detect copy number variations (CNV)
  \item Detect chromosomic aberrations
  \end{itemize}

  \bigskip
  \paragraph{AgriArray} \\
  {\sl A complete CGH protocol will be developed in order to
    characterize maize breeding lines; this protocol, baseyd on the
    Genoplante 57 K maize array and in a second time the results of
    WP1, will give access to gene duplication or deletion events
    between lines. Its utility as a global identification tool will
    also be evaluated, the genomic profiling being used as a plant
    barcode.}

  \bigskip
  \paragraph{Disparition of BioGemma} 
  \begin{itemize}
  \item No more precise biological question or demand
  \item \rightarrow Development of generic tools
  \end{itemize}

  }

%====================================================================
\frame{ \frametitle{}
%==================================================================== 
  }

%====================================================================
\section{Breakpoints in one profile}
\frame{ \frametitle{Breakpoints in one profile}
%==================================================================== 
  }

%====================================================================
\section{Regression model for multiple profile}
\frame{ \frametitle{Regression model for multiple profile}
%==================================================================== 
  }

%====================================================================
\section{Work in progress}
\subsection{CGH-seg package}
\frame{ \frametitle{CGH-seg package}
%==================================================================== 
  }

%====================================================================
\subsection{Looking for recurrent aberrations}
\frame{ \frametitle{Looking for recurrent aberrations}
%==================================================================== 
  }

%====================================================================
\subsection{Exploring the segmentation space}
\frame{ \frametitle{Exploring the segmentation space}
%==================================================================== 
  }

%====================================================================
\section{ChIP-chip}
\frame{ \frametitle{ChIP-chip}
%==================================================================== 
  }

%====================================================================
\frame{ \frametitle{References}
%==================================================================== 
  \nocite{PRL07}\nocite{PRL05}\nocite{RoS08}
  \bibliography{/biblio/AST}
%   \bibliograhystyle{/latex/astats}
  }

%====================================================================
%====================================================================
\end{document}
%====================================================================
%====================================================================

%====================================================================
\frame{ \frametitle{}
%==================================================================== 
  }

