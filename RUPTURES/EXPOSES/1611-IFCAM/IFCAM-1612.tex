\documentclass[10pt]{beamer}

% Beamer style
%\usetheme[secheader]{Madrid}
% \usetheme{CambridgeUS}
\useoutertheme{infolines}
\usecolortheme[rgb={0.65,0.15,0.25}]{structure}
% \usefonttheme[onlymath]{serif}
\beamertemplatenavigationsymbolsempty
%\AtBeginSubsection

% Packages
%\usepackage[french]{babel}
\usepackage[latin1]{inputenc}
\usepackage{color}
\usepackage{xspace}
\usepackage{dsfont, stmaryrd}
\usepackage{amsmath, amsfonts, amssymb}
\usepackage{epsfig}
\usepackage{tikz}
\usepackage{url}
\usepackage{/home/robin/LATEX/Biblio/astats}
%\usepackage[all]{xy}
\usepackage{graphicx}

% Commands
% TikZ
\newcommand{\nodesize}{1.8em}
\newcommand{\edgeunit}{2.2*\nodesize}
\tikzstyle{hidden}=[draw, circle, fill=gray!50, minimum width=\nodesize, inner sep=0]
\tikzstyle{observed}=[draw, circle, minimum width=\nodesize, inner sep=0]
\tikzstyle{eliminated}=[draw, circle, minimum width=\nodesize, color=gray!50, inner sep=0]
\tikzstyle{empty}=[]
\tikzstyle{arrow}=[->, >=latex, line width=1pt]
\tikzstyle{edge}=[-, line width=1pt]
\tikzstyle{dashedarrow}=[->, >=latex, dashed, line width=1pt]
\tikzstyle{lightarrow}=[->, >=latex, line width=1pt, fill=gray!50, color=gray!50]


\definecolor{darkred}{rgb}{0.65,0.15,0.25}
\newcommand{\emphase}[1]{\textcolor{darkred}{#1}}
% \newcommand{\emphase}[1]{{#1}}
\newcommand{\paragraph}[1]{\textcolor{darkred}{#1}}
\newcommand{\refer}[1]{{{\textcolor{blue}{{[\cite{#1}]}}}}}
\newcommand{\Refer}[1]{{{\textcolor{blue}{{[#1]}}}}}
\renewcommand{\newblock}{}

% Symbols
\newcommand{\Abf}{{\bf A}}
\newcommand{\Beta}{\text{B}}
\newcommand{\Bcal}{\mathcal{B}}
\newcommand{\BIC}{\text{BIC}}
\newcommand{\Ccal}{\mathcal{C}}
\newcommand{\dd}{\text{~d}}
\newcommand{\dbf}{{\bf d}}
\newcommand{\Dcal}{\mathcal{D}}
\newcommand{\Esp}{\mathbb{E}}
\newcommand{\Ebf}{{\bf E}}
\newcommand{\Ecal}{\mathcal{E}}
\newcommand{\Gcal}{\mathcal{G}}
\newcommand{\Gam}{\mathcal{G}\text{am}}
\newcommand{\Hcal}{\mathcal{H}}
\newcommand{\Ibb}{\mathbb{I}}
\newcommand{\Ibf}{{\bf I}}
\newcommand{\ICL}{\text{ICL}}
\newcommand{\Cov}{\mathbb{C}\text{ov}}
\newcommand{\Corr}{\mathbb{C}\text{orr}}
\newcommand{\Var}{\mathbb{V}}
\newcommand{\Vsf}{\mathsf{V}}
\newcommand{\pen}{\text{pen}}
\newcommand{\Fcal}{\mathcal{F}}
\newcommand{\Hbf}{{\bf H}}
\newcommand{\Jcal}{\mathcal{J}}
\newcommand{\Kbf}{{\bf K}}
\newcommand{\Lcal}{\mathcal{L}}
\newcommand{\Mcal}{\mathcal{M}}
\newcommand{\mbf}{{\bf m}}
\newcommand{\mum}{\mu(\mbf)}
\newcommand{\Ncal}{\mathcal{N}}
\newcommand{\Nbf}{{\bf N}}
\newcommand{\Nm}{N(\mbf)}
\newcommand{\Ocal}{\mathcal{O}}
\newcommand{\Obf}{{\bf 0}}
\newcommand{\Omegas}{\underset{s}{\Omega}}
\newcommand{\Pbf}{{\bf P}}
\newcommand{\Pt}{\widetilde{P}}
\newcommand{\Pcal}{\mathcal{P}}
\newcommand{\Qcal}{\mathcal{Q}}
\newcommand{\Rbb}{\mathbb{R}}
\newcommand{\Rcal}{\mathcal{R}}
\newcommand{\Scal}{\mathcal{S}}
\newcommand{\Tcal}{\mathcal{T}}
\newcommand{\Ucal}{\mathcal{U}}
\newcommand{\Vcal}{\mathcal{V}}
\newcommand{\BP}{\text{BP}}
\newcommand{\EM}{\text{EM}}
\newcommand{\VEM}{\text{VEM}}
\newcommand{\VBEM}{\text{VBEM}}
\newcommand{\cst}{\text{cst}}
\newcommand{\obs}{\text{obs}}
\newcommand{\ra}{\emphase{\mathversion{bold}{$\rightarrow$}~}}
%\newcommand{\transp}{\text{{\tiny $\top$}}}
\newcommand{\transp}{\text{{\tiny \mathversion{bold}{$\top$}}}}
\newcommand{\logit}{\text{logit}\xspace}

% Directory
\newcommand{\fignet}{/home/robin/Bureau/RECHERCHE/RESEAUX/EXPOSES/FIGURES}
\newcommand{\figchp}{/home/robin/Bureau/RECHERCHE/RUPTURES/EXPOSES/FIGURES}


%====================================================================
%====================================================================

%====================================================================
%====================================================================
\begin{document}
%====================================================================
%====================================================================

%====================================================================
\title[Detecting change-points in the structure of a network]{Detecting change-points in the structure of a network: Exact Bayesian inference}

\author[S. Robin]{L. Schwaller, S. Robin \\ ~\\
    {\small (and also A. Cleynen, E. Lebarbier, G. Rigaill, M. Stumpf)}
  }

\institute[INRA / AgroParisTech]{%INRA / AgroParisTech \\
%   \vspace{-.2\textheight}
  \begin{tabular}{ccccc}
    \includegraphics[height=.2\textheight]{\fignet/LogoINRA-Couleur} & 
    \hspace{.02\textheight} &
    \includegraphics[height=.05\textheight]{\fignet/logagroptechsolo} & 
  \end{tabular} 
  }

\date[Nov. 2016, Bengalore]{IFCAM workshop, Nov. 2016, Bengalore}

%====================================================================
%====================================================================
\maketitle
%====================================================================

%====================================================================
%====================================================================
\section*{Motivating example}
%====================================================================
\frame{\frametitle{Example: Gene regulatory network along time} 

  \paragraph{Data:} 
  $$
  Y_{jt} = \text{expression of gene $j$ at time $t$}
  $$
  
  \bigskip \pause
  \paragraph{'Model':} 
  \begin{eqnarray*}
    G_t & = & \text{gene regulatory network at time $t$} \\
    & = & \text{graphical model of $(Y_{jt})_j$}
  \end{eqnarray*}

  \bigskip \pause
  \paragraph{Questions:} 
  \begin{itemize}
   \item Is $G_t$ constant along time or is there some 'gene rewiring'?
   \item If not, when does it change?
   \item And what is the network within each period?
  \end{itemize}
}

%====================================================================
\frame{\frametitle{Example of output} 

  \paragraph{Data:} $N = 67$ time points, $p = 11$ genes, four expected regions

  \pause \bigskip
  Posterior probability of change-points:
  $$
  \includegraphics[width=.7\textwidth]{\figchp/SrS16-Fig8-chgpt_EMP_ap10_au1}
  $$

  \pause
  Inferred networks:
  $$
  \includegraphics[width=.8\textwidth]{\figchp/SrS16-Fig9-network_seuil02}
  $$

}

%====================================================================
%====================================================================
\section{Bayesian inference with discrete parameters}
\frame{\frametitle{Outline} \tableofcontents[currentsection]}
%====================================================================
\frame{\frametitle{A reminder on Bayesian inference} 

  \paragraph{Parametric inference.} A statistical model describes how the distribution of the observed data $Y$ depends on a parameter of interest:
  $$
  Y \sim p(\cdot; \theta)
  $$

  \pause \bigskip
  \paragraph{'Classical' or 'frequentist' framework:} $\theta$ supposed to be fix.
  \begin{itemize}
   \item Try to provide a estimate $\widehat{\theta}$ not too far from the true $\theta^*$.
  \end{itemize}

  \pause \bigskip \bigskip
  \paragraph{Bayesian framework:} $\theta$ is random with  \emphase{\sl prior} distribution $p(\theta)$.
  \begin{itemize}
   \item Try to establish its conditional distribution given the data: 
   $$
   p(\theta|Y) = \frac{p(\theta) p(Y|\theta)}{p(Y)}
   \qquad = \text{\emphase{\sl posterior} distribution.}
   $$
  \end{itemize}

}

% %====================================================================
% \frame{\frametitle{Illustration: Beta binomial (1/2)}
% 
%   \paragraph{Aim:} Infer a success probability $\theta$ based on $n$ independent trials.
%   
%   \bigskip
%   \begin{itemize}
%    \item \pause Define a prior distribution $p(\theta)$, e.g.
%    $$
%    \theta \sim \Beta(a, b)
%    $$
%    \item \pause Write the conditional distribution of $Y =$ number of success:
%    $$
%    Y|\theta \sim \Bcal(n, \theta)
%    $$
%    \item \pause Deduce the conditional distribution of $\theta$ given $Y$ (posterior):
%    $$
%    p(\theta|Y=y) = \frac{p(\theta) P(Y=y | \theta)}{P(Y=y)}
%    $$
%    where $P(Y=y) = \int P(Y=y|\theta) p(\theta) \dd \theta$
%   \end{itemize} \pause 
%   In this case we get
%   $$
%   (\theta|Y=y) \sim \Beta(\widetilde{a}, \widetilde{b}): \qquad \widetilde{a}=a+y, \quad
%   \widetilde{b}=b+n-y
%   $$
% 
% }
% 
% %====================================================================
% \frame{\frametitle{Illustration: Beta binomial (2/2)}
% 
%   \begin{tabular}{cc}
%     \begin{tabular}{p{.5\textwidth}}
%       \begin{enumerate}
%        \onslide+<2->{\item Define a (flat) prior for $\theta$}
%        \onslide+<3->{\item Get the data ($n = 20$, $y = 4$)}
%        \onslide+<4->{\item Compute the posterior distribution}
%        \onslide+<5->{\item Compute, e.g. a credibility interval: $CI_{95\%}(\theta|Y)$.}
%       \end{enumerate}
%     \end{tabular}
%     & 
%     \hspace{-.125\textwidth}
%     \begin{tabular}{p{.5\textwidth}}
%       \begin{overprint}
%        \onslide<2>\includegraphics[width=.5\textwidth]{\figchp/Fig-BetaBinom-Prior1}
%        \onslide<3>\includegraphics[width=.5\textwidth]{\figchp/Fig-BetaBinom-Prior1-Data}
%        \onslide<4>\includegraphics[width=.5\textwidth]{\figchp/Fig-BetaBinom-Prior1-Data-Posterior1}
%        \onslide<5>\includegraphics[width=.5\textwidth]{\figchp/Fig-BetaBinom-Prior1-Data-ConfInter1}
%        \onslide<6>\includegraphics[width=.5\textwidth]{\figchp/Fig-BetaBinom-Prior2}
%        \onslide<7>\includegraphics[width=.5\textwidth]{\figchp/Fig-BetaBinom-Prior2-Data-Posterior2}
%       \end{overprint}
%     \end{tabular}
%   \end{tabular}
% 
%   \onslide<6->{A different prior} \onslide<7->{gives in a different posterior.} 
% }
% 
% 
%====================================================================
\frame{\frametitle{Bayesian inference} 

  \paragraph{Generic Bayesian framework:}
  \begin{align*}
   \text{prior:} & \quad p(\theta) \\
   \text{likelihood:} & \quad p(Y|\theta) \qquad \text{(given by the model)}\\
   \text{\ra posterior:} & \quad p(\theta|Y)
  \end{align*}
  
  \bigskip \pause
  \paragraph{A typical issue:} Deriving
  $$
  p(\theta|Y) = {p(\theta) p(Y|\theta)} \ / \ {p(Y)}
  $$
  is often not trivial, typically because
  $$
  p(Y) = \int_\Theta p(Y|\theta) p(\theta) \dd \theta
  $$
  is intractable when the parameter space $\Theta$ is huge.
}
%====================================================================
\frame{\frametitle{Posterior distribution} 

  \paragraph{3 main approaches}
  \begin{enumerate}
   \item \pause {Sampling} (Monte Carlo, Monte Carlo - Markov chain, sequential MC, Importance sampling, ...): 
   $$
   \text{sample} \quad (\theta^b) \sim p(\theta | Y).
   $$ \\~ 
   \item \pause {Approximation} (e.g. Variational Bayes, Expectation propagation, ...): 
   $$
   \text{find} \qquad \widetilde{p}_Y(\theta) \simeq p(\theta | Y).
   $$ \\~ 
   \item \pause {Exact:} 
   $$
   \text{actually compute} \qquad p(\theta|Y)
   $$
   or some marginals of interest.
  \end{enumerate}
  }

%====================================================================
\frame{\frametitle{Models with discrete parameters}

  \paragraph{Mixed parameter:} $ \theta \rightarrow (\theta, T)$
  $$
  \theta \in \Theta = \text{continuous set}, 
  \qquad
  T \in \Tcal = \text{discrete (countable) set}, 
  $$
  $$
  \Rightarrow \qquad p(Y) = \emphase{\sum_{T \in \Tcal}} \int_\Theta p(Y, \theta, T) \dd \theta
  $$

  \bigskip \bigskip \pause
  \paragraph{Size of $\Tcal$.}
  \begin{itemize}
   \item No big deal of $\Tcal$ is small (e.g. model selection within a small collection).
%    $$
%    p(T|Y) = \int_\Theta p(Y, \theta, T) \dd \theta \left/ \sum_{T' \in \Tcal} \int_\Theta p(Y, \theta, T') \dd \theta \right.
%    $$
  \\ ~
   \item Big issue if $|\Tcal|$ grows (super-)exponentially with the number of observations $n$ or the number of variables $p$.
  \end{itemize}

}

%====================================================================
\frame{\frametitle{Main issue}

  Suppose that the integration wrt $\theta$ raise no issue\footnote{Using e.g. conjugate priors.}, the calculation of 
  $$
  \sum_{T \in \Tcal}
  $$
  can often not be achieved in a naive way because of the combinatorial complexity\footnote{The frequentist counterpart often raises similar issues.}. 
  \begin{center}
   \ra Need to find algorithmic or algebraic shortcuts
  \end{center}
  
  \bigskip \pause
  \paragraph{Examples.}
  \begin{itemize}
   \item Change-point detection \\~
   \item 'Network inference' = inference of the structure of a graphical model 
  \end{itemize}
  }

%====================================================================
%====================================================================
\section{Change-point detection}
\frame{\frametitle{Outline} \tableofcontents[currentsection]}

%====================================================================
\subsection*{Change-point detection model}
%====================================================================
\frame{\frametitle{A change-point detection model} 
  \begin{tabular}{cc}
    \hspace{-.5cm}
    \begin{tabular}{p{.5\textwidth}}
      \onslide+<1->{\paragraph{Model.} 
        \begin{itemize}
        \item \onslide+<2->{$K$ segments \\~}
        \item \onslide+<3->{$T = (\tau_k)_k$ change points\\~\\
        $r_k = \llbracket\tau_{k-1}+1; \tau_k\rrbracket$\\~}
        \item \onslide+<4->{$\theta = (\theta_k)_k$ parameters\\~}
        \item \onslide+<5->{$Y = (Y_t)_{1 \leq t \leq n}$ observed data \\~\\
        $Y^r = (Y_t)_{t \in r}$}
        \end{itemize}}
    \end{tabular}
    & 
    \hspace{-1cm}
    \begin{tabular}{c}
	 \hspace{-6cm}
	 \onslide+<5->{$\{Y^r\}_r \text{ indep}, \quad Y^r \sim p(\cdot | \theta_r)$}
	 \\
      \begin{overprint}
        \onslide<2>
        \includegraphics[width=.5\textwidth]{\figchp/FigSeg-Budapest-1} 
        \onslide<3>
        \includegraphics[width=.5\textwidth]{\figchp/FigSeg-Budapest-2} 
        \onslide<4>
        \includegraphics[width=.5\textwidth]{\figchp/FigSeg-Budapest-3} 
        \onslide<5>
        \includegraphics[width=.5\textwidth]{\figchp/FigSeg-Budapest-4} 
        \onslide<6->
        \includegraphics[width=.5\textwidth]{\figchp/FigSeg-Budapest-0} 
      \end{overprint}
    \end{tabular}
  \end{tabular}

  \onslide+<6->{
  \paragraph{Bayesian version:} on the top of this, add
  $
  p(K), p(T|K), p(\theta|K).
  $
  } 

}

%====================================================================
\subsection*{Maximum likelihood inference}
\frame{\frametitle{Maximum likelihood inference (1/2)} 

  \paragraph{Log-likelihood:}
  $$
  \log p(Y; \theta, T) = \sum_{r \in T} \log p(Y^r; \theta^r) 
  $$
  
  \pause
  \paragraph{Inference}
  \begin{itemize}
   \item continuous part ($\theta$):
   $$
   \widehat{\theta}_r = \arg\max_{\theta_r} \; \log p(Y^r; \theta^r)
   \qquad \text{\ra standard MLE}
   $$
   \item \pause discrete part ($T$):
   $$
   \widehat{T} = \arg\max_{T} \; \sum_{r \in T} \log p(Y^r; \widehat{\theta}^r)
   = \arg\max_{T} \; \sum_{r \in T} \log \widehat{p}(Y^r)
   $$
   \ra discrete optimization problem
   \end{itemize}
}

%====================================================================
\frame{\frametitle{Maximum likelihood inference (2/2)} 

  \paragraph{Segmentation space} $\Tcal = \Tcal_{1:n}^K =$ set of all possible segmentations of $\llbracket 1; n \rrbracket$ with $K$ segments:
  $$
  |\Tcal| = \binom{n-1}{K-1} \approx \left(\frac{n}{K}\right)^K
  $$
  \ra exhaustive search is prohibited.

  \bigskip \pause
  \paragraph{Dynamic programming} allows to retrieve $\widehat{T}$ \refer{AuL89} using 
  $$
  \max_{T \in \Tcal_{1:j}^K} \sum_{r \in T} \log \widehat{p}(Y^r)
  =
  \max_{K-1 \leq i < j} \left(\max_{T \in \Tcal_{1:i}^{K-1}} \sum_{r \in T} \log
  \widehat{p}(Y^r)\right) + \log \widehat{p}(Y^{\llbracket i+1; j\rrbracket})
  $$

  \bigskip \bigskip \pause
  \paragraph{Further inference} is hard to carry out \\
  \ra Standard likelihood theory does not apply to discrete parameters \\
  \qquad (no simple confidence intervals for the $\tau_k$). \\
  \ra Bayesian inference can circumvent some difficulties.
  }
%====================================================================
\subsection*{Bayesian inference}
\frame{\frametitle{Bayesian inference} 

  \paragraph{Factorability assumptions} 
  \begin{itemize}
  \item Prior distribution for the segmentation:
  $$
  p(T|K) = \prod_{r \in T} a_r , \qquad \text{e.g. } a_r = n_r^\alpha
  $$
  ~
  \item Independent parameters in each segment:
  $$
  p(\theta|T) = \prod_{r \in T} p(\theta_r)
  $$
  ~
  \item Data are independent from one segment to another
  $$
  p(Y | T, \theta) = \prod_{r \in T} p(Y^r | \theta_r) 
  $$
  \end{itemize}

}

%====================================================================
\frame{ \frametitle{Some quantities of interest}

  \paragraph{Marginal likelihood.}
  $$
  p(Y|K) = \sum_{T \in \Tcal^K} \int p(Y, \theta, T | K) \dd \theta \propto \sum_{T \in \Tcal^K} \prod_{r \in T} a_r p(Y^r)
  $$
  where $p(Y^r) = \int p(Y^r|\theta_r) p(\theta^r) \dd \theta_r$ 
%   (supposed to be easy to compute using e.g. conjugate priors) 
  and the normalizing constant is
  $$
  \sum_{T \in \Tcal^K} \prod_{r \in T} a_r.
  $$

  \pause \bigskip
  \paragraph{Posterior distribution of a change-point.}
  $$
  \Pr\{\tau_k = t | Y, K\} \propto \left(\sum_{T \in \Tcal^k_{1:t}} \prod_{r \in T} a_r p(Y^r) \right) \left(\sum_{T \in \Tcal^{K-k}_{t+1:n}} \prod_{r \in T} a_r p(Y^r) \right)
  $$

}

%====================================================================
\frame{ \frametitle{Summing over segmentations \refer{RLR11}}

  \paragraph{Property:} {\sl 
%   To compute
%   $$
%   \sum_{T \in \Tcal^K_{1:n}} \prod_{r \in T} f_r,
%   $$
  Define the upper triangular $(n+1) \times (n+1)$ matrix $A$:
  $$
  A_{i, j+1} = f_r \qquad \text{for } r = \llbracket i, j \rrbracket
  $$ \pause
  Then 
  $$
  \left[A^K\right]_{1,n+1} = \sum_{T \in \Tcal^K_{1:n}} \prod_{r \in T} f_r
  $$
  \ra all terms are computed in {$O(K n^2)$}. }
  
  \bigskip
  \begin{itemize}
   \item \pause To compute $p(Y)$, take $f_r = a_r p(Y^r)$. 
   \item \pause 'sum-product' = counterpart of 'max-sum' in the dynamic programming algorithm.
   \item \pause Similar ideas in \refer{Fea06}. 
  \end{itemize}
  
  \bigskip \pause
  \ra R package EBS (exact Bayesian segmentation) \refer{ClR14}

  }

%====================================================================
\frame{\frametitle{Illustration: Of exons, introns and UTR's}
  
  Regions for a same gene are not adjacent along the genome 
  $$
  \includegraphics[width=.8\textwidth]{\figchp/Pre-mRNA_to_mRNA}
  $$
  \Refer{Wikipedia}
  
  \bigskip \bigskip \pause
  \begin{itemize}
   \item The transcribed regions are made of both exons and untranslated regions (UTR)
   \item Alternative splicing: some exons can be skipped or the boundaries may vary.
  \end{itemize}

  }

%====================================================================
\frame{\frametitle{Posterior distribution of transcript boundaries in yeast}

  \vspace{-.05\textheight}
  \begin{tabular}{p{.2\textwidth}p{.7\textwidth}}
    \begin{tabular}{p{.3\textwidth}}
	 \paragraph{RNA-seq data:} \\
	 \\ ~\\
	 One gene \\
	 \\
	 $\times$ \\
	 \\
	 Three growth \\
	 conditions \\ 
	 $A$, $B$, $C$
    \end{tabular}
    &
    \begin{tabular}{p{.7\textwidth}}
    \includegraphics[width=.7\textwidth]{\figchp/compyeastresult.pdf}
    \end{tabular}
  \end{tabular}
}

%====================================================================
\frame{\frametitle{Comparing change-point locations \refer{ClR14}}

  \paragraph{One series.} We know how to compute (in $O(Kn^2)$)
  $$
  \Pr\{\tau_k = t | Y, K\} \qquad \text{or} \qquad \Pr\{\tau_k = t | Y\}.
  $$
  
  \bigskip \bigskip \pause
  \paragraph{Two series ($Y^A, Y^B$):} Consider the shift of the $k$th change-point 
  $$
  \Pr\{\tau^A_k - \tau^B_k = 0 | Y^A, Y^B, K^A, K^B\}
  $$
  
  \bigskip \bigskip \pause
  \paragraph{$I$ series ($Y^A, \dots Y^I$):} Check if the $k$th change-point is conserved\footnote{Requires a probability change, as $Y^A, \dots Y^I$ are not independent conditionally on $\tau^A_k = \dots = \tau^I_k$.}:
  $$
  \Pr\{\tau^A_k = \dots = \tau^I_k | Y^A, \dots Y^I, K^A, \dots, K^I\}
  $$
  
}

%====================================================================
\frame{\frametitle{Boundary shifts between conditions}

  3 comparisons ($A/B$, $A/C$, $B/C$) $\times$ 4 change points:
  
\centerline{\includegraphics[width=.8\textwidth, height=.8\textheight]{\figchp/cred-yeast.pdf}}

}

%====================================================================
\frame{\frametitle{Comparing transcript boundaries}

Setting $\Pr\{\tau^A_k = \tau^B_k | K\} = 1/2$.

$$
\begin{array}{lccccc}
& \tau_1 & \tau_2 & \tau_3 & \tau_4 \\ 
\hline \\
\Pr\{\tau^A_k = \tau^B_k | Y, K\} & 0.32 & 0.30 &0.99 & 10^{-5} \\ \\
\Pr\{\tau^A_k = \tau^C_k | Y, K\} & 4 \; 10^{-4} & 0.99 &0.99 & 6 \; 10^{-3} \\ \\
\Pr\{\tau^B_k = \tau^C_k | Y, K\} & 5 \; 10^{-2} & 0.60 & 0.99 & 0.99 \\ \\
\Pr\{\tau^A_k = \tau^B_k = \tau^C_k | Y, K\} & 10^{-3} & 0.99  & 0.99 & 6 \; 10^{-3} \\ 
\end{array}
$$

\bigskip
\ra Differences at the UTR's end but not at internal exon boundaries.
}

%====================================================================
\frame{\frametitle{Various isoforms in yeast?} 

  \paragraph{$\Pr\{\tau^A_k = \tau^B_k = \tau^C_k | Y, K\}$} for all yeast genes with 2 expressed exons
  $$
  \begin{tabular}{cc}
  \includegraphics[width=.4\textwidth]{\figchp/statall-all} 
  & 
  \includegraphics[width=.4\textwidth]{\figchp/statall2} 
  \\
   $p_0 = (.5, \;.5, \;.5, \;.5)$
   &
   $p_0 = (.9, \;.99, \;.99, \;.9)$
  \end{tabular}
  $$
}

%====================================================================
%====================================================================
\section{Network inference}
\frame{\frametitle{Outline} \tableofcontents[currentsection]}
%====================================================================
\subsection*{Graphical model framework}
\frame{\frametitle{Graphical model framework} 

  \paragraph{Property [Hammersley-Clifford].} The joint distribution $p(Y) = p(Y_1, \dots Y_p)$ is Markov wrt the (decomposable) graph $G$ iff it factorizes wrt the maximal cliques of $G$:
  $$
  p(Y) \propto \prod_{C \in \Ccal(G)} \psi_c(Y^c), 
  \qquad Y^c = (Y_j)_{j \in C}.
  $$
  
  \bigskip 
  \ra $G$ reveals the structure of conditional independences between the variables $Y_1, \dots Y_p$.
  
}

%====================================================================
\frame{\frametitle{Graphical model} 

  \begin{tabular}{cc}
    \begin{tabular}{p{.5\textwidth}}
      \begin{tikzpicture}
      \node[observed] (Y1) at (0, 0) {$Y_1$};
      \node[observed] (Y2) at (\edgeunit, 0) {$Y_2$};
      \node[observed] (Y3) at (.5*\edgeunit, .87*\edgeunit) {$Y_3$};
      \node[observed] (Y4) at (-.5*\edgeunit, -.87*\edgeunit) {$Y_4$};
      \node[observed] (Y5) at (.5*\edgeunit, -.87*\edgeunit) {$Y_5$};
      \node[observed] (Y8) at (-.5*\edgeunit, .87*\edgeunit) {$Y_8$};
      \node[observed] (Y6) at (1.5*\edgeunit, -.87*\edgeunit) {$Y_6$};
      \node[observed] (Y7) at (1.5*\edgeunit, -1.87*\edgeunit) {$Y_7$};
      
      \draw[edge] (Y1) to (Y2);  \draw[edge] (Y1) to (Y3);  \draw[edge] (Y2) to (Y3);
      \draw[edge] (Y1) to (Y4);  \draw[edge] (Y1) to (Y5);
      \draw[edge] (Y2) to (Y6);  \draw[edge] (Y6) to (Y7);
      \draw[edge] (Y3) to (Y8);
      \end{tikzpicture}
    \end{tabular}
    & 
    \hspace{-.2\textwidth}
    \begin{tabular}{p{.5\textwidth}}
    Means that
    \begin{eqnarray*}
      p(Y_1, \dots, Y_8) & \propto & \psi_1(Y_1, Y_2, Y_3) \\
      & \times & \psi_2(Y_1, Y_4) \ \psi_3(Y_1, Y_5) \\
      & \times & \psi_4(Y_2, Y_6) \ \psi_5(Y_6, Y_7) \\
      & \times & \psi_6(Y_3, Y_8)
    \end{eqnarray*}
    which implies\footnote{Under fairly general assumptions on $p$} that
    \begin{eqnarray*}
     Y_4 \perp Y_3 & | & Y_1 \\
     (Y_6, Y_7) \perp Y_3 & | & Y_2 \\
     \dots
    \end{eqnarray*}
    \end{tabular}
  \end{tabular}
  
  \pause
  \paragraph{'Network inference' problem:} Based on $\{(Y_{i1}, \dots Y_{ip})\}_i$ iid $\sim p$, infer $G$.

}

%====================================================================
\frame{\frametitle{Tree-structured network} 

  Suppose the graph $G$ is a tree $T$, $p(Y)$ is Markov wrt $T$ iff
  \begin{eqnarray*}
   p(Y|\theta) 
   & = & \prod_j p(Y_j|\theta_j) \prod_{(j, k) \in T} \frac{p(Y_j, Y_k|\theta_{jk})}{p(Y_j|\theta_j)p(Y_k|\theta_k)} \\
   & = & \prod_{(j, k) \in T} p(Y_j, Y_k|\theta_{jk}) \left/ \prod_j p^{d_j-1}(Y_j|\theta_j) \right.
  \end{eqnarray*}
  where $d_j$ is the degree (number of neighbors in $T$) of node $j$.
  
  \bigskip \bigskip \pause
  \paragraph{Tree structure assumption.}
  \begin{itemize}
   \item Consistent with the usual assumption that the graph is sparse (although much stronger). \\~
   \item Not true in general, but may be sufficient for the \emphase{inference on local structures}, such as the existence of a given edge.
  \end{itemize}

}

%====================================================================
\subsection*{Maximum likelihood inference}
\frame{\frametitle{Maximum likelihood inference (1/2)} 

  \paragraph{Log-likelihood.}
  \begin{eqnarray*}
    \log p(Y; \theta, T) & = & \sum_{(j, k) \in T} \log p(Y_j, Y_k|\theta_{jk}) - \sum_j ({d_j-1}) \log p(Y_j|\theta_j) \\
    & = & \sum_j \log p(Y_j|\theta_j) + \sum_{(j, k) \in T} \log \frac{p(Y_j, Y_k|\theta_{jk})}{p(Y_j|\theta_j) p(Y_j|\theta_j)}
  \end{eqnarray*}
  
  \bigskip \pause
  \paragraph{Inference:}
  \begin{itemize}
   \item continuous part ($\theta$): MLE
   $$
   \widehat{\theta}_j = \arg\max_{\theta_j} \log p(\{Y_{ij}\}_i; \theta_j), 
   \qquad
   \widehat{\theta}_{jk} = \arg\max_{\theta_{jk}} \log p(\{(Y_{ij}, Y_{ik})\}_i; \theta_{jk})
   $$
   \item \pause discrete part ($T$)
   $$
   \widehat{T} = \arg\max_T \sum_{(j, k) \in T} \log \frac{p(Y_j, Y_k|\widehat{\theta}_{jk})}{p(Y_j|\widehat{\theta}_j)p(Y_k|\widehat{\theta}_k)}  $$
  \end{itemize}
}

%====================================================================
\frame{\frametitle{Maximum likelihood inference (2/2)} 

  \paragraph{Chow \& Liu algorithm \refer{ChL68}:} Taking
  $$
  f_{jk} = \log \frac{p(Y_j, Y_k|\widehat{\theta}_{jk})}{p(Y_j|\widehat{\theta}_j) \ p(Y_j|\widehat{\theta}_j)}
  $$
  as the weight of edge $(j, k)$, 
  $$
  \widehat{T} = \arg\max_T \sum_{(j, k) \in T} f_{jk}
  $$
  is the \emphase{maximum spanning tree} with weights $\{f_{jk}\}$, which can be retrieved by Kruskal's algorithm in $O(p^2)$ \refer{Kru56}.
  
  \pause \bigskip \bigskip
  Retrieves the maximum likelihood tree but with no measure of uncertainty. \\
  \ra Exploring the whole tree space allows to evaluate uncertainty. \\
  \ra Bayesian inference can again be a solution.
  }
  
%====================================================================
\subsection*{Bayesian inference}
\frame{\frametitle{Bayesian setting \refer{SRS15}} 

  \bigskip 
  \hspace{-.02\textwidth}\begin{tabular}{lrlcrl}
  \paragraph{Model: \qquad } 
  & prior on $T$: & $\quad p(T) $ \\
  & prior on $\theta$: & $\quad p(\theta|T)$ & \ra & posterior: & $\quad p(T|Y)$ \\
  & likelihood: & $\quad p(Y|\theta, T)$
  \end{tabular}
  
  \bigskip \bigskip \pause
  \paragraph{Prior on $T$:} factorizes over the edges:
  $$
  p(T) \propto \prod_{(j, k) \in T} a_{jk}
  $$
  
  \bigskip \pause
  \paragraph{Prior on $\theta$:} displays factorability properties, i.e. needs to satisfy
  $$
  p(\theta_{jk}|T) \equiv p(\theta_{jk}) \quad \text{for all } T \ni (j, k).
  $$ 
  \ra Compatible family of strong Markov hyper-distributions \refer{DaL93}:	multinomial-Dirichlet (conjugacy), normal-Wishart (conjugacy), Gaussian copulas (numerical integration), ...?

}

%====================================================================
\frame{\frametitle{Quantities of interest}

  \paragraph{Marginal distribution.}
  $$
  p(Y) \propto 
%   \sum_{T \in \Tcal} \prod_{j, k} \frac{a_{jk} \int p(Y_j, Y_k, \theta_{jk}) \dd \theta_{jk}}{\int p(Y_j, \theta_j) \dd \theta_j \times \int p(Y_k, \theta_k) \dd \theta_k}
  \sum_{T \in \Tcal} \prod_{j, k} \frac{a_{jk} p(Y_j, Y_k)}{p(Y_j) p(Y_k)}
  $$
  where $\Tcal$ stands for the set of all spanning trees and $p(Y_j)$, $p(Y_j, Y_k)$ are intergals wrt $\theta_j$ and $\theta_{j, k}$, resp. 
  
  \bigskip \pause
  \paragraph{Posterior probability for an edge to be absent.}
  $$
  \Pr\{(j, k) \notin T |Y\} \propto 
%   \sum_{T \in \Tcal: (j, k) \notin T} \prod_{j, k} \frac{a_{jk} \int p(Y_j, Y_k, \theta_{jk}) \dd \theta_{jk}}{\int p(Y_j, \theta_j) \dd \theta_j \times \int p(Y_k, \theta_k) \dd \theta_k}
  \sum_{T \in \Tcal: (j, k) \notin T} \prod_{j, k} \frac{a_{jk} p(Y_j, Y_k)}{p(Y_j) p(Y_k) }
  $$

  \bigskip \bigskip \pause
  \paragraph{Typical form:} 
  $$\displaystyle{\sum_{T \in \Tcal} \prod_{(j, k) \in T} f_{jk}},$$
  with \emphase{cardinality of $\Tcal = p^{p-2}$}.


}

%====================================================================
\frame{\frametitle{Summing over spanning trees} 

  \paragraph{Matrix-tree theorem.} \refer{Cha82} {\sl
  \begin{itemize}
   \item $F = [f_{jk}]$: a symmetric matrix with $f(j, j) = 0, f_{jk} > 0$;
   \item $\Delta = [\Delta_{jk}]$ its Laplacian: $\Delta_{jj} = \sum_k f_{jk}, \Delta_{jk} = -f_{jk}$. 
  \end{itemize} \pause
  Then the minors $|\Delta^{uv}|$ of $\Delta$ are equal and
  $$
  |\Delta^{uv}| = \sum_{T \in \Tcal} \prod_{(j, k) \in T} f_{jk}.
  $$}
  
  \pause \bigskip
  \begin{itemize}
   \item Can be used to compute $p(Y)$, the normalizing constant of $p(T)$, ... at the cost of computing a $p \times p$ determinant.
   \item Already used in \refer{MeJ06} for tree learning.
   \item Again 'sum-product' in place of 'max-sum'.
  \end{itemize}
}

%====================================================================
\frame{\frametitle{Posterior probability of an edge} 

  The existence of an edge between variables $Y_j$ and $Y_k$ can be assessed by
  $$
  \Pr\{(j, k) \in T | Y\} \propto \sum_{T \ni (j, k)} p(T) p(Y|T)
  $$
  which depends on the prior $p(T)$.
  
  \bigskip \bigskip
  The prior probability $\Pr\{(j, k) \in T\}$ can be tuned
  \begin{itemize}
   \item with the prior coefficient $a_{jk}$ 
   \item or set to an arbitrary value using an edge-specific probability change.
  \end{itemize}
  

  \pause \bigskip \bigskip
  All posterior probabilities can be computed in $O(p^3)$. \\
  \ra R package Saturnin (spanning trees used for network inference) \refer{SRS15}

}

%====================================================================
\frame{\frametitle{Simulations: ROC curves for edge detection} 

For various graph topologies ($p=25$, $n = 25, 50, 200$, $B = 100$ simulations) \\
\begin{tabular}{cccc}
\includegraphics[width=0.2\linewidth]{\fignet/roc_curves_tree_multinomial.pdf}
& \includegraphics[width=0.2\linewidth]{\fignet/roc_curves_ER2p_multinomial.pdf}
& \includegraphics[width=0.2\linewidth]{\fignet/roc_curves_ER4p_multinomial.pdf}
& \includegraphics[width=0.2\linewidth]{\fignet/roc_curves_ER8p_multinomial.pdf} \\
Tree & Erd\"{o}s-R\'{e}nyi & Erd\"{o}s-R\'{e}nyi & Erd\"{o}s-R\'{e}nyi \\
& $p_c = 2/p$ & $p_c = 4/p$ &  $p_c = 8/p$ 
\end{tabular}

}

%====================================================================
\frame{\frametitle{Simulations: Comparison with sampling among DAGs} 

\refer{NPK11}: MCMC sampling over the directed acyclic graphs (multinomial case) \\~

\begin{tabular}{cccc}
\includegraphics[width=0.2\linewidth]{\fignet/boxplot_tree.pdf}
& \includegraphics[width=0.2\linewidth]{\fignet/boxplot_ER2p.pdf}
& \includegraphics[width=0.2\linewidth]{\fignet/boxplot_ER4p.pdf}
& \includegraphics[width=0.2\linewidth]{\fignet/boxplot_ER8p.pdf} \\
Tree & Erd\"{o}s-R\'{e}nyi & Erd\"{o}s-R\'{e}nyi & Erd\"{o}s-R\'{e}nyi \\
& $p_c = 2/p$ & $p_c = 4/p$ &  $p_c = 8/p$ 
\end{tabular} \\~

Area under the curves: top=ROC, bottom=PR\\
light grey = multinomial trees (\emphase{2.2''}), dark grey: multinomial DAGs (\emphase{1393''}) 

}	

%====================================================================
\frame{\frametitle{Illustration: Raf pathway}

Flow cytometry data for $p = 11$ proteins from the Raf signaling pathway \refer{SPP05} \\ ~

\begin{tabular}{cc}
  \includegraphics[trim = 12mm 35mm 12mm 18mm, clip,width=0.45\linewidth]{\fignet/RAF.pdf}
  & 
  \includegraphics[width=0.45\linewidth]{\fignet/RAF_edge_prob_graph_q0_05.pdf} \\
  'ground truth' & posterior probabilities \\ 
  ~\\
  \includegraphics[width=0.45\linewidth]{\fignet/best_1.pdf} 
  &
  \includegraphics[width=0.45\linewidth]{\fignet/best_2.pdf} \\
  most likely tree & second most likely tree
\end{tabular}

}

%====================================================================
%====================================================================
\section{Detecting changes in a graphical model}
\frame{\frametitle{Outline} \tableofcontents[currentsection]}

%====================================================================
\frame{\frametitle{Change-point in a graphical model} 

  \pause \paragraph{Problem:} \refer{ScR16}
  \begin{itemize}
  \item Consider $p$ variables observed along time; 
  \item Consider the graph $G_t$ supporting the graphical model at time $t$; 
  \item Does the graph $G_t$ remain the same along time? 
  \end{itemize}
  $$
  \includegraphics[height=.3\textheight]{\figchp/ChangePoint-Tree.pdf}
  $$

 \pause \paragraph{Examples:}
  \begin{enumerate}
   \item Gene regulatory network along the {\sl Drosophila} life cycle? \\~
   \item Connections between brain regions along different tasks?
  \end{enumerate}
%   }
  
}

%====================================================================
\frame{\frametitle{Handling two sums} 

  \paragraph{Double discrete structure:}
  \begin{itemize}
   \item $\approx (N/K)^K$ possible segmentations into $K$ segments; \\~
   \item $p^{K(p-2)}$ possible combination of $K$ trees \\~
  \end{itemize}
  \ra sum over $\approx (N/K)^K p^{K(p-2)}$ terms.

  \pause \bigskip \bigskip
  \paragraph{Combining the two preceding tricks:}
  \begin{itemize}
   \item Summing of all segmentations in $O(KN^2)$, \\~
   \item Summing over all trees in $O(p^3)$ (one tree per possible segment) \\~
  \end{itemize}
  \ra Global complexity $= O(\max\{K, p^3\}N^2)$

}

%====================================================================
\frame{\frametitle{Inference} 

  \paragraph{Quantities of interest} can be computed in $O(p^3N^2)$: \\
  \begin{itemize}
   \item $P(\text{change-point at time $t$} |K, Y)$ \\~
   \item $P(\text{edge $(i, j)$ present at time $t$} | K, Y)$ \\~
   \item $P(\text{edge $(i, j)$ remains present along time} |Y)$ \\~
   \item $P(\text{$K$ segments} | Y)$.
  \end{itemize}
  
  \pause \bigskip 
  \paragraph{+ Network comparison} \\
  \begin{itemize}
   \item $P(T_1 = T_2 | Y_1, Y_2)$ \\~
   \item $P(\text{edge $(i, j)$ present in both $T_1$ and $T_2$} |Y_1 ,Y_2)$.
  \end{itemize}

}
%====================================================================
\frame{\frametitle{Some simulations} 

  \begin{tabular}{ccc}
   Tree & Erd\"os ($\pi = 2/p$) & Erd\"os ($\pi = 4/p$) \\
   \begin{tabular}{c}
    \includegraphics[width=.3\textwidth]{\figchp/ScR16-Fig4-Tree_N_p10_Tree}
   \end{tabular}
   &
   \hspace{-.05\textwidth}
   \begin{tabular}{c}
    \includegraphics[width=.3\textwidth]{\figchp/ScR16-Fig4-ER2p_N_p10_Tree}
   \end{tabular}
   &
   \hspace{-.05\textwidth}
   \begin{tabular}{c}
    \includegraphics[width=.3\textwidth]{\figchp/ScR16-Fig4-ER4p_N_p10_Tree}
   \end{tabular}
   \\
   \begin{tabular}{c}
    \includegraphics[width=.3\textwidth]{\figchp/ScR16-Fig4-Tree_N_p10_Complete}
   \end{tabular}
   &
   \hspace{-.05\textwidth}
   \begin{tabular}{c}
    \includegraphics[width=.3\textwidth]{\figchp/ScR16-Fig4-ER2p_N_p10_Complete}
   \end{tabular}
   &
   \hspace{-.05\textwidth}
   \begin{tabular}{c}
    \includegraphics[width=.3\textwidth]{\figchp/ScR16-Fig4-ER4p_N_p10_Complete}
   \end{tabular}
  \end{tabular}
  From top to bottom: N = 70, 140, 210. \\
   \textcolor{blue}{Tree-structured} network. \textcolor{red}{Complete} network.

}

%====================================================================
\frame{\frametitle{Gene regulatory network} 

  \pause
  \paragraph{Data:} $N = 67$ time points, $p = 11$ genes, four expected regions

  \pause \bigskip
  Posterior probability of change-points:
  $$
  \includegraphics[width=.7\textwidth]{\figchp/SrS16-Fig8-chgpt_EMP_ap10_au1}
  $$

  \pause
  Inferred networks:
  $$
  \includegraphics[width=.8\textwidth]{\figchp/SrS16-Fig9-network_seuil02}
  $$

}

% %====================================================================
% \frame{\frametitle{Brain connectivity} 
% }

%====================================================================
%====================================================================
\section{Discussion}
\frame{\frametitle{Outline} \tableofcontents[currentsection]}

%====================================================================
\frame{\frametitle{Discussion} 

  \paragraph{To summarize.}
  \begin{itemize}
   \item Exact Bayesian inference can still be achieved for some fairly complex models with discrete parameter.
   \item Do not have to care about sampling and convergence.
   \item No systematic way to check when this is possible \ra ad-hoc developments.
  \end{itemize}

  \bigskip \bigskip \pause
  \paragraph{Future works.}
  \begin{itemize}
%    \item Combining the two problems: finding change-points in a network structure.
   \item Dealing with dependency along time.
   \item Influence of the prior: $p(T)$ depends on $n$ and/or $p$.
   \item The exact evaluation of the key quantity raises numerical issues.
  \end{itemize}
}


%====================================================================
\frame[allowframebreaks]{ \frametitle{References}
{\tiny
  \bibliography{/home/robin/Biblio/BibGene}
  %\bibliographystyle{/home/robin/LATEX/Biblio/astats}
  \bibliographystyle{plain}
  }
}

% %====================================================================
% %====================================================================
% \section{Appendix}
% 
% %====================================================================
% \frame{\frametitle{Bayesian inference: Example 1.0.0 (1/2)} 
% 
%   \paragraph{Beta/binomial setting.} Aim: estimate a success probability $\theta$
%   
%   \bigskip
%   \begin{itemize}
%    \item \pause Prior distribution on $\theta$:
%    $$
%    \theta \sim \text{Beta}(a, b): \qquad p(\theta) = \left. \theta^{a-1} (1-\theta)^{b-1} \right/ \Beta(a+b)
%    $$ \bigskip
%    \item \pause Model: $n$ independent trials, $Y =$ number of successes:
%    $$
%    (Y | \theta) \sim \Bcal(n, \theta): \qquad P(Y=y|\theta) = \binom{n}{y} \theta^y (1-\theta)^{n-y}
%    $$
%   \end{itemize}
% 
% }
% 
% %====================================================================
% \frame{\frametitle{Bayesian inference: Example 1.0.0 (2/2)} 
% 
%   \begin{itemize}
%    \item Marginal distribution of $Y$:
%    \begin{eqnarray*}
%    P(Y=y) & = & \int P(Y=y|\theta) p(\theta) \dd \theta \\
%    & = & \frac1{\Beta(a, b)} \binom{n}{y} \int \theta^{a+y-1} (1-\theta)^{b+n-y-1} \dd \theta \\
%    & = & \frac1{\Beta(a, b)} \binom{n}{y} \Beta(a+y, b+n-y) \\
%    \end{eqnarray*} \bigskip
%    \item \pause Posterior distribution of $\theta$:
%    $$
%    p(\theta|Y=y) = \frac{p(\theta) p(Y=y|\theta)}{p(Y=y)} = \frac{\theta^{a+y-1} (1-\theta)^{b+n-y-1}}{\Beta(a+y, b+n-y)}
%    $$
%    \pause so we recognize
%    $$
%    (\theta|Y) \sim \Beta(\widetilde{a}, \widetilde{b}): \qquad \widetilde{a}=a+Y, \quad \widetilde{b}=b+n-Y
%    $$
%   \end{itemize}
% 
% }


%====================================================================
%====================================================================
\end{document}
%====================================================================
%====================================================================

  \begin{tabular}{cc}
    \begin{tabular}{p{.5\textwidth}}
    \end{tabular}
    & 
    \hspace{-.02\textwidth}
    \begin{tabular}{p{.5\textwidth}}
    \end{tabular}
  \end{tabular}

