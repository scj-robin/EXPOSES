\documentclass[12pt]{article}
%
%
% Retirez le caractere "%" au debut de la ligne ci--dessous si votre
% editeur de texte utilise des caracteres accentues
\usepackage[latin1]{inputenc}
%
% Retirez le caractere "%" au debut des lignes ci--dessous si vous
% utiisez les symboles et macros de l'AMS
\usepackage{amsmath}
\usepackage{amsfonts}
%
%
\setlength{\textwidth}{16cm}
\setlength{\textheight}{21cm}
\setlength{\hoffset}{-1.4cm}

\newcommand{\ymt}{y_{mt}}
\newcommand{\Ymt}{Y_{mt}}
\newcommand{\ec}[1]{\mathbb{E}_{\phi^{(h)}}\left\{#1 |\mathbf{Y} \right\}}
\newcommand{\vc}[1]{\mathbb{V}_{\phi^{(h)}}\left\{#1 |\mathbf{Y} \right\}}

\newcommand{\alphabf}{\text{\mathversion{bold}{$\alpha$}}}
\newcommand{\thetabf}{\text{\mathversion{bold}{$\theta$}}}
\newcommand{\Sigmabf}{\text{\mathversion{bold}{$\Sigma$}}}
\newcommand{\Psibf}{\text{\mathversion{bold}{$\Psi$}}}
\newcommand{\mubf}{\text{\mathversion{bold}{$\mu$}}}
\newcommand{\xbf}{\mathbf{x}}
\newcommand{\Xbf}{\mathbf{X}}
\newcommand{\Ybf}{\mathbf{Y}}
\newcommand{\Zbf}{\mathbf{Z}}
\newcommand{\Bbf}{\mathbf{B}}
\newcommand{\Bbfp}{\mathbf{B}^{\prime}}
\newcommand{\Zbfp}{\mathbf{Z}^{\prime}}
\newcommand{\Wbf}{\mathbf{W}}
\newcommand{\ZA}{\mathbf{Z}_A}
\newcommand{\ZB}{\mathbf{Z}_B}
\newcommand{\WB}{\mathbf{W}_B}
\newcommand{\WA}{\mathbf{W}_A}
\newcommand{\WAB}{\mathbf{W}_{AB}}
\newcommand{\UA}{\mathbf{U}_A}
\newcommand{\UB}{\mathbf{U}_B}
\newcommand{\Vbf}{\mathbf{V}}
\newcommand{\Rbf}{\mathbf{R}}
\newcommand{\Ubf}{\mathbf{U}}
\newcommand{\Ebf}{\mathbf{E}}
\newcommand{\Gbf}{\mathbf{G}}
\newcommand{\Tbf}{\mathbf{T}}
\newcommand{\Ibf}{\mathbf{I}}
\newcommand{\RMSE}{\mbox{RMSE}}
\newcommand{\FPR}{\mbox{FNR}}
\newcommand{\FNR}{\mbox{FNR}}
\newcommand{\LogL}{\mathcal{L}}
\def\1{1\!{\rm l}}
%\newtheorem{prop}{Proposition}
%\newproof{pf}{Proof}Les probl�mes de segmentation se rencontrent dans de nombreux domaines (�conomie, �cologie, biologie). 


\def\argmin{\mathop{\mathrm{argmin}}}
%
%
%-------------------------------------------------------------------------
%-------------------------------------------------------------------------
\begin{document}
%-------------------------------------------------------------------------
%-------------------------------------------------------------------------
%
\begin{center}
{\Large
        {\sc A factor model approach for the segmentation of correlated time series 
        }
}
\bigskip

Emilie Lebarbier \& St�phane Robin

\medskip
{\it

UMR 518 AgroParisTech / INRA Math. Info. Appli. \\
16, rue Claude Bernard \\
75005 Paris {\sc France}
}
\end{center}
\bigskip
\noindent

\paragraph{R�sum�.} On s'int�resse � la segmentation d'un ensemble de
s�ries d'observations corr�l�es, typiquement organis�es spatialement.
On propose de mod�liser la d�pendance entre les s�ries au moyen d'un
mod�le � facteur. Cette mod�lisation de la d�pendance autorise
l'utilisation d'algorithmes de segmentation efficaces pour obtenir les
estimateur du maximum de vraisemblance. On propose �galement une
proc�dure de s�lection de mod�le pour d�terminer le nombre de points
de ruptures ainsi que le nombre de facteurs.

\paragraph{Abstract.} We consider the segmentation of set of
correlated time-series, typically with some spatial structure. We
propose to model the between-series dependency with a factor model.
This modelling allows us to use efficient segmentation algorithms to
obtain the maximum likelihood estimates. We also propose a model
selection procedure to determine the number of change point in each
series and the number of factors.

\paragraph{Keywords.}   E-M algorithm, Factor model, Segmentation.

%-------------------------------------------------------------------------
%-------------------------------------------------------------------------
\section{Introduction}
%-------------------------------------------------------------------------
We consider the detection of change-points in a set of time-series. We
are typically interested in the case where this data consist in series
of measurements observed along time in different locations. Each
series is supposed to be affected changes at series-specific
breakpoints and the signals observed at each location are supposed to
be correlated due to their spatial organisation.

One of the difficulty here is to propose a modelling that leads to an
efficient estimation algorithm. Indeed, the inference of segmentation
models often require to search over the space of all possible
segmentations, which is prohibitive in terms of computational time.
Dynamic programming (DP) strategies remain among the most efficient
but can only be applied when the contrast to be optimised is additive
with respect to the segments. In presence of dependency, the contrast
(e.g. the log-likelihood) is generally not additive. Our strategy
consists in 'removing' the dependency so that, at a given step of the
estimation algorithm, dynamic programming can be applied to
transformed data.

A similar setting is considered in \cite{PLB11} where a variance
component model is used to account for the dependecy between the
series.  Our purpose here is broaden the set of possible dependency
structures that the modelling can account for. The factor model
provides a convenient and efficient way to describe various covariance
matrices, still limiting the number of parameters. It can be viewed as
a generalisation of variance components models, where the components
are unknown and need to be estimated, together with the associated
variances. It is based on the spectral decomposition of the covariance
matrix and has been successfully applied in situation where very
little is known about the correlation structure (see e.g.
\cite{FKC09}).  The inference of factor models is often achieved via
an EM algorithm.

We present here a general model for correlated Gaussian time series,
based on a factor model for the covariance matrix. We show that some
by-product of the EM algorithm can be used to 'remove' the dependency
between the series. This allows use to combine EM and DP algorithms
together. We then discuss the issue of choosing both the number of
breakpoints and the number of factors. Simulation and illustrations
will be presented in the oral presentation.

%-------------------------------------------------------------------------
%-------------------------------------------------------------------------
\section{Data, notations and model}
%-------------------------------------------------------------------------

\paragraph{Data and notations.} We consider $M$ series with $n_m$ points
each. We note $y_{tm}$ the observed signal of series $m$ at time $t$.
The total number of observations is $N=\sum_{m=1}^M n_m$. In the
following, we consider that $n_m=n$ whatever the series. The data $y$
are modeled by a random Gaussian process $Y$ with size $[n \times M]$.
In general we denote $A_t$ the row vector with size $M$ of the matrix
$A$ and $A^m$ its column vector with size $n$. Thus $Y^m$ represents
whole series $m$, while $Y_t$ stands for the observations at time $t$
in all the series.

\paragraph{Segmentation.} We consider here that each series has its
own segmentation: the mean of the series $\{Y_{tm}\}_t$ is subject to
$K_{m}-1$ specific abrupt changes at breakpoints $\{t_{k}^{m}\}$ (with
convention $t^m_0=0$ and $t^m_{K_m}=n_{\max}$) and is constant between
two breakpoints within the interval $I_{k}^{m} = ]t_{k-1}^{m},
t_{k}^{m}]$. In the following we denote by $K=\sum_{m}^M K_{m}$ the
total number of segments and $n_k^{m} = t_{k}^{m}- t_{k-1}^{m}$ the
length of segment $k$ for series $m$ ($k=1,\ldots,K_m$). The
segmentation model is written as follows:
\begin{equation} \label{eq:model1}
Y_{tm}=\mu_{km} +F_{tm} \qquad \forall t \in I_k^{m}
\end{equation}
where the error vectors $\{F_t\}_t$ has a centered Gaussian
distribution with a covariance matrix $\Sigmabf$ to be specified.

\paragraph{Correlations between series.} We want to take
into account the correlations that can exit between series. In the
case of spatial data, the correlation can take the form
$e^{-d(m,m^{\prime})}$ where $d(m,m^{\prime})$ is the distance between
the locations of series $m$ and $m^{\prime}$. However this correlation
structure (or other structures) hampers the use of the DP algorithm
to obtain the best segmentation. \\
We propose here to consider the factor analysis model framework.  That
consists in identifying a linear space of $Q$ random vectors that
captures the dependence among the series (see \cite{FKC09}).  In
others words, we set
$$
\Sigmabf=\Bbf \mathbb{V}(\Zbf) \Bbfp+\Psibf,
$$
where $\Zbf$ corresponds to the $Q$ random vectors and $B$ the
associated coefficients. The dependence between series is then free
from any spatial structure. In the above decomposition, $\Bbf
\mathbb{V}(\Zbf) \Bbfp$ refers to the shared variance and $\Psibf$ to
the specific one. 
% Note that, in this case $Q$, can not be equal to the number of series
% $M$.

\paragraph{Model.} With the previous decomposition of the variability,
the model (\ref{eq:model1}) can be rewritten as a mixed linear model:
\begin{equation*}
Y_{tm}=\mu_{km} + \sum_{q=1}^{Q} Z_{tq} b_{qm}+E_{tm} \qquad \forall t
\in I_k^{m}
\end{equation*}
where the $\{E_{t}\}_t$ are i.i.d. centered Gaussian vectors with
variance matrix $\Psibf$ and the $\{Z_{t}\}_t$ are i.i.d. centered
Gaussian vectors with variance $I_Q$ (without loss of generality), the
two sets of vectors being independent.  So $\Sigmabf$ can be
decomposed as
\begin{eqnarray} \label{eq:variance_decomposition}
\Sigmabf=\Bbf \Bbfp+\Psibf.
\end{eqnarray}
Here $\Psibf$ is supposed to be diagonal $\Psibf=\sigma^2 I_M$. This
will allow us to use the DP algorithm for the segmentation parameter
estimation. The matrix formulation of this linear model is
\begin{equation} \label{eq:model}
\Ybf=\Tbf \mubf +\Zbf \Bbfp+\Ebf
\end{equation}
where
\begin{itemize}
\item $\Ybf$, with size $[n \times M]$, stands for the observed data,
\item $\Tbf$ is the incidence matrix of breakpoints with size $[n\times
K]$,
$
{T^m} = \text{Bloc} \left[ \1_{n_{K_m}^{m}} \right],
$
and
$
{\Tbf} = \left[ T^1 \ T^2 \ \ldots \ T^M \right].
$
\item $\mubf$ the means with size $[K\times M]$ (and $\mu_{k}^m$ the mean of the segment $k$ for series $m$) such that
$
{\mu^{m}}= \text{Bloc} \left[ \mu_{K_m}^m \right],
$
and
$
{\mubf}  = \left[ \mu^1  \ \mu^2  \ \ldots  \ \mu^M \right].
$
\item $\Zbf$ with size $[n\times Q]$ and $\Bbf$
with size $[M \times Q]$,
\item $\Ebf$ with size $[n\times M]$. \\
\end{itemize}
The main difference between this model and a classical mixed linear
model is that both the incidence matrix $\Tbf$ and the factor matrix
$\Bbf$ are unknown.

%-------------------------------------------------------------------------
%-------------------------------------------------------------------------
\section{Estimation using the EM algorithm}
%-------------------------------------------------------------------------

We now consider the maximum-likelihood inference of model
\eqref{eq:model}. This can be done via an EM algorithm, considering
that $\Zbf$ represents the missing data (hidden and unobserved). The
parameters of the model are $\phi=(\Tbf,\mubf,\sigma^2,\Bbf)$. In this
setting, the complete-data log-likelihood is:
\begin{eqnarray*}
\log \mathcal{L}(\Ybf,\Zbf; \phi) = \log \mathcal{L}_0(\Ybf|\Zbf;
\phi) + \log \mathcal{L}_1(\Zbf).
\end{eqnarray*}
Since the distribution of $\Zbf$ does not depend on the parameters
$\phi$, only the first term will be considered which is written:
\begin{eqnarray*}
-2 \log \mathcal{L}_0(\Ybf|\Zbf; \phi)=N  \log{(2 \pi)}+ n
\log{(|\Psibf|)}+\sum_{t=1}^n \|Y_t - \mu_{k(t)} - Z_t
\Bbfp\|^2_{\Psibf^{-1}},
\end{eqnarray*}
then its conditional expectation, $Q_0(\phi; \phi^{(h)})=\ec{\log
\mathcal{L}_0(\Ybf|\Zbf; \phi)}$ satisfies
\begin{eqnarray*}
-2 Q_0(\phi; \phi^{(h)}) &=& N  \log{(2 \pi)}+ n
\log{(|\Psibf|)}+\sum_{t=1}^n \left [ \|Y_t - \mu_{k(t)} -
\widehat{\Zbf}_t^{(h)} \Bbfp\|^2_{\Psibf^{-1}} +\text{Tr} \left( \Bbfp
  \Psibf^{-1} \Bbf \Wbf_t^{(h)} \right) \right ],
\end{eqnarray*}
where $\mathbb{E}_{\phi}\{\cdot\}$ is the expectation operator using
$\phi$ as the parameter value and $\mathbb{V}_{\phi}\{\cdot\}$ the
corresponding variance, $\widehat{\Zbf}_t^{(h)} = \ec{\Zbf_t}$,
$\text{Tr}(A)$ is the trace of matrix $A$, $|A|$ its determinant and
$\Wbf_t^{(h)}=\vc{\Zbf_t}$. \\
%The second $\log$-likelihood is written:
%\begin{eqnarray*}
%-2 \log \mathcal{L}_1(\Zbf)=N  \log{(2 \pi)}+ \sum_{t=1}^n \|Z_t
%\|^2,
%\end{eqnarray*}
%then its conditional version is
%\begin{eqnarray*} -2 Q_1(\phi; \phi^{(h)}) &=& N  \log{(2 \pi)}+
%\sum_{t=1}^n \left [ \widehat{\Zbf}_t^{(h)} \widehat{\Zbfp}_t^{(h)}
%+\text{Tr} \left( \Wbf_t^{(h)} \right) \right ].
%\end{eqnarray*}

\paragraph{E-step} This step consists in the calculation of the
conditional expectation which only requires the calculation of
$\widehat{\Zbf}$ and $\Wbf$: at iteration $(h+1)$, we get\\
%$$
%\begin{cases}
%\widehat{\Zbf}_t^{(h+1)} = E_t^{(h)} \Psibf^{-1,(h)} \Bbf^{(h)}
%\Wbf_t^{(h)}/{\sigma^{2,(h)}}, \\
%\Wbf_t^{(h+1)}  = \left (I_M+ \Bbf'^{(h)} \Psibf^{-1,(h)} \Bbf^{(h)}
%\right )^{-1}.
%\end{cases}
%$$
$$
\begin{cases}
\widehat{\Zbf}_t^{(h+1)} = \tilde{Y}_t^{(h)} \Bbf^{(h)}
\Wbf_t^{(h)}/{\sigma^{2,(h)}}, \\
\Wbf_t^{(h+1)}  = \left (I_M+ \Bbf'^{(h)} \Bbf^{(h)} /\sigma^{2,(h)}
\right )^{-1}.
\end{cases}
$$
where $\tilde{Y}_t^{(h)}=Y_t-\mu_{k(t)}^{(h)}$.
\paragraph{M-step} This step consists in the estimation of the
parameters by maximizing the obtained conditional expectation.
\begin{itemize}
\item Estimation of the variance component $\sigma^{2}$:
%\begin{eqnarray*}
%\Psibf^{(h+1)}&=&\arg \max_{\Psibf} Q_0,\\
%&=& \frac{1}{n} \sum_{t=1}^n (Y_t-\mu_{k(t)}^{(h)})'
%(Y_t-\mu_{k(t)}^{(h)})- \frac{1}{n} \sum_{t=1}^n
%(Y_t-\mu_{k(t)}^{(h)})' \widehat{\Zbf}_t^{(h+1)}  \Bbf'^{(h)}
%\end{eqnarray*}
%In the case where $\Psibf=\sigma^2 I_M$, we get
\begin{eqnarray*}
\sigma^{2,(h+1)} &=& \frac{1}{N} \sum_{t=1}^n \left [ E_t^{(h)}
E_t'^{(h)}+ \text{Tr} \left( \Bbf'^{(h)}
 \Bbf^{(h)}  \Wbf_t^{(h+1)} \right)\right ],
\end{eqnarray*}
where $E_t^{(h)}=Y_t-\mu_{k(t)}^{(h)}-\widehat{\Zbf}_t^{(h)}
\Bbf'^{(h)}$.

\item Estimation of $\Bbf$.
\begin{eqnarray*}
\Bbf^{(h+1)}= \sum_{t=1}^n (Y_t-\mu_{k(t)}^{(h)})'
\widehat{\Zbf}_t^{(h+1)} \ \left [ \sum_{t=1}^n
(\widehat{Z}_t'^{(h+1)} \widehat{Z}_t^{(h+1)}+ W_t^{(h+1)})\right
]^{-1}.
\end{eqnarray*}

\item Estimation of the segmentation parameters $\Tbf \mubf$.
\begin{eqnarray*}
\left\{ \Tbf^{(h+1)},\mubf^{(h+1)} \right\} & = & \arg
\max_{\Tbf,\mubf} Q_0, \\
&=& \arg \max_{\Tbf,\mubf} \sum_{t=1}^n \| Y_t- (\Tbf \mubf)_t-
\widehat{\Zbf}^{(h+1)}_t \Bbf'^{(h+1)}\|^2_{\Psibf^{-1,(h+1)}}, \\
&=& \arg \max_{\Tbf,\mubf} \sum_{t=1}^n \| \tilde{\tilde{Y}}_t-
(\Tbf \mubf)_t\|^2_{\Psibf^{-1,(h+1)}}.
\end{eqnarray*}
where $\tilde{\tilde{Y}}_t=Y_t- \widehat{\Zbf}_t^{(h+1)}
\Bbf'^{(h+1)}$. This last term can be viewed as a correction to remove
dependency between the series. Since $\Psibf$ is diagonal the DP (in
particular the two-stages DP) can be used to obtain the segmentation
parameter estimations.
\end{itemize}

%\max_{\Tbf,\mubf} \sum_{m=1}^M \sum_{k=1}^{K_m} \sum_{t \in I^k_m}
%\left (\tilde{Y}_{tm}-\mu_{km} \right )^2,

% The EM algorithm is known to depend strongly on the initial value of
% the parameters, which is here the segmentation one. This phenomenon
% often leads to erratic behavior of the likelihood and consequently
% hampers the use of model selection criteria based on this likelihood. \\
% In the application of the procedure on the real data, this phenomenon
% is almost often observed on the likelihood with respect to $q$ for
% each considered value of $K$. To solve this problem, for the values of
% $q$ for which the likelihood does not increase, we restart the EM
% algorithm with initial parameters, the parameters obtained with eithe
% $q-1$ or $q+1$.


%-------------------------------------------------------------------------
%-------------------------------------------------------------------------
\section{Model selection}
%-------------------------------------------------------------------------

Here either the number of factors $q$ and the number of segments $K$
should be estimated. The joint estimation of two parameters is not
classical. We propose here a heuristic to select both these two
parameters. First, for each $K$, $q$ is selected by using the BIC
criterion:
\begin{eqnarray*}
\hat{q}_K&=&\argmin_{q \in \{1,\ldots,M-1\}} BIC(q_K),\\
&=& \argmin_{q \in \{1,\ldots,M-1\}}-2 \LogL(\widehat{\Tbf \mubf}_K,
\hat{\Sigmabf}_q)+Dq \log{(n)},
\end{eqnarray*}
where $\LogL(\widehat{\Tbf \mubf}_K, \hat{\Sigmabf}_q)$ is the
$\log$-likelihood calculated at its maximum for a fixed $K$ and a
fixed $q$, and $D_q$ is the number of parameters in a model with $q$
factors, $D_q=q(2 M -q+1)/2+1$. Indeed for the variance components,
according to the variance decomposition (cf equation
(\ref{eq:variance_decomposition})), the number of parameters are $M
\times q$ for $B$ and one for $\sigma^2$. Moreover, with the
orthogonality condition on $B$, only $M q -q(q-1)/2$ need to be
estimated. \\
% Note that the number of segments is not be considered since
% does not affected this part of the model selection.\\
To select the number of segments, we use the modified BIC proposed by
\cite{ZhS07} adapted to the joint segmentation by \cite{PLB11}
\begin{eqnarray*}
mBIC_{\text{JointSeg}}(K)&= &\left ( \frac{N+1}{2} \right
)\log{SS_{\text{all}}} -\left ( \frac{N-K+1}{2} \right )
\log{SS_{\text{wg}}(\hat{t})} + \log{\left [\Gamma \left
(\frac{N-K+1}{2} \right)\right ]}
\\
&&   -\frac{1}{2} \sum_{m=1}^M \sum_{k=1}^{k_m} \log
{\hat{n}^m_k}+\left ( \frac{1}{2}-(K-M) \right ) \log{(N)},
\end{eqnarray*}
where
\begin{eqnarray*}
&&SS_{\text{wg}}(\hat{t})=\sum_{t=1}^n (\Ybf_t-\hat{\mu}_{k(t)})
\hat{\Sigmabf}_q^{-1}(\Ybf_t-\hat{\mu}_{k(t)})', \\
&& SS_{\text{all}}=\sum_{t=1}^n (\Ybf_t-\bar{Y})
\hat{\Sigmabf}_q^{-1}(\Ybf_t-\bar{Y})',
\end{eqnarray*}
with $\hat{n}^m_k$ is the length of segment $k$ in profile $m$
($\hat{n}^m_k=\hat{t}^m_k-\hat{t}^m_{k-1}+1$), and $\hat{\mu}_{k(t)}$
is a vector of size $M$ with the component $m$ is
$\bar{y}_{mk}=(\hat{n}^m_k)^{-1}
\sum_{t=\hat{t}^m_{k-1}+1}^{\hat{t}^m_{k}} y_m(t)$ if $t \in
\hat{I}^m_k$.






%We use two criteria: the ones developed by \cite{Lav05} and
%\cite{L05} which the penalty depends on the number of parameters in
%a model with $K$ segments, denoted by $D_K$. Here
%$$
%D_K=K+\frac{Q}{2}(2 M -Q+1)+1.
%$$
%Indeed, for the mean parameters, we have $K$ means to estimate and
%for the variance components, according to the variance decomposition
%(cf equation (\ref{eq:variance_decomposition})), the number of
%parameters are $M \times Q$ for $B$ and one for $\sigma^2$.
%Moreover, with the orthogonality condition on $B$, only $M Q
%-Q(Q-1)/2$ need to be estimated. \\


%The criterion proposed by \cite{Lav05} is
%begin{equation*} Crit_{\text{Lav}}(K)= -2 \log {\mathcal{L}_K(\Ybf;
%\widehat{\phi})}+\beta D_K
%\end{equation*}
%where $\log { \mathcal{L}_K(\Ybf; \widehat{\phi})}$ stands for the
%log-likelihood calculated at its maximum. The constant $\beta$ is
%chosen using an adaptative method which involves a threshold $s$. As
%Lavielle suggests $s \in [0.5, 0.75]$, we used $s=0.7$ throughout
%the simulation study. \\

%The one proposed by \cite{L05} is
%\begin{equation*}
%Crit_{\text{Leb}}(K)=-2 \log {\mathcal{L}_K(\Ybf;
%\widehat{\phi})}+\alpha \left[ 5 D_K +2 (K-M) \log \left
%(\frac{N-M}{K-M} \right) \right]
%\end{equation*}
%This penalty also depends on a constant $\alpha$ which can be
%calibrated in practice (cf \cite{BM04}).



% %-------------------------------------------------------------------------
% %-------------------------------------------------------------------------
% \section{Simulation study}
% %-------------------------------------------------------------------------


% %-------------------------------------------------------------------------
% %-------------------------------------------------------------------------
% \section{Application to temperature data series}
% %-------------------------------------------------------------------------

%-------------------------------------------------------------------------
%-------------------------------------------------------------------------
% \bibliography{/Biblio/AST}
% \bibliographystyle{amsplain}

\begin{thebibliography}{1}

\bibitem{FKC09}
C.~Friguet, M.~Kloareg, and D.~Causeur, \emph{A factor model approach to
  multiple testing under dependence}, J. Amer. Statist. Assoc. \textbf{104}
  (2009), no.~488, 1406--15, {\tt DOI:10.1198/jasa.2009.tm08332}.

\bibitem{PLB11}
F.~Picard, E.~Lebarbier, E.~Budinska, and S.~Robin, \emph{Joint segmentation of
  multivariate gaussian processes using mixed linear models}, Comput. Statist.
  and Data Analysis \textbf{55} (2011), no.~2, 1160--70.

\bibitem{ZhS07}
N.~R. Zhang and D.~O. Siegmund, \emph{A modified {B}ayes information criterion
  with applications to the analysis of comparative genomic hybridization data},
  Biometrics \textbf{63} (2007), no.~1, 22--32.

\end{thebibliography}

%-------------------------------------------------------------------------
%-------------------------------------------------------------------------
\end{document}
%-------------------------------------------------------------------------
%-------------------------------------------------------------------------


TexteLong  de quatre \`a six pages qui sera mis en ligne sur le site
du congr\`es et reproduit sur le CD ROM distribu\'e aux
participants. Ce TexteLong devra mentionner le titre, le ou les
auteurs, leurs affiliations et leurs adresses. Il comprendra un
r\'esum\'e en fran\c cais, un r\'esum\'e en anglais, une liste de
mots cl\'es et une bibliographie.
\medskip

Les r\'ef\'erences bibliographiques seront donn\'ees dans le texte
sous la forme pr\'esent\'ee dans l'exemple suivant.
\bigskip

\noindent{EXEMPLE}

La n\'ecessit\'e de produire des r\'esum\'es clairs et bien
r\'ef\'erenc\'es a \'et\'e d\'emontr\'ee par Achin et Quidont~(2000). Le
r\'ecent article de Noteur~(2003) met en \'evidence \dots

%Quelques rappels :
%%
%\begin{center}
%%
%\begin{tabular}{lr} \hline
%%
%Accent aigu :              &  \'e; \\
%Accent grave :             &  \`a;\\
%Accent circonflexe :       &  \^o mais \^{\i};\\
%Tr\'emas :                 &  \"o mais \"{\i};\\
%C\'edille :                &  \c{c}. \\ \hline
%\end{tabular}
%%
%\end{center}

\bigskip

\noindent {\large{\bf Bibliographie}}
\medskip
%

\noindent [1] Auteurs (ann\'ee) Titre, revue, localisation.

\noindent [2] Achin, M. et Quidont, C. (2000) {\it Th\'eorie des
Catalogues}, Editions du Soleil, Montpellier.

\noindent [3] Noteur, U. N. (2003) Sur l'int\'er\^et des
r\'esum\'es. {\it Revue des Organisateurs de Congr\`es}, 34, 67--89.


