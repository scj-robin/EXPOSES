\documentclass{beamer}

% Beamer style
%\usetheme[secheader]{Madrid}
\usetheme{CambridgeUS}
\usecolortheme[rgb={0.65,0.15,0.25}]{structure}
%\usefonttheme[onlymath]{serif}
\beamertemplatenavigationsymbolsempty
%\AtBeginSubsection

% Packages
%\usepackage[french]{babel}
\usepackage[latin1]{inputenc}
\usepackage{color}
\usepackage{dsfont, stmaryrd}
\usepackage{amsmath, amsfonts, amssymb}
\usepackage{stmaryrd}
\usepackage{epsfig}
\usepackage{array}
\usepackage{url}
\usepackage{/Latex/astats}
%\usepackage[all]{xy}
\usepackage{graphicx}

% Commands
\definecolor{darkred}{rgb}{0.65,0.15,0.25}
\newcommand{\emphase}[1]{\textcolor{darkred}{#1}}
\newcommand{\ppause}{}
%\newcommand{\emphase}[1]{{#1}}
\newcommand{\paragraph}[1]{\textcolor{darkred}{#1}}
\newcommand{\refer}[1]{\textcolor{blue}{\sl \cite{#1}}}
\newcommand{\Refer}[1]{\textcolor{blue}{\sl #1}}
\newcommand{\newblock}{}

% Symbols
\newcommand{\Abf}{{\bf A}}
\newcommand{\Bias}{\mathbb{B}}
\newcommand{\Bbf}{{\bf B}}
\newcommand{\Beta}{\text{B}}
\newcommand{\Bcal}{\mathcal{B}}
\newcommand{\BIC}{\text{BIC}}
\newcommand{\dd}{\text{d}}
\newcommand{\dbf}{{\bf d}}
\newcommand{\Dcal}{\mathcal{D}}
\newcommand{\Esp}{\mathbb{E}}
\newcommand{\Ebf}{{\bf E}}
\newcommand{\Ecal}{\mathcal{E}}
\newcommand{\Fbf}{{\bf F}}
\newcommand{\Gcal}{\mathcal{G}}
\newcommand{\Gbf}{{\bf G}}
\newcommand{\Gam}{\mathcal{G}\mbox{am}}
\newcommand{\Ibb}{\mathbb{I}}
\newcommand{\Ibf}{{\bf I}}
\newcommand{\ICL}{\text{ICL}}
\newcommand{\Jbf}{{\bf J}}
\newcommand{\Cov}{\mathbb{C}\text{ov}}
\newcommand{\Corr}{\mathbb{C}\text{orr}}
\newcommand{\Var}{\mathbb{V}}
\newcommand{\Vsf}{\mathsf{V}}
\newcommand{\pen}{\text{pen}}
\newcommand{\Fcal}{\mathcal{F}}
\newcommand{\Hbf}{{\bf H}}
\newcommand{\Hcal}{\mathcal{H}}
\newcommand{\Jcal}{\mathcal{J}}
\newcommand{\Kbf}{{\bf K}}
\newcommand{\Lcal}{\mathcal{L}}
\newcommand{\Mcal}{\mathcal{M}}
\newcommand{\mbf}{{\bf m}}
\newcommand{\mum}{\mu(\mbf)}
\newcommand{\Ncal}{\mathcal{N}}
\newcommand{\Nbf}{{\bf N}}
\newcommand{\Nm}{N(\mbf)}
\newcommand{\Ocal}{\mathcal{O}}
\newcommand{\Obf}{{\bf 0}}
\newcommand{\Omegas}{\underset{s}{\Omega}}
\newcommand{\Pbf}{{\bf P}}
\newcommand{\Pcal}{\mathcal{P}}
\newcommand{\Qcal}{\mathcal{Q}}
\newcommand{\Rbb}{\mathbb{R}}
\newcommand{\Rcal}{\mathcal{R}}
\newcommand{\sbf}{{\bf s}}
\newcommand{\Sbf}{{\bf S}}
\newcommand{\Scal}{\mathcal{S}}
\newcommand{\Ucal}{\mathcal{U}}
\newcommand{\Vcal}{\mathcal{V}}
\newcommand{\Tbf}{{\bf T}}
\newcommand{\ubf}{{\bf u}}
\newcommand{\Ubf}{{\bf U}}
\newcommand{\Vbf}{{\bf V}}
\newcommand{\Wbf}{{\bf W}}
\newcommand{\xbf}{{\bf x}}
\newcommand{\Xbf}{{\bf X}}
\newcommand{\ybf}{{\bf y}}
\newcommand{\Ybf}{{\bf Y}}
\newcommand{\zbf}{{\bf z}}
\newcommand{\Zbf}{{\bf Z}}
\newcommand{\betabf}{\mbox{\mathversion{bold}{$\beta$}}}
\newcommand{\pibf}{\mbox{\mathversion{bold}{$\pi$}}}
\newcommand{\Sigmabf}{\mbox{\mathversion{bold}{$\Sigma$}}}
\newcommand{\gammabf}{\mbox{\mathversion{bold}{$\gamma$}}}
\newcommand{\mubf}{\mbox{\mathversion{bold}{$\mu$}}}
\newcommand{\nubf}{\mbox{\mathversion{bold}{$\nu$}}}
\newcommand{\Thetabf}{\mbox{\mathversion{bold}{$\Theta$}}}
\newcommand{\thetabf}{\mbox{\mathversion{bold}{$\theta$}}}
\newcommand{\BP}{\text{BP}}
\newcommand{\EM}{\text{EM}}
\newcommand{\VEM}{\text{VEM}}
\newcommand{\VBEM}{\text{VB}}
\newcommand{\cst}{\text{cst}}
\newcommand{\obs}{\text{obs}}
\newcommand{\ra}{\emphase{\mathversion{bold}{$\rightarrow$}~}}
\newcommand{\QZ}{Q_{\Zbf}}
\newcommand{\Qt}{Q_{\thetabf}}

%====================================================================
\title[Factor model for segmentation]{A factor model
  approach for the joint segmentation of correlated series}

\author[Lebarbier \& Robin]{E. Lebarbier \& S. Robin}

\institute[INRA/AgroParisTech]{INRA / AgroParisTech \\
  \bigskip
  \begin{tabular}{ccccc}
    \epsfig{file=/RECHERCHE/RESEAUX/Exposes/Figures/LogoINRA-Couleur.ps,
      width=2.5cm} & 
    \hspace{.5cm} &
    \epsfig{file=/RECHERCHE/RESEAUX/Exposes/Figures/logagroptechsolo.eps,
      width=3.75cm} & 
    \hspace{.5cm} &
    \epsfig{file=/RECHERCHE/RESEAUX/Exposes/Figures/Logo-SSB.eps,
      width=2.5cm} \\ 
  \end{tabular} \\
  \bigskip
  }

\date[JDS 2011]{Journ�es de Statistique, Mai 2011, Tunis}
%====================================================================

%====================================================================
%====================================================================
\begin{document}
%====================================================================
%====================================================================

%====================================================================
%====================================================================
\section {Introduction}
%====================================================================
\frame{\titlepage}

%====================================================================
\frame{ \frametitle{Segmentation of multiple series}
  \paragraph{Aim:} 
  \begin{itemize}
  \item Find abrupt change-points in the distribution of a signal,
  \item Within multiple series of measurements,
  \item In presence of correlation between the series.
  \end{itemize}

  \pause\bigskip
%   \begin{tabular}{lccc}
  \begin{tabular}{p{.2\textwidth}
      >{\centering\arraybackslash}m{.22\textwidth}
    >{\centering\arraybackslash}m{.22\textwidth}
    >{\centering\arraybackslash}m{.22\textwidth}} 
    \emphase{Examples:} & \emphase{Genomics} &
    \emphase{Meteorology} & \emphase{Geology} \\
    \hline    
    \emphase{Series $m$} & Patients & Station & Station \\
    \hline    
    \emphase{Time $t$} & Position along & Time & Time \\
    & the genome &  &  \\
    \hline    
    \emphase{Signal $Y_{tm}$} & Micro-array & Temperature & GPS
    location \pause \\ 
    \hline    
    \emphase{Breakpoints} & Endpoints of & Change of &
    Earth's crust \\ 
    \emphase{$\{t_k^m\}$} & altered regions & instrument & shifts
    \pause \\
    \hline    
    \emphase{Correlation $\Sigmabf$} & Probe effect & Spatial & Spatial \\
  \end{tabular}
  }

%====================================================================
\frame{ \frametitle{Basic segmentation model}

  \paragraph{For one single series:} \\
  \begin{tabular}{cc}
    \hspace{-.5cm}
    \begin{tabular}{p{.5\textwidth}}
      \emphase{$Y_t =$} signal at time $t$,\\
      \\
      \emphase{$t_k=$} $k$-th breakpoints ($k=1..K-1$), \\
      \\
      \emphase{$I_k =$} $]t_{k-1}; t_k]:$ $k$-th interval, 
      $$
      \forall t \in I_k, \qquad Y_{t}=\mu_{k} +F_{t}
      $$
      where the $\{F_t\}$ are i.i.d. $\Ncal(0, \sigma^2)$.
    \end{tabular}
    & 
    \hspace{-.5cm}
    \begin{tabular}{p{.5\textwidth}}
      \paragraph{A genomic example:} \\
      \\
      \epsfig{file=../Figures/bt474_c1_seg_homo_K10.eps, clip=,
      width=.45\textwidth} 
    \end{tabular}
  \end{tabular}
  }

%====================================================================
\frame{ \frametitle{Computational issue}

  \paragraph{One series.} There are ${{n-1}\choose{K-1}}$ possible
  ways to divide a series with length $n$ into $K$ segments.

  \bigskip\bigskip\pause
  \paragraph{Dynamic programming (DP)} allows to recover the \emphase{optimal
  segmentation} with $\Ocal(Kn^2)$ complexity, provided that the
  contrast to be optimized is additive, e.g.
  $$
  \log p (\Ybf) = \sum_k \log p (\Ybf^k)
  $$
  \ra Assumption: Independence between the segments.
  
  \bigskip\bigskip\pause
  \paragraph{2-stage dynamic programming.} A 2-stage algorithm can be
  derived to recover the optimal segmentation of $M$
  \emphase{independent series}, under the same assumption
  (\refer{PLH11}).  

  }

%====================================================================
%====================================================================
\section {Joint segmentation of multiple series}
%====================================================================
\frame{ \frametitle{Segmentation of multiple series}
  \paragraph{Simultaneous segmentation} refers to the case where the
  breakpoints are the \emphase{same in all series} $m = 1..M$:
  $$
  \forall t \in I_k, \qquad Y_{tm}=\mu_{km} +
  F_{tm}, 
  $$
  where the residual vectors $\Fbf_t = (F_{tm})_m$ are
  i.i.d. $\Ncal_M(\Obf, \Sigmabf)$.
  
  \bigskip\bigskip\pause
  \paragraph{Joint segmentation} refers to the case where the
  breakpoints are \emphase{specific to each series}:
  $$
  \forall t \in I_k^{\emphase{m}}, \qquad Y_{tm}=\mu_{km} +
  F_{tm}, 
  \qquad
  \{\Fbf_t\}_t \text{ i.i.d. } \Ncal_M(\Obf, \Sigmabf).
  $$

  \bigskip\pause
  \begin{itemize}
  \item We \emphase{allow between series} correlation (e.g. spatial) :
    $\Sigmabf$. 
  \item We \emphase{forbid within series} correlation: $\{\Fbf_t\}_t$ i.i.d..
  \end{itemize}
  }

%====================================================================
\frame{ \frametitle{Graphical model perspective}
  
  \begin{tabular}{cc}
    \hspace{-.5cm}
    \begin{tabular}{p{.4\textwidth}}
      \vspace{-1cm}      
      \onslide+<1->{\paragraph{One series:} OK with
        classical DP. \\}
      \onslide+<2->{\medskip
        \paragraph{Independent series:}
        OK with 2-stage DP ($\Sigmabf = \sigma^2 \Ibf$). \\}
      \onslide+<3->{\medskip
        \paragraph{Simultaneous segmentation:} Same complexity as for
      one series \ra classical DP. \\} 
      \onslide+<4->{\medskip
        \paragraph{Joint segmentation:} Non-additivity
          of the likelihood\\ \ra DP cannot apply as such.} 
    \end{tabular}
    & 
    \hspace{-.5cm}
    \begin{tabular}{p{.5\textwidth}}
      %\vspace{1cm}      
      \begin{overprint}
        \onslide<1>
        \epsfig{file=../Figures/SegFA-ModGraph-X1.eps, clip=,
          width=0.7\textwidth}
        \onslide<2>
        \epsfig{file=../Figures/SegFA-ModGraph-X1m.eps, clip=,
          width=0.7\textwidth}
        \onslide<3>
        \epsfig{file=../Figures/SegFA-ModGraph-XcovSimult.eps, clip=,
          width=0.7\textwidth}
        \onslide<4>
        \epsfig{file=../Figures/SegFA-ModGraph-Xcov.eps, clip=,
          width=0.7\textwidth}
        \onslide<5>
        \epsfig{file=../Figures/SegFA-ModGraph-Xcov.eps, clip=,
          width=0.7\textwidth}
      \end{overprint}
      \onslide+<5>{
        \vspace{-1.25cm}
        \paragraph{Previous work:} Random effect model
        $$
        \Sigmabf = \sigma_U^2 \Jbf + \sigma_E^2 \Ibf
        $$
        (\refer{PLB11}).
        }
    \end{tabular}
  \end{tabular}  
  }

%====================================================================
%====================================================================
\section {Factor model for correlated series}
%====================================================================
\frame{ \frametitle{Factor model}

  \paragraph{Joint segmentation model.}
  $$
  Y_{tm} = \mu_{km} + F_{tm},   
  \qquad \forall t \in I_k^{\emphase{m}}
  \qquad \Fbf_t \text{ \emphase{i.i.d.} } \sim \Ncal_M(\Obf,
  \emphase{\Sigmabf}). 
  $$

  \bigskip\pause
  \paragraph{Factor model:} If $\Sigmabf$ can be written as
   $$
   \emphase{\Sigmabf = \Bbf \Bbf' + \sigma^2 \Ibf}, 
   \qquad \text{with } \Bbf = [b_{qm}]: M \times Q, 
   \qquad Q < M
   $$ \pause
   (always true for $Q = M-1$), then the model can be rewritten as
   $$
   Y_{tm} = \mu_{km} + \sum_{q=1}^Q Z_{tq} b_{qm} + E_{tm},   
   \qquad \forall t \in I_k^{m}
   $$
   where
   $
   \Zbf_t \text{ \emphase{i.i.d.} } \sim \Ncal_Q(\Obf, \emphase{\Ibf}), 
   \quad
   \Ebf_t \text{ \emphase{i.i.d.} } \sim \Ncal_M(\Obf,
   \emphase{\sigma^2 \Ibf}), 
   \quad
   (\{\Zbf_t\}, \{\Ebf_t\}) \text{ indep.}
   $

   \bigskip\bigskip
   The same trick is used in \refer{FKC09} in a different context.
  }

%====================================================================
\frame{ \frametitle{Breaking down dependency}

      \vspace{-1cm}      
  \begin{tabular}{cc}
    \hspace{-.5cm}
    \begin{tabular}{p{.4\textwidth}}
      \onslide+<1->{\paragraph{Marginal likelihood:} \\ $\log p(\Ybf)$ is
      not additive w.r.t. the segments $I_k^m$. \\}
      \onslide+<2->{\medskip
        \paragraph{Joint likelihood:} $\log p(\Ybf, \Zbf)$ is
      still not additive. \\}
      \onslide+<3->{\medskip
        \paragraph{Conditional likelihood:} \\
        $\log p(\Ybf|\Zbf)$ is additive w.r.t. the segments. \\} 
    \end{tabular}
    & 
    \hspace{-.5cm}
    \begin{tabular}{p{.5\textwidth}}
      \vspace{1cm}      
      \begin{overprint}
        \onslide<1>
        \epsfig{file=../Figures/SegFA-ModGraph-Xcov.eps, clip=,
          width=0.7\textwidth}
        \onslide<2>
        \epsfig{file=../Figures/SegFA-ModGraph-XZ.eps, clip=,
          width=0.7\textwidth}
        \onslide<3>
        \epsfig{file=../Figures/SegFA-ModGraph-XcondZ.eps, clip=,
          width=0.7\textwidth}
        \onslide<4>
        \epsfig{file=../Figures/SegFA-ModGraph-XcondZ.eps, clip=,
          width=0.7\textwidth}
      \end{overprint}
    \end{tabular}
  \end{tabular}
  
  \onslide+<4>{
    \vspace{-1.25cm}
    \paragraph{E-M algorithm:} Dynamic programming can be applied
    within the M step, when maximizing
    $$
    \Esp[\log p(\Ybf | \Zbf) | \Ybf].
    $$
    }
  
  }

%====================================================================
%====================================================================
\section {Inference}
%====================================================================
\frame{ \frametitle{EM algorithm (1/2)}

  \paragraph{Matrix form.} The model can be written as
  $$
  \Ybf = \Tbf \mubf + \Zbf \Bbf' + \Ebf
  $$
  where $\Tbf$ provides the breakpoint locations and $\mubf$
  contains the segment means.  \\
  \ra We need to estimate $\thetabf = (\Tbf, \mubf, \Bbf, \sigma^2)$.

  \bigskip\bigskip\ppause
  \paragraph{E-M strategy.} In presence of a latent variable $\Zbf$, it
  aims at maximizing
  $$
  \log p(\Ybf; \thetabf) = \Esp[\log p(\Ybf, \Zbf; \thetabf) | \Ybf] +
  \Hcal_{\thetabf}[p(\Zbf|\Ybf)]. 
  $$
  }

%====================================================================
\frame{ \frametitle{EM algorithm (2/2)}

  \paragraph{E-step:} Calculate the conditional distribution 
  $$
  p(\Zbf | \Ybf; \thetabf^h)
  $$
  e.g. the moments $\Esp_{\thetabf^h}(\Zbf_t | \Ybf)$ and
  $\Var_{\thetabf^h}(\Zbf_t | \Ybf)$. 

  \bigskip\bigskip\pause
  \paragraph{M-step:} Update the parameter value as
  $$
  \thetabf^{h+1} = \arg\max_{\thetabf} \Esp[\log p(\Ybf, \Zbf;
  \thetabf) | \Ybf]
  $$
  where 2 maximizations can be achieved separately:
  $$ \pause
  \begin{array}{rclcl}
    \Esp[\log p(\Ybf, \Zbf; \thetabf) | \Ybf] 
    & = & \Esp[\log p(\Ybf| \Zbf; \thetabf) | \Ybf] & + & \Esp[\log
    p(\Zbf; \thetabf) | \Ybf] \\ 
    \\
    & = & \Esp[\log p(\Ybf| \Zbf; \emphase{\Tbf, \mubf, \sigma^2}) |
    \Ybf] & + &
    \Esp[\log p(\Zbf; \emphase{\Bbf}) | \Ybf] \\
    \\
    & & 
    \text{(2 stage) dynamic}
    & & 
    \simeq \text{ diagonalization}
    \\
    & & 
    \text{programming,}
    & & 
    \text{problem.}
  \end{array}
   $$
  }

%====================================================================
\frame{ \frametitle{Model selection: a heuristic}

  \begin{description}
  \item[$Q$:] dimension of the latent vectors $\{\Zbf_t\}$.
  \item[$K = \sum_m K_m$:] total number of segment.
  \end{description}

  \bigskip\pause
  \paragraph{Factor model.} For a fixed given $K$:
  $$
  BIC_K(Q) = -2 \log p(\Ybf; \widehat{\Tbf\mubf}_K,
  \widehat{\Sigmabf}) + [1 + Q(2M-Q+1)/2] \log n.
  $$

  \bigskip\pause
  \paragraph{Segmentation.} \refer{ZhS07} proposed a modified BIC
  ($mBIC$) for one series. We use an extension similar to this of
  \refer{PLB11}:
  $$
  mBIC_Q(K)= f(K, \widehat{\Sigmabf}_Q, \{n_k^m\}).
  $$

  \bigskip
  \paragraph{Heuristic.}
  $$
  \widehat{Q}_K = \arg\max_Q BIC_K(Q), 
  \qquad \widehat{K} = \arg\max_K mBIC_{\widehat{Q}_K}(K), 
  \qquad \widehat{Q} = \widehat{Q}_{\widehat{K}}.
  $$
  }

%====================================================================
%====================================================================
\section {Simulation study}
%====================================================================
\frame{ \frametitle{Simulations: an example}
  \begin{tabular}{cc}
    \hspace{-.5cm}
    \begin{tabular}{p{.5\textwidth}}
      $M = 5$ stations, $n = 50$ times; \\
      $\mu_k^m \in \{-2, -1, 0, 1, 2\}$ %\\
      %$d(m, m') = $ distance from $m$ to $m'$; 
      \begin{eqnarray*}
        \Sigma(m, m') & = & (1-\lambda) \rho^{d(m, m')} \\
        & + & \lambda \sigma^2 \Ibb\{m = m'\}
      \end{eqnarray*}
      \\
      \onslide+<2->{
        \paragraph{Relative fit} as compared to $(K_0, Q_0)$: 
        $$
        \text{e.g.} \qquad \frac{RMSE(K, Q)}{RMSE(K_0,
          Q_0)} - 1 
        $$
      %\vspace{-.25cm}
        $$
        \begin{array}{lccc}
          \onslide+<3->{
            (\%) & (K_0, \widehat{Q}) & 
            (\widehat{K}, Q_0) & (\widehat{K}, \widehat{Q}) \\
            \hline
%         \emphase{\Tbf\mubf} & \emphase{-0.15} & -0.06  & 0.10 \\
%         \emphase{\Sigmabf} & 23.9 & \emphase{3.0} & 78.7 \\
            \emphase{\Tbf\mubf} & -16 & -8 & -17 \\
            }
          \onslide+<4->{
            \emphase{\Sigmabf} & -0.8 & -1.8 & -0.7
            }
        \end{array}
        $$
        }
    \end{tabular}
    & 
    \hspace{-.5cm}
    \begin{tabular}{p{.5\textwidth}}
      \onslide+<3->{
        \vspace{-2cm}
        \epsfig{file=../Figures/FaSegSimEx.n49.M5kmean3FitTmu.ps,
          clip=, angle=270, width=.48\textwidth}  \\
        $\qquad \quad K_0 = 17, \qquad \widehat{K} = 16$ \\ 
        }
      \onslide+<4->{
      \vspace{-.5cm}
      \epsfig{file=../Figures/FaSegSimEx.n49.M5kmean3FitSigma.ps,
        clip=, angle=270, width=.48\textwidth} \\
      \vspace{-.5cm}
      $\quad Q_0 = M-1 = 4, \qquad \widehat{Q} = 1$
      }
%       \begin{overprint}
%       \end{overprint}
    \end{tabular}
  \end{tabular}

  }

%====================================================================
\frame{ \frametitle{Estimation of $K$ and $Q$}

  \vspace{-.25cm}
  {2 conditions:} \textcolor{red}{$M=5, n=50$},
  \textcolor{green}{$M=10, n=100$}, $Q_0 = M-1$, increasing $\sigma^2$
  \\
  \pause
  \begin{tabular}{cc}
%     $\widehat{K}-K_0$ for $Q = \widehat{Q}$ &
%     $\widehat{K}-K_0$ for $Q = Q_0$ \\
    \epsfig{file=../Figures/FaSegSimJDS-EstimK-Qest.ps,
      clip=, angle=270, width=.45\textwidth} &
    \epsfig{file=../Figures/FaSegSimJDS-EstimK-Q0.ps,
      clip=, angle=270, width=.45\textwidth} 
  \end{tabular}\\
  \vspace{-.5cm} \pause
  \begin{tabular}{cc}
%     $\widehat{Q}-Q_0$ for $K = \widehat{K}$ &
%     $\widehat{Q}-Q_0$ for $K = K_0$ \\
    \epsfig{file=../Figures/FaSegSimJDS-EstimQ-Kest.ps,
      clip=, angle=270, width=.45\textwidth} &
    \epsfig{file=../Figures/FaSegSimJDS-EstimQ-K0.ps,
      clip=, angle=270, width=.45\textwidth} 
  \end{tabular}
  }

%====================================================================
\frame{ \frametitle{Fit of the segmentation $\Tbf\mubf$}

  \begin{tabular}{cc}
    \hspace{-.5cm}
    \begin{tabular}{p{.6\textwidth}}
      \paragraph{Fit of the segmentation.} Measured in terms of RMSE:
      $$
      RMSE = \|\widehat{\Tbf\mubf} - \Tbf\mubf \|
      $$
      relatively to the reference case: $(K_0, Q_0)$. \\
      \\
      \onslide+<2->{
        \paragraph{Results} are overall quite good:
        \begin{itemize}
        \item Results are competitive when using $(\widehat{K},
          \widehat{Q})$;          
          \onslide+<3->{
          \item Using the true $Q_0$ does not improve much;
            }
          \onslide+<4->{
          \item Using the right number of segments $K_0$ does improve the
            fit.
            }
        \end{itemize}
        }
    \end{tabular}
    & 
    \hspace{-.5cm}
    \begin{tabular}{p{.3\textwidth}}
      \vspace{-1.5cm}
        \onslide+<2->{
          \epsfig{file=../Figures/FaSegSimJDS-EstimTmu-KeQe.ps,
            clip=, angle=270, width=.37\textwidth} \\
          }
        \onslide+<3->{
          \vspace{-.75cm}
          \epsfig{file=../Figures/FaSegSimJDS-EstimTmu-KeQo.ps,
            clip=, angle=270, width=.37\textwidth} \\
          }
        \onslide+<4->{
          \vspace{-.75cm}
          \epsfig{file=../Figures/FaSegSimJDS-EstimTmu-KoQe.ps,
            clip=, angle=270, width=.37\textwidth} 
          }
%       \begin{overprint}
%       \end{overprint}
    \end{tabular}
  \end{tabular}
  }

%====================================================================
\frame{ \frametitle{Fit of the variance structure $\Sigmabf$}

  \begin{tabular}{cc}
    \hspace{-.5cm}
    \begin{tabular}{p{.6\textwidth}}
      \paragraph{Fit of the variance.} Also measured in terms of RMSE
      relatively to the reference case: $(K_0, Q_0)$. \\
      \\
      \onslide+<2->{
        \paragraph{Results} are overall not that good:
        \begin{itemize}
        \item Results are bad when using $(\widehat{K},
          \widehat{Q})$;          
          \onslide+<3->{
          \item Using the \emphase{true $Q_0$ does not improve much
              (...!)}; 
            }
          \onslide+<4->{
          \item Using the right number of segments $K_0$ really does
            improve the fit.
            }
        \end{itemize}
        }
    \end{tabular}
    & 
    \hspace{-.5cm}
    \begin{tabular}{p{.3\textwidth}}
      \vspace{-1.5cm}
        \onslide+<2->{
          \epsfig{file=../Figures/FaSegSimJDS-EstimSigma-KeQe.ps,
            clip=, angle=270, width=.37\textwidth} \\
          }
        \onslide+<3->{
          \vspace{-.75cm}
          \epsfig{file=../Figures/FaSegSimJDS-EstimSigma-KeQo.ps,
            clip=, angle=270, width=.37\textwidth} \\
          }
        \onslide+<4->{
          \vspace{-.75cm}
          \epsfig{file=../Figures/FaSegSimJDS-EstimSigma-KoQe.ps,
            clip=, angle=270, width=.37\textwidth} 
          }
%       \begin{overprint}
%       \end{overprint}
    \end{tabular}
  \end{tabular}
  }

%====================================================================
%====================================================================
\section {Comments \& future works}
%====================================================================
\frame{ \frametitle{Some practical limitations}
  \paragraph{(Practical) identifiability.} Although the model is
  theoretical identifiable, simultaneous breakpoints in a large
  fraction of series can be \emphase{confounded with the 'random
    effect'} $\Zbf_t$ (see \refer{PLH11}). \\
  \ra Observed in French temperature series.

  \bigskip\bigskip
  \paragraph{Missing data.} Calculations can be made ... but formulas
  are much uglier.

  \bigskip\bigskip
  \paragraph{Exploration of the $(K, Q)$ space.} The 2 BIC
  criteria \emphase{cannot be combined into a single one}, as they do
  not rely on the same contrast (likelihood vs within/between sum of
  squares). \\ 
  \ra Time consuming exploration over a grid for $(K, Q)$.

  }

%====================================================================
\frame{ \frametitle{Some theoretical issues}

  \paragraph{Model selection.} The proposed strategy is only
  heuristic. No criterion with theoretical guaranty exist at this time
  for such a dependency structure.


  \bigskip\bigskip
  \paragraph{Within series correlation.} 
  \begin{itemize}
  \item To which extent does it matter as for the segmentation?
  \item Can we manage it computationally? 
  \item Doable for heterogeneous autoregressive models $AR(\mu_k,
    \phi_k, \sigma_k^2)$.
  \item Ongoing work with S. Chakar for autoregressive models.
  \end{itemize}

  \bigskip\bigskip
  \paragraph{Post-doc position.} \\
  ANR project CNV-Maize: Determination of genetic variant in maize
  varieties. \\
  }

%====================================================================
\frame{ \frametitle{}
  \tiny{
    \bibliography{/Biblio/AST,/Biblio/ARC}
    \bibliographystyle{/Latex/astats}
    }
  }

%====================================================================
%====================================================================
\end{document}
%====================================================================
%====================================================================

  \begin{tabular}{cc}
    \hspace{-.5cm}
    \begin{tabular}{p{.5\textwidth}}
    \end{tabular}
    & 
    \hspace{-.5cm}
    \begin{tabular}{p{.5\textwidth}}
      \begin{overprint}
      \end{overprint}
    \end{tabular}
  \end{tabular}


