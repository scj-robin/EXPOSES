\documentclass[11pt,a4paper,english]{article}

\usepackage{amsmath, amsfonts, amssymb}
\usepackage{enumerate}
\usepackage{epsfig}
\usepackage[french]{babel}
\usepackage[latin1]{inputenc}
\def\1{1\!{\rm l}}
\def\0{0\!{\rm 0}}
\textwidth 16cm
\textheight 22cm
\topmargin 0cm
\oddsidemargin 0cm
\evensidemargin 0cm

%\title{Segmentation et mod�le lin�aire}

\newcommand{\ymt}{y_{mt}}
\newcommand{\Ymt}{Y_{mt}}
\newcommand{\ec}[1]{\mathbb{E}_{\phi^{(h)}}\left\{#1 |\mathbf{Y} \right\}}
\newcommand{\vc}[1]{\mathbb{V}_{\phi^{(h)}}\left\{#1 |\mathbf{Y} \right\}}

\newcommand{\alphabf}{\text{\mathversion{bold}{$\alpha$}}}
\newcommand{\thetabf}{\text{\mathversion{bold}{$\theta$}}}
\newcommand{\mubf}{\text{\mathversion{bold}{$\mu$}}}
\newcommand{\xbf}{\mathbf{x}}
\newcommand{\Xbf}{\mathbf{X}}
\newcommand{\Ybf}{\mathbf{Y}}
\newcommand{\Zbf}{\mathbf{Z}}
\newcommand{\Bbf}{\mathbf{B}}
\newcommand{\Bbfp}{\mathbf{B}^{\prime}}
\newcommand{\Zbfp}{\mathbf{Z}^{\prime}}
\newcommand{\Wbf}{\mathbf{W}}
\newcommand{\ZA}{\mathbf{Z}_A}
\newcommand{\ZB}{\mathbf{Z}_B}
\newcommand{\WB}{\mathbf{W}_B}
\newcommand{\WA}{\mathbf{W}_A}
\newcommand{\WAB}{\mathbf{W}_{AB}}
\newcommand{\UA}{\mathbf{U}_A}
\newcommand{\UB}{\mathbf{U}_B}
\newcommand{\Vbf}{\mathbf{V}}
\newcommand{\Rbf}{\mathbf{R}}
\newcommand{\Ubf}{\mathbf{U}}
\newcommand{\Ebf}{\mathbf{E}}
\newcommand{\Gbf}{\mathbf{G}}
\newcommand{\Tbf}{\mathbf{T}}
\newcommand{\Ibf}{\mathbf{I}}
\newcommand{\RMSE}{\mbox{RMSE}}
\newcommand{\FPR}{\mbox{FNR}}
\newcommand{\FNR}{\mbox{FNR}}
\newcommand{\LogL}{\mathcal{L}}
\def\1{1\!{\rm l}}
\newtheorem{prop}{Proposition}
%\newproof{pf}{Proof}


\def\argmin{\mathop{\mathrm{argmin}}}

%%%%%%%%%%%%%%%%%%%%%%%%%%%%%%%%%%%%%%%%%%%%%%%%%%%%%%%
\begin{document}

%%%%%%%%%%%%%%%%%%%%%%%%%%%%%%%%%%%%%%%%%%%%%%%%%%%%%%%
\section{Data, notations and model}

\paragraph{Data and notations.} We have $M$ climatic series with $n_m$ points
each. We note $y_{tm}$ the observed temperature of the station $m$
at time $t$. The total number of observations is $N=\sum_{m=1}^M
n_m$. In the following, we consider that $n_m=n$ whatever the
station. The data $y$ are modeled by a random Gaussian process $Y$
with size $[n \times M]$. In general we note $A_t$ to be the row
vector with size $M$ of the matrix $A$ and $A^m$ its column vector
with size $n$. Thus $Y^m$ represents the temperature series of the
station $m$ (so for each time $t$) and $Y_t$ the temperatures of the
$M$ series at a fixed time $t$. \\


\paragraph{Segmentation.} We consider here that each series has its
own segmentation: the mean of the series $\{Y_{tm}\}_t$ is subject
to $K_{m}-1$ abrupt changes at breakpoints $\{t_{k}^{m}\}$ for
series $m$, (with convention $t^m_0=0$ and $t^m_{K_m}=n_{\max}$) and
is constant between two breakpoints within the interval $I_{k}^{m} =
]t_{k-1}^{m}, t_{k}^{m}]$. In the following we denote by
$K=\sum_{m}^M K_{m}$ the total number of segments and $n_k^{m} =
t_{k}^{m}- t_{k-1}^{m}$ the length of the segment $k$ for series $m$
($k=1,\ldots,K_m$). The segmentation model is written as follows:
\begin{equation} \label{eq:model1}
Y_{tm}=\mu_{km} +F_{tm} \ \  \forall t \in I_k^{m}
\end{equation}
where the error vectors $\{F_t\}_t$ follows a centered Gaussian with
a covariance matrix to be specified. Denote $\Sigma$ this matrix:
\begin{equation*}
\mathbb{V}(F_t)=\Sigma.
\end{equation*}

\paragraph{Spatial correlations between series.} We want to take
into account spatial correlations that can exit between series. As
example, the corresponding correlation matrix can be of the form
$e^{-d(m,m^{\prime})}$ where $d(m,m^{\prime})$ is the distance
between the series $m$ and $m^{\prime}$. However this correlation
structure (or otherwise structure) hampers the use of the DP
algorithm to obtain the best
segmentation. \\
We propose here to consider the factor analysis model framework.
That consists in identifying a linear space of $Q$ (to be fixed)
random vectors that captures the dependence among the data, here the
series (ref). In others words, we set
$$
\Sigma=\Bbf \mathbb{V}(\Zbf) \Bbfp+\psi,
$$
where $\Zbf$ corresponds to the $Q$ random vectors and $B$ the
associated coefficients. The dependence between series is then free
from any spatial structure. In the above decomposition, $\Bbf
\mathbb{V}(\Zbf) \Bbfp$ referred to the shared variance and $\psi$
to the specific one. Remark that in this case $Q$ can not be equal
to the number of series $M$.\\




\paragraph{Model.} With the previous decomposition of the variability, the model (\ref{eq:model1}) can be rewritten
using a mixed linear model as follows:
\begin{equation*}
Y_{tm}=\mu_{km} + \sum_{q=1}^{M} Z_{tq} b_{qm}+E_{tm} \ \  \forall t
\in I_k^{m}
\end{equation*}
with $E_{t}$ a centered Gaussian noise with variance matrix $\psi$
and with $Z_{t}$ a centered Gaussian with variance $I_Q$. Here
$\psi$ is supposed to be diagonal $\psi=\sigma^2 I_M$ (otherwise we
can not use the DP algorithm for the segmentation parameter
estimation). Using the matricial formulation of linear models, we
have
\begin{equation*}
\Ybf=\Tbf \mubf +\Zbf \Bbfp+\Ebf
\end{equation*}
where
\begin{itemize}
\item $\Ybf$, with size $[n \times M]$, stands for the observed data,
\item $\Tbf$ is the incidence matrix of breakpoints with size $[n\times
K]$,
\begin{eqnarray*}
{T^m} & = & \text{Bloc} \left[ \1_{n_{K_m}^{m}} \right],
\end{eqnarray*}
and
\begin{eqnarray*}
{\Tbf} & = &\left[ T^1 \ T^2 \ \ldots \ T^M \right].
\end{eqnarray*}
\item $\mubf$ the means with size $[K\times M]$ (and $\mu_{k}^m$ the mean of the segment $k$ for series $m$) such that
\begin{eqnarray*}
{\mu^{m}} & = & \text{Bloc} \left[ \mu_{K_m}^m \right],
\end{eqnarray*}
and
\begin{eqnarray*}
{\mubf} & = &\left[ \mu^1  \ \mu^2  \ \ldots  \ \mu^M \right].
\end{eqnarray*}
\item $\Zbf$ with size $[n\times Q]$ and $\Bbf$
with size $[M \times Q]$,
\item $\Ebf$ with size $[n\times M]$. \\
\end{itemize}
So we have that $\Sigma$ is structured as follows:
\begin{eqnarray} \label{eq:variance_decomposition}
\Sigma=\Bbf \Bbfp+\psi.
\end{eqnarray}
Moreover, as in a classical mixed linear model, $Z^q$ and $E^m$ are
supposed to be independent. The main difference between this model
and a classical mixed linear model is that both the incidence
matrices $\Tbf$ and $\Bbf$ are unknown. \\



\section{Estimation using the EM algorithm.}


In order to get the MLE in the EM framework, $\Zbf$ represents the
missing data (hidden and unobserved), and parameters of the model
are $\phi=(\Tbf,\mubf,\sigma^2,\Bbf)$. In this setting, the
complete-data log-likelihood is:
\begin{eqnarray*}
\log \mathcal{L}(\Ybf,\Zbf; \phi) = \log \mathcal{L}_0(\Ybf|\Zbf;
\phi) + \log \mathcal{L}_1(\Zbf).
\end{eqnarray*}
Since the distribution of $Z$ does not depend on the parameters
$\phi$, only the first term will be considered which is written:
\begin{eqnarray*}
-2 \log \mathcal{L}_0(\Ybf|\Zbf; \phi)=N  \log{(2 \pi)}+ n
\log{(|\psi|)}+\sum_{t=1}^n \|Y_t - \mu_{k(t)} - Z_t
\Bbfp\|^2_{\psi^{-1}},
\end{eqnarray*}
then its conditional expectation, $Q_0(\phi; \phi^{(h)})=\ec{\log
\mathcal{L}_0(\Ybf|\Zbf; \phi)}$ where $\mathbb{E}_{\phi}\{\cdot\}$
the expectation operator using $\phi$ as the parameter value and
$\mathbb{V}_{\phi}\{\cdot\}$ the corresponding variance, is
\begin{eqnarray*}
-2 Q_0(\phi; \phi^{(h)}) &=& N  \log{(2 \pi)}+ n
\log{(|\psi|)}+\sum_{t=1}^n \left [ \|Y_t - \mu_{k(t)} -
\widehat{\Zbf}_t^{(h)} \Bbfp\|^2_{\psi^{-1}} +\text{Tr} \left( \Bbfp
  \psi^{-1} \Bbf \Wbf_t^{(h)} \right) \right ],
\end{eqnarray*}
where $\widehat{\Zbf}_t^{(h)} = \ec{\Zbf_t}$, $\text{Tr}(A)$ for the
trace of matrix $A$, $|A|$ for its determinant and where
$\Wbf_t^{(h)}=\vc{\Zbf_t}$. \\
%The second $\log$-likelihood is written:
%\begin{eqnarray*}
%-2 \log \mathcal{L}_1(\Zbf)=N  \log{(2 \pi)}+ \sum_{t=1}^n \|Z_t
%\|^2,
%\end{eqnarray*}
%then its conditional version is
%\begin{eqnarray*} -2 Q_1(\phi; \phi^{(h)}) &=& N  \log{(2 \pi)}+
%\sum_{t=1}^n \left [ \widehat{\Zbf}_t^{(h)} \widehat{\Zbfp}_t^{(h)}
%+\text{Tr} \left( \Wbf_t^{(h)} \right) \right ].
%\end{eqnarray*}

\paragraph{E-step} This step consists in the calculation of the
conditional expectation which only requires the calculation of
$\widehat{\Zbf}$ and $\Wbf$: at iteration $(h+1)$, we get\\
%$$
%\begin{cases}
%\widehat{\Zbf}_t^{(h+1)} = E_t^{(h)} \psi^{-1,(h)} \Bbf^{(h)}
%\Wbf_t^{(h)}/{\sigma^{2,(h)}}, \\
%\Wbf_t^{(h+1)}  = \left (I_M+ \Bbf'^{(h)} \psi^{-1,(h)} \Bbf^{(h)}
%\right )^{-1}.
%\end{cases}
%$$
$$
\begin{cases}
\widehat{\Zbf}_t^{(h+1)} = \tilde{Y}_t^{(h)} \Bbf^{(h)}
\Wbf_t^{(h)}/{\sigma^{2,(h)}}, \\
\Wbf_t^{(h+1)}  = \left (I_M+ \Bbf'^{(h)} \Bbf^{(h)} /\sigma^{2,(h)}
\right )^{-1}.
\end{cases}
$$
where $\tilde{Y}_t^{(h)}=Y_t-\mu_{k(t)}^{(h)}$.
\paragraph{M-step} This step consists in the estimation of the
parameters by maximizing the obtained conditional expectation.
\begin{itemize}
\item Estimation of the variance component $\sigma^{2}$:
%\begin{eqnarray*}
%\psi^{(h+1)}&=&\arg \max_{\psi} Q_0,\\
%&=& \frac{1}{n} \sum_{t=1}^n (Y_t-\mu_{k(t)}^{(h)})'
%(Y_t-\mu_{k(t)}^{(h)})- \frac{1}{n} \sum_{t=1}^n
%(Y_t-\mu_{k(t)}^{(h)})' \widehat{\Zbf}_t^{(h+1)}  \Bbf'^{(h)}
%\end{eqnarray*}
%In the case where $\psi=\sigma^2 I_M$, we get
\begin{eqnarray*}
\sigma^{2,(h+1)} &=& \frac{1}{N} \sum_{t=1}^n \left [ E_t^{(h)}
E_t'^{(h)}+ \text{Tr} \left( \Bbf'^{(h)}
 \Bbf^{(h)}  \Wbf_t^{(h+1)} \right)\right ],
\end{eqnarray*}
where $E_t^{(h)}=Y_t-\mu_{k(t)}^{(h)}-\widehat{\Zbf}_t^{(h)}
\Bbf'^{(h)}$.

\item Estimation of $\Bbf$.
\begin{eqnarray*}
\Bbf^{(h+1)}= \sum_{t=1}^n (Y_t-\mu_{k(t)}^{(h)})'
\widehat{\Zbf}_t^{(h+1)} \ \left [ \sum_{t=1}^n
(\widehat{Z}_t'^{(h+1)} \widehat{Z}_t^{(h+1)}+ W_t^{(h+1)})\right
]^{-1}.
\end{eqnarray*}

\item Estimation of the segmentation parameters $\Tbf \mubf$.
\begin{eqnarray*}
\left\{ \Tbf^{(h+1)},\mubf^{(h+1)} \right\} & = & \arg
\max_{\Tbf,\mubf} Q_0, \\
&=& \arg \max_{\Tbf,\mubf} \sum_{t=1}^n \| Y_t- (\Tbf \mubf)_t-
\widehat{\Zbf}^{(h+1)}_t \Bbf'^{(h+1)}\|^2_{\psi^{-1,(h+1)}}, \\
&=& \arg \max_{\Tbf,\mubf} \sum_{t=1}^n \| \tilde{\tilde{Y}}_t-
(\Tbf \mubf)_t\|^2_{\psi^{-1,(h+1)}}.
\end{eqnarray*}
where $\tilde{\tilde{Y}}_t=Y_t- \widehat{\Zbf}_t^{(h+1)}
\Bbf'^{(h+1)}$. Since $\psi$ is diagonal the DP (in particular the
two-stages DP) can be used to obtain the segmentation parameter
estimations. \\

%\max_{\Tbf,\mubf} \sum_{m=1}^M \sum_{k=1}^{K_m} \sum_{t \in I^k_m}
%\left (\tilde{Y}_{tm}-\mu_{km} \right )^2,




It is well known that the EM algorithm depends strongly on the
initial parameters which is here the segmentation one. This
phenomenon often leads to erratic behavior of the likelihood and
consequently hampers the use of model selection criteria based on
this likelihood. \\
In the application of the procedure on the real data, this
phenomenon is almost often observed on the likelihood with respect
to $q$ for each considered value of $K$. To solve this problem, for
the values of $q$ for which the likelihood does not increase, we
propose to perform the EM algorithm by considering as initial
parameters the obtained segmentation ones of their neighbors. \\

\end{itemize}

\section{Model selection}

Here either the number of factors $q$ and the number of segments $K$
should be estimated. The joint estimation of two parameters is not
classical. We propose here an heuristic to select both these two
parameters. First, for each $K$, $q$ is selected by using the BIC
criterion:
\begin{eqnarray*}
\hat{q}_K&=&\argmin_{q \in \{1,\ldots,M-1\}} BIC(q_K),\\
&=& \argmin_{q \in \{1,\ldots,M-1\}}-2 \LogL(\widehat{\Tbf \mubf}_K,
\hat{\Sigma}_q)+Dq \log{(n)},
\end{eqnarray*}
where $\LogL(\widehat{\Tbf \mubf}_K, \hat{\Sigma}_q)$ is the
$\log$-likelihood calculated at its maximum for a fixed $K$ and a
fixed $q$, and $D_q$ is the number of parameters in a model with $q$
factors, $D_q=\frac{q}{2}(2 M -q+1)+1$. Indeed for the variance
components, according to the variance decomposition (cf equation
(\ref{eq:variance_decomposition})), the number of parameters are $M
\times q$ for $B$ and one for $\sigma^2$. Moreover, with the
orthogonality condition on $B$, only $M q -q(q-1)/2$ need to be
estimated. Remark that the number of segments is not be considered
since does not affected
the selection.\\
To select the number of segments, we use the modified BIC proposed
by \cite{ZhS07} adapted to the joint segmentation by \cite{PLBR11}
\begin{eqnarray*}
mBIC_{\text{JointSeg}}(K)&= &\left ( \frac{N+1}{2} \right
)\log{SS_{\text{all}}} -\left ( \frac{N-K+1}{2} \right )
\log{SS_{\text{wg}}(\hat{t})} + \log{\left [\Gamma \left
(\frac{N-K+1}{2} \right)\right ]}
\\
&&   -\frac{1}{2} \sum_{m=1}^M \sum_{k=1}^{k_m} \log
{\hat{n}^m_k}+\left ( \frac{1}{2}-(K-M) \right ) \log{(N)},
\end{eqnarray*}
where
\begin{eqnarray*}
&&SS_{\text{wg}}(\hat{t})=\sum_{t=1}^n (\Ybf_t-\hat{\mu}_{k(t)})
\hat{\Sigma}_q^{-1}(\Ybf_t-\hat{\mu}_{k(t)})', \\
&& SS_{\text{all}}=\sum_{t=1}^n (\Ybf_t-\bar{Y})
\hat{\Sigma}_q^{-1}(\Ybf_t-\bar{Y})',
\end{eqnarray*}
with $\hat{n}^m_k$ is the length of segment $k$ in profile $m$
($\hat{n}^m_k=\hat{t}^m_k-\hat{t}^m_{k-1}+1$), and $\hat{mu}_{k(t)}$
is a vector of size $M$ with the component $m$ is
$\bar{y}_{mk}=\frac{1}{\hat{n}^m_k}
\sum_{t=\hat{t}^m_{k-1}+1}^{\hat{t}^m_{k}} y_m(t)$ if $t \in
\hat{I}^m_k$.






%We use two criteria: the ones developed by \cite{Lav05} and
%\cite{L05} which the penalty depends on the number of parameters in
%a model with $K$ segments, denoted by $D_K$. Here
%$$
%D_K=K+\frac{Q}{2}(2 M -Q+1)+1.
%$$
%Indeed, for the mean parameters, we have $K$ means to estimate and
%for the variance components, according to the variance decomposition
%(cf equation (\ref{eq:variance_decomposition})), the number of
%parameters are $M \times Q$ for $B$ and one for $\sigma^2$.
%Moreover, with the orthogonality condition on $B$, only $M Q
%-Q(Q-1)/2$ need to be estimated. \\


%The criterion proposed by \cite{Lav05} is
%begin{equation*} Crit_{\text{Lav}}(K)= -2 \log {\mathcal{L}_K(\Ybf;
%\widehat{\phi})}+\beta D_K
%\end{equation*}
%where $\log { \mathcal{L}_K(\Ybf; \widehat{\phi})}$ stands for the
%log-likelihood calculated at its maximum. The constant $\beta$ is
%chosen using an adaptative method which involves a threshold $s$. As
%Lavielle suggests $s \in [0.5, 0.75]$, we used $s=0.7$ throughout
%the simulation study. \\

%The one proposed by \cite{L05} is
%\begin{equation*}
%Crit_{\text{Leb}}(K)=-2 \log {\mathcal{L}_K(\Ybf;
%\widehat{\phi})}+\alpha \left[ 5 D_K +2 (K-M) \log \left
%(\frac{N-M}{K-M} \right) \right]
%\end{equation*}
%This penalty also depends on a constant $\alpha$ which can be
%calibrated in practice (cf \cite{BM04}).



\section{Simulation study}


\section{Application to temperature data series}

\bibliography{modlin}
\bibliographystyle{amsplain}


\end{document}
