\documentclass[11pt,a4paper,french]{article}

\usepackage{amsmath, amsfonts, amssymb}
\usepackage{enumerate}
\usepackage{epsfig}
\usepackage[french]{babel}
\usepackage[latin1]{inputenc}
\def\1{1\!{\rm l}}
\def\0{0\!{\rm 0}}
\textwidth 16cm \textheight 24cm \topmargin -1cm \oddsidemargin 0cm
\evensidemargin 0cm


%Proposition Workshop on Statistical Analysis of Post Genomic Data \\
%15-16 January 2007
\newcommand{\yglt}{y_{g \ell t}}
\newcommand{\Yglt}{Y_{g \ell t}}
\newcommand{\ec}[1]{\mathbb{E}_{\phi^{(h)}}\left\{#1 |\mathbf{Y} \right\}}
\newcommand{\vc}[1]{\mathbb{V}_{\phi^{(h)}}\left\{#1 |\mathbf{Y} \right\}}

\newcommand{\alphabf}{\text{\mathversion{bold}{$\alpha$}}}
\newcommand{\thetabf}{\text{\mathversion{bold}{$\theta$}}}
\newcommand{\Xbf}{\mathbf{X}}
\newcommand{\Ybf}{\mathbf{Y}}
\newcommand{\Zbf}{\mathbf{Z}}
\newcommand{\Ubf}{\mathbf{U}}
\newcommand{\Ebf}{\mathbf{E}}
\newcommand{\Tbf}{\mathbf{T}}
\newcommand{\mubf}{\mathbf{\mu}}

\title{\large {\bf {Mod�le lin�aire mixte avec segmentation :
application aux puces CGH}}}

\author{ \underline{E. Lebarbier}, S. Robin \\
      \footnotesize {UMR INA P-G/ENGREF/INRA MIA 518, Paris, France} \\
      F. Picard \\
      \footnotesize {UMR CNRS-8071/INRA-1152/Universit\'e d'\'Evry, \'Evry, France}\\
      Eva Budinsk\`a \\
      \footnotesize {Centre of Biostatistics and Analyses, Faculty of Science and Faculty
of Medicine,}\\
 \footnotesize {Masaryk University, Brno} }


\date{}

%%%%%%%%%%%%%%%%%%%%%%%%%%%%%%%%%%%%%%%%%%%%%%%%%%%%%%%
\begin{document}
%%%%%%%%%%%%%%%%%%%%%%%%%%%%%%%%%%%%%%%%%%%%%%%%%%%%%%%\

\maketitle

L'objectif des experiences de microarrays CGH (Comparative Genomic
Hybridization) est de d�tecter des aberrations chromosomiques
(deletion ou amplification) en comptant le nombre relatif de copies
d'une s�quence d'ADN le long du g�nome par rapport � une r�f�rence.
Le profil CGH ainsi obtenu peut �tre vu comme une succession de
segments dont le nombre de copies est homog�ne en moyenne. Le but de
l'analyse statistique consiste donc � d�tecter et localiser ces
segments. \\

Les nouvelles exp�riences de microarrays CGH am�nent d�sormais �
analyser simultan�ment plusieurs �chantillons biologiques. Dans le
cadre de l'�tude du cancer, l'objectif est d'identifier des
aberrations r�currentes et de les relier aux stades de progression
de la maladie. Pour cette analyse simultan�e de plusieurs profils
CGH, nous proposons un mod�le lin�aire mixte avec segmentation.
L'introduction d'un effet al�atoire dans un mod�le de segmentation
permet de mod�liser la structure de variance/covariance des
observations en prenant notamment en compte la d�pendance qu'il
peut exister entre les profils de patients � une position donn�e. \\
Nous proposons une strat�gie algorithmique pour estimer les
param�tres par maximum de vraisemblance. Cette strat�gie consiste �
alterner l'algorithme EM avec un algorithme de programmation
dynamique.\\
Cette m�thode est appliqu�e pour l'analyse de donn�es CGH dans le
cas du cancer (Nakao et al. (2004))











\end{document}
