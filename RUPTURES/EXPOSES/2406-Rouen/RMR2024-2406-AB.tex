\documentclass[french,english]{beamer}
%\usepackage{mathpazo}
\renewcommand{\sfdefault}{lmss}
\usepackage[T1]{fontenc}
\usepackage[utf8]{inputenc}
\setcounter{secnumdepth}{2}
\setcounter{tocdepth}{2}
\usepackage{amsfonts,amsmath,amssymb,amsthm,enumerate,mathtools,color,hyperref,bbm,tabularx,ifthen,twoopt,mathabx,multirow}
\usepackage{mathrsfs}
\usepackage{graphicx}
\usepackage{xcolor}
\usepackage{array}
\usepackage[authoryear]{natbib}
\PassOptionsToPackage{normalem}{ulem}
\usepackage{ulem}
\usepackage{tikz}
\usepackage{multirow}
\makeatletter

%%%%%%%%%%%%%%%%%%%%%%%%%%%%%% LyX specific LaTeX commands.
%% Because html converters don't know tabularnewline
\providecommand{\tabularnewline}{\\}

%%%%%%%%%%%%%%%%%%%%%%%%%%%%%% Textclass specific LaTeX commands.
 % this default might be overridden by plain title style

%%%%%%%%%%%%%%%%%%%%%%%%%%%%%% User specified LaTeX commands.
\newcolumntype{C}[1]{>{\centering\arraybackslash}m{#1}}
\newcommand{\nodesize}{2em}
\newcommand{\edgeunit}{4*\nodesize}
\definecolor{burgundy}{rgb}{0.5, 0.0, 0.13}
\definecolor{cadmiumgreen}{rgb}{0.0, 0.50, 0.24}
\definecolor{skobeloff}{rgb}{0.0, 0.48, 0.45}
\definecolor{dimgray}{rgb}{0.95, 0.95, 0.95}
\definecolor{twilightlavender}{rgb}{0.54, 0.29, 0.42}
\definecolor{eggplant}{rgb}{0.38, 0.25, 0.32}
\colorlet{eggplant_light}{eggplant!50!white}
\definecolor{viridian}{rgb}{0.25, 0.51, 0.43}
\definecolor{warmblack}{rgb}{0.0, 0.26, 0.26}
\definecolor{darkorange}{rgb}{1.0, 0.55, 0.0}
\definecolor{darkpastelpurple}{rgb}{0.59, 0.44, 0.84}
\usetheme{boxes}
\useinnertheme{rectangles}
\setbeamertemplate{blocks}[default]
\setbeamercolor{block body alerted}{fg=black,bg=dimgray}
\setbeamercolor{block body}{fg=black,bg=dimgray}
%\setbeamercolor{block title alerted}{fg=black,bg=white }
\setbeamercolor{block title}{fg=warmblack,bg=black!10}
\setbeamercolor{itemize item}{fg=skobeloff}
\setbeamercolor{itemize subitem}{fg=warmblack}
\setbeamercolor{palette tertiary}{fg=white,bg=warmblack}
\setbeamercolor{palette secondary}{fg=warmblack,bg=lightgray}
\setbeamertemplate{itemize subitem}[ast]
\setbeamercolor{enumerate item}{fg=warmblack,bg=warmblack}
\setbeamertemplate{enumerate item}[triangle]
%\setbeamertemplate{itemize subitem}{\tiny\raise3.5pt\hbox{\donotcoloroutermaths$\blacktriangleright$}}

\setbeamercolor{palette primary}{fg=white,bg=dimgray}
\setbeamercolor{structure}{fg=warmblack,bg=black!10}
\setbeamercolor{title}{fg=viridian,bg=white}
\setbeamercolor{frametitle}{fg=viridian,bg=white}
\setbeamertemplate{bibliography item}{[\theenumiv]}
\setbeamertemplate{navigation symbols}{}
\setbeamertemplate{title page}[default]





\usepackage[]{graphicx}
\usepackage[]{color}
%% maxwidth is the original width if it is less than linewidth
%% otherwise use linewidth (to make sure the graphics do not exceed the margin)




\usepackage{multicol}
\usepackage{fix-cm}




\def\PE{\mathbb{E}}
\def\PP{\mathbb{P}}
\def\Var{\mathop{\rm Var}\nolimits}
\def\Tr{\mathop{\rm Tr}\nolimits}
\def\rmd{\mathrm{d}}
\def \1{\mathbbm{1}}
\def\Y{\mathbf{Y}}
\def\betabold{\boldsymbol{\beta}}
\def\X{\mathbf{X}}
\def\u{\mathbf{u}}
\def\Z{\mathbf{Z}}
\def\R{\mathbf{R}}
\def\W{\mathbf{W}}
\def\D{\mathbf{D}}
\def\G{\mathbf{G}}
\def\g{\mathbf{g}}
\def\B{\mathbf{B}}
\def\M{\mathbf{M}}
\def\H{\mathbf{H}}
\def\U{\mathbf{U}}
\def\T{\mathbf{T}}
\def\J{\mathbf{J}}
\def\e{\mathbf{e}}
\def\v{\mathbf{v}}
\def\t{\mathbf{t}}
\def\V{\mathbf{V}}
\def\L{\mathbf{L}}
\def\l{\mathbf{l}}
\def\w{\mathbf{w}}
\def\eps{\boldsymbol{\varepsilon}}
\def\Id{\textrm{Id}}
\def\rset{\mathbb{R}}
\def \1{\mathbbm{1}}
\def\II{{\mathds 1}}

\title[Modeling spatial interactions] % (optional, use only with long paper titles)
{
Markov switching Hawkes process with applications in ecology
}
\institute[LPSM]{LPSM, Sorbonne Université}
\date{Rencontres statistiques du CEREMADE, 6 mai 2024}

%\author[Anna Bonnet] % (optional, use only with lots of authors)
%{Anna Bonnet, Franck Picard, Vincent Rivoirard}

\author{Anna Bonnet \and Stéphane Robin}


%\institute[AgroParisTech]{}
% - Give the names in the same order as the appear in the paper.



\begin{document}

\begin{frame}
\maketitle

\end{frame}


\begin{frame}{Point process}

\begin{minipage}{6cm}
\includegraphics[scale=0.1,trim=130mm 900mm 20mm 150mm]{figures/random_univ.png}
\end{minipage}
\begin{minipage}{4.5cm}
\begin{itemize}
\item $(T_k)_{k \geq 1}$ a random collection of points
\item Intensity function $\lambda(t)$: immediate probability of observing an event at time $t$
\end{itemize}
\end{minipage}

\vspace{0.5 cm}
\begin{block}{Hawkes process}
    \begin{itemize}
      The intensity function $\lambda(t)$ depends on the past history $\mathcal{H}_t$
    \end{itemize}
\end{block}



\end{frame}


\begin{frame}{Linear univariate Hawkes process (Hawkes, 1971)}

\vspace{2cm}

\includegraphics[scale=0.1,trim=40mm 900mm 20mm 250mm]{figures/random_univ_Ht.png}


\begin{block}{(Conditional) intensity function}
$$\lambda(t \vert \mathcal{H}_t)= \lambda(t)= \lambda_0 + \underset{T_k < t}{\sum} h(t-T_k)$$
\end{block}



\begin{itemize}
\item $\lambda_0$ is called the baseline 
\item $h$ describes the influence of past events
\end{itemize}
\end{frame}

\begin{frame}{Self-exciting exponential Hawkes process}


$$ \lambda(t)= \lambda_0 + \underset{T_k < t}{\sum} a e^{-b(t-T_k)}$$



\begin{minipage}{5.6cm}

\vspace{1cm}
\includegraphics[scale=0.13,trim=10mm 120mm 0mm 0mm,clip]{figures/point_process.png} \\


\end{minipage}
\begin{minipage}{5cm}
\begin{alertblock}{}
\small
\begin{itemize}
\item Classical exponential kernel function $h(t)= a e^{-b t}$
\item $ a \geq 0$, which ensures that $\lambda$ is non negative 
%\item This model is called \textbf{self-exciting} process or \textbf{linear} Hawkes process
\item An event increases the probability of observing another event
\item Stationarity condition: $a/b <1$
\item Applications: Sismology, epidemiology, neuroscience...

\end{itemize}
\end{alertblock}

\end{minipage}

%\begin{itemize}
%\item Cluster representation and results from branching processes
%\end{itemize}
%
%\begin{center}
%\includegraphics[scale=0.15]{figures/cluster_representation.png}
%\end{center}
\end{frame}



\begin{frame}{Application in ecology}

\begin{itemize}
    \item Sounds emission of an animal
    \item Observations: times when each call is recorded (one process for each species)

\begin{itemize}
    \item  Bats  monitoring (Vigie-Chiro program, CESCO-MNHN)


\begin{center}
      \includegraphics[scale=0.38,trim=0mm 5mm 5mm 0mm, clip]{figures/images_data_description_buzz_sonogramme2.png}  \\
   \footnotesize{Denis et al. (2023)} 
\end{center}
 


\vspace{0.3cm}

\item Narwhals monitoring in Greenland (Greenland Institute of Natural Resources)

\vspace{0.1cm}
\centering
  \includegraphics[scale=0.4]{figures/narwhals.jpg}
\end{itemize}
  \end{itemize}
  
\end{frame}
\begin{frame}{Modeling different phases}
\centering 
\footnotesize{Count process} 

\vspace{-0.2cm}

    \includegraphics[scale=0.3]{figures/changement_phase.pdf}

    \begin{block}{Goals}
    \begin{itemize}
        \item New modeling to integrate different phases while keeping the Hawkes dependence structure
        \item Estimating all parameters and the change points
    \end{itemize}
        
    \end{block}
\end{frame}



\begin{frame}{Interest in ecology}
     \begin{itemize}
        
              \item[$\blacktriangleright$] Bats: modeling animal activity (commuting/foraging phases)
              \begin{center}

             
             \begin{center}
            \includegraphics[scale=0.25,trim=0mm 0mm 0mm 0mm, clip]{figures/chiro.png} 
             \end{center}       
             
                 
              \end{center}
              \item[$\blacktriangleright$] Narwhals: modeling the impact of external factors on sounds emission while fishing
              \begin{itemize}
                 \item[$\blacktriangleright$]  Sound exposure
                 \item[$\blacktriangleright$]   Distance from ships
                     \item[$\blacktriangleright$]  Environmental changes
              \end{itemize}
              
         
        \end{itemize} 

    


  
\end{frame}
\begin{frame}{When covariates are observed}
  \vspace{-0.5cm}
\begin{minipage}{7cm}
    $\blacktriangleright$ Narwhals' behaviour
\end{minipage}
\begin{minipage}{3cm}
 \includegraphics[scale=0.08,trim=0mm 30mm 0mm 50mm,clip]{figures/noun-narwhal-3564834.png}   
\end{minipage}

       
        \begin{block}{Goals}
            \begin{itemize}
                \item Integrate the covariates in the intensity function
                \item Estimate the effect of covariates, select the covariates with major impact
            \end{itemize}
        \end{block}
        
    \vspace{-0.5cm}

             $$ \lambda(t)= \lambda_0(X_t) + \underset{T_k < t}{\sum} h(t-T_k,X_t)$$

             \vspace{-0.2cm}
    \begin{block}{Observations}
          \begin{itemize}
             \item  $(T_k)_{k \geq 1}$ event times
         \item  $X_t$ observed covariates at time $t$
         \end{itemize} 
     \end{block} 
 
$\blacktriangleright$  Collaboration with C. Dion-Blanc, A. Samson, M. Sadeler          
      

\end{frame}

\begin{frame}{When covariates are not observed}
\begin{minipage}{5cm}
     $\blacktriangleright$ Bats monitoring  
\end{minipage}
\begin{minipage}{5cm}
     \includegraphics[scale=0.12,trim=0mm 30mm 0mm 50mm,clip]{figures/noun-bat-1110.png}   
\end{minipage}

       
    \begin{itemize}
        \item Introduce a latent variable $Z$ that encodes the changes of phases
    \end{itemize}

\small 
\begin{figure}
   \begin{centering}
\input{HMM}\
\end{centering}
\end{figure}

\begin{block}{Discretized version}
      \begin{itemize}
            \item[$\blacktriangleright$] Discrete Markov Chain for the hidden dynamics
             \item[$\blacktriangleright$]   Discretized version of the Hawkes process
        \end{itemize}
\end{block}
       
        
\end{frame}




\begin{frame}{Discrete time Hawkes process}
\small
\vspace{-0.2cm}
\begin{block}{Exponential Hawkes process}
$$\lambda(t)=\lambda_0 + \underset{T_k < t}{\sum} a e^{-b(t-T_k)} $$    
\end{block}
\begin{itemize}
    \item $I_k=[ \tau_{k-1};\tau_k]$ with $\tau_k=k\Delta$
    \item $N_k=N(I_k)$ the number of events on $I_k$


\vspace{-0.2cm}
\begin{center}
\includegraphics[scale=0.16]{figures/discrete_process.png}
\end{center}

\vspace{-0.3cm}

\item Distribution of $N_k$?
\end{itemize}



\end{frame}

\begin{frame}{Cluster representation (Hawkes and Oakes, 1974)}

\includegraphics[scale=0.3]{figures/cluster_representation.png}

\begin{itemize}
\item Immigrants arrive at rate $\lambda_0$ 
\item All individuals $T$ (immigrant and descendant) produce a new individual at rate $h(t - T)$
\end{itemize}
\end{frame}
\begin{frame}{Discrete time Hawkes process}
$N_k=N(I_k)$ the number of events on $I_k=[ \tau_{k-1};\tau_k]$
\begin{block}{Count distribution}
$$N_k \overset{\Delta}{=} B_k + \sum_{\ell \leq k-1} \sum_{T \in I_\ell} M_T(I_k) + R_k
$$
\end{block}
\small
   \begin{itemize}
       \item $B_k \sim \mathcal{P}(\lambda_0 \Delta)$ discrete immigrant process
 \item $ M_T(I_k) \sim \mathcal{P}\left(c(a,b,\Delta) e^{-b(\tau_{k-1}-T)}\right)$ descendants of $T<\tau_{k-1}$

 \item $R_k$ number of descendants of points $T \in I_k$
   \end{itemize}

\begin{block}{Approximation when $\Delta$ is small}
    \begin{itemize}
        \item  $ M_T(I_k) \simeq \mathcal{P}\left(c(a,b,\Delta) e^{-b(\tau_{k-1}-\tau_{\ell-1)}}\right)$ for $T \in  I_\ell=[\tau_{\ell-1}; \tau_\ell]$
        \item We neglect $R_k$
    \end{itemize}
\end{block}
\end{frame}

\begin{frame}{Markovian reformulation}

 \begin{block}{Approximation of $N_k$}
$$
Y_k \mid \{Y_\ell\}_{\ell \leq k-1} \sim \mathcal{P}\left(\mu + \sum_{\ell = 1}^{\infty} \alpha \beta^\ell Y_{k-\ell} \right),
$$
  with $\mu=\lambda_0 \Delta$ and $\alpha$, $\beta$ depending on $a$, $b$, $\Delta$.
   \end{block}
\begin{itemize}
    \item $\{Y_k\}_{k\geq 1}$ is not a Markov chain
    \item If we define $U_k$ such that 
  \end{itemize}  
  
  $$
 \left\{
 \begin{array}{l}
  U_1=0 \\
    U_k= \alpha Y_{k-1} + \beta U_{k-1}
 \end{array}
\right.
$$  
    
\begin{alertblock}{ }
$$Y_k \vert (U_{k-1}, Y_{k-1}) \sim  \mathcal{P}\left(\alpha Y_{k-1} +  \beta U_{k-1} \right)$$   
\end{alertblock}

$\rightarrow$ $(Y_k,U_k)_{\geq 1}$ is a Markov Chain    





\end{frame}

    
\begin{frame}{Discrete time Hawkes HMM}
\begin{itemize}
    \item Hidden path: $\{Z_k\}_{k \geq 1}$ homogeneous Markov chain with transition matrix $\pi$
    \item Observed counts: for $k\geq 1$, set $U_1=0$ and

    $$Y_k \mid \{Y_\ell\}_{\ell \leq k-1} \sim \mathcal{P}\left(\mu_{Z_k} + \sum_{\ell = 1}^{\infty} \alpha \beta^\ell Y_{k-\ell} \right)$$

    \begin{itemize}
        \item The immigration rate varies with the hidden path
        \item The number of offspring does not vary with the hidden path
    
    \end{itemize}
\end{itemize}

\footnotesize
\begin{figure}
   \begin{centering}
\begin{tikzpicture}
\node[] (Zt_2) at (-\edgeunit, \edgeunit) {}; 
\node[] (Zt_1) at (0, \edgeunit) {$Z_{i-1}$}; 
\node[] (Zt) at (\edgeunit, \edgeunit) {$Z_{i}$}; 
\node[] (Zt1) at (2*\edgeunit, \edgeunit) {$Z_{i+1}$}; 
\node[] (Zt2) at (3*\edgeunit, \edgeunit) {}; 
\node[] (Ut_1) at (-0.5*\edgeunit, 0.5*\edgeunit) {$U_{i-1}$}; 
\node[] (Ut) at (0.5*\edgeunit, 0.5*\edgeunit) {$U_{i}$}; 
\node[] (Ut1) at (1.5*\edgeunit, 0.5*\edgeunit) {$U_{i+1}$}; 
\node[] (Ut2) at (2.5*\edgeunit, 0.5*\edgeunit) {$U_{i+2}$}; 
\node[] (Ut3) at (3.5*\edgeunit, 0.5*\edgeunit) {}; 
\node[] (Yt_2) at (-\edgeunit, 0) {}; 
\node[] (Yt_1) at (0, 0) {$Y_{i-1}$}; 
\node[] (Yt) at (\edgeunit, 0) {$Y_{i}$}; 
\node[] (Yt1) at (2*\edgeunit, 0) {$Y_{i+1}$}; 
\node[] (Yt2) at (3*\edgeunit, 0) {}; 

\draw[->,dashed] (Zt_2) -- (Zt_1); \draw[->] (Zt_1) -- (Zt); \draw[->] (Zt) -- (Zt1); \draw[->,dashed] (Zt1) -- (Zt2);
\draw[->] (Zt_1) -- (Yt_1); \draw[->] (Zt) -- (Yt); \draw[->] (Zt1) -- (Yt1);
\draw[->] (Ut_1) -- (Yt_1); \draw[->] (Ut) -- (Yt); \draw[->] (Ut1) -- (Yt1);
\draw[->] (Ut_1) -- (Ut); \draw[->] (Ut) -- (Ut1); \draw[->] (Ut1) -- (Ut2); \draw[->,dashed] (Ut2) -- (Ut3);
\draw[->,dashed] (Yt_2) -- (Ut_1); \draw[->] (Yt_1) -- (Ut); \draw[->] (Yt) -- (Ut1); \draw[->] (Yt1) -- (Ut2);
\end{tikzpicture}


\end{centering}
\end{figure}

\end{frame}

\begin{frame}{Inference}

\begin{block}{Goals}
    \begin{itemize}
        \item Estimate $\mu=(\mu_1, \dots, \mu_Q)$, $\alpha$, $\beta$, $\pi$
        \item Learn $p(Z\vert Y) \rightarrow$ detect the change points
    \end{itemize}
\end{block}
    \begin{itemize}
        \item Introducing $(U_k)_{k\geq 1} \rightarrow$  EM algorithm for HMM

        \begin{block}{EM algorithm (Dempster et al, 1977)}
       Estimator $\hat{\theta}_k$ of  $\theta=(\mu,\alpha,\beta,\pi)$ at step $k$
            \begin{itemize}
                \item E-Step : Compute $\mathbb{E}_{\hat{\theta}_k} [log(p_\theta(Y,Z)\vert Y)]$
                \item M-Step : Define $\hat{\theta}_{k+1}=\underset{\theta}{\textrm{ argmax }} \mathbb{E}_{\hat{\theta}_k}[log(p_\theta(Y,Z)\vert Y)]$
            \end{itemize}
        \end{block}
    
    \end{itemize}
\end{frame}

\begin{frame}{EM algorithm for HMM}

    \begin{block}{Forward-Backward recursion for the E-step}
        \begin{itemize}
            \item Forward : Compute $p(Z_{k}\vert Y)$ as function of $p(Z_{k-1}\vert Y)$ 
            \item Backward: Compute $p(Z_{k}\vert Y)$ as function of $p(Z_{k+1}\vert Y)$
        \end{itemize}
    \end{block}
    \vspace{-0.3cm}
\footnotesize
\begin{figure}
   \begin{centering}
\begin{tikzpicture}
\node[] (Zt_2) at (-\edgeunit, \edgeunit) {}; 
\node[] (Zt_1) at (0, \edgeunit) {$Z_{i-1}$}; 
\node[] (Zt) at (\edgeunit, \edgeunit) {$Z_{i}$}; 
\node[] (Zt1) at (2*\edgeunit, \edgeunit) {$Z_{i+1}$}; 
\node[] (Zt2) at (3*\edgeunit, \edgeunit) {}; 
\node[] (Ut_1) at (-0.5*\edgeunit, 0.5*\edgeunit) {$U_{i-1}$}; 
\node[] (Ut) at (0.5*\edgeunit, 0.5*\edgeunit) {$U_{i}$}; 
\node[] (Ut1) at (1.5*\edgeunit, 0.5*\edgeunit) {$U_{i+1}$}; 
\node[] (Ut2) at (2.5*\edgeunit, 0.5*\edgeunit) {$U_{i+2}$}; 
\node[] (Ut3) at (3.5*\edgeunit, 0.5*\edgeunit) {}; 
\node[] (Yt_2) at (-\edgeunit, 0) {}; 
\node[] (Yt_1) at (0, 0) {$Y_{i-1}$}; 
\node[] (Yt) at (\edgeunit, 0) {$Y_{i}$}; 
\node[] (Yt1) at (2*\edgeunit, 0) {$Y_{i+1}$}; 
\node[] (Yt2) at (3*\edgeunit, 0) {}; 

\draw[->,dashed] (Zt_2) -- (Zt_1); \draw[->] (Zt_1) -- (Zt); \draw[->] (Zt) -- (Zt1); \draw[->,dashed] (Zt1) -- (Zt2);
\draw[->] (Zt_1) -- (Yt_1); \draw[->] (Zt) -- (Yt); \draw[->] (Zt1) -- (Yt1);
\draw[->] (Ut_1) -- (Yt_1); \draw[->] (Ut) -- (Yt); \draw[->] (Ut1) -- (Yt1);
\draw[->] (Ut_1) -- (Ut); \draw[->] (Ut) -- (Ut1); \draw[->] (Ut1) -- (Ut2); \draw[->,dashed] (Ut2) -- (Ut3);
\draw[->,dashed] (Yt_2) -- (Ut_1); \draw[->] (Yt_1) -- (Ut); \draw[->] (Yt) -- (Ut1); \draw[->] (Yt1) -- (Ut2);
\end{tikzpicture}


\end{centering}
\end{figure}


  \vspace{-0.5cm}

   \begin{block}{3-step initialization}
       \begin{itemize}
          \item Homogeneous Hawkes for the reproduction parameters $\alpha$ and $\beta$
          \item Poisson-HMM for the rates $\mu_1, \dots, \mu_Q$
          \item Correction: $\mu_k \rightarrow \widetilde{\mu}_k$ to account for reproduction rate
      \end{itemize}
   \end{block}
      
\end{frame}
\begin{frame}{Numerical results: classification}


\begin{center}
 \small
           \begin{tabular}{c||c|cccc}
              & States &  1 &2  & 3\\
              \hline
                \multirow{3}{*}{Hawkes HMM}& 1 & 273.7 & 28.7 & 14.2\\
    &2 &  37 & 166.3 & 96.9\\
    & 3 &  4.7  & 24.7 & 353.8 \\
    \hline 
    \hline
                   \multirow{3}{*}{Poisson HMM}& 1 & 181 & 122.8 & 12.8\\
    &2 &  136 & 111.1 & 53.1\\
    & 3 &  45.4 & 115.2 & 222.6
               \end{tabular} 

\vspace{0.5cm}
\begin{tabular}{cc}
    \includegraphics[scale=0.25]{figures/traj_mediane_Q3_true.pdf}    & 
    \includegraphics[scale=0.25]{figures/traj_mediane_Q3_estimated.pdf}
    \end{tabular}
\end{center}
               
               


\end{frame}

\begin{frame}{Numerical results: parameters estimation}

    \begin{center}
       \includegraphics[scale=0.3]{figures/estim_param_discr_Q3.pdf}

    \end{center}
\end{frame}

\begin{frame}{Goodness-of-fit}

\begin{block}{Time-change theorem (Daley and Vere-Jones, 2003)}
\small
 A sequence $(T_k)_{k\ge 1}$ is a realization of $N$ if and only if $(\Lambda(T_k))_{k \ge 1}$ is a realization of a homogeneous Poisson process with unit intensity.
\end{block}
where $$\Lambda(t)=\int_0^t \lambda(u) du \quad \quad \quad \textrm{(Compensator)}$$

\begin{block}{Goodness-of-fit test}
\begin{itemize}
\item $H_0$: ``$(T_k)_{k\ge 1}$ is a realization of a HMM-Hawkes process with parameter \(\theta\)''.
\item Kolmogorov-Smirnov test between $\left( \Lambda_\theta(T_{k+1}) - \Lambda_\theta(T_k) \right)_{k \ge 1}$ and an exponential distribution $\mathcal{E}(1)$.
\end{itemize}
\end{block}

$\blacktriangleright$ Same test for alternative models (Hawkes, HMM-Poisson)  \\
   $\blacktriangleright$ Train/test samples (resampling procedure, Reynaud-Bouret et al. (2014)) 
\end{frame}


\begin{frame}{Results Goodness-of-fit}

\begin{minipage}{4cm}
\centering
\includegraphics[scale=0.2]{figures/ex_traj_Q3.pdf}
\end{minipage}
\begin{minipage}{6cm}
\centering
    \includegraphics[scale=0.22]{figures/compar_pval_ks_Q3.pdf}
\end{minipage}

\begin{block}
$\blacktriangleright$ The GoF test does not differentiate the homogeneous Hawkes process from the HMM-Hawkes process.
\end{block}
\end{frame}

\begin{frame}{Results Goodness-of-fit}

\begin{minipage}{4cm}
\centering
\includegraphics[scale=0.2]{figures/ex_traj.pdf}
\end{minipage}
\begin{minipage}{6cm}
\centering
    \includegraphics[scale=0.22]{figures/compar_pval_ks.pdf}
\end{minipage}

\begin{block}
$\blacktriangleright$ The GoF test is able to detect that the point process is neither an homogeneous Hawkes nor a HMM-Poisson process
\end{block}
\end{frame}

\begin{frame}{Preliminary results on bats data}
\begin{minipage}{6.6cm}
    \includegraphics[scale=0.28]{figures/compar_Q_bats.pdf} 
\end{minipage}
   \begin{minipage}{3.8cm}
       \begin{itemize}
       \item $Q=2$ or $Q=3$ give visually good results
        \item Choice for $Q$:  BIC criterion, likelihood ratio test?
       \end{itemize}
   \end{minipage}
\end{frame}


\begin{frame}{Extension I : Multivariate Hawkes process}



\includegraphics[scale=0.1,trim=40mm 1050mm 20mm 2050mm]{figures/random_multiv.png}

\begin{block}{Conditional intensity}

$$ \lambda^{(i)}(t)=  \lambda_{0}^i + \underset{i=1}{\overset{M}{\sum}} \underset{T_k^j  < t}{\sum} h_{i,j}(t-T_k^j) 
$$    
\end{block}


\begin{itemize}
    \item One dimension for each neuron/species
    \item Allows to model interactions 
\end{itemize}


\end{frame}

\begin{frame}{Extension I : Multivariate Hawkes process}
\small
\begin{minipage}{6cm}
Example of a bivariate Hawkes process:
\centering
    \includegraphics[scale=0.35]{figures/anna_tlse_24.pdf}    
\end{minipage}
\begin{minipage}{3.5cm}
\centering
    Connectivity graph 
\includegraphics[scale=0.2,trim=20mm 10mm 10mm 10mm,clip]{figures/communities_d10-2.pdf}
\end{minipage}

    \begin{block}{Modeling choice}
       \begin{itemize}
           \item Only one $(Z_k)_{k \geq 1}$ for the multivariate process
           \item Each dimension has its own hidden path
       \end{itemize} 
    \end{block}
\end{frame}
\begin{frame}{Extension II : Changing memory/interaction functions }
    
\begin{itemize}

    \item The hidden path can influence the memory/interaction functions
   \item  $ U_k$ depends on $Z_{k-1}$
\end{itemize}
  
\footnotesize
\begin{figure}
   \begin{centering}
\input{DirectedGM}
\end{centering}
\end{figure}
\end{frame}


\begin{frame}{Extension III : Nonlinear Hawkes process}

\begin{itemize}
    \item Modeling inhibition ($h \leq 0$)

\begin{center}
    \begin{tabular}{c}
\includegraphics[scale=0.2]{figures/ann_1.png}
    \end{tabular}
\end{center}

      $$ \lambda(t)= \phi \left( \lambda_0+  \underset{T_k \leq t}{\sum} h(t-T_k) \right)
$$


\end{itemize}
    

\end{frame}


\end{document}

\end{document}
