\documentclass[12pt]{article}
\usepackage[a4paper,top=2.5cm,bottom=2.5cm,left=2cm,right=2cm]{geometry}
\usepackage[utf8]{inputenc}
\usepackage{amsmath,amsfonts,amssymb}
\usepackage{natbib,hyperref}
\usepackage{xcolor}
\usepackage{tikz}
\usepackage{comment}


%------------------------------------------------------------------------------------
%------------------------------------------------------------------------------------

%------------------------------------------------------------------------------------
%------------------------------------------------------------------------------------
\begin{document}
%------------------------------------------------------------------------------------
%------------------------------------------------------------------------------------

\begin{center}
{\bf {\large Discrete-time Markov Switching Hawkes Process}}

\bigskip
{Anna {\sc Bonnet}, Stéphane {\sc Robin}}

\bigskip
{Sorbonne Université, Laboratoire de Probabilités, Statistique et Modélisation} 

\bigskip\url{<anna.bonnet@sorbonne-universite.fr>}, \url{<stephane.robin@sorbonne-universite.fr>}
\end{center}

\bigskip
Over the last few decades, the Hawkes process has become an important framework for modeling successions of self-exciting (or self-inhibiting) events that occur over time. It has been used in a wide variety of applications, such as earth sciences (seismology, vulcanology), economics, genomics, ecology, to name but a few. The conditional intensity of a Hawkes process comprises a base function (or immigration rate) and the sum of the influences of all past events, encoded in the so-called kernel function. 

\bigskip
We consider here a discrete-time version of the Hawkes process, with an exponential kernel, where the immigration term varies according to a latent Markov chain. We prove that this model is identifiable and that, thanks to the specificities of the exponential kernel, it can be reformulated in terms of a hidden Markov model, with Poisson emissions. Based on these properties, we show that maximum likelihood inference of the model's parameters can be performed using an EM algorithm, which involves a recursive M-step.

\bigskip
We evaluate the accuracy of parameter estimates through a simulation study when signal level and discretization step vary. We also address the selection of the number of hidden states. Finally, we illustrate the use of the proposed modeling to distinguish different bat behaviors, based on the recording of their cries.

%------------------------------------------------------------------------------------
%------------------------------------------------------------------------------------
\end{document}
%------------------------------------------------------------------------------------
%------------------------------------------------------------------------------------
