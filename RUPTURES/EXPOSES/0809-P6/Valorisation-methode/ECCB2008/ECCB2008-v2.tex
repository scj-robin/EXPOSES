\documentclass{bioinfo}
\copyrightyear{2005}
\pubyear{2005}
\usepackage{natbib}

\begin{document}
\firstpage{1}

\title[ChIPmix]{ChIPmix: Mixture model of regressions for two-color ChIP-chip analysis}
\author[Sample \textit{et~al}]{Marie-Laure Martin-Magniette\,$^{1,2,+,}$\footnote{to whom correspondence should be addressed},
Tristan Mary-Huard\,$^{1,+}$,  Caroline B\'erard$^{1,2}$
and St\'ephane Robin\,$^1$}


\address{$^{1}$UMR AgroParisTech/INRA MIA 518, 16 rue Claude Bernard,
  75231 Paris Cedex 05, France.\\
  $^{2}$URGV UMR INRA/CNRS/UEVE, 2 rue Gaston Cr\'emieux, CP5708,
  91057, Evry Cedex, France.\\
  $^+$ Both authors contributed equally to this work.}

\history{Received on XXXXX; revised on XXXXX; accepted on XXXXX}

\editor{Associate Editor: XXXXXXX}

\maketitle

\begin{abstract}

\section{Motivation:}
Chromatin immunoprecipitation (ChIP) combined with DNA microarray is
a high-throughput technology to investigate DNA-protein-binding or
chromatin/histone modifications. ChIP-chip data require adapted
statistical method in order to identify enriched regions. All methods
already proposed are based on the analysis of the log-ratio (Ip/Input).
Nevertheless the assumption that the log-ratio is
a pertinent quantity to assess the probe status is not always verified and
it leads to a poor data interpretation.

\section{Results:}
Instead of working on the log-ratio, we directly work with the Ip and
Input signals of each probe by modeling the distribution of the Ip
signal conditional to the Input signal.
We propose a method named ChIPmix based on a linear regression
mixture model to identify actual binding targets of the protein under
study. Moreover we are able to control the proportion of false-positives.
The efficiency of ChIPmix is illustrated on several datasets obtained
from different organisms and hybridized either on tiling or promoter arrays.
This validation shows that ChIPmix is convenient for any two-color array
whatever its density and provides promising results.

\section{Availability:}
The ChIPmix method is implemented in R and is available at
{{http://www.agroparistech.fr/mia/outil\_A.html}}

\section{Contact:} \href{marie\_laure.martin@agroparistech.fr}{marie\_laure.martin@agroparistech.fr}
\end{abstract}


%%%%%%%%%%%%%%%%%%%%%%%%%%%%%%%%%%%%%%%%%%%%%%%%%%%%%%%%%%%%%%%%%%%%%%%%%%%%%%%
%%%%%%%%%    Introduction
%%%%%%%%%%%%%%%%%%%%%%%%%%%%%%%%%%%%%%%%%%%%%%%%%%%%%%%%%%%%%%%%%%%%%%%%%%%%%%%


\section{Introduction}
Chromatin immunoprecipitation (ChIP) is a well-established procedure
used to investigate proteins associated with DNA. ChIP on chip
involves analysis of DNA recovered from ChIP experiments by
hybridization to microarray. In a two-color ChIP-chip experiment, two
samples are compared: DNA fragments crosslinked to a protein of
interest (IP), and genomic DNA (Input). The two samples are
differentially labeled and then co-hybridized on a single array. The
goal is then to identify actual binding targets of the protein of
interest, i.e. probes whose IP signal is significantly larger than the
Input signal. {\par} Many authors have already pointed out the need
for efficient statistical procedures to detect enriched probes
\citep{BuckLieb04,Keles07}. Recently, two strategies have been widely
applied for the detection of enriched DNA regions. The first strategy
takes advantage of the spatial structure of the data. Since probes are
positioned all along the genome, if one region is enriched we expect
several adjacent probes to obtain high ratio measurements, resulting
in a ``peak'' of intensity. Spatial methods such as sliding windows
\citep{Cawley04,Keles07} or Hidden Markov Models
\citep{JiWong05,LiMeyerLiu05} have been proposed to detect these peaks.
Alternatively, the second strategy is to consider that the whole
population of probes can be divided into two components: the
population of IP-enriched genomic fragments, and the population of
genomic DNA that is not IP enriched. Different statistical methods
have been proposed to distinguish between the two populations by
considering the distribution of the ratios (or their associated rank).
Assuming that a non negligible proportion of the fragments are
enriched, the logratio distribution is bimodal, the highest mode
corresponding to the enriched population. A probe is then declared
enriched when its ratio exceeds a selected cutoff, which is fixed
according to the data distribution \citep{BuckLieb04}.  {\par}
Importantly, both strategies assume that the logratio measurement is a
pertinent statistical quantity to assess the probe status (enriched or
not). This assumption is correct if the distribution of the ratio
mostly depends on the status (normal/enriched) of the probe. Figure
\ref{Figure:Histogramme} (left) shows the ideal situation described in
\cite{BuckLieb04}, where the distribution is bimodal. In many
applications the distribution of the logratios is closer to Figure
\ref{Figure:Histogramme} (center), and the performance of
logratio-based methods may be poor. At least two technical reasons may
explain the difference between the ideal and real cases.
First, there are some technical difficulties to
obtain the IP sample: it requests the use of a very specific
antibody and a careful experimental process to avoid a high level of
contamination. The second reason comes from the possible cross-hybridization
phenomena.

\begin{center}
  Figure \ref{Figure:Histogramme} here.
\end{center}

From observation of Figure \ref{Figure:Histogramme}, we argue that it is
worth working directly
with the two measurements of each probe (Input and IP) rather than
with the logratio. In Figure \ref{Figure:Histogramme}(right), we
observe that the relationship between the two measurements is almost
linear. Working on logratio amounts to stating that the slope of the
linear relationship is the same whatever the status of the probe. In
many cases the slopes are different: Figure
\ref{ComparaisonHistoTheoriques} (synthetic data) shows that even a
slight difference between the two slopes may turn the distribution
of the logratios into unimodal rather than bimodal, as observed for
the Nimblegen slide in Figure \ref{Figure:Histogramme}.

In this work, we propose a new statistical method that we call
ChIPmix, based on a mixture model of regressions. This framework
allows us to well characterize the IP-Input relationship, and to
provide a statistical procedure to control the proportion of probes
wrongly classified as enriched. The article is organized as follows.
The statistical model and the procedure for false positive control
are described in Section \ref{methodo}.  In Section \ref{appli}, we
consider several large datasets obtained from different organisms
and hybridized on different array types (tiling or promoter). We
show that the method outperforms competing methods in terms of
sensitivity. The main conclusions and some possible extensions are
discussed in Section \ref{discu}.

\begin{center}
  Figure \ref{ComparaisonHistoTheoriques} here.
\end{center}


%%%%%%%%%%%%%%%%%%%%%%%%%%%%%%%%%%%%%%%%%%%%%%%%%%%%%%%%%%%%%%%%%%%%%%%%%%%%%%%
%%%%%%%%%    Statistical framework
%%%%%%%%%%%%%%%%%%%%%%%%%%%%%%%%%%%%%%%%%%%%%%%%%%%%%%%%%%%%%%%%%%%%%%%%%%%%%%%


\section{Statistical framework}\label{methodo}
\subsection{Model and inference}
Let $(x_i,Y_i)$ be the log-Input and log-IP intensities of probe $i$,
respectively. The  (unknown) status of the probe is characterized
through a label $Z_i$ which is 1 if the probe is enriched and 0 if
it is normal (not enriched). We assume the Input-IP relationship to
be linear whatever the population, but with different slope and
intercept. More precisely, we have:
\begin{eqnarray}
Y_i &=& a_0 + b_0 x_i + \epsilon_i  \qquad \text{if } Z_i=0 \text{ (normal) } \nonumber\\
 &=& a_1 + b_1 x_i + \epsilon_i  \qquad \text{if } Z_i=1 \text{  (enriched) } \nonumber
\end{eqnarray}
where $\epsilon_i$ is a Gaussian random variable with mean 0 and
variance $\sigma^2$. Such a model is named a mixture model of
regressions.

The marginal distribution of $Y_i$ for a given level of Input $x_i$
is
\begin{eqnarray}
(1-\pi) \phi_0(Y_i | x_i) + \pi \phi_1(Y_i | x_i),
\label{Equation:Melange} %\leadsto
\end{eqnarray}
where $\pi$ is the proportion of enriched probes, and
$\phi_j(\cdot|x)$ stands for the probability density function (pdf) of
a Gaussian distribution with mean $a_j+b_j x$ and variance $\sigma^2$.

The mixture model is used to classify probes as normal or enriched. To do this, we calculate
the probability of a probe to be enriched given its Input and IP intensities. This probability
is called the \textit{posterior} probability and is defined from Equation
(\ref{Equation:Melange}) by
\begin{eqnarray}
\tau_i = \Pr\{Z_i=1 \;|\; x_i,Y_i\}=\frac{\pi
  \phi_1(Y_i |x_i)}{(1-\pi) \phi_0(Y_i|x_i) + \pi \phi_1(Y_i|x_i)}. \label{Equation:Posterior}
\end{eqnarray}

The mixture parameters (proportion, intercepts, slopes and variance)
are estimated using the EM algorithm. The EM algorithm is dedicated to
the class of incomplete data models where the status of the
observations is unknown. In the E step, the posterior probability for
each observation to belong to each class is calculated. In the M step,
the parameters of each class are estimated using a weighted
regression, in which the weights are given by the posterior
probabilities. This algorithm is implemented in the mixreg function of
the mixreg R package \citep{Turner00}. Figure \ref{Figure:Resultats} (left) shows the application of Chipmix on the NimbleGen high-density array data, presented in section \ref{HistModifH3K9me3}.


In the mixreg function, the initial values of the parameters must be
given by the users otherwise they are chosen randomly. Nevertheless
the EM algorithm is well-known to be sensitive to the initial values \citep{Karlis03, Bohning03} and
to solve this difficulty, we propose initial values derived from the first
axis of the Principal Component Analysis (PCA) of the whole dataset
(see the ChIPmix R function for details).

The mixture model with two linear regressions is adapted if the
protein under study has some targets. When the protein has no
target, all probes belong to the normal class. In this case a simple
linear regression is sufficient to fit the data. For each dataset
the two models  (one or two classes) are fitted and the best model
is selected according to the BIC criterion \citep{Schwarz78}.

\subsection{False discovery control}  \label{fdc}
Posterior probabilities are used to classify probes into the normal
or enriched class, using the following classification rule
$$
{\tau}_{i} > s \qquad \Rightarrow \qquad \widehat{Z}_i=1 \text{ classified as enriched},
$$
where $s$ is an arbitrary  threshold that has to be fixed. In the
context of mixture models, $s$ is usually fixed to 1/2 ({\it Maximum
A Posteriori} rule) which implicitly means that misclassifications
in population 0 or in population 1 have the same cost.

In ChIP-chip experiments where false positives are of concern, it is
important to control the false positive proportion and to fix $s$
accordingly. In the hypothesis test theory, the false discovery
control is performed by controlling the probability to reject
wrongly the null hypothesis. We propose an analogous concept in the
mixture model framework. Our aim is to control the probability for a
probe to be wrongly assigned to the enriched class. Therefore we
want $\Pr\{\tau_{i} > s \;|\; x_i, Z_i =0\}$ to be equal to a
predefined level $\alpha$. In practice, we fix $\alpha$ and we find the
threshold $s$ depending on $\alpha$ and $x_i$ (see Appendix).


%%%%%%%%%%%%%%%%%%%%%%%%%%%%%%%%%%%%%%%%%%%%%%%%%%%%%%%%%%%%%%%%%%%%%%%%%%%%%%%
%%%%%%%%%    Statistical framework
%%%%%%%%%%%%%%%%%%%%%%%%%%%%%%%%%%%%%%%%%%%%%%%%%%%%%%%%%%%%%%%%%%%%%%%%%%%%%%%


\section{Results}\label{appli}

We present three applications of ChIPmix to assess the performance of the method whatever the specificity and density of the array (tiling or promoter array). The first two applications validate the method on already
published data. The third dataset is used to compare ChIPmix with existing methods.

\subsection{Promoter DNA methylation in the human genome}
\cite{Schubeler07} measured DNA methylation using a NimbleGen
microarray representing 15 609 promoter regions of the human genome.
Each promoter region is covered by 15 probes and is classified into
a category according to its CpG rate. We focus on the analysis of
the class ICP (intermediate CpG promoter). Weber et al. based their
classification on the mean logratio value for the 15 probes per
promoter region. If this value was larger than 0.4 (threshold based
on bisulfite sequencing), the promoter region was declared
hypermethylated. Among the 2056 promoter regions under study, 460
were declared hypermethylated. {\par} We applied ChIPmix to these
data without averaging the 15 values per promoter region. The
estimated proportion $\pi$ of enriched probes was 0.794. This is in
keeping with a large proportion of targets expected in such
experiments. The estimated regression slopes were
$\widehat{b}_0=0.613$ for the normal class and $\widehat{b}_1=1.162$
for the enriched one, which shows that the Input-Ip relations
substantially differ between the two status. At the level
$\alpha=0.01$, a total of 1706 promoter regions were found to have
at least 1 probe enriched. Except for one region, all the promoter
regions of the Weber's list have at least 1 enriched probes, and 403
have 5 or more enriched probes. Besides, ChIPmix identified 38
promoter regions with 9 probes or more classified as enriched that
were not detected in \cite{Schubeler07}.

\subsection{Histone modification in \textit{Arabidopsis thaliana}}
\cite{Turck07} studied several histone modifications of
\textit{Arabidopsis thaliana} using a custom genomic tiling array of
chromosome 4. To declare a tile enriched, they developed a two-step
method based on a Gaussian mixture model and a total of 2775 tiles
were found to be marked by histone H3 tri-methylated at lysine 27
(H3K27me3) according to their analysis.

We analyzed the same dataset using ChIPmix. The estimated proportion
and slopes were $\widehat{\pi}=0.361$, $\widehat{b}_0=0.907$ and
$\widehat{b}_1=1.167$. The tiles classified as enriched at risk
$\alpha=0.01$ include all the tiles found by \cite{Turck07} plus
2346 others: 1404 tiles extend the genomic region already found
marked by H3K27me3 and 942 tiles form 62 new genomic regions. The
difference between the two slopes enables us to better discriminate
the two classes for high Input intensities. This may explain the
higher number of enriched probes detected by ChIPmix.


\subsection{NimbleGen high-density array (Histone modification H3K9me3)}\label{HistModifH3K9me3}
In this last example we considered Chip-chip data produced on a
two-color NimbleGen array of 1\;132\;140 probes. Each chromosome of
the model plant \textit{Arabidopsis thaliana} is covered by about
200\;000 probes. Such very high density arrays are more and more
popular, so we need to assess the efficiency of ChIPmix on such a
very large dataset.

From a biological point of view, the same IP and Input samples were
already hybridized on a custom genomic tilling array covering the
chromosome 4 \citep{Turck07}. Regions identified in \cite{Turck07}
were biologically validated and are used as true positives. In
addition, the chloroplastic genome can be used as a negative
control, since no histone modification target is expected in this
region. We did not use the mitochondrial genome as a negative control
since some regions have been duplicated in the nuclear genome.
From a statistical point of view, since ChIPmix does not take  the
spatial structure into account, it is important to compare it with methods
using this information. We compared our
results with those provided by the NimbleGen software and ChIPOTle method \citep{BuckNobelLieb05}. NimbleGen software uses a permutation-based algorithm to
find statistically significant peaks, using scaled logratio data, and
ChIPOTle method uses a sliding window approach. \\
{\par} Two biological replicates were available, for which
hybridizations were performed in dye-swap. We performed a
normalization step to remove technical biases as well as dye bias.
Since the Input and IP samples differed substantially, array-by-array
normalization such as lowess could not be applied.  We quantified biases
by an ANOVA model \citep{Kerr02}, and removed them from the raw data.
The IP and Input signals for each biological replicate were averaged
on the dye-swap to remove the gene-specific dye bias. Analyses per chromosome were performed on the
normalized data.

\begin{center}
  Figure \ref{Figure:SignalMAP} here.
\end{center}

For a risk $\alpha=0.01$, a total of 30\;477 probes were detected in
the first replicate and 27\;553 in the second. The intersection
contains more than two thirds of the probes declared enriched in at
least one replicate (23\;546 probes). Although ChIPmix does not take
the spatial structure of the genome into account, enriched probes
are clustered in genomic regions (see Figure
\ref{Figure:SignalMAP}). These regions are rich in genes and
corroborate the results of \cite{Turck07}, who have shown that
H3K9me3 is actually a euchromatin mark. Moreover more than 80 \% of
the probes classified as enriched in this experiment cover genomic
regions already found in \cite{Turck07}.

For the chloroplastic genome, the BIC criterion selected a two component
regression model. For the first biological replicate, two 2-probe clusters and one 3-probe cluster were declared enriched. On the same replicate, ChIPOTle (window=500 and step=100) found five 2-probe clusters, two 3-probe cluster and one 6-probe cluster. With other parameters (window=200 and step=50), the number of detected peaks increased. Results are similar for the second biological replicate, and one cluster was declared enriched with ChIPmix in both biological replicates (two with ChIPOTle). Nimblegen did not provide the analysis of the chloroplastic genome.

We also compared ChIPmix to the results given by NimbleGen and
ChIPOTle on chromosome 4, studied in \cite{Turck07}.
The probes declared enriched by ChIPmix include almost all of
those found enriched by the NimbleGen software, but cover much
larger genomic regions. Moreover ChIPmix identifies other genomic
regions not found by the NimbleGen software (see Figure
\ref{Figure:SignalMAP}), that are validated by a comparison with
results of  \cite{Turck07}. ChIPmix detects 30\;477 enriched probes,
including 24\;575 in common with \cite{Turck07}. ChIPOTle detects
24\;357 probes (20\;866 common with \cite{Turck07}) and NimbleGen
detects 19\;837 probes (16\;600 common with \cite{Turck07}) (see Figure \ref{Figure:Resultats}, right). Among
the three methods ChIPmix provides the closest results to the
reference publication.

\begin{center}
  Figure \ref{Figure:Resultats} here.
\end{center}


%%%%%%%%%%%%%%%%%%%%%%%%%%%%%%%%%%%%%%%%%%%%%%%%%%%%%%%%%%%%%%%%%%%%%%%%%%%%%%%
%%%%%%%%%    Discussion
%%%%%%%%%%%%%%%%%%%%%%%%%%%%%%%%%%%%%%%%%%%%%%%%%%%%%%%%%%%%%%%%%%%%%%%%%%%%%%%


\section{Discussion}\label{discu}
We propose a statistical method based on mixture of regression to
classify probes in ChIP-chip experiments. Our approach accounts for
different relations between IP and Input intensity in the two
classes of probes (enriched and normal). The ChIPmix method
outperforms the standard approaches based on the logratio.

We presented various applications each dedicated to one specific
biological question (histone modification and DNA methylation on
different organisms). ChIPmix can also be applied to the detection
of transcription factor binding sites (TFBS, results not shown). The
method is valid when the proportion of positive probes is expected
to be large (e.g. histone modification), or small (e.g. TFBS).
Through the examples we have shown that ChIPmix is convenient for
any two color chip whatever its density (array size from thousands
to hundreds of thousands of probes) and the nature of the probe
(tiling and promoter arrays).

ChIPmix does not account for the spatial structure of the data.
While this could be seen as a drawback, we showed that enriched
probes are clustered into genomic regions in the presented
applications. Moreover, this may become perfectly relevant for
specific experiments as well as RIP-chip, which investigates
interactions between protein and RNA (see
\cite{Schmitz-Linneweber05}) or ChIP-chip experiments performed on
array where promoter are represented by only one probe (see project
SAP at {\tt www.psb.ugent.be/SAP/})

The only parameter of the ChIPmix method is the risk $\alpha$, which
can be easily interpreted. In contrast, two parameters have to be
tuned in the ChIPOTle method (window size and step). The tuning of
this two parameters depends on both the experimental protocol and
the array type. The results are very sensitive to this tuning.

The proposed strategy can be extended in different ways.  The ChIPmix
extension to the unequal variance case is straightforward. However,
the equal variance case provides an efficient framework for the false
discovery control. If the equality of variance is not assumed, the
calculations given in Section 7 do not hold anymore, and the solving
of the equation system becomes much more complex. \\
The proposed regression models allow us to correct the IP
intensity with respect to the Input one. Other elements may influence
the level of IP signal. \cite{Schubeler07} show that the CpG rate has
to be taken into account to classify probes. The specificity of the
probes (number of hits) may also alter the IP intensity. All this
information can be considered as covariates and added in the model.
This will lead to a mixture of multiple regression for which the
statistical framework is almost the same as the one we propose.\\
The proportion of false negative results can be controlled in the same
way as the false discovery described in Section \ref{fdc}. This allows
us to evaluate the sensitivity of the classification at each Input
level. Moreover, the two criteria (false negative and false discovery)
can be combined to derive a threshold $s$ that optimizes some
trade-off between them.

\section{Funding}
This work was supported by the TAG ANR/Genoplante project.

\section{Acknowledgements}
The authors want to thank Vincent Colot, Alain Lecharny and Michel
Caboche from the URGV unit for helpful discussions and advice.

\section{Appendix}\label{append}
We propose to control the probability for a normal probe $i$ to be
wrongly assigned to the enriched class:
\begin{equation}\label{equation}
\Pr\{\tau_{i} > s \;|\; x_i, Z_i =0\}=\alpha.
\end{equation}
In practice, we fix $\alpha$ and we find the threshold $s$ depending
on $\alpha$ and $x_i$. Using definition \ref{Equation:Posterior},
$\Pr\{\tau_{i} > s \;|\; x_i, Z_i =0\}$ can be rewritten as
\begin{equation*}
\Pr\{ (1-\pi) \phi_0(Y_i|x_i) (1-s) - s \pi \phi_1(Y_i|x_i) > 0
\;|\; x_i, Z_i =0\}.
\end{equation*}
Replacing the probability density functions $\phi_0(Y_i|x_i)$ and
$\phi_1(Y_i|x_i)$ with their expression, we get Equation
(\ref{equation}) equivalent to
\begin{equation}\label{exp1}
\Pr\left(2\frac{(a_0-a_1)+(b_0-b_1)x_i}{\sigma^{2}}Y_{i} +
\gamma(s,x_i) > 0 \;|\; x_i,Z_i=0\right)=\alpha,
\end{equation}
where
\begin{eqnarray*}
\gamma(s,x_i) &=& \frac{(a_0+b_0x_i)^{2}-(a_1+b_1x_i)^{2}}{\sigma^2} \\
&&-2\log
\{s(1-\pi)\} + 2\log \{(1-s)\pi\}.
\end{eqnarray*}
Since the status of probe $i$ is normal ($Z_i=0$), the distribution
of $Y_i$ is a Gaussian with mean $a_0+b_0x_i$ and variance
$\sigma^2$, and we deduce that Equation (\ref{exp1}) is equivalent
to solve
$$\gamma(s, x_i) = \frac{2(a_0-a_1+(b_0-b_1)x_i)}{\sigma} \left\{u_{1-\alpha} + \frac{(a_0+b_0x_i)}{\sigma}\right\},$$  where
$u_{1-\alpha}$ is the  $(1-\alpha)$-quantile  of Gaussian with mean
0 and variance 1.

Using the definition of $\gamma(s, x_i)$, the expression of
threshold $s$ is given by
$$ s = \frac{e^{\lambda}}{1+e^{\lambda}},$$
where
\begin{eqnarray*}
\lambda &=& \left(\frac{a_1-a_0+(b_1-b_0)x_i}{\sigma}\right)
\left(u_{1-\alpha} - \frac{a_1-a_0+(b_1-b_0)x_i}{2\sigma}\right)\\
&&- \log\left(\frac{1-\pi}{\pi}\right)\,\cdot
\end{eqnarray*}

\bibliographystyle{natbib}
\begin{thebibliography}{}
\bibitem[Berard (2007)]{Berard07}
C.~Berard.
\newblock Analyse des donn\'ees tiling array issues de la technologie
  nimblegen.
\newblock Master's thesis, Universit\'e Paris-Sud 11 / AgroParisTech, 2007.

\bibitem[Bohning and Seidel (2003)]{Bohning03}
D.~Bohning, and W.~Seidel,
\newblock Editorial: recent developments in mixture models.
\newblock {\em Computational Statistics and Data Analysis}, 41(3):349--357, 2003.
\bibitem[Buck and Lieb (2004)]{BuckLieb04}
M.J. Buck and J.D. Lieb.
\newblock Chip-chip: considerations for the design, analysis, and application
  of genome-wide chromatin immunoprecipitation experiments.
\newblock {\em Genomics}, 83(3):349--360, 2004.

\bibitem[Buck et al. (2005)]{BuckNobelLieb05}
M.J. Buck, A.B. Nobel, and J.D. Lieb.
\newblock Ch{IPOT}le: a user-friendly tool for the analysis of {C}h{IP}-chip
  data.
\newblock {\em Genome Biol.}, 6(11), 2005.

\bibitem[Cawley et al. (2004)]{Cawley04}
S.~Cawley, S.~Bekiranov, H.~Ng, and et~al.
\newblock Unbiased mapping of transcription factor binding sites along human
  chromosomes 21 and 22 points to widespread regulation of noncoding rnas.
\newblock {\em Cell}, 116(4):499--509, 2004.

\bibitem[Ji and Wong (2005)]{JiWong05}
H.~Ji and W.H. Wong.
\newblock Tilemap: create chromosomal map of tiling array hybridizations.
\newblock {\em Bioinformatics}, 21(18):3629--3636, 2005.

\bibitem[Karlis and Xekalaki (2003)]{Karlis03}
D.~Karlis and E.~ Xekalaki.
\newblock Choosing initial values for the EM algorithm for finite mixtures.
\newblock {\em Computational Statistics and Data Analysis}, 41(3):577--590, 2003.

\bibitem[Keles (2007)]{Keles07}
S.~Keles.
\newblock Mixture modeling for genome-wide localization of transcription
  factors.
\newblock {\em Biometrics}, 63(1):10�--21, 2007.

\bibitem[Kerr et al. (2002)]{Kerr02}
M.~K. Kerr, C.~A. Afshari, L.~Bennett, P.~Bushel, J.~Martinez, N.~J. Walker,
  and G.~A. Churchill.
\newblock Statistical analysis of a gene expression microarray experiment with
  replication.
\newblock {\em Statistica Sinica}, 12:203--217, 2002.

\bibitem[Li et al. (2005)]{LiMeyerLiu05}
W.~Li, C.A. Meyer, and X.S. Liu.
\newblock A hidden markov model for analyzing chip-chip experiments on genome
  tiling arrays and its application to p53 binding sequences.
\newblock {\em Bioinformatics}, 21:i274--i282, 2005.

\bibitem[Weber et al. (2007)]{Schubeler07}
Weber M., I.~Hellmann, M.B. Stadler, Svante-P. Ramos, L., M.~Rebhan, and
  D.~Schubeler.
\newblock Distribution, silencing potential and evolutionary impact of promoter
  dna methylation in the human genome.
\newblock {\em Nature Genetics}, 39(4):457--466, 2007.

\bibitem[Schmitz-Linneweber et al. (2005)]{Schmitz-Linneweber05}
C.~Schmitz-Linneweber, R.~Williams-Carrier, and A.~Barkan.
\newblock Rna immunoprecipitation and microarray analysis show a chloroplast
  pentatricopeptide repeat protein to be associated with the 5' region of mrnas
  whose translation it activates.
\newblock {\em Plant Cell}, 17::2791--2804, 2005.

\bibitem[Schwarz (1978)]{Schwarz78}
G.~Schwarz.
\newblock Estimating the dimension of a model.
\newblock 6:461--464, 1978.

\bibitem[Turck et al. (2007)]{Turck07}
F.~Turck, F.~Roudier, S.~Farrona, M.-L. Martin-Magniette, and et~al.
\newblock Arabidopsis tfl2/lhp1 specifically associates with genes marked by
  trimethylation of histone h3 lysine 27.
\newblock {\em PLoS Genet.v}, 3(6), 2007.

\bibitem[Turner (2000)]{Turner00}
T.~R. Turner.
\newblock Estimating the rate of spread of a viral infection of potato plants
  via mixtures of regressions.
\newblock {\em Appl. Statist.}, 49(3):371--384, 2000.

\end{thebibliography}

\newpage

\begin{figure}
\begin{center}
%\hspace{-2cm}
\begin{onecolumn}
\begin{tabular}{ccc}
\hspace*{-0.5cm}
\begin{tabular}{c}
\hspace{-0.5cm}\includegraphics[scale=0.25]{BelleDistribution.ps}
\end{tabular} &
\hspace*{-.5cm}
\begin{tabular}{c}
\hspace{-0.1cm}\includegraphics[scale=0.23]{HistogrammePourriNimblegen.ps}
\end{tabular} &
\hspace*{-.5cm}
\begin{tabular}{c}
\hspace{-0.1cm}\includegraphics[angle=270,scale=0.23]{Nuage.ps}
\end{tabular}
\end{tabular}
\end{onecolumn}
\end{center}
\caption{\textbf{Left:} Ideal logratio distribution with two
distinct peaks. \textbf{Center:} Logratio distribution on a real
example (Nimblegen array). \textbf{Right:} Associated plot of IP
versus Input (Nimblegen array). \label{Figure:Histogramme}}
\end{figure}

%\clearpage

\begin{figure}
\begin{center}
\includegraphics[scale=0.6]{ComparaisonHistoTheoriques.ps}
\end{center}
\caption{Synthetic data. \textbf{Top:} Two populations with linear
relationship and equal slopes. The corresponding logratio histogram
is bimodal. \textbf{Bottom:} Two populations with linear
relationship but different slopes. The corresponding logratio
histogram is unimodal. \label{ComparaisonHistoTheoriques}}
\end{figure}

\begin{figure}
\begin{center}
\begin{onecolumn}
\includegraphics[scale=0.3]{8516_annotation.ps}
\end{onecolumn}
\end{center}
\caption{Genomic region of chromosome 4 of \textit{Arabidopsis
thaliana} visualized with SignalMap$^{TM}$. In the first line
annotation is given, the boxes are the genes, the second line
shows the genomic regions found by the NimbleGen software. Thick
bars are not enriched and the others bars are colored according to a
FDR value and are all enriched. The third line gives the probes
declared enriched by ChIPmix with $\alpha=0.01$. The fourth line
gives the results of ChIPOTle (window=500, step=100).
\label{Figure:SignalMAP}}
\end{figure}

\begin{figure}
  \begin{center}
    \begin{tabular}{cc}
%       \includegraphics[angle=270,scale=0.3]{Graph_regression_Rep1_chr4bis.ps}%, angle=90 &
\hspace*{-.5cm}
\begin{tabular}{c}
       \includegraphics[angle=270,scale=0.3]{Graph_regression_Rep1_chr4bis.ps}%, angle=90
\end{tabular}&
\hspace*{-1cm}
\begin{tabular}{c}
       \includegraphics[scale=0.3]{DiagrammeVenn_couleur_legend2.ps}
\end{tabular}
    \end{tabular}
  \end{center}
\caption{\textbf{left:} IP intensities vs Input Intensities colored according to
the posterior probabilities $\hat{\tau}_i$. Colors change every 20
\% (blue: $\hat{\tau}_i<20\%$, red: $\hat{\tau}_i>80\%$). The two lines are the two estimated linear regressions of the
mixture. \textbf{right:} Venn diagram summarizing the results of the three methods.\label{Figure:Resultats}}
\end{figure}







\end{document}




