\documentclass[dvips,lscape]{foils}
%\documentclass[dvips, french]{slides}
\textwidth 18.5cm
\textheight 25cm 
\topmargin -1cm 
\oddsidemargin  -1cm 
\evensidemargin  -1cm

% Maths
\usepackage{amsfonts, amsmath, amssymb,graphics}

\newcommand{\coefbin}[2]{\left( 
    \begin{array}{c} #1 \\ #2 \end{array} 
  \right)}
\newcommand{\Bcal}{\mathcal{B}}
\newcommand{\Ccal}{\mathcal{C}}
\newcommand{\Dcal}{\mathcal{D}}
\newcommand{\Ecal}{\mathcal{E}}
\newcommand{\Gcal}{\mathcal{G}}
\newcommand{\Mcal}{\mathcal{M}}
\newcommand{\Ncal}{\mathcal{N}}
\newcommand{\Pcal}{\mathcal{P}}
\newcommand{\Rcal}{\mathcal{R}}
\newcommand{\Lcal}{\mathcal{L}}
\newcommand{\Tcal}{\mathcal{T}}
\newcommand{\Ucal}{\mathcal{U}}
\newcommand{\alphabf}{\mbox{\mathversion{bold}{$\alpha$}}}
\newcommand{\betabf}{\mbox{\mathversion{bold}{$\beta$}}}
\newcommand{\gammabf}{\mbox{\mathversion{bold}{$\gamma$}}}
\newcommand{\mubf}{\mbox{\mathversion{bold}{$\mu$}}}
\newcommand{\Pibf}{\mbox{\mathversion{bold}{$\Pi$}}}
\newcommand{\psibf}{\mbox{\mathversion{bold}{$\psi$}}}
\newcommand{\Sigmabf}{\mbox{\mathversion{bold}{$\Sigma$}}}
\newcommand{\taubf}{\mbox{\mathversion{bold}{$\tau$}}}
\newcommand{\Hbf}{{\bf H}}
\newcommand{\Ibf}{{\bf I}}
\newcommand{\Sbf}{{\bf S}}
\newcommand{\mbf}{{\bf m}}
\newcommand{\ubf}{{\bf u}}
\newcommand{\vbf}{{\bf v}}
\newcommand{\xbf}{{\bf x}}
\newcommand{\Xbf}{{\bf X}}
\newcommand{\Esp}{{\mathbb E}}
\newcommand{\Var}{{\mathbb V}}
\newcommand{\Cov}{{\mathbb C}\mbox{ov}}
\newcommand{\Ibb}{{\mathbb I}}
\newcommand{\Rbb}{\mathbb{R}}

% sommes
\newcommand{\sumk}{\sum_k}
\newcommand{\sumt}{\sum_{t \in I_k}}
\newcommand{\sumth}{\sum_{t=t_{k-1}^{(h)}+1}^{t_k^{(h)}}}
\newcommand{\sump}{\sum_{p=1}^{P}}
\newcommand{\suml}{\sum_{\ell=1}^{P}}
\newcommand{\sumtau}{\sum_k \hat{\tau}_{kp}}

% Couleur et graphiques
\usepackage{color}
\usepackage{graphics}
\usepackage{epsfig} 
\usepackage{pstcol}
\newcommand{\Example}{tfl2_1_ResReg}

% Texte
\usepackage{lscape}
\usepackage{fancyheadings, rotating, enumerate}
%\usepackage[french]{babel}
\usepackage[latin1]{inputenc}
\definecolor{darkgreen}{cmyk}{0.5, 0, 0.5, 0.5}
\definecolor{orange}{cmyk}{0, 0.6, 0.8, 0}
\definecolor{jaune}{cmyk}{0, 0.5, 0.5, 0}
\newcommand{\textblue}[1]{\textcolor{blue}{#1}}
\newcommand{\textred}[1]{\textcolor{red}{#1}}
\newcommand{\textgreen}[1]{\textcolor{green}{ #1}}
\newcommand{\textlightgreen}[1]{\textcolor{green}{#1}}
%\newcommand{\textgreen}[1]{\textcolor{darkgreen}{#1}}
\newcommand{\textorange}[1]{\textcolor{orange}{#1}}
\newcommand{\textyellow}[1]{\textcolor{yellow}{#1}}
\newcommand{\refer}[2]{{\sl #1}}

% Sections
%\newcommand{\chapter}[1]{\centerline{\LARGE \textblue{#1}}}
% \newcommand{\section}[1]{\centerline{\Large \textblue{#1}}}
% \newcommand{\subsection}[1]{\noindent{\Large \textblue{#1}}}
% \newcommand{\subsubsection}[1]{\noindent{\large \textblue{#1}}}
% \newcommand{\paragraph}[1]{\noindent {\textblue{#1}}}
% Sectionsred
\newcommand{\chapter}[1]{
  \addtocounter{chapter}{1}
  \setcounter{section}{0}
  \setcounter{subsection}{0}
  {\centerline{\LARGE \textblue{\arabic{chapter} - #1}}}
  }
\newcommand{\section}[1]{
  \addtocounter{section}{1}
  \setcounter{subsection}{0}
  {\centerline{\Large \textblue{\arabic{chapter}.\arabic{section} - #1}}}
  }
\newcommand{\subsection}[1]{
  \addtocounter{subsection}{1}
  {\noindent{\large \textblue{#1}}}
  }
% \newcommand{\subsection}[1]{
%   \addtocounter{subsection}{1}
%   {\noindent{\large \textblue{\arabic{chapter}.\arabic{section}.\arabic{subsection} - #1}}}
%   }
\newcommand{\paragraph}[1]{\noindent{\textblue{#1}}}
\newcommand{\emphase}[1]{\textblue{#1}}

%%%%%%%%%%%%%%%%%%%%%%%%%%%%%%%%%%%%%%%%%%%%%%%%%%%%%%%%%%%%%%%%%%%%%%
%%%%%%%%%%%%%%%%%%%%%%%%%%%%%%%%%%%%%%%%%%%%%%%%%%%%%%%%%%%%%%%%%%%%%%
%%%%%%%%%%%%%%%%%%%%%%%%%%%%%%%%%%%%%%%%%%%%%%%%%%%%%%%%%%%%%%%%%%%%%%
%%%%%%%%%%%%%%%%%%%%%%%%%%%%%%%%%%%%%%%%%%%%%%%%%%%%%%%%%%%%%%%%%%%%%%
\begin{document}
%%%%%%%%%%%%%%%%%%%%%%%%%%%%%%%%%%%%%%%%%%%%%%%%%%%%%%%%%%%%%%%%%%%%%%
%%%%%%%%%%%%%%%%%%%%%%%%%%%%%%%%%%%%%%%%%%%%%%%%%%%%%%%%%%%%%%%%%%%%%%
%%%%%%%%%%%%%%%%%%%%%%%%%%%%%%%%%%%%%%%%%%%%%%%%%%%%%%%%%%%%%%%%%%%%%%
%%%%%%%%%%%%%%%%%%%%%%%%%%%%%%%%%%%%%%%%%%%%%%%%%%%%%%%%%%%%%%%%%%%%%%
\landscape
\newcounter{chapter}
\newcounter{section}
\newcounter{subsection}
\setcounter{chapter}{0}
\headrulewidth 0pt 
\pagestyle{fancy} 
\cfoot{}
\rfoot{\begin{rotate}{90}{
      %\hspace{1cm} \tiny S. Robin: Segmentation-clustering for CGH
      }\end{rotate}}
\rhead{\begin{rotate}{90}{
      \hspace{-.5cm} \tiny \thepage
      }\end{rotate}}

%%%%%%%%%%%%%%%%%%%%%%%%%%%%%%%%%%%%%%%%%%%%%%%%%%%%%%%%%%%%%%%%%%%%%%
%%%%%%%%%%%%%%%%%%%%%%%%%%%%%%%%%%%%%%%%%%%%%%%%%%%%%%%%%%%%%%%%%%%%%%
\begin{center}
  \textblue{\LARGE ChIPmix : Mixture model of regressions}
  \textblue{\LARGE   for ChIP-chip experiment analysis}
  
  \vspace{1cm}
  {\large M.-L. Martin-Magniette$^{1,2}$, T. Mary-Huard$^{1}$,}
  
  {\large C.  B�rard$^{2}$, S. Robin$^{1}$}

  \vspace{1cm}
  ($^1$) UMR AgroParisTech/INRA MIA, team Statistics and Genome, Paris, France \\
  ($^2$) UMR INRA/CNRS/UEVE Plant Genomic Research Unit, Evry, France

\end{center}

%%%%%%%%%%%%%%%%%%%%%%%%%%%%%%%%%%%%%%%%%%%%%%%%%%%%%%%%%%%%
%%%%%%%%%%%%%%%%%%%%%%%%%%%%%%%%%%%%%%%%%%%%%%%%%%%%%%%%%%%%
\newpage
\chapter{Detecting Hybridization in CHip-Chip data}
%%%%%%%%%%%%%%%%%%%%%%%%%%%%%%%%%%%%%%%%%%%%%%%%%%%%%%%%%%%%
%%%%%%%%%%%%%%%%%%%%%%%%%%%%%%%%%%%%%%%%%%%%%%%%%%%%%%%%%%%%

%%%%%%%%%%%%%%%%%%%%%%%%%%%%%%%%%%%%%%%%%%%%%%%%%%%%%%%%%%%%
\bigskip
\section{CHip-Chip technology and biological question}
%%%%%%%%%%%%%%%%%%%%%%%%%%%%%%%%%%%%%%%%%%%%%%%%%%%%%%%%%%%%
$$
\begin{tabular}{cc}
  \begin{tabular}{c}
    \epsfig{file =/home/Q/math_recherche/StatGenome/chIP-chip/Valorisation-methode/Figures/Chip-chip.ps, width=12cm,
    height=12cm, clip=} 
  \end{tabular}
  \begin{tabular}{p{11cm}}
    IP sample: immuno-precipited~DNA\\
    INPUT sample: total DNA\\\\
    Both samples are co-hybridized on a same array\\\\
    Aim to determine which probes have an IP signal 
significantly higher than the INPUT signal. 
  \end{tabular}
 \end{tabular}
$$

%%%%%%%%%%%%%%%%%%%%%%%%%%%%%%%%%%%%%%%%%%%%%%%%%%%%%%%%%%%%
\newpage
\section{Unsupervised Classification Problem}
%%%%%%%%%%%%%%%%%%%%%%%%%%%%%%%%%%%%%%%%%%%%%%%%%%%%%%%%%%%%

\paragraph{Data:} For each probe, we hence get the IP signal and the INPUT 
signal\\\\
\bigskip
\paragraph{Question:} According to these two signals, we \emphase{have to}  
\begin{itemize}
\item \vspace{-0.5cm} Account for the \emphase{link between the IP and
    Input} signals
\item \vspace{-0.5cm} \emphase{Classify each probe} into the
  \emphase{enriched} group or into the \emphase{normal} group,

\end{itemize}

\bigskip
\noindent In addition we \emphase{would like to}
\begin{itemize}
\item \vspace{-0.5cm} Evaluate the quality of the classification;
\item \vspace{-0.5cm} Avoid numerous false detections (false enriched probes).
\end{itemize}

\bigskip
\paragraph{Bibliography:}
 \begin{itemize}
    \item model based on the spatial structure of the data (sliding window, peak detection),
    \item model considering that the whole population can be divided 
      into two groups.
 \end{itemize}
%%%%%%%%%%%%%%%%%%%%%%%%%%%%%%%%%%%%%%%%%%%%%%%%%%%%%%%%%%%%
\newpage
\section{About logratios...}
%%%%%%%%%%%%%%%%%%%%%%%%%%%%%%%%%%%%%%%%%%%%%%%%%%%%%%%%%%%%
  A link between the two signals is observed in all studies. \\ 
  $\leadsto$ All methods already published are based 
  on the logratio IP/Input, considered informative about the probe status.\\
\begin{tabular}{ccc}
  \begin{tabular}{c}
\hspace{-3.0cm}\includegraphics[scale=0.35]{/home/Q/math_recherche/StatGenome/chIP-chip/Valorisation-methode/Figures/BelleDistribution.ps}
  \end{tabular}
  &
  \begin{tabular}{c}
   \epsfig{file=/home/Q/math_recherche/StatGenome/chIP-chip/Valorisation-methode/Figures/Graph_Histogramme_LogRatio_MoyDye_Rep2_chr4.ps,width=5cm,height=6cm,angle=-90,clip=}
  \end{tabular}  
  & 
\begin{tabular}{c}
   \epsfig{file=/home/Q/math_recherche/StatGenome/chIP-chip/Valorisation-methode/Figures/Graph_MoyDye_Rep2_chr4_projection.ps,width=5cm,height=6cm,angle=-90,clip=}
 \end{tabular} \\
 Good case (Buck \& Lieb, 2004) & Bad case  & log(IP) versus log(Input) \\
\end{tabular}\\
Explanation of the poor performance of the log-ratio
\begin{itemize}
\item Technical difficulties to obtainthe IP sample
\item Possible cross-hybridization phenomena
\end{itemize}

%%%%%%%%%%%%%%%%%%%%%%%%%%%%%%%%%%%%%%%%%%%%%%%%%%%%%%%%%%%%
\newpage
\paragraph{About logratios on synthetic data.}
%%%%%%%%%%%%%%%%%%%%%%%%%%%%%%%%%%%%%%%%%%%%%%%%%%%%%%%%%%%%

\begin{tabular}{c}
  \includegraphics[scale=0.6]{/home/Q/math_recherche/StatGenome/chIP-chip/Valorisation-methode/Figures/ComparaisonHistoTheoriques.ps}
\end{tabular}

\noindent \textbf{Top:} Two populations with linear relationship and equal slopes. \\
The corresponding logratio histogram is bimodal. \\
\noindent \textbf{Bottom:} Two populations with linear relationship but different slopes.\\ 
The corresponding logratio histogram is unimodal.

%%%%%%%%%%%%%%%%%%%%%%%%%%%%%%%%%%%%%%%%%%%%%%%%%%%%%%%%%%%%
\newpage
\chapter{ChIPmix: Mixture Model of Regressions}
%%%%%%%%%%%%%%%%%%%%%%%%%%%%%%%%%%%%%%%%%%%%%%%%%%%%%%%%%%%%
%%%%%%%%%%%%%%%%%%%%%%%%%%%%%%%%%%%%%%%%%%%%%%%%%%%%%%%%%%%%
\begin{tabular}{cc}
\begin{tabular}{p{10cm}}
    A closer look shows that the relation between IP and Input
    may differ between the 'normal' group and 'enriched' group.\\ \\
    $\leadsto$ it is worth working directly with the 2 signals rather than the logratio
  \end{tabular}
  &
  \begin{tabular}{c}
    \epsfig{file = /home/Q/math_recherche/StatGenome/chIP-chip/Valorisation-methode/Figures/NuageIPInput.ps,width=12cm,height=12cm, angle=0,clip=} 
  \end{tabular}
\end{tabular}
%%%%%%%%%%%%%%%%%%%%%%%%%%%%%%%%%%%%%%%%%%%%%%%%%%%%%%%%%%%%
\newpage
\section{Mixture model}
%%%%%%%%%%%%%%%%%%%%%%%%%%%%%%%%%%%%%%%%%%%%%%%%%%%%%%%%%%%%
We assume that each probe $i$ has probability $\pi$
to be enriched:
$$
\Pr\{\text{Probe $i$ enriched}\} = \pi, 
\qquad
\Pr\{\text{Probe $i$ normal}\} = 1 - \pi, 
$$
and that the relation between log-IP ($Y_i$) and log-Input ($X_i$) depends on
the status of the probe:
$$
Y_i = \left\{ \begin{array}{ll}
    a_0 + b_0 X_i + E_i & \quad \text{if $i$ is normal} \\
    \\
    a_1 + b_1 X_i + E_i & \quad \text{if $i$ is enriched}
  \end{array} \right.
$$

The marginal distribution of $Y_i$ for a given $X_i$ is
$(1-\pi) \phi_0(Y_i | x_i) + \pi \phi_1(Y_i | x_i),$
where $\phi_j(\cdot|x)$ stands for the probability density function of
a Gaussian distribution with mean $a_j+b_j x$ and variance $\sigma^2$.

\paragraph{Comparison with the 'Logratio' analysis.} 
Analysis based on the logratio is the particular case of the
regression mixture model where
$b_0 = b_1 = 1.$

%%%%%%%%%%%%%%%%%%%%%%%%%%%%%%%%%%%%%%%%%%%%%%%%%%%%%%%%%%%%
\newpage
\section{Parameter Estimation and Probe Classification}
%%%%%%%%%%%%%%%%%%%%%%%%%%%%%%%%%%%%%%%%%%%%%%%%%%%%%%%%%%%%

\bigskip
\paragraph{Task.} We have to estimate
\begin{itemize}
\item \vspace{-0.5cm} the proportion of enriched probes: $\pi$;
\item \vspace{-0.5cm} the regression parameters: intercepts ($a_0$,
  $a_1$), slopes ($b_0$, $b_1$) and the variance $\sigma^2$.
\end{itemize}

\bigskip\bigskip
\paragraph{Algorithm.} This can be done using the E-M algorithm which
alternates
\begin{description}
\item[E-step:] \vspace{-0.5cm} prediction of the probe status given
  the parameters;
\item[M-step:] \vspace{-0.5cm} estimation of the parameters given the
  (predicted) probe status.
\end{description}

\bigskip\bigskip
\paragraph{Posterior probability.} The status prediction is based on
the posterior probability $\tau_i$
$$
\textred{\tau_i = \Pr\{\text{$i$ enriched} \; |\; X_i, Y_i\}},
\qquad 1 - \tau_i = \Pr\{\text{$i$ normal} \; |\; X_i, Y_i\}.
$$
This probability provides a \emphase{probe classification rule}.

%%%%%%%%%%%%%%%%%%%%%%%%%%%%%%%%%%%%%%%%%%%%%%%%%%%%%%%%%%%%
\newpage
\begin{center}
  \epsfig{file=/home/Q/math_recherche/StatGenome/chIP-chip/Valorisation-methode/Figures/PosteriorPlot_MoyDye_Rep2_chr4.eps}
\end{center}

How to classify a probe according to its posterior probability ?

%%%%%%%%%%%%%%%%%%%%%%%%%%%%%%%%%%%%%%%%%%%%%%%%%%%%%%%%%%%%
%%%%%%%%%%%%%%%%%%%%%%%%%%%%%%%%%%%%%%%%%%%%%%%%%%%%%%%%%%%%
\newpage
\chapter{Limiting False Detections}
%%%%%%%%%%%%%%%%%%%%%%%%%%%%%%%%%%%%%%%%%%%%%%%%%%%%%%%%%%%%
%%%%%%%%%%%%%%%%%%%%%%%%%%%%%%%%%%%%%%%%%%%%%%%%%%%%%%%%%%%%

\bigskip
\paragraph{Maximum A Posteriori (MAP) rule.} Probes are usually classified
into their most probable class, using the 50\% threshold:
$$
\begin{array}{rclcl}
  \widehat{\tau_i} & \geq & 50\% & \Rightarrow & \text{$i$ classified as 'enriched'}, \\ 
  \widehat{\tau_i} & < & 50\% & \Rightarrow & \text{$i$ classified as 'normal'}. 
\end{array}
$$
Misclassifications in 'enriched' group or in 'normal' group 
have the same cost.  \\
\paragraph{Controlling false detections.} We want to control the
probability for the $\tau_i$ of a normal probe to fall above the
classification threshold. 

For a \emphase{fixed risk $\alpha$} we calculate the threshold $s$
such that
$$
\textred{s: \qquad \Pr\{\tau_i > s \;|\; \text{$i$ normal}, \;
  \log(\text{Input})=X_i\} \leq \alpha}
$$
and if $\tau_i > s$ then $i$ is classified as 'enriched'.\\
The threshold $s$ depends on both $\alpha$ and the log-Input $X_i$.\\
This control focuses on misclassifications in 'enriched' group.
%%%%%%%%%%%%%%%%%%%%%%%%%%%%%%%%%%%%%%%%%%%%%%%%%%%%%%%%%%%%
%%%%%%%%%%%%%%%%%%%%%%%%%%%%%%%%%%%%%%%%%%%%%%%%%%%%%%%%%%%%
\newpage
\chapter{Applications}
\section{Promoter DNA methylation in the human genome}

\paragraph{Data}(Weber \emph{et al.} 2007)

\noindent NimbleGen tiling array of promoter regions of 15\;609
human genes.

\noindent Each promoter region has 15 probes.

\noindent We  consider the Intermediate CpG class of promoters
(2056).

\paragraph{Results}

\noindent A high proportion of enriched probes
($\widehat{\pi}\approx$80\%).

\noindent 403 of the 460 promoter regions found by Weber have 5
enriched probes or more.

\noindent 38 new promoter candidates with 9 enriched probes or more.

\noindent 1 promoter region found by Weber has no enriched probe.

%%%%%%%%%%%%%%%%%%%%%%%%%%%%%%%%%%%%%%%%%%%%%%%%%%%%%%%%%%%%
%%%%%%%%%%%%%%%%%%%%%%%%%%%%%%%%%%%%%%%%%%%%%%%%%%%%%%%%%%%%
\newpage
\section{Histone modification of \textit{Arabidopsis thaliana}}  

\noindent NimbleGen tiling array of very high density, 
more than 1,000,000 probes ($\approx$200,000 per chromosome)

\noindent Two biological replicates with dye-swap, normalized according to Kerr et al., 2002

\paragraph{Results of ChIPmix}

\noindent At level $\alpha=0.01$, $\sim 20\%$ of the probes are enriched on each chromosome. \\
 
\noindent Overlap of 2/3 between the two biological replicates. \\

\noindent Probes are clustered in genomic regions. \\

\noindent For chromosome 4 : $\widehat{\pi}=0.28$, $\widehat{b}_0=0.75 ,\widehat{b}_1=1.05$\\

\newpage
\paragraph{Comparison with two other methods}

\begin{tabular}{cc}
\begin{tabular}{r}
    \\ \\
    Annotation \\ \\
    NimbleGen  \\ \\
    ChIPmix    \\ \\
    ChIPOTle
\end{tabular}
&
\begin{tabular}{c}
\includegraphics[scale=0.5]{/home/Q/math_recherche/StatGenome/chIP-chip/Valorisation-methode/Figures/8516_chr4_sortie_ecran_annotation.eps}\\
\end{tabular}
\end{tabular} \\

\noindent \textbf{NimbleGen:} finds smaller genomic regions, and fail to identify some targets.

\noindent \textbf{ChIPOTle: } agrees with ChIPmix, not with NimbleGen.
\newpage
\paragraph{Comparison with results of Turck \textit{et al.} (2007)}

Home-made tiling array of chr 4 of \textit{Arabidopsis thaliana} with the same sample

More than 75 $\%$ of the probes declared enriched with the NimbleGen's array
cover genomic regions already found in Turck \textit{et al.} (2007).

\begin{tabular}{c}
\includegraphics[scale=0.6]{/home/Q/math_recherche/StatGenome/chIP-chip/Valorisation-methode/Figures/DiagrammeVenn_couleur_legend.ps}
\end{tabular} \\

For NimbleGen, less than 20\% of the specific probes are
validated targets. In contrast, for ChIPmix, more than 55\% of 
the specific probes are validated targets.

%%%%%%%%%%%%%%%%%%%%%%%%%%%%%%%%%%%%%%%%%%%%%%%%%%%%%%%%%%%%
%%%%%%%%%%%%%%%%%%%%%%%%%%%%%%%%%%%%%%%%%%%%%%%%%%%%%%%%%%%%
\newpage
\chapter{Conclusions}

New method for analyzing chIP-chip data based on mixture model of regressions

A control of the false discovery is provided

Other explicative variables in the regressions can be added

Applications to ChIP-chip data from different arrays and different organisms 
are promising: our results are coherent with results already published and 
new candidates are found.

R program of ChIPmix is available at 
$$\texttt{http://www.agroparistech.fr/mia/outil\_A.html}$$

\newpage
\chapter{In progress}

ChIPmix is generalized to analyze simultaneously the independant biological replicates.
 
The marginal distribution of $Y_i$ for a given $X_i$ is
$$(1-\pi) \phi_{0,k}(Y_i | x_i) + \pi \phi_{1,k}(Y_i | x_i),$$
where $\phi_{j,k}(\cdot|x)$ stands for the probability density function of
a Gaussian distribution with mean $a_{j,k}+b_{j,k} x$ and variance $\sigma^2_k$.

\begin{tabular}{c}
\includegraphics[scale=0.35]{/home/Q/math_recherche/StatGenome/chIP-chip/Valorisation-methode/Figures/capture_SignalMap1.ps}
\end{tabular} 

%%%%%%%%%%%%%%%%%%%%%%%%%%%%%%%%%%%%%%%%%%%%%%%%%%%%%%%%%%%%
%%%%%%%%%%%%%%%%%%%%%%%%%%%%%%%%%%%%%%%%%%%%%%%%%%%%%%%%%%%%

%%%%%%%%%%%%%%%%%%%%%%%%%%%%%%%%%%%%%%%%%%%%%%%%%%%%%%%%%%%%%%%%%%%%%%
%%%%%%%%%%%%%%%%%%%%%%%%%%%%%%%%%%%%%%%%%%%%%%%%%%%%%%%%%%%%%%%%%%%%%%
%%%%%%%%%%%%%%%%%%%%%%%%%%%%%%%%%%%%%%%%%%%%%%%%%%%%%%%%%%%%%%%%%%%%%%
%%%%%%%%%%%%%%%%%%%%%%%%%%%%%%%%%%%%%%%%%%%%%%%%%%%%%%%%%%%%%%%%%%%%%%
\end{document}
%%%%%%%%%%%%%%%%%%%%%%%%%%%%%%%%%%%%%%%%%%%%%%%%%%%%%%%%%%%%%%%%%%%%%%
%%%%%%%%%%%%%%%%%%%%%%%%%%%%%%%%%%%%%%%%%%%%%%%%%%%%%%%%%%%%%%%%%%%%%%
%%%%%%%%%%%%%%%%%%%%%%%%%%%%%%%%%%%%%%%%%%%%%%%%%%%%%%%%%%%%%%%%%%%%%%
%%%%%%%%%%%%%%%%%%%%%%%%%%%%%%%%%%%%%%%%%%%%%%%%%%%%%%%%%%%%%%%%%%%%%%
