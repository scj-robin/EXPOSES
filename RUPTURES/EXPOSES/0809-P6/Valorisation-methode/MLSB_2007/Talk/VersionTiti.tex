\documentclass[dvips, lscape]{foils}
%\documentclass[dvips, french]{slides}
\textwidth 18.5cm \textheight 25cm \topmargin -1cm \oddsidemargin
-1cm \evensidemargin  -1cm

% Maths
\usepackage{amsfonts, amsmath, amssymb,graphics}

\newcommand{\coefbin}[2]{\left(
    \begin{array}{c} #1 \\ #2 \end{array}
  \right)}


% Couleur et graphiques
\usepackage{color}
\usepackage{graphics}
\usepackage{epsfig}
\usepackage{pstcol}
\newcommand{\Example}{tfl2_1_ResReg}

% Texte
\usepackage{lscape}
\usepackage{fancyheadings, rotating, enumerate}
%\usepackage[french]{babel}
\usepackage[latin1]{inputenc}
\definecolor{darkgreen}{cmyk}{0.5, 0, 0.5, 0.5}
\definecolor{orange}{cmyk}{0, 0.6, 0.8, 0}
\definecolor{jaune}{cmyk}{0, 0.5, 0.5, 0}
\newcommand{\textblue}[1]{\textcolor{blue}{#1}}
\newcommand{\textred}[1]{\textcolor{red}{#1}}
\newcommand{\textgreen}[1]{\textcolor{green}{ #1}}
\newcommand{\textlightgreen}[1]{\textcolor{green}{#1}}
%\newcommand{\textgreen}[1]{\textcolor{darkgreen}{#1}}
\newcommand{\textorange}[1]{\textcolor{orange}{#1}}
\newcommand{\textyellow}[1]{\textcolor{yellow}{#1}}
\newcommand{\refer}[2]{{\sl #1}}

% Sections
%\newcommand{\chapter}[1]{\centerline{\LARGE \textblue{#1}}}
% \newcommand{\section}[1]{\centerline{\Large \textblue{#1}}}
% \newcommand{\subsection}[1]{\noindent{\Large \textblue{#1}}}
% \newcommand{\subsubsection}[1]{\noindent{\large \textblue{#1}}}
% \newcommand{\paragraph}[1]{\noindent {\textblue{#1}}}
% Sectionsred
\newcommand{\chapter}[1]{
  \addtocounter{chapter}{1}
  \setcounter{section}{0}
  \setcounter{subsection}{0}
  {\centerline{\LARGE \textblue{\arabic{chapter} - #1}}}
  }
\newcommand{\section}[1]{
  \addtocounter{section}{1}
  \setcounter{subsection}{0}
  {\centerline{\Large \textblue{\arabic{chapter}.\arabic{section} - #1}}}
  }
\newcommand{\subsection}[1]{
  \addtocounter{subsection}{1}
  {\noindent{\large \textblue{#1}}}
  }
% \newcommand{\subsection}[1]{
%   \addtocounter{subsection}{1}
%   {\noindent{\large \textblue{\arabic{chapter}.\arabic{section}.\arabic{subsection} - #1}}}
%   }
\newcommand{\paragraph}[1]{\noindent{\textblue{#1}}}
\newcommand{\emphase}[1]{\textblue{#1}}

%%%%%%%%%%%%%%%%%%%%%%%%%%%%%%%%%%%%%%%%%%%%%%%%%%%%%%%%%%%%%%%%%%%%%%
%%%%%%%%%%%%%%%%%%%%%%%%%%%%%%%%%%%%%%%%%%%%%%%%%%%%%%%%%%%%%%%%%%%%%%
%%%%%%%%%%%%%%%%%%%%%%%%%%%%%%%%%%%%%%%%%%%%%%%%%%%%%%%%%%%%%%%%%%%%%%
%%%%%%%%%%%%%%%%%%%%%%%%%%%%%%%%%%%%%%%%%%%%%%%%%%%%%%%%%%%%%%%%%%%%%%
\begin{document}
%%%%%%%%%%%%%%%%%%%%%%%%%%%%%%%%%%%%%%%%%%%%%%%%%%%%%%%%%%%%%%%%%%%%%%
%%%%%%%%%%%%%%%%%%%%%%%%%%%%%%%%%%%%%%%%%%%%%%%%%%%%%%%%%%%%%%%%%%%%%%
%%%%%%%%%%%%%%%%%%%%%%%%%%%%%%%%%%%%%%%%%%%%%%%%%%%%%%%%%%%%%%%%%%%%%%
%%%%%%%%%%%%%%%%%%%%%%%%%%%%%%%%%%%%%%%%%%%%%%%%%%%%%%%%%%%%%%%%%%%%%%
\landscape
\newcounter{chapter}
\newcounter{section}
\newcounter{subsection}
\setcounter{chapter}{0} \headrulewidth 0pt \pagestyle{fancy}
\cfoot{} \rfoot{\begin{rotate}{90}{
      %\hspace{1cm} \tiny S. Robin: Segmentation-clustering for CGH
      }\end{rotate}}
\rhead{\begin{rotate}{90}{
      \hspace{-.5cm} \tiny \thepage
      }\end{rotate}}

%%%%%%%%%%%%%%%%%%%%%%%%%%%%%%%%%%%%%%%%%%%%%%%%%%%%%%%%%%%%%%%%%%%%%%
%%%%%%%%%%%%%%%%%%%%%%%%%%%%%%%%%%%%%%%%%%%%%%%%%%%%%%%%%%%%%%%%%%%%%%
\begin{center}
  \textblue{\LARGE Regression Mixture Model }

  \textblue{\LARGE  for ChIP-chip Analysis}

  \vspace{1cm}
  {\large M.-L. Martin-Magniette$^{1,2}$, \underline{T. Mary-Huard}$^{1}$,}

  {\large C.  B�rard$^{2}$, S. Robin$^{1}$}

  \vspace{1cm}
  ($^1$) UMR 518 AgroParisTech/INRA, Paris, France \\
  ($^2$) UMR INRA/CNRS/UEVE Plant Genomic Research Unit, Evry, France

\end{center}

%%%%%%%%%%%%%%%%%%%%%%%%%%%%%%%%%%%%%%%%%%%%%%%%%%%%%%%%%%%%
%%%%%%%%%%%%%%%%%%%%%%%%%%%%%%%%%%%%%%%%%%%%%%%%%%%%%%%%%%%%
\newpage
\chapter{Detecting Hybridization in ChIP-Chip data}
%%%%%%%%%%%%%%%%%%%%%%%%%%%%%%%%%%%%%%%%%%%%%%%%%%%%%%%%%%%%
%%%%%%%%%%%%%%%%%%%%%%%%%%%%%%%%%%%%%%%%%%%%%%%%%%%%%%%%%%%%

%%%%%%%%%%%%%%%%%%%%%%%%%%%%%%%%%%%%%%%%%%%%%%%%%%%%%%%%%%%%
\bigskip
\section{ChIP-Chip technology and biological question}
%%%%%%%%%%%%%%%%%%%%%%%%%%%%%%%%%%%%%%%%%%%%%%%%%%%%%%%%%%%%
$$
\begin{tabular}{cc}
  \begin{tabular}{c}
  \epsfig{file = ../../Figures/Chip-chip.ps, width=12cm,height=12cm, clip=}
 %\includegraphics{HistogrammeChr4.png}%, width=12cm,height=12cm, clip=}
  \end{tabular}
  \begin{tabular}{p{11cm}}
    IP sample: immuno-precipited~DNA\\
    INPUT sample: total DNA\\\\
    Both samples are co-hybridized on a same array\\\\
    Aim to determine which probes have an IP signal
significantly higher than the INPUT signal.
  \end{tabular}
 \end{tabular}
$$

%%%%%%%%%%%%%%%%%%%%%%%%%%%%%%%%%%%%%%%%%%%%%%%%%%%%%%%%%%%%
\newpage
\section{Unsupervised Classification Problem}
%%%%%%%%%%%%%%%%%%%%%%%%%%%%%%%%%%%%%%%%%%%%%%%%%%%%%%%%%%%%

\paragraph{Data:} For each probe, we hence get the IP signal and the INPUT
signal\\\\
\bigskip
\paragraph{Question:} According to these two signals, we \emphase{have to}
\begin{itemize}
%\item \vspace{-0.5cm} Account for the \emphase{link between the IP and
%    Input} signals
\item \vspace{-0.5cm} \emphase{Classify each probe} into the
  \emphase{enriched} group or into the \emphase{normal} group,
\end{itemize}

\bigskip
\noindent In addition we \emphase{would like to}
\begin{itemize}
\item \vspace{-0.5cm} Evaluate the quality of the classification,
\item \vspace{-0.5cm} Avoid numerous false detections (false enriched probes).
\end{itemize}

\bigskip
\paragraph{Bibliography:}
 \begin{itemize}
    \item model based on the spatial structure of the data (sliding window, HMM),
    \item model considering that the whole population can be divided
      into two groups.
 \end{itemize}
%%%%%%%%%%%%%%%%%%%%%%%%%%%%%%%%%%%%%%%%%%%%%%%%%%%%%%%%%%%%
%%%%%%%%%%%%%%%%%%%%%%%%%%%%%%%%%%%%%%%%%%%%%%%%%%%%%%%%%%%%
\newpage

%%%%%%%%%%%%%%%%%%%%%%%%%%%%%%%%%%%%%%%%%%%%%%%%%%%%%%%%%%%%
%%%%%%%%%%%%%%%%%%%%%%%%%%%%%%%%%%%%%%%%%%%%%%%%%%%%%%%%%%%%

%%%%%%%%%%%%%%%%%%%%%%%%%%%%%%%%%%%%%%%%%%%%%%%%%%%%%%%%%%%%
\bigskip
\section{About logratios...}
%%%%%%%%%%%%%%%%%%%%%%%%%%%%%%%%%%%%%%%%%%%%%%%%%%%%%%%%%%%%
%$$
%\begin{tabular}{cc}
%  \begin{tabular}{p{10cm}}
%All methods already published are based on the logratio IP/Input \\
%$\Rightarrow$ Assume that the logratio distribution is informative
%about the status.
%  \end{tabular}
%  &
%  \begin{tabular}{c}
%   \epsfig{file = ../../Figures/Graph_Histogramme_LogRatio_MoyDye_Rep2_chr4.ps,width=12cm,    height=12cm, angle=-90,clip=}
%  \end{tabular}
%\end{tabular}
%$$
\begin{tabular}{cc}
  \begin{tabular}{c}
\hspace{-3.5cm}\includegraphics[scale=0.6]{BelleDistribution.ps}
  \end{tabular}
  &
  \begin{tabular}{c}
   \epsfig{file = ../../Figures/Graph_Histogramme_LogRatio_MoyDye_Rep2_chr4.ps,width=10cm,    height=12cm, angle=-90,clip=}
  \end{tabular} \\ Good case (Buck \& Lieb, 2004) & Bad case \\ & \\
\end{tabular}\\
All methods already published are based on the logratio IP/Input \\
$\Rightarrow$ Assume that the logratio distribution is informative
about the status.

%%%%%%%%%%%%%%%%%%%%%%%%%%%%%%%%%%%%%%%%%%%%%%%%%%%%%%%%%%%%
\newpage
$$
\begin{tabular}{cc}
  \begin{tabular}{p{10cm}}
    A closer look shows that the relation between IP and Input
    may differ between the 'normal' group and 'enriched' group.
  \end{tabular}
  &
  \begin{tabular}{c}
    \epsfig{file = ./Nuage.ps,width=12cm,height=12cm,clip=}
  \end{tabular}
\end{tabular}
$$

%%%%%%%%%%%%%%%%%%%%%%%%%%%%%%%%%%%%%%%%%%%%%%%%%%%%%%%%%%%%
\newpage
\chapter{Mixture Model of Regressions}
\section{Mixture model}
%%%%%%%%%%%%%%%%%%%%%%%%%%%%%%%%%%%%%%%%%%%%%%%%%%%%%%%%%%%%

\bigskip
\paragraph{Model.} We assume that each probe $i$ has probability $\pi$
to be enriched:
$$
\Pr\{\text{Probe $i$ enriched}\} = \pi, \qquad \Pr\{\text{Probe $i$
normal}\} = 1 - \pi,
$$
and that the relation between log-IP ($Y_i$) and log-Input ($X_i$)
depends on the status of the probe:
$$
Y_i = \left\{ \begin{array}{ll}
    a_0 + b_0 X_i + E_i & \quad \text{if $i$ is normal} \\
    \\
    a_1 + b_1 X_i + E_i & \quad \text{if $i$ is enriched}
  \end{array} \right. \ \ \ V(Y_i)=\sigma^2
$$

\bigskip\bigskip
\paragraph{Comparison with the 'Logratio' analysis.} The standard
analysis based on the log(IP/Input) is the particular case of the
regression mixture model where
$$
b_0 = b_1 = 1.
$$

%%%%%%%%%%%%%%%%%%%%%%%%%%%%%%%%%%%%%%%%%%%%%%%%%%%%%%%%%%%%
\newpage
\section{Parameter Estimation and Probe Classification}
%%%%%%%%%%%%%%%%%%%%%%%%%%%%%%%%%%%%%%%%%%%%%%%%%%%%%%%%%%%%

\bigskip
\paragraph{Task.} We have to estimate proportion $\pi$, intercepts
$a_0$ and $a_1$, slopes $b_0$ and $b_1$, and variance $\sigma^2$
%\begin{itemize}
%\item \vspace{-0.5cm} the proportion of enriched probes: $\pi$;
%\item \vspace{-0.5cm} the regression parameters: intercepts ($a_0$,
%  $a_1$), slopes ($b_0$, $b_1$) and variances ($\sigma^2_0$, $\sigma_1^2$).
%\end{itemize}

\bigskip\bigskip
\paragraph{Algorithm.} This can be done using the E-M algorithm which
alternates
\begin{description}
\item[E-step:] \vspace{-0.5cm} prediction of the probe status given
  the parameters;
\item[M-step:] \vspace{-0.5cm} estimation of the parameters given the
  (predicted) probe status.
\end{description}

\bigskip\bigskip
\paragraph{Model selection.} BIC criterion to select between a 1 or
2 population model.

\bigskip\bigskip
\paragraph{Posterior probability.} The status prediction is based on
the posterior probability $\tau_i$
$$
\textred{\tau_i = \Pr\{\text{$i$ enriched} \; |\; X_i, Y_i\}},
\qquad 1 - \tau_i = \Pr\{\text{$i$ normal} \; |\; X_i, Y_i\}.
$$
%This probability provides a \emphase{probe classification rule}.

%%%%%%%%%%%%%%%%%%%%%%%%%%%%%%%%%%%%%%%%%%%%%%%%%%%%%%%%%%%%
\newpage
\begin{center}
    %\epsfig{file=../../Figures/PosteriorPlot_MoyDye_Rep2_chr4.eps}
    \includegraphics[scale=1]{Graph_regression_Rep1_chr4bis.ps}
%,
      %      width=12cm, height=12cm, angle=0, clip=}
\end{center}

How to classify a probe according to its posterior probability ?

%%%%%%%%%%%%%%%%%%%%%%%%%%%%%%%%%%%%%%%%%%%%%%%%%%%%%%%%%%%%
%%%%%%%%%%%%%%%%%%%%%%%%%%%%%%%%%%%%%%%%%%%%%%%%%%%%%%%%%%%%
\newpage
\chapter{Limiting False Detections}
%%%%%%%%%%%%%%%%%%%%%%%%%%%%%%%%%%%%%%%%%%%%%%%%%%%%%%%%%%%%
%%%%%%%%%%%%%%%%%%%%%%%%%%%%%%%%%%%%%%%%%%%%%%%%%%%%%%%%%%%%

\bigskip
\paragraph{Maximum A Posteriori (MAP) rule.}
$$
\begin{array}{rclcl}
  \widehat{\tau_i} & \geq & 50\% & \Rightarrow & \text{$i$ classified as
  'enriched'}, \\
  \widehat{\tau_i} & < & 50\% & \Rightarrow & \text{$i$ classified as
  'normal'}.
\end{array}
$$
Misclassifications in 'enriched' group or in 'normal' group have the
same cost.

\paragraph{Controlling false detections.} Control the
probability for the $\tau_i$ of a normal probe to fall above the
classification threshold.

For a \emphase{fixed risk $\alpha$} we calculate the threshold $s$
such that
$$
\textred{s: \qquad \Pr\{\tau_i > s \;|\; \text{$i$ normal}, \;
  \log(\text{Input})=X_i\} = \alpha}
$$
and if $\widehat{\tau_i} > s$ then $i$ is classified as
'enriched'.\\
The threshold $s$ depends on both $\alpha$ and the log-Input $X_i$.\\

%%%%%%%%%%%%%%%%%%%%%%%%%%%%%%%%%%%%%%%%%%%%%%%%%%%%%%%%%%%%
%%%%%%%%%%%%%%%%%%%%%%%%%%%%%%%%%%%%%%%%%%%%%%%%%%%%%%%%%%%%
\newpage

\newpage
\chapter{Applications}

\section{Promoter DNA methylation in the human genome}

\paragraph{Data}(Weber \emph{et al.} 2007)

\noindent NimbleGen tiling array of promoter regions of 15\;609
human genes.

\noindent Each promoter region has 15 probes.

\noindent We  consider the Intermediate CpG class of promoters
(2056).

\paragraph{Results}

\noindent A high proportion of enriched probes
($\widehat{\pi}\approx$80\%).

\noindent 403 of the 460 promoter regions found by Weber have 5
enriched probes or more.

\noindent 38 new promoter candidates with 9 enriched probes or more.

\noindent 1 promoter region found by Weber has no enriched probe.

\newpage


\section{NimbleGen tiling array of \textit{Arabidopsis thaliana}}

\paragraph{Data}\\
Study of the histone modification H3K9me3.

\noindent Very high density: more than 1,000,000 probes
($\approx$200,000 per chromosome)

\noindent 2 biological replicates with dye-swap, normalized
according to Kerr et al., 2002

\noindent 3 different analysis methods: NimbleGen software,
ChIPOTle, and Mixture of regression (ChIPmix).

\paragraph{Chloroplastic genome}\\
No target is expected $\Rightarrow$ negative control.

\noindent \textbf{ChIPmix:} BIC selects a single regression model,
i.e. no probe is declared enriched.

\noindent \hspace{-0.7cm} \textbf{ ChIPOTle:} finds 10 to 16 peaks
(depending on parameters).

\noindent \textbf{Nimblegen:} not available.

\newpage

\paragraph{Peak detection}\\
\begin{tabular}{cc}
\begin{tabular}{r}
    \\ \\
    Annotation \\ \\
    NimbelGen  \\ \\
    ChIPmix    \\ \\
    ChIPOTle
\end{tabular}
&
\begin{tabular}{c}
\includegraphics[scale=0.5]{8516_annotation.ps}
%\epsfig{file=../../Figures/8516_chr4_sortie_ecran_annotation.eps,scale=0.5}
\\
\end{tabular}
\end{tabular} \\
\textbf{ChIPmix: } $\alpha=0.01$. Probes are clustered in genomic regions.

\noindent \textbf{NimbleGen:} finds smaller genomic regions and
fail to identify some targets.

\noindent \textbf{ChIPOTle: } agrees with ChIPmix, not with NimbleGen.

\newpage
\paragraph{Biological validation} (Turck \textit{et al.}, 2007)

\begin{tabular}{c}
\includegraphics[scale=0.6]{DiagrammeVenn_couleur.ps}
%\epsfig{file=../../Figures/8516_chr4_sortie_ecran_annotation.eps,scale=0.5}
\end{tabular} \\
\textbf{ChIPmix:} + 55\% of the specific probes are validated
targets.

\noindent \textbf{Nimblegen:} - 20\% of the specific probes are
validated targets.



%\noindent More than 75 $\%$ of the probes found MMR and ChIPOTle
%cover genomic regions were biologically validated in Turck
%\textit{et al.} (2007).


\newpage
\chapter{Conclusions}

\noindent New method for analyzing chIP-chip data based on mixture
model of regressions

\paragraph{May be applied to:}

$\bullet$ different biological questions (histone modification, DNA
methylation...)

$\bullet$ different organisms (Arabidopsis, Human)

$\bullet$ different 2-color technologies (tiling/promoter arrays,
high density arrays)

$\bullet$ different situations, whatever the proportion $\pi$

\paragraph{Extensions}

To estimate the proportion of undiscovered enriched probes

Other explicative variables in the regressions can be added (Hits,
CpG rate)





%%%%%%%%%%%%%%%%%%%%%%%%%%%%%%%%%%%%%%%%%%%%%%%%%%%%%%%%%%%%
%%%%%%%%%%%%%%%%%%%%%%%%%%%%%%%%%%%%%%%%%%%%%%%%%%%%%%%%%%%%

%%%%%%%%%%%%%%%%%%%%%%%%%%%%%%%%%%%%%%%%%%%%%%%%%%%%%%%%%%%%%%%%%%%%%%
%%%%%%%%%%%%%%%%%%%%%%%%%%%%%%%%%%%%%%%%%%%%%%%%%%%%%%%%%%%%%%%%%%%%%%
%%%%%%%%%%%%%%%%%%%%%%%%%%%%%%%%%%%%%%%%%%%%%%%%%%%%%%%%%%%%%%%%%%%%%%
%%%%%%%%%%%%%%%%%%%%%%%%%%%%%%%%%%%%%%%%%%%%%%%%%%%%%%%%%%%%%%%%%%%%%%
\end{document}
%%%%%%%%%%%%%%%%%%%%%%%%%%%%%%%%%%%%%%%%%%%%%%%%%%%%%%%%%%%%%%%%%%%%%%
%%%%%%%%%%%%%%%%%%%%%%%%%%%%%%%%%%%%%%%%%%%%%%%%%%%%%%%%%%%%%%%%%%%%%%
%%%%%%%%%%%%%%%%%%%%%%%%%%%%%%%%%%%%%%%%%%%%%%%%%%%%%%%%%%%%%%%%%%%%%%
%%%%%%%%%%%%%%%%%%%%%%%%%%%%%%%%%%%%%%%%%%%%%%%%%%%%%%%%%%%%%%%%%%%%%%
