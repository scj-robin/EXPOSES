\documentclass{llncs}
\usepackage{makeidx}  % allows for indexgeneration
\usepackage{graphicx,graphics,color, epsfig}
\usepackage{amsfonts,amssymb,amsmath}
\usepackage{latexsym}

\newcommand{\replace}[2]{\textcolor{red}{{#1}} \textcolor{blue}{#2}}

\begin{document}
\frontmatter          % for the preliminaries
\pagestyle{headings}  % switches on printing of running heads
%\addtocmark{Hamiltonian Mechanics} % additional mark in the TOC

\title{Mixture model of regressions for ChIP-chip experiment analysis}
\titlerunning{ChIP-chip analysis} \author{Marie-Laure
  Martin-Magniette\inst{1,2, *}\and Tristan Mary-Huard\inst{1, *} \and
  Caroline B\'erard\inst{1,2} \and St\'ephane Robin\inst{1}}

\authorrunning{Martin-Magniette \textit{et al.}}   % abbreviated author list (for running head)
\institute{UMR AgroParisTech/INRA MIA 518\\ 16 rue Claude Bernard
75231 Paris Cedex 05, France
\\
\email{marie\_laure.martin@agroparistech.fr}\\
%WWW home page:\texttt{http://www.inapg.fr/ens_rech/maths/index.html}
\and URGV UMR INRA/CNRS/UEVE\\ 2 rue Gaston Cr\'emieux, CP5708,
91057, Evry Cedex, France \\
(*) These authors equally contributed to this work.}

\maketitle              % typeset the title of the contribution

\begin{abstract}
  Chromatin immunoprecipitation (ChIP) is a well-established procedure
  to investigate proteins associated with DNA. We propose a
  method based on a linear regression mixture model to identify
  actual binding targets of the protein under study and we are able to
  control the number of false-positives. We apply our method
  on NimbleGen technology data in order to determine genomic regions
  marked by a specific histone modification.
\end{abstract}

\section{Introduction}
Chromatin immunoprecipitation (ChIP) is a well-established procedure
used to investigate proteins associated with DNA. ChIP on chip
involves analysis of DNA recovered from ChIP experiments by
hybridization to microarray. In a two-color ChIP-chip experiment,
two samples are compared: DNA fragments crosslinked to a protein of
interest (IP), and genomic DNA (Input). The two samples are
differentially labeled and then co-hybridized on a single array. The
goal is then to identify actual binding targets of the protein of
interest, i.e. probes whose IP signal is significantly larger
than the Input signal. {\par}
%the two samples are DNA fragments crosslinked to a protein of
%interest (IP) and genomic DNA (Input). They are differentially
%labeled and co-hybridized on the same array. This technique allows
%the determination of DNA binding sites for any given protein.
Many authors already pointed out the need of efficient statistical
procedures to detect enriched probes \cite{BuckLieb04,Keles07}.
Recently, two strategies have been widely applied for the detection of
enriched DNA regions. The first strategy takes advantage of the
spatial structure of the data. Since probes are positioned all along
the genome, if one region is enriched we expect several adjacent
probes to obtain high ratio measurements, resulting in a ``peak'' of
intensity. Spatial methods such as sliding windows
\cite{Cawley04,Keles07} or Hidden Markov Models
\cite{JiWong05,LiMeyerLiu05} have been proposed to detect these peaks.
Alternatively, the second strategy is to consider that the whole
population of probes can be divided into two components: the
population of IP-enriched genomic fragments, and the population of
genomic DNA that is not IP enriched. Different statistical methods
have been proposed to distinguish between the two populations by
considering the distribution of the ratios (or their associated rank).
Assuming that a non negligible proportion of the fragments are
enriched, the ratio distribution is bimodal, the highest mode
corresponding to the enriched population. A probe is then declared
enriched when its ratio exceeds a selected cutoff, that is fixed
according to the data distribution \cite{BuckLieb04}. {\par} It is
worth to mention that in both strategies the statistical methods are
based on the assumption that the ratio measurement is a pertinent
statistical quantity of interest to assess the probe status (enriched
or not).  In practice, this is debatable.
%First from a biological
%point of view, it is well known that quantity of DNA recovered in the
%IP fraction depends on the antibody used and the type of protein
%analyzed and cannot thus be properly controlled. Second on the same
%array, where probes are not specific \cite{Turck07}, it is shown
%that it has an impact on the ratio value.
Figure
\ref{Figure:NuageIPInput} shows that the IP signal does not only depend on
the probe status but also on the Input signal.  This dependency is not
taken into account when the statistical model directly describes the
IP/Input ratio. For these reasons, ratios may be a partial indicator
of the probe status. {\par}
\begin{figure}
\begin{center}
\includegraphics[scale=0.3]{NuageIPInput.ps}
\end{center}
\caption{IP intensities vs Input Intensities
\label{Figure:NuageIPInput}}
\end{figure}

In this work, we propose a new method to analyse ChIP-chip data based
on mixture model of regressions.  This framework allows us to well
characterize the IP-Input relationship, and to provide a statistical
procedure to control the number of probes wrongly declared enriched.
The article is organized as follows. The statistical model and the
procedure for false positive control are described in Section
\ref{methodo} Application to real data are presented in Section
\ref{appli}.

\section{Statistical framework}\label{methodo}
\subsection{Model and inference}
Let $(X_i,Y_i)$ be the Input and IP intensities of probe $i$,
respectively, and let $Z_i$ be the (unknown) status of the probe,
with $Z_i=1$ if $i$ belongs to the enriched population, and $Z_i=0$
otherwise. We assume the Input-IP relationship to be linear whatever
the population, but with different slope and intercept. More
precisely, we have :
\begin{eqnarray}
Y_i = a_j + b_j X_i + \epsilon_i  \qquad \text{if } Z_i=j,
\qquad j=0,1,
\end{eqnarray}
where $\epsilon_i$ is a Gaussian random variable with mean 0 and
variance $\sigma^2_j$. The overall distribution of $Y_i$ for a
given level of Input $X_i$ is a mixture of two regressions is
\begin{eqnarray}
\pi_0 \phi(Y_i | X_i; \theta_0) + \pi_1 \phi(Y_i | X_i;
\theta_1),\label{Equation:Melange} %\leadsto
\end{eqnarray}
where $\pi_1 = 1-\pi_0$ is the proportion of enriched probes, and
$\theta_j$ stands for $(a_j, b_j, \sigma^2_j)$.
% To describe the relationship between IP and Input, we
%consider a mixture model of two linear regression. For a given probe
%$i$ with Input intensity $Input_i$, the IP intensity distribution is
%given by
%\begin{eqnarray}
%IP_i = \pi_0 \phi(Input_i, \theta_0) + \pi_1 \phi(Input_i, \theta_1)
%\ \ ,
%\end{eqnarray}
%where $\pi_1 = 1-\pi_0$ is the proportion of enriched probes, and
%$\phi(Input_i, \theta_j)$ is the gaussian distribution with mean
%$$a_j + b_j Input_i, \ \ j=0,1 $$ and variance $\sigma^2_j$.
%The parameter $\theta_j$ stands for $(a_j, b_j, \sigma^2_j)$. The
%first component (resp. the second) characterizes the distribution of
%the non-enriched (resp. enriched) probes. This model states that the
%relationship is linear for both components, but with different
%parameters.
According to the biological interpretation, for a given Input
intensity we expect the IP intensity to be higher when the probe is
enriched, and the prior probability $\pi_1$ to be lower than $\pi_0$
since most of the probes are supposed not to be targeted by the
protein. After estimation, these two characterizations will help to
designate the enriched population. {\par} From Equation
\ref{Equation:Melange}, we can define the posterior probability for a
given probe to be enriched by
\begin{eqnarray}
\tau_i = \Pr\{Z_i=1 \;|\; X_i,Y_i\} = \frac{\pi_0
  \phi(Y_i |X_i; \theta_0)}{\pi_0 \phi(Y_i|X_i; \theta_0) + \pi_1 \phi(Y_i|X_i;
  \theta_1)}. \label{Equation:Posterior}
\end{eqnarray}
The mixture parameters can be estimated using the classical EM
algorithm, and the posterior probability $\tau_i$ for a given probe
with Input and IP intensities $(X_i,Y_i)$ can then be estimated with
Formula (\ref{Equation:Posterior}) by replacing parameters with their
estimates. {\par}

\subsection{False discovery control}
Posterior probabilities can be used to decide which probes are
enriched, using a classification rule such as
$$
\hat{\tau}_{i} > s \qquad \Rightarrow \qquad
\hat{Z}_i =1,
$$
where $s$ is a threshold that has to be fixed by the experimenter.
In the context of mixture models, $s$ is usually fixed to 1/2 ({\it
  Maximum A Posteriori} rule) which implicitly involves that
misclassifications in population 0 or in population 1 have the same
cost. In ChIP-chip experiments where false positives are under
concern, it is better to fix $s$ in order to control the false
positive rate.

In the hypothesis test theory, the false discovery control is
performed by controlling the probability to reject wrongly the null
hypothesis. An analogous false discovery control in the framework of
the mixture models is to control the probability to assign a probe
in the enriched population wrongly. We want therefore $\Pr\{\tau_{i}
> s \;|\; X_i, Z_i =0\}$ to be equal to predefined level $\alpha$.
It can be rewritten as
\begin{equation}
\Pr\{ \pi_0 \phi(Y_i|X_i; \theta_0) (1-s) - s \pi_1 \phi(Y_i|X_i;
\theta_1) > 0 \;|\; X_i, Z_i =0\}  =  \alpha.
\label{Equation:FDR}
\end{equation}
%where $\alpha$ is a probability given by the user.
In practice, we fix $\alpha$ and find the threshold $s(\alpha, X_i)$
satisfying Equation (\ref{Equation:FDR}).  This leads to study the
sign of a second degree polynomial in $Y_i$.
% In practice, we fix $\alpha$ and find the threshold $s(\alpha, X_i)$
% satisfying Equation (\ref{Equation:FDR}).  Straightforward
% calculations show that $s(\alpha, X_i)$ satisfies
% \begin{equation}
% \Pr\left\{
% \left(\frac{1}{\sigma^2_0}-\frac{1}{\sigma^2_1}\right)Y_i^2 +
% 2\left(\frac{\mu_{i0}}{\sigma^2_0}-\frac{\mu_{i1}}{\sigma^2_1}\right) Y_i +
% \gamma[s(\alpha, X_i)] \geq 0 \;|\; X_i, Z_i =0 \right\} = \alpha,
% \end{equation}
% where for $j=0,1$, $\mu_{ij}=a_j+b_j X_i$ and $\gamma(s)
% =\left(\frac{\mu_{i0}^2}{\sigma^2_0}-\frac{\mu_{i1}^2}{\sigma^2_1}\right)
% -2\log\left(\frac{ \pi_0 \sigma_1 s}{\pi_1 \sigma_0 (1-s) }\right)$.

Note that the same strategy can be applied to control the probability
to assign a probe in the non-enriched population wrongly.

\section{Application}\label{appli}

We applied our method on Chip-chip data produced on a two-color
NimbleGen array.
% This array is a tiling array where probes were
% designed in a window of 110 nt such that the probe size is between 50
% and 75 nt and Tm is about 76$^°$.  The total number of probes is 1
% 132 140, each chromosome being covered by about 200 000 probes.
This tiling array consists of 1\;132\;140 probes, each chromosome
being covered by about 200\;000 probes.  The biological objective is
to detect genomic regions of the model plant \textit{Arabidopsis
  thaliana} marked by histone H3 tri-methylated at lysine 9.
Available data are two biological replicates, for which hybridizations
were performed in dye-swap. Since this experiment was already
performed on a genomic tilling array covering the chromosome 4
\cite{Turck07}, we focus on this chromosome.

Normalization of Chip-chip data is essential to remove technical biases
as well as dye bias. Since the Input and IP samples differ
substantially, array-by-array normalization such as lowess cannot be applied.
We just quantify biases by an anova model \cite{Kerr02}, and
remove them to the raw data. The IP and Input signals for each
biological replicate are averaged on the dye-swap.

Estimations of the parameter of our model on the data of the first
replicate are $\widehat{\pi}_1=0.32$, $\widehat{\theta}_0=(1.48,\;
0.81,\; 0.32)$ and $\widehat{\theta}_1=(-0.29,\; 1.13,\; 0.64)$, and
on the data of the second replicate are $\widehat{\pi}_1=0.28$,
$\widehat{\theta}_0=(2.32,\; 0.75,\; 0.34)$ and
$\widehat{\theta}_1=(0.61,\; 1.05,\; 0.52)$.
Figure \ref{Figure:Resultats} displays the results of our analysis for
the first replicate.
% paragraphe sur alpha

\begin{figure}
  \begin{center}
    \begin{tabular}{cc}
%       \includegraphics[scale=0.3, angle=90]{Graph_Regression_MoyDye_Rep1_chr4.ps}
%       &
%       \includegraphics[scale=0.3, angle=90]{Graph_Regression_MoyDye_Rep1_chr4.ps}
      \epsfig{file=Graph_Regression_MoyDye_Rep1_chr4_degrade.ps, angle=270,
      width=6cm, clip=}
      &
      \epsfig{file=Graph_Regression_MoyDye_Rep1_chr4.ps, angle=270,
      width=6cm, clip=}
    \end{tabular}
  \end{center}
\caption{Left panel: IP intensities vs Input Intensities colored
  according to the posterior probabilities $\hat{\tau}_i$. Colors change every
20 \% (blue: $\hat{\tau}_i<20\%$, red: $\hat{\tau}_i>80\%$). Right
  panel: Number enriched probes as a function of the risk $\alpha$.
\label{Figure:Resultats}}
\end{figure}

To assess the reproducibility of the analysis, we compared the lists
of enriched probes in the two biological replicates. A total of 37\;360
probes where detected in the first replicate and 31\;124 in the second.
The intersection contains 27\;560 probes, i.e.
more than two third of probes declared enriched in at least in one
replicate. We point out that although our method does not take spatial structure of the
genome into account, enriched probes are clustered in genomic regions
(data not shown). These regions are rich in genes and
corroborate the results of \cite{Turck07}, who
have shown that H3K9me3 is actually an euchromatin mark.

\bibliographystyle{plain}
\bibliography{Chipchip}

\end{document}
