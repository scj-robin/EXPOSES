\documentclass[dvips, lscape]{foils}
%\documentclass[dvips, french]{slides}
\textwidth 18.5cm \textheight 25cm \topmargin -1cm \oddsidemargin
-1cm \evensidemargin  -1cm

% Maths
\usepackage{amsfonts, amsmath, amssymb,graphics, url}

\newcommand{\coefbin}[2]{\left(
    \begin{array}{c} #1 \\ #2 \end{array}
  \right)}


% Couleur et graphiques
\usepackage{color}
\usepackage{graphics}
\usepackage{epsfig}
\usepackage{pstcol}
\newcommand{\Example}{tfl2_1_ResReg}

% Texte
\usepackage{lscape}
\usepackage{../../../../Latex/fancyheadings, rotating, enumerate}
%\usepackage[french]{babel}
\usepackage[latin1]{inputenc}
\definecolor{darkgreen}{cmyk}{0.5, 0, 0.5, 0.5}
\definecolor{orange}{cmyk}{0, 0.6, 0.8, 0}
\definecolor{jaune}{cmyk}{0, 0.5, 0.5, 0}
\newcommand{\textblue}[1]{\textcolor{blue}{#1}}
\newcommand{\textred}[1]{\textcolor{red}{#1}}
\newcommand{\textgreen}[1]{\textcolor{green}{ #1}}
\newcommand{\textlightgreen}[1]{\textcolor{green}{#1}}
%\newcommand{\textgreen}[1]{\textcolor{darkgreen}{#1}}
\newcommand{\textorange}[1]{\textcolor{orange}{#1}}
\newcommand{\textyellow}[1]{\textcolor{yellow}{#1}}
\newcommand{\refer}[2]{{\sl #1}}

% Sections
%\newcommand{\chapter}[1]{\centerline{\LARGE \textblue{#1}}}
% \newcommand{\section}[1]{\centerline{\Large \textblue{#1}}}
% \newcommand{\subsection}[1]{\noindent{\Large \textblue{#1}}}
% \newcommand{\subsubsection}[1]{\noindent{\large \textblue{#1}}}
% \newcommand{\paragraph}[1]{\noindent {\textblue{#1}}}
% Sectionsred
\newcommand{\chapter}[1]{
  \addtocounter{chapter}{1}
  \setcounter{section}{0}
  \setcounter{subsection}{0}
  {\centerline{\LARGE \textblue{\arabic{chapter} - #1}}}
  }
\newcommand{\section}[1]{
  \addtocounter{section}{1}
  \setcounter{subsection}{0}
  {\centerline{\Large \textblue{\arabic{chapter}.\arabic{section} - #1}}}
  }
\newcommand{\subsection}[1]{
  \addtocounter{subsection}{1}
  {\noindent{\large \textblue{#1}}}
  }
% \newcommand{\subsection}[1]{
%   \addtocounter{subsection}{1}
%   {\noindent{\large \textblue{\arabic{chapter}.\arabic{section}.\arabic{subsection} - #1}}}
%   }
\newcommand{\paragraph}[1]{\noindent{\textblue{#1}}}
\newcommand{\emphase}[1]{\textblue{#1}}

%%%%%%%%%%%%%%%%%%%%%%%%%%%%%%%%%%%%%%%%%%%%%%%%%%%%%%%%%%%%%%%%%%%%%%
%%%%%%%%%%%%%%%%%%%%%%%%%%%%%%%%%%%%%%%%%%%%%%%%%%%%%%%%%%%%%%%%%%%%%%
%%%%%%%%%%%%%%%%%%%%%%%%%%%%%%%%%%%%%%%%%%%%%%%%%%%%%%%%%%%%%%%%%%%%%%
%%%%%%%%%%%%%%%%%%%%%%%%%%%%%%%%%%%%%%%%%%%%%%%%%%%%%%%%%%%%%%%%%%%%%%
\begin{document}
%%%%%%%%%%%%%%%%%%%%%%%%%%%%%%%%%%%%%%%%%%%%%%%%%%%%%%%%%%%%%%%%%%%%%%
%%%%%%%%%%%%%%%%%%%%%%%%%%%%%%%%%%%%%%%%%%%%%%%%%%%%%%%%%%%%%%%%%%%%%%
%%%%%%%%%%%%%%%%%%%%%%%%%%%%%%%%%%%%%%%%%%%%%%%%%%%%%%%%%%%%%%%%%%%%%%
%%%%%%%%%%%%%%%%%%%%%%%%%%%%%%%%%%%%%%%%%%%%%%%%%%%%%%%%%%%%%%%%%%%%%%
\landscape
\newcounter{chapter}
\newcounter{section}
\newcounter{subsection}
\setcounter{chapter}{0} 
\headrulewidth 0pt \pagestyle{fancy}
\cfoot{} \rfoot{\begin{rotate}{90}{
      %\hspace{1cm} \tiny S. Robin: Segmentation-clustering for CGH
      }\end{rotate}}
\rhead{\begin{rotate}{90}{
      \hspace{-.5cm} \tiny \thepage
      }\end{rotate}}

%%%%%%%%%%%%%%%%%%%%%%%%%%%%%%%%%%%%%%%%%%%%%%%%%%%%%%%%%%%%%%%%%%%%%%
%%%%%%%%%%%%%%%%%%%%%%%%%%%%%%%%%%%%%%%%%%%%%%%%%%%%%%%%%%%%%%%%%%%%%%
\begin{center}
  \textblue{\LARGE Mixture models for ChIP-chip Experiments} 

   \vspace{1cm}
   {\large S. {Robin}} \\
   robin@agroparistech.fr

   \vspace{1.5cm}
   {UMR 518 AgroParisTech / INRA MIA} \\
   {Statistics for Systems Biology group}
   
   \vspace{1.5cm}
   \paragraph{Joint work with} C. B�rard, M.-L. Martin-Magniette,
   T. Mary-Huard 

    \vspace{1.5cm}
    {Ecole INSERM, ENS, September 2008} \\

    \vspace{1.5cm}
    \begin{tabular}{ccccc}
      \epsfig{file=../Figures/LogoINRA-Couleur.ps, width=5cm} &
      \hspace{2cm} &
      \epsfig{file=../Figures/logagroptechsolo.eps, width=7.5cm} &
      \hspace{2cm} &
      \epsfig{file=../Figures/Logo-SSB.eps, width=5cm} \\
    \end{tabular} \\
\end{center}


%%%%%%%%%%%%%%%%%%%%%%%%%%%%%%%%%%%%%%%%%%%%%%%%%%%%%%%%%%%%
%%%%%%%%%%%%%%%%%%%%%%%%%%%%%%%%%%%%%%%%%%%%%%%%%%%%%%%%%%%%
\newpage
\chapter{Detecting Hybridization in ChIP-Chip data}
%%%%%%%%%%%%%%%%%%%%%%%%%%%%%%%%%%%%%%%%%%%%%%%%%%%%%%%%%%%%
%%%%%%%%%%%%%%%%%%%%%%%%%%%%%%%%%%%%%%%%%%%%%%%%%%%%%%%%%%%%

%%%%%%%%%%%%%%%%%%%%%%%%%%%%%%%%%%%%%%%%%%%%%%%%%%%%%%%%%%%%
\bigskip
\section{Biological context}
%%%%%%%%%%%%%%%%%%%%%%%%%%%%%%%%%%%%%%%%%%%%%%%%%%%%%%%%%%%%
\bigskip
\begin{itemize}
\item Gene expression study $\rightarrow$ DNA microarray
\item \emphase{Control mecanisms of gene expression} \\ \\
  \paragraph{Examples :} 
  \begin{itemize}
  \item Transcription factors and co-factors
  \item DNA methylation
  \item Histone modifications $\rightarrow$ alter chromatin structure
    $\rightarrow$ alter gene expression 
  \end{itemize}
\end{itemize}

\vspace{2cm}
\begin{center}
$\Longrightarrow$ Interest to study \emphase{interactions} between \emphase{proteins} and \emphase{DNA} \\ 
\end{center}

\begin{center}
$\Longrightarrow$ Technique of \emphase{ChIP-chip}
\end{center}

 

%%%%%%%%%%%%%%%%%%%%%%%%%%%%%%%%%%%%%%%%%%%%%%%%%%%%%%%%%%%%
\newpage
\section{ChIP-Chip technology}
%%%%%%%%%%%%%%%%%%%%%%%%%%%%%%%%%%%%%%%%%%%%%%%%%%%%%%%%%%%%

\noindent
\begin{tabular}{cc}
  \begin{tabular}{c}
    \epsfig{file = ../Figures/Chip-chip.ps, width=14cm,height=14cm, clip=}
 %\includegraphics{HistogrammeChr4.png}%, width=12cm,height=12cm, clip=}
  \end{tabular}
  \begin{tabular}{p{10cm}}
    \emphase{IP sample} : immuno-precipitated DNA\\
    \emphase{INPUT sample} : total DNA\\\\
    Both samples are co-hybridized on a same array\\\\
    \emphase{Purpose} : to determine which probes have a significantly
    higher IP signal than the INPUT signal.
  \end{tabular}
\end{tabular}

%%%%%%%%%%%%%%%%%%%%%%%%%%%%%%%%%%%%%%%%%%%%%%%%%%%%%%%%%%%%
\newpage
\section{Unsupervised Classification Problem}
%%%%%%%%%%%%%%%%%%%%%%%%%%%%%%%%%%%%%%%%%%%%%%%%%%%%%%%%%%%%

\paragraph{Data:} For each probe, we hence get the IP signal and the INPUT
signal\\\\
\bigskip
\paragraph{Question:} According to these two signals, we have to
\begin{itemize}
%\item \vspace{-0.5cm} Account for the \emphase{link between the IP and
%    Input} signals
\item \vspace{-0.5cm} \emphase{Classify each probe} into the
  \emphase{enriched} group or into the \emphase{normal} group,
\end{itemize}

\bigskip
\noindent In addition we would like to
\begin{itemize}
\item \vspace{-0.5cm} Avoid numerous false detections (false enriched probes).
\end{itemize}

\bigskip
\paragraph{Bibliography:}
\begin{itemize}
\item Model based on the spatial structure of the data (sliding
  window [S.Keles, \textit{Biometrics} 2007], HMM [W.Li,
  C.A.Meyer, X.S.Liu, \textit{Bioinformatics} 2005]), 
\item \vspace{-0.5cm} Model considering that the whole population
  can be divided into two groups.
\end{itemize}

%%%%%%%%%%%%%%%%%%%%%%%%%%%%%%%%%%%%%%%%%%%%%%%%%%%%%%%%%%%%
%%%%%%%%%%%%%%%%%%%%%%%%%%%%%%%%%%%%%%%%%%%%%%%%%%%%%%%%%%%%
\newpage
\chapter{Mixture Model}
%%%%%%%%%%%%%%%%%%%%%%%%%%%%%%%%%%%%%%%%%%%%%%%%%%%%%%%%%%%
%%%%%%%%%%%%%%%%%%%%%%%%%%%%%%%%%%%%%%%%%%%%%%%%%%%%%%%%%%%%

%%%%%%%%%%%%%%%%%%%%%%%%%%%%%%%%%%%%%%%%%%%%%%%%%%%%%%%%%%%%
\bigskip
\section{Distribution of the log-ratios}
%%%%%%%%%%%%%%%%%%%%%%%%%%%%%%%%%%%%%%%%%%%%%%%%%%%%%%%%%%%%
$$
\includegraphics[scale=0.8]{../Figures/BelleDistribution.ps}
$$
This plot suggests the existence of 2 populations of probes: 
\emphase{normal / enriched. } \\
$\emphase{\Rightarrow}$ We have to characterize the
disctribution of the log-ratio (= log(IP / Input)) in each of them.

%%%%%%%%%%%%%%%%%%%%%%%%%%%%%%%%%%%%%%%%%%%%%%%%%%%%%%%%%%%%
\newpage
\section{Mixture Model}
%%%%%%%%%%%%%%%%%%%%%%%%%%%%%%%%%%%%%%%%%%%%%%%%%%%%%%%%%%%%

\hspace{-2.5cm}
\begin{tabular}{cc}
  \begin{tabular}{p{10cm}}
    \paragraph{How do we see the data?} 
    \begin{itemize}
    \item Probes ($i=1..n$) are spread into \emphase{2 populations} 0
      = normal / 1 = enriched
    \item \vspace{-0.5cm} With \emphase{respective proportions}
      $(1-\pi)$ and $\pi$.
    \item \vspace{-0.5cm} The distribution of the log-ratio $R_i$
      \emphase{depends on the population}, e.g.
      $$
      \begin{array}{rrcl}
        i \text{ normal:} & R_i & \sim & \mathcal{N}(\mu_0,
        \sigma^2) \\
%        \\
        i \text{ enriched:} & R_i & \sim & \mathcal{N}(\mu_1, \sigma^2),
      \end{array}
      $$
    \end{itemize}
  \end{tabular}
  &
  \begin{tabular}{c}
    \epsfig{file = ../Figures/EM_ChipChip.eps, clip=, width=12cm}
  \end{tabular}
\end{tabular}

\paragraph{Statistical inference.} We have to
\begin{itemize}
\item \vspace{-0.5cm} Estimate all the parameters: $\pi, \mu_0, \mu_1,
  \sigma^2$;
\item \vspace{-0.5cm} Classify all the probes or at least give a
  \emphase{probability for each probe to be enriched}.
\end{itemize}

%%%%%%%%%%%%%%%%%%%%%%%%%%%%%%%%%%%%%%%%%%%%%%%%%%%%%%%%%%%%
%%%%%%%%%%%%%%%%%%%%%%%%%%%%%%%%%%%%%%%%%%%%%%%%%%%%%%%%%%%%
\newpage
\section{Estimation with the EM algorithm} 
%%%%%%%%%%%%%%%%%%%%%%%%%%%%%%%%%%%%%%%%%%%%%%%%%%%%%%%%%%%%

\noindent The EM algorithm is dedicated to \emphase{incomplete data
  problems}. Here, the status $Z_i = 0/1$ of each probe is unknown.

\paragraph{E step:} Given the distribution of the log-ratio in each
population, we can estimate the status $Z_i$, i.e. calculate the
probability for probe $i$ to be enriched:
$$
\widehat{Z}_i = \Pr\{Z_i = 1 \;|\; R_i\}
= 
\frac{\pi \; \phi(R_i; \mu_1, \sigma^2)}{(1-\pi) \; \phi(R_i; \mu_0,
  \sigma^2) + \pi \; \phi(R_i; \mu_1, \sigma^2)}
$$ \\
$\widehat{Z}_i$ is called the \emphase{posterior probability} (or
conditional probability).

\bigskip\bigskip\bigskip
\paragraph{M step:} Given the estimated status, the parameters can be
estimated as weighted sums or means:
$$
\widehat{\pi}_1 = \frac1n \sum_i \widehat{Z}_i, 
\qquad
\widehat{\mu}_1 = \frac1{\sum_i \widehat{Z}_i} \sum_i \widehat{Z}_i R_i.
$$

%%%%%%%%%%%%%%%%%%%%%%%%%%%%%%%%%%%%%%%%%%%%%%%%%%%%%%%%%%%%
%%%%%%%%%%%%%%%%%%%%%%%%%%%%%%%%%%%%%%%%%%%%%%%%%%%%%%%%%%%%
\newpage
\paragraph{Posterior probabilities:} 
%%%%%%%%%%%%%%%%%%%%%%%%%%%%%%%%%%%%%%%%%%%%%%%%%%%%%%%%%%%%
$$
  \begin{tabular}{cc}
    Distributions: & Posterior probabilities: \\
    \\
    $f(x) = \textred{\pi_1 \phi_1(x)} + \textgreen{\pi_2 \phi_2(x)} +
    \textblue{\pi_3 \phi_3(x)}$ & $\tau_{gk} = \Pr\{g \in \phi_k \;|\; x_g\}
    = \pi_k \phi_k(x_g) / f(x_g)$\\
    \\
    \epsfig{file=../Figures/Melange-densite.ps, height=6cm,
      width=12cm, bbllx=77, bblly=328, bburx=549, bbury=528, clip=}
    &
    \epsfig{file=../Figures/Melange-posteriori.ps, height=6cm, width=12cm,
      bbllx=83, bblly=320, bburx=549, bbury=537, clip=} 
  \end{tabular}
$$
$$
\begin{array}{cccc}
  \quad p_g(k)~~(\%) \quad & \qquad g=1 \qquad & \qquad g=2 \qquad &
  \qquad g=3 \qquad \\
  \hline
  k = 1 & \textred{65.8} & 0.7 & 0.0 \\
  k = 2 & 34.2 & 47.8 & 0.0 \\
  k = 3 & 0.0 & \textblue{51.5} & \textblue{1.0} \\
\end{array}
$$

\centerline{\emphase{See demo.}}

%%%%%%%%%%%%%%%%%%%%%%%%%%%%%%%%%%%%%%%%%%%%%%%%%%%%%%%%%%%%
%%%%%%%%%%%%%%%%%%%%%%%%%%%%%%%%%%%%%%%%%%%%%%%%%%%%%%%%%%%%
\newpage
\chapter{Mixture Model of Regressions}
%%%%%%%%%%%%%%%%%%%%%%%%%%%%%%%%%%%%%%%%%%%%%%%%%%%%%%%%%%%%
%%%%%%%%%%%%%%%%%%%%%%%%%%%%%%%%%%%%%%%%%%%%%%%%%%%%%%%%%%%%

%%%%%%%%%%%%%%%%%%%%%%%%%%%%%%%%%%%%%%%%%%%%%%%%%%%%%%%%%%%%
\bigskip
\section{About log-ratios...}
%%%%%%%%%%%%%%%%%%%%%%%%%%%%%%%%%%%%%%%%%%%%%%%%%%%%%%%%%%%%

\paragraph{Correlation IP / Input.} 
All methods already published are based on the log-ratio IP/Input \\
$\Rightarrow$ Assume that the log-ratio distribution is informative
about the status. \\
\begin{tabular}{cc}
  \begin{tabular}{c}
\hspace{-1.5cm}\includegraphics[scale=0.6]{../Figures/BelleDistribution.ps}
  \end{tabular}
  &
  \begin{tabular}{c}
   \epsfig{file =
   ../Figures/Graph_Histogramme_LogRatio_MoyDye_Rep2_chr4.ps,width=10cm,
   height=12cm, angle=-90,clip=} 
  \end{tabular} \\ Good case (Buck \& Lieb, 2004) & Bad case \\ & \\
\end{tabular}\\


%%%%%%%%%%%%%%%%%%%%%%%%%%%%%%%%%%%%%%%%%%%%%%%%%%%%%%%%%%%%
\newpage
$$
\begin{tabular}{cc}
  \begin{tabular}{p{10cm}}
    \paragraph{Mixture of regression.} \\ 
    \\
    A closer look shows that the relation between IP and Input
    may \emphase{differ between the normal and enriched} groups. \\
    \\
    \emphase{Working on log-ratios} is equivalent to assume that 
    \begin{itemize}
    \item these relations are the same
    \item and that the slope is 1 in both.
    \end{itemize}
  \end{tabular}
  &
  \begin{tabular}{c}
    \epsfig{file = ../Figures/Nuage.ps,width=12cm,height=12cm,clip=}
  \end{tabular}
\end{tabular}
$$


%%%%%%%%%%%%%%%%%%%%%%%%%%%%%%%%%%%%%%%%%%%%%%%%%%%%%%%%%%%%
\newpage
\section{Mixture model}
%%%%%%%%%%%%%%%%%%%%%%%%%%%%%%%%%%%%%%%%%%%%%%%%%%%%%%%%%%%%

\bigskip
\paragraph{Model.} We assume that each probe $i$ has a probability $\pi$
to be enriched:
$$
\Pr\{\text{Probe $i$ enriched}\} = \pi, \qquad \Pr\{\text{Probe $i$
normal}\} = 1 - \pi,
$$
and that the relation between log-IP ($Y_i$) and log-Input ($X_i$)
depends on the status of the probe:
$$
Y_i = \left\{ \begin{array}{ll}
    a_0 + b_0 X_i + E_i & \quad \text{if $i$ is normal} \\
    \\
    a_1 + b_1 X_i + E_i & \quad \text{if $i$ is enriched}
  \end{array} \right. \ \ \ V(Y_i)=\sigma^2
$$

\bigskip\bigskip
\paragraph{Comparison with the 'Log-Ratio' analysis.} The standard
analysis based on the log(IP/Input) is the special case of the
regression mixture model where
$$
b_0 = b_1 = 1.
$$

%%%%%%%%%%%%%%%%%%%%%%%%%%%%%%%%%%%%%%%%%%%%%%%%%%%%%%%%%%%%
\newpage
\section{Parameter Estimation and Probe Classification}
%%%%%%%%%%%%%%%%%%%%%%%%%%%%%%%%%%%%%%%%%%%%%%%%%%%%%%%%%%%%

\bigskip
\paragraph{Task.} We have to estimate
\begin{itemize}
\item \vspace{-0.5cm} the proportion of enriched probes $\pi$
\item \vspace{-0.5cm} the regression parameters: intercepts ($a_0$,
  $a_1$), slopes ($b_0$, $b_1$) and variance ($\sigma^2$).
\end{itemize}


\bigskip\bigskip
\paragraph{Algorithm.} This can be done using the E-M algorithm which
alternates
\begin{description}
\item[E-step:] \vspace{-0.5cm} prediction of the probe status given
  the parameters;
\item[M-step:] \vspace{-0.5cm} estimation of the parameters given the
  (predicted) probe status.
\end{description}

\bigskip\bigskip
\paragraph{Model selection.} BIC criterion to select between a 1 or
2 population model.

\bigskip\bigskip
\paragraph{Posterior probability.} The status prediction is based on
the posterior probability $\tau_i$
$$
\textred{\tau_i = \Pr\{\text{$i$ enriched} \; |\; X_i, Y_i\}},
\qquad 1 - \tau_i = \Pr\{\text{$i$ normal} \; |\; X_i, Y_i\}.
$$
%This probability provides a \emphase{probe classification rule}.

%%%%%%%%%%%%%%%%%%%%%%%%%%%%%%%%%%%%%%%%%%%%%%%%%%%%%%%%%%%%
\newpage
\paragraph{Probe classification.}
$$
    %\epsfig{file=../../Figures/PosteriorPlot_MoyDye_Rep2_chr4.eps}
\includegraphics[scale=1]{../Figures/Graph_regression_Rep1_chr4bis.ps}
%,
      %      width=12cm, height=12cm, angle=0, clip=}
$$

How to classify a probe according to its posterior probability ?

%%%%%%%%%%%%%%%%%%%%%%%%%%%%%%%%%%%%%%%%%%%%%%%%%%%%%%%%%%%%
%%%%%%%%%%%%%%%%%%%%%%%%%%%%%%%%%%%%%%%%%%%%%%%%%%%%%%%%%%%%
\newpage
\chapter{Limiting False Detections}
%%%%%%%%%%%%%%%%%%%%%%%%%%%%%%%%%%%%%%%%%%%%%%%%%%%%%%%%%%%%
%%%%%%%%%%%%%%%%%%%%%%%%%%%%%%%%%%%%%%%%%%%%%%%%%%%%%%%%%%%%

\bigskip
\paragraph{Maximum A Posteriori (MAP) rule.}
$$
\begin{array}{rclcl}
  \widehat{\tau_i} & \geq & 50\% & \Rightarrow & \text{$i$ classified as
  'enriched'}, \\
  \widehat{\tau_i} & < & 50\% & \Rightarrow & \text{$i$ classified as
  'normal'}.
\end{array}
$$
Misclassifications in 'enriched' group or in 'normal' group have the
same cost.

\paragraph{Controlling false detections.} We want to control the
probability for the $\tau_i$ of a normal probe to fall above the
classification threshold.

For a \emphase{fixed risk $\alpha$} we calculate the threshold $s$
such that
$$
\textred{s: \qquad \Pr\{\tau_i > s \;|\; \text{$i$ normal}, \;
  \log(\text{Input})=X_i\} = \alpha}
$$
and if $\widehat{\tau_i} > s$ then the probe $i$ is classified as 'enriched'.\\
The threshold $s$ depends on both $\alpha$ and the log-Input $X_i$.\\
This control focuses on misclassifications in 'enriched' group.

%%%%%%%%%%%%%%%%%%%%%%%%%%%%%%%%%%%%%%%%%%%%%%%%%%%%%%%%%%%%
%%%%%%%%%%%%%%%%%%%%%%%%%%%%%%%%%%%%%%%%%%%%%%%%%%%%%%%%%%%%
\newpage

\newpage
\chapter{Applications}
\bigskip
\section{Promoter DNA methylation in the human genome}

\paragraph{Data} (Weber \emph{et al.} 2007)

\noindent NimbleGen tiling array of promoter regions of 15\;609
human genes.

\noindent Each promoter region has 15 probes.

\noindent We  consider the Intermediate CpG class of 2056 promoters.


\paragraph{Results}

\noindent A high proportion of enriched probes
($\widehat{\pi}\approx$80\%).

\noindent 403 of the 460 promoter regions found by Weber have 5
enriched probes or more.

\noindent 38 new promoter candidates with 9 enriched probes or more.

\noindent 1 promoter region found by Weber has no enriched probe.

\newpage


\section{NimbleGen tiling array of \textit{Arabidopsis thaliana}}

\paragraph{Data.} Study of the histone modification H3K9me3. 

\noindent Very high density: more than 1,000,000 probes ($\approx$200,000 per chromosome), 
2 biological replicates with dye-swap, normalized according to Kerr et al., 2002. \\

\paragraph{Results.}\\
At level $\alpha=0.01$, $\sim 20\%$ of the probes are enriched on each chromosome.
Estimated parameters for chromosome 4 : $\widehat{\pi}=0.28$, $\widehat{b}_0=0.75 ,\widehat{b}_1=1.05$

\noindent Study performed on two biological replicates : overlap of 2/3 

\noindent Enriched probes are clustered in genomic regions.

%\paragraph{Chloroplastic genome.}\\
%No target is expected $\Rightarrow$ negative control.

%\noindent \textbf{ChIPmix:} BIC selects a single regression model,
%i.e. no probe is declared enriched.

%\noindent \hspace{-0.7cm} \textbf{ ChIPOTle:} finds 10 to 16 peaks
%(depending on parameters).

%\noindent \textbf{Nimblegen:} not available.


\newpage

\paragraph{Comparison with other methods}
% A dire � l'oral :
% \noindent 3 different analysis methods: NimbleGen's software, ChIPOTle , and Mixture of regression (ChIPmix).
%(uses a permutation-based algorithm, takes spatial structure into account)
%(takes spatial structure into account too)

\begin{tabular}{cc}
\begin{tabular}{r}
    \\ \\
    Annotation \\ \\
    NimbleGen  \\ \\
    ChIPmix    \\ \\
    ChIPOTle
\end{tabular}
&
\begin{tabular}{c}
\includegraphics[scale=0.5]{../Figures/8516_annotation.ps}
%\epsfig{file=../../Figures/8516_chr4_sortie_ecran_annotation.eps,scale=0.5}
\\
\end{tabular}
\end{tabular} \\
\textbf{ChIPmix: } $\alpha=0.01$. Probes are clustered in genomic
regions, which cover especially the genes.

\noindent \textbf{NimbleGen:} finds smaller genomic regions, and
fail to identify some targets.

\noindent \textbf{ChIPOTle\footnote[1]{ChIPOTle : M.J.Buck, A.B.Nobel, J.D.Lieb, \textit{Genome Biol.}, 2005}: } agrees with ChIPmix, not with
NimbleGen.


\newpage
\paragraph{Biological validation} (Turck \textit{et al.}, 2007)

\begin{tabular}{c}
\includegraphics[scale=0.6]{../Figures/DiagrammeVenn_couleur_legend.ps}
%\epsfig{file=../../Figures/8516_chr4_sortie_ecran_annotation.eps,scale=0.5}
\end{tabular} 

\begin{small}
\noindent ChIPmix detects the greatest number of probes and more than 80\% of the probes declared enriched cover genomic regions already found in Turck \textit{et al.} (2007).

\noindent \textbf{Probes detected only by ChIPmix:} $>$ 55\% are validated
targets.

\noindent \textbf{Probes detected only by NimbleGen:} $<$ 20\% are
validated targets.

\noindent \textbf{Probes detected only by ChIPOTle:} only 716 and 50\% are validated targets.
\end{small}


\newpage
\chapter{Conclusions}

\noindent New method for analyzing ChIP-chip data based on mixture
model of regressions. The ChIPmix method is implemented in R and available on : \textit{www.agroparistech.fr/mmip/maths/outil.html}

\paragraph{May be applied to:}

$\bullet$ different biological questions (histone modification, DNA
methylation...)

$\bullet$ different organisms (Arabidopsis, Human)

$\bullet$ different 2-color technologies (tiling/promoter arrays,
high density arrays)

$\bullet$ different situations, whatever the proportion $\pi$

\paragraph{Extensions}

Other explicative variables in the regressions can be added (Hits,
CpG rate)



%%%%%%%%%%%%%%%%%%%%%%%%%%%%%%%%%%%%%%%%%%%%%%%%%%%%%%%%%%%%
%%%%%%%%%%%%%%%%%%%%%%%%%%%%%%%%%%%%%%%%%%%%%%%%%%%%%%%%%%%%

%%%%%%%%%%%%%%%%%%%%%%%%%%%%%%%%%%%%%%%%%%%%%%%%%%%%%%%%%%%%%%%%%%%%%%
%%%%%%%%%%%%%%%%%%%%%%%%%%%%%%%%%%%%%%%%%%%%%%%%%%%%%%%%%%%%%%%%%%%%%%
%%%%%%%%%%%%%%%%%%%%%%%%%%%%%%%%%%%%%%%%%%%%%%%%%%%%%%%%%%%%%%%%%%%%%%
%%%%%%%%%%%%%%%%%%%%%%%%%%%%%%%%%%%%%%%%%%%%%%%%%%%%%%%%%%%%%%%%%%%%%%
\end{document}
%%%%%%%%%%%%%%%%%%%%%%%%%%%%%%%%%%%%%%%%%%%%%%%%%%%%%%%%%%%%%%%%%%%%%%
%%%%%%%%%%%%%%%%%%%%%%%%%%%%%%%%%%%%%%%%%%%%%%%%%%%%%%%%%%%%%%%%%%%%%%
%%%%%%%%%%%%%%%%%%%%%%%%%%%%%%%%%%%%%%%%%%%%%%%%%%%%%%%%%%%%%%%%%%%%%%
%%%%%%%%%%%%%%%%%%%%%%%%%%%%%%%%%%%%%%%%%%%%%%%%%%%%%%%%%%%%%%%%%%%%%%
