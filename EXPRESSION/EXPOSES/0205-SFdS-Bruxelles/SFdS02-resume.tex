%
%       INSTRUCTIONS :
%
%       1) DANS LE TEXTE CI-DESSOUS COMPRIS ENTRE LES COMMANDES
%       \begin{document} ET \end{document}, REMPLACER PAR LES RENSEIGNEMENTS
%       PERTINENTS CHAQUE BRIBE D'INFORMATION.
%
%       2) SAUVEGARDEZ LE FICHIER SOUS LE NOM :
%
%       res***.tex    en rempla�ant *** par votre nom
%
%      3) EXPEDIEZ LE FICHIER  :
%
%       PAR COURRIER ELECTRONIQUE A:  jsbl2002@stat.ucl.ac.be
%       (IL N'EST PAS NECESSAIRE DE SUPPRIMER L'EN-TETE.)
%       et
%       SOUS FORME PAPIER EN 3 EXEMPLAIRES A  :
%       JSBL200
%       Institut de statistique
%       Universit� catholique de Louvain
%       20 voie du Roman Pays
%       B-1348 Louvain-la-Neuve
%       Belgique
%
%
%       VOUS ETES INVITES A VERIFIER LA QUALITE DU PRODUIT FINI,
%       POUR CELA, COMPILEZ D'ABORD LE FICHIER ,
%       PUIS, VISUALISEZ-LE.
%
\documentclass[12pt]{article}

\setlength{\topmargin}{0cm}
\setlength{\headheight}{0.5cm}
\setlength{\headsep}{0.5cm}
\setlength{\textheight}{22cm}
\setlength{\footskip}{2cm}

\pagestyle{empty}
\begin{document}
\begin{center}
{\Large {\sc

Analyse de donn\'ees issues de puces \`a ADN

}}
\bigskip 

Robin St\'ephane

\medskip
{\it

 INA-PG / INRA Biom\'etrie\\
 16, rue Claude Bernard \\
 75005 Paris

}
\end{center}
\bigskip


Parmi les avanc\'ees technologiques r\'ecentes en biologie
mol\'eculaire, les ``biopuces'' ou ``puces \`a ADN'' constituent sans
doute l'une des plus prometteuses.  Ces puces permettent notamment de
mesurer le niveau d'expression de la quasi totalit\'e des g\`enes d'un
organisme.  Cette technologie est aujourd'hui un des piliers de la
{\sl g\'enomique fonctionnelle} qui vise \`a d\'eterminer la fonction
des g\`enes.

\bigskip 

Cependant, l'abondance de ces mesures, leur sensibilit\'e aux
conditions exp\'erimentales, leur variabilit\'e intrins\`eque en
rendent l'interpr\'etation tr\`es d\'elicate et le recours \`a des
outils statistiques s'ompose aujourd'hui \`a tous les biologistes.

Une exp\'erience typique d'analyse de transcriptome consiste \`a
mesurer le niveau d'expression des g\`enes d'un organisme dans un
ensemble de conditions, ou \`a des instants successifs, ou dans un
ensemble de tissus. L'analyse mobilise un tr\`es grand nombres de
m\'ethodes statistiques
pour \\
$\bullet$ la normalisation des donn\'ees,  \\
$\bullet$ la classification des g\`enes ou des tissus, \\
$\bullet$ la d\'etection des g\`enes diff\'erentiellement exprim\'es
entre deux conditions, \\
$\bullet$ les comparaisons multiples des expressions des g�nes, \\
$\bullet$ la discrimination de diff\'erents types de tissus,  \\
$\bullet$ la reconnaissance de r\'eseau de r\'egulations. \\

Apr\`es avoir pr\'esent\'e le principe de la technologie des biopuces,
on pr\'e\-sentera diff\'erents exemples d'analyses en s'appuyant
notamment sur la bibliographie


\bigskip


\begin{center}
{\large{\bf Bibliographie}}
\end{center}
{\tt http://www.biostat.wisc.edu/geda/literature.html}

\end{document}

