\documentclass[dvips]{slides}
\textwidth 20cm
\textheight 25.7cm 
\topmargin -1.7cm 
\oddsidemargin  -2cm 
\evensidemargin  -2cm

% Maths
\usepackage{amsfonts, amsmath, amssymb}
\newcommand{\Acal}{\mathcal{A}}
\newcommand{\Ccal}{\mathcal{C}}
\newcommand{\Dcal}{\mathcal{D}}
\newcommand{\Ecal}{\mathcal{E}}
\newcommand{\Ncal}{\mathcal{N}}
\newcommand{\Pcal}{\mathcal{P}}
\newcommand{\Ucal}{\mathcal{U}}
\newcommand{\Hbf}{{\bf H}}
\newcommand{\Bcal}{\mathcal{B}}
\newcommand{\Lcal}{\mathcal{L}}
\newcommand{\Tcal}{\mathcal{T}}
\newcommand{\alphabf}{\mbox{\mathversion{bold}{$\alpha$}}}
\newcommand{\betabf}{\mbox{\mathversion{bold}{$\beta$}}}
\newcommand{\gammabf}{\mbox{\mathversion{bold}{$\gamma$}}}
\newcommand{\psibf}{\mbox{\mathversion{bold}{$\psi$}}}
\newcommand{\taubf}{\mbox{\mathversion{bold}{$\tau$}}}
\newcommand{\Rbb}{\mathbb{R}}
\newcommand{\Sbf}{{\bf S}}
\newcommand{\bps}{\mbox{bps}}
\newcommand{\ubf}{{\bf u}}
\newcommand{\vbf}{{\bf v}}
\newcommand{\Esp}{{\mathbb E}}
\newcommand{\Var}{{\mathbb V}}
\newcommand{\Indic}{{\mathbb I}}
\newcommand{\liste}{$\bullet \quad$}

% Couleur et graphiques
\usepackage{color}
\usepackage{graphics}
\usepackage{epsfig} 
\usepackage{pstcol}

% Texte
\usepackage{lscape, ../../../fancyheadings, rotating, enumerate}
%\usepackage[french]{babel}
%\usepackage[latin1]{inputenc}
\definecolor{darkgreen}{cmyk}{0.5, 0, 0.5, 0.5}
\definecolor{orange}{cmyk}{0, 0.6, 0.8, 0}
\definecolor{jaune}{cmyk}{0, 0.5, 0.5, 0}
\newcommand{\textblue}[1]{\textcolor{blue}{#1}}
\newcommand{\textred}[1]{\textcolor{red}{#1}}
%\newcommand{\textgreen}[1]{\textcolor{green}{\bf #1}}
\newcommand{\textlightgreen}[1]{\textcolor{green}{#1}}
\newcommand{\textgreen}[1]{\textcolor{darkgreen}{#1}}
\newcommand{\textorange}[1]{\textcolor{orange}{#1}}
\newcommand{\textyellow}[1]{\textcolor{yellow}{#1}}

% Sections
\newcommand{\chapter}[1]{\centerline{\LARGE \textgreen{#1}}}
\newcommand{\section}[1]{\centerline{\Large \textblue{#1}}}
\newcommand{\subsection}[1]{\noindent{\large \textblue{#1}}}
\newcommand{\paragraph}[1]{{\textgreen{#1}}}

%%%%%%%%%%%%%%%%%%%%%%%%%%%%%%%%%%%%%%%%%%%%%%%%%%%%%%%%%%%%%%%%%%%%%%
%%%%%%%%%%%%%%%%%%%%%%%%%%%%%%%%%%%%%%%%%%%%%%%%%%%%%%%%%%%%%%%%%%%%%%
\begin{document}
\landscape
\headrulewidth 0pt 
\pagestyle{fancy} 
\cfoot{}
\rfoot{\begin{rotate}{90}{
      \vspace{.5cm}
      \hspace{-0.5cm} \tiny S. Robin (Tehran 2005)
      }\end{rotate}}
\rhead{\begin{rotate}{90}{
      \hspace{-.5cm} \tiny \thepage
      }\end{rotate}}

%%%%%%%%%%%%%%%%%%%%%%%%%%%%%%%%%%%%%%%%%%%%%%%%%%%%%%%%%%%%%%%%%%%%%%
%%%%%%%%%%%%%%%%%%%%%%%%%%%%%%%%%%%%%%%%%%%%%%%%%%%%%%%%%%%%%%%%%%%%%%

% 2 Differential analysis of microarray data, multiple testing problems
% and False Discovery Rate (FDR)

% Microarray (DNA chips) data is a new technology that measures the
% expression level of thousands of genes simultaneously. Many microarray
% experiment aim at detecting differentially expressed genes between two
% conditions. Because of the large number of genes, multiple testing
% problems immediately rise in the statistical analysis of these
% experiments. One of the crucial points is the control of the number of
% false positive genes. We present the general setting of differential
% analysis in microarray experiments and the multiple testing problem.
% We then introduce the recent developments about the false discovery
% rate and propose a new method, based on mixture models, to estimate
% it.

\begin{center}
  \chapter{Differential analysis of microarray data,} 
  \bigskip
  \chapter{Multiple testing problems} 
  \bigskip
  \chapter{and Local False Discovery Rate.}
  \bigskip

   {\large S. {\sc Robin}, J.-J. {\sc Daudin}, A. {\sc Bar-Hen},
   L. {\sc Pierre}} \\
   robin@inapg.inra.fr

   {UMR INA-PG / INRA, Paris} \\
   {Math�matique et Informatique Appliqu�es}
   
   {Bio-Info-Math Workshop, Tehran, April 2005}
\end{center}

%%%%%%%%%%%%%%%%%%%%%%%%%%%%%%%%%%%%%%%%%%%%%%%%%%%%%%%%%%%%%%%%%%%%%%
%%%%%%%%%%%%%%%%%%%%%%%%%%%%%%%%%%%%%%%%%%%%%%%%%%%%%%%%%%%%%%%%%%%%%%
%%%%%%%%%%%%%%%%%%%%%%%%%%%%%%%%%%%%%%%%%%%%%%%%%%%%%%%%%%%%%%%%%%%%%%
%%%%%%%%%%%%%%%%%%%%%%%%%%%%%%%%%%%%%%%%%%%%%%%%%%%%%%%%%%%%%%%%%%%%%%
\newpage
\chapter{Microarray data and differential analysis}
\bigskip

%%%%%%%%%%%%%%%%%%%%%%%%%%%%%%%%%%%%%%%%%%%%%%%%%%%%%%%%%%%%%%%%%%%%%%
%%%%%%%%%%%%%%%%%%%%%%%%%%%%%%%%%%%%%%%%%%%%%%%%%%%%%%%%%%%%%%%%%%%%%%
\section{Molecular biology central dogma} 

\vspace{1cm}
\centerline{
  \begin{tabular}{c}
    \framebox{DNA molecule (gene)} \\
    $\mid$ \\
    \textred{\sl transcription} \\
    $\downarrow$ \\
    \textgreen{\framebox{messenger RNA (transcript)}} \\
    $\mid$ \\
    \textred{\sl translation} \\
    $\downarrow$ \\
    \framebox{Protein (biological function)}
  \end{tabular}
}

\vspace{1cm}
\paragraph{``Definition'':}
$
\left(\begin{tabular}{c} Expression level\\ of a gene
  \end{tabular}\right)
\propto
\left(\begin{tabular}{c} number of \\copies of mRNA \\ in the cell
  \end{tabular}\right) 
$

%%%%%%%%%%%%%%%%%%%%%%%%%%%%%%%%%%%%%%%%%%%%%%%%%%%%%%%%%%%%%%%%%%%%%%
%%%%%%%%%%%%%%%%%%%%%%%%%%%%%%%%%%%%%%%%%%%%%%%%%%%%%%%%%%%%%%%%%%%%%%
\newpage
\section{Microarray technology}

\begin{tabular}{ll}
  \begin{tabular}{p{11cm}}
    Aims to monitor the expression level of several thousands of genes
    simultaneously \\
    \\
    \\
    1 spot = 1 gene \\
    \\
    \\
    Expression level in the cell: \\
    ~\\
    \liste at given time, \\
    ~\\
    \liste in a given condition \\
  \end{tabular}
  &
  \begin{tabular}{c}
    \psfig{file = ../DNAchip-green.ps, width=12cm, height =12cm,
      bbllx=02, bblly=20, bburx=410, bbury=435, clip=}
  \end{tabular}
\end{tabular}

\newpage 
\psfig{file = ../MicroarrayTech.ps, angle=90, width=25cm, height=18cm}

%%%%%%%%%%%%%%%%%%%%%%%%%%%%%%%%%%%%%%%%%%%%%%%%%%%%%%%%%%%%%%%%%%%%%%
%%%%%%%%%%%%%%%%%%%%%%%%%%%%%%%%%%%%%%%%%%%%%%%%%%%%%%%%%%%%%%%%%%%%%%
\newpage
\section{Differential analysis}

\paragraph{Elementary data:} $Y_{itr} = $ expression level of gene
$i$ in condition $t$ ($t=1$ or $2$) at replicate $r$

\paragraph{Differentially expressed genes} are genes for which
$Y_{i1r}$ is not distributed as $Y_{i2r}$. 

Null hypothesis for gene $i$:
$
\Hbf_0(i) = \{Y_{i1r} \overset{\Lcal}{=} Y_{i2r}\}
$

\paragraph{Statistical test:} Student, Wilcoxon, permutation, {\it
  etc.} \\
(idem with more then 2 conditions: Fisher, Kruskall-Wallis, {\it
  etc.}.)

For each gene we get:
$$
\begin{tabular}{ll}
  the value of the test statistic & $T_i$ \\
  \\
  the corresponding $p$-value & $P_i = \Pr\{\Tcal > T_i
  \;|\;\Hbf_0(i)\}$ 
\end{tabular}
$$

%%%%%%%%%%%%%%%%%%%%%%%%%%%%%%%%%%%%%%%%%%%%%%%%%%%%%%%%%%%%%%%%%%%%%%
%%%%%%%%%%%%%%%%%%%%%%%%%%%%%%%%%%%%%%%%%%%%%%%%%%%%%%%%%%%%%%%%%%%%%%
%%%%%%%%%%%%%%%%%%%%%%%%%%%%%%%%%%%%%%%%%%%%%%%%%%%%%%%%%%%%%%%%%%%%%%
%%%%%%%%%%%%%%%%%%%%%%%%%%%%%%%%%%%%%%%%%%%%%%%%%%%%%%%%%%%%%%%%%%%%%%
\newpage
\chapter{Multiple testing problem}
\bigskip
%%%%%%%%%%%%%%%%%%%%%%%%%%%%%%%%%%%%%%%%%%%%%%%%%%%%%%%%%%%%%%%%%%%%%%
%%%%%%%%%%%%%%%%%%%%%%%%%%%%%%%%%%%%%%%%%%%%%%%%%%%%%%%%%%%%%%%%%%%%%%
%\section{General problem}

\paragraph{Rejection rule:} For a given level $\alpha$, 
$$
\begin{tabular}{rcl}
  $P_i < \alpha$ & $\qquad \Longrightarrow \qquad$ & gene $i$ is
    declared positive \\
  & & (i.e. differentially expressed)
\end{tabular}
$$

\paragraph{Multiple testing:} When performing $n$ simultaneous tests
$$
\begin{tabular}{c|cc|c}
  & \multicolumn{2}{c}{(Decision (random)} & \\
  & $\Hbf_0$ accepted & $\Hbf_0$ rejected & \\
  \hline
  $\Hbf_0$ true 
  & \begin{tabular}{c} $TN$ \\ true negatives \end{tabular} 
  & \begin{tabular}{c} $FN$ \\ false negatives \end{tabular}
  & \begin{tabular}{c} $n_0$ \\ negatives \end{tabular} \\
  $\Hbf_0$ false 
  & \begin{tabular}{c} $FP$ \\ false positives \end{tabular} 
  & \begin{tabular}{c} $TP$ \\ true positives \end{tabular} 
  & \begin{tabular}{c} $n_1$ \\ positives \end{tabular} \\
  \hline
  & $N$ negatives & $P$ positives & $n$
\end{tabular}
$$

All the random quantities (capital) depend on the data and the
pre-fixed level $\alpha$.

\newpage
\paragraph{Microarray experiment:} Typically $n=10\;000$ tests are
performed simultaneously 

For $\alpha = 5\%$, if no gene is actually differentially expressed
($n_1 = 0, n_0 = n$), we expect
$$
0.05 \times 10\;000 = 500 \mbox{``positive'' genes}
$$
which are \paragraph{all false positives}.

\paragraph{Problem:} We'd like to control some ``global risk'' $\alpha^*$
such as \\
\\
\liste the probability of having some false positive (FWER), \\
\\
\liste or the proportion of false positives (FDR).

(Dudoit \& al., Stat. Sci., 2003)

%%%%%%%%%%%%%%%%%%%%%%%%%%%%%%%%%%%%%%%%%%%%%%%%%%%%%%%%%%%%%%%%%%%%%%
%%%%%%%%%%%%%%%%%%%%%%%%%%%%%%%%%%%%%%%%%%%%%%%%%%%%%%%%%%%%%%%%%%%%%%
\newpage
\section{Family Wise Error Rate (FWER)}
\bigskip
\centerline{$FWER = \Pr\{FP \geq 1\}$}

\paragraph{Sidak:} If the $n$ tests are independent, 
$$
FP \sim \Bcal(n, \alpha) 
\qquad \Longrightarrow \qquad 
\Pr\{FP \geq 1\} = 1 - (1- \alpha)^n.
$$
Fixing level at $\alpha = 1 - (1- \alpha^*)^{1/n} (\simeq \alpha^*/n)$
ensures $FWER = \alpha^*$.

\paragraph{Bonferroni:} In any case
$$
FWER = \Pr\left\{\bigcup_i i \mbox{ false positive}\right\}
\leq \sum_i \Pr\left\{i \mbox{ false positive}\right\} = n \alpha
$$ 
Fixing level at $\alpha = \alpha^*/n$ ensures $FWER \leq
\alpha^*$.

\paragraph{Remark:} The independent case is, in some sens, the worst case.

%%%%%%%%%%%%%%%%%%%%%%%%%%%%%%%%%%%%%%%%%%%%%%%%%%%%%%%%%%%%%%%%%%%%%%
%%%%%%%%%%%%%%%%%%%%%%%%%%%%%%%%%%%%%%%%%%%%%%%%%%%%%%%%%%%%%%%%%%%%%%
\newpage
\section{False Discovery Rate (FDR)}
\bigskip
\centerline{$FDR = \Esp(FP/P)$}

\paragraph{Idea:} Instead of preventing any error, just control the
proportion of errors 
\centerline{$\Longrightarrow$ less conservative}

\paragraph{Benjamini \& Hochberg (95) procedure:} Given the sorted
$p$-values
$$
P_{(1)} \leq \dots \leq P_{(i)} \leq \dots \leq P_{(n)} ,
$$
rejecting $\Hbf_0$ for all $(i)$ such as
$$
P_{(i)} \leq \frac{i \alpha^*}n
\qquad \Longrightarrow \qquad
FDR \leq \frac{n_0}n \alpha^* \leq \alpha^*
$$

\paragraph{Benjamini \& Yakutieli (01):} For positively correlated
test statistics
$$
P_{(i)} \leq \frac{i \alpha^*}{n (\sum_j 1/j)}.
$$

\newpage
\paragraph{Number of positive genes:} Hedenfalk data ($n = 3226$, $\alpha^* = 5\%$)

\begin{tabular}{ccc}
  \begin{tabular}{l}
    $p$-value: \\
    {\bf $\centerdot\centerdot\centerdot$}  612 \\
    \\
    Bonferroni: \\
    \textred{\bf ---} 2 \\
    \\
    Sidak: \\
    \textlightgreen{\bf ---} 2 \\
    \\
    Holm: \\
    \textred{$\centerdot\centerdot\centerdot$} 2 \\
    \\
    Sidak adp.: \\
    \textlightgreen{$\centerdot\centerdot\centerdot$} 2 \\
    \\
    FDR: \\
    \textblue{$\centerdot\centerdot\centerdot$}   70 \\
    \\
  \end{tabular}
  &
  ~\vspace{1cm} \begin{rotate}{90} $\log_{10} p$-value \end{rotate}
  &
  \vspace{-1cm}
  \begin{tabular}{c}
    \epsfig{
      file=/RECHERCHE/EXPRESSION/EXEMPLES/HEDENFALK/FDR-FWER.eps,
      height=17cm, width=15cm, bbllx=65, bblly=290, bburx=285,
      bbury=480, angle=90, clip=} \\ 
    \small{gene rank}
  \end{tabular}
\end{tabular}

%   {\bf --} $5\%$ \\
%   \textred{--} Bonferroni \\
%   \textred{\dots} Holm \\
%   \textlightgreen{--} Sidak \\
%   \textlightgreen{\dots} Sidak ad. \\
%   \textblue{\dots} FDR

%%%%%%%%%%%%%%%%%%%%%%%%%%%%%%%%%%%%%%%%%%%%%%%%%%%%%%%%%%%%%%%%%%%%%%
%%%%%%%%%%%%%%%%%%%%%%%%%%%%%%%%%%%%%%%%%%%%%%%%%%%%%%%%%%%%%%%%%%%%%%
\newpage
\section{Local False Discovery Rate}

FDR provides a general information about the risk of the whole
procedure (up to step $i$).

We are interested in a specific risk, associated to each gene.

\paragraph{Local FDR ($\ell FDR$).} Initially defined by Efron \&
al. (JASA, 2001) in a mixture model framework:
$$
\ell FDR(i) := \Pr\{\Hbf_0(i) \mbox{ is false} \;|\; T_i\}.
$$


\paragraph{Derivative of the FDR:} $\ell FDR(i)$ can be also defined as the
derivative of the FDR, which can be estimated by 
$$
\widehat{n}_0 (P_{(i)} - P_{(i-1)})
$$
(Aubert \& al., BMC Bioinfo., 04).

%%%%%%%%%%%%%%%%%%%%%%%%%%%%%%%%%%%%%%%%%%%%%%%%%%%%%%%%%%%%%%%%%%%%%%
%%%%%%%%%%%%%%%%%%%%%%%%%%%%%%%%%%%%%%%%%%%%%%%%%%%%%%%%%%%%%%%%%%%%%%
%%%%%%%%%%%%%%%%%%%%%%%%%%%%%%%%%%%%%%%%%%%%%%%%%%%%%%%%%%%%%%%%%%%%%%
%%%%%%%%%%%%%%%%%%%%%%%%%%%%%%%%%%%%%%%%%%%%%%%%%%%%%%%%%%%%%%%%%%%%%%
\newpage
\chapter{Semi-parametric mixture model}
\bigskip
%%%%%%%%%%%%%%%%%%%%%%%%%%%%%%%%%%%%%%%%%%%%%%%%%%%%%%%%%%%%%%%%%%%%%%
%%%%%%%%%%%%%%%%%%%%%%%%%%%%%%%%%%%%%%%%%%%%%%%%%%%%%%%%%%%%%%%%%%%%%%
\section{Mixture model}

\paragraph{Property of the test statistic.} The standard hypotheses
testing theory implies that, under $\Hbf_0(i)$, $P_i$ is uniformly
distributed over $[0, 1]$: \\
\begin{tabular}{cc}
  \begin{tabular}{p{13.5cm}}
    $$
    P_i \underset{\Hbf_0(i)}{\sim} \Ucal_{[0, 1]}
    $$
    \\
    ~\\
    The $P_i$'s are distributed according to a mixture distribution
    with density 
    $$
    g(p) = a f(p) + (1-a) 
    $$
    \\
    ~\\
    The problem is then to estimate
  \end{tabular}
  &
  \begin{tabular}{c}
    \epsfig{
      file=/RECHERCHE/EXPRESSION/EXEMPLES/HEDENFALK/HistoPval.eps,
      height=10cm, width=10cm, bbllx=82, bblly=304, bburx=277,
      bbury=460, angle=90, clip=} 
  \end{tabular}
\end{tabular}
$$
\begin{tabular}{ll}
  \paragraph{$a$:} & the proportion of differentially expressed genes
  \\
  \\
  \paragraph{$f$:} & the  (alternative) density $f$ \\
\end{tabular}
$$

\newpage
\paragraph{Generalization:} We consider an i.i.d. sample $\{X_1,
\dots, X_n\}$ with mixture density
$$
g(x) = a f(x) + (1-a) \phi(x)
$$
\begin{tabular}{lr}
  The proportion $a$ is unknown &  $\longrightarrow$ \paragraph{parametric part} \\ 
  \\
  The density $f$ is completely unknown & \quad $\longrightarrow$ \paragraph{non
    parametric part} \\ 
  \\
  The density $\phi$ in completely specified & ($\Ucal_{[0, 1]}$,
  $\Ncal(0, 1)$, {\it etc.})
\end{tabular}

\bigskip \bigskip
\paragraph{Posterior probability.} We are interested in the
estimation of 
$$
\tau_i = \Pr\{Z_i = 1 \;|\; x_i\} = \Esp(Z_i \;|\; x_i) = \frac{a
  f(x_i)}{g(x_i)}
$$
where $Z_i =
\left\{
\begin{tabular}{rll}
  $Z_i = 1$ & if $i$ comes from $f$ & ($\Hbf_0(i)$ false), \\
  \\
  $Z_i = 0$ & otherwise & ($\Hbf_0(i)$ true).
\end{tabular}
\right.$


%%%%%%%%%%%%%%%%%%%%%%%%%%%%%%%%%%%%%%%%%%%%%%%%%%%%%%%%%%%%%%%%%%%%%%
%%%%%%%%%%%%%%%%%%%%%%%%%%%%%%%%%%%%%%%%%%%%%%%%%%%%%%%%%%%%%%%%%%%%%%
\newpage
\section{Density estimation}

\paragraph{Kernel estimate.} A natural non-parametric estimate of $f$
is
$$
\widehat{f}(x) = \frac1{\sum_i Z_i} \sum_i Z_i k_i(x)
$$
where
$$
k_i(x) = \frac1h k\left(\frac{x-x_i}h\right)
$$ 
$k$ being a kernel, i.e. a symmetric density function with mean 0.

\paragraph{Weighted kernel estimate.} Since the $Z_i$'s are unknown,
we propose to replace them by their conditional expectations:
$$
\widehat{f}(x) = \frac1{\sum_i \tau_i}\sum_i \tau_i k_i(x)
$$
$\tau_i$ is the weight of observation $i$ in the estimation of $f$.
\newpage
\paragraph{Property of the $\widehat{\tau}_i$}. The estimates of the
$\tau_i$'s must satisfy
$$
\widehat{\tau}_j = \frac{a \widehat{f}(x_j)}{\widehat{g}(x_j)} 
= \frac{a \sum_i \widehat{\tau}_i k_i(x_j)}{a \sum_i \widehat{\tau}_i
  k_i(x_j) + (1-a) \phi(x_j) \sum_i \widehat{\tau}_i}
$$
or 
$$
\widehat{\tau}_j =\frac{\sum_i \widehat{\tau}_i b_{ij}}{\sum_i \widehat{\tau}_i b_{ij}
  + \sum_i \widehat{\tau}_i}
\qquad \mbox{with} \qquad
b_{ij} = \frac{a}{1-a} \frac{k_i(x_j)}{\phi(x_j)} \geq 0
$$

\paragraph{Function $\psibf$.}
$$
\begin{array}{rcl}
  \psibf: \Rbb^n & \rightarrow & \Rbb^n \\
  \ubf & \rightarrow &  \displaystyle{\psibf(\ubf): \psi_j(\ubf) = \frac{\sum_i
  u_i b_{ij}}{\sum_i u_i b_{ij}+ \sum_i u_i}}  
\end{array}
$$

\bigskip \bigskip 
\centerline{$\widehat{\taubf} = (\widehat{\tau}_1, \dots,
  \widehat{\tau}_n)$ should be a fixed point of $\psibf$.}

\newpage
\paragraph{Estimation algorithm of $\taubf$.} Given some initial
$\widehat{\taubf}^{0}$, iterate $\psibf$:
$$
\widehat{\taubf}^{h+1} = \psibf(\widehat{\taubf}^{h}).
$$

\paragraph{$a$ remains fix:} Must to be estimate
independently.

\paragraph{2 steps of the algorithm:} \\
{\bf ''E'' step:} given $\widehat{f}^h$ and $\widehat{g}^h$, calculate
$$
\widehat{\taubf}^{h+1} = a \widehat{f}^h(x_i) \left/
  \widehat{g}^h(x_i) \right..
$$
{\bf Other step:} given $\widehat{\taubf}^{h}$, estimate $f$ and
$g$:
$$
\widehat{f}^h(x) = \sum_i \widehat{\tau}^h_i k_i(x) \left/ \sum_i
  \widehat{\tau}^h_i \right., 
\qquad 
\widehat{g}^h(x) = a \widehat{f}^h(x) + (1-a) \phi(x).
$$
This second step does not maximize the likelihood \\
\centerline{$\longrightarrow$ not an E-M algorithm.}

\newpage
\centerline{\framebox{
    \begin{tabular}{rl}
      \paragraph{Theorem:} & $\psibf$ is contracting \\
      \\
      $\Longrightarrow$ & the algorithm converges toward its unique
      fix point. 
    \end{tabular}
    }}

\paragraph{Sketch of proof.} $\psibf = \alphabf \circ \betabf \circ
\gammabf$:
$$
\alpha_j(\ubf) = \frac{u_j}{u_j + 1}, 
\qquad
\beta_j(\ubf) = \sum_i b_{ij} u_i, 
\qquad 
\gamma_j(\ubf) = \frac{u_j}{\sum_i u_i},
$$

\begin{description}
\item[1.] Simplex $\Ecal = \{\ubf: \sum_i u_i = 1\}$ ($\gammabf = $
  projection on $\Ecal$)
  $$
  \ubf^* \in \Rbb^n: \psibf(\ubf) = \ubf
  \qquad \Longleftrightarrow \qquad
  \vbf^* = \gammabf(\ubf^*) \in \Ecal: \gammabf \circ \psibf (\vbf) = \vbf
  $$
  $\rightarrow$ Just consider $\gammabf \circ \psibf$ on the
  simplex $\Ecal$.
\item[2.] Brouwer's theroem: $\Ecal$ is compact and $\gammabf \circ
  \psibf$ is continuous, so at least one fix point exists.
\end{description}

\newpage
\begin{description}
\item[3.] Interior of $\Ecal$: $\Ecal' = \{\ubf \in \Ecal: \forall i, u_i
  > 0\}$. 
  $$
  d(\ubf, \vbf) = \log \left[ \max_i \left(\frac{u_i}{v_i}\right)
    \left/ \min_i \left(\frac{u_i}{v_i}\right) \right. \right]
  $$
  is a distance on $\Ecal'$.
\item[4.]  $ d[\gammabf \circ \psibf(\ubf), \gammabf \circ
  \psibf(\vbf)] < d(\ubf, \vbf)$ \qquad (except if $\ubf =
  \vbf$). \\
  $\rightarrow$ $\gammabf \circ \psibf$ admits at most one fix point
  in $\Ecal'$.
\item[5.] If $k_{ij} > 0$ for all $(i, j)$: 
  $$
  \{\ubf \in \Ecal \setminus \Ecal'\} \quad \Longrightarrow \quad
  \{\gammabf \circ \psibf(\ubf) \in \Ecal'\}.
  $$
\end{description}

$\gammabf \circ \psibf$ (and therefore for $\psibf$) admits one unique
fix point toward which the algorithm converges.

%%%%%%%%%%%%%%%%%%%%%%%%%%%%%%%%%%%%%%%%%%%%%%%%%%%%%%%%%%%%%%%%%%%%%%
%%%%%%%%%%%%%%%%%%%%%%%%%%%%%%%%%%%%%%%%%%%%%%%%%%%%%%%%%%%%%%%%%%%%%%
\newpage
\section{Estimation $a$ (and $h$)}

\paragraph{Analogy with EM. } $a$ could be estimated iteratively:
$$
\widehat{a}^h = \frac1n \sum_i \widehat{\tau}_i^h
$$
but
$$
\widehat{\taubf} = (\begin{array}{ccc}1 & \dots & 1 \end{array}), 
\qquad
\widehat{a} = 1
$$
is a fixed point of this algorithm.

\paragraph{Remark.} For a given $a$, there is a unique
$\widehat{\taubf}$. \\
In some sense, $a$ is the unique parameter of the problem.

\newpage
\paragraph{Cross-validation.} $a$ (and $h$) are estimated as follows
\begin{enumerate}
\item Split the dataset $\Dcal$ into $V$ subsets $\Dcal_1, \dots,
  \Dcal_V$. \\
  Typically, $V = 5$ or $10$.
\item For $v=1 \dots V$ \\
  \liste estimate $f$ and $g$ with the data from $\Dcal \setminus \Dcal_v$
  ($\rightarrow \widehat{f}_v, \widehat{g}_v$), \\
  \liste calculate
  $$
  \Lcal_{CV}(\Dcal; a) = \frac1V \sum_v \sum_{i \in \Dcal_v} \log \widehat{g}_v(x_i).
  $$
\item Miximize $\Lcal_{CV}$ (numerically):
  $$
  \widehat{a} = \arg\max_a \Lcal_{CV}(\Dcal; a).
  $$
\end{enumerate}

%%%%%%%%%%%%%%%%%%%%%%%%%%%%%%%%%%%%%%%%%%%%%%%%%%%%%%%%%%%%%%%%%%%%%%
%%%%%%%%%%%%%%%%%%%%%%%%%%%%%%%%%%%%%%%%%%%%%%%%%%%%%%%%%%%%%%%%%%%%%%
%%%%%%%%%%%%%%%%%%%%%%%%%%%%%%%%%%%%%%%%%%%%%%%%%%%%%%%%%%%%%%%%%%%%%%
%%%%%%%%%%%%%%%%%%%%%%%%%%%%%%%%%%%%%%%%%%%%%%%%%%%%%%%%%%%%%%%%%%%%%%
\newpage
\chapter{$FDR$ and local $FDR$ estimation}
\bigskip
%%%%%%%%%%%%%%%%%%%%%%%%%%%%%%%%%%%%%%%%%%%%%%%%%%%%%%%%%%%%%%%%%%%%%%
%%%%%%%%%%%%%%%%%%%%%%%%%%%%%%%%%%%%%%%%%%%%%%%%%%%%%%%%%%%%%%%%%%%%%%
%%%%%%%%%%%%%%%%%%%%%%%%%%%%%%%%%%%%%%%%%%%%%%%%%%%%%%%%%%%%%%%%%%%%%%
%%%%%%%%%%%%%%%%%%%%%%%%%%%%%%%%%%%%%%%%%%%%%%%%%%%%%%%%%%%%%%%%%%%%%%
\newpage
\chapter{Applications}
\bigskip
%%%%%%%%%%%%%%%%%%%%%%%%%%%%%%%%%%%%%%%%%%%%%%%%%%%%%%%%%%%%%%%%%%%%%%
%%%%%%%%%%%%%%%%%%%%%%%%%%%%%%%%%%%%%%%%%%%%%%%%%%%%%%%%%%%%%%%%%%%%%%
\section{Probit transform}
%Instead of modeling the distribution of the $P_i$', we consider the
%$$
\begin{tabular}{lcr}
  $P_i \in [0, 1]$
  &   
  \qquad 
  &
  $X_i = \Phi^{-1}(P_i) \in \Rbb$ 
  \\
  \multicolumn{3}{c}{(Efron, JASA, 2005)}
  \\
  \textred{$\phi = \Ucal_{[0; 1]}$}
  & 
  & 
  \textred{$\phi = \Ncal(0, 1)$}
  \\
  \psfig{file =
  /RECHERCHE/EXPRESSION/COMPMULT/EMpondere/Article/HistoPval.ps, 
  width=11.5cm, height=10cm, bbllx=211, bblly=322, bburx=424,
  bbury=480, clip=} 
  &
  &
%  \vspace{-2cm}
  \psfig{file =
  /RECHERCHE/EXPRESSION/COMPMULT/EMpondere/Article/HistoProbit.ps,
  width=11.5cm, height=10cm, bbllx=211, bblly=322, bburx=424,
  bbury=480, clip=} 
  \\
\end{tabular}
%$$

%%%%%%%%%%%%%%%%%%%%%%%%%%%%%%%%%%%%%%%%%%%%%%%%%%%%%%%%%%%%%%%%%%%%%%
%%%%%%%%%%%%%%%%%%%%%%%%%%%%%%%%%%%%%%%%%%%%%%%%%%%%%%%%%%%%%%%%%%%%%%
\newpage
\section{Simulations}

%%%%%%%%%%%%%%%%%%%%%%%%%%%%%%%%%%%%%%%%%%%%%%%%%%%%%%%%%%%%%%%%%%%%%%
%%%%%%%%%%%%%%%%%%%%%%%%%%%%%%%%%%%%%%%%%%%%%%%%%%%%%%%%%%%%%%%%%%%%%%
\newpage
\section{Hedenfalk data}

\paragraph{Comparison of 2 breast cancers (BRCA1 / BRCA2):} $n = 3226$
genes
%%%%%%%%%%%%%%%%%%%%%%%%%%%%%%%%%%%%%%%%%%%%
% CHANGER COCHRAN -> VARMIXT
%%%%%%%%%%%%%%%%%%%%%%%%%%%%%%%%%%%%%%%%%%%%
$$
\begin{tabular}{cc} 
  \multicolumn{2}{c}{log-likelihood $\Lcal(a, h)$, \quad ($V = 5$)} \\
  training set & test set: $\Lcal_{CV}$ \\
  \begin{tabular}{c}
    \epsfig{
      file=/RECHERCHE/EXPRESSION/EXEMPLES/HEDENFALK/Cochran-Gaus.eps,
      height=12cm, width=12cm, bbllx=330, bblly=500, bburx=550,
      bbury=710, clip=, angle=90} 
  \end{tabular}
  &
  \begin{tabular}{c} 
    \epsfig{
      file=/RECHERCHE/EXPRESSION/EXEMPLES/HEDENFALK/Cochran-Gaus.eps,
      height=12cm, width=12cm, bbllx=330, bblly=290, bburx=550,
      bbury=500, clip=, angle=90} 
  \end{tabular}
  \\
  $\widehat{a} \rightarrow 1, \quad \widehat{h} \rightarrow 0$
  & $\widehat{a} = ???, \quad \widehat{h} = ???$
\end{tabular}
$$

\newpage
$$
\begin{tabular}{cc} 
  \begin{tabular}{c}
    Mixture model \\
    $\textblue{\widehat{g}(x)}  = a \textred{\widehat{f}(x)} + (1-a)
    \textgreen{\widehat{f}(x)}$ \\ 
    \epsfig{
      file=/RECHERCHE/EXPRESSION/EXEMPLES/HEDENFALK/Cochran-Gaus.eps,
      height=12cm, width=12cm, bbllx=66, bblly=510, bburx=283,
      bbury=691, clip=, angle=90} 
  \end{tabular}
  &
  \begin{tabular}{c} 
    $\textred{\widehat{FDR}_i}, \textblue{\widehat{\ell FDR}_i} \times
    \Phi^{-1}(P_i)$ \\ 
    \epsfig{
      file=/RECHERCHE/EXPRESSION/EXEMPLES/HEDENFALK/Cochran-Gaus.eps,
      height=12cm, width=6cm, bbllx=66, bblly=295, bburx=283,
      bbury=480, clip=, angle=90} 
    \\
    $\textred{\widehat{FDR}_i}, \textblue{\widehat{\ell FDR}_i} \times
    P_i$ \\ 
    \epsfig{
      file=/RECHERCHE/EXPRESSION/EXEMPLES/HEDENFALK/Cochran-Gaus.eps,
      height=12cm, width=6cm, bbllx=66, bblly=85, bburx=283,
      bbury=265, clip=, angle=90} 
  \end{tabular}
\end{tabular}
$$

%%%%%%%%%%%%%%%%%%%%%%%%%%%%%%%%%%%%%%%%%%%%%%%%%%%%%%%%%%%%%%%%%%%%%%
%%%%%%%%%%%%%%%%%%%%%%%%%%%%%%%%%%%%%%%%%%%%%%%%%%%%%%%%%%%%%%%%%%%%%%
%%%%%%%%%%%%%%%%%%%%%%%%%%%%%%%%%%%%%%%%%%%%%%%%%%%%%%%%%%%%%%%%%%%%%%
%%%%%%%%%%%%%%%%%%%%%%%%%%%%%%%%%%%%%%%%%%%%%%%%%%%%%%%%%%%%%%%%%%%%%%
\end{document}
%%%%%%%%%%%%%%%%%%%%%%%%%%%%%%%%%%%%%%%%%%%%%%%%%%%%%%%%%%%%%%%%%%%%%%
%%%%%%%%%%%%%%%%%%%%%%%%%%%%%%%%%%%%%%%%%%%%%%%%%%%%%%%%%%%%%%%%%%%%%%
%%%%%%%%%%%%%%%%%%%%%%%%%%%%%%%%%%%%%%%%%%%%%%%%%%%%%%%%%%%%%%%%%%%%%%
%%%%%%%%%%%%%%%%%%%%%%%%%%%%%%%%%%%%%%%%%%%%%%%%%%%%%%%%%%%%%%%%%%%%%%

