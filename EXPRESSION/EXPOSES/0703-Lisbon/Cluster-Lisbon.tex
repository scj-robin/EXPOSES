\documentclass[dvips, lscape]{foils}
%\documentclass[dvips, french]{slides}
\textwidth 18.5cm
\textheight 25cm 
\topmargin -1cm 
\oddsidemargin  -1cm 
\evensidemargin  -1cm

% Maths
\usepackage{amsfonts, amsmath, amssymb}

\newcommand{\coefbin}[2]{\left( 
    \begin{array}{c} #1 \\ #2 \end{array} 
  \right)}
\newcommand{\Bcal}{\mathcal{B}}
\newcommand{\Ccal}{\mathcal{C}}
\newcommand{\Dcal}{\mathcal{D}}
\newcommand{\Ecal}{\mathcal{E}}
\newcommand{\Mcal}{\mathcal{M}}
\newcommand{\Ncal}{\mathcal{N}}
\newcommand{\Pcal}{\mathcal{P}}
\newcommand{\Lcal}{\mathcal{L}}
\newcommand{\Tcal}{\mathcal{T}}
\newcommand{\Ucal}{\mathcal{U}}
\newcommand{\alphabf}{\mbox{\mathversion{bold}{$\alpha$}}}
\newcommand{\betabf}{\mbox{\mathversion{bold}{$\beta$}}}
\newcommand{\gammabf}{\mbox{\mathversion{bold}\newcommand{\psibf}{\mbox{\mathversion{bold}{$\psi$}}}
{$\gamma$}}}
\newcommand{\mubf}{\mbox{\mathversion{bold}{$\mu$}}}
\newcommand{\psibf}{\mbox{\mathversion{bold}{$\psi$}}}
\newcommand{\Sigmabf}{\mbox{\mathversion{bold}{$\Sigma$}}}
\newcommand{\taubf}{\mbox{\mathversion{bold}{$\tau$}}}
\newcommand{\Hbf}{{\bf H}}
\newcommand{\Ibf}{{\bf I}}
\newcommand{\Sbf}{{\bf S}}
\newcommand{\mbf}{{\bf m}}
\newcommand{\ubf}{{\bf u}}
\newcommand{\vbf}{{\bf v}}
\newcommand{\xbf}{{\bf x}}
\newcommand{\Xbf}{{\bf X}}
\newcommand{\Esp}{{\mathbb E}}
\newcommand{\Var}{{\mathbb V}}
\newcommand{\Cov}{{\mathbb C}\mbox{ov}}
\newcommand{\Ibb}{{\mathbb I}}
\newcommand{\Rbb}{\mathbb{R}}

% Couleur et graphiques
\usepackage{color}
\usepackage{graphics}
\usepackage{epsfig} 
\usepackage{pstcol}

% Texte
\usepackage{lscape}
\usepackage{../../../../Latex/fancyheadings, rotating, enumerate}
%\usepackage[french]{babel}
\usepackage[latin1]{inputenc}
\definecolor{darkgreen}{cmyk}{0.5, 0, 0.5, 0.5}
\definecolor{orange}{cmyk}{0, 0.6, 0.8, 0}
\definecolor{jaune}{cmyk}{0, 0.5, 0.5, 0}
\newcommand{\textblue}[1]{\textcolor{blue}{#1}}
\newcommand{\textred}[1]{\textcolor{red}{#1}}
\newcommand{\textgreen}[1]{\textcolor{green}{ #1}}
\newcommand{\textlightgreen}[1]{\textcolor{green}{#1}}
%\newcommand{\textgreen}[1]{\textcolor{darkgreen}{#1}}
\newcommand{\textorange}[1]{\textcolor{orange}{#1}}
\newcommand{\textyellow}[1]{\textcolor{yellow}{#1}}
\newcommand{\refer}[2]{{\sl #1}}

% Sections
%\newcommand{\chapter}[1]{\centerline{\LARGE \textblue{#1}}}
% \newcommand{\section}[1]{\centerline{\Large \textblue{#1}}}
% \newcommand{\subsection}[1]{\noindent{\Large \textblue{#1}}}
% \newcommand{\subsubsection}[1]{\noindent{\large \textblue{#1}}}
% \newcommand{\paragraph}[1]{\noindent {\textblue{#1}}}
% Sectionsred
\newcommand{\chapter}[1]{
  \addtocounter{chapter}{1}
  \setcounter{section}{0}
  \setcounter{subsection}{0}
  {\centerline{\LARGE \textblue{\arabic{chapter} - #1}}}
  }
\newcommand{\section}[1]{
  \addtocounter{section}{1}
  \setcounter{subsection}{0}
  {\centerline{\Large \textblue{\arabic{chapter}.\arabic{section} - #1}}}
  }
\newcommand{\subsection}[1]{
  \addtocounter{subsection}{1}
  {\noindent{\large \textblue{\arabic{chapter}.\arabic{section}.\arabic{subsection} - #1}}}
  }
\newcommand{\paragraph}[1]{\noindent{\textblue{#1}}}

%%%%%%%%%%%%%%%%%%%%%%%%%%%%%%%%%%%%%%%%%%%%%%%%%%%%%%%%%%%%%%%%%%%%%%
%%%%%%%%%%%%%%%%%%%%%%%%%%%%%%%%%%%%%%%%%%%%%%%%%%%%%%%%%%%%%%%%%%%%%%
%%%%%%%%%%%%%%%%%%%%%%%%%%%%%%%%%%%%%%%%%%%%%%%%%%%%%%%%%%%%%%%%%%%%%%
%%%%%%%%%%%%%%%%%%%%%%%%%%%%%%%%%%%%%%%%%%%%%%%%%%%%%%%%%%%%%%%%%%%%%%
\begin{document}
%%%%%%%%%%%%%%%%%%%%%%%%%%%%%%%%%%%%%%%%%%%%%%%%%%%%%%%%%%%%%%%%%%%%%%
%%%%%%%%%%%%%%%%%%%%%%%%%%%%%%%%%%%%%%%%%%%%%%%%%%%%%%%%%%%%%%%%%%%%%%
%%%%%%%%%%%%%%%%%%%%%%%%%%%%%%%%%%%%%%%%%%%%%%%%%%%%%%%%%%%%%%%%%%%%%%
%%%%%%%%%%%%%%%%%%%%%%%%%%%%%%%%%%%%%%%%%%%%%%%%%%%%%%%%%%%%%%%%%%%%%%
\landscape
\newcounter{chapter}
\newcounter{section}
\newcounter{subsection}
\setcounter{chapter}{0}
\headrulewidth 0pt 
\pagestyle{fancy} 
\cfoot{}
\rfoot{\begin{rotate}{90}{
      \hspace{1cm} \tiny S. Robin: Clustering for microarrays data
      }\end{rotate}}
\rhead{\begin{rotate}{90}{
      \hspace{-.5cm} \tiny \thepage
      }\end{rotate}}

%%%%%%%%%%%%%%%%%%%%%%%%%%%%%%%%%%%%%%%%%%%%%%%%%%%%%%%%%%%%%%%%%%%%%%
%%%%%%%%%%%%%%%%%%%%%%%%%%%%%%%%%%%%%%%%%%%%%%%%%%%%%%%%%%%%%%%%%%%%%%
\begin{center}
  \textblue{\LARGE Clustering analysis of microarray data} 

   \vspace{1cm}
   {\large S. {Robin}} \\
   robin@inapg.inra.fr

   {UMR INA-PG / ENGREF / INRA, Paris} \\
   {Math�matique et Informatique Appliqu�es}
   
%   \vspace{3cm}
    \vspace{1cm}
    {Microarray Design and Statistical Analysis} \\
    {Lisbon, \today}
\end{center}

\paragraph{Outline}

1 - General problem

2 - Distance based methods

3 - Model based methods

%%%%%%%%%%%%%%%%%%%%%%%%%%%%%%%%%%%%%%%%%%%%%%%%%%%%%%%%%%%%%%%%%%%%%%
%%%%%%%%%%%%%%%%%%%%%%%%%%%%%%%%%%%%%%%%%%%%%%%%%%%%%%%%%%%%%%%%%%%%%%
%%%%%%%%%%%%%%%%%%%%%%%%%%%%%%%%%%%%%%%%%%%%%%%%%%%%%%%%%%%%%%%%%%%%%%
%%%%%%%%%%%%%%%%%%%%%%%%%%%%%%%%%%%%%%%%%%%%%%%%%%%%%%%%%%%%%%%%%%%%%%
\newpage
\chapter{Introduction} 
%%%%%%%%%%%%%%%%%%%%%%%%%%%%%%%%%%%%%%%%%%%%%%%%%%
To understand the genes' functions, clusters of genes having
similar expression profiles in a set of conditions are defined \\
\epsfig{figure=../Figures/SLA01-Fig3.ps,height=24cm,width=12cm,angle=270,
  clip=}

\refer{Schaffer \& al. (01)}{Diurnal and Circadian-Regulated Genes in
Arabidopsis}

%%%%%%%%%%%%%%%%%%%%%%%%%%%%%%%%%%%%%%%%%%%%%%%%%%
\newpage
\section{General problem} 
%%%%%%%%%%%%%%%%%%%%%%%%%%%%%%%%%%%%%%%%%%%%%%%%%%

\paragraph{Unsupervised classification:} 
Let $\Pcal$ denote the ``population'' of the $G$ genes. The problem is
to find a partition (\textblue{``class discovery''}) $\{\Ccal_1, \dots,
\Ccal_K\}$ of $\Pcal$:
$$
\bigcup_{k=1}^K \Ccal_k = \Pcal, 
$$
where groups $\Ccal_k$ are \\
$\bullet \quad$ homogeneous (low within-group variability) \\
$\bullet \quad$ well separated (high between-group variability)

\paragraph{Remark:} Most clustering techniques provide disjoint clusters: 
$$
\Ccal_k \cap \Ccal_{k'} = \varnothing \quad \mbox{if} \quad k \neq k'.
$$
This property is not always biologically desirable.

\newpage
\paragraph{Main difficulty:} Number of partitions in $K$ groups of a set of
size $G =$
$$
\frac{1}{K!} \sum_{k=0}^K (-1)^k (K-k)^G 
\coefbin{K}{k}
$$

\centerline{$\simeq 10^{47}$ partitions of $G=100$ genes into $K=3$ groups} 
\centerline{$\simeq 10^{68}$ partitions of $G=100$ genes into $K=5$ groups} 

\paragraph{Distance based (heuristic) methods:}
\begin{itemize}
\item  Hierarchical algorithms ($K$ unknown)
\item $K$ means ($K$ known)
\end{itemize}

\paragraph{Model based (statistical) methods:} 
\begin{itemize}
\item  Mixture models
\end{itemize}


%%%%%%%%%%%%%%%%%%%%%%%%%%%%%%%%%%%%%%%%%%%%%%%%%%
\newpage
\chapter{Distance based methods} 
\bigskip\bigskip
%%%%%%%%%%%%%%%%%%%%%%%%%%%%%%%%%%%%%%%%%%%%%%%%%%

%%%%%%%%%%%%%%%%%%%%%%%%%%%%%%%%%%%%%%%%%%%%%%%%%%
\section{Hierarchical clustering} 
%%%%%%%%%%%%%%%%%%%%%%%%%%%%%%%%%%%%%%%%%%%%%%%%%%

The clusters are built iteratively joining the two ``closest'' genes
or groups at each step. The result is a \textblue{tree} (dendrogram).

These methods are based on three basic elements: 
\begin{itemize}
\item a distance between genes 
  $$
  d(g, g')
  $$
\item  a distance between groups 
  $$
  d(\Ccal, \Ccal')
  $$
\item a stopping rule
\end{itemize}

\newpage
\paragraph{Typical output:} free software ``{\sl Cluster}''

\begin{pspicture}(24, 10)
  \rput[bl](0, 0){\epsfig{figure=../Figures/ESB98-Fig1.ps, height=24cm,
      width=10cm, angle=270, bbllx=22, bblly=15, bburx=300, bbury=825,
      clip=}
    }
  
  \rput[tr](23, 10){\begin{tabular}{r}
      \refer{Eisen \& al. (98)}{} \\
      \\
      \\
    \end{tabular}}
\end{pspicture}

Genes are clustered according to their expression profile over a given
period. \\
Every algorithm will \textblue{always provide a tree} \dots even if the
data are \textblue{not structured} according to a tree. \\

\centerline{\begin{tabular}{rl} But: & \textblue{\sl If all a man
      has is a hammer} \\
    & \textblue{\sl then every problem looks like a nail.}
      \end{tabular}}

%%%%%%%%%%%%%%%%%%%%%%%%%%%%%%%%%%%%%%%%%%%%%%%%%%
\newpage
\subsection{Distance between genes}

Typically, euclidian distance:
$$
\begin{tabular}{ll}
  simple: &  $\displaystyle{d^2(g, g') = \sum_t (x_{gt} -
    x_{g't})^2}$, \\
  \\
  standardized: & $\displaystyle{d^2(g, g') = \sum_t (x_{gt} - x_{g't})^2 /
    \sigma^2_t}$.
\end{tabular}
$$
Some algorithms work with only a dissimilarity (or a similarity).

\paragraph{Dissimilarity:} 3 properties 
$$
d(g, g') \geq 0, \qquad d(g, g) = 0, \qquad d(g, g') = d(g', g)
$$

\paragraph{Distance:} 1 additional property
$$
d(g, g'') \leq d(g, g') + d(g', g'')
$$

%%%%%%%%%%%%%%%%%%%%%%%%%%%%%%%%%%%%%%%%%%%%%%%%%%
\newpage 
Similarity can be based on the correlation coefficient
$$
\begin{tabular}{ll}
  centered: & $\displaystyle{r(g, g') = \frac{\sum_t (x_{gt} -
        \bar{x}_g)(x_{g't} - \bar{x}_{g'})}{\sqrt{\sum_t (x_{gt} -
        \bar{x}_g)^2} \sqrt{\sum_t (x_{g't} - \bar{x}_{g'})^2}}}$ \\
  \\
  non centered: & $\displaystyle{r'(g, g') = \frac{\sum_t x_{gt}
      x_{g't}}{\sqrt{\sum_t x_{gt}^2} \sqrt{\sum_t x_{g't}^2}}}$.
\end{tabular}
$$

Different similarities derived from $r(g, g')$ may have (very)
different significations: \\
%$$
\begin{tabular}{lccc}
  & \epsfig{file=../Figures/GraphCorr09.ps, height=4cm, width=6cm, bbllx=123,
    bblly=347, bburx=505, bbury=510, clip=}  
  & \epsfig{file=../Figures/GraphCorr00.ps, height=4cm, width=6cm, bbllx=123,
    bblly=347, bburx=505, bbury=510, clip=} 
  & \epsfig{file=../Figures/GraphCorr-09.ps, height=4cm, width=6cm, bbllx=123,
    bblly=347, bburx=505, bbury=510, clip=} \\
  $r(g, g')$ & 0.90 & 0.0 & -0.90 \\
%  $\displaystyle{\frac{1 + r(g, g')}{2}}$ & 0.95 & 0.5 & 0.05 \\
  $\left[1 + r(g, g')\right]$ / 2 & 0.95 & 0.5 & 0.05 \\
  $\left[r(g, g')\right]^2$ & 0.81 & 0.0 & 0.81 
\end{tabular}
%$$

%%%%%%%%%%%%%%%%%%%%%%%%%%%%%%%%%%%%%%%%%%%%%%%%%%
\newpage
\subsection{Distance between groups} 

This criterion gives the general name of the clustering algorithm

\begin{tabular}{llcl}
  Single linkage: & $\displaystyle{d(\Ccal, \Ccal')}$ & = & $\displaystyle{
    \min_{g \in \Ccal, g' \in \Ccal'} d(g, g')}$ \\
  \\
  Average linkage: & $\displaystyle{d(\Ccal, \Ccal')}$ & = & $\displaystyle{
    \frac{1}{|\Ccal| |\Ccal'|} \sum_{g \in \Ccal} \sum_{g' \in \Ccal'}
    d(g, g')}$ \\
  \\
  Complete linkage: & $\displaystyle{d(\Ccal, \Ccal')}$ & = &
    $\displaystyle{ \max_{g \in \Ccal, g' \in \Ccal'} d(g, g')}$ \\
  \\
  Centroid: & $\displaystyle{d(\Ccal, \Ccal')}$ & = & $\displaystyle{
    d(\overline{g}, \overline{g}')}$ \\
  \\
  Ward: & $\displaystyle{d^2(\Ccal, \Ccal')}$ & = & $\displaystyle{
    \frac{|\Ccal| |\Ccal'|}{|\Ccal| + |\Ccal'|} d^2(\overline{g},
    \overline{g}')}$ \\
  \\
  & & = & within variance of the groups
\end{tabular}

\newpage
\paragraph{Ward method:} based on the inertia $I_n$ at step $n$

Step 0: $p_g \equiv 1/G$

Step $n$:
\begin{equation*}
  \hspace{-1cm}
  \begin{split}
    & I_n = \sum_g p_g \|\xbf_g - \mbf\|^2 \\
    & = p_1 \|\xbf_1 - \mbf\|^2 + p_2 \|\xbf_2 - \mbf\|^2
    + \sum_{g> 2} p_g \|\xbf_g - \mbf\|^2  \\
    & = \underset{d^2(1, 2)}{\underbrace{p_1 \left\|\xbf_1 -
          \tilde{\xbf}^{12}\right\|^2 + p_2 \left\|\xbf_2 -
          \tilde{\xbf}^{12}\right\|^2}} +
    \underset{I_{n+1}}{\underbrace{(p_1+p_2) \|\tilde{\xbf}^{12} -
        \mbf\|^2 + \sum_{g> 2} p_g \|\xbf_g - \mbf\|^2}}
  \end{split}
\end{equation*}
where
$$
\xbf_g = (x_{g1}\dots x_{gT}), 
\quad 
\mbf = (\bar{x}_1 \dots \bar{x}_T), 
\quad
\tilde{\xbf}^{12} = \frac{p_1 \xbf_1 + p_2 \xbf_2}{p_1 + p_2}
$$

The new element ``12'' has weight $(p_1 + p_2)$ and coordinates $\tilde{\xbf}^{12}$

%%%%%%%%%%%%%%%%%%%%%%%%%%%%%%%%%%%%%%%%%%%%%%%%%%
\newpage
\subsection{Stopping rule} 

\paragraph{Local:} for ex. distance between the two clusters to be
joined at the current step
$$
d^2(g, g') > \mbox{threshold}
$$

\paragraph{Global:} for ex. remaining variance or inertia
$$
I_n > \mbox{threshold}
$$

\newpage
\paragraph{Other methods:} 

\begin{tabular}{ll}
  UPGMA: & Average linkage\\
  \\
  UPGMC: & $\displaystyle{d(\Ccal, \Ccal'\Ccal'') = \frac{|\Ccal'| d(\Ccal, \Ccal') +
    |\Ccal''| d(\Ccal, \Ccal'')}{|\Ccal'| + |\Ccal''|} - 
  \frac{|\Ccal'| |\Ccal''| d(\Ccal', \Ccal'')}{(|\Ccal'| + |\Ccal''|)^2}}$ \\
  \\
 WPGMA: & $\displaystyle{d(\Ccal, \Ccal'\Ccal'') = \frac{d(\Ccal, \Ccal') + d(\Ccal,
    \Ccal'')}{2}}$ \\
 \\
 WPGMC: & $\displaystyle{d(\Ccal, \Ccal'\Ccal'') = \frac{d(\Ccal, \Ccal') +
    d(\Ccal, \Ccal'')}{2} - \frac{d(\Ccal', \Ccal'')}{4}}$ \\
\end{tabular}

$$
\left. \begin{tabular}{r} Unweighted \\ Weighted \end{tabular} \right\}
\mbox{Pairwise Group Method}
\left\{ \begin{tabular}{l} Average \\ Centroid \end{tabular} \right.
$$
%\refer{Sokal \& Michener (58)}{}, 

\refer{Sokal \& Sneath (63)}{}

%%%%%%%%%%%%%%%%%%%%%%%%%%%%%%%%%%%%%%%%%%%%%%%%%%
\newpage
\subsection{Toy example} 
$$
\begin{array}{c|ccccc|c}
  d(g,g') & a & b & c & \textblue{d} & \textblue{e} & f\\ 
  p(g) & (1) & (1) & (1) & \textblue{(1)} & \textblue{(1)} & (?)\\ 
  \hline \hline
  a &   & 3 & 7 & \textblue{3} & \textblue{4} & ? \\ 
  b & 3 &   & 4 & \textblue{4} & \textblue{1} & ? \\ 
  c & 7 & 4 &   & \textblue{2} & \textblue{6} & ? \\ 
  d & \textblue{3} & \textblue{4} & \textblue{2} &   & \textblue{.5} & \textblue{*} \\ 
  e & \textblue{4} & \textblue{1} & \textblue{6} & \textblue{.5} &   & \textblue{*} \\
  \hline
  f &  ? &  ? &  ? & \textblue{*} & \textblue{*} &   
\end{array}
$$

At the first step of the algorithm, elements $d$ and $e$ are gathered
into a new element $f$.

\begin{itemize}
\item What are the distances $d(\cdot, f)$ ?
\item What is the weight of $f$ ?
\end{itemize}
\refer{Bouroche \& Saporta (98)}{L'analyse des donn�es}

\newpage
\paragraph{Different methods provide different trees:} 
$$
\begin{tabular}{c||c||c}
  Single linkage & Average linkage & Complete linkage \\
                                % LIEN MINIMUM
  \begin{pspicture}(4, 5)(-1, -1)
                                % Ordonn�es   
    \psline(0, 0)(0, 4.25)
    \psline(-0.25, 0)(0, 0) \rput[B]{0}(-0.5, -0.1){0}
    \psline(-0.25, 0.5)(0, 0.5) 
    \psline(-0.25, 1)(0, 1) \rput[B]{0}(-0.5, 0.9){2}
    \psline(-0.25, 1.5)(0, 1.5) 
    \psline(-0.25, 2)(0, 2) \rput[B]{0}(-0.5, 1.9){4}
    \psline(-0.25, 2.5)(0, 2.5) 
    \psline(-0.25, 3)(0, 3) \rput[B]{0}(-0.5, 2.9){6}
    \psline(-0.25, 3.5)(0, 3.5) 
    \psline(-0.25, 4)(0, 4) \rput[B]{0}(-0.5, 3.9){8}
                                % Abscisses
    \psline(0, 0)(4, 0)
    \rput[B]{0}(0.5, -0.75){$d$}
    \rput[B]{0}(1.25, -0.75){$e$}
    \rput[B]{0}(2, -0.75){$b$}
    \rput[B]{0}(2.75, -0.75){$c$}
    \rput[B]{0}(3.5, -0.75){$a$}
                                % Arbre
    \psline[linestyle=dashed](0, 0.25)(0.5, 0.25)
    \psline[linewidth=.075](0.5, 0)(0.5, 0.25)(1.25, 0.25)(1.25, 0)
    \psline[linestyle=dashed](0, 0.5)(2, 0.5)
    \psline[linewidth=.075](0.875, 0.25)(0.875, 0.5)(2, 0.5)(2, 0)
    \psline[linestyle=dashed](0, 1)(1.44, 1)
    \psline[linewidth=.075](1.44, 0.5)(1.44, 1)(2.75, 1)(2.75, 0)
    \psline[linestyle=dashed](0, 1.5)(2.1, 1.5)
    \psline[linewidth=.075](2.1, 1)(2.1, 1.5)(3.5, 1.5)(3.5, 0)
    \psline[linewidth=.075](2.8, 1.5)(2.8, 1.75)
  \end{pspicture}
  &
                                % LIEN MOYEN
  \begin{pspicture}(4, 5)(-1, -1)
                                % Ordonn�es   
    \psline(0, 0)(0, 4.25)
    \psline(-0.25, 0)(0, 0) \rput[B]{0}(-0.5, -0.1){0}
    \psline(-0.25, 0.5)(0, 0.5) 
    \psline(-0.25, 1)(0, 1) \rput[B]{0}(-0.5, 0.9){2}
    \psline(-0.25, 1.5)(0, 1.5) 
    \psline(-0.25, 2)(0, 2) \rput[B]{0}(-0.5, 1.9){4}
    \psline(-0.25, 2.5)(0, 2.5) 
    \psline(-0.25, 3)(0, 3) \rput[B]{0}(-0.5, 2.9){6}
    \psline(-0.25, 3.5)(0, 3.5) 
    \psline(-0.25, 4)(0, 4) \rput[B]{0}(-0.5, 3.9){8}
                                % Abscisses
    \psline(0, 0)(4, 0)
    \rput[B]{0}(0.5, -0.75){$d$}
    \rput[B]{0}(1.25, -0.75){$e$}
    \rput[B]{0}(2, -0.75){$b$}
    \rput[B]{0}(2.75, -0.75){$a$}
    \rput[B]{0}(3.5, -0.75){$c$}
                                % Arbre
    \psline[linestyle=dashed](0, 0.25)(0.5, 0.25)
    \psline[linewidth=.075](0.5, 0)(0.5, 0.25)(1.25, 0.25)(1.25, 0)
    \psline[linestyle=dashed](0, 1.25)(0.875, 1.25)
    \psline[linewidth=.075](0.875, 0.25)(0.875, 1.25)(2, 1.25)(2, 0)
    \psline[linestyle=dashed](0, 1.625)(1.44, 1.625)
    \psline[linewidth=.075](1.44, 1.25)(1.44, 1.625)(2.75, 1.625)(2.75, 0)
    \psline[linestyle=dashed](0, 2.75)(2.1, 2.75)
    \psline[linewidth=.075](2.1, 1.625)(2.1, 2.75)(3.5, 2.75)(3.5, 0)
    \psline[linewidth=.075](2.8, 2.75)(2.8, 3)
  \end{pspicture}
  &
                                % LIEN MAXIMAL
  \begin{pspicture}(4, 5)(-1, -1)
                                % Ordonn�es   
    \psline(0, 0)(0, 4.25)
    \psline(-0.25, 0)(0, 0) \rput[B]{0}(-0.5, -0.1){0}
    \psline(-0.25, 0.5)(0, 0.5) 
    \psline(-0.25, 1)(0, 1) \rput[B]{0}(-0.5, 0.9){2}
    \psline(-0.25, 1.5)(0, 1.5) 
    \psline(-0.25, 2)(0, 2) \rput[B]{0}(-0.5, 1.9){4}
    \psline(-0.25, 2.5)(0, 2.5) 
    \psline(-0.25, 3)(0, 3) \rput[B]{0}(-0.5, 2.9){6}
    \psline(-0.25, 3.5)(0, 3.5) 
    \psline(-0.25, 4)(0, 4) \rput[B]{0}(-0.5, 3.9){8}
                                % Abscisses
    \psline(0, 0)(4, 0)
    \rput[B]{0}(0.5, -0.75){$d$}
    \rput[B]{0}(1.25, -0.75){$e$}
    \rput[B]{0}(2, -0.75){$b$}
    \rput[B]{0}(2.75, -0.75){$a$}
    \rput[B]{0}(3.5, -0.75){$c$}
                                % Arbre 
    \psline[linestyle=dashed](0, 0.25)(0.5, 0.25)
    \psline[linewidth=.075](0.5, 0)(0.5, 0.25)(1.25, 0.25)(1.25, 0)
    \psline[linestyle=dashed](0, 2)(0.875, 2)
    \psline[linewidth=.075](0.875, 0.25)(.875, 2)(2.325, 2)(2.325, 1.5)
    \psline[linestyle=dashed](0, 3.5)(1.6, 3.5)
    \psline[linewidth=.075](1.6, 2)(1.6, 3.5)(3.5, 3.5)(3.5, 0)
    \psline[linestyle=dashed](0, 1.5)(2, 1.5)
    \psline[linewidth=.075](2, 0)(2, 1.5)(2.75, 1.5)(2.75, 0) 
    \psline[linewidth=.075](2.55, 3.5)(2.55, 3.75)
  \end{pspicture}
  \\
                                % LIEN MINIMUM
  $\begin{array}{ccccc}
    a & b & c & d & e \\
    \hline
    0  &  3  &  3  &  3  &  3  \\
    &  0  &  2  &  1  &  1  \\
    &     &  0  &  2  &  2  \\
    &     &     &     & 0.5 \\
    &     &     &     &  0    
  \end{array}$
  &
                                % LIEN MOYEN
  $\begin{array}{ccccc}
    a & b & c & d & e \\
    \hline
    0  & 3.25 & 5.5 & 3.25 & 3.25 \\
    &  0  & 5.5 & 2.5 & 2.5 \\
    &     &  0  & 5.5 & 5.5 \\
    &     &     &  0  & 0.5 \\
    &     &     &     &  0  \\
  \end{array}$
  &
                                % LIEN MAXIMAL
  $\begin{array}{ccccc}
    a & b & c & d & e \\
    \hline
    0  &  3  &  7  &  4  &  4   \\
    &  0  &  7  &  4  &  4   \\
    &     &  0  &  7  &  7    \\
    &     &     &     & 0.5 \\
    &     &     &     &  0  
  \end{array}$
\end{tabular}
$$

\newpage
\paragraph{Comparison of trees:} 
Denoting $d$ the original distance and $\hat{d}$ the distance in the
tree, the ``cophenetic'' correlation is the correlation between $d$
and $\hat{d}$:
$$
\rho(d, \hat{d}) = \frac{\sum_{i<j}
  \left[d(i,j)-\bar{d}\right]\left[\hat{d}(i,j)-\bar{\hat{d}}\right]}{\sqrt{\sum_{i<j}
    \left[d(i,j)-\bar{d}\right]^2 \sum_{i<j}
    \left[\hat{d}(i,j)-\bar{\hat{d}}\right]^2}}.
$$
In the toy example we have
$$
\begin{tabular}{lcccc}
  & original & \begin{tabular}{c}simple\\ linkage\end{tabular} 
  & \begin{tabular}{c}average\\ linkage\end{tabular}  
  & \begin{tabular}{c}complete\\ linkage\end{tabular}  \\
  \hline
  \begin{tabular}{l}mean\\ distance\end{tabular} & 3.45 & 2.05 & 3.73
  & 4.75 \\
  \\
  \begin{tabular}{l}standard\\ deviation\end{tabular} & 2.03 & 0.96 &
  1.72 & 2.20 \\
  \\
  \begin{tabular}{l}phenotic\\ correlation\end{tabular}  & 1 &  0.54 & 0.67 & 0.65
\end{tabular}
$$

\newpage
\hspace{-1cm}
\begin{tabular}{cc}
  \begin{tabular}{l}
    \subsection{Example} \\
    \\
    \textblue{Clustering genes / } \\
    \\
    \textblue{clustering patients:} \\
    \\ \\ \\
    rows = genes \\ 
    \\
    columns = patients\\
    (blue = BLBCL) \\
    \\ \\ \\
    \refer{Alizadeh \& al. (00)}{Distinct Type of Diffuse Large
    B-cell Lymphoma ... } \\   
  \end{tabular}
  & 
  \begin{tabular}{l}
  \epsfig{figure=../Figures/AED00-Fig1.ps, height=18cm, width=15cm, bbllx=85, bblly=115,
    bburx=510, bbury=570, clip=} 
  \end{tabular}
\end{tabular} 

\newpage 
BLBCL subgroups respond differentially to therapy:

\begin{tabular}{cc}
  \begin{tabular}{l}
  \epsfig{figure=../Figures/AED00-Fig3.ps, height=15cm, width=10cm, bbllx=45, bblly=170,
    bburx=290, bbury=480, clip=} 
  \end{tabular}
  & 
  \begin{tabular}{r}
    \refer{Alizadeh \& al. (00)}{} \\
    \\ \\
  \epsfig{figure=../Figures/AED00-Fig5.ps, height=10cm, width=10cm, bbllx=370, bblly=80,    bburx=490, bbury=205, clip=} 
  \end{tabular}
\end{tabular} \\

%\epsfig{figure=../Figures/AEDM00-p5.ps, height=6cm, width=18cm, bbllx=47, bblly=165,
%  bburx=290, bbury=260, clip=} 


%%%%%%%%%%%%%%%%%%%%%%%%%%%%%%%%%%%%%%%%%%%%%%%%%%
\newpage
\section{$K$ means} 
%%%%%%%%%%%%%%%%%%%%%%%%%%%%%%%%%%%%%%%%%%%%%%%%%%
The number $K$ of group is a priori known 

\paragraph{Start:} (In general) choose $K$ points at random among $\xbf_1
\dots \xbf_G$ that become the means $\mbf^0_1 \dots \mbf^0_K$ of the
$K$ groups

\paragraph{Step $n$:} \\
$a) \quad$ Affect each element $g$ to the closest group $\Ccal^n_k$:
$$
d(\xbf_g, \mbf^n_k) = \min_{k'} d(\xbf_g, \mbf^n_{k'})
$$
$b) \quad$ Update the mean of each group:
$$
\mbf^{n+1}_k = \frac{1}{|\Ccal_k|}\sum_{g \in \Ccal_k} \xbf_g
$$

\newpage
\paragraph{Property:} The within-group inertia decreases at each step\\
Let $I_n$ denote it at step $n$:
$$
I_n = \sum_{k=1}^K \sum_{g \in \Ccal^n_k} d^2\left(\xbf_g - \mbf^n_k\right)^2
$$
$a) \quad$ If element $g$ moves from group $\Ccal_k$ to group $\Ccal_{k'}$
then
$$
d^2\left(\xbf_g - \mbf^n_{k'}\right)^2 \leq d^2\left(\xbf_g - \mbf^n_k\right)^2
$$
$b) \quad$ Since $\mbf^{n+1}$ is the mean of group $\Ccal^{n+1}_k$
$$
\sum_{g \in \Ccal^{n+1}_k} d^2\left(\xbf_g - \mbf^{n+1}_k\right)^2
\leq
\sum_{g \in \Ccal^{n+1}_k} d^2\left(\xbf_g - \mbf^n_k\right)^2
$$

The $K$ means algorithm converges but some groups may empty.

\newpage
\paragraph{Drawback:} 
Clusters provided by the $K$ means algorithm are very sensitive to the
starting points $\mbf^0_1 \dots \mbf^0_K$

\hspace{-1cm}
\begin{tabular}{ccc}
  Selection 1 & Selection 2 & Selection 3 \\
  \epsfig{file = ../Figures/Kmeans-tra1.eps, height = 5cm,width = 7.5cm}
  & \epsfig{file = ../Figures/Kmeans-tra3.eps, height = 5cm,width = 7.5cm}
  & \epsfig{file = ../Figures/Kmeans-tra2.eps, height = 5cm,width = 7.5cm} 
\end{tabular}


\paragraph{In practice:} Use $K$ mean 
\begin{itemize}
\item when you have good prior informations to choose
  $\mbf^0_1 \dots \mbf^0_K$ 
\item after a hierarchical clustering to allow small changes 
\item  at least, try several starting points
\end{itemize}

%%%%%%%%%%%%%%%%%%%%%%%%%%%%%%%%%%%%%%%%%%%%%%%%%%
\newpage
\chapter{Mixture models}
%%%%%%%%%%%%%%%%%%%%%%%%%%%%%%%%%%%%%%%%%%%%%%%%%%

Statistical modeling of the data $\Longrightarrow$ the variability is
considered

%%%%%%%%%%%%%%%%%%%%%%%%%%%%%%%%%%%%%%%%%%%%%%%%%%
\bigskip
\section{General model} 

Each element $g$ has probability $\pi_k$ to belong to group $\Ccal_k$
\\
If element $g$ belongs to group $\Ccal$, then
$$
(\Xbf_g \;|\; g \in \Ccal_k) \sim \phi(\cdot; \theta_k)
\qquad \Leftrightarrow \qquad
\Xbf_g \sim \sum_k \pi_k \phi(\cdot; \theta_k)
$$
where $\phi$ is a distribution parameterized by $\theta$.

\paragraph{``Complete'' data:} 
In an ideal world, we should observe an i.i.d. sample $\{(\Xbf_1,
Z_1)$, \dots, $(\Xbf_G, Z_G)\}$ with distribution
$$
Z_g \sim \Mcal(1; \pi_1, \dots, \pi_K), 
\qquad
(\Xbf_g \; | \; Z_g = k)  \sim \phi(\cdot; \theta_k)
$$

\newpage 
\hspace{-2cm}
\begin{tabular}{c|c|c}
  the model & what we would like & what we have \\
  & & \\
  \epsfig{file = ../Figures/Melange-modele.eps, height = 8cm,width = 8cm,
    clip=} 
  & \epsfig{file = ../Figures/Melange-complet.eps, height = 8cm,width = 8cm,
    bbllx=90, bblly=224, bburx=540, bbury=580, clip=} 
  & \epsfig{file = ../Figures/Melange-incomplet.eps, height = 8cm,width = 8cm,
    bbllx=90, bblly=224, bburx=540, bbury=580, clip=} 
  \\
  & & \\
  & complete data & incomplete data \\
  & $Z: 1 = \textblue{\bf \circ}, 2 = \textred{+}, 3 = {\bf *}$
  & $Z =$ ?
\end{tabular}

%%%%%%%%%%%%%%%%%%%%%%%%%%%%%%%%%%%%%%%%%%%%%%%%%%
\newpage
\subsection{Likelihoods}


\paragraph{Incomplete likelihood:} data $\{\Xbf_g\}$ 
$$
\log \Lcal\left(\{\Xbf_g\}; \{\pi_k, \theta_k\}\right) = \sum_g
\log \left[\sum_k \pi_k \phi(\Xbf_g; \theta_k)\right]
$$

\paragraph{Complete likelihood:} data $\{(\Xbf_g, Z_g)\}$ \\
\begin{equation*}
  \begin{split}
    \log \Lcal\left(\{\Xbf_g, Z_g\}; \{\pi_k, \theta_k\}\right) & =
    \sum_k \sum_{Z_g = k} \log\left[\pi_k \phi(\Xbf_g;
      \theta_k)\right]
    \\
    \\
    & \begin{array}{ccc}
      = \displaystyle{\sum_g} & \underbrace{\Ibb\{Z_g =  k\}} 
      & \underbrace{\log\left[\pi_k \phi(\Xbf_g;\theta_k)\right]}
    \\
    & \mbox{Expectation} & \mbox{Maximization} \\
    & \mbox{(E) step} & \mbox{(M) step}
    \end{array}
  \end{split}
\end{equation*}
    

%%%%%%%%%%%%%%%%%%%%%%%%%%%%%%%%%%%%%%%%%%%%%%%%%%
\newpage
\subsection{Estimation with the EM algorithm} 

\paragraph{E step:} Estimates missing data $\{Z_g\}$ 
denoting $\{\pi^n_k, \theta^n_k\}$ the estimates at step $n$, the
\textblue{posterior probability} to belong to group $k$ is
$$
p^n_g(k) = \Pr\left\{Z_g = k \;|\; \Xbf_g, \{\pi^n_k,
  \theta^n_k\}\right\}
= 
\frac{\pi^n_k \phi(\Xbf_g; \theta^n_k)}{\sum_{k'} \pi^n_{k'}
  \phi(\Xbf_g; \theta^n_{k'})}
$$

Several possibilities 
$$
\begin{tabular}{ll}
  $a)$ deterministic (CEM): & $\widehat{Z}^n_g =
  \underset{k}{\arg\max} \left[p^n_g(k)\right]$ \\
  \\
  $b)$ stochastic (SEM): & $\widehat{Z}^n_g \sim \Mcal[p^n_g(1), \dots,
  p^n_g(K)]$ \\
  \\
  $c)$ weighted (EM): & use $p^n_g(k) =
  \widehat{\Esp}^n\left(\Ibb\{Z_g=k\} \;|\; \Xbf_g\right)$ \\
\end{tabular}
$$

\newpage
\paragraph{Posterior probabilities:} 
$$
  \begin{tabular}{cc}
    Distributions: & Posterior probabilities: \\
    \\
    $f(x) = \textred{\pi_1 f_1(x)} + \textgreen{\pi_2 f_2(x)} +
    \textblue{\pi_3 f_3(x)}$ & $\tau_{gk} = \Pr\{g \in f_k \;|\; x_g\}
    = \pi_k f_k(x_g) / f(x_g)$\\
    \\
    \epsfig{file=../Figures/Melange-densite.ps, height=6cm,
      width=12cm, bbllx=77, bblly=328, bburx=549, bbury=528, clip=}
    &
    \epsfig{file=../Figures/Melange-posteriori.ps, height=6cm, width=12cm,
      bbllx=83, bblly=320, bburx=549, bbury=537, clip=} 
  \end{tabular}
$$

% \centerline{
%   \begin{pspicture}(20, 11)
%     \rput[B](8, 0){$x_1$}
%     \rput[B](12.2, 0){$x_2$}
%     \rput[B](17, 0){$x_3$}
%     \rput[bl](0, 1){\epsfig{file=../Figures/Melange-densite.ps, height=10cm,
%         width=20cm, bbllx=77, bblly=328, bburx=549, bbury=528, clip=}}
%   \end{pspicture}
% }

\centerline{$
  \begin{array}{cccc}
    \quad p_g(k)~~(\%) \quad & \qquad g=1 \qquad & \qquad g=2 \qquad &
    \qquad g=3 \qquad \\
    \hline
    k = 1 & \textred{65.8} & 0.7 & 0.0 \\
    k = 2 & 34.2 & 47.8 & 0.0 \\
    k = 3 & 0.0 & \textblue{51.5} & \textblue{1.0} \\
  \end{array}
$}

\newpage
\paragraph{M step:} Maximize $\Lcal(\{\Xbf_g \; | \;Z_g=k\})$
(separately for each $k$)
$$
\log \Lcal(\{\Xbf_g \; | \;Z_g=k\}) 
=
\sum_g \Ibb\{Z_g=k\} \log\left[\phi(\Xbf_g; \theta_k)\right]
$$
$$
\begin{array}{lrcl}
  a) \mbox{ or } b): \qquad & 
  \theta^{n+1}_k & = &
  \displaystyle{\underset{\theta_k}{\arg\max}\sum_{\widehat{Z}^n_g = k}
  \log\left[\phi(\Xbf_g; \theta_k)\right]} \\
  & \pi^{n+1}_k & = & \displaystyle{\frac{1}{G} \sum_g \widehat{Z}^n_g} \\
  \\
  c): \qquad & 
  \theta^{n+1}_k & = & \displaystyle{\underset{\theta_k}{\arg\max} \sum_g
  p^n_g(k) + \log\left[\phi(\Xbf_g; \theta_k)\right]}
  \\
  & \pi^{n+1}_k & = & \displaystyle{\frac{1}{G} \sum_g p^n_g(k)}
\end{array}
$$

%%%%%%%%%%%%%%%%%%%%%%%%%%%%%%%%%%%%%%%%%%%%%%%%%%
\newpage
\paragraph{Univariate Gaussian mixture:}
$ \theta_k = (\mu_k, \sigma^2_k), \quad \phi_k = \Ncal(\mu_k,
\sigma^2_k) $
\begin{equation*}
  \begin{split}
    \phi(X_k; \mu_k, \sigma^2_k) = &
    \frac{1}{\sigma_k \sqrt{2 \pi}}\exp\left[(X_k - \mu_k)^2 / (2 \sigma^2_k)\right] \\
    \mu^{n+1}_k = &
    \frac{1}{\sum_g p^n_g(k) } \sum_g p^n_g(k) X_g \\
    (\sigma^2_k)^{n+1} = &
    \frac{1}{\sum_g p^n_g(k) } \sum_g p^n_g(k) \left(X_g - \mu^{n+1}_k\right)^2\\
  \end{split}
\end{equation*}

\paragraph{$K$ means:} $\pi_k \equiv 1/K$, $\sigma^2_k \equiv 
\sigma^2$, E step = CEM
$$
\underset{k}{\arg\max} \left[\pi_k \phi(X_g; \mu_k, \sigma^2_k)\right] =
\underset{k}{\arg\min} (X_g - \mu_k)^2
$$
multivariate version ($\Sigmabf = \sigma^2 \Ibf$):
$$
\underset{k}{\arg\max} \left[\pi_k \phi(\Xbf_g; \mubf_k, \Sigmabf)\right] =
\underset{k}{\arg\min} \|\Xbf_g - \mubf\|^2
$$

\newpage
\subsection{Simulated data} 

2 groups with different means $\mubf_k$ and variances $\Sigmabf_k$

\paragraph{Posterior probabilities $\tau_{g1}$:} 
$$
\epsfig{figure=../Figures/FigMixtureGauss2.eps,height=15cm,width=12cm,
  clip=, bbllx=340, bblly=75, bburx=555, bbury=350, angle=90} 
$$

\newpage
%\vspace{2cm}
\paragraph{Comparison Mixture model / $K$ means:}
$$
%\hspace{-1cm}
\begin{tabular}{cc}
  MAP classification & $K$-mean classification \\
  \epsfig{figure=../Figures/FigMixtureGauss2.eps,height=12.5cm,width=12.5cm,
    clip=, bbllx=65, bblly=420, bburx=285, bbury=690, angle=90} 
  & \epsfig{figure=../Figures/FigMixtureGauss2.eps,height=12.5cm,width=12.5cm,
    clip=, bbllx=65, bblly=75, bburx=285, bbury=350, angle=90} 
\end{tabular}
$$
$K$ means classification would lead to biased estimates of both
$\mubf_k$ and $\Sigmabf_k$.
% \newpage
% \begin{tabular}{lc}
%   \begin{tabular}{p{10cm}}
%     \paragraph{Comparison on simulated data:} 
%     2 groups with different 
%     means $\mubf_k$ 
%     and variances $\Sigmabf_k$ 
%   \end{tabular}
%   &
%   \begin{tabular}{c}
%     \epsfig{figure=../Figures/FigMixtureGauss2.eps,height=8cm,width=6cm,
%       clip=, bbllx=340, bblly=75, bburx=555, bbury=350, angle=90} 
%   \end{tabular}
% \end{tabular} 

% \begin{tabular}{cc}
%   MAP classification & $K$-mean classification \\
%   \epsfig{figure=../Figures/FigMixtureGauss2.eps,height=10cm,width=10cm,
%     clip=, bbllx=65, bblly=420, bburx=285, bbury=690, angle=90} 
%   & \epsfig{figure=../Figures/FigMixtureGauss2.eps,height=10cm,width=10cm,
%     clip=, bbllx=65, bblly=75, bburx=285, bbury=350, angle=90} 
% \end{tabular}


\newpage
\paragraph{Properties of EM:}
\begin{itemize}
\item always converges (at each step, $\log \Lcal$
  increases)
\item but not necessarily to the true maximum
\end{itemize}

\paragraph{Practical use of EM:} 
\begin{itemize}
\item easy to compute 
\item  sometimes slow to converge
\item  try several starting points
\end{itemize}

\refer{McLahlan \& Peel (00)}{}

%\comment{Autre ref : cf FP}

%%%%%%%%%%%%%%%%%%%%%%%%%%%%%%%%%%%%%%%%%%%%%%%%%%
\newpage
\subsection{Choice of the number of groups}

\paragraph{Likelihood ratio test:}
\begin{itemize}
\item the $K$-group model is nested in the ($K+1$)-group model:
  $\Lcal_{K+1} \geq \Lcal_K$
\item but LRT is not valid (border of the parameter space)
\end{itemize}

\paragraph{Penalized likelihood criterion:} 
\begin{eqnarray*}
AIC & = & -2 \log \Lcal + 2 \times \mbox{number of parameters} \\
BIC & = & -2 \log \Lcal + \log G \times \mbox{number of of parameters}
\end{eqnarray*}
\refer{Burnahm \& Anderson}{}

\paragraph{MCMC methods:} \\
estimate the parameters and the number $K$ as any parameter using a
random walk over the parameter space (reversible jump algorithm)

%%%%%%%%%%%%%%%%%%%%%%%%%%%%%%%%%%%%%%%%%%%%%%%%%%
\newpage
\section{Mixture models for time course experiments} \qquad
\qquad \refer{Luan \& Li (03)}{}

\paragraph{Data:}  $X_{gt} = $ expression level of gene $g$ ($=1..G$)
at time $t$ ($=1..T$).

\paragraph{Idea:} Each cluster of genes is characterized by a ``mean''
profile.

\paragraph{Trick:} Mean profiles and clusters can be estimated at
simultaneously. 
 
\paragraph{Model:} denote $\Xbf_g = (X_{g1}, \dots, X_{gT})$ the
  espression profile of gene $g$
$$
\Xbf_g \;|\; g \in \Ccal_k \sim \Ncal_T
$$
where 
\begin{itemize}
\item $\Esp(X_{gt}) = F_k(t)$ having some general form (polynomial,
  B-spline, etc), 
\item $\Var(X_{gt}) = \sigma^2_k$ (constant variance), 
\item $\Cov(X_{gt}, X_{gs}) = \sigma^2 \delta(|s-t|)$, with $\delta$
  some decreasing function.
\end{itemize}

\newpage
\paragraph{Application to yeast cell cycle data} \refer{Spellmann \&
  al. (98)}{} 

\hspace{-2cm}
\begin{tabular}{cc}
  \begin{tabular}{p{12cm}}
    \epsfig{figure=../Figures/LuL03-Fig2.ps, height=12cm, width=12cm, 
      bbllx=120, bblly=460, bburx=470, bbury=750,
      clip=} \\
    BIC selects $K=7$ groups (top left) \\
    \\
    Other plots = group mean profiles
  \end{tabular}
  &
  \begin{tabular}{p{12cm}}
    \epsfig{figure=../Figures/LuL03-Tab2.ps, height=8cm, width=12cm, 
      bbllx=37, bblly=616, bburx=282, bbury=757,
      clip=} \\
    \\
  Some of them are consistent with biologically relevent groups
  defined independently
  ~\\
  ~\\
  ~\\
  ~\\
  \end{tabular} 
\end{tabular}

\newpage
\paragraph{Application to human fibroblast data} \refer{Iyer \&
  al. (99)}{} 
\begin{itemize}
\item BIC again select 7 groups.
\item Posterior probabilities measure the confidence with which
  genes are classified into groups:
  $$
  \epsfig{figure=../Figures/LuL03-Fig3.ps, height=8cm, width=16cm, 
    bbllx=306, bblly=479, bburx=464, bbury=548,
    clip=} \\
  $$
\end{itemize}

%%%%%%%%%%%%%%%%%%%%%%%%%%%%%%%%%%%%%%%%%%%%%%%%%%%%%%%%%%%%%%%%%%%%%%
%%%%%%%%%%%%%%%%%%%%%%%%%%%%%%%%%%%%%%%%%%%%%%%%%%%%%%%%%%%%%%%%%%%%%%
%%%%%%%%%%%%%%%%%%%%%%%%%%%%%%%%%%%%%%%%%%%%%%%%%%%%%%%%%%%%%%%%%%%%%%
%%%%%%%%%%%%%%%%%%%%%%%%%%%%%%%%%%%%%%%%%%%%%%%%%%%%%%%%%%%%%%%%%%%%%%
\end{document}
%%%%%%%%%%%%%%%%%%%%%%%%%%%%%%%%%%%%%%%%%%%%%%%%%%%%%%%%%%%%%%%%%%%%%%
%%%%%%%%%%%%%%%%%%%%%%%%%%%%%%%%%%%%%%%%%%%%%%%%%%%%%%%%%%%%%%%%%%%%%%
%%%%%%%%%%%%%%%%%%%%%%%%%%%%%%%%%%%%%%%%%%%%%%%%%%%%%%%%%%%%%%%%%%%%%%
%%%%%%%%%%%%%%%%%%%%%%%%%%%%%%%%%%%%%%%%%%%%%%%%%%%%%%%%%%%%%%%%%%%%%%
