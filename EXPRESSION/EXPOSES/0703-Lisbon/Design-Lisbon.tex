\documentclass[dvips, lscape]{foils}
%\documentclass[dvips, french]{slides}
\textwidth 18.5cm
\textheight 25cm 
\topmargin -1cm 
\oddsidemargin  -1cm 
\evensidemargin  -1cm

% Maths
\usepackage{amsfonts, amsmath, amssymb}

\newcommand{\coefbin}[2]{\left( 
    \begin{array}{c} #1 \\ #2 \end{array} 
  \right)}
\newcommand{\bbullet}{\bullet\bullet}
\newcommand{\bbbullet}{\bbullet\bullet}
\newcommand{\bbbbullet}{\bbbullet\bullet}
\newcommand{\Bcal}{\mathcal{B}}
\newcommand{\Ccal}{\mathcal{C}}
\newcommand{\Dcal}{\mathcal{D}}
\newcommand{\Ecal}{\mathcal{E}}
\newcommand{\Mcal}{\mathcal{M}}
\newcommand{\Ncal}{\mathcal{N}}
\newcommand{\Pcal}{\mathcal{P}}
\newcommand{\Lcal}{\mathcal{L}}
\newcommand{\Tcal}{\mathcal{T}}
\newcommand{\Ucal}{\mathcal{U}}
\newcommand{\alphabf}{\mbox{\mathversion{bold}{$\alpha$}}}
\newcommand{\betabf}{\mbox{\mathversion{bold}{$\beta$}}}
\newcommand{\gammabf}{\mbox{\mathversion{bold}\newcommand{\psibf}{\mbox{\mathversion{bold}{$\psi$}}}
{$\gamma$}}}
\newcommand{\mubf}{\mbox{\mathversion{bold}{$\mu$}}}
\newcommand{\psibf}{\mbox{\mathversion{bold}{$\psi$}}}
\newcommand{\Sigmabf}{\mbox{\mathversion{bold}{$\Sigma$}}}
\newcommand{\taubf}{\mbox{\mathversion{bold}{$\tau$}}}
\newcommand{\thetabf}{\mbox{\mathversion{bold}{$\theta$}}}
\newcommand{\Dbf}{{\bf D}}
\newcommand{\Ebf}{{\bf E}}
\newcommand{\Hbf}{{\bf H}}
\newcommand{\Ibf}{{\bf I}}
\newcommand{\Sbf}{{\bf S}}
\newcommand{\mbf}{{\bf m}}
\newcommand{\ubf}{{\bf u}}
\newcommand{\vbf}{{\bf v}}
\newcommand{\xbf}{{\bf x}}
\newcommand{\Xbf}{{\bf X}}
\newcommand{\Esp}{{\mathbb E}}
\newcommand{\Corr}{{\mathbb C}\mbox{orr}}
\newcommand{\Var}{{\mathbb V}}
\newcommand{\Ibb}{{\mathbb I}}
\newcommand{\Rbb}{\mathbb{R}}

% Couleur et graphiques
\usepackage{color}
\usepackage{graphics}
\usepackage{epsfig} 
\usepackage{pstcol}

% Texte
\usepackage{lscape}
\usepackage{../../../../Latex/fancyheadings, rotating, enumerate}
%\usepackage[french]{babel}
\usepackage[latin1]{inputenc}
%\definecolor{darkgreen}{cmyk}{0.5, 0, 0.5, 0.5}
%\definecolor{green}{cmyk}{0.5, 0, 0.5, 0.5}
\definecolor{orange}{cmyk}{0, 0.6, 0.8, 0}
\definecolor{jaune}{cmyk}{0, 0.5, 0.5, 0}
\newcommand{\textblue}[1]{\textcolor{blue}{#1}}
\newcommand{\textred}[1]{\textcolor{red}{#1}}
\newcommand{\textgreen}[1]{\textcolor{green}{ #1}}
\newcommand{\textlightgreen}[1]{\textcolor{green}{#1}}
%\newcommand{\textgreen}[1]{\textcolor{darkgreen}{#1}}
\newcommand{\textorange}[1]{\textcolor{orange}{#1}}
\newcommand{\textyellow}[1]{\textcolor{yellow}{#1}}
\newcommand{\refer}[2]{{\sl #1}}

% Sections
%\newcommand{\chapter}[1]{\centerline{\LARGE \textblue{#1}}}
% \newcommand{\section}[1]{\centerline{\Large \textblue{#1}}}
% \newcommand{\subsection}[1]{\noindent{\Large \textblue{#1}}}
% \newcommand{\subsubsection}[1]{\noindent{\large \textblue{#1}}}
% \newcommand{\paragraph}[1]{\noindent {\textblue{#1}}}
% Sectionsred
\newcommand{\chapter}[1]{
  \addtocounter{chapter}{1}
  \setcounter{section}{0}
  \setcounter{subsection}{0}
  {\centerline{\LARGE \textblue{\arabic{chapter} - #1}}}
  }
\newcommand{\section}[1]{
  \addtocounter{section}{1}
  \setcounter{subsection}{0}
  {\centerline{\Large \textblue{\arabic{chapter}.\arabic{section} - #1}}}
  }
\newcommand{\subsection}[1]{
  \addtocounter{subsection}{1}
  {\noindent{\large \textblue{\arabic{chapter}.\arabic{section}.\arabic{subsection} - #1}}}
  }
\newcommand{\paragraph}[1]{\noindent{\textblue{#1}}}

%%%%%%%%%%%%%%%%%%%%%%%%%%%%%%%%%%%%%%%%%%%%%%%%%%%%%%%%%%%%%%%%%%%%%%
%%%%%%%%%%%%%%%%%%%%%%%%%%%%%%%%%%%%%%%%%%%%%%%%%%%%%%%%%%%%%%%%%%%%%%
%%%%%%%%%%%%%%%%%%%%%%%%%%%%%%%%%%%%%%%%%%%%%%%%%%%%%%%%%%%%%%%%%%%%%%
%%%%%%%%%%%%%%%%%%%%%%%%%%%%%%%%%%%%%%%%%%%%%%%%%%%%%%%%%%%%%%%%%%%%%%
\begin{document}
%%%%%%%%%%%%%%%%%%%%%%%%%%%%%%%%%%%%%%%%%%%%%%%%%%%%%%%%%%%%%%%%%%%%%%
%%%%%%%%%%%%%%%%%%%%%%%%%%%%%%%%%%%%%%%%%%%%%%%%%%%%%%%%%%%%%%%%%%%%%%
%%%%%%%%%%%%%%%%%%%%%%%%%%%%%%%%%%%%%%%%%%%%%%%%%%%%%%%%%%%%%%%%%%%%%%
%%%%%%%%%%%%%%%%%%%%%%%%%%%%%%%%%%%%%%%%%%%%%%%%%%%%%%%%%%%%%%%%%%%%%%
\landscape
\newcounter{chapter}
\newcounter{section}
\newcounter{subsection}
\setcounter{chapter}{0}
\headrulewidth 0pt 
\pagestyle{fancy} 
\cfoot{}
\rfoot{\begin{rotate}{90}{
      \hspace{1cm} \tiny S. Robin: Designs for microarray experiments 
      }\end{rotate}}
\rhead{\begin{rotate}{90}{
      \hspace{-.5cm} \tiny \thepage
      }\end{rotate}}

%%%%%%%%%%%%%%%%%%%%%%%%%%%%%%%%%%%%%%%%%%%%%%%%%%%%%%%%%%%%%%%%%%%%%%
%%%%%%%%%%%%%%%%%%%%%%%%%%%%%%%%%%%%%%%%%%%%%%%%%%%%%%%%%%%%%%%%%%%%%%
\begin{center}
  \textblue{\LARGE Experimental designs for microarray experiments} 

   \vspace{1cm}
   {\large S. {Robin}} \\
   robin@inapg.inra.fr

   {UMR INA-PG / ENGREF / INRA, Paris} \\
   {Math�matique et Informatique Appliqu�es}
   
    \vspace{1cm}
    {Microarray Design and Statistical Analysis} \\
    {Lisbon, \today}
\end{center}

\paragraph{Outline.} 

\begin{tabular}{lll}
  1 - Introduction & \hspace{2.5cm} & 3 - Designs for two channels \\
  \\
  2 - Designs for one channel & & 4 - Example with dependent data \\
\end{tabular}




%%%%%%%%%%%%%%%%%%%%%%%%%%%%%%%%%%%%%%%%%%%%%%%%%%%%%%%%%%%%%%%%%%%%%%
%%%%%%%%%%%%%%%%%%%%%%%%%%%%%%%%%%%%%%%%%%%%%%%%%%%%%%%%%%%%%%%%%%%%%%
\newpage
\chapter{Introduction}
%%%%%%%%%%%%%%%%%%%%%%%%%%%%%%%%%%%%%%%%%%%%%%%%%%%%%%%%%%%%%%%%%%%%%%
%%%%%%%%%%%%%%%%%%%%%%%%%%%%%%%%%%%%%%%%%%%%%%%%%%%%%%%%%%%%%%%%%%%%%%

%%%%%%%%%%%%%%%%%%%%%%%%%%%%%%%%%%%%%%%%%%%%%%%%%%%%%%%%%%%%%%%%%%%%%%
\bigskip\bigskip
\section{A design, what for?}
%%%%%%%%%%%%%%%%%%%%%%%%%%%%%%%%%%%%%%%%%%%%%%%%%%%%%%%%%%%%%%%%%%%%%%

\paragraph{Aim of experimental designs:} Organize comparisons in order to 
\begin{itemize}
\item get precise results from a limited number of experiments,
\item avoid biases
\end{itemize}

The design depends on
\begin{itemize}
\item the model with which data will be analyzed (ex: anova)
\item the kind of information one wants to get (ex: contrasts)
\end{itemize}

%\paragraph{Independence of the experiments} is a crucial assumption. 

\refer{Churchill (02)}{Fundamentals of experimental designs for c{DNA}
  microarray}, \refer{Kerr \& Churchill (01)}{Experimental design of gene expression microarrays}

%%%%%%%%%%%%%%%%%%%%%%%%%%%%%%%%%%%%%%%%%%%%%%%%%%%%%%%%%%%%%%%%%%%%%%
\newpage
\paragraph{General approach.}
\begin{description}
\item[1 - Modeling:] translate biological questions into
  equations and parameters
  $$
  \rightarrow \mbox{ statistical model}
  $$
\item[2 - Constraints list:] number of slides, systematic biases to
  be controlled, {\it etc.}
  $$
  \rightarrow \mbox{ propose one or several designs}
  $$
\item[3 - Parameter estimates:] mimic how the data will be analyzed
  when they will be available
  $$
  \rightarrow \mbox{ evaluate the precision of the estimates with the
    proposed design}
  $$
\end{description}

%%%%%%%%%%%%%%%%%%%%%%%%%%%%%%%%%%%%%%%%%%%%%%%%%%%%%%%%%%%%%%%%%%%%%%
\newpage
\section{Statistical model}
%%%%%%%%%%%%%%%%%%%%%%%%%%%%%%%%%%%%%%%%%%%%%%%%%%%%%%%%%%%%%%%%%%%%%%

\paragraph{Measurement for one gene:} We denote
\begin{eqnarray*}
  X_{tr} & = & \mbox{expression level of the gene} \\
  & & \mbox{under condition $t$} \\
  & & \mbox{for the $r$-th replicate.}
\end{eqnarray*}

\paragraph{Model:}
$$
\begin{array}{ccccc}
  X_{tr} & = & \mu_t & + & E_{tr} \\
  \\
  \mbox{measurement} & = & \mbox{signal} & + & \mbox{noise}
\end{array}
$$
where
\begin{description}
\item[$\mu_t = $] mean expression level of the gene in condition $t$,
\item[$E_{tr} =$] residual term due to the random variability
  affecting the $r$th replicate 
\end{description}

%%%%%%%%%%%%%%%%%%%%%%%%%%%%%%%%%%%%%%%%%%%%%%%%%%%%%%%%%%%%%%%%%%%%%%
\newpage
\subsection{Definition of the variability}

\paragraph{Reference variability:}
The variability between replicates is measured by the variance of the
residual term $E_{tr}$
$$
\Var(E_{tr}) = \sigma^2
$$
\paragraph{Replicates = different individuals:}
$$
\Rightarrow  \quad \sigma = \mbox{'biological' variability}
$$
\paragraph{Replicates = different samples within a common individual:}
$$
\Rightarrow \quad \sigma = \mbox{'technological' variability}
$$

\paragraph{Assumption $\sigma = $ constant}
for all replicates and treatments \\
$\quad \Rightarrow \quad$ log-transform to stabilize it:
$$
\sqrt{\Var(X)} \propto \Esp(X) 
\qquad \Rightarrow \qquad 
\Var(\log X) \simeq \mbox{cst}.
$$
Other typical transform
$$
\Var(X) \propto \Esp(X) 
\qquad \Rightarrow \qquad 
\Var(\sqrt{X}) \simeq \mbox{cst}.
$$

%%%%%%%%%%%%%%%%%%%%%%%%%%%%%%%%%%%%%%%%%%%%%%%%%%%%%%%%%%%%%%%%%%%%%%
\newpage
\subsection{One basic statistical result}

Suppose we have $R$ replicates ($r = 1..R$). A natural estimate of
$\mu_t$ is
$$
\widehat{\mu}_t = \overline{X}_t = \frac1R \sum_r X_{tr}
$$
\paragraph{Variance of one measurement:} 
$$
\Var(X_{tr}) = \Var(\mu_t + E_{tr}) = \Var(E_{tr}) = \sigma^2.
$$
\paragraph{Variance of a mean:} 
If the replicates are independent, we have
$$
\Var(\widehat{\mu}_t) = \frac1{R^2} \sum_r \Var(X_{tr}) = \frac1{R^2}
R\sigma^2 = \frac{\sigma^2}R
\qquad
\Rightarrow
\qquad
\sqrt{\Var(\widehat{\mu}_t)} = \frac{\sigma}{\sqrt{R}}.
$$
The precision of the mean increases as the \textblue{square root of
  the number of replicates}.

%%%%%%%%%%%%%%%%%%%%%%%%%%%%%%%%%%%%%%%%%%%%%%%%%%%%%%%%%%%%%%%%%%%%%%
%%%%%%%%%%%%%%%%%%%%%%%%%%%%%%%%%%%%%%%%%%%%%%%%%%%%%%%%%%%%%%%%%%%%%%
\newpage
\chapter{One channel designs}
%%%%%%%%%%%%%%%%%%%%%%%%%%%%%%%%%%%%%%%%%%%%%%%%%%%%%%%%%%%%%%%%%%%%%%
%%%%%%%%%%%%%%%%%%%%%%%%%%%%%%%%%%%%%%%%%%%%%%%%%%%%%%%%%%%%%%%%%%%%%%

\bigskip
\paragraph{Nylon membranes or Affymetrix slides:} 1 slide $\rightarrow$ one
measurement per gene.

%%%%%%%%%%%%%%%%%%%%%%%%%%%%%%%%%%%%%%%%%%%%%%%%%%%%%%%%%%%%%%%%%%%%%%
\bigskip\bigskip
\section{Comparing two conditions}
%%%%%%%%%%%%%%%%%%%%%%%%%%%%%%%%%%%%%%%%%%%%%%%%%%%%%%%%%%%%%%%%%%%%%%

\paragraph{Aim:} Comparing two conditions denoted $t = 1, 2$

\paragraph{Contrast of interest:} We want to estimate precisely the
difference 
$$
\delta = \mu_1 - \mu_2
$$

\paragraph{Contrast estimate:} 
$$
\widehat{\delta} = \widehat{\mu}_1 - \widehat{\mu}_2
$$

\paragraph{Variance of the contrast:} If $R_t$ independent replicates are
performed in condition $t$
$$
\Var(\widehat{\delta}) = \Var(\widehat{\mu}_1) + \Var(\widehat{\mu}_2)
= \left( \frac1{R_1} + \frac1{R_2} \right) \sigma^2
$$

%%%%%%%%%%%%%%%%%%%%%%%%%%%%%%%%%%%%%%%%%%%%%%%%%%%%%%%%%%%%%%%%%%%%%%
\newpage
\subsection{How many replicates?} 

The number of replicates is fixed according to some desirable property
of the results.

\paragraph{Precision of the estimated contrast.} 
One may fix a desired value $s^2$ of $\Var(\widehat{\delta})$.

If the number of replicates is the same in both conditions ($R_1 = R_2
= R$):
$$
\Var(\widehat{\delta}) = s^2 \qquad \Longrightarrow \qquad R = \frac{2\sigma^2}{s^2}
$$
\begin{enumerate}
\item $\sigma^2$ has to be known a priori, or $s^2$ should be
  expressed as a function of $\sigma^2$,
\item $s^2 = 0$ leads to $R = \infty$.
\end{enumerate}

%%%%%%%%%%%%%%%%%%%%%%%%%%%%%%%%%%%%%%%%%%%%%%%%%%%%%%%%%%%%%%%%%%%%%%
\newpage
\paragraph{Power of the test.}
One may fix a value of $\delta$ we want to detect with high
probability.

Using the $t$-test, we have \\
%\vspace{-5cm}
\centerline{
  \begin{pspicture}(25, 14)
    \rput[br](24, 13){unpaired data, $R = $ number of replicates
      $\rightarrow 2R$ data}
    \rput[bl](0, 0){ 
      \epsfig{figure=../Figures/PowerT.ps, height=15cm, width=25cm, clip=}    
      }
    \rput[B](14, 1){$\delta/\sigma$}
    \rput[B](16.25, 8){$R = 2$}
    \rput[B](10.75, 9){$4$}
    \rput[B](8.5, 9.5){$8$}
    \rput[B](4.75, 10){$64$}
  \end{pspicture}
}
\paragraph{Remark.} Again, $\delta$ is expressed as a function of $\sigma$

%%%%%%%%%%%%%%%%%%%%%%%%%%%%%%%%%%%%%%%%%%%%%%%%%%%%%%%%%%%%%%%%%%%%%%
\newpage
\section{Factorial designs}
%%%%%%%%%%%%%%%%%%%%%%%%%%%%%%%%%%%%%%%%%%%%%%%%%%%%%%%%%%%%%%%%%%%%%%

%%%%%%%%%%%%%%%%%%%%%%%%%%%%%%%%%%%%%%%%%%%%%%%%%%%%%%%%%%%%%%%%%%%%%%
\bigskip
\subsection{Anova model} 

We want to study the simultaneous effects of 2 factors:
\begin{itemize}
\item Type of patient: $t = $ sick / healthy,
\item Sex of the patients: $s = $ male / female. 
\end{itemize}
The expression level of the gene may be decomposed as
$$
X_{tsr} = \mu + \alpha_t + \beta_s + (\alpha\beta)_{ts} + E_{tsr}
$$
\begin{tabular}{rcl}
  where \qquad \qquad $\mu$ & $ = $ & constant term (typically, the global mean) \\
  $\alpha_t$ & $ = $ & main effect of type $t$ \\
  $\beta_s$ & $ = $ &  main effect of sex $s$ \\
  $(\alpha\beta)_{ts}$ & $ = $ & interaction effect between type $t$ and sex $s$ \\
  $E_{tsr}$ & $ = $ & residual term
\end{tabular}

%%%%%%%%%%%%%%%%%%%%%%%%%%%%%%%%%%%%%%%%%%%%%%%%%%%%%%%%%%%%%%%%%%%%%%
\newpage
\paragraph{Additive case: no interaction.} $\Esp(X_{tsr}) = \mu +
\alpha_t + \beta_s$:
$$
  \begin{pspicture}(0,1)(24, 16.5)
    
                                % axes
    \psline[linewidth=0.05]{->}(0, 2)(24, 2)
    \rput[br](24, 2.1){Factor $A$}
    \psline[linewidth=0.05]{->}(0, 2)(0, 16)
    \rput[bl](0.1, 15.5){Mean expression level}
    
                                % mu
    \psline[linewidth=0.05, linestyle=dashed]{-}(6, 9)(18, 9)
    \rput[bc](11.5, 9.25){$\mu$}

                                % alpha effect
    \psline[linewidth=0.1, linestyle=dashed, linecolor=blue]{-}(6, 7)(18, 11)
    \psline[linewidth=0.075, linecolor=blue]{<->}(6, 7)(6, 9) 
    \rput[br](5.75, 8){\textblue{$\alpha_2=-\alpha$}}
    \rput[bc](6, 6.75){\textblue{$\mu-\alpha$}}
    \psline[linewidth=0.075, linecolor=blue]{<->}(18, 11)(18, 9) 
    \rput[bl](18.25, 10){{$\alpha_1=+\alpha$}}
    \rput[bc](18, 11.25){{$\mu+\alpha$}}
    \psline[linewidth=0.05, linestyle=dotted]{-}(6, 2)(6, 10)
    \rput[bc](6, 1){Level $2$ $(-)$}
    \psline[linewidth=0.05, linestyle=dotted]{-}(18, 2)(18, 14)
    \rput[bc](18, 1){Level $1$ $(+)$}

                                % beta effect
    \psline[linewidth=0.1, linecolor=red]{-}(6, 4)(18, 8)
    \psline[linewidth=0.075, linecolor=red]{<->}(12, 9)(12, 6) 
    \rput[br](11.75, 7){\textred{$\beta_2=-\beta$}}
    \rput[bc](12, 5.75){{$\mu-\beta$}}
    \psline[linewidth=0.1, linecolor=red]{-}(6, 10)(18, 14)
    \psline[linewidth=0.075, linecolor=red]{<->}(12, 9)(12, 12) 
    \rput[bl](12.25, 10.5){\textred{$\beta_1=+\beta$}}
    \rput[bc](12, 12.25){{$\mu+\beta$}}

                                % additive case
    \rput[bc](18.25, 14.25){{$\mu+\alpha+\beta$}}
    \rput[bc](18.25, 8.25){{$\mu+\alpha-\beta$}}
    \rput[bc](5.75, 3.75){{$\mu-\alpha-\beta$}}
    \rput[bc](5.75, 10.25){{$\mu-\alpha+\beta$}}

  \end{pspicture}
$$

%%%%%%%%%%%%%%%%%%%%%%%%%%%%%%%%%%%%%%%%%%%%%%%%%%%%%%%%%%%%%%%%%%%%%%
\newpage
\paragraph{General case: with interaction.} $\Esp(X_{tsr}) = \mu +
\alpha_t + \beta_s + (\alpha\beta)_{ts}$:
%\vspace{0.5cm}
$$
  \begin{pspicture}(0,1)(24, 16.5)
    
                                % axes
    \psline[linewidth=0.05]{->}(0, 2)(24, 2)
    \rput[br](24, 2.1){Factor $A$}
    \psline[linewidth=0.05]{->}(0, 2)(0, 16)
    \rput[bl](0.1, 15.5){Mean expression level}
    
                                % mu
    %\psline[linewidth=0.05, linestyle=dashed]{-}(12, 2)(12, 9)
    \psline[linewidth=0.05, linestyle=dashed]{-}(6, 9)(18, 9)
    \rput[bc](11.5, 9.25){$\mu$}

                                % alpha effect
    \psline[linewidth=0.1, linestyle=dashed, linecolor=blue]{-}(6, 7)(18, 11)
%     \psline[linewidth=0.075, linecolor=blue]{<->}(6, 7)(6, 9) 
%     \rput[br](5.75, 8){\textblue{$\alpha_2=-\alpha$}}
%     \psline[linewidth=0.075, linecolor=blue]{<->}(18, 11)(18, 9) 
%     \rput[bl](18.25, 10){\textblue{$\alpha_1=+\alpha$}}

    %\psline[linewidth=0.05, linestyle=dotted]{-}(6, 2)(6, 10)
    \rput[bc](6, 1){Level $2$ $(-)$}
    %\psline[linewidth=0.05, linestyle=dotted]{-}(18, 2)(18, 14)
    \rput[bc](18, 1){Level $1$ $(+)$}

                                % beta effect
    \psline[linewidth=0.1, linecolor=red]{-}(6, 4)(18, 8)
    %\psline[linewidth=0.075, linecolor=red]{<->}(12, 9)(12, 6) 
    %\rput[br](11.75, 7){\textred{$\beta_2=-\beta$}}
    \psline[linewidth=0.1, linecolor=red]{-}(6, 10)(18, 14)
    %\psline[linewidth=0.075, linecolor=red]{<->}(12, 9)(12, 12) 
    %\rput[bl](12.25, 10.5){\textred{$\beta_1=+\beta$}}

                                % additive case
    \rput[bc](18.25, 13.75){{$\mu+\alpha+\beta$}}
    \rput[bc](18.25, 8.25){{$\mu+\alpha-\beta$}}
    \rput[bc](5.75, 3.75){{$\mu-\alpha-\beta$}}
    \rput[bc](5.75, 10.25){{$\mu-\alpha+\beta$}}

                                % interaction effect
    \psline[linewidth=0.1, linecolor=green]{-}(6, 8)(18, 16)
    \psline[linewidth=0.1, linecolor=green]{-}(6, 6)(18, 6)
    \psline[linewidth=0.075, linecolor=green]{<->}(18, 14)(18, 16) 
    \rput[bl](18.25, 14.75){\textgreen{$(\alpha\beta)_{11}=+(\alpha\beta)$}}
    \rput[bc](18, 16.25){{$\mu +\alpha + \beta  +(\alpha\beta)$}}
    \psline[linewidth=0.075, linecolor=green]{<->}(6, 4)(6, 6) 
    \rput[br](5.75, 4.75){\textgreen{$(\alpha\beta)_{22}=+(\alpha\beta)$}}
    \rput[bc](6., 6.25){{$\mu -\alpha - \beta  +(\alpha\beta)$}}
    \psline[linewidth=0.075, linecolor=green]{<->}(18, 8)(18, 6) 
    \rput[bl](18.25, 6.75){\textgreen{$(\alpha\beta)_{12}=-(\alpha\beta)$}}
    \rput[bc](18, 5.5){{$\mu +\alpha - \beta  -(\alpha\beta)$}}
     \psline[linewidth=0.075, linecolor=green]{<->}(6, 10)(6, 8) 
     \rput[br](5.75, 8.75){\textgreen{$(\alpha\beta)_{21}=-(\alpha\beta)$}}
     \rput[bc](6, 8){{$\mu -\alpha + \beta  -(\alpha\beta)$}}
  \end{pspicture}
$$

%%%%%%%%%%%%%%%%%%%%%%%%%%%%%%%%%%%%%%%%%%%%%%%%%%%%%%%%%%%%%%%%%%%%%%
\newpage
\paragraph{$+ / -$ notation.} An equivalent expression of the model is
$$
X_{tsr} = \mu \pm \alpha \pm \beta \pm (\alpha\beta) + E_{tsr},
$$
meaning that, for the four types of individuals, we have:
$$
\vspace{-0.5cm} \begin{array}{cc|c}
  \mbox{Type} & \mbox{Sex} & \mbox{Model} \\
  \hline
  + & + & X = \mu + \alpha +\beta +(\alpha\beta) +E \\
  + & - & X = \mu + \alpha -\beta -(\alpha\beta) +E \\
  - & + & X = \mu - \alpha +\beta +(\alpha\beta) +E \\
  - & - & X = \mu - \alpha -\beta -(\alpha\beta) +E \\
\end{array}
\qquad \qquad
\begin{array}{rcl}
  \multicolumn{3}{c}{\mbox{Parameters}} \\
  \hline
  \alpha & = & \alpha_1 = -\alpha_2 \\
  \beta & = & \beta_1 = - \beta_2 \\
  (\alpha\beta) & = & (\alpha\beta)_{11} = (\alpha\beta)_{22}  \\ 
  & = & -(\alpha\beta)_{12} = -(\alpha\beta)_{21}  
\end{array}
$$
\paragraph{Basic factorial design.} It is described by the
\textblue{design matrix}:
$$
\begin{array}{c|cccc}
  \# \mbox{ individual} & \mu & \alpha & \beta & (\alpha\beta) \\
  \hline
  1 & + & + & + & + \\
  2 & + & + & - & - \\
  3 & + & - & + & - \\
  4 & + & - & - & + \\
\end{array}
\vspace{-0.5cm}
$$
Number of experiments = number of parameters: \textblue{saturated design}.

%%%%%%%%%%%%%%%%%%%%%%%%%%%%%%%%%%%%%%%%%%%%%%%%%%%%%%%%%%%%%%%%%%%%%%
\newpage
\paragraph{Linear model and matrix representation.}
Denote 
\begin{description}
\vspace{-0.5cm}
\item[$\Dbf =$] the design matrix
\item[$\Xbf =$] the vector containing all the expression levels
  ($X_{tsr}$)
\item[$\thetabf = $] the vector containing the parameters:
  $
  \thetabf = \left[ \mu \quad \alpha \quad \beta \quad (\alpha\beta) \right]'
  $
\end{description}
The anova model can be written as
$$
\Xbf = \Dbf \thetabf + \Ebf 
$$
The least square estimate of $\thetabf$ is
$$
\widehat{\thetabf} = (\Dbf' \Dbf)^{-1} \Dbf' \Xbf
$$
Its variance is
$$
\Var\left(\widehat{\thetabf}\right) = \sigma^2 (\Dbf' \Dbf)^{-1}
$$
\paragraph{Remark: } $\Var\left(\widehat{\thetabf}\right)$ mainly
depends on $\Dbf$. It depends on $\Xbf$ only through $\sigma^2$.

%%%%%%%%%%%%%%%%%%%%%%%%%%%%%%%%%%%%%%%%%%%%%%%%%%%%%%%%%%%%%%%%%%%%%%
\newpage
\paragraph{Back to the 2 factor design.} The parameter estimates are
$$
\begin{array}{rclcrcl}
  \widehat{\mu} & = & \frac14(+X_1+X_2+X_3+X_4)
  & \qquad &
  \widehat{\alpha} & = & \frac14(+X_1+X_2-X_3-X_4) \\
  \widehat{\beta} & = & \frac14(+X_1-X_2+X_3-X_4)
  & \qquad &
  \widehat{(\alpha\beta)} & = & \frac14(+X_1-X_2-X_3+X_4) \\
  \multicolumn{7}{c}{\mbox{Saturated design $\Rightarrow$ no estimation of $\sigma^2$}}
\end{array}
$$
\paragraph{Precision of the estimates.} 
If $R$ replicates are performed in each combination ($n = 4R$
experiments):
$$
\Var(\widehat{\mu}) = \Var(\widehat{\alpha}) =
\Var\left(\widehat{\beta}\right) = 
\Var\left(\widehat{(\alpha\beta)}\right) = \frac{\sigma^2}{4R}.
$$
\paragraph{Design size.} To estimate the 4 parameters, we need at
least 4 experiments. 

When studying
$$
k = 3, 4, 5 \mbox{ factors}
$$
we need at least 
$$
2^k = 8, 16, 32 \mbox{ experiments}
$$
corresponding to each possible combination.

%%%%%%%%%%%%%%%%%%%%%%%%%%%%%%%%%%%%%%%%%%%%%%%%%%%%%%%%%%%%%%%%%%%%%%
\newpage
\section{Reducing the number of experiments}
%%%%%%%%%%%%%%%%%%%%%%%%%%%%%%%%%%%%%%%%%%%%%%%%%%%%%%%%%%%%%%%%%%%%%%

%%%%%%%%%%%%%%%%%%%%%%%%%%%%%%%%%%%%%%%%%%%%%%%%%%%%%%%%%%%%%%%%%%%%%%
\bigskip
\subsection{Fractional designs and confounded effects}

The number of experiments may be reduce if some effects are neglected,
i.e. if we renounce to estimate them.

\paragraph{Design matrix for $k=3$ factors.} $\alpha = $ type, $\beta
= $ sex, $\gamma = $ age
$$
\begin{array}{c|cccccccc}
 \# & \mu & \alpha & \beta & \gamma & (\alpha\beta) & (\alpha\gamma) &
 (\beta\gamma) & (\alpha\beta\gamma) \\
 \hline
 1 & + & - & - & - & + & + & + & - \\
 2 & + & - & - & + & + & - & - & + \\
 3 & + & - & + & - & - & + & - & + \\
 4 & + & - & + & + & - & - & + & - \\
 5 & + & + & - & - & - & - & + & + \\
 6 & + & + & - & + & - & + & - & - \\
 7 & + & + & + & - & + & - & - & - \\
 8 & + & + & + & + & + & + & + & + \\
\end{array}
$$
Note that the column $(\alpha\beta)$ is the product of columns
$\alpha$ and $\beta$.

%%%%%%%%%%%%%%%%%%%%%%%%%%%%%%%%%%%%%%%%%%%%%%%%%%%%%%%%%%%%%%%%%%%%%%
\newpage 
\paragraph{Confounded effects.} To neglect all the interaction terms
$(\alpha\beta), (\alpha\gamma), (\beta\gamma), (\alpha\beta\gamma)$
we systematically confound them with main effects:
$$
(\alpha\beta) = \gamma, \qquad (\alpha\gamma) = \beta, 
\qquad (\beta\gamma) = \alpha, \qquad (\alpha\beta\gamma) = \mu
$$
to keep only 4 experiments
$$
\begin{array}{c|cccccccc}
 \# & \mu & \alpha & \beta & \gamma & (\alpha\beta) & (\alpha\gamma) &
 (\beta\gamma) & (\alpha\beta\gamma) \\
 \hline
 2 & + & - & - & + & + & - & - & + \\
 3 & + & - & + & - & - & + & - & + \\
 5 & + & + & - & - & - & - & + & + \\
 8 & + & + & + & + & + & + & + & + 
\end{array}
$$
This design is the half of the full design and is denoted $2^{3-1}$.

\paragraph{Aliasing.} The parameter $\alpha$ is actually an alias for
$\alpha + (\beta\gamma)$:
$$
\widehat{\alpha} = \widehat{\alpha + (\beta\gamma)}, 
\qquad \widehat{\beta} = \widehat{\beta + (\alpha\gamma)}, 
\qquad \widehat{\gamma} = \widehat{\gamma + (\alpha\beta)}, 
\qquad \widehat{\mu} = \widehat{\mu + (\alpha\beta\gamma)}
$$

%%%%%%%%%%%%%%%%%%%%%%%%%%%%%%%%%%%%%%%%%%%%%%%%%%%%%%%%%%%%%%%%%%%%%%
\newpage 
\subsection{Example of a $2^{5-2}$ fractional design}

\noindent \hspace{-1cm} \begin{tabular}{lc}
  \begin{tabular}{l}
    \paragraph{$SAS$ program:}  \\ \\ \\ \\ \\
  \end{tabular}
  & 
  {\small
    \begin{tabular}{p{12cm}} 
\begin{verbatim}
proc Factex; 
        factors A B C D E; 
        size design = 8; 
        model est = (A B C D E); 
        examine aliasing confounding design; 
\end{verbatim}
    \end{tabular}
}
\end{tabular}  \\
\paragraph{$SAS$ output:} \\
\begin{tabular}{c|c}
  {\small
    \begin{tabular}{p{12cm}} 
\begin{verbatim}
                    Design Points

Experiment       A       B       C       D       E

         1      -1      -1      -1      -1       1
         2      -1      -1       1       1      -1
         3      -1       1      -1       1      -1
         4      -1       1       1      -1       1
         5       1      -1      -1       1       1
         6       1      -1       1      -1      -1
         7       1       1      -1      -1      -1
         8       1       1       1       1       1
\end{verbatim}
    \end{tabular}
    \hspace{4cm}
    }
    &
    {\small
      \begin{tabular}{p{12cm}} 
\begin{verbatim}
Aliasing Structure

 A = D*E
 B = C*E
 C = B*E
 D = A*E
 E = A*D = B*C
 A*B = C*D
 A*C = B*D



\end{verbatim}
    \end{tabular}
    }
\end{tabular}

%%%%%%%%%%%%%%%%%%%%%%%%%%%%%%%%%%%%%%%%%%%%%%%%%%%%%%%%%%%%%%%%%%%%%%
%%%%%%%%%%%%%%%%%%%%%%%%%%%%%%%%%%%%%%%%%%%%%%%%%%%%%%%%%%%%%%%%%%%%%%
\newpage
\chapter{Two channels designs}
%%%%%%%%%%%%%%%%%%%%%%%%%%%%%%%%%%%%%%%%%%%%%%%%%%%%%%%%%%%%%%%%%%%%%%
%%%%%%%%%%%%%%%%%%%%%%%%%%%%%%%%%%%%%%%%%%%%%%%%%%%%%%%%%%%%%%%%%%%%%%

%%%%%%%%%%%%%%%%%%%%%%%%%%%%%%%%%%%%%%%%%%%%%%%%%%%%%%%%%%%%%%%%%%%%%%
\bigskip
\section{Data specificity}
%%%%%%%%%%%%%%%%%%%%%%%%%%%%%%%%%%%%%%%%%%%%%%%%%%%%%%%%%%%%%%%%%%%%%%
Glass slide experiments ($r = $ slide):
$$
X_{t r} = \mu_t + \beta_r + \varepsilon_{tr}
$$
where $\beta_r$ is the effect of slide $r$.

If $t$ and $t'$ are compared on the same slide $r$, $\beta_r$ vanishes
in the difference:
$$
\begin{array}{cccccc}
  \underbrace{X_{t r} - X_{t' r}} & = & \underbrace{\mu_t - \mu_{t'}} & + &
  \underbrace{\varepsilon_{tr} - \varepsilon_{t'r}} \\
  \\
  Y_{tt', r} & = &  \delta_{tt'} & + & E_{tt', r}
\end{array}
$$
with $\{E_{tt', r}\}$ iid, $\Var(E) = \sigma^2$.

\textblue{Statistical analysis will consider the 'log-ratio' $Y_{tt', r}$.}

%%%%%%%%%%%%%%%%%%%%%%%%%%%%%%%%%%%%%%%%%%%%%%%%%%%%%%%%%%%%%%%%%%%%%%
\newpage
\section{Swap designs}
%%%%%%%%%%%%%%%%%%%%%%%%%%%%%%%%%%%%%%%%%%%%%%%%%%%%%%%%%%%%%%%%%%%%%%
\subsection{Bias in cDNA labeling}
$$
\begin{tabular}{l}
  \textblue{M-A plot:} \\
  \\
  $R = $ red signal \\
  \qquad (cond. 1) \\ 
  \\
  $V = $ green signal \\
  \qquad (cond. 2) \\
  \\
  $M = R - V$ \\
  $A = R + V$ \\
  \\
  lowess correction \\
  for each tip \\
  \\
\end{tabular}
\begin{tabular}{c}
  \epsfig{figure=../Figures/DYCS00-Fig4.ps, height=12cm, width=15cm,
    bbllx=80, bblly=290, bburx=520, bbury=560, clip=} 
\end{tabular}
$$
\refer{Dudoit \& al. (00)}{Statistical methods for identifying
  differentially expressed genes...}, \refer{Yang \& al. (02)}{}

%%%%%%%%%%%%%%%%%%%%%%%%%%%%%%%%%%%%%%%%%%%%%%%%%%%%%%%%%%%%%%%%%%%%%%
\newpage
\subsection{Swap design}

\paragraph{Statistical model.} $t = $ treatment, $r = $ slide, $d
= $ dye:
$$
X_{trd} = \mu + \alpha_t + \beta_{r} + \gamma_d + E_{trd} 
$$
Swap designs aim to correct the bias due to cDNA labeling ($\gamma_d$).

\paragraph{Design.} Two slides, with inverted labeling.
$$
\begin{array}{ccc}
  \begin{tabular}{ll|cc}
    \multicolumn{2}{l|}{dye $d$} & \multicolumn{2}{c}{condition $t$} \\
    & & 1 & 2 \\
    \hline
    & 1 & 1 (red) & 2 (green) \\
    slide $r$ & & \\
    & 2 & 2 (green) & 1 (red)
  \end{tabular}
  &
  \begin{array}{c} 1 \rightarrow + \\ 2 \rightarrow - \end{array} 
  &
  \begin{array}{ccc}
    \multicolumn{3}{c}{\mbox{Design matix}} \\
    \alpha & \beta & \gamma \\
    \hline
    + & + & + \\
    + & - & - \\
    - & + & - \\
    - & - & + \\
  \end{array}
\end{array}
$$
\paragraph{Aliasing.} Such a design is refereed to as a
\textblue{latin square} design. 

The treatment effect $\alpha$ is \textblue{confounded} with the dye*slide
interaction $(\beta\gamma)$.

%%%%%%%%%%%%%%%%%%%%%%%%%%%%%%%%%%%%%%%%%%%%%%%%%%
\newpage
\paragraph{Removing the dye effect.} The slide effects $\beta_s$ vanishes
in the log-ratios (Red - Green) observed on both slides: 
$$
Y_1 = + (\alpha_1 - \alpha_2) + (\gamma_1 - \gamma_2) + E_1, \qquad
Y_2 = - (\alpha_1 - \alpha_2) + (\gamma_1 - \gamma_2) + E_2
$$
The dye effects $\gamma_d$ vanishes in their difference:
$$
\Delta Y = Y_1 - Y_2 = 2 (\alpha_1 - \alpha_2) + (E_1 - E_2) \qquad
\Rightarrow \qquad \widehat{\alpha_1 - \alpha_2} = 2 \widehat{\alpha}
= \frac{\Delta Y}4
$$
\paragraph{Application.} This simple difference remove most part of
the Gene*Dye interaction. 
$$
\begin{tabular}{ccc}
  \begin{tabular}{l}
    \hspace{-1.5cm} 
    \epsfig{figure=..//Figures/MAplot-DyeSwap-ECabannes.ps,
      bbllx=21, bblly=19, bburx=289, bbury=140, width=12cm, height=6cm,
      clip=}    
  \end{tabular}
  &
  \begin{tabular}{c}
    $\hspace{-1cm} \overset{\mbox{difference}}{\longrightarrow} \hspace{-1cm}$ 
  \end{tabular}
  &
  \begin{tabular}{l}
    \epsfig{figure=../Figures/MAplot-DyeSwap-ECabannes.ps,
      bbllx=345, bblly=19, bburx=574, bbury=143, width=10cm, height=6cm,
      clip=}
  \end{tabular}
\end{tabular}   
$$
The labeling bias is supposed to be the same on each slide \\
\centerline{\textblue{$\Rightarrow$ no dye*slide interaction}}

%%%%%%%%%%%%%%%%%%%%%%%%%%%%%%%%%%%%%%%%%%%%%%%%%%
\newpage
\subsection{Lowess regression} 

The Gene*Dye interaction is \textblue{modeled} as a function of the
mean expression level of the gene:
$$
M_g = f(A_g).
$$
The regression function $f(A)$ is estimated using a
\textblue{robust, locally weighted} linear regression $A = a_g + b_g
M$.

Parameters $a_g$ and $b_g$ are obtained minimizing (for gene $g$)
$$
\sum_{g'} w(g, g')[M_{g'} - a_g + b_g A_{g'}]^2
$$
where \\
$\bullet \quad$ $w(g, g')$ is a decreasing function of $|A_g - A_{g'}|$,
\\
$\bullet \quad$ ``large'' residuals $(M_{g'} - \widehat{M}_{g'})$ are
excluded from the sum.

\refer{Cleveland (79)}{Robust locally weighted regression...}

%%%%%%%%%%%%%%%%%%%%%%%%%%%%%%%%%%%%%%%%%%%%%%%%%%
\newpage
\paragraph{Back to the swap design.} 
% The lowess transform can be viewed as a \textblue{modeling of the
%   Dye*Gene interaction}
Generally, a lowess is performed separately for each slide.
\centerline{$\Rightarrow$ Correction of the \textblue{Slide*Gene*Dye
    interaction}}

\paragraph{Anova table.} Real data, 2 slides, $\simeq$ 10\;000 genes
$$
  \begin{tabular}{lcccccc}
    & & \multicolumn{2}{l}{\qquad before lowess \qquad\qquad~} &
    \multicolumn{2}{l}{\qquad after lowess\qquad\qquad~} \\
    Effect & df & SS & MS & SS & MS \\
    \hline
    Chip & 1 & 165 & 165 & 165 & 165 \\ 
    {\sl Dye} & 1 & {\sl 19149} & 19149 & {\sl 0.0386} & 0.0386 \\ 
    {\sl Treatment} & 1 & {\sl 480} & 480 & {\sl 0.00242} & 0.00242 \\ 
    Gene & 9983 & 60826 & 6.09 & 60826 & 6.09 \\ 
    Chip*Gene & 9983 & 2330 & 0.233 & 2330 & 0.233 \\ 
    {\bf Dye*Gene} & 9983 & {\bf 5882} & 0.589 & {\bf 299} & 0.0299 \\ 
    {\sl Treatment*Gene} & 9983 & {\sl 4781} & 0.479 & {\sl 3471} & 0.348 \\ 
  \end{tabular} 
$$
\begin{itemize}
\item The Gene*Dye interaction is strongly reduced $\rightarrow$ Loess
  does its job.
\item The Dye and Treatment effects are also strongly reduced.
\item The Gene*Treatment interaction is also (slightly) reduced.
\end{itemize}

%%%%%%%%%%%%%%%%%%%%%%%%%%%%%%%%%%%%%%%%%%%%%%%%%%
\newpage
\paragraph{Practical conclusions.}
\begin{itemize}
\item If the Gene*Dye interaction is the same on all slides,
  no specific normalization is needed.
\item The lowess normalization is needed when the Gene*Dye varies with
  the slides ie, when a Slide*Gene*Dye interaction exists.
\end{itemize}
\paragraph{Aliasing.} Due to the latin square structure, the Dye*Slide 
interaction is confounded with the condition effect
$$
\Longrightarrow \mbox{\textblue{Slide*Gene*Dye = Gene*Treatment}}
$$
The lowess normalization should be very smooth since it may
\textblue{correct the signal} we are interested in!

\paragraph{Laser tuning.}
The Slide*Gene*Dye interaction can be due to slide specific tuning of
the lasers (PMT)
$$
\mbox{\textblue{Avoid tuning the laser for each slide}}
$$


%%%%%%%%%%%%%%%%%%%%%%%%%%%%%%%%%%%%%%%%%%%%%%%%%%%%%%%%%%%%%%%%%%%%%%
\newpage
\section{Comparing $T$ conditions}
%%%%%%%%%%%%%%%%%%%%%%%%%%%%%%%%%%%%%%%%%%%%%%%%%%%%%%%%%%%%%%%%%%%%%%

\bigskip
\begin{tabular}{l}
  \hspace{-0.5cm}\subsection{`Star' design} \\
  \\
  Each condition \\
  is compared to \\
  a common \\
  reference $A_0$ \\
  \\
  ($TR$ slides)
\end{tabular}
\hspace{2cm}
\begin{tabular}{c}
  \begin{pspicture}(8, 8)
    \rput[B](4, 3.8){$A_0$}
    
    \rput[B](1, 1){$A_1$}
    \psline[linewidth=0.1]{<->}(1.5, 1.5)(3.5, 3.5)
    
    \psline[linewidth=0.1, linestyle=dashed]{<->}(3.5, 4)(1, 5.5)
    \psline[linewidth=0.1, linestyle=dashed]{<->}(4.5, 4)(7, 5.5)
    
    \rput[B](4, 8){$A_t$}
    \psline[linewidth=0.1]{<->}(4, 4.7)(4, 7.5)
    
    \rput[B](7, 1){$A_T$}
    \psline[linewidth=0.1]{<->}(6.5, 1.5)(4.5, 3.5)
  \end{pspicture}
\end{tabular}
\hspace{-2cm}
\begin{tabular}{l}
  Typically, \\
  \\
  $A_0 = $ wild type \\
  \\
  $A_t = $ mutant\\
\end{tabular}
$$
% \begin{array}{lcccc}
%   \mbox{parameter}\qquad & \qquad \delta_{tt'}
%   \qquad & \qquad  \delta_{t0} \qquad & \qquad \mu_t \qquad & \qquad
%   \mu_0 \qquad \\
%   \hline
%   \\
%   \mbox{estimate} & \overline{Y}_{t0} - \overline{Y}_{t'0} &
%   \overline{Y}_{t0} & \overline{X}_t & \overline{X}_0 \\
%   \\
%   \mbox{variance}  & 2\sigma^2 / R & \sigma^2 / R & \gamma^2 / R &
%   \gamma^2 / (TR) \\
%   \\
%   \\
%   & \multicolumn{4}{r}{\gamma^2 = \sigma^2/2 + \mbox{variance due to 
%       the slides.}}
% \end{array}
\begin{array}{lcc}
  \mbox{parameter}\qquad & \qquad \delta_{tt'}
  \qquad & \qquad  \delta_{t0} \qquad  \\
  \hline
  \\
  \mbox{estimate} & \overline{Y}_{t0} - \overline{Y}_{t'0} &
  \overline{Y}_{t0} \\
  \\
  \mbox{variance}  & 2\sigma^2 / R & \sigma^2 / R  \\
  \\
%  \\
%  & \multicolumn{4}{r}{\gamma^2 = \sigma^2/2 + \mbox{variance due to 
%      the slides.}}
\end{array}
$$

%%%%%%%%%%%%%%%%%%%%%%%%%%%%%%%%%%%%%%%%%%%%%%%%%%
\newpage
\begin{tabular}{l}
  \hspace{-0.5cm}\subsection{`Loop' design} \\
  \\
  Each condition \\
  is compared to \\
  two (arbitrary) \\
  others \\
  \\
  ($TR$ slides)
\end{tabular}
%\hspace{-2cm}
\begin{tabular}{c}
  \begin{pspicture}(9, 9)
    \rput[B](3, 0.5){$A_1$}
    \psline[linewidth=0.1, linestyle=dashed]{<->}(2.5, 1)(1, 2)
%    \rput[B](0.5, 2.5){$o$}
    \psline[linewidth=0.1, linestyle=dashed]{<->}(0.5, 3)(0.5, 5)
    \rput[B](0.5, 5.5){$A_{t-1}$}
    \psline[linewidth=0.1]{<->}(1, 6)(2.5, 7)
    \rput[B](3, 7.5){$A_t$}
    \psline[linewidth=0.1]{<->}(3.7, 7.5)(6.3, 7.5)
    \rput[B](7, 7.5){$A_{t+1}$}
    \psline[linewidth=0.1, linestyle=dashed]{<->}(7.5, 7)(9, 6)
%    \rput[B](9.5, 5.5){$o$}
    \psline[linewidth=0.1, linestyle=dashed]{<->}(9.5, 5)(9.5, 3)
%    \rput[B](9.5, 2.5){$o$}
    \psline[linewidth=0.1, linestyle=dashed]{<->}(9, 2)(7.5, 1)
    \rput[B](7, 0.5){$A_T$}
    \psline[linewidth=0.1]{<->}(6.3, 0.5)(3.7, 0.5)
  \end{pspicture}
\end{tabular}
\hspace{1.5cm}
\begin{tabular}{l}
  Typically, \\
  \\
  $A_t = $ time $t$
\end{tabular}
$$
\begin{array}{lll}
  \mbox{parameter} \qquad &  \mbox{estimate} & \mbox{variance} \\
  \hline
  \\
  \delta_{t, t+1} & \overline{Y}_{t, t+1} & \sigma^2 / R \\ 
  \\
  \delta_{t, t+d} &  \overline{Y}_{t, t+1} + \dots +
  \overline{Y}_{t+d-1, t+d} \qquad & d \sigma^2 / R \\
\end{array}
$$

%%%%%%%%%%%%%%%%%%%%%%%%%%%%%%%%%%%%%%%%%%%%%%%%%
\newpage 
\noindent The variance of the estimates can be reduced considering the
two possible paths from $A_t$ to $A_{t+d}$ in the graph: \vspace{0.5cm}\\
$\begin{array}{clc}
  \mbox{parameter} &  \mbox{estimate (least square)} & \mbox{variance} \\
  \hline
  \\
  \delta_{t, t+1} & \displaystyle{ \frac{T-1}{T} \overline{Y}_{t,
      t+1} + \frac{1}{T} \sum_{u=1}^{T-1} \overline{Y}_{t+u, t+u+1} }
  & \displaystyle{\frac{T-1}{T} \frac{\sigma^2}{R}} \\
  \\
  \delta_{t, t+d} & \displaystyle{ \frac{T-d}{T} \sum_{u=1}^{d}
    \overline{Y}_{t+u, t+u+1} + \frac{d}{T} \sum_{u=d+1}^{T}
    \overline{Y}_{t+u, t+u+1} } & \displaystyle{\frac{d(T-d)}{T}
    \frac{\sigma^2}{R}}
\end{array}$ 
\vspace{-1cm} \\
\centerline{
  \begin{tabular}{l}
    $T = 10$ \\
    \\
    $\sigma^2 = 1$ \\
    \\
    $R = 4$
  \end{tabular}
  \begin{tabular}{l}
    \psfig{file=../Figures/VarLoop.ps, width=15cm, height=9cm}
  \end{tabular}
  }

%%%%%%%%%%%%%%%%%%%%%%%%%%%%%%%%%%%%%%%%%%%%%%%%%%%%%%%%%%%%%%%%%%%%%%
\newpage
\section{Case of a two factor design}
%%%%%%%%%%%%%%%%%%%%%%%%%%%%%%%%%%%%%%%%%%%%%%%%%%%%%%%%%%%%%%%%%%%%%%

%%%%%%%%%%%%%%%%%%%%%%%%%%%%%%%%%%%%%%%%%%%%%%%%%%%%%%%%%%%%%%%%%%%%%%
\bigskip
\paragraph{Biological question.} 
\begin{itemize}
\item We want to study the effects of two factors: type ($\alpha_i$)
  and sex ($\beta_j$) and of their interaction $(\alpha\beta)_{ij}$).
\item We are mainly interested in the status effect. 
\end{itemize}

\paragraph{Practical constraints.} 
\begin{itemize}
\item We can only afford 8 slides.
\item We want to estimate all the effects: $\alpha_i, \beta_j,
  (\alpha\beta)_{ij}$.
\item We would like to avoid normalizing steps (e.g. loess).
\end{itemize}

%%%%%%%%%%%%%%%%%%%%%%%%%%%%%%%%%%%%%%%%%%%%%%%%%%%%%%%%%%%%%%%%%%%%%%
\newpage
\subsection{Statistical model}

\paragraph{One channel.} Slide $s$, dye $d$
$$
X_{ijsd} = \mu + \alpha_i + \beta_j + (\alpha\beta)_{ij} + \gamma_s +
\delta_d + E_{ijsd}
$$
\paragraph{Log-ratio.} Slide $s$
\begin{eqnarray*}
  Y_s & = & X_{ijs1} - X_{i'j's2} \\
  & = & \mu + (\alpha_i-\alpha_{i'}) +
  (\beta_j-\beta_{j'}) + ((\alpha\beta)_{ij}-(\alpha\beta)_{i'j'}) \\
  & & + (\delta_1 - \delta_2) + (E_{ijs1} - E_{i'j's2})
\end{eqnarray*}
\begin{itemize}
\item The slide effect $\gamma_s$ vanishes
\item The labeling bias $(\delta_1 - \delta_2)$ is always present
\item The effect $(\alpha_i-\alpha_{i'})$ appears only if the
  individuals have different status: $i \neq i'$.
\end{itemize}

%%%%%%%%%%%%%%%%%%%%%%%%%%%%%%%%%%%%%%%%%%%%%%%%%%%%%%%%%%%%%%%%%%%%%%
\newpage
\paragraph{Proposed design.} Avoiding loess $\Rightarrow$ swap
experiments.

8 slides = 4 swaps = 8 individuals:
  $$
  \begin{array}{c|ccccc|ccc}
    \mbox{swap} & \multicolumn{2}{c}{\mbox{indiv. 1}} & / &
    \multicolumn{2}{c|}{\mbox{indiv. 2}} 
    & \multicolumn{3}{c}{\mbox{Design matrix}} \\
    & (\mbox{Type} & \mbox{Sex}) & / & (\mbox{Type} & \mbox{Sex}) &
    \alpha & \beta & (\alpha\beta) \\
    \hline
    1 & 1 & 1 & / & 2 & 1 & +4 &    & +4 \\
    2 & 1 & 1 & / & 2 & 2 & +4 & +4 &  \\
    3 & 1 & 2 & / & 2 & 1 & +4 & -4 &  \\
    4 & 1 & 2 & / & 2 & 2 & +4 &    & -4 \\
  \end{array}
  $$
\paragraph{Estimates.} 
$$
\widehat{\alpha} = \frac{\Delta Y_1 + \Delta Y_2 + \Delta Y_3 + \Delta
  Y_4}{16},  
\qquad
\widehat{\beta} = \frac{\Delta Y_2 - \Delta Y_3}{8}, 
\qquad
\widehat{(\alpha\beta)} = \frac{\Delta Y_1 - \Delta Y_4}{8}.
$$
\paragraph{Variance.} $\sigma^2 = $
mixture of technological and biological variances:
$$
\Var(\widehat{\alpha}) = \frac12\Var\left(\widehat{\beta}\right) =
\frac12\Var\left(\widehat{(\alpha\beta)}\right).
$$

%%%%%%%%%%%%%%%%%%%%%%%%%%%%%%%%%%%%%%%%%%%%%%%%%%%%%%%%%%%%%%%%%%%%%%
%%%%%%%%%%%%%%%%%%%%%%%%%%%%%%%%%%%%%%%%%%%%%%%%%%%%%%%%%%%%%%%%%%%%%%
\newpage
\chapter{Example with dependent data}
%%%%%%%%%%%%%%%%%%%%%%%%%%%%%%%%%%%%%%%%%%%%%%%%%%%%%%%%%%%%%%%%%%%%%%
%%%%%%%%%%%%%%%%%%%%%%%%%%%%%%%%%%%%%%%%%%%%%%%%%%%%%%%%%%%%%%%%%%%%%%

%%%%%%%%%%%%%%%%%%%%%%%%%%%%%%%%%%%%%%%%%%%%%%%%%%%%%%%%%%%%%%%%%%%%%%
\bigskip
\section{Biological problem}
%%%%%%%%%%%%%%%%%%%%%%%%%%%%%%%%%%%%%%%%%%%%%%%%%%%%%%%%%%%%%%%%%%%%%%

\paragraph{Question.} Measure the effects of trisomy and sex on the
transcription of genes located on chromosome 21. 
\begin{itemize}
\item The major attention is paid to the trisomy effect.
\item We want to evaluate the between-individual variability.
\end{itemize}

\paragraph{Patients.} 10 patients with trisomy (T), 10 normal (N); 5 men
(M) / 5 women (F) in each group.

\paragraph{Constraints.}
\begin{itemize}
\item About 40 slides available.
\item Avoid swap to augment the number of individuals to be studied.
\end{itemize}


%%%%%%%%%%%%%%%%%%%%%%%%%%%%%%%%%%%%%%%%%%%%%%%%%%%%%%%%%%%%%%%%%%%%%%
\newpage
\subsection{Statistical model}
%%%%%%%%%%%%%%%%%%%%%%%%%%%%%%%%%%%%%%%%%%%%%%%%%%%%%%%%%%%%%%%%%%%%%%

\paragraph{Mixed model for one channel measurements.}
$$
X_{tsir} = \mu + \alpha_t + \beta_s + (\alpha\beta)_{ts} +
\textblue{A_{tsi}} + \textred{E_{tsir}}
$$
$A_{tsi}$ is the random effect of individual $i$ (type $t$, sex
$s$) affecting all samples coming from it.
% $$
% \vspace{-0.5cm}
% \Var(E) = \sigma^2, \qquad \Var(A) = \gamma^2
% \qquad \Rightarrow \qquad
% \Corr(X_{tsir}, X_{tsir}) = \frac{\gamma^2}{\gamma^2 + \sigma^2}
% $$
$$
\vspace{-1cm} \hspace{-1cm}
\begin{tabular}{cc}
  \begin{tabular}{ll}
    \multicolumn{2}{c}{$
      \begin{array}{l}
        \Var(E) = \textred{\sigma^2}, \qquad \Var(A) = \textblue{\gamma^2} \\
        \\
        \Corr(X_{tsir}, X_{tsir}) =
        \displaystyle{\frac{\gamma^2}{\gamma^2 + \sigma^2}} \\ 
        \\
      \end{array}
      $}
    \\
    {\Large \bf ---}: & fixed effects ($\mu_{ts}$)\\ 
    \textblue{\Large \bf ---}: & biological variability ($\gamma^2$)\\
    \textred{\Large \bf $\cdots$}: & technological variability \\
    & ($\sigma^2$) \\
    \\
  \end{tabular}
  & 
  \hspace{-1cm}
  \begin{tabular}{c}
    \epsfig{figure=../Figures/FigMixedModel.eps, height=9cm, width=15cm,
      bbllx=101, bblly=225, bburx=532, bbury=571, clip=}
  \end{tabular}
\end{tabular}
$$

%%%%%%%%%%%%%%%%%%%%%%%%%%%%%%%%%%%%%%%%%%%%%%%%%%%%%%%%%%%%%%%%%%%%%%
\newpage
\paragraph{Slides / Log-ratios.} 
\begin{itemize}
\item As for the preceding design, only T/N comparison are performed.
\item Log-ratios coming from slides involving a same individual are
  correlated.
\end{itemize}
\paragraph{Proposed design;}
$$
{\small
  \begin{array}{ccc|ccccc|ccccc}
    & & & \multicolumn{10}{c}{$ Type = Trisomy$} \\
    \multicolumn{2}{c}{+=R/G} & & \multicolumn{5}{c|}{$ Sex = M$} 
    & \multicolumn{5}{c}{$ Sex = F$} \\
    \multicolumn{2}{c}{-=G/R} & & 1 & 2 & 3 & 4 & 5 & 1 & 2 & 3 & 4 & 5 \\
    \hline
    &            & 1 &  + &  - &    &    &    &  + &  - &    &    &    \\
    &            & 2 &    &  + &  - &    &    &    &  + &  - &    &    \\
    & $ Sex = M$ & 3 &    &    &  + &  - &    &    &    &  + &  - &    \\
    &            & 4 &    &    &    &  + &  - &    &    &    &  + &  - \\
    $ Type =$  & & 5 &  - &    &    &    &  + &  - &    &    &    &  +
    \\
    \hline
    $ Normal $ & & 1 &  + &  - &    &    &    &  + &  - &    &    &    \\
    &            & 2 &    &  + &  - &    &    &    &  + &  - &    &    \\
    & $ Sex = F$ & 3 &    &    &  + &  - &    &    &    &  + &  - &    \\
    &            & 4 &    &    &    &  + &  - &    &    &    &  + &  - \\
    &            & 5 &  - &    &    &    &  + &  - &    &    &    &  + \\
  \end{array}
}
$$

%%%%%%%%%%%%%%%%%%%%%%%%%%%%%%%%%%%%%%%%%%%%%%%%%%%%%%%%%%%%%%%%%%%%%%
\newpage
\subsection{Variance of the estimates}

\paragraph{2 scenarios.} 
$$
  \begin{array}{c|c}
    \gamma^2 = 0 & \gamma^2 = 2 \sigma^2 \\
    \hline
    \Var(\widehat{\thetabf}) = \sigma^2 \left( 
      \begin{array}{ccc}
        0.0125  &          0   &         0 \\
        0  &      0.025   &         0 \\
        0  &          0   &     0.025 \\
      \end{array}    
    \right)
    & 
    \Var(\widehat{\thetabf}) = \sigma^2 \left( 
      \begin{array}{ccc}
        0.2125 & 0 & 0 \\
        0 & 0.225 & 0 \\
        0 & 0 & 0.225 \\
      \end{array}
    \right)
  \end{array}
$$

\paragraph{Full design.} If $\gamma^2 = 2 \sigma^2$
$$
\begin{tabular}{cc}
  ${\small
    \begin{array}{c|ccccc|ccccc}
      & \multicolumn{5}{c|}{$TM$} 
      & \multicolumn{5}{c}{$TF$} \\
      \hline
            & + & - & + & - & + & - & + & - & + & - \\
            & - & + & - & + & - & + & - & + & - & + \\
       $NM$ & + & - & + & - & + & - & + & - & + & - \\
            & - & + & - & + & - & + & - & + & - & + \\
            & + & - & + & - & + & - & + & - & + & - \\
      \hline
            & - & + & - & + & - & + & - & + & - & + \\
            & + & - & + & - & + & - & + & - & + & - \\
       $NF$ & - & + & - & + & - & + & - & + & - & + \\
            & + & - & + & - & + & - & + & - & + & - \\
            & - & + & - & + & - & + & - & + & - & + \\
    \end{array}
    }
  $
  &
  \begin{tabular}{p{8cm}}
    $\Var(\widehat{\thetabf}) = \sigma^2 \times$ \\
    $\left( 
      \begin{array}{ccc}
        0.205 & 0 & 0 \\
        0 & 0.21 & 0 \\
        0 & 0 & 0.21 \\
      \end{array}
    \right)$ \\
    \\
    $\Rightarrow$ No need for more slides, \\
    \\
    Need for more individuals.
  \end{tabular}
\end{tabular}
$$ 
%%%%%%%%%%%%%%%%%%%%%%%%%%%%%%%%%%%%%%%%%%%%%%%%%%%%%%%%%%%%%%%%%%%%%%
\newpage
\section{Preliminary results}

\paragraph{Estimates of $\gamma^2$ and $\sigma^2$.} 

\hspace{-2cm}
\begin{tabular}{cc}
  \begin{tabular}{p{10cm}}
    Distribution of the ratio
    $\widehat{\gamma}^2/\widehat{\sigma}^2$ over 700 genes. \\   
    \\
    \epsfig{file = ../Figures/FigOBagland-P56.ps, width=10cm, height=10cm,
      bbllx=174, bblly=326, bburx=419, bbury=542, clip=}
  \end{tabular}
  & 
  \hspace{-2cm}
  \begin{tabular}{p{15cm}}
    \begin{itemize}
    \item For many genes $\widehat{\gamma}^2 = 0$:
      small biological variance or estimation problems?
    \item The mean ratio is about 0.7 (instead of 2): should the
      design be reconsidered?
    \item Simulations show that the design is still satisfying (but the
      interest of the full design increases).
    \item The definition of biological / technological variances is
      not that clear: for example, the cell growth is a
      'technological' step.
    \end{itemize}
  \end{tabular}
\end{tabular}

%%%%%%%%%%%%%%%%%%%%%%%%%%%%%%%%%%%%%%%%%%%%%%%%%%%%%%%%%%%%%%%%%%%%%%%%
%%%%%%%%%%%%%%%%%%%%%%%%%%%%%%%%%%%%%%%%%%%%%%%%%%%%%%%%%%%%%%%%%%%%%%%%
%%%%%%%%%%%%%%%%%%%%%%%%%%%%%%%%%%%%%%%%%%%%%%%%%%%%%%%%%%%%%%%%%%%%%%%%
%%%%%%%%%%%%%%%%%%%%%%%%%%%%%%%%%%%%%%%%%%%%%%%%%%%%%%%%%%%%%%%%%%%%%%%%
\end{document}
%%%%%%%%%%%%%%%%%%%%%%%%%%%%%%%%%%%%%%%%%%%%%%%%%%%%%%%%%%%%%%%%%%%%%%%%
%%%%%%%%%%%%%%%%%%%%%%%%%%%%%%%%%%%%%%%%%%%%%%%%%%%%%%%%%%%%%%%%%%%%%%%%
%%%%%%%%%%%%%%%%%%%%%%%%%%%%%%%%%%%%%%%%%%%%%%%%%%%%%%%%%%%%%%%%%%%%%%%%
%%%%%%%%%%%%%%%%%%%%%%%%%%%%%%%%%%%%%%%%%%%%%%%%%%%%%%%%%%%%%%%%%%%%%%%%

%%% Local Variables: 
%%% mode: latex
%%% TeX-master: "Microarray-tra"
%%% End: 
