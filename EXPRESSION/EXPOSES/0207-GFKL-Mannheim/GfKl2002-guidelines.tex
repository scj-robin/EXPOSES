%%%%%%%%%%%%%%%%%%%%%%%%%%%%%%%%%%%%%%%%%%%%%%%%%%%%%%%%%%%%%%%
% Guidelines for authors for the 26th Annual Conference 
% Version 14.0.2.2001
%%%%%%%%%%%%%%%%%%%%%%%%%%%%%%%%%%%%%%%%%%%%%%%%%%%%%%%%%%%%%%%

%%%%%%%%%%%%%%%%%%%%%%%%%%%%%%%%%%%%%%%%%%%%%%%%%%%%%%%%%%%%%%%
% This is a sample input file for your contribution to the 
% proceedings volume to be published by Springer-Verlag.
%
% Please, use it as a template for your own input, and, please,
% follow the instructions in this document.
%
% Please, send the Tex and figure files of your manuscript
% to the editor.
%
%%%%%%%%%%%%%%%%%%%%%%%%%%%%%%%%%%%%%%%%%%%%%%%%%%%%%%%%%%%%%%%

%PACKAGES%%%%%%%%%%%%%%%%%%%%%%%%%%%%%%%%%%%%%%%%%%%%%%%%%%%%


\documentclass[runningheads]{d:/latex/gfkl2002}

\usepackage{amsmath}
\usepackage{amsfonts}
\usepackage{latexsym}  
\usepackage{psfig}
\usepackage{graphicx} 

%AUTHOR_STYLES_AND_DEFINITIONS%%%%%%%%%%%%%%%%%%%%%%%%%%%%%%%
%
% Please, reduce your own definitions and macros to an absolute
% minimum. 
%
%%%%%%%%%%%%%%%%%%%%%%%%%%%%%%%%%%%%%%%%%%%%%%%%%%%%%%%%%%%%%

\begin{document}

%
% \title* lets you specify the title of your manuscript. 
% Use \protect\linebreak to force a line break in your title.
\title*{Guidelines for Authors of the\protect\linebreak  
Proceedings Volume of the\protect\linebreak  
26th Annual Conference of the GfKl
}

%
% \toctitle specifies the title as it will be printed in the table of 
% contents. 
% Use \protect\linebreak to force a line break in your title.
\toctitle{Guidelines for Authors of the Proceedings Volume}

%
% \titlerunning defines the title in the running head. 
% Abbreviate your title, if the full title is too long to fit in the running 
% head.
\titlerunning{Guidelines for Authors}

%  
% \authors specifies the authors. 
% Authors are separated by the \and command. Use \inst{1}, \inst{2}, ...
% to define the reference mark to your affiliation. 
\author{
  Wolfgang Gaul\inst{1}
  \and 
  Martin Schader\inst{2}
  \and 
  Maurizio Vichi\inst{3}
}

%
% \authorrunning specifies the author names in the running heads.
% If there is one author: Gaul
% If there are two autors: Gaul and Schader
% If there are more than two authors: Gaul et al.
\authorrunning{Gaul et al.} 

%
% The \institute command lets you specify the your affiliation and
% your address. Separate two or more different affiliations by the
% \and command. 
\institute{
  Institut f\"ur Entscheidungstheorie und Unternehmensforschung, \\
  Universit\"at Karlsruhe, D-76128 Karlsruhe, Germany
  \and 
  Lehrstuhl f\"ur Wirtschaftsinformatik III,\\
  Universit\"at Mannheim, D-68131 Mannheim, Germany
  \and 
  Dip.\ Statistica, Probabilit\`{a} e Statistiche Applicate,\\
  Universit\`{a} di Roma, I-00185 Roma, Italy
}

% Typeset the title
\maketitle             

\begin{abstract}
A proceedings volume of the 26th Annual Conference of the GfKl will be published 
as a special volume in the Series ``Studies in Classification, Data Analysis, 
and Knowledge Organization'' by Springer-Verlag.
In these guidelines we describe the format instructions and the submission 
procedure both to be followed seriously. The text of these guidelines is 
written in the prescribed format and can be used as a specimen.
\end{abstract}

\section{The format of the text}
Please, use the style file \verb@gfkl2002.cls@. This file takes care of 
all the formatting.\\
{\bf Please, use only the \verb@gfkl2002.cls@ style file and standard
fonts.\\
Please, do not include other packages than the ones used above.}\\
If you have any problems, do not hesitate to
contact:
\begin{center}
Ingo Ott\\
Lehrstuhl f\"ur Wirtschaftsinformatik III\\
Universit\"at Mannheim, Schlo\ss, D-68131 Mannheim, Germany\\
E-mail: ott@wifo3.uni-mannheim.de\\
\end{center}

\section{Heading, abstract, and sections}
For the heading, specify the following items: 
\begin{itemize}
\item \verb@\title*@ to specify the title of your manuscript,
\item \verb@\toctitle@ to specify the title to be used in the table 
 of contents,
\item \verb@\titlerunning@ to specify the title in the running head,
\item \verb@\author@ to specify the authors. Authors are 
 separated by the \verb@\and@ command. Use the \verb@\inst{1}@,  
 \verb@\inst{2}@, \ldots\ commands to define the reference mark to your affiliation. 
\item \verb@\authorrunning@ to specify the author names in the running heads.
 If there are more than two authors, please, abbreviate the authors' list
 (e.g., Opitz et al.).
\item \verb@\institute@ to specify your affiliation and
 your address. Separate two or more different affiliations by the
 \verb@\and@ command. 
\end{itemize}
An abstract (6--9 lines) is to be included after the heading using the 
commands \verb@\begin{abstract} ... \end{abstract}@. Do not specify 
keywords in your paper.

Please, use only the \LaTeX\ sectioning commands \verb@\section@ and 
\verb@\subsec@- \verb@tion@. Do not use a deeper hierarchy. Considering the 
length of the ma\-nu\-script, we recommend only using the \verb@\section@  
command. Only capitalize the first word of the (sub)section title. 

\section{Length of the paper, figures, tables, and equations}
The manuscripts should have a length of approximately {\bf 8 pages} using the 
\verb@gfkl2002.cls@ style file.

Each figure or table must have a caption, explaining in brief your 
figure or table, e.g., ``Fig. 3. Plot of ...  .'' or ``Table 1. Data 
of ...  .'' (arabic numbering).  Use capitalization when referring to a 
figure or a table in the text, for example, ``In the center of Figure 
3, the clusters $\ldots$''.

Figures and tables must be included at the appropriate place in the 
text. Please, do not use colored figures. Please, provide the 
figures only using the \LaTeX\ picture commands or in \verb@eps@ 
(Encapsulated PostScript) format.  A figure contained in the \verb@eps@-file  
\verb@yourfile.eps@ can be included as follows: 
\begin{verbatim}
\begin{figure}[t]
  \centerline{
    \includegraphics[width=.8\textwidth]{yourfile.eps}}
  \caption{This is a sample of how to include an  
  eps graphics file in your manuscript.}
  \label{fig:ExampleEPSFile}
\end{figure}
\end{verbatim}
The \verb@[width=.8\textwidth]@ option specifies that the graphics will 
to be reduced to 80\%.  The \verb@\includegraphics@ command makes use 
of the \verb@graphicx@ style file, which is a standard \LaTeX\ graphics tool 
for including \verb@eps@-figure files.

A figure containing only \LaTeX\ picture commands can be 
specified as
\begin{verbatim}
\begin{figure}[t]
  \setlength{\unitlength}{4cm}
  \centerline{
    \begin{picture}(1,1)(0,0)
      \linethickness{.5pt}
      \put(0,0){\framebox(1,1){}}
      \put(0.25,0.25){\circle*{.02}}
      \put(0.25,0.75){\circle*{.02}}
      \put(0.75,0.75){\circle*{.02}}
      \put(0.75,0.25){\circle*{.02}}
    \end{picture}}
  \caption{This is an example of a figure that uses only 
  the \LaTeX\ picture commands. It contains four points.}
  \label{fig:ExamplePicture}
\end{figure}
\end{verbatim}
%
\begin{figure}[t]
  \setlength{\unitlength}{4cm}
  \centerline{
    \begin{picture}(1,1)(0,0)
      \linethickness{.5pt}
      \put(0,0){\framebox(1,1){}}
      \put(0.25,0.25){\circle*{.02}}
      \put(0.25,0.75){\circle*{.02}}
      \put(0.75,0.75){\circle*{.02}}
      \put(0.75,0.25){\circle*{.02}}
    \end{picture}}
  \caption{This is an example of a figure that uses only 
  the \LaTeX\ picture commands. It contains four points.}
  \label{fig:ExamplePicture}
\end{figure}
%
Equations should be typed using the \LaTeX\ commands 
\verb@\begin{equation}@ \verb@...\end{equation}@. For example: 
\begin{equation}
   g({\cal C}) := \sum_{i=1}^m \sum_{k\in C_i} \|x_k - 
   \bar{x}_{C_i}\|^2.
\end{equation}
If you do not want the equation to be numbered, use \verb@\begin{displaymath}@
\verb@...\end{displaymath}@ 
instead of \verb@{equation}@.

\section{Submission of papers, deadlines}

Send your paper by e-mail to Prof. Dr. Martin Schader at
\begin{verbatim} 
mscha@wifo.uni-mannheim.de or gfkl2002@gfkl.de
\end{verbatim}
The deadline for submission is {\bf August 1, 2002}.
At that time, we only need a \verb@pdf@ or \verb@ps@ version of your correctly formatted paper for the review process. If the paper passes the review, we will also need the following: 
\begin{itemize}
\item your paper in \LaTeX,

\item the \verb@eps@ files for figures (if any), and

\item a list of approximately 10 keywords for the index of the volume.
\end{itemize}

While submitting their paper, authors automatically leave
the copyright to the editors or the publisher.

\section{Specifying references}

Please, use the following citation style within the text: Bock (1974) 
(in parentheses: Bock (1974)), Gaul and Schader (1994). 
In the case of more than two authors: first author et al.

The list of references at the end of your paper must be in {\bf 
alphabetical order}. 
Please, use the style of the following specimen which shows examples 
for the citation of books, articles, and papers in proceedings volumes.

\begin{thebibliography}{1}

\item[]
% Example for the citation of books:
BOCK, H.H. (1974): 
{\em Automatische Klassifikation}. 
Vandenhoeck \& Ruprecht, G\"ottingen. 

\item[]
% Example for the citation of articles:
GAUL, W. and SCHADER, M. (1994): 
Pyramidal Classification Based on Incomplete Dissimilarity Data. 
{\em Journal of Classification, 11, 171--193}. 

\item[]
% Example for the citation of papers in proceedings volumes:
IEZZI, D. and VICHI, M. (1999): 
Forecasting a Classification. 
In: M. Vichi and O. Opitz (Eds.): 
{\em Classification and Data Analysis}. 
Springer, Heidelberg, 27--34.

\end{thebibliography}

\end{document}


