\documentclass[runningheads]{d:/Latex/gfkl2002}
%\documentclass[12pt]{article}

%\textwidth 18cm
%\textheight 23cm 
%\topmargin 0 cm 
%\oddsidemargin  -1cm 
%\evensidemargin  -1cm
%\renewcommand{\baselinestretch}{1.1}     

                                % %%%%%%%%%%%%%%%%%%%%%%%%%%%%%%%%%%%%%%%%
                                % Graphics Gallery :                     %
                                % http://www.accessexcellence.com/AB/GG/ %
                                % %%%%%%%%%%%%%%%%%%%%%%%%%%%%%%%%%%%%%%%%


% Maths
\usepackage{amsmath} 
\usepackage{amsfonts} 
\usepackage{amssymb} 
\usepackage{psfig} 
\usepackage{graphicx} 
%\usepackage{d:/Latex/astats}

\newcommand{\Esp}{\mathbb{E}}
\newcommand{\Var}{\mathbb V}
\renewcommand{\a}{{\tt a}}
\newcommand{\cc}{{\tt c}}
\newcommand{\g}{{\tt g}}
\renewcommand{\t}{{\tt t}}

\renewcommand{\subsubsection}[1]{\medskip \noindent {\bf #1.}}
\renewcommand{\paragraph}[1]{\noindent {\it #1:}}

%%%%%%%%%%%%%%%%%%%%%%%%%%%%%%%%%%%%%%%%%%%%%%%%%%%%%%%%%%%%%%%%%%%%%%
%%%%%%%%%%%%%%%%%%%%%%%%%%%%%%%%%%%%%%%%%%%%%%%%%%%%%%%%%%%%%%%%%%%%%%
\begin{document}
%%%%%%%%%%%%%%%%%%%%%%%%%%%%%%%%%%%%%%%%%%%%%%%%%%%%%%%%%%%%%%%%%%%%%%
%%%%%%%%%%%%%%%%%%%%%%%%%%%%%%%%%%%%%%%%%%%%%%%%%%%%%%%%%%%%%%%%%%%%%%

\title*{Some Statistical Issues in Microarray Data Analysis}

\toctitle{Statistical Analysis of Microarray Data }

\titlerunning{Statistical Analysis of Microarray Data}

\author{St\'ephane Robin} 

\institute{INA-PG / INRA, Biom\'etrie \& Intelligence Artificielle, 16
  rue Claude Bernard, F-75005, Paris, {\sc France}}

\authorrunning{Robin}

\maketitle

\abstract{DNA chips give a direct access to the expression levels of
  thousands of genes at the same time. This promising technology is a
  key point of functional genomics. However, the abundance and
  variability of the data it provides require proper statistical
  analysis.\\
  We first introduce the DNA chip technology and present what
  biologists hope to learn thanks to it. Some statistical problems
  raised by these data are then discussed with references to recent
  bibliography.  }
%%%%%%%%%%%%%%%%%%%%%%%%%%%%%%%%%%%%%%%%%%%%%%%%%%%%%%%%%%%%%%%%%%%%%%
%%%%%%%%%%%%%%%%%%%%%%%%%%%%%%%%%%%%%%%%%%%%%%%%%%%%%%%%%%%%%%%%%%%%%%
\section{Introduction}
%%%%%%%%%%%%%%%%%%%%%%%%%%%%%%%%%%%%%%%%%%%%%%%%%%%%%%%%%%%%%%%%%%%%%%
%%%%%%%%%%%%%%%%%%%%%%%%%%%%%%%%%%%%%%%%%%%%%%%%%%%%%%%%%%%%%%%%%%%%%%
Among the recent advances in molecular biology, the microarray
technology seems to be one of the most promising.  This `high
throughput' technology allows to measure the expression level of
several thousands of genes in one single experiment. This information
is crucial in view of understanding the function of the genes and is,
therefore, a key point of functional genomics.

%%%%%%%%%%%%%%%%%%%%%%%%%%%%%%%%%%%%%%%%%%%%%%%%%%%%%%%%%%%%%%%%%%%%%%
\subsubsection{Large scaled sequencing programs}
%%%%%%%%%%%%%%%%%%%%%%%%%%%%%%%%%%%%%%%%%%%%%%%%%%%%%%%%%%%%%%%%%%%%%%
An major biological work has been made in the last fifteen years by
sequencing the genomes of numerous organisms. Sequencing means reading
a very long text written with the alphabet of famous four letters
(bases) $\{\a, \cc, \g, \t\}$. Genome lengths vary between
millions of bases for bacteria to billions for superior organisms.\\
The problem of finding genes in these sequences raised immediately
after their publication. Both biological and bioinformatic tools to
detect genes now exist; it is therefore possible to evaluate the
number of genes of a given species. Table \ref{Tab:NbGenes} shows how
this number can vary.

\begin{table}
  \centerline{
    \begin{tabular}{lc}
      Organism &  Number of genes \\
      \hline
      {\sl Escherichia coli} (Bacteria) & 4\;000 \\
      {\sl Saccharomyces cerevisiae} (Yeast) & 6\;000 \\
      {\sl Caenorhabditis elegans} (Nematode) & 19\;000 \\
      {\sl Drosophila melanogaster} (Fly) & 13\;000 \\
      {\sl Arabidopsis thaliana} (Plant) & 25--30\;000 \\
      Human & 30--50\;000 \\
    \end{tabular}
    }
  \caption{Approximate number of genes in several species}
  \label{Tab:NbGenes}
\end{table}

%%%%%%%%%%%%%%%%%%%%%%%%%%%%%%%%%%%%%%%%%%%%%%%%%%%%%%%%%%%%%%%%%%%%%%
\subsubsection{Functional genomics}
%%%%%%%%%%%%%%%%%%%%%%%%%%%%%%%%%%%%%%%%%%%%%%%%%%%%%%%%%%%%%%%%%%%%%%
One of the main issue for molecular biologists is now to understand
the functions of all these genes. Although a gene is mainly
characterized by its sequence (i.e. the portion of the genome sequence
that corresponds to it), its function can not be directly deduced from
this sequence. At the present time, in `superior' organisms, only 5 \%
of the genes have a `known' function that has been biologically
confirmed by some experimental or phenotypic evidence; 65 \% of the
genes have a predicted function (most of the time, the prediction is
based on an homology between the sequence of the gene and the sequence
of few genes the functions of which are known); the functions of the
remaining 30 \% are still unknown.  Biologists hope to be able to
understand gene functions thanks to an intensive use of the DNA chip
technology.

\medskip

Section \ref{Sec:DNAChips} recalls some basic biological principles
and presents DNA chips. Typical biological questions are addressed in
Section \ref{Sec:BioQuestions}. Section \ref{Sec:StatIssues} proposes
a overview of some statistical problems raised by microarray data,
emphasizing differential analysis.

%%%%%%%%%%%%%%%%%%%%%%%%%%%%%%%%%%%%%%%%%%%%%%%%%%%%%%%%%%%%%%%%%%%%%%
%%%%%%%%%%%%%%%%%%%%%%%%%%%%%%%%%%%%%%%%%%%%%%%%%%%%%%%%%%%%%%%%%%%%%%
\section{DNA chips}\label{Sec:DNAChips}
%%%%%%%%%%%%%%%%%%%%%%%%%%%%%%%%%%%%%%%%%%%%%%%%%%%%%%%%%%%%%%%%%%%%%%
%%%%%%%%%%%%%%%%%%%%%%%%%%%%%%%%%%%%%%%%%%%%%%%%%%%%%%%%%%%%%%%%%%%%%%
We briefly describe here the aim and principle of the DNA chip
technology.  A large presentation is proposed by \cite{BrB99}.

%%%%%%%%%%%%%%%%%%%%%%%%%%%%%%%%%%%%%%%%%%%%%%%%%%%%%%%%%%%%%%%%%%%%%%
\subsubsection{Central dogma of molecular biology}
%%%%%%%%%%%%%%%%%%%%%%%%%%%%%%%%%%%%%%%%%%%%%%%%%%%%%%%%%%%%%%%%%%%%%%
Figure \ref{Fig:CentralDogma} presents the two steps that leads
from the gene to the protein, i.e. to the molecule that has an
effective biological function. The gene DNA sequence is first copied
({\it transcripted}) and then translated (according to the genetic
code) into a protein. Depending on the condition or tissue, zero, one
or several copies of a given gene can be made. The expression level of
a gene is related to this number, i.e. to the concentration of
corresponding messenger RNA (mRNA) in the cell. The transcriptome of a
cell is define as the set of all the mRNA (transcripts) present in the
cell.
\begin{figure}[hbt]
  \centerline{
    \begin{tabular}{ccccc}
      \framebox{\begin{tabular}{c} {DNA} \\{molecule} \\ {(gene)}
        \end{tabular}}
      &
      {$\underrightarrow{~transcription~}$}
      &
      \framebox{\begin{tabular}{c} {messenger} \\ {RNA} \\ {(transcript)}
        \end{tabular}}
      &
      {$\underrightarrow{~translation~}$}
      &
      \framebox{\begin{tabular}{c} {Protein} \\ {(biological } \\ {function)} \end{tabular}}
    \end{tabular}
    }
  \caption{Two steps from gene to protein}
  \label{Fig:CentralDogma}
\end{figure} \\
Transcriptome analysis measures the abundance of the transcripts
corresponding to each genes and provides therefore an estimation of
their expression levels. DNA chips are designed to capture separately
transcripts corresponding to each of the genes.

%%%%%%%%%%%%%%%%%%%%%%%%%%%%%%%%%%%%%%%%%%%%%%%%%%%%%%%%%%%%%%%%%%%%%%
\subsubsection{DNA Chips}
%%%%%%%%%%%%%%%%%%%%%%%%%%%%%%%%%%%%%%%%%%%%%%%%%%%%%%%%%%%%%%%%%%%%%%
The separation of the transcripts is based on the hybridization
reaction that leads two complementary RNA (or DNA) fragments to match
spontaneously.  The fragments match according to the Watson-Crick rule
that associates {\a} with {\t} and {\cc} with {\g} (see Figure
\ref{Fig:Hybrid}).
\begin{figure}[hbt]
  \centerline{
    $
    \begin{array}{rcccccccccccccccccccl}
      \mbox{{target} (solution)}
      & ~ & \a & ~ & \t & ~ & \g  & ~ & \g  & ~ & \t & ~ & \a & ~ & \g  & ~ & \cc & ~ & \a \\
      & ~ & |  & ~ & |  & ~ & |   & ~ & |   & ~ & |  & ~ & |  & ~ & |   & ~ & |   & ~ & |  \\ 
      & ~ & \t & ~ & \a & ~ & \cc & ~ & \cc & ~ & \a & ~ & \t & ~ & \cc & ~ & \g  & ~ & \t & ~ &  
      \mbox{{probe} (chip)}
    \end{array}
    $
    }
  \caption{Hybridization of a single strain of DNA to its
    complementary}
  \label{Fig:Hybrid}
\end{figure}
Complementary fragments corresponding to each gene are spotted on a
chip; all this spots constitute an array. The solution (target)
containing transcripts is spread on the chip (probe). The quantity of
transcripts hybridized on each spot of the array measures the
expression level of the corresponding gene.

%%%%%%%%%%%%%%%%%%%%%%%%%%%%%%%%%%%%%%%%%%%%%%%%%%%%%%%%%%%%%%%%%%%%%%
\subsubsection{Quantification}
%%%%%%%%%%%%%%%%%%%%%%%%%%%%%%%%%%%%%%%%%%%%%%%%%%%%%%%%%%%%%%%%%%%%%%
Before being spread on the chip, transcripts are labeled in order to
be quantified. The two most popular labeling techniques are
fluorescence and radioactivity. Most of the time, chips used with
fluorescence are glass slides (`microarray') while chips used with
radioactivity are nylon membranes (`macroarray', see Figure
\ref{Fig:Chip}). A microarray can contain about 10\;000 spots; a
macroarray contains a bit less.\\
\begin{figure}[hbt]
  \centerline{
    \begin{tabular}{ccc}
      \includegraphics[height=5cm,width=5cm]{Rob02-GfKl-Fig3a.ps}
      & \hspace{1cm}
      & \includegraphics[height=5cm,width=5cm]{Rob02-GfKl-Fig3b.ps}  \\
      Glass slide (fluorescence) & & Nylon membrane (radioactivity)
    \end{tabular}
    }
  \caption{Partial pictures of a Microarray (left) and a macroarray
    (right). The left picture results from the superimposition of
    two pictures : one green and one red. Equal green and red
    signals result in yellow spots; absence of both green and red
    signals give black spots.}
  \label{Fig:Chip}
\end{figure} 
The fluorescence technique can use two different dyes at the same time
(green and red). In a typical experiment, transcripts extracted from
tumors are labeled with the red dye while normal transcripts are
labeled with the green dye; the two solutions are then spread on a
same chip.  The abundance of the transcripts of each type of cell is
quantified by exciting the chip with two different lasers, one green
and one red.  Unfortunately, there are many bias in the labeling
technique and normalization of the data is strongly required.  Most of
the time, the log-transform is applied to the signal in order to
stabilized its variance. In the following, all numeric values will
refer to log-signals.

%%%%%%%%%%%%%%%%%%%%%%%%%%%%%%%%%%%%%%%%%%%%%%%%%%%%%%%%%%%%%%%%%%%%%%
%%%%%%%%%%%%%%%%%%%%%%%%%%%%%%%%%%%%%%%%%%%%%%%%%%%%%%%%%%%%%%%%%%%%%%
\section{Biological questions}\label{Sec:BioQuestions}
%%%%%%%%%%%%%%%%%%%%%%%%%%%%%%%%%%%%%%%%%%%%%%%%%%%%%%%%%%%%%%%%%%%%%%
%%%%%%%%%%%%%%%%%%%%%%%%%%%%%%%%%%%%%%%%%%%%%%%%%%%%%%%%%%%%%%%%%%%%%%

A basic idea of functional genomics is that, to understand the
function of a gene, one need to know in which conditions or in which
tissues it is highly (or lowly) expressed.  The `high throughput'
technology of DNA chips allows to explore in a reasonable time (and
with a reasonable cost) many conditions or tissues. 

%%%%%%%%%%%%%%%%%%%%%%%%%%%%%%%%%%%%%%%%%%%%%%%%%%%%%%%%%%%%%%%%%%%%%%
\subsubsection{Expression profiles} 
%%%%%%%%%%%%%%%%%%%%%%%%%%%%%%%%%%%%%%%%%%%%%%%%%%%%%%%%%%%%%%%%%%%%%%
Historically, one of the first question raised by microarray data has
been to define groups (clusters) of genes having similar expression
profiles in a given set of conditions. Gene belonging to a same
cluster probably participate to the same physiologic mechanism or are
involved in the same regulation network (see below). The pioneer paper
of \cite{ESB98}, that presents several studies of this kind, has
revealed clustering techniques to the molecular biology
community.\\
The dual clustering question is also interesting when `conditions' are
patients. In their very famous paper, \cite{AED00} compare different
types of lymphoma. Within patients of one type, they detect two
subtypes of patients on the base of their expression profiles. Those
two subtypes of patients turn out to have very different responses to
the standard therapy.

%%%%%%%%%%%%%%%%%%%%%%%%%%%%%%%%%%%%%%%%%%%%%%%%%%%%%%%%%%%%%%%%%%%%%%
\subsubsection{Genes affected by a condition change} 
%%%%%%%%%%%%%%%%%%%%%%%%%%%%%%%%%%%%%%%%%%%%%%%%%%%%%%%%%%%%%%%%%%%%%%
A natural way to discover gene involved in a given mechanism is to
detect `differentially expressed' genes. For example, \cite{TTC01}
study human genes affected by ionizing radiations: this study is lead
by comparing the transcriptomes of exposed and non-exposed human
cells. This kind of study is sometimes called `differential analysis'.

%%%%%%%%%%%%%%%%%%%%%%%%%%%%%%%%%%%%%%%%%%%%%%%%%%%%%%%%%%%%%%%%%%%%%%
\subsubsection{Tissues or patients classification}
%%%%%%%%%%%%%%%%%%%%%%%%%%%%%%%%%%%%%%%%%%%%%%%%%%%%%%%%%%%%%%%%%%%%%%
The DNA chip technology can also be used for diagnosis since patients
can be characterized by their expression profiles. \cite{GST99} have
made the most famous study of that type: their aim was to classify
patients into one of two types of acute leukemia. Many supervised
classification techniques have been tested on their data. \\
The problem of selecting a small subset of genes sufficient to make a
good classification is also important in view of designing reduced
chips for routine analysis.

%%%%%%%%%%%%%%%%%%%%%%%%%%%%%%%%%%%%%%%%%%%%%%%%%%%%%%%%%%%%%%%%%%%%%%
\subsubsection{Regulation networks}
%%%%%%%%%%%%%%%%%%%%%%%%%%%%%%%%%%%%%%%%%%%%%%%%%%%%%%%%%%%%%%%%%%%%%%
The most exciting but also ambitious question for biologist is to
determine regulation activities (activation or inhibition) between
genes. On the base of many expression profiles, biologist hope to be
able to draw huge graphs (networks) showing, for example, that gene
$A$ induces gene $B$ that itself inhibits gene $C$, etc.  Such a
network would be a key to understand metabolic pathways.  However,
this project will probably require tremendous datasets and, at the
present time, only reduced networks can be analyzed.

%%%%%%%%%%%%%%%%%%%%%%%%%%%%%%%%%%%%%%%%%%%%%%%%%%%%%%%%%%%%%%%%%%%%%%
%%%%%%%%%%%%%%%%%%%%%%%%%%%%%%%%%%%%%%%%%%%%%%%%%%%%%%%%%%%%%%%%%%%%%%
\section{Statistical issues}\label{Sec:StatIssues}
%%%%%%%%%%%%%%%%%%%%%%%%%%%%%%%%%%%%%%%%%%%%%%%%%%%%%%%%%%%%%%%%%%%%%%
%%%%%%%%%%%%%%%%%%%%%%%%%%%%%%%%%%%%%%%%%%%%%%%%%%%%%%%%%%%%%%%%%%%%%%
Each biological question presented in Section \ref{Sec:BioQuestions}
can be reformulated in terms of statistical analysis that has to be
lead very carefully because of the high variability of microarray
data.  Although many of these analysis can be made with standard
statistical methods, the dimension of microarray data imply a new way
to look at standard statistical notions. For example, the distinction
between `individuals' and `variables' is not clear about genes: genes
will be considered as individuals in many clustering analysis and as
variables in supervised classification. \\
Furthermore, classical statistical problems as multiple comparison or
variable selection are restated by microarray data because we have to
deal here with several thousands of levels (for multiple comparison)
or variables (for variable selection) at the same time. This section
presents a non exhaustive list of problems. Crucial steps such as
probe sampling, design of experiments and image analysis are not
discussed here.

%%%%%%%%%%%%%%%%%%%%%%%%%%%%%%%%%%%%%%%%%%%%%%%%%%%%%%%%%%%%%%%%%%%%%%
%%%%%%%%%%%%%%%%%%%%%%%%%%%%%%%%%%%%%%%%%%%%%%%%%%%%%%%%%%%%%%%%%%%%%%
\subsubsection{Clustering}
%%%%%%%%%%%%%%%%%%%%%%%%%%%%%%%%%%%%%%%%%%%%%%%%%%%%%%%%%%%%%%%%%%%%%%
The definition of genes (or tissues) having similar profiles has been
one of the first problem to be treated with statistical methods. The
need of automatic clustering techniques appeared very early because of
the size of the datasets. The great popularity of the paper of
\cite{ESB98} is partly due the free `{\it Cluster}' software that was
associated to it (see web-sites given at the end of this paper).  This
software performs many analysis such as hierarchical clustering (based
on correlation coefficients), $K$-means, self-organized maps
(Kohonen's networks), etc.  Many biological paper present analysis
looking like Figure \ref{Fig:ESB98} (reproduced
from \cite{ESB98}) that are sometimes called `Eisenifications'!\\
\begin{figure}[hbt]
  \centerline{
    \includegraphics[height=12cm, width=5cm, angle=270, bbllx=22,
    bblly=15, bburx=300, bbury=825, clip=]{Rob02-GfKl-Fig4.ps}
    }
  \caption{Example of clustering of microarray data (reproduction of
    Fig. 1 of \cite{ESB98}). The red and green matrix represents the
    dataset: each column corresponds to a gene and each row to a
    condition (a time in this case); negative values are colored in
    green, positive are colored in red. A biological interpretation of
    clusters A to E is given in the original paper.}
  \label{Fig:ESB98}
\end{figure}
The limitations of theses approaches is that they do not take into
account the great experimental variability of these data, and because
the result is very method-dependent and need biological assessment.
Other approaches such as bootstrap on trees or mixture models are
proposed by statistical papers but they still have few applications in
the biological literature.

%%%%%%%%%%%%%%%%%%%%%%%%%%%%%%%%%%%%%%%%%%%%%%%%%%%%%%%%%%%%%%%%%%%%%%
%%%%%%%%%%%%%%%%%%%%%%%%%%%%%%%%%%%%%%%%%%%%%%%%%%%%%%%%%%%%%%%%%%%%%%
\subsubsection{Data normalization}
%%%%%%%%%%%%%%%%%%%%%%%%%%%%%%%%%%%%%%%%%%%%%%%%%%%%%%%%%%%%%%%%%%%%%%
All practitioners are now convinced that the variability of the data
can not be neglected. Many papers have shown that experimental
artifacts induce huge variations of the signal. For example,
\cite{SRD01} aim to detect genes affected be a change in the sulfur
source in {\sl B. subtilis}.  The analysis of variance table of this
experiment (not shown here) shows that the mean square of the
interaction of interest (Sulfur source$\times$Gene) is 10~000 times
smaller than the variability due to the date of the experiment. \\
This analysis (and many others) proves that the data have to be
normalized in some way. The analysis of variance model is now commonly
used to analyze such designs (see \cite{KeC01}). However, the status
of the different effects can be discussed between fixed and random.
More recently, mixed models have been introduced to deal with repeated
data (\cite{WGW01}).

%%%%%%%%%%%%%%%%%%%%%%%%%%%%%%%%%%%%%%%%%%%%%%%%%%%%%%%%%%%%%%%%%%%%%%
\paragraph{Bias in DNA labeling}
The analysis of variance model is often used to eliminate artifacts.
A typical experimental bias is observed in experiments using two dyes
(red and green) because the two dyes have not the same efficiency, and
because the difference depends on the gene. The correction of the mean
red/green difference can be made simply and with a good precision
thanks to the size of the dataset. However, the correction of this
bias for each gene can not be made so simply because of the small
number of (or even sometimes {`absence of')
  replicates.\\
  Figure \ref{Fig:Swap} shows that this bias seems to be related to
  the mean signal of the gene. Let us denote $R_g$ the signal obtained
  for gene $g$ in the condition labeled in red and $G_g$ the signal
  obtained for the same gene in the conditioned labeled in green.  The
  mean signal is $M_g = (R_g + G_g)/2$ and the (half) difference is
  $D_g = (R_g - G_g)/2$. \cite{DYC02} suggest to use the `loess'
  (\cite{Cle74}) robust regression $D_g = f(M_g)$ to correct the
  Gene$\times$Dye interaction. Such a correction reduces significantly
  the Gene$\times$Dye interaction and is proposed by many commercial
  softwares.
\begin{figure}[hbt]
  \centerline{
    \includegraphics[height=6cm, width=12cm]{Rob02-GfKl-Fig5.eps}
    }
  \caption{Plot of the half difference $D$ between the red and green signals
    versus their mean $M$ for a microarray of {\sl A. thaliana}
    (10~000 spots).  Solid line = loess regression curve.}
  \label{Fig:Swap}
\end{figure} 
                                
%%%%%%%%%%%%%%%%%%%%%%%%%%%%%%%%%%%%%%%%%%%%%%%%%%%%%%%%%%%%%%%%%%%%%%
%%%%%%%%%%%%%%%%%%%%%%%%%%%%%%%%%%%%%%%%%%%%%%%%%%%%%%%%%%%%%%%%%%%%%%
\subsubsection{Differential analysis}
%%%%%%%%%%%%%%%%%%%%%%%%%%%%%%%%%%%%%%%%%%%%%%%%%%%%%%%%%%%%%%%%%%%%%%
As said in Section \ref{Sec:BioQuestions}, the discovery of genes
having different expression levels under two conditions is one of the
typical question in microarray data analysis. Examples of typical
analysis including description of the experimental design, data
normalization and differential analysis can be found in \cite{DYC02}
or \cite{RLM02}.The question can be formulated in following way: let
$A$ and $B$ denote the two conditions and $A_{gi}$ denote the
expression level of gene $g$ in the $i$-th replicate in condition $A$
(resp. $B_{gi}$). The statistical problem is then to test the
hypothesis ${\bf H}_0$ = ``$A_{gi}$'s and $B_{gi}$'s have the same
distribution''. Let $\Delta_g$ denote the test statistic for gene $g$:
typically $\Delta_g = \bar{A}_g - \bar{B}_g$ where $\bar{A}_g$ is the
mean expression level of $g$ in condition $A$ (resp. $\bar{B}_g$).

%%%%%%%%%%%%%%%%%%%%%%%%%%%%%%%%%%%%%%%%%%%%%%%%%%%%%%%%%%%%%%%%%%%%%%
\paragraph{Distribution of $\Delta_g$ under ${\bf H}_0$} 
%%%%%%%%%%%%%%%%%%%%%%%%%%%%%%%%%%%%%%%%%%%%%%%%%%%%%%%%%%%%%%%%%%%%%%
A non parametric approach requires a set of non differentially
expressed genes to estimate distribution of $\Delta_g$ under ${\bf
  H}_0$. Generally, one assumes that only few genes are differentially
expressed, so that the distribution of $\Delta_g$ under ${\bf H}_0$ is
estimated using all the genes. \cite{DYC02}, \cite{TTC01} or, more
recently, \cite{StT01} propose several approaches to limit the
influence of differentially expressed genes in this estimation.

%%%%%%%%%%%%%%%%%%%%%%%%%%%%%%%%%%%%%%%%%%%%%%%%%%%%%%%%%%%%%%%%%%%%%%
\paragraph{Variance estimate} 
%%%%%%%%%%%%%%%%%%%%%%%%%%%%%%%%%%%%%%%%%%%%%%%%%%%%%%%%%%%%%%%%%%%%%%
In a parametric framework, the parameters characterizing this
distribution have to be estimated. The normal distribution of the data
is frequently assumed; in this framework, the question is then to
estimate the variance $\sigma_g^2$ of $\Delta_g$ that appears in the
statistic $T_g= \Delta_g / \sqrt{\hat{\sigma}_g}$. We present here
several possible approaches.
\begin{description}
\item[{\it Homoscedasticity:}] The simplest approach is based on the
  homoscedasticity hypothesis ``$\sigma_g \equiv \sigma$''. In this
  case, the common estimate $\hat{\sigma}$ has a high precision thanks
  to the large number of genes and the associated t-test is highly
  powerful.  In the data of \cite{SRD01}, a careful analysis leads to
  set apart about 50 genes with high variance $\sigma_g^2$ among
  4~600. For the to the remaining 4~550 the homoscedasticity
  hypothesis is quite reasonable. However, a strong heteroscedasticity
  is observed in many experiments and standard transforms (log,
  Box-Cox, etc) are unable to correct it.
\item[{\it Gene specific estimate:}] In case of heteroscedasticity, a
  specific estimate $\hat{\sigma}_g$ can be calculated for each gene.
  When the number of replicates is small (less than 5), this
  estimation is very bad and the power of the test is dramatically
  low. In the data of \cite{SRD01} again, this approach detects only 1
  differentially expressed gene instead of 50 under the
  homoscedasticity hypothesis.  This great loss of power suggests to
  look for an intermediate way between the two preceeding approaches
  (see \cite{RLM02} for a discussion).
\item[{\it Combined estimate:}] In order to stabilize the gene
  specific variance estimate $\hat{\sigma}_g$, \cite{TTC01} propose to
  add some constant to it and to use $s^2_g = \sigma^2_0 +
  \hat{\sigma}^2_g$.  $\sigma^2_0$ is chosen sufficiently high to
  avoid false positive due to too small estimate $\hat{\sigma}_g$.
\item[{\it Variance modeling:}] Figure \ref{Fig:Swap} suggests that
  the variance $\sigma^2_g$ is related to the mean expression level
  $M_g$. This relation is frequently observed and \cite{RLM02} propose
  a exponential modeling of the relation $\sigma^2_g = f(M_g)$ while
  \cite{HHS02} suggest a quadratic form.
\item[{\it Mixture model:}] The modeling of $\sigma^2_g$ can be made
  independently from the expression level by using a mixture model
  (see \cite{DRD02}). In this framework, genes are supposed to belong
  to one of $K$ classes, each characterized by a variance $s^2_k$
  ($k=1..K$).
\end{description}
In the last three approaches, the distribution of the statistic $T_g$
is still difficult to obtain and is generally approximated by a
Gaussian or a Student distribution. The comparison of all these
methods is also difficult to make, mainly because of the lack of test
datasets in which differentially expressed genes are clearly
identified.  Simulation comparisons are not really satisfying since
the simulation model has a major influence on the performance of the
different methods.  \cite{Pan02} compares the results obtained by
several methods on \cite{GST99} data.

%%%%%%%%%%%%%%%%%%%%%%%%%%%%%%%%%%%%%%%%%%%%%%%%%%%%%%%%%%%%%%%%%%%%%%
\paragraph{Mixture model / Bayesian classification} 
The hypothesis testing framework is not the only way to detect
differentially expressed genes. A mixture model approach is proposed
by \cite{ETG00} or \cite{PGA02}. The components of the mixture
correspond, for example, to up-regulated ($\Esp(\Delta_g)>0$),
down-regulated ($\Esp(\Delta_g)<0$) and non differentially expressed
($\Esp(\Delta_g)=0$) genes. Differentially expressed genes are then
detected according to their posterior probability of belonging to one
of the first two components.
  
%%%%%%%%%%%%%%%%%%%%%%%%%%%%%%%%%%%%%%%%%%%%%%%%%%%%%%%%%%%%%%%%%%%%%%
\subsubsection{Multiple comparison} 
%%%%%%%%%%%%%%%%%%%%%%%%%%%%%%%%%%%%%%%%%%%%%%%%%%%%%%%%%%%%%%%%%%%%%%
Differential analysis always imply a multiple testing problem since
several thousands of tests (one for each gene) are made at the same
time. Controlling the overall level or the false positive rate is one
of the most crucial statistical problem raised by microarray data. The
very popular approach (`SAM' for Statistical Analysis of Microarray,
\cite{TTC01}) is based on permutation tests. The same permutation is
applied on all the genes at the same time to preserve the correlation
structure, which is obviously necessary since the genes can not be
considered as independent. Yet, it seems that the estimation of the
False Discovery Rate (FDR) can be improved (\cite{StT01},
\cite{KTM01}).

%%%%%%%%%%%%%%%%%%%%%%%%%%%%%%%%%%%%%%%%%%%%%%%%%%%%%%%%%%%%%%%%%%%%%%
%%%%%%%%%%%%%%%%%%%%%%%%%%%%%%%%%%%%%%%%%%%%%%%%%%%%%%%%%%%%%%%%%%%%%%
\subsubsection{Classification}
%%%%%%%%%%%%%%%%%%%%%%%%%%%%%%%%%%%%%%%%%%%%%%%%%%%%%%%%%%%%%%%%%%%%%%
The problem of classifying tissues or patients is addressed by many
biological papers. Many supervised classification methods (Fisher
discriminant analysis, nearest neighbors, support vector machines
(SVM), classification trees, etc.) have been tested and all the
statisticians working in this domain know the \cite{GST99} or
\cite{AED00} datasets. \cite{DFS00} or \cite{BGL00} propose
comparisons. The statistical renewal of SVM came together with the
apparition of microarrays and these two techniques seems to be made
for each other since SVM generally perform very well on transcriptome
data. However, one of the key principle of SVM is to augment
artificially the dimension of the data space in order to make the
groups linearly separable. In microarray experiments, samples or
tissues (individuals) are much less numerous as genes (variables), so
the groups are linearly separable in the
original data space. \\
In several papers, the selection of a subset of discriminating genes
is made on the bases of a differential analysis. This approach is not
satisfying because of the correlation between genes that may lead to a
very redundant subsets. Stepwise selection can be used for Fisher
discriminant analysis.  A selection procedure for SVM is proposed by
\cite{WMC00} and applied to \cite{GST99} data by \cite{GWB02}. The
reduction of the number of genes improved the performance of the
classifier on the test data.

%%%%%%%%%%%%%%%%%%%%%%%%%%%%%%%%%%%%%%%%%%%%%%%%%%%%%%%%%%%%%%%%%%%%%%
\medskip
\subsubsection{Acknowledgments} 
%%%%%%%%%%%%%%%%%%%%%%%%%%%%%%%%%%%%%%%%%%%%%%%%%%%%%%%%%%%%%%%%%%%%%%
The author thanks Pr M. Eisen for authorizing the reproduction of
Figure \ref{Fig:ESB98} and Pr J.-J. Daudin for his comments.

%%%%%%%%%%%%%%%%%%%%%%%%%%%%%%%%%%%%%%%%%%%%%%%%%%%%%%%%%%%%%%%%%%%%%%
%%%%%%%%%%%%%%%%%%%%%%%%%%%%%%%%%%%%%%%%%%%%%%%%%%%%%%%%%%%%%%%%%%%%%%
\section*{Web-sites}
%%%%%%%%%%%%%%%%%%%%%%%%%%%%%%%%%%%%%%%%%%%%%%%%%%%%%%%%%%%%%%%%%%%%%%
%%%%%%%%%%%%%%%%%%%%%%%%%%%%%%%%%%%%%%%%%%%%%%%%%%%%%%%%%%%%%%%%%%%%%%
The intense activity of the statistical community about microarray
data makes it unreasonable to wait for the final publication of
papers. Published papers do not give a good idea of the up-to-date
bibliography. This motivates the presence of technical reports and
web-sites among the references. A frequent look at some web-sites is
therefore necessary. Of course, the following list is not exhaustive but
proposes some interesting pages in connection with the topics
discussed in this paper.
\begin{description}
\item[{\sc Stanford Microarray Database}] {\tt <www.dnachip.org/>}:
  test datasets (\cite{AED00}, \cite{GST99}, ...) can be downloaded
  from this database.
\item[{\sc Eisen-lab}] {\tt <rana.lbl.gov/>}: the `{\it Cluster}'
  software is available there.
\item[{\sc Gene Expression Data Analysis}] (Department of
  Biostatistics \& Medical Informatics, University of Wisconsin -
  Madison) {\tt <www.biostat.wisc.edu/geda/>}: the seminar page
  presents a good up-to-date bibliography.
\item[{\sc Terry Speed's Microarray Data Analysis Group}] {\tt
    <www.stat.berkeley.edu/ users/terry/zarray/Html/>}: this team is
  one of the most active in the field and proposes an overview of
  statistical problems.
\item[{\sc Significance Analysis of Microarrays}] {\tt
    <www-stat.stanford.edu/\~{ }tibs/SAM/>}: presentation of the SAM
  method for differential analysis.
\end{description}

%%%%%%%%%%%%%%%%%%%%%%%%%%%%%%%%%%%%%%%%%%%%%%%%%%%%%%%%%%%%%%%%%%%%%%
%%%%%%%%%%%%%%%%%%%%%%%%%%%%%%%%%%%%%%%%%%%%%%%%%%%%%%%%%%%%%%%%%%%%%%
%\bibliography{../AST}
%\bibliographystyle{d:/latex/astats}
\bibliographystyle{plain}
\begin{thebibliography}{Tototo}

\bibitem[Alizadeh {\em et~al.} (2000)]{AED00}
{\sc Alizadeh, A.}, {\sc Eisen, M.}, {\sc Davis, R.~E.}, {\sc Ma, C.~A.}, {\sc
  Lossos, I.}, {\sc Rosenwald, A.}, {\sc Boldrick, J.}, {\sc Sabet, H.}, {\sc
  Tran, T.~.}, {\sc Yu, X.}, {\sc Powell, J.}, {\sc Yang, L.}, {\sc Marti, G.},
  {\sc Moore, T.}, {\sc Hudson, J.}, {\sc Chan, W.~C.}, {\sc Greiner, T.~C.},
  {\sc Weissenberger, D.~D.}, {\sc Armitage, J.~O.}, {\sc Levy, R.}, {\sc
  Grever, M.~R.}, {\sc Byrd, J.~C.}, {\sc Botstein, D.}, {\sc Brown, P.~O.} and
  {\sc Staudt, L.~M.}
\newblock (2000).
\newblock Distinct types of diffuse large {B}-cell lymphoma identified by gene
  expression profiling.
\newblock {\em Nature}.
\newblock {\bf 403} 503--511.

\item[Brown {\em et~al.} (2000)]{BGL00}
{\sc Brown, M. P.~S.}, {\sc Grundy, W.~N.}, {\sc Lin, D.}, {\sc Cristianini,
  N.}, {\sc Sugnet, C.~W.}, {\sc Furey, T.~S.}, {\sc Ares, M.} and {\sc
  Haussler, D.}
\newblock (2000).
\newblock Knowledge-based analysis of microarray gene expression data by using
  support vector machines.
\newblock {\em Proc. Natl. Acad. Sci. USA}.
\newblock {\bf 97} 262--267.

\item[Brown and Botstein (1999)]{BrB99}
{\sc Brown, P.} and {\sc Botstein, D.}
\newblock (1999).
\newblock Exploring the new world of the genome with {DNA} microarrays.
\newblock {\em Nature Genetics}.
\newblock {\bf Supplement 21} 33--37.

\item[Cleveland (1979)]{Cle74}
{\sc Cleveland, W.~S.}
\newblock (1979).
\newblock Robust locally weighted regression and smoothing scatterplots.
\newblock {\em Journal of the American Statistical Association}.
\newblock {\bf 74} 829--836.

\item[Delmar {\em et~al.} (2002)]{DRD02}
{\sc Delmar, P.}, {\sc Robin, S.} and {\sc Daudin, J.-J.}
\newblock (2002).
\newblock Mixture model on the variance for the differential analysis of gene
  expression.
\newblock {\it submitted}.

\item[Dudoit {\em et~al.} (2000)]{DFS00}
{\sc Dudoit, S.}, {\sc Fridlyand, J.} and {\sc Speed, T.~P.}
\newblock (2000), Comparison of discrimination methods for the classification
  of tumors using gene expression data.
\newblock Technical Report 576, Statistics Department, University of
  California, Berkeley.

\item[Dudoit {\em et~al.} (2002)]{DYC02}
{\sc Dudoit, S.}, {\sc Yang, Y.}, {\sc Callow, M.~J.} and {\sc Speed, T.~P.}
\newblock (2002).
\newblock Statistical methods for identifying differentially expressed genes in
  replicated c{DNA} microarray experiments.
\newblock {\em Statistica Sinica}.
\newblock {\bf 12} 111--139.

\item[Efron {\em et~al.} (2000)]{ETG00}
{\sc Efron, B.}, {\sc Tibshirani, R.}, {\sc Goss, V.} and {\sc Chu, G.}
\newblock (2000), Microarrays and their use in a comparative experiment.
\newblock Technical report, Department of Statistics, Stanford University.
\newblock www-stat.stanford.edu/\~{ }tibs/research.html.

\item[Eisen {\em et~al.} (1998)]{ESB98}
{\sc Eisen, M.~B.}, {\sc Spellman, P.~T.}, {\sc Brown, P.~O.} and {\sc
  Botstein, D.}
\newblock (1998).
\newblock Cluster analysis and display of genome-wide expression patterns.
\newblock {\em Proc. Natl. Acad. Sci. USA}.
\newblock  14863--14868.

\item[Golub {\em et~al.} (1999)]{GST99}
{\sc Golub, T.~R.}, {\sc Slonim, D.~K.}, {\sc Tamayo, P.}, {\sc Huard, C.},
  {\sc Gaasenbeek, M.}, {\sc Mesirov, J.~P.}, {\sc Coller, H.}, {\sc Loh, M.},
  {\sc Downing, J.~R.}, {\sc Caligiuri, M.~A.}, {\sc Bloomfield, C.~D.} and
  {\sc Lander, E.~S.}
\newblock (1999).
\newblock Molecular classification of cancer: Class discovery and class
  prediction by gene expression.
\newblock {\em Science}.
\newblock {\bf 286} 531--537.

\item[Guyon {\em et~al.} (2002)]{GWB02}
{\sc Guyon, I.}, {\sc Weston, J.}, {\sc Barnhill, S.} and {\sc Vapnik, V.}
\newblock (2002).
\newblock Gene selection for cancer classification using support vector
  machines.
\newblock {\em Machine Learning}.
\newblock {\bf 46} 389--422.

\item[Huber {\em et~al.} (2002)]{HHS02}
{\sc Huber, W.}, {\sc von Heydebreck, A.}, {\sc S�ltmann, H.}, {\sc Poustka,
  A.} and {\sc Vingron, M.}
\newblock (2002).
\newblock Variance stabilization applied to microarray data calibration and to
  the quantification of differential expression.
\newblock {\em Bioinformatics}.
\newblock {\bf 18} 96--104.

\item[Kerr and Churchill (2001)]{KeC01}
{\sc Kerr, M.~K.} and {\sc Churchill, G.}
\newblock (2001).
\newblock Experimental design for gene expression microarrays.
\newblock {\em Biostatistics}.
\newblock {\bf 2} 183--201.

\item[Korn {\em et~al.} (2001)]{KTM01}
{\sc Korn, E.~L.}, {\sc Troendle, J.~F.}, {\sc McShane, L.~M.} and {\sc Simon,
  R.}
\newblock (2001), Controlling the number of false discoveries:application to
  high-dimensional genomic data.
\newblock Technical Report 003, National Cancer Institute, Biometric Research
  Branch, Dicision of Cancer Tratment and Diagnosis.
\newblock linus.nci.nih.gov/\~{ }brb/TechReport.htm.

\item[Pan (2002)]{Pan02}
{\sc Pan, W.}
\newblock (2002).
\newblock A comparative review of statistical methods for discovering
  differentially expressed genes in replicated microarray experiments.
\newblock {\em Bioinformatics}.
\newblock {\bf 18}~{\bf (4)} 546--554.

\item[Parmigiani {\em et~al.} (2002)]{PGA02}
{\sc Parmigiani, G.}, {\sc Garret, E.}, {\sc Anbazhagan, R.} and {\sc E., G.}
\newblock (2002).
\newblock A statistical framework for expression-based molecular classification
  in cancer.
\newblock {\em J. R. Statist. Soc. B}.
\newblock {\bf 64}~{\bf (4)} 1--20.

\item[Rudemo {\em et~al.} (2002)]{RLM02}
{\sc Rudemo, M.}, {\sc Lobovkina, T.}, {\sc Mostad, P.}, {\sc Scheidl, S.},
  {\sc Nilsson, S.} and {\sc Lindahl, P.}
\newblock (2002).
\newblock Variance models for microarray data.
\newblock http://www.math.chalmers.se/\~{ }rudemo/.

\item[Sekowska {\em et~al.} (2001)]{SRD01}
{\sc Sekowska, A.}, {\sc Robin, S.}, {\sc Daudin, J.-J.}, {\sc H\'enaut, A.}
  and {\sc Danchin, A.}
\newblock (2001).
\newblock Extracting biological information from {DNA} arrays: an unexpected
  link between arginine and methionine metabolism in {\sl {b}acillus subtilis}.
\newblock {\em Genome Biology}.
\newblock {\bf 2}~{\bf (6)}~{\tt
  http://genomebiology.com/2001/2/6/research/0019.1}.

\item[Storey and Tibshirani (2001)]{StT01}
{\sc Storey, J.~D.} and {\sc Tibshirani, R.}
\newblock (2001), Estimating false discovery rates under dependence, with
  applications to {DNA} microarrays.
\newblock Technical Report 2001-28, Department of Statistics, Stanford
  University.
\newblock {\tt www.stat.berkeley.edu/\~{ }storey/}.

\item[Tusher {\em et~al.} (2001)]{TTC01}
{\sc Tusher, V.~G.}, {\sc Tibshirani, R.} and {\sc Chu, G.}
\newblock (2001).
\newblock Significance analysis of microarrays applied to the ionizing
  radiation response.
\newblock {\em Proc. Natl. Acad. Sci. USA}.
\newblock {\bf 98} 5116--5121.

\item[Weston {\em et~al.} (2000)]{WMC00}
{\sc Weston, J.}, {\sc Mukherjee, S.}, {\sc Chapelle, O.}, {\sc Pontil, M.},
  {\sc Poggio, T.} and {\sc Vapnik, V.}
\newblock (2000).
\newblock Feature selection for {SVMs}.
\newblock In {\em {Neural Information Processing Systems}}, 668--674.
\newblock {\tt{http://www.ai.mit.edu/people/sayan/webPub/feature.pdf}}.

\item[Wolfinger {\em et~al.} (2001)]{WGW01}
{\sc Wolfinger, R.~D.}, {\sc Gibson, G.}, {\sc Wolfinger, E.}, {\sc Bennett,
  L.}, {\sc Hamadeh, H.}, {\sc Bushel, P.}, {\sc Afshari, C.} and {\sc Paules,
  R.~S.}
\newblock (2001).
\newblock Assessing gene significance from c{DNA} microarray expression data
  via mixed models.
\newblock {\em J. Comp. Biol.}
\newblock {\bf 8}~{\bf (6)} 625--637.

\end{thebibliography}

%%%%%%%%%%%%%%%%%%%%%%%%%%%%%%%%%%%%%%%%%%%%%%%%%%%%%%%%%%%%%%%%%%%%%%
%%%%%%%%%%%%%%%%%%%%%%%%%%%%%%%%%%%%%%%%%%%%%%%%%%%%%%%%%%%%%%%%%%%%%%
\end{document}
%%%%%%%%%%%%%%%%%%%%%%%%%%%%%%%%%%%%%%%%%%%%%%%%%%%%%%%%%%%%%%%%%%%%%%
%%%%%%%%%%%%%%%%%%%%%%%%%%%%%%%%%%%%%%%%%%%%%%%%%%%%%%%%%%%%%%%%%%%%%%


