\documentclass[12pt]{article}
%
% Retirez le caractere "%" au debut de la ligne ci--dessous si votre
% editeur de texte utilise des caracteres accentues 
% \usepackage[latin1]{inputenc}
%
% Retirez le caractere "%" au debut des lignes ci--dessous si vous
% utiisez les symboles et macros de l'AMS
% \usepackage{amsmath}
% \usepackage{amsfonts}
%
\pagestyle{empty}
%
% ATTENTION : les 3 commandes ci-dessous definissent un cadre de 10x16
% pour votre texte et ne doivent en aucun cas etre modifiees. 
%
\setlength{\textwidth}{16cm}
\setlength{\textheight}{10cm}
\setlength{\hoffset}{-1.4cm}
%
%
%
\begin{document} 
%
%%%%%%%%%%%%%%%%%%%%%%%%%%%%%%%%%%%%%%%%%%%%%%%%%%%%%%%%%%%%%%%%%%
%                                                                %
%           NE METTEZ PAS DE TITRE A CE RESUME                   %
%                                                                %
%%%%%%%%%%%%%%%%%%%%%%%%%%%%%%%%%%%%%%%%%%%%%%%%%%%%%%%%%%%%%%%%%%
%

% \noindent Le pr\'esent r\'esum\'e en fran\c cais est destin\'e au
% livret du participant o\`u il occupera, sur une demi--page A4, un
% cadre de 10cm~x~16cm, ce qui correspond approximativement \`a 15
% lignes ou 250 mots. Il devra permettre au lecteur de se faire
% rapidement une id\'ee pr\'ecise du sujet que vous traitez. 

% \noindent Vous n'avez \`a mentionner ici ni le titre, ni l'orateur, ni
% le(s) co--auteur(s). Si votre communication est accept\'ee, ils seront
% lus, ainsi que la localisation de l'expos\'e (salle, date et heure)
% dans la base de donn\'ees du congr\`es.

% \noindent Vous n'avez pas non plus \`a donner ici de traduction en
% anglais, ni de  mots cl\'es. Par contre, vous pouvez \'eventuellement
% indiquer une r\'ef\'erence MAJEURE de mani\`ere \`a situer votre
% travail.
% \medskip

% \noindent ATTENTION : le gabarit {\tt TexteCourt.tex} est con\c cu pour
% produire un texte contenu dans un cadre de hauteur 10~cm et de largeur
% 16~cm. Si votre r\'esum\'e est trop long pour ce cadre, il sera
% rejet\'e. 
% \medskip
%  
% \noindent{\bf R\'ef\'erence}

% \noindent Noteur, U. N. (2003) Sur l'int\'er\^et des r\'esum\'es. {\it
% Revue des Organisateurs de Congr\`es}, 34, 67--89.  
%
%

\bigskip Le probl\`eme g\'en\'eral des comparaisons multiples est de
contr\^oler le nombre de faux positifs quand on effectue
simultan\'ement un grand nombre de tests. On peut consid\'erer ce
probl\`eme dans le cadre d'un mod\`ele de m\'elange en profitant du
fait que la distribution des probabilit\'es critiques est connue sous
$H_0$. Nous aboutissons ainsi a un mod\`ele de la forme $ g(x) = a
f(x) + (1-a) \phi(x) $ o\`u $a$ est la proportion inconnue de
d'hypoth\`eses $H_1$, $f$ est une densit\'e quelconque et $\phi$ est une
densit\'e parfaitement sp\'ecifi\'ee.

Nous proposons d'estimer la densit\'e $f$ par un estimateur \`a noyaux
pond\'er\'es, les poids \'etant estim\'es de fa\c con adaptative.
Nous montrons que l'algorithme d'estimation converge vers un optimum
unique. Nous d\'eduisons une estimation du taux de faux positifs
($FDR$). Le mod\`ele de m\'elange fournit de plus, au travers des
probabilit\'es {\it a posteriori} une estimation naturelle du $FDR$
local associ\'e \`a chaque hypoth\`ese test\'ee.

Nous pr\'esentons des applications de cette proc\'edure \`a des
donn\'ees simul\'ees ainsi qu'\`a des donn\'ees issues d'exp\'eriences
de transcriptome.



\end{document} 


%%%%%%%%%%%%%%%%%%%%%%%%%%%%%%%%%%%%%%%%%%%%%%%
%%%%%%%%%%%%%%%%%%%%%%%%%%%%%%%%%%%%%%%%%%%%%%%
% \centerline{\bf \Large Une approche semi-param\'etrique pour les mod\`eles
%   de m\'elange }

% \smallskip \centerline{\bf \Large avec une composante connue~:
%   Application au FDR local}

% \smallskip \centerline{{\sc S. Robin$^1$, A. Bar-Hen$^2$, J.-J.
%     Daudin$^1$, L. Pierre$^3$}}

% \smallskip \centerline{($^1$) INA-PG / INRA, ($^2$), Univ. Marseille
%   III, ($^3$) Univ. Paris X}

% \bigskip {\sl Autres mots cl\'es~: Comparaison multiple, Estimation de
%   la densit\'e}
%%%%%%%%%%%%%%%%%%%%%%%%%%%%%%%%%%%%%%%%%%%%%%%
%%%%%%%%%%%%%%%%%%%%%%%%%%%%%%%%%%%%%%%%%%%%%%%


