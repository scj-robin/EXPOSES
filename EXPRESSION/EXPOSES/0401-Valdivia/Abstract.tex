\documentclass[a4paper, 12pt]{article}

\usepackage{amsfonts, amsmath, amssymb}
\usepackage{astats}

\usepackage{enumerate}
\usepackage{epsfig}
%\usepackage[french]{babel}
\usepackage[latin1]{inputenc}

\textwidth 16cm
\textheight 24cm 
\topmargin -1 cm 
\oddsidemargin 0cm 
\evensidemargin 0cm

%\renewcommand{\baselinestretch}{1.3}

%%%%%%%%%%%%%%%%%%%%%%%%%%%%%%%%%%%%%%%%%%%%%%%%%%%%%%%%%%%%
%%%%%%%%%%%%%%%%%%%%%%%%%%%%%%%%%%%%%%%%%%%%%%%%%%%%%%%%%%%%
\begin{document}
%%%%%%%%%%%%%%%%%%%%%%%%%%%%%%%%%%%%%%%%%%%%%%%%%%%%%%%%%%%%
%%%%%%%%%%%%%%%%%%%%%%%%%%%%%%%%%%%%%%%%%%%%%%%%%%%%%%%%%%%%
\begin{center}
{\Large \bf Statistical analysis of microarray data} \\
\bigskip
{\large \sc S. Robin} \\
\bigskip
{INA PG / INRA Biom�trie, Paris, {\sc France}} \\
\bigskip
{\tt robin@inapg.inra.fr} \\
\bigskip
{\large \sl Preliminary program: \today}
\bigskip
\bigskip
\bigskip
\end{center}

The aim of this course is to present the main issues in statistical
analysis of microarray. Most popular and standard tools will be
presented and discussed and some more recent approaches will be
introduced. This program gives a sketch of the content of each of the
four lectures.

%%%%%%%%%%%%%%%%%%%%%%%%%%%%%%%%%%%%%%%%%%%%%%%%%%%%%%%%%%%%
\section{Introduction and preliminary studies}
%%%%%%%%%%%%%%%%%%%%%%%%%%%%%%%%%%%%%%%%%%%%%%%%%%%%%%%%%%%%
The first lecture will be devoted to a general introduction to
microarray technology, to underlying biological issues and their
translation into statistical terms. Planing of experiments and
normalization techniques will be presented as necessary preliminary
steps before further statistical analysis.
\begin{description}
\item[Biological context:] functional genomics.
\item[Microarray technology:] hybridization reaction, different
  technologies.
\item[Statistical issues:] the high variability of the data requires
  careful statistical analysis. Typical problems are class discovery,
  class comparison and class prediction.
\item[Experimental designs:] how to organize the experiments in view
  of analyzing a specific contrast. A special attention will be paid
  to the glass slide technology (loop designs, star designs, swaps).
\item[Normalization:] due to numerous technical biases, data have to
  be normalized. Which effects should be normalized and by which mean?
\end{description}

\paragraph{\slshape Background:}
{Elementary biology, basic statistics and analysis of variance.}

\paragraph{\slshape Bibliography:} \cite{BrB99}, \cite{Chu02}, \cite{KAB02}

%%%%%%%%%%%%%%%%%%%%%%%%%%%%%%%%%%%%%%%%%%%%%%%%%%%%%%%%%%%%
\section{Class discovery}
%%%%%%%%%%%%%%%%%%%%%%%%%%%%%%%%%%%%%%%%%%%%%%%%%%%%%%%%%%%%
The second lecture will deal with the discovery of groups of genes
having similar expression profile in a given set of conditions.
Clustering techniques have become most popular tools in microarray
data analysis. However their relevance is not always obvious and
underlying hypotheses are generally not clearly stated.
\begin{description}
\item[Similarity and dissimilarity:] the definition of groups of genes
  is always based on some (arbitrary) measurement of the `distance'
  between genes.
\item[Hierarchical clustering:] this technique, which provides the famous
  `Eisen' plots, will be described in details and discussed.
\item[$K$ means] is another frequently used algorithm. Its properties
  will be presented, partly as an introduction to mixture models.
\item[Mixture models] allow to take into account the variability of
  the data in a clustering procedure. The
  `expectation-maximization'(EM) algorithm will be described.
\item[How many groups?] Some standard criterion to choose the number
  of groups will be presented.
\end{description}

\paragraph{\slshape Background:}
{Statistics and probability (likelihood, conditional
  probability and Bayes rule).}

\paragraph{\slshape Bibliography:} \cite{ESB98}, \cite{GST99}

%%%%%%%%%%%%%%%%%%%%%%%%%%%%%%%%%%%%%%%%%%%%%%%%%%%%%%%%%%%%
\section{Class comparison}
%%%%%%%%%%%%%%%%%%%%%%%%%%%%%%%%%%%%%%%%%%%%%%%%%%%%%%%%%%%%
A typical problem is the detection of genes associated with some
disease. Such genes will be revealed by their differential expression
between, says, normal tissues and tumors. This is often called
differential analysis.
\begin{description}
\item[Hypothesis testing:] the basic principles of statistical testing
  will be briefly reminded (test statistic, risk, power, etc).
\item[Variance modeling:] the estimation of the residual variability
  has strong consequences on the power of the test. Gene by gene
  estimates lead to un-powerful tests while unique estimate is often
  unrealistic. Intermediate solution must be found
\item[Multiple testing:] performing thousands of tests at the same
  time may lead to numerous false positives. Some multiple testing
  procedure will be discussed.
\end{description}

\paragraph{\slshape Background:}
{Hypothesis testing, risks, $t$-test, $p$-value.}

\paragraph{\slshape Bibliography:} \cite{DYC02}, \cite{RLM02},
\cite{DSB02}, \cite{Pan02}

%%%%%%%%%%%%%%%%%%%%%%%%%%%%%%%%%%%%%%%%%%%%%%%%%%%%%%%%%%%%
\section{Class prediction}
%%%%%%%%%%%%%%%%%%%%%%%%%%%%%%%%%%%%%%%%%%%%%%%%%%%%%%%%%%%%
A last typical problem is the prediction of the type of tissue (normal
or tumor) according to their expression profiles. Classifiers are
fitted to tissues of known types and then applied to unknown tissues.
\begin{description}
\item[Discriminant analysis:] linear (LDA) and quadratic (QDA) methods
  are based on a gaussian modeling of the data.
\item[`Algorithmic' approaches:] support vector machines (SVM),
  classification trees (CART) or other methods that do not make
  distribution assumptions have often better performances.
\item[Generalization risk:] classifiers always present good
  performances on the training data set. The problem is to evaluate
  error risk on new data.
\item[Gene selection:] how to select a reduced subset of genes to
  predict the type of the tissue with a reasonably low error rate.
\end{description}

\paragraph{\slshape Background:}
{Conditional probability, Bayes rule and multivariate statistics.
  Other notions will be introduced during the lecture.}

\paragraph{\slshape Bibliography:} \cite{GWB02}, \cite{DFS02}, \cite{BGL00}

%%%%%%%%%%%%%%%%%%%%%%%%%%%%%%%%%%%%%%%%%%%%%%%%%%%%%%%%%%%%
\bibliography{../../../AST}
\bibliographystyle{astats}
%%%%%%%%%%%%%%%%%%%%%%%%%%%%%%%%%%%%%%%%%%%%%%%%%%%%%%%%%%%%

\paragraph{Other sources.}
Interesting papers appear also in {\em Bioinformatics} or {\em J.
  Comp. Biology}. Vol. {\bf 12} of Statistica Sinica and {\bf 32} of
Nature Genetics in 2002 were specially devoted to the analysis of
microarray data.  Recent and up-to-date methods are generally
available on the web, several months (or years!) before their
publication in regular journals.

Some groups working on statistical analysis of microarray data give
access to their manuscripts or slides. See for example  \\
$\bullet$ \quad {\tt www.bioconductor.org}, \\
$\bullet$ \quad {\tt www.stat.berkeley.edu/users/terry/zarray/Html}, \\
$\bullet$ \quad {\tt www.jax.org/staff/churchill/labsite/research/expression/}.

%%%%%%%%%%%%%%%%%%%%%%%%%%%%%%%%%%%%%%%%%%%%%%%%%%%%%%%%%%%%
%%%%%%%%%%%%%%%%%%%%%%%%%%%%%%%%%%%%%%%%%%%%%%%%%%%%%%%%%%%%
\end{document}
%%%%%%%%%%%%%%%%%%%%%%%%%%%%%%%%%%%%%%%%%%%%%%%%%%%%%%%%%%%%
%%%%%%%%%%%%%%%%%%%%%%%%%%%%%%%%%%%%%%%%%%%%%%%%%%%%%%%%%%%%

