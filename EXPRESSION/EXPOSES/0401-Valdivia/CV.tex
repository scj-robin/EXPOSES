\documentclass[a4paper, 10pt]{article}

%\usepackage[french]{babel}
\usepackage[latin1]{inputenc}

\textwidth 16cm
\textheight 26cm 
\topmargin -1 cm 
\oddsidemargin 0cm 
\evensidemargin 0cm

%\renewcommand{\baselinestretch}{1.3}

%%%%%%%%%%%%%%%%%%%%%%%%%%%%%%%%%%%%%%%%%%%%%%%%%%%%%%%%%%%%
%%%%%%%%%%%%%%%%%%%%%%%%%%%%%%%%%%%%%%%%%%%%%%%%%%%%%%%%%%%%
\begin{document}
\pagestyle{empty}
%%%%%%%%%%%%%%%%%%%%%%%%%%%%%%%%%%%%%%%%%%%%%%%%%%%%%%%%%%%%
%%%%%%%%%%%%%%%%%%%%%%%%%%%%%%%%%%%%%%%%%%%%%%%%%%%%%%%%%%%%
\subsection*{Curriculum Vitae of St�phane Robin}

\paragraph{Position:} 
Professor of Statistics and Applied Mathematics at the Institut
National Agronomique Paris-Grignon, Paris, {\sc France}

\paragraph{Email:} {\tt robin@inapg.inra.fr}

\paragraph{Cursus:}~\\ 
{1989:} Graduate of the Ecole Nationale de la Statistique et de
l'Administration Economique \\
{1989:} Master in Statistics (university Paris VI,
Jussieu) \\
{1993:} PhD on ``Sensitivity analysis of some non-parametric tests for
censored data'' \\
{2002:} HDR on ``Motifs distribution in DNA sequences''


\subsection*{Research interests}
All my research activities are related to the biometric field. Since
several years I have been mainly working on statistical problems in
molecular biology with two main topics: \\
$\bullet$~words or motifs distribution in biological sequences, \\
$\bullet$~statistical analysis of microarray data.

\paragraph{Word statistics in DNA sequence:}
In most species, crucial functions such as DNA repairing, gene
expression, genome protection against restriction enzymes, etc are
controlled by short sequences of nucleotides called words or motifs.
For some problems, these words are unknown but can be detected by
studying their frequency or their presence/absence in specific
regions. In other problems, the word is known and the analysis of its
distribution along the genome may help to discover new regions of
interest. I worked on the exact distribution of the positions and
distances between occurrences of a motif in a random sequence of
letters. These results have been obtained in the Markov chain
framework using a generating function approach. I also work on the
detection of poor or rich regions using (heterogeneous) compound
Poisson processes.

\paragraph{Statistics and microarrays:}
Due to their great quantity and high variability, gene expression data
require an intensive use of statistical analysis. Differential
analysis aims to discover genes that are affected by a condition
change. In this field I work on \\
$\bullet$~variance modeling for $t$-test,\\
$\bullet$~an iterative procedure controlling the FDR. \\
I am also interested in the class prediction problem and, more
precisely, in the selection of a relevant subset of genes that would
provide reliable predictions based on discriminant analysis or support
vector machines.

\paragraph{Other topics:}
Away from the genomic field, I also worked on the comparison of plant
varieties. The problem is to estimate the phenotypic similarity
between two varieties on the basis of genetic data (numerous markers).
The method we
derived involves multivariate generalized linear models. \\
I also worked on the comparison of plant lines on the basis of
infra-red spectra. The common charateristic is the high dimension of
the data.

\subsection*{Some publications}
$\bullet$~{\sc Nuel, G.}, {\sc Robin, S.} et {\sc Baril, C.}  \newblock
  (2001a).  \newblock Predicting phenotypic distances using a linear
  model: the case of varietal distinctness.  \newblock {\em J. Appl.
    Statist.}  \newblock {\bf 28}~{\bf (5)} 607--621.  \\
$\bullet$~{\sc Robin, S.}  \newblock (2002).  \newblock A compound
  {P}oisson model for words occurrences in {DNA} sequences.  \newblock
  {\em J.  R. Statist. Soc. C}.  \newblock {\bf 51} 437--451.  \\
$\bullet$~{\sc Robin, S.}  \newblock (2003).  \newblock Some statistical
  issues in microarray data analysis.  \newblock In {\em Between data
    science and applied data analysis}, (Martin,
  S. and Gaul, W. and Vichi, M.,  ed) \newblock 337--347.  \\
$\bullet$~{\sc Robin, S.} et {\sc Daudin, J.-J.}  \newblock (1999).
  \newblock Exact distribution of word occurrences in a random
  sequence of letters.  \newblock {\em J. Appl. Prob.}  \newblock {\bf
    36} 179--193.  \\
$\bullet$~{\sc Robin, S.}, {\sc Daudin, J.-J.}, {\sc Richard, H.}, {\sc
    Sagot, M.-F.} et {\sc Schbath, S.}  \newblock (2002).  \newblock
  Occurrence probability of structured motifs in random sequences.
  \newblock {\em J. Comp. Biol.}  \newblock {\bf 9} 761--773.  \\
$\bullet$~{\sc Sekowska, A.}, {\sc Robin, S.}, {\sc Daudin, J.-J.}, {\sc
    H�naut, A.} et {\sc Danchin, A.}  \newblock (2001).  \newblock
  Extracting biological information from {DNA} arrays: an unexpected
  link between arginine and methionine metabolism in {\sl {b}acillus
    subtilis}.  \newblock {\em Genome Biology}.  \newblock {\bf
    2}~{\bf (6)}. {\tt genomebiology.com/2001/2/6/research/0019.1.} %ok

%%%%%%%%%%%%%%%%%%%%%%%%%%%%%%%%%%%%%%%%%%%%%%%%%%%%%%%%%%%%
%%%%%%%%%%%%%%%%%%%%%%%%%%%%%%%%%%%%%%%%%%%%%%%%%%%%%%%%%%%%
\end{document}
%%%%%%%%%%%%%%%%%%%%%%%%%%%%%%%%%%%%%%%%%%%%%%%%%%%%%%%%%%%%
%%%%%%%%%%%%%%%%%%%%%%%%%%%%%%%%%%%%%%%%%%%%%%%%%%%%%%%%%%%%

