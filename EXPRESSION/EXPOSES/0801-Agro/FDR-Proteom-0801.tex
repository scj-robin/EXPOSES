\documentclass[dvips, lscape]{foils}
%\documentclass[dvips, french]{slides}
\textwidth 18.5cm
\textheight 25cm 
\topmargin -1cm 
\oddsidemargin  -1cm 
\evensidemargin  -1cm

% Maths
\usepackage{amsfonts, amsmath, amssymb}

\newcommand{\coefbin}[2]{\left( 
    \begin{array}{c} #1 \\ #2 \end{array} 
  \right)}
\newcommand{\Dcal}{\mathcal{D}}
\newcommand{\Ecal}{\mathcal{E}}
\newcommand{\Ncal}{\mathcal{N}}
\newcommand{\Hbf}{{\bf H}}
\newcommand{\Bcal}{\mathcal{B}}
\newcommand{\Fcal}{\mathcal{F}}
\newcommand{\Lcal}{\mathcal{L}}
\newcommand{\Tcal}{\mathcal{T}}
\newcommand{\Ucal}{\mathcal{U}}
\newcommand{\alphabf}{\mbox{\mathversion{bold}{$\alpha$}}}
\newcommand{\betabf}{\mbox{\mathversion{bold}{$\beta$}}}
\newcommand{\gammabf}{\mbox{\mathversion{bold}{$\gamma$}}}
\newcommand{\psibf}{\mbox{\mathversion{bold}{$\psi$}}}
\newcommand{\taubf}{\mbox{\mathversion{bold}{$\tau$}}}
\newcommand{\Rbb}{\mathbb{R}}
\newcommand{\Sbf}{{\bf S}}
% \newcommand{\bps}{\mbox{bps}}
\newcommand{\ubf}{{\bf u}}
\newcommand{\vbf}{{\bf v}}
\newcommand{\Esp}{{\mathbb E}}
% \newcommand{\Var}{{\mathbb V}}
\newcommand{\Ibb}{{\mathbb I}}
%\newcommand{\liste}{$\bullet \quad$}
\newcommand{\lFDR}{\ell FDR}

% Couleur et graphiques
\usepackage{color}
\usepackage{graphics}
\usepackage{epsfig} 
\usepackage{pstcol}

% Texte
\usepackage{lscape}
\usepackage{../../../../Latex/fancyheadings, rotating, enumerate}
%\usepackage[french]{babel}
\usepackage[latin1]{inputenc}
\definecolor{darkgreen}{cmyk}{0.5, 0, 0.5, 0.5}
\definecolor{orange}{cmyk}{0, 0.6, 0.8, 0}
\definecolor{jaune}{cmyk}{0, 0.5, 0.5, 0}
\newcommand{\textblue}[1]{\textcolor{blue}{#1}}
\newcommand{\textred}[1]{\textcolor{red}{#1}}
\newcommand{\textgreen}[1]{\textcolor{green}{ #1}}
\newcommand{\textlightgreen}[1]{\textcolor{green}{#1}}
\newcommand{\textdarkgreen}[1]{\textcolor{darkgreen}{#1}}
\newcommand{\textorange}[1]{\textcolor{orange}{#1}}
\newcommand{\textyellow}[1]{\textcolor{yellow}{#1}}
\newcommand{\refer}[2]{{\sl #1}}
\newcommand{\emphase}[1]{\textblue{#1}}

% Sections
%\newcommand{\chapter}[1]{\centerline{\LARGE \textblue{#1}}}
% \newcommand{\section}[1]{\centerline{\Large \textblue{#1}}}
% \newcommand{\subsection}[1]{\noindent{\Large \textblue{#1}}}
% \newcommand{\subsubsection}[1]{\noindent{\large \textblue{#1}}}
% \newcommand{\paragraph}[1]{\noindent {\textblue{#1}}}
% Sectionsred
\newcommand{\chapter}[1]{
  \addtocounter{chapter}{1}
  \setcounter{section}{0}
  \setcounter{subsection}{0}
%   {\centerline{\LARGE \textblue{\arabic{chapter} - #1}}}
  {\centerline{\LARGE \textblue{#1}}}
  }
\newcommand{\section}[1]{
  \addtocounter{section}{1}
  \setcounter{subsection}{0}
%   {\centerline{\Large \textblue{\arabic{chapter}.\arabic{section} - #1}}}
  {\centerline{\Large \textblue{#1}}}
  }
\newcommand{\subsection}[1]{
  \addtocounter{subsection}{1}
%   {\noindent{\large \textblue{\arabic{chapter}.\arabic{section}.\arabic{subsection} - #1}}}
  {\noindent{\large \textblue{#1}}}
  }
\newcommand{\paragraph}[1]{\noindent{\textblue{#1}}}

%%%%%%%%%%%%%%%%%%%%%%%%%%%%%%%%%%%%%%%%%%%%%%%%%%%%%%%%%%%%%%%%%%%%%%
%%%%%%%%%%%%%%%%%%%%%%%%%%%%%%%%%%%%%%%%%%%%%%%%%%%%%%%%%%%%%%%%%%%%%%
%%%%%%%%%%%%%%%%%%%%%%%%%%%%%%%%%%%%%%%%%%%%%%%%%%%%%%%%%%%%%%%%%%%%%%
%%%%%%%%%%%%%%%%%%%%%%%%%%%%%%%%%%%%%%%%%%%%%%%%%%%%%%%%%%%%%%%%%%%%%%
\begin{document}
%%%%%%%%%%%%%%%%%%%%%%%%%%%%%%%%%%%%%%%%%%%%%%%%%%%%%%%%%%%%%%%%%%%%%%
%%%%%%%%%%%%%%%%%%%%%%%%%%%%%%%%%%%%%%%%%%%%%%%%%%%%%%%%%%%%%%%%%%%%%%
%%%%%%%%%%%%%%%%%%%%%%%%%%%%%%%%%%%%%%%%%%%%%%%%%%%%%%%%%%%%%%%%%%%%%%
%%%%%%%%%%%%%%%%%%%%%%%%%%%%%%%%%%%%%%%%%%%%%%%%%%%%%%%%%%%%%%%%%%%%%%
\landscape
\newcounter{chapter}
\newcounter{section}
\newcounter{subsection}
\setcounter{chapter}{0}
\headrulewidth 0pt 
\pagestyle{fancy} 
\cfoot{}
\rfoot{\begin{rotate}{90}{
      \hspace{1cm} \tiny S. Robin: Multiple Testing
      }\end{rotate}}
\rhead{\begin{rotate}{90}{
      \hspace{-.5cm} \tiny \thepage
      }\end{rotate}}

%%%%%%%%%%%%%%%%%%%%%%%%%%%%%%%%%%%%%%%%%%%%%%%%%%%%%%%%%%%%%%%%%%%%%%
%%%%%%%%%%%%%%%%%%%%%%%%%%%%%%%%%%%%%%%%%%%%%%%%%%%%%%%%%%%%%%%%%%%%%%
\begin{center}
  \textblue{\LARGE Multiple Testing}

  \textblue{\LARGE in High Throughput Data Analysis} 

   \vspace{1cm}
   {\large S. {Robin}} \\
   robin@agroparistech.fr

   {UMR 518 AgroParisTech / INRA MIA} \\
   {Statistics for Systems Biology group}
   
    \vspace{1cm}
    {Journ�e 'Prot�ome Vert'} \\
    {Paris, 16 Janvier, 2008, AgroParisTech}
\end{center}

\vspace{1cm}
\paragraph{Outline}
$$
\begin{tabular}{lcl}
  1 - High Throughput Data & \qquad \qquad \qquad & 3 - False Discovery Rate \\
  \\
  2 - Multiple Testing & & 4 - Local FDR
\end{tabular}
$$


%%%%%%%%%%%%%%%%%%%%%%%%%%%%%%%%%%%%%%%%%%%%%%%%%%%%%%%%%%%%%%%%%%%%%%
%%%%%%%%%%%%%%%%%%%%%%%%%%%%%%%%%%%%%%%%%%%%%%%%%%%%%%%%%%%%%%%%%%%%%%
\newpage
\chapter{High Throughput Data} 
\bigskip
%%%%%%%%%%%%%%%%%%%%%%%%%%%%%%%%%%%%%%%%%%%%%%%%%%%%%%%%%%%%%%%%%%%%%%
%%%%%%%%%%%%%%%%%%%%%%%%%%%%%%%%%%%%%%%%%%%%%%%%%%%%%%%%%%%%%%%%%%%%%%

\paragraph{Proteomic, transcriptomic approaches:} Measure the
expression level of a large number (typically hundreds or thousands)
of proteins or genes, simultaneously.
 
%%%%%%%%%%%%%%%%%%%%%%%%%%%%%%%%%%%%%%%%%%%%%%%%%%%%%%%%%%%%%%%%%%%%%%
\vspace{2cm}
\section{Some Typical Questions} 
%%%%%%%%%%%%%%%%%%%%%%%%%%%%%%%%%%%%%%%%%%%%%%%%%%%%%%%%%%%%%%%%%%%%%%

\paragraph{A - Effect of a condition change.} \\ 
%\paragraph{Question.}
Which proteins/genes are differentially expressed between
condition $A$ (treatment) and condition $B$ (control)? \\
% \paragraph{Experiment.} Measure the abundance of each protein/gene on
% several replicates in each condition.

\paragraph{B - Comparing several conditions.} \\
%\paragraph{Question.}
Which genes are involved in the early response to hydric stress?
(Condition = Time)\\

\paragraph{C- Finding proteins/genes associated to a given trait.} \\
Which proteins explains (is correlated with) the value of a given
phenotypic trait.

%%%%%%%%%%%%%%%%%%%%%%%%%%%%%%%%%%%%%%%%%%%%%%%%%%%%%%%%%%%%%%%%%%%%%%
\newpage
\section{Statistical Significance (1/2)} 
%%%%%%%%%%%%%%%%%%%%%%%%%%%%%%%%%%%%%%%%%%%%%%%%%%%%%%%%%%%%%%%%%%%%%%

\paragraph{Data.} Replicate measurements of the abundance of each
protein in each condition, at each time, or in several individuals. \\

\paragraph{Test statistic.} For each protein $g$, calculate some
criterion that is supposed to measure its implication in the biological
process: \\ \\
\emphase{2 conditions:} Student $T_g$, Mean rank of treated samples
$M_g$; \\ \\
\emphase{Time course:} Fisher $F_g$, Kruskall-Wallis $W_g$; \\ \\
\emphase{Quantitative trait:} Correlation coefficient $R_g$ \\ \\

\paragraph{Null distribution =} distribution of the test statistic
under the null hypothesis:
$$
\mbox{\emphase{$\Hbf_0$ = `the protein is not involved'}.}
$$

%%%%%%%%%%%%%%%%%%%%%%%%%%%%%%%%%%%%%%%%%%%%%%%%%%%%%%%%%%%%%%%%%%%%%%
\newpage
\section{Statistical Significance (2/2)} 
%%%%%%%%%%%%%%%%%%%%%%%%%%%%%%%%%%%%%%%%%%%%%%%%%%%%%%%%%%%%%%%%%%%%%%

\paragraph{$p$-value.} The significance of each gene is given by
$$
P_g = \Pr_{\Hbf_0}\{|\Tcal| > |T_g|\} 
\quad \mbox{or} \quad
P_g = \Pr_{\Hbf_0}\{\Fcal > F_g\}
\quad \mbox{or ...}
$$

\hspace{-1.8cm} \begin{tabular}{lcrcl}
  \paragraph{Decision rule:} 
  & \hspace{3cm} & Accept $\Hbf_0$ & & if $P_g \leq \alpha$ \qquad
  (e.g. $\alpha = 1\%, 5\%$) \\
  \\
  & & Reject $\Hbf_0$ & & otherwise
\end{tabular} \\

\paragraph{Risks:} \\
\centerline{
  \begin{tabular}{cc|cc}
    \multicolumn{2}{c|}{risk} & \multicolumn{2}{c}{decision} \\
    & & accept $\Hbf_0$ & reject $\Hbf_0$ \\
    \hline
    & $\Hbf_0$ true & $1-\alpha$ & $\alpha$ \\
    truth & \\
    & $\Hbf_0$ false & $\beta$ & $1-\beta$ \\
  \end{tabular}
}

\paragraph{Conservative approach} aims to control the risk $\alpha$ first.

%%%%%%%%%%%%%%%%%%%%%%%%%%%%%%%%%%%%%%%%%%%%%%%%%%%%%%%%%%%%%%%%%%%%%%
\newpage
\subsection{Power study}
%\bigskip
%%%%%%%%%%%%%%%%%%%%%%%%%%%%%%%%%%%%%%%%%%%%%%%%%%%%%%%%%%%%%%%%%%%%%%

The power of a test is the probability for a given difference $\delta$
\emphase{to be actually detected}. It can be
studied when the distribution of $T_g$ is known (e.g. $t$-test) \\
\centerline{
  \begin{pspicture}(25, 15)
    \rput[br](24, 13){unpaired data, $R = $ number of replicates
    $\rightarrow 2R$ data}
    \rput[bl](0, 0){ 
      \epsfig{figure=../Figures/PowerT.ps, height=15cm, width=25cm, clip=}    
      }
    \rput[B](12.5, 1){$\delta = (\mu_A - \mu_B)/\sigma$}
    \rput[B](16.25, 8){$R = 2$}
    \rput[B](10.75, 9){$4$}
    \rput[B](8.5, 9.5){$8$}
    \rput[B](4.75, 10){$64$}
  \end{pspicture}
}
% $$
% \begin{array}{c}
%   1 - \beta = \Pr\{\mbox{gene declared significant} \;|\; \delta\} \\
%   \epsfig{figure=../Figures/PowerT.ps, height=15cm, width=25cm, clip=} \\
%   \vspace{-2.5cm}
%   \\
%   \delta = (\mu_A - \mu_B)/\sigma
% \end{array}
% $$

%%%%%%%%%%%%%%%%%%%%%%%%%%%%%%%%%%%%%%%%%%%%%%%%%%%%%%%%%%%%%%%%%%%%%%
\newpage
\subsection{Typical Results: Golub data}

$R_A = 27$ patients with AML, $R_B = 11$ with ALL, 7070 genes.

\paragraph{Gene ranking}
$$
\begin{tabular}{crrrrrr}
  & \quad~ & \multicolumn{2}{c}{student test } & \quad~ & \multicolumn{2}{c}{Welch
  test} \\
  rank & & $T_1$ & $p$-value & & $T_1$ & $p$-value \\
  \hline
  1  & & -8.32 &    $< 10^{-16}$  & & -8.09 & 6.66\;$10^{-16}$ \\
  2  & &  5.16 & 2.37\;$10^{-7}$  & &  7.90 & 2.66\;$10^{-15}$ \\
  3  & &  4.57 & 4.81\;$10^{-6}$  & &  6.80 & 1.02\;$10^{-11}$ \\
  4  & & -8.86 &    $< 10^{-16}$  & & -6.43 & 1.22\;$10^{-10}$ \\
  5  & &  4.45 & 8.37\;$10^{-6}$  & &  6.29 & 3.09\;$10^{-10}$ \\
  6  & &  4.37 & 1.20\;$10^{-5}$  & &  6.28 & 3.35\;$10^{-10}$ \\
  7  & & -6.74 & 1.55\;$10^{-11}$ & & -6.26 & 3.65\;$10^{-10}$ \\
  8  & &  4.10 & 4.11\;$10^{-5}$  & &  6.21 & 5.05\;$10^{-10}$ \\
  9  & &  3.94 & 7.97\;$10^{-5}$  & &  6.18 & 6.34\;$10^{-10}$ \\
  10 & & -5.79 & 6.86\;$10^{-9}$  & & -6.14 & 7.93\;$10^{-10}$ \\
\end{tabular}
$$

%%%%%%%%%%%%%%%%%%%%%%%%%%%%%%%%%%%%%%%%%%%%%%%%%%%%%%%%%%%%%%%%%%%%%%
\newpage
\begin{tabular}{l}
  \paragraph{Comparison} ($\alpha = 5 \%$) \\
  \\
  \\
  \begin{tabular}{c|cc|c}
    & $N_2$ & $P_2$ & \\
    & & & \\
    \hline
    & & & \\
    $N_1$ & 4974 &  345 & 5319 \\
    & & & \\
    $P_1$  &  209 & 1542 & 1751 \\
    & & & \\
    \hline
    & & & \\
    &    5183 & 1887 & 7070
  \end{tabular} \\
  \\
  $P = $ positive  (significant) \\
  \\
  $N = $ negative \\
\end{tabular}
\begin{tabular}{l}
  \epsfig{figure=../Figures/Golub-pval.eps, height=15cm, width=15cm,
    bbllx=64, bblly=209, bburx=542, bbury=585, clip}
\end{tabular}

% %%%%%%%%%%%%%%%%%%%%%%%%%%%%%%%%%%%%%%%%%%%%%%%%%%%%%%%%%%%%%%%%%%%%%%
% \newpage
% \subsection{Non-parametric approach} 

% \vspace{-0.5cm}
% \paragraph{Permutation tests / bootstrap:} 
% avoid parametric assumption (e.g. Gaussian distribution)

% The distribution of $T_g$ under $\Hbf_0$ is estimated assigning ``a
% large number of times'' (denoted $S$) the $(R_A + R_B)$ values
% ($X_{A1}$, \dots, $X_{AR_A}$, $X_{B1}$, \dots, $X_{BR_B}$) randomly to
% conditions $A$ or $B$.

% Each permutation $s$ provides a pseudo value $\tilde{T}^s_g$.

% The $p$-value associated to $T_g$ is estimated by the proportion of
% pseudo values $T_g^s$ exceeding $T_g$: 
% $$
% \hat{p} = \frac{\mbox{number of permutations where }|T^s_g| >
%   |T_g|}{\mbox{total number of permutations }(S)}.
% $$
% $(R_A+R_B)$ has to be large enough to estimate small $p$-values: $S \leq
% \coefbin{R_A+R_B}{R_B}$:
% $$
% \coefbin{8}{4} = 70, \qquad\coefbin{10}{5} = 252.
% $$


%%%%%%%%%%%%%%%%%%%%%%%%%%%%%%%%%%%%%%%%%%%%%%%%%%%%%%%%%%%%%%%%%%%%%%
%%%%%%%%%%%%%%%%%%%%%%%%%%%%%%%%%%%%%%%%%%%%%%%%%%%%%%%%%%%%%%%%%%%%%%
\newpage
\chapter{Multiple testing} 
%%%%%%%%%%%%%%%%%%%%%%%%%%%%%%%%%%%%%%%%%%%%%%%%%%%%%%%%%%%%%%%%%%%%%%
%%%%%%%%%%%%%%%%%%%%%%%%%%%%%%%%%%%%%%%%%%%%%%%%%%%%%%%%%%%%%%%%%%%%%%

%%%%%%%%%%%%%%%%%%%%%%%%%%%%%%%%%%%%%%%%%%%%%%%%%%%%%%%%%%%%%%%%%%%%%%
\bigskip
\section{Testing simultaneously $G$ genes} 
%%%%%%%%%%%%%%%%%%%%%%%%%%%%%%%%%%%%%%%%%%%%%%%%%%%%%%%%%%%%%%%%%%%%%%
$$
\begin{tabular}{c|cc}
  truth & \begin{tabular}{c}declared\\ non diff. exp.\end{tabular} 
  & \begin{tabular}{c}declared\\ diff. exp.\end{tabular} \\
  & & \\
  \hline
  & & \\
  $G_0$ \begin{tabular}{c}non differentially\\ expressed genes\end{tabular}
  & $TN$ \begin{tabular}{c}true\\ negatives\end{tabular}
  & $FP$ \begin{tabular}{c}false\\ positives\end{tabular} \\
  & & \\
  & & \\
  $G_1$ \begin{tabular}{c}differentially\\ expressed genes\end{tabular}
  & $FN$ \begin{tabular}{c}false\\ negatives\end{tabular} 
  & $TP$ \begin{tabular}{c}true\\ positives\end{tabular} \\
  & & \\
  \hline
  & & \\
  $G$ genes & $N$ negatives  & $P$ positives 
\end{tabular}
$$
\refer{Dudoit \& al (03)}{}

%%%%%%%%%%%%%%%%%%%%%%%%%%%%%%%%%%%%%%%%%%%%%%%%%%%%%%%%%%%%%%%%%%%%%%
\newpage
\paragraph{Basic problem:} 
If
\begin{itemize}
\item all the genes are non differentially expressed ($G_0 =
  G$),
\item all tests are made with level $\alpha$
\end{itemize}
then
$$
\Esp(FP) = G \alpha.
$$
$G=10\;000, \alpha = 5\% \Longrightarrow \Esp(FP) = 500$ genes to be
studied for nothing.

\paragraph{Global risk $\alpha^*$:} 
\begin{itemize}
\item Family-Wise Error Rate:
$$
FWER = \Pr\{FP > 0\}
$$
\item False Discovery Rate:
  $$
  \begin{array}{rcll}
    FDR & = & \Esp(FP / P) & \qquad \mbox{if } P > 0, \\
    & =& 1 & \qquad \mbox{otherwise.}
  \end{array}
  $$
\end{itemize}

%%%%%%%%%%%%%%%%%%%%%%%%%%%%%%%%%%%%%%%%%%%%%%%%%%%%%%%%%%%%%%%%%%%%%%
\newpage
\section{Family-Wise Error Rate (FWER)}
%%%%%%%%%%%%%%%%%%%%%%%%%%%%%%%%%%%%%%%%%%%%%%%%%%%%%%%%%%%%%%%%%%%%%%

\paragraph{Sidak:} 
If tests are independent and if $G_0 = m$, we have
$$
FWER = 1 - \Pr\{FP = 0\} = 1 - (1-\alpha)^G.
$$
that insures $FWER = \alpha^*$ if each test is performed at level
$$
\emphase{\alpha = 1 - (1-\alpha^*)^{1/G}}.
$$
But proteins abundances or genes expressions (and tests) are
\textblue{not independent}.

\paragraph{Bonferroni correction}
is based on the inequality
$$
\Pr\left\{\bigcup_i A_i\right\} \leq \sum_i \Pr\{A_i\}
$$
and implies that performing each test at level \emphase{$\alpha =
  \alpha^*/G$} insures $FWER \leq \alpha^*$.

% %%%%%%%%%%%%%%%%%%%%%%%%%%%%%%%%%%%%%%%%%%%%%%%%%%%%%%%%%%%%%%%%%%%%%%
% \newpage
% \subsection{Adaptive procedure for FWER}

% \paragraph{Idea:} \\
% One step procedure are designed for the smallest $p$-value \\
% \centerline{$\Longrightarrow$ they are too conservative}

% \paragraph{Principle:} \\
% Order the $G$ $p$-values
% $$
% p_{(1)} < p_{(2)} < \dots < p_{(G)}.
% $$
% \begin{enumerate}
% \item[1.] Apply Sidak or Bonferroni correction to $p_{(1)}$, 
% \item[2.] Apply the same correction to $p_{(2)}$, 
% replacing $G$ by $G-1$ 
% \item[$\vdots$]
% \item[$k.$] Apply the same correction to $p_{(k)}$, replacing $G$ by
%   $G-k+1$.
% \end{enumerate}

%%%%%%%%%%%%%%%%%%%%%%%%%%%%%%%%%%%%%%%%%%%%%%%%%%%%%%%%%%%%%%%%%%%%%%
\newpage
\paragraph{Thresholds for Golub data}   (Welch test)

\begin{tabular}{l}
  \epsfig{figure=../Figures/Golub-seuil.eps, height=15cm, width=20cm,
    bbllx=64, bblly=209, bburx=542, bbury=585, clip}
\end{tabular}
\begin{tabular}{l}
  $\bullet$ $p$-value \\
  {\bf --} $5\%$ \\
  \textred{--} Bonferroni \\
  \textred{\dots} Holm \\
  \textlightgreen{--} Sidak \\
  \textlightgreen{\dots} Sidak ad.
\end{tabular}

%%%%%%%%%%%%%%%%%%%%%%%%%%%%%%%%%%%%%%%%%%%%%%%%%%%%%%%%%%%%%%%%%%%%%%
\newpage
\hspace{-1.8cm}\begin{tabular}{ll}
  \begin{tabular}{p{8cm}}
    \paragraph{Adjusted $p$-values} can be directly
    compared to the desired FWER $\alpha^*$. \\ \\
    Bonferroni: \\
    $p_g \leq \alpha^* / G$ \\
    $\Rightarrow \tilde{p}_g = G p_g \leq \alpha^*$ \\ \\
    Sidak: \\
    $p_g \leq 1 - (1 - \alpha^*)^{1/G}$ \\
    $\Rightarrow \tilde{p}_g = 1-(1-p_g)^G  \leq \alpha^*$ \\ \\ 
    \paragraph{Right plot.}\\
    $\bullet$ $p$-value, \\
    {\bf --} $5\%$, \\
    \textred{--} Bonferroni, \\
    \textlightgreen{--} Sidak
  \end{tabular}
  &
  \begin{tabular}{p{15cm}}    
    \paragraph{Golub data} \\ \\
    \epsfig{figure=../Figures/Golub-adjp.eps, height=15cm, width=16cm,
      bbllx=64, bblly=209, bburx=542, bbury=585, clip} 
  \end{tabular}
\end{tabular}

%%%%%%%%%%%%%%%%%%%%%%%%%%%%%%%%%%%%%%%%%%%%%%%%%%%%%%%%%%%%%%%%%%%%%%
\newpage
\paragraph{Accounting for dependency} \\
The Westfall \& Young procedure preserves the correlation between
genes using permutation tests and applying the \textblue{same
  permutations} to all the genes.

Adjusted $p$-values are estimated by
$$
\begin{array}{rll}
  \hat{\tilde{p}} =
    &
    \displaystyle{\frac1S \sum_s \Ibb\{p^s_{(g)} < p_g\}}
    & 
    \qquad\mbox{"minP'' procedure} \\
    \\
    &
    \displaystyle{\frac1S \sum_s \Ibb\{|T^s_{(g)}| > |T_g|\}}
    & 
    \qquad \mbox{"maxT'' procedure}
  \end{array}
$$

The number of replicates in each condition is, again, a limitation of
this estimation procedure.

\refer{Westfall \& Young (93)}{}

%%%%%%%%%%%%%%%%%%%%%%%%%%%%%%%%%%%%%%%%%%%%%%%%%%%%%%%%%%%%%%%%%%%%%%
\newpage
\chapter{False Discovery Rate (FDR)} 
%%%%%%%%%%%%%%%%%%%%%%%%%%%%%%%%%%%%%%%%%%%%%%%%%%%%%%%%%%%%%%%%%%%%%%

\paragraph{Idea:} Not to control the risk of one error, but the
proportion of errors $\Rightarrow$ \textblue{less conservative} than
controlling FWER. 

\paragraph{Principle:} The number of declared positive genes $P$ is
given by the largest $g$ such as
$$
p_{(g)} \leq g \alpha^* / G.
$$

\paragraph{Property:} \\
If tests are independent, this guarantees that
$$
FDR \leq (G_0 / G) \alpha^* \leq \alpha^*.
$$

\refer{Benjamini \& Hochberg (95)}{}

%%%%%%%%%%%%%%%%%%%%%%%%%%%%%%%%%%%%%%%%%%%%%%%%%%%%%%%%%%%%%%%%%%%%%%
\newpage 
\paragraph{Central idea:}
The $p$-values of null genes are uniformly distributed: 
$$
g \mbox{ null} \qquad \Longrightarrow \qquad p_g \sim \Ucal[0, 1].
$$

The histogram of the $p$-values must look like this: \\
%$$
\centerline{
  \epsfig{figure=../Figures/HistoPvalRef.ps, height=14cm, width=22cm,
    clip=}%, bbllx=25.3, bblly=16, bburx=243, bbury=185}   
}
%$$

\newpage 
$$
\begin{tabular}{cc}
  not like this: & nor like this: \\
  \epsfig{figure=../Figures/TDdiff.eps, height=10cm, width=12cm,
    clip=, bbllx=25, bblly=28, bburx=176, bbury=131} & 
  \epsfig{figure=../Figures/CompExcepLRT.eps, height=10cm, width=12cm,
    clip=, bbllx=108, bblly=663, bburx=509, bbury=730.5} 
\end{tabular}
$$

The absence of a 'large' uniform part (on the right) reveals a
\textblue{lack of fit} for the null distribution of test statistic.

%%%%%%%%%%%%%%%%%%%%%%%%%%%%%%%%%%%%%%%%%%%%%%%%%%%%%%%%%%%%%%%%%%%%%%
\newpage 
\paragraph{Remarks:} 
\begin{itemize}
\item Estimating $G_0$ improves the procedure.
\item The Benjamini \& Yekutieli (01) criterion: \qquad
  $p_{(g)} \leq g \alpha^* \left/ \left( G \sum_j 1/j\right) \right.$ \\
  controls the FDR for some positive correlations between tests.
\end{itemize}

\paragraph{Adjusted $p$-value:} \\
For the Benjamini \& Hochberg procedure:
$$
\tilde{p}_{(g)} = \min_{j \geq g}\{\min[G p_{(j)} / j, 1]\}.
$$

\paragraph{Statistical Analysis of Microarray  (SAM):} \\
Proposed by Tusher \& al. (01) aims to control FDR using a Westfall \&
Young estimates of the adjusted $p$-value.

%%%%%%%%%%%%%%%%%%%%%%%%%%%%%%%%%%%%%%%%%%%%%%%%%%%%%%%%%%%%%%%%%%%%%%
\newpage
\paragraph{Adjusted $p$-values for Golub data} 

\begin{tabular}{l}
  \epsfig{figure=../Figures/Golub-adjp.eps, height=15cm, width=20cm,
    bbllx=64, bblly=209, bburx=542, bbury=585, clip}
\end{tabular}
\begin{tabular}{l}
  $\bullet$ $p$-value \\
  {\bf --} $5\%$ \\
  \textred{--} Bonferroni \\
  \textred{\dots} Holm \\
  \textlightgreen{--} Sidak \\
  \textlightgreen{\dots} Sidak ad. \\
  \textblue{\dots} FDR
\end{tabular}

%%%%%%%%%%%%%%%%%%%%%%%%%%%%%%%%%%%%%%%%%%%%%%%%%%%%%%%%%%%%%%%%%%%%%%
\newpage
\paragraph{Number of positive genes} 

\begin{tabular}{l}
  $p$-value: \\
  \qquad 1887 \\
  \\
  Bonferroni: \\
  \qquad 111 \\
  \\
  Sidak: \\
  \qquad 113 \\
  \\
%   Holm: \\\\
%   \qquad 112 \\
%   \\
%   Sidak adp.: \\
%   \qquad 113 \\
%   \\
  FDR: \\
  \qquad  903 \\
\end{tabular}
\begin{tabular}{l}
  \epsfig{figure=../Figures/Golub-adjp-zoom.eps, height=15cm, width=20cm,
    bbllx=64, bblly=209, bburx=542, bbury=585, clip}
\end{tabular}

%%%%%%%%%%%%%%%%%%%%%%%%%%%%%%%%%%%%%%%%%%%%%%%%%%%%%%%%%%%%%%%%%%%%%%
\newpage
\subsection{Estimation of the proportion of 'null' genes $\pi_0$}

\noindent
\begin{tabular}{p{12cm}p{12cm}}
  \paragraph{Empirical proportion.} Storey \& al propose an estimate
  of $a$ based on this approximation:  
  &
  \paragraph{Linear regression.}
  $\pi_0$ can also be estimated by the coefficient of the linear
  regression of the empirical cdf  for $p > \lambda$: \\
  \qquad $\widehat{\pi}_0 = \left.[R(\lambda)/G] \right/ (1-\lambda)$ 
  & \qquad $\widehat{F}(p) \simeq \widehat{\pi}_0 p + \mbox{cst}$ 
\\
  \epsfig{file=../Figures/RegGenoWas.eps, width=12cm, height=12cm,
  clip=, angle=90, bbllx=66, bblly=424, bburx=555, bbury=694}
  & \epsfig{file=../Figures/RegGenoWas.eps, width=12cm, height=12cm,
    clip=, angle=90, bbllx=66, bblly=85, bburx=555, bbury=353} 
\end{tabular} \\
Both method provide upwardly biased estimates $\Rightarrow$
conservative estimates of FDR.

%%%%%%%%%%%%%%%%%%%%%%%%%%%%%%%%%%%%%%%%%%%%%%%%%%%%%%%%%%%%%%%%%%%%%%
%%%%%%%%%%%%%%%%%%%%%%%%%%%%%%%%%%%%%%%%%%%%%%%%%%%%%%%%%%%%%%%%%%%%%%
\newpage
\chapter{Local False Discovery Rate} 
%%%%%%%%%%%%%%%%%%%%%%%%%%%%%%%%%%%%%%%%%%%%%%%%%%%%%%%%%%%%%%%%%%%%%%
%%%%%%%%%%%%%%%%%%%%%%%%%%%%%%%%%%%%%%%%%%%%%%%%%%%%%%%%%%%%%%%%%%%%%%

\noindent FDR provides a general information about the risk of the whole
procedure (up to step $i$). But we are actually interested in a
specific risk, associated to each gene.


\paragraph{Local FDR ($\lFDR$).} First defined by Efron \&
al. (JASA, 2001) in a mixture model framework:
$$
\lFDR_g := \Pr\{\Hbf_0(g) \mbox{ is false} \;|\; T_g\}.
$$

\vspace{1cm}
\hspace{-1.8cm}
\begin{tabular}{cc}
  \begin{tabular}{p{13.5cm}}
    \paragraph{Under $\Hbf_0(g)$,} 
    $P_g$ is uniformly distributed over $[0, 1]$: 
    $$
    P_g \underset{\Hbf_0(g)}{\sim} \Ucal_{[0, 1]}
    $$ 
    ~\\
    \paragraph{Mixture model.} 
    The $P_g$'s are distributed according to density 
    $$
    g(p) = \textred{(1-\pi_0) f(p)} + \textdarkgreen{\pi_0} 
    $$
  \end{tabular}
  &
  \begin{tabular}{c}
  \epsfig{figure=../Figures/HistoPvalRef.ps, height=8cm, width=8cm,
    clip=, bbllx=26, bblly=40, bburx=176, bbury=130}   
%     \epsfig{
%       file=../Figures/HistoPval.ps,
%       height=8cm, width=8cm, bbllx=82, bblly=304, bburx=277,
%       bbury=460, angle=90, clip=} 
  \end{tabular}
\end{tabular}

%%%%%%%%%%%%%%%%%%%%%%%%%%%%%%%%%%%%%%%%%%%%%%%%%%%%%%%%%%%%%%%%%%%%%%
\newpage
\section{Mixture for transformed $p$-values}
%%%%%%%%%%%%%%%%%%%%%%%%%%%%%%%%%%%%%%%%%%%%%%%%%%%%%%%%%%%%%%%%%%%%%%
%%%%%%%%%%%%%%%%%%%%%%%%%%%%%%%%%%%%%%%%%%%%%%%%%%%%%%%%%%%%%%%%%%%%%%
\subsection{The probit transform}
$$
\begin{tabular}{cc}
    \begin{tabular}{p{7cm}}
      $p$-value are distributed between 0 and 1. \\
      \\
      One may use the \textblue{inverse Gaussian cdf (probit)} to get
      transformed 
      $p$-value: 
      $$
      X = \Phi^{-1}(P)
      $$
      The probit transform 'zooms' on the tails of the
      distributions and provides a \textblue{better separation} of the two
      populations.       
  \end{tabular}
  &
  \begin{tabular}{cc}
    \begin{tabular}{c}
      \epsfig{file = ../Figures/ProbitTransform-P.eps, width=7cm,
        height=7cm, angle=90}
    \end{tabular}
    &
    \begin{tabular}{c}
      \epsfig{file = ../Figures/ProbitTransform-Phi.eps, width=7cm, height=7cm}
    \end{tabular}
    \\
    & 
    \begin{tabular}{c}
      \epsfig{file = ../Figures/ProbitTransform-X.eps, width=7cm, height=7cm}
    \end{tabular}
  \end{tabular}
\end{tabular}
$$

%%%%%%%%%%%%%%%%%%%%%%%%%%%%%%%%%%%%%%%%%%%%%%%%%%%%%%%%%%%%%%%%%%%%%%
\newpage
\paragraph{Effect of the Probit transform}

%Instead of modeling the distribution of the $P_g$', we consider the
%$$
\begin{tabular}{lcr}
  $P_g \in [0, 1]$
  &   
  \qquad 
  &
  $X_g = \Phi^{-1}(P_g) \in \Rbb$ 
  \\
  \multicolumn{3}{c}{(Efron, JASA, 2005)}
  \\
  \textdarkgreen{$\phi = \Ucal_{[0; 1]}$}
  & 
  & 
  \textdarkgreen{$\phi = \Ncal(0, 1)$}
  \\
  \\
  \\
  \epsfig{figure=../Figures/HistoPval.ps, height=8cm, width=11.5cm,
    clip=, bbllx=26, bblly=40, bburx=176, bbury=130}   
  &
  &
%  \vspace{-2cm}
  \epsfig{figure=../Figures/HistoProbit.ps, height=8cm, width=11.5cm,
    clip=, bbllx=26, bblly=39, bburx=176, bbury=130}   
\end{tabular}
%$$

%%%%%%%%%%%%%%%%%%%%%%%%%%%%%%%%%%%%%%%%%%%%%%%%%%%%%%%%%%%%%%%%%%%%%%
\newpage
\subsection{Prior and posterior probability}
$$
  \begin{tabular}{cc}
    \textblue{Model:} & \textblue{Posterior probability:} \\
    \\
    $f(x) = \textred{\pi_1 f_1(x)} + \textgreen{\pi_2 f_2(x)} +
    \textblue{\pi_3 f_3(x)}$ & $\tau_{gk} = \Pr\{g \in f_k \;|\; x_g\}
    = \pi_k f_k(x_g) / f(x_g)$\\
    \\
    \epsfig{file=../Figures/Melange-densite.ps, height=6cm,
      width=12cm, bbllx=77, bblly=328, bburx=549, bbury=528, clip=}
    &
    \epsfig{file=../Figures/Melange-posteriori.ps, height=6cm, width=12cm,
      bbllx=83, bblly=320, bburx=549, bbury=537, clip=} 
  \end{tabular}
$$
$$  
\begin{array}{cccc}
  \quad \tau_{gk}~~(\%) \quad & \qquad g=1 \qquad & \qquad g=2 \qquad &
  \qquad g=3 \qquad \\
  \hline
  k = 1 & 65.8 & 0.7 & 0.0 \\
  k = 2 & 34.2 & 47.8 & 0.0 \\
  k = 3 & 0.0 & 51.5 & 1.0
\end{array}
$$

%%%%%%%%%%%%%%%%%%%%%%%%%%%%%%%%%%%%%%%%%%%%%%%%%%%%%%%%%%%%%%%%%%%%%
\newpage
\subsection{Gaussian mixture for transformed $p$-values}

$$
\begin{tabular}{cc}
    \begin{tabular}{p{10cm}}
      McLachlan et al. (05) suggest to use a \textblue{Gaussian}
      mixture to separate null genes from differentially expressed ones:
      $$
      \Phi^{-1}(P) \sim \textred{\pi_1 \Ncal(\mu, \sigma^2)} +
      \textgreen{\pi_0 \Ncal(0, 1)} 
      $$
      Posterior probabilities provide estimates of the local FDR.
    \end{tabular}
    &
    \begin{tabular}{c}
      \epsfig{file = ../Figures/Mix2Gauss.eps, width=12cm,
        height=10cm}
    \end{tabular}
\end{tabular}
$$

%%%%%%%%%%%%%%%%%%%%%%%%%%%%%%%%%%%%%%%%%%%%%%%%%%%%%%%%%%%%%%%%%%%%%%
\newpage
\section{Semi-parametric mixture model}
%%%%%%%%%%%%%%%%%%%%%%%%%%%%%%%%%%%%%%%%%%%%%%%%%%%%%%%%%%%%%%%%%%%%%%

\bigskip
\paragraph{Property of the test statistic.} The standard hypotheses
testing theory implies that, under $\Hbf_0(g)$, $P_g$ is uniformly
distributed over $[0, 1]$: \\
\begin{tabular}{cc}
  \begin{tabular}{p{13.5cm}}
    $$
    P_g \underset{\Hbf_0(g)}{\sim} \Ucal_{[0, 1]}
    $$
    \\
    ~\\
    The $P_g$'s are distributed according to a mixture distribution
    with density 
    $$
    g(p) = (1-\pi_0) f(p) + \pi_0 
    $$
    \\
    The problem is then to estimate
  \end{tabular}
  &
  \begin{tabular}{c}
  \epsfig{figure=../Figures/HistoPvalRef.ps, height=8cm, width=8cm,
    clip=, bbllx=26, bblly=40, bburx=176, bbury=130}   
%     \epsfig{
%       file=../Figures/HistoPval.ps,
%       height=8cm, width=8cm, bbllx=82, bblly=304, bburx=277,
%       bbury=460, angle=90, clip=} 
  \end{tabular}
\end{tabular}
$$
\begin{tabular}{ll}
  \textblue{$\pi_0$:} & the proportion of non-differentially expressed genes
  \\
  \textblue{$f$:} & the alternative density \\
\end{tabular}
$$

%%%%%%%%%%%%%%%%%%%%%%%%%%%%%%%%%%%%%%%%%%%%%%%%%%%%%%%%%%%%%%%%%%%%%%
\newpage
\subsection{Density estimation}

\hspace{-2cm}
\begin{tabular}{ll}
  \begin{tabular}{p{10cm}}
  \paragraph{Kernel estimate.} A natural non-parametric estimate of $f$
  is \\
  $\displaystyle{\widehat{f}(x) = \frac1{\sum_g Z_g} \sum_g Z_g k_g(x)}$ \\
  where \\
  $\displaystyle{k_g(x) = \frac1h k\left(\frac{x-x_g}h\right)}$  \\
  \\
  $k$ being a kernel, i.e. a symmetric density function with mean
  0. \\
  \\
  \end{tabular}
  &
  \begin{tabular}{c}
    \epsfig{file=../Figures/FigKernelEstim.eps, width=9cm, height=13cm,
    angle=90, clip=, bbllx=77, bblly=61, bburx=552, bbury=676}
  \end{tabular}
\end{tabular}

\vspace{-0.5cm}
\paragraph{Weighted kernel estimate.} Since the $Z_g$'s are unknown,
we propose to replace them by their conditional expectations:
$$
\widehat{f}(x) = \frac1{\sum_g \tau_g}\sum_g \tau_g k_g(x)
$$
$\tau_g$ is the weight of observation $i$ in the estimation of $f$.

%%%%%%%%%%%%%%%%%%%%%%%%%%%%%%%%%%%%%%%%%%%%%%%%%%%%%%%%%%%%%%%%%%%%%%
\newpage
\subsection{Application to Hedenfalk data}

\paragraph{Student t-test} with variance modeling ($K = 5$ groups of
variances). 

\hspace{-2cm}
\begin{tabular}{cc} 
  \begin{tabular}{c}
    $\widehat{\pi_0} = 69.5 \%$ \\
    $\textblue{\widehat{g}(x)}  = \textred{(1-\pi_0) \widehat{f}(x)} +
    \textgreen{\pi_0 \widehat{f}(x)}$ \\ 
    \epsfig{
      file=/RECHERCHE/EXPRESSION/EXEMPLES/HEDENFALK/Ainit/Asup-1.0/Varmixt-Gaus.eps,
      height=12cm, width=9cm, bbllx=66, bblly=510, bburx=283,
      bbury=691, clip=, angle=90} 
  \end{tabular}
  &
  \begin{tabular}{c} 
    $\textred{\widehat{FDR}_g}, \textblue{\widehat{\tau}_g} \times
    \Phi^{-1}(P_g)$ \\ 
    \epsfig{
      file=/RECHERCHE/EXPRESSION/EXEMPLES/HEDENFALK/Ainit/Asup-1.0/Varmixt-Gaus.eps,
      height=12cm, width=4.5cm, bbllx=66, bblly=295, bburx=283,
      bbury=480, clip=, angle=90} 
    \\
    $\textred{\widehat{FDR}_g}, \textblue{\widehat{\tau}_g} \times
    P_g$ \\ 
    \epsfig{
      file=/RECHERCHE/EXPRESSION/EXEMPLES/HEDENFALK/Ainit/Asup-1.0/Varmixt-Gaus.eps,
      height=12cm, width=4.5cm, bbllx=66, bblly=85, bburx=283,
      bbury=265, clip=, angle=90} 
  \end{tabular}
\end{tabular}

\centerline{$
  \begin{array}{ccccc}
    \quad \widehat{FDR}_{(g)} \quad & \qquad i \qquad & \quad P_{(g)}
    \quad & \quad \widehat{\tau}_{(g)} \quad & \quad
    \widehat{FNR}_{(g)} \quad \\  
    \hline
    1\% & 4 & 2.5\;10^{-5} & 0.988 & 31.5 \% \\
    5\% & 142 & 3.1\;10^{-3} & 0.914 & 28.7 \% \\
    10\% & 296 & 1.3\;10^{-2} & 0.798 & 25.7 \% \\
  \end{array}
$}
% \vspace{-0.5cm}
% $\widehat{FDR}_{(g)} = \widehat{FNR}_{(g)} = 19.7 \%$ for $(g) = 633,
% P_{(g)} = 5.4 \%, \widehat{\tau}_{(g)} = 43.5 \%$.

%%%%%%%%%%%%%%%%%%%%%%%%%%%%%%%%%%%%%%%%%%%%%%%%%%%%%%%%%%%%%%%%%%%%%%
\newpage
\subsection{R package {\tt kerFDR}.} 

Provides semi-parametric estimates of the local FDR. \\
\centerline{    
  \epsfig{file=../FIGURES/mgeni_fdr_pv.eps, clip=, width=15cm,
    height=20cm, angle=270}
}

%%%%%%%%%%%%%%%%%%%%%%%%%%%%%%%%%%%%%%%%%%%%%%%%%%%%%%%%%%%%%%%%%%%%%%
%%%%%%%%%%%%%%%%%%%%%%%%%%%%%%%%%%%%%%%%%%%%%%%%%%%%%%%%%%%%%%%%%%%%%%
%%%%%%%%%%%%%%%%%%%%%%%%%%%%%%%%%%%%%%%%%%%%%%%%%%%%%%%%%%%%%%%%%%%%%%
%%%%%%%%%%%%%%%%%%%%%%%%%%%%%%%%%%%%%%%%%%%%%%%%%%%%%%%%%%%%%%%%%%%%%%
\end{document}
%%%%%%%%%%%%%%%%%%%%%%%%%%%%%%%%%%%%%%%%%%%%%%%%%%%%%%%%%%%%%%%%%%%%%%
%%%%%%%%%%%%%%%%%%%%%%%%%%%%%%%%%%%%%%%%%%%%%%%%%%%%%%%%%%%%%%%%%%%%%%
%%%%%%%%%%%%%%%%%%%%%%%%%%%%%%%%%%%%%%%%%%%%%%%%%%%%%%%%%%%%%%%%%%%%%%
%%%%%%%%%%%%%%%%%%%%%%%%%%%%%%%%%%%%%%%%%%%%%%%%%%%%%%%%%%%%%%%%%%%%%%
