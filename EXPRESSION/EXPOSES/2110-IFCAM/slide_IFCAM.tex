\documentclass{beamer}
\usepackage{color}
\usepackage{bbm}
\usetheme{boxes}
%  
\definecolor{ceruleanblue}{rgb}{0.16, 0.32, 0.75}
\usecolortheme[rgb={0.16, 0.32, 0.75}]{structure}
%\setbeamertemplate{footline}[text line]{
%\parbox{\linewidth}{\vspace*{-8pt}\textcolor{ceruleanblue}{\insertpagenumber \ \ \ \  ANITI DAYS 2020}}}
\setbeamertemplate{navigation symbols}{}
%
\title{\textcolor{blue}{Statistical detection of regulation using multi-omics}}
\author{\textcolor{blue}{Indranil Mukhopadhyay, Indian Statistical Institute, India\\ St\'ephane Robin, MIA-Paris, INRAE, AgroParisTech, Universit� Paris-Saclay, France.}} 
\date{\textcolor{blue}{Project dates: August 2018 - July 2021}}

%%%%%%%%%%%%%%%%%%%%%%%%%%%%%%%%%%%%%%%%%%%%%%%%%%%%%%%%%%%%%%%%%%%%%%%%%%%%%%%%
%%%%%%%%%%%%%%%%%%%%%%%%%%%%%%%%%%%%%%%%%%%%%%%%%%%%%%%%%%%%%%%%%%%%%%%%%%%%%%%%
\begin{document}
%%%%%%%%%%%%%%%%%%%%%%%%%%%%%%%%%%%%%%%%%%%%%%%%%%%%%%%%%%%%%%%%%%%%%%%%%%%%%%%%
%%%%%%%%%%%%%%%%%%%%%%%%%%%%%%%%%%%%%%%%%%%%%%%%%%%%%%%%%%%%%%%%%%%%%%%%%%%%%%%%
\setbeamertemplate{background canvas}{\includegraphics
[width=\paperwidth,height=\paperheight]{fig/template_page1_IFCAM.pdf}}
    
%%%%%%%%%%%%%%%%%%%%%%%%%%%%%%%%%%%%%%%%%%%%%%%%%%%%%%%%%%%%%%%%%%%%%%%%%%%%%%%%
\begin{frame}
%[plain]
\titlepage
\end{frame}

%
\setbeamertemplate{footline}[text line]{
\parbox{\linewidth}{\vspace*{-8pt}\textcolor{ceruleanblue}{\insertpagenumber \ \ \ \  short project name }}}

{
\setbeamertemplate{background canvas}{\includegraphics
[width=\paperwidth,height=\paperheight]{fig/frame_IFCAM.pdf}}
%

%%%%%%%%%%%%%%%%%%%%%%%%%%%%%%%%%%%%%%%%%%%%%%%%%%%%%%%%%%%%%%%%%%%%%%%%%%%%%%%%
\section{Scientific perimeter of the project}
%%%%%%%%%%%%%%%%%%%%%%%%%%%%%%%%%%%%%%%%%%%%%%%%%%%%%%%%%%%%%%%%%%%%%%%%%%%%%%%%

%%%%%%%%%%%%%%%%%%%%%%%%%%%%%%%%%%%%%%%%%%%%%%%%%%%%%%%%%%%%%%%%%%%%%%%%%%%%%%%%
\begin{frame}
\frametitle{\normalsize Scientific perimeter of the project}

Combine various omic data to improve the understanding of regulation

\begin{tabular}{cc}
  \begin{tabular}{p{.45\textwidth}}
    \begin{itemize}
        \item \textcolor{brown}{Thread 1} 
    A generic statistical framework for testing composed hypotheses
    %
    % \item \textcolor{brown}{Thread 2 etc} 
    \end{itemize}
  \end{tabular}
  &  
  \begin{tabular}{p{.45\textwidth}}
    \includegraphics[width=.4\textwidth]{fig/OmicsIntegration}
  \end{tabular}   
\end{tabular}

\end{frame}

%%%%%%%%%%%%%%%%%%%%%%%%%%%%%%%%%%%%%%%%%%%%%%%%%%%%%%%%%%%%%%%%%%%%%%%%%%%%%%%%
\section{One particular result you are proud of}
%%%%%%%%%%%%%%%%%%%%%%%%%%%%%%%%%%%%%%%%%%%%%%%%%%%%%%%%%%%%%%%%%%%%%%%%%%%%%%%%
\begin{frame}
\frametitle{One particular result you are proud of}


 
 \begin{tabular}{cc}
  \begin{tabular}{p{.5\textwidth}}
    {\footnotesize
    \begin{itemize}
     \item Consider $n$ entities (genes), 
     \item $Q$ tests (based on transcriptome, epigenome, proteome, ...) with null $H_0^1, \dots H_0^Q$,
     \item test for the composed null 
     $$
     \mathcal{H}_0 = \{\text{at least $Q-L$ nulls hold}\}.
     $$
    \end{itemize}
     (or any other combination of nulls) \cite{MDM21}}
  \end{tabular}
  &  
  \begin{tabular}{p{.4\textwidth}}
    \includegraphics[width=.4\textwidth]{fig/MDM21-ArXiv-Fig1b}
    {\tiny ROC curve: proposed (red) vs intersection-union test (blue) for $Q=8$, $L=1$}
  \end{tabular}   
\end{tabular}


\end{frame}

%%%%%%%%%%%%%%%%%%%%%%%%%%%%%%%%%%%%%%%%%%%%%%%%%%%%%%%%%%%%%%%%%%%%%%%%%%%%%%%%
\section{publication}
%%%%%%%%%%%%%%%%%%%%%%%%%%%%%%%%%%%%%%%%%%%%%%%%%%%%%%%%%%%%%%%%%%%%%%%%%%%%%%%%
\begin{frame}
\frametitle{\small Publications}
  \nocite{MDM21}
  \bibliography{biblio.bib}
  \bibliographystyle{plain}
%     \begin{itemize}
%     \item Tristan Mary-Huard, Sarmistha Das, Indranil Mukhopadhyay and St\'ephane Robin. ``Querying multiple sets of p-values through composed hypothesis testing." (2021). Accepted in Bioinformatics.
%     \end{itemize}
\end{frame}

%%%%%%%%%%%%%%%%%%%%%%%%%%%%%%%%%%%%%%%%%%%%%%%%%%%%%%%%%%%%%%%%%%%%%%%%%%%%%%%%
%%%%%%%%%%%%%%%%%%%%%%%%%%%%%%%%%%%%%%%%%%%%%%%%%%%%%%%%%%%%%%%%%%%%%%%%%%%%%%%%
\end{document}
%%%%%%%%%%%%%%%%%%%%%%%%%%%%%%%%%%%%%%%%%%%%%%%%%%%%%%%%%%%%%%%%%%%%%%%%%%%%%%%%
%%%%%%%%%%%%%%%%%%%%%%%%%%%%%%%%%%%%%%%%%%%%%%%%%%%%%%%%%%%%%%%%%%%%%%%%%%%%%%%%

